\documentclass{beamer}
% Use DS9 global theme
\usepackage{../../../../shared/templates/ds9_theme}
\definecolor{ds9blue}{HTML}{0066CC}
\definecolor{ds9red}{HTML}{CC0000}

% Additional packages for algorithms
\usepackage{listings}
\usepackage{algorithm}
\usepackage{algpseudocode}

% Title page configuration
\title[Sorting Algorithms]{C++ Sorting Algorithms}
\subtitle{Implementation and Visualization}
\author[Mr. Gullo]{Mr. Gullo}
\date[\today]{\today}

% Code listing style
\lstset{
  language=C++,
  basicstyle=\ttfamily\small,
  keywordstyle=\color{ds9blue},
  stringstyle=\color{ds9red},
  commentstyle=\color{ds9grey},
  numbers=left,
  numberstyle=\tiny,
  frame=single,
  breaklines=true
}

\begin{document}

\frame{\titlepage}

\begin{frame}{Learning Objectives}
\begin{block}{After this presentation, you will:}
\begin{itemize}
\item Understand five different sorting algorithms
\item Be able to implement each sorting algorithm in C++
\item Know the advantages and disadvantages of each method
\item Recognize the time complexity of different algorithms
\end{itemize}
\end{block}
\end{frame}

\begin{frame}{Initial Array}
\begin{block}{Numbers to Sort}
Our example will use these 9 numbers: [7, 2, 9, 4, 5, 8, 3, 6, 10]
\end{block}
\end{frame}

\begin{frame}[fragile]{Bubble Sort}
\begin{block}{Algorithm Description}
\begin{itemize}
\item Repeatedly steps through the list
\item Compares adjacent elements and swaps them if needed
\item Continues until no swaps are needed
\end{itemize}
\end{block}

\begin{lstlisting}
void bubbleSort(int arr[], int n) {
    for (int i = 0; i < n-1; i++)
        for (int j = 0; j < n-i-1; j++)
            if (arr[j] > arr[j+1])
                swap(arr[j], arr[j+1]);
}
\end{lstlisting}
\end{frame}

\begin{frame}{Bubble Sort Steps}
\begin{center}
\begin{tabular}{|c|c|c|c|c|c|c|c|c|}
\hline
7 & 2 & 9 & 4 & 5 & 8 & 3 & 6 & 10 \\
\hline
2 & 7 & 4 & 5 & 8 & 3 & 6 & 9 & 10 \\
\hline
2 & 4 & 5 & 7 & 3 & 6 & 8 & 9 & 10 \\
\hline
2 & 4 & 5 & 3 & 6 & 7 & 8 & 9 & 10 \\
\hline
2 & 4 & 3 & 5 & 6 & 7 & 8 & 9 & 10 \\
\hline
2 & 3 & 4 & 5 & 6 & 7 & 8 & 9 & 10 \\
\hline
\end{tabular}
\end{center}
\end{frame}

\begin{frame}[fragile]{Selection Sort}
\begin{block}{Algorithm Description}
\begin{itemize}
\item Divides array into sorted and unsorted regions
\item Finds minimum element in unsorted region
\item Swaps it with first element of unsorted region
\end{itemize}
\end{block}

\begin{lstlisting}
void selectionSort(int arr[], int n) {
    for (int i = 0; i < n-1; i++) {
        int min_idx = i;
        for (int j = i+1; j < n; j++)
            if (arr[j] < arr[min_idx])
                min_idx = j;
        swap(arr[min_idx], arr[i]);
    }
}
\end{lstlisting}
\end{frame}

% Continue with remaining frames similarly...

\begin{frame}{Time Complexity Comparison}
\begin{block}{Time Complexity}
\begin{itemize}
\item Bubble Sort: O(n²)
\item Selection Sort: O(n²)
\item Insertion Sort: O(n²)
\item Quick Sort: O(n log n) average, O(n²) worst case
\item Merge Sort: O(n log n)
\end{itemize}
\end{block}
\end{frame}

\begin{frame}{Summary}
\begin{block}{Key Points}
\begin{itemize}
\item Bubble Sort: Simple but inefficient
\item Selection Sort: Simple and performs well on small lists
\item Insertion Sort: Efficient for small data sets
\item Quick Sort: Generally the fastest in practice
\item Merge Sort: Consistent performance but requires extra space
\end{itemize}
\end{block}
\end{frame}

\end{document}
