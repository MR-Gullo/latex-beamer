\documentclass{beamer}
% Use DS9 global theme (includes pgfplots for visualization)
\usepackage{../../../shared/templates/ds9_theme}

% Title page configuration
\title[Functions]{CS12 CH11: Functions}
\subtitle{Writing Reusable Code}
\author[Mr. Gullo]{Mr. Gullo}
\date[Oct 2025]{October 2025}

\begin{document}
\frame{\titlepage}

\section{Introduction to Functions}

\begin{frame}
\frametitle{Learning Objectives}
By the end of this lesson, you will be able to:
\begin{itemize}
    \item Identify the components of a function (return type, name, parameters, body)
    \item Write functions with various return types (int, bool, char, void)
    \item Implement functions that accept multiple parameters
    \item Use return statements correctly to send values back to the caller
    \item Apply functions to solve computational problems involving iteration and logic
\end{itemize}
\end{frame}

\begin{frame}
\frametitle{What is a Function?}
A \textbf{function} is a reusable block of code that performs a specific task.\pause

\textbf{Benefits:}
\begin{itemize}
    \item \textbf{Reusability}: Write once, use multiple times
    \item \textbf{Modularity}: Break complex problems into smaller pieces
    \item \textbf{Readability}: Makes code easier to understand
    \item \textbf{Maintenance}: Update code in one place
\end{itemize}
\end{frame}

\section{Function Anatomy}

\begin{frame}
\frametitle{Function Anatomy}
Every function has four main components:\pause

\begin{enumerate}
    \item \textbf{Return Type}: The type of data the function returns (int, bool, char, void, etc.)
    \item \textbf{Function Name}: How you reference the function (use descriptive names)
    \item \textbf{Parameters}: Input values the function receives (in parentheses)
    \item \textbf{Function Body}: The code that executes when called (in curly braces)
\end{enumerate}
\end{frame}

\begin{frame}[fragile]
\frametitle{Function Components: Example}
\begin{minted}[fontsize=\small]{cpp}
bool isEven(int a) {
    if (a % 2 == 0)
        return true;
    else
        return false;
}
\end{minted}
\pause

\textbf{Breaking it down:}
\begin{itemize}
    \item \textbf{Return type}: bool (returns true or false)
    \item \textbf{Name}: isEven (descriptive name)
    \item \textbf{Parameter}: int a (input value)
    \item \textbf{Return statement}: returns the result
\end{itemize}
\end{frame}

\begin{frame}[fragile]
\frametitle{Calling a Function}
After defining a function, you call it by name with arguments:\pause

\begin{minted}[fontsize=\small]{cpp}
cout << isEven(22) << endl;  // Output: 1 (true)
\end{minted}
\pause

\textbf{What happens:}
\begin{itemize}
    \item The value 22 is passed to parameter \texttt{a}
    \item Function executes: \texttt{22 \% 2 == 0} is true
    \item Function returns true (displayed as 1)
\end{itemize}
\end{frame}

\section{Return Types and Parameters}

\begin{frame}
\frametitle{Return Types}
Functions can return different types of data:\pause

\begin{itemize}
    \item \textbf{int}: Returns integer values
    \item \textbf{bool}: Returns true or false
    \item \textbf{char}: Returns a single character
    \item \textbf{float/double}: Returns decimal numbers
    \item \textbf{void}: Returns nothing (performs action only)
\end{itemize}
\pause

\textbf{Important}: The return statement must match the return type!
\end{frame}

\begin{frame}
\frametitle{Parameters vs Arguments}
Understanding the terminology:\pause

\textbf{Parameters:}
\begin{itemize}
    \item Variables listed in the function definition
    \item Placeholders for values the function will receive
    \item Example: \texttt{int a} in \texttt{bool isEven(int a)}
\end{itemize}
\pause

\textbf{Arguments:}
\begin{itemize}
    \item Actual values passed when calling the function
    \item Example: \texttt{22} in \texttt{isEven(22)}
\end{itemize}
\end{frame}

\begin{frame}[fragile]
\frametitle{Multiple Parameters}
Functions can accept multiple parameters:\pause

\begin{minted}[fontsize=\small]{cpp}
int max(int a, int b) {
    if (a > b)
        return a;
    else
        return b;
}
\end{minted}
\pause

\textbf{Usage:}
\begin{minted}[fontsize=\small]{cpp}
cout << max(10, 20) << endl;  // Output: 20
\end{minted}

Parameters are separated by commas in both definition and call.
\end{frame}

\section{Code Examples}

\begin{frame}[fragile]
\frametitle{Example 1: max Function}
\textbf{Goal}: Return the larger of two integers\pause

\begin{minted}[fontsize=\small]{cpp}
int max(int a, int b) {
    // Returns the bigger of the two numbers
    if (a > b)
        return a;
    else
        return b;
}
\end{minted}
\pause

\textbf{Key concepts:}
\begin{itemize}
    \item Two parameters of same type
    \item Conditional logic determines return value
    \item Only one return statement executes
\end{itemize}
\end{frame}

\begin{frame}[fragile]
\frametitle{Example 2: numberOfDigits Function}
\textbf{Goal}: Count digits in an integer\pause

\begin{minted}[fontsize=\scriptsize]{cpp}
int numberOfDigits(int a) {
    // Works for both positive and negative numbers
    if (a == 0) return 1;
    int numDigits = 0;
    
    while (a != 0) {
        a /= 10;
        numDigits++;
    }
    return numDigits;
}
\end{minted}
\pause

\textbf{Algorithm:}
\begin{itemize}
    \item Divide by 10 repeatedly to remove digits
    \item Count iterations until no digits remain
    \item Special case: zero has one digit
\end{itemize}
\end{frame}

\begin{frame}[fragile]
\frametitle{Example 3: void Functions}
Functions don't always need to return a value:\pause

\begin{minted}[fontsize=\small]{cpp}
void displayAuthor() {
    cout << "This program was written by Mark Vuorela\n";
}

// Calling the function
displayAuthor();
\end{minted}
\pause

\textbf{void functions:}
\begin{itemize}
    \item Perform actions without returning values
    \item Use \texttt{return;} (no value) or omit return
    \item Useful for output, modification, or side effects
\end{itemize}
\end{frame}

\section{Practice Exercises}

\begin{frame}[fragile]
\frametitle{Exercise 1: myAbsolute}
\textbf{Goal}: Return the absolute value of an integer\pause

\textbf{Template with TODOs:}
\begin{minted}[fontsize=\small, frame=lines]{cpp}
int myAbsolute(int a) {
    // TODO 1: Check if a is positive
    // TODO 2: If positive, return a
    // TODO 3: If negative, return a * -1
}
\end{minted}
\pause

\textbf{Expected behavior:}
\begin{itemize}
    \item \texttt{myAbsolute(25)} returns 25
    \item \texttt{myAbsolute(-25)} returns 25
\end{itemize}
\end{frame}

\begin{frame}[fragile]
\frametitle{Exercise 2: myPower}
\textbf{Goal}: Calculate $a^b$ where $b \geq 0$\pause

\textbf{Template with TODOs:}
\begin{minted}[fontsize=\scriptsize, frame=lines]{cpp}
int myPower(int a, int b) {
    int retVal = 1;
    
    // TODO 1: Create a loop that runs b times
    // TODO 2: Inside loop, multiply retVal by a
    // TODO 3: Return retVal
}
\end{minted}
\pause

\textbf{Expected behavior:}
\begin{itemize}
    \item \texttt{myPower(2, 3)} returns 8
    \item \texttt{myPower(-3, 3)} returns -27
    \item \texttt{myPower(5, 0)} returns 1
\end{itemize}
\end{frame}

\begin{frame}[fragile]
\frametitle{Exercise 3: sumDigits}
\textbf{Goal}: Return the sum of digits in an integer\pause

\textbf{Template with TODOs:}
\begin{minted}[fontsize=\scriptsize, frame=lines]{cpp}
int sumDigits(int a) {
    int sum = 0;
    a = abs(a);  // Handle negative numbers
    
    // TODO 1: Create loop while a > 0
    // TODO 2: Add rightmost digit (a % 10) to sum
    // TODO 3: Remove rightmost digit (a /= 10)
    // TODO 4: Return sum
}
\end{minted}
\pause

\textbf{Expected behavior:}
\begin{itemize}
    \item \texttt{sumDigits(33333)} returns 15
    \item \texttt{sumDigits(-123)} returns 6
\end{itemize}
\end{frame}

\begin{frame}[fragile]
\frametitle{Exercise 4: revDigits}
\textbf{Goal}: Reverse the digits of a positive integer\pause

\textbf{Template with TODOs:}
\begin{minted}[fontsize=\scriptsize, frame=lines]{cpp}
int revDigits(int a) {
    int reversed = 0;
    
    // TODO 1: Create loop while a > 0
    // TODO 2: Multiply reversed by 10
    // TODO 3: Add rightmost digit (a % 10) to reversed
    // TODO 4: Remove rightmost digit from a (a /= 10)
    // TODO 5: Return reversed
}
\end{minted}
\pause

\textbf{Expected behavior:}
\begin{itemize}
    \item \texttt{revDigits(1234)} returns 4321
    \item \texttt{revDigits(5005)} returns 5005
\end{itemize}
\end{frame}

\begin{frame}[fragile]
\frametitle{Exercise 5: fibonacci}
\textbf{Goal}: Return the $n^{th}$ Fibonacci number where $f_0 = f_1 = 1$\pause

\textbf{Template with TODOs:}
\begin{minted}[fontsize=\scriptsize, frame=lines]{cpp}
int fibonacci(int n) {
    int current = 1, last1 = 0, last2;
    
    // TODO 1: Create loop while n > 0
    // TODO 2: Store last1 in last2
    // TODO 3: Store current in last1
    // TODO 4: Calculate new current (last1 + last2)
    // TODO 5: Decrement n
    // TODO 6: Return current
}
\end{minted}
\pause

\textbf{Expected behavior:}
First 10 Fibonacci numbers: 1, 1, 2, 3, 5, 8, 13, 21, 34, 55, 89
\end{frame}

\begin{frame}[fragile]
\frametitle{Exercise 6: changeCase}
\textbf{Goal}: Convert uppercase to lowercase and vice versa\pause

\textbf{Template with TODOs:}
\begin{minted}[fontsize=\scriptsize, frame=lines]{cpp}
char changeCase(char c) {
    int caseDifference = 'z' - 'Z';
    
    // TODO 1: Check if c is lowercase (a-z)
    // TODO 2: If lowercase, return c - caseDifference
    // TODO 3: Check if c is uppercase (A-Z)
    // TODO 4: If uppercase, return c + caseDifference
    // TODO 5: Otherwise, return c unchanged
}
\end{minted}
\pause

\textbf{Expected behavior:}
\begin{itemize}
    \item \texttt{changeCase('E')} returns 'e'
    \item \texttt{changeCase('f')} returns 'F'
    \item \texttt{changeCase('4')} returns '4'
\end{itemize}
\end{frame}

\section{Summary}

\begin{frame}
\frametitle{Key Takeaways}
\textbf{Functions enable code reuse and modularity:}
\begin{itemize}
    \item Every function has a return type, name, parameters, and body
    \item Return type must match the value returned
    \item void functions perform actions without returning values
    \item Parameters receive values when function is called
    \item Multiple parameters separated by commas
\end{itemize}
\pause

\textbf{Common patterns:}
\begin{itemize}
    \item Conditional returns (max, myAbsolute)
    \item Loop-based algorithms (numberOfDigits, sumDigits)
    \item Digit manipulation (revDigits, sumDigits)
    \item Character operations (changeCase)
\end{itemize}
\end{frame}

\end{document}