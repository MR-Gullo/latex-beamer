\documentclass{beamer}
% Use DS9 global theme (includes pgfplots for visualization)
\usepackage{../../../shared/templates/ds9_theme}

% Title page configuration
\title[Problem Set 1]{CS12 CH:10}
\subtitle{Problem Solving with Loops}
\author[Mr. Gullo]{Mr. Gullo}
\date[Nov 5, 2025]{November 5, 2025}

\begin{document}
\frame{\titlepage}

\begin{frame}
\frametitle{Learning Objectives}
\begin{itemize}
\item Apply the 6-step problem-solving methodology to complex programming challenges
\item Implement while and for loops to solve iterative problems
\item Understand and apply optimization techniques to improve algorithm efficiency
\item Master unsigned data types for handling large positive integers
\item Develop algorithms for mathematical sequences and number theory problems
\item Implement template-based code structure with clear TODO sections
\end{itemize}
\end{frame}

\begin{frame}
\frametitle{The 6-Step Problem Solving Process}
\framesubtitle{A Systematic Approach to Complex Problems}
\begin{enumerate}
\item \textbf{Explore the Question}
   \begin{itemize}
   \item Understand all requirements
   \item Identify constraints and edge cases
   \item Simplify when possible
   \end{itemize}
\pause
\item \textbf{Code Something Simple First}
   \begin{itemize}
   \item Start with a simplified version
   \item Test with smaller numbers/simpler cases
   \end{itemize}
\pause
\item \textbf{Run With Examples}
   \begin{itemize}
   \item Verify against known solutions
   \item Test edge cases
   \end{itemize}
\pause
\item \textbf{Get Something Working}
   \begin{itemize}
   \item Focus on correctness over efficiency
   \item Ensure the core algorithm functions
   \end{itemize}
\pause
\item \textbf{Make Optimizations}
   \begin{itemize}
   \item Improve efficiency after correctness is confirmed
   \item Reduce unnecessary computations
   \end{itemize}
\item \textbf{Try Another Approach}
   \begin{itemize}
   \item Consider alternative algorithms
   \item Compare performance characteristics
   \end{itemize}
\end{enumerate}
\end{frame}

\begin{frame}
\frametitle{Problem Set Overview}
\framesubtitle{Today's Challenges}
\begin{enumerate}
\item \textbf{Basics of while and for loops}
   \begin{itemize}
   \item Generate sequences with specific patterns
   \end{itemize}
\item \textbf{Division without operators}
   \begin{itemize}
   \item Implement division and modulus using subtraction
   \end{itemize}
\item \textbf{Largest Prime Factor}
   \begin{itemize}
   \item Factor large numbers and identify prime factors
   \end{itemize}
\item \textbf{Smallest Multiple}
   \begin{itemize}
   \item Find LCM of numbers 1-20
   \end{itemize}
\item \textbf{Sum of Powers}
   \begin{itemize}
   \item Calculate geometric series
   \end{itemize}
\item \textbf{Greatest Common Factor}
   \begin{itemize}
   \item Implement Euclidean algorithm
   \end{itemize}
\item \textbf{Fibonacci Sequence}
   \begin{itemize}
   \item Generate terms in specified ranges
   \end{itemize}
\item \textbf{Palindromic Numbers}
   \begin{itemize}
   \item Check and find palindrome products
   \end{itemize}
\end{enumerate}
\end{frame}

\begin{frame}[fragile]
\frametitle{Key Concept: Unsigned Data Types}
\framesubtitle{Handling Large Positive Integers}
\begin{columns}
\column{0.5\textwidth}
\textbf{Signed Integers}
\begin{itemize}
\item Use one bit for sign
\item Range: $-2^{n-1}$ to $2^{n-1}-1$
\item Example: \texttt{int} (32-bit)
   \begin{itemize}
   \item Range: $-2,147,483,648$ to $2,147,483,647$
   \end{itemize}
\end{itemize}

\column{0.5\textwidth}
\textbf{Unsigned Integers}
\begin{itemize}
\item All bits for magnitude
\item Range: $0$ to $2^n-1$
\item Example: \texttt{unsigned int} (32-bit)
   \begin{itemize}
   \item Range: $0$ to $4,294,967,295$
   \end{itemize}
\end{itemize}
\end{columns}

\vspace{0.5cm}
\textbf{Common Issue:}
\begin{lstlisting}[language=C++,basicstyle=\small]
int testNum = 600851475143;  // ERROR! Too large
unsigned long long testNum = 600851475143;  // Works!
\end{lstlisting}
\end{frame}

\begin{frame}[fragile]
\frametitle{Problem 1: Loop Basics}
\framesubtitle{Template Exercise}

\textbf{Exercise File:} \texttt{problem1\_loops.cpp} (Template with TODOs)

\textbf{Objective:} Complete the TODOs to generate sequences.\pause

\begin{lstlisting}[language=C++,basicstyle=\scriptsize,numbers=left,frame=single,breaklines=true]
#include <iostream>
using namespace std;

int main()
{
    // TODO 1: Use a while loop to output multiples of 7 between 0 and 77 inclusive
    // Expected output: 0, 7, 14, 21, 28, 35, 42, 49, 56, 63, 70, 77
    
    // TODO 2: Use a for loop to output: 1000, 800, 600, ..., -1000
    // Expected output: 1000, 800, 600, 400, 200, 0, -200, -400, -600, -800, -1000
    
    return 0;
}
\end{lstlisting}

\end{frame}

\begin{frame}[fragile]
\frametitle{Problem 4: Largest Prime Factor}
\framesubtitle{Template Exercise}

\textbf{Exercise File:} \texttt{problem4\_primeFactor.cpp} (Template with TODOs)

\textbf{Objective:} Find the largest prime factor of a number.\pause

\begin{lstlisting}[language=C++,basicstyle=\scriptsize,numbers=left,frame=single,breaklines=true]
#include <iostream>
using namespace std;

int main()
{
    unsigned long long testNumber;
    unsigned long long largestPrime = 0;
    unsigned long long currentFactor = 2;
    
    cout << "Enter a positive integer greater than 1: ";
    cin >> testNumber;
    
    // TODO 1: Implement algorithm to find largest prime factor
    // Hint: Divide out factors starting from 2
    // When testNumber % currentFactor == 0:
    //    - Update largestPrime
    //    - Divide testNumber by currentFactor
    // Otherwise: increment currentFactor
    // Continue until testNumber equals 1
    
    cout << "The largest prime factor is: " << largestPrime << endl;
    
    return 0;
}
\end{lstlisting}
\end{frame}

\begin{frame}[fragile]
\frametitle{Problem 5: Smallest Multiple}
\framesubtitle{Template Exercise}

\textbf{Exercise File:} \texttt{problem5\_smallestMultiple.cpp} (Template with TODOs)

\textbf{Objective:} Find smallest number divisible by 1-20.\pause

\begin{lstlisting}[language=C++,basicstyle=\scriptsize,numbers=left,frame=single,breaklines=true]
#include <iostream>
using namespace std;

int main()
{
    int smallestNumber = 0;
    bool areWeDone = false;
    
    // TODO 1: Optimization - identify what numbers we need to check
    // Hint: What's special about prime numbers 2,3,5,7,11,13,17,19?
    
    while(!areWeDone) {
        areWeDone = true;
        smallestNumber += /* TODO 2: Fill in optimized increment value */;
        
        // TODO 3: Check divisibility by all numbers from 1 to 20
        // If not divisible by any number, set areWeDone to false and break
    }
    
    cout << "The smallest number divisible by 1..20 is: " << smallestNumber << endl;
    return 0;
}
\end{lstlisting}
\end{frame}

\begin{frame}
\frametitle{Problem 8: Fibonacci Sequence}
\framesubtitle{Understanding the Sequence}

\textbf{Definition:}
\[f_n = \begin{cases} 
f_0 = f_1 = 1 \\
f_n = f_{n-1} + f_{n-2}
\end{cases}\]

\textbf{First 10 terms:} 1, 1, 2, 3, 5, 8, 13, 21, 34, 55

\textbf{Key Implementation Strategy:}
\begin{itemize}
\item Keep track of three consecutive terms
\item Use variables for: $f_n$, $f_{n-1}$, and $f_{n-2}$
\item Shift values forward in each iteration
\end{itemize}

\alert{[Diagram showing Fibonacci sequence calculation]}
\end{frame}

\begin{frame}[fragile]
\frametitle{Problem 8: Fibonacci Implementation}
\framesubtitle{Template Exercise}

\textbf{Exercise File:} \texttt{problem8\_fibonacci.cpp} (Template with TODOs)

\textbf{Objective:} Display Fibonacci numbers from $f_a$ to $f_b$.\pause

\begin{lstlisting}[language=C++,basicstyle=\scriptsize,numbers=left,frame=single,breaklines=true]
#include <iostream>
using namespace std;

int main()
{
    int fibN = 1;    // Current term
    int fibN_1 = 0;  // Previous term
    int fibN_2 = 0;  // Two terms back
    
    int a, b;
    cout << "Enter integers a and b (where 0 <= a <= b <= 50): ";
    cin >> a >> b;
    
    // TODO 1: Find the ath term (don't print yet)
    // Use a loop to calculate up to position a
    
    // TODO 2: Print from ath to bth term
    // Handle both cases: a <= b and a > b
    // Remember to shift the three terms appropriately
    
    return 0;
}
\end{lstlisting}
\end{frame}

\begin{frame}
\frametitle{Problem 9: Palindromic Numbers}
\framesubtitle{Definition and Properties}

\textbf{Palindrome:} A number that reads the same forward and backward

\textbf{Examples:} 9, 232, 7007, 12321

\textbf{Algorithm to Check Palindrome:}
\begin{enumerate}
\item Make a copy of the original number
\item Reverse the digits using modulo and division
\item Compare original with reversed number
\end{enumerate}

\textbf{Largest Palindrome Product Problem:}
\begin{itemize}
\item Find largest palindrome from product of two 3-digit numbers
\item Brute force approach: Check all products from 999×999 down to 100×100
\item Optimization: Inner loop can start at current outer loop value
\end{itemize}

\alert{[Diagram showing palindrome check algorithm]}
\end{frame}

\begin{frame}[fragile]
\frametitle{Problem 9: Palindrome Implementation}
\framesubtitle{Template Exercise}

\textbf{Exercise File:} \texttt{problem9\_palindrome.cpp} (Template with TODOs)

\textbf{Objective:} Find largest palindrome from two 3-digit numbers.\pause

\begin{lstlisting}[language=C++,basicstyle=\scriptsize,numbers=left,frame=single,breaklines=true]
#include <iostream>
using namespace std;

bool isPalindrome(int number) {
    // TODO 1: Implement palindrome check function
    // Return true if number is palindrome, false otherwise
    int original = number;
    int reversed = 0;
    
    // Reverse the digits without destroying original
    // Compare original with reversed
    
    return false; // Replace with actual condition
}

int main() {
    int largestPalindrome = 0;
    
    // TODO 2: Implement nested loops to find largest palindrome
    // Outer loop: from 999 down to 100
    // Inner loop: from current outer value down to 100 (optimization)
    // Check if product is palindrome and larger than current max
    
    cout << "The largest palindrome is: " << largestPalindrome << endl;
    return 0;
}
\end{lstlisting}
\end{frame}

\begin{frame}
\frametitle{Optimization Techniques}
\framesubtitle{Making Your Code More Efficient}

\textbf{Problem 5 (Smallest Multiple) Optimization:}
\begin{itemize}
\item Check only multiples of primes $<$ 20
\item Product: $2 \cdot 3 \cdot 5 \cdot 7 \cdot 11 \cdot 13 \cdot 17 \cdot 19$
\item Reduces checks by $\sim$99\%
\end{itemize}

\textbf{Problem 9 (Palindrome) Optimization:}
\begin{itemize}
\item Inner loop starts at outer loop value
\item Avoids duplicate products (e.g., 123×456 and 456×123)
\item Reduces iterations by approximately 50\%
\end{itemize}

\textbf{General Optimization Principles:}
\begin{enumerate}
\item Get it working first, then optimize
\item Profile to identify bottlenecks
\item Consider mathematical properties of the problem
\item Reduce redundant calculations
\end{enumerate}
\end{frame}

\begin{frame}
\frametitle{Summary}
\framesubtitle{Key Takeaways}

\textbf{Problem-Solving Methodology:}
\begin{itemize}
\item Break complex problems into smaller, manageable parts
\item Start with simple cases and verify your approach
\item Implement a working solution before optimizing
\end{itemize}

\textbf{Loop Implementation:}
\begin{itemize}
\item While loops: When iteration count is unknown
\item For loops: When iteration count is known
\item Proper initialization, condition, and increment are essential
\end{itemize}

\textbf{Algorithm Design:}
\begin{itemize}
\item Consider data type limitations (unsigned for large positives)
\item Look for mathematical properties to optimize
\item Test with edge cases and known solutions
\end{itemize}

\textbf{Next Steps:}
\begin{itemize}
\item Complete all template exercises
\item Experiment with different optimization approaches
\item Apply problem-solving methodology to new challenges
\end{itemize}
\end{frame}

\end{document}
