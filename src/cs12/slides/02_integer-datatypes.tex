\documentclass{beamer}
% Use DS9 global theme (includes pgfplots for visualization)
\usepackage{../../../../shared/templates/ds9_theme}
\usepackage{minted} % For code highlighting

% Title page configuration
\title[Integer Datatypes]{CS12 CH:02}
\subtitle{C++ Introduction \& Integer Datatypes}
\author[Mr. Gullo]{Mr. Gullo}
\date[\today]{\today}

\begin{document}
\frame{\titlepage}

\section{Learning Objectives}

\begin{frame}
\frametitle{Learning Objectives}
\begin{itemize}
    \item Understand the purpose and syntax of comments in C++.\pause
    \item Identify the fundamental integer-based datatypes: \texttt{int}, \texttt{char}, and \texttt{bool}.\pause
    \item Recognize valid operations and behaviors for each datatype.\pause
    \item Understand ASCII representation and special characters.\pause
    \item Apply integer division, modulo operations, and comparison operators.\pause
    \item Declare and initialize variables of different datatypes.
\end{itemize}
\end{frame}

\section{Comments in C++}

\begin{frame}
\frametitle{What Are Comments?}
\begin{block}{For Human Eyes Only}
Comments are text in your code that is completely ignored by the compiler. They are meant for you and other programmers to read and understand the code better.
\end{block}
\pause

\begin{columns}[T]
    \begin{column}{0.5\textwidth}
        \begin{alertblock}{Single-Line Comments}
        Start with \texttt{//}. The compiler ignores everything from \texttt{//} to the end of the line.
        \end{alertblock}
    \end{column}\pause
    \begin{column}{0.5\textwidth}
        \begin{exampleblock}{Multi-Line Comments}
        Start with \texttt{/*} and end with \texttt{*/}. Can span multiple lines. Useful for longer explanations or "commenting out" large blocks of code.
        \end{exampleblock}
    \end{column}
\end{columns}
\end{frame}

\begin{frame}[fragile]
\frametitle{Code Example: Comments}
\textbf{Demo File:} \texttt{02\_comments.cpp} (Interactive - run and modify)
\begin{minted}[fontsize=\small, frame=lines, linenos, breaklines]{cpp}
#include <iostream>
using namespace std;

// Use a double forward slash to create a single line comment
/*
    Use a forward slash and a star to make a multiline comment
    these are useful for large pieces of text or to remove sections
    from your code without deleting it.
*/
int main()
{
  // cout << "Hello World" << endl; // This line is disabled by a comment
    cout << "goodbye comments" << endl;
    return 0;
}
\end{minted}
\end{frame}

\section{Numeric Datatypes}

\begin{frame}
\frametitle{Integer Datatypes (The Basics)}
Today we will look at three fundamental types that store whole numbers or concepts based on them.\pause

\begin{columns}[T]
    \begin{column}{0.33\textwidth}
        \begin{alertblock}{int}
        Short for 'integer'. Stores positive and negative whole numbers.
        \vspace{1em}
        \textbf{Examples:} \texttt{23}, \texttt{19}, \texttt{-3}, \texttt{0}
        \end{alertblock}
    \end{column}\pause
    \begin{column}{0.33\textwidth}
        \begin{exampleblock}{char}
        Short for 'character'. Stores a \textit{single} character.
        \vspace{1em}
        \textbf{Examples:} \texttt{'a'}, \texttt{'Z'}, \texttt{'\textbackslash n'}
        \end{exampleblock}
    \end{column}\pause
    \begin{column}{0.33\textwidth}
        \begin{block}{bool}
        Short for 'Boolean'. Stores logical values.
        \vspace{1em}
        \textbf{Examples:} \texttt{true}, \texttt{false}
        \end{block}
    \end{column}
\end{columns}
\end{frame}

\section{The \texttt{int} Datatype}

\begin{frame}
\frametitle{Operations on \texttt{int}s}
\begin{columns}[T]
    \begin{column}{0.5\textwidth}
        \begin{alertblock}{Binary Operations}
        \begin{itemize}
            \item \texttt{+} (Addition)
            \item \texttt{-} (Subtraction)
            \item \texttt{*} (Multiplication)
            \item \texttt{/} (Integer Division - no remainder!)
            \item \texttt{\%} (Modulo - gives the remainder)
        \end{itemize}
        \end{alertblock}
    \end{column}\pause
    \begin{column}{0.5\textwidth}
        \begin{exampleblock}{Comparison Operations}
        \begin{itemize}
            \item \texttt{==} (Equal to)
            \item \texttt{!=} (Not equal to)
            \item \texttt{<} (Less than)
            \item \texttt{<=} (Less than or equal to)
            \item \texttt{>} (Greater than)
            \item \texttt{>=} (Greater than or equal to)
        \end{itemize}
        \end{exampleblock}
    \end{column}
\end{columns}
\end{frame}

\begin{frame}[fragile]
\frametitle{I Do: Integer Operations Demo}
\textbf{Demo File:} \texttt{02\_dataTypesIntegers.cpp} (Interactive - comprehensive demo)
\\Let's walk through this code and see what it does.\pause
\begin{minted}[fontsize=\scriptsize, frame=lines, linenos, breaklines]{cpp}
int main()
{
   int x = 34;
   int y = 5;

   // Integer addition
   cout << "x + y = " << x + y << endl;
   // Integer subtraction
   cout << "x - y = " << x - y << endl;
   // Integer multiplication
   cout << "x * y = " << x * y << endl;
   // Integer division (rounds down)
   cout << "x / y = " << x / y << endl;
   // Integer modulo division (remainder)
   cout << "x % y = " << x % y << endl;
   // Integer comparison ==
   cout << "x == y is " << (x == y) << endl;
   return 0;
}
\end{minted}
\end{frame}


\section{The \texttt{char} Datatype}

\begin{frame}
\frametitle{Characters and ASCII}
\begin{alertblock}{Characters are Numbers!}
A \texttt{char} stores a single character like \texttt{'a'} or \texttt{'5'}. But behind the scenes, the computer stores it as an integer code. The most common system is \textbf{ASCII} (American Standard Code for Information Interchange).
\end{alertblock}\pause
\begin{exampleblock}{Example}
The character \texttt{'A'} is stored as the number 65.
\newline
The character \texttt{'a'} is stored as the number 97.\pause
\newline
This means we can perform math on characters! \texttt{'a' - 32} would result in \texttt{'A'}.
\end{exampleblock}
\end{frame}

\begin{frame}
\frametitle{Common Special Characters}
Some characters are not printable. We use an "escape sequence" (a backslash followed by a letter) to represent them.\pause
\begin{center}
\scriptsize
\begin{tabular}{|c|c|l|}
\hline
\textbf{Escape} & \textbf{Character} & \textbf{Meaning} \\
\textbf{Character} & \textbf{Represented} & \\
\hline
\texttt{\textbackslash n} & Newline & Move to a new line \\
\texttt{\textbackslash t} & Horizontal Tab & Move to the next horizontal tab setting \\
\texttt{\textbackslash a} & Alert & Issue an alert (annoying bell sound) \\
\texttt{\textbackslash\textbackslash} & Backslash & Since \textbackslash{} is an escape character we need \\
& & to use two of them if we actually want \\
& & an \textbackslash{} to show up \\
\texttt{\textbackslash?} & Question Mark & ? Character \\
\texttt{\textbackslash'} & Single quote & ' character \\
\texttt{\textbackslash"} & Double quote & " character \\
\texttt{\textbackslash 0} & Null character & Insert the null character; zero or \\
& & 00000000 in ASCII. We will use this \\
& & a lot later in the course. \\
\hline
\end{tabular}
\end{center}
\end{frame}

\begin{frame}
\frametitle{ASCII Codes (For Reference Only)}
\begin{center}
\scriptsize
\begin{tabular}{|c|c|c|c|c|c|c|c|}
\hline
\textbf{Lowercase} & \textbf{Binary} & \textbf{Lowercase} & \textbf{Binary} & \textbf{Uppercase} & \textbf{Binary} & \textbf{Uppercase} & \textbf{Binary} \\
\textbf{Letter} & \textbf{Code} & \textbf{Letter} & \textbf{Code} & \textbf{Letter} & \textbf{Code} & \textbf{Letter} & \textbf{Code} \\
\hline
a & 01100001 & n & 01101110 & A & 01000001 & N & 01001110 \\
b & 01100010 & o & 01101111 & B & 01000010 & O & 01001111 \\
c & 01100011 & p & 01110000 & C & 01000011 & P & 01010000 \\
d & 01100100 & q & 01110001 & D & 01000100 & Q & 01010001 \\
e & 01100101 & r & 01110010 & E & 01000101 & R & 01010010 \\
f & 01100110 & s & 01110011 & F & 01000110 & S & 01010011 \\
g & 01100111 & t & 01110100 & G & 01000111 & T & 01010100 \\
h & 01101000 & u & 01110101 & H & 01001000 & U & 01010101 \\
i & 01101001 & v & 01110110 & I & 01001001 & V & 01010110 \\
j & 01101010 & w & 01110111 & J & 01001010 & W & 01010111 \\
k & 01101011 & x & 01111000 & K & 01001011 & X & 01011000 \\
l & 01101100 & y & 01111001 & L & 01001100 & Y & 01011001 \\
m & 01101101 & z & 01111010 & M & 01001101 & Z & 01011010 \\
\hline
\end{tabular}
\end{center}
\end{frame}

\begin{frame}
\frametitle{Converting from Binary to Decimal}
\begin{alertblock}{Example: Character 'M'}
'M' is represented by: 01001101 in binary\\
Let's convert this to decimal
\end{alertblock}

\begin{columns}[T]
    \begin{column}{0.6\textwidth}
        \begin{block}{Binary Powers Calculation}
        \scriptsize
        \begin{align*}
        0 \times 2^7 &= 0 \\
        1 \times 2^6 &= 64 \\
        0 \times 2^5 &= 0 \\
        0 \times 2^4 &= 0 \\
        1 \times 2^3 &= 8 \\
        1 \times 2^2 &= 4 \\
        0 \times 2^1 &= 0 \\
        1 \times 2^0 &= 1
        \end{align*}
        \end{block}
    \end{column}
    \begin{column}{0.4\textwidth}
        \begin{exampleblock}{Result}
        \Large
        64 + 8 + 4 + 1 = 77
        
        So 'M' = ASCII 77
        \end{exampleblock}
    \end{column}
\end{columns}
\end{frame}


\section{The \texttt{bool} Datatype}

\begin{frame}
\frametitle{Boolean Logic}
\begin{block}{\texttt{bool}: True or False}
A Boolean variable can only hold one of two values: \texttt{true} or \texttt{false}. They are the foundation of decision-making in programs.
\end{block}\pause

\begin{alertblock}{Booleans are also Numbers!}
In C++, \texttt{true} is represented by the integer \texttt{1}, and \texttt{false} is represented by the integer \texttt{0}.
\end{alertblock}\pause

\begin{exampleblock}{Important Note}
When evaluating a condition, C++ considers any non-zero integer to be \texttt{true} and only \texttt{0} to be \texttt{false}.
\end{exampleblock}
\end{frame}


\section{Common Errors}

\begin{frame}
\frametitle{Common Errors and How to Fix Them}
I'm going to compile these as we come across common errors.
\end{frame}

\begin{frame}
\frametitle{Error 1: Missing Semicolon}
\begin{alertblock}{Error: \texttt{expected ';' before '\}' token}}
One of your lines is likely missing the semicolon at the end.\\
Look at the lines before the line number of the error. (in this case just before 59)
\end{alertblock}
\begin{itemize}
    \item \textbf{Meaning:} You missed a semicolon at the end of a line.
    \item \textbf{Fix:} Look at the line \textit{before} the error number and add a \texttt{;}
\end{itemize}
\end{frame}

\begin{frame}[fragile]
\frametitle{Error 2: Variable Redeclaration}
\begin{alertblock}{Error: \texttt{redeclaration of 'int x'}}
You have used 'int x =' more than once. After the declaration you only need 'x ='
\end{alertblock}
\begin{minted}[fontsize=\small, frame=lines]{cpp}
int x = 0;
.
.
.
x = 100;
\end{minted}
\begin{itemize}
    \item \textbf{Meaning:} You declared the same variable twice.
    \item \textbf{Fix:} Declare the type only once. After that, just use the variable name.
\end{itemize}
\end{frame}

\begin{frame}[fragile]
\frametitle{Error 3: Undeclared Variable}
\begin{alertblock}{Error: \texttt{'y' was not declared in this scope}}
You have used 'y =' without declaring a type. Fix this by adding a type (int, float, char, etc.)
\end{alertblock}
\begin{minted}[fontsize=\small, frame=lines]{cpp}
int y = 100;
\end{minted}
\begin{itemize}
    \item \textbf{Meaning:} You tried to use a variable before creating it.
    \item \textbf{Fix:} Make sure you have a declaration line like \texttt{int y;} before you try to use \texttt{y}.
\end{itemize}
\end{frame}

\begin{frame}[fragile]
\frametitle{Error 4: Wrong Quote Type for Char}
\begin{alertblock}{Error: \texttt{invalid conversion from 'const char*' to 'char'}}
You have used double quotes "a" instead of 'a' for a char datatype.
\end{alertblock}
\begin{minted}[fontsize=\small, frame=lines]{cpp}
char c = 'a';
\end{minted}
\begin{itemize}
    \item \textbf{Meaning:} You used double quotes (\texttt{"a"}) for a \texttt{char}.
    \item \textbf{Fix:} \texttt{char} types must use single quotes: \texttt{char c = 'a';}
\end{itemize}
\end{frame}

\section{Exercises}

\begin{frame}[fragile]
\frametitle{We Do: Predict the Output!}
\textbf{Answer File:} \texttt{02\_quizQuestions.cpp} (\textcolor{red}{HIDDEN - contains solutions})
\\What will the output be for each question?\pause
\begin{minted}[fontsize=\small, frame=lines, linenos, breaklines]{cpp}
int main()
{
   int x;
   // Question 1
   x = 99 / 20;
   cout << "Question 1:\t" << x << endl;
   // Question 2
   x = 27 % 10;
   cout << "Question 2:\t" << x << endl;
   // Question 3
   x = 100 / 10 / 2;
   cout << "Question 3:\t" << x << endl;
   return 0;
}
\end{minted}
\end{frame}

\begin{frame}[fragile]
\frametitle{We Do: Boolean Predictions}
\textbf{Answer File:} \texttt{02\_quizQuestions.cpp} (\textcolor{red}{HIDDEN - contains solutions})
\\What will the output be? Remember \texttt{true} is 1, \texttt{false} is 0.\pause
\begin{minted}[fontsize=\small, frame=lines, linenos, breaklines]{cpp}
int main()
{
   int x;

   // Question 4
   x = true;
   cout << "Question 4:\t" << x << endl;
   
   // Question 5
   x = false;
   cout << "Question 5:\t" << x << endl;
   
   // Question 6
   x = (99 == 99); // Is 99 equal to 99? This is true.
   cout << "Question 6:\t" << x << endl;
   return 0;
}
\end{minted}
\end{frame}

\begin{frame}[fragile]
\frametitle{We Do: Character Predictions}
\textbf{Answer File:} \texttt{02\_quizQuestions.cpp} (\textcolor{red}{HIDDEN - contains solutions})
\\Remember, characters are just numbers. What is the output here?\pause
\begin{minted}[fontsize=\small, frame=lines, linenos, breaklines]{cpp}
int main()
{
   int x;
   char c;

   // Question 7
   x = 'a'; // 'a' has ASCII value 97
   cout << "Question 7:\t" << x << endl;
   
   // Question 8
   x = 'z' - 'a'; // (122 - 97)
   cout << "Question 8:\t" << x << endl;

   // Question 9
   c = 100; // ASCII 100 is 'd'
   cout << "Question 9:\t" << c << endl;
   return 0;
}
\end{minted}
\end{frame}

\begin{frame}[fragile]
\frametitle{You Do: Integer Exercises}
\textbf{Template File:} \texttt{02\_quizQuestionsTemplate.cpp} (Student starting point)
\begin{block}{Your Turn}
Complete the integer exercises on Schoology. Use the following template to organize your solutions. Remember to declare a variable before you use it!
\end{block}
\begin{minted}[fontsize=\small, frame=lines, linenos]{cpp}
#include <iostream>
using namespace std;

int main()
{
   int x;   // only use "int x =" once, after that only use "x ="
   char c;

   // Question 0 (example)
   x = 1 + 1;
   cout << "Question 0:\t" << x << endl;

   // Your code for other questions goes here...

   return 0;
}
\end{minted}
\end{frame}

\begin{frame}[fragile]
\frametitle{Exercises: INT}
\textbf{Template File:} \texttt{02\_quizQuestionsTemplate.cpp} (Student template)
\begin{minted}[fontsize=\scriptsize, frame=lines, linenos, breaklines]{cpp}
#include <iostream>

using namespace std;

int main()
{
    int x;    // only use "int x =" once, after that only use "x = "
    char c;

    // Question 0 (example)
    x = 1 + 1;
    cout << "Question 0:\t" << x << endl;

    return 0;
}
\end{minted}
\end{frame}

\section{Homework and Challenge}

\begin{frame}
\frametitle{Homework}
\begin{block}{Complete the Exercises on Schoology}
\begin{itemize}
    \item I have included a short sample program on Schoology showing how you can organize your solutions.
    \item I am purposefully not giving you text you can copy/paste for the first few assignments.
    \item The goal is to improve your typing, especially finding the special characters like \texttt{\{\}}, \texttt{;}, \texttt{<}, \texttt{>}, etc. on the keyboard.
\end{itemize}
\end{block}
\end{frame}

\begin{frame}[fragile]
\frametitle{Challenge Question}
\begin{exampleblock}{Find the Range!}
Write C++ code to determine the range (smallest and largest values) for each of the following integer types:
\begin{enumerate}
    \item \texttt{unsigned int}
    \item \texttt{int}
    \item \texttt{short int}
\end{enumerate}
Note: These questions are intended as a way to challenge students who already have some programming experience. Don't worry if you have no idea how to do this, we'll get there.
\end{exampleblock}
\end{frame}

\begin{frame}[fragile]
\frametitle{Extension Question (Challenging)}
\textbf{Starter File:} \texttt{Challenge.cpp} (Partial solution)
\\\textbf{Complete Answer:} \texttt{02\_intMinMax.cpp} (\textcolor{red}{HIDDEN - advanced solution})
\\Here's a hint for the first one... what happens when you subtract 1 from 0 with an unsigned integer?
\pause
\begin{minted}[fontsize=\small, frame=lines]{cpp}
#include <iostream>
using namespace std;

int main()
{
   unsigned int a = 0;
   int b;
   short int c;
   
   cout << "The range of an unsigned int is: [0, " << a - 1 << "]\n";
   
   // these are incomplete, you will likely need loops to find these values
   cout << "The range of an int is: [" << 0 << ", " << 0 << "]\n";
   cout << "The range of a short int is: [" << 0 << " , " << 0 << "]\n";
   return 0;
}
\end{minted}
\end{frame}

\begin{frame}[fragile]
\frametitle{Loop Structures for Advanced Students}
For the challenge question, you may need these loop structures:
\begin{minted}[fontsize=\small, frame=lines]{cpp}
// Different loop structures
while(<test>)
{
}
//-----------------------------------------------------
do
{
}while(<test>); // this statement has the semicolon
//-----------------------------------------------------
for(int i = 0; i < 100; i = i + 1)
{
}
\end{minted}
\end{frame}


\end{document}
  