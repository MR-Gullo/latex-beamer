\documentclass{beamer}
% Use DS9 global theme (includes pgfplots for visualization)
\usepackage{../../../shared/templates/ds9_theme}
\usepackage{minted} % For code highlighting

% Title page configuration
\title[Integer Datatypes]{CS12 CH:02}
\subtitle{C++ Introduction \& Integer Datatypes}
\author[Mr. Gullo]{Mr. Gullo}
\date[\today]{\today}

\begin{document}
\frame{\titlepage}

\section{Comments in C++}

\begin{frame}
\frametitle{Learning Objectives}
\begin{itemize}
    \item Understand the purpose of comments in code.
    \item Learn the syntax for single-line (\texttt{//}) and multi-line (\texttt{/* ... */}) comments in C++.
    \item Use comments to document code and temporarily disable code for testing.
\end{itemize}
\end{frame}

\begin{frame}
\frametitle{What Are Comments?}
\begin{block}{For Human Eyes Only}
Comments are text in your code that is completely ignored by the compiler. They are meant for you and other programmers to read and understand the code better.
\end{block}

\begin{columns}[T]
    \begin{column}{0.5\textwidth}
        \begin{alertblock}{Single-Line Comments}
        Start with \texttt{//}. The compiler ignores everything from \texttt{//} to the end of the line.
        \end{alertblock}
    \end{column}
    \begin{column}{0.5\textwidth}
        \begin{exampleblock}{Multi-Line Comments}
        Start with \texttt{/*} and end with \texttt{*/}. Can span multiple lines. Useful for longer explanations or "commenting out" large blocks of code.
        \end{exampleblock}
    \end{column}
\end{columns}
\end{frame}

\begin{frame}[fragile]
\frametitle{Code Example: Comments}
\begin{minted}[fontsize=\small, frame=lines, linenos, breaklines]{cpp}
#include <iostream>

using namespace std;

// Use a double forward slash to create a single line comment

/*
    Use a forward slash and a star to make a multiline comment
    these are useful for large pieces of text or to remove sections
    from your code without deleting it.
*/

int main()
{
  // cout << "Hello World" << endl; // This line is disabled by a comment
    cout << "goodbye comments" << endl;
    return 0;
}
\end{minted}
\end{frame}

\section{Numeric Datatypes}

\begin{frame}
\frametitle{Learning Objectives}
\begin{itemize}
    \item Identify the fundamental integer-based datatypes in C++.
    \item Differentiate between \texttt{int}, \texttt{char}, and \texttt{bool}.
    \item Understand the kind of data each type is designed to store.
\end{itemize}
\end{frame}

\begin{frame}
\frametitle{Integer Datatypes (The Basics)}
Today we will look at three fundamental types that store whole numbers or concepts based on them.

\begin{columns}[T]
    \begin{column}{0.33\textwidth}
        \begin{alertblock}{int}
        Short for 'integer'. Stores positive and negative whole numbers.
        \vspace{1em}
        \textbf{Examples:} \texttt{23}, \texttt{19}, \texttt{-3}, \texttt{0}
        \end{alertblock}
    \end{column}
    \begin{column}{0.33\textwidth}
        \begin{exampleblock}{char}
        Short for 'character'. Stores a \textit{single} character.
        \vspace{1em}
        \textbf{Examples:} \texttt{'a'}, \texttt{'Z'}, \texttt{'\textbackslash n'}
        \end{exampleblock}
    \end{column}
    \begin{column}{0.33\textwidth}
        \begin{block}{bool}
        Short for 'Boolean'. Stores logical values.
        \vspace{1em}
        \textbf{Examples:} \texttt{true}, \texttt{false}
        \end{block}
    \end{column}
\end{columns}
\end{frame}

\section{The \texttt{int} Datatype}

\begin{frame}
\frametitle{Learning Objectives: \texttt{int}}
\begin{itemize}
    \item Recognize valid mathematical and comparison operators for the \texttt{int} datatype.
    \item Predict the outcome of integer division (\texttt{/}) and modulo (\texttt{\%}) operations.
    \item Declare and initialize integer variables.
\end{itemize}
\end{frame}

\begin{frame}
\frametitle{Operations on \texttt{int}s}
\begin{columns}[T]
    \begin{column}{0.5\textwidth}
        \begin{alertblock}{Binary Operations}
        \begin{itemize}
            \item \texttt{+} (Addition)
            \item \texttt{-} (Subtraction)
            \item \texttt{*} (Multiplication)
            \item \texttt{/} (Integer Division - no remainder!)
            \item \texttt{\%} (Modulo - gives the remainder)
        \end{itemize}
        \end{alertblock}
    \end{column}
    \begin{column}{0.5\textwidth}
        \begin{exampleblock}{Comparison Operations}
        \begin{itemize}
            \item \texttt{==} (Equal to)
            \item \texttt{!=} (Not equal to)
            \item \texttt{<} (Less than)
            \item \texttt{<=} (Less than or equal to)
            \item \texttt{>} (Greater than)
            \item \texttt{>=} (Greater than or equal to)
        \end{itemize}
        \end{exampleblock}
    \end{column}
\end{columns}
\end{frame}

\begin{frame}[fragile]
\frametitle{I Do: Integer Operations Demo}
Let's walk through this code and see what it does.
\begin{minted}[fontsize=\scriptsize, frame=lines, linenos, breaklines]{cpp}
#include <iostream>
using namespace std;
int main()
{
   int x = 34;
   int y = 5;

   // Integer addition
   cout << "x + y = " << x + y << endl;
   // Integer subtraction
   cout << "x - y = " << x - y << endl;
   // Integer multiplication
   cout << "x * y = " << x * y << endl;
   // Integer division (rounds down)
   cout << "x / y = " << x / y << endl;
   // Integer modulo division (remainder)
   cout << "x % y = " << x % y << endl;
   // Integer comparison ==
   cout << "x == y is " << (x == y) << endl;
   return 0;
}
\end{minted}
\end{frame}

\begin{frame}[fragile]
\frametitle{We Do: Predict the Output!}
What will the output be for each question?
\begin{minted}[fontsize=\small, frame=lines, linenos, breaklines]{cpp}
#include <iostream>
using namespace std;
int main()
{
   int x;
   // Question 1
   x = 99 / 20;
   cout << "Question 1:\t" << x << endl;

   // Question 2
   x = 27 % 10;
   cout << "Question 2:\t" << x << endl;

   // Question 3
   x = 100 / 10 / 2;
   cout << "Question 3:\t" << x << endl;

   return 0;
}
\end{minted}
\end{frame}

\begin{frame}[fragile]
\frametitle{You Do: Integer Exercises}
\begin{block}{Your Turn}
Complete the integer exercises on Schoology. Use the following template to organize your solutions. Remember to declare a variable before you use it!
\end{block}
\begin{minted}[fontsize=\small, frame=lines, linenos]{cpp}
#include <iostream>
using namespace std;

int main()
{
   int x;   // Declare variables here
   char c;

   // Question 0 (example)
   x = 1 + 1;
   cout << "Question 0:\t" << x << endl;

   // Your code for other questions goes here...

   return 0;
}
\end{minted}
\end{frame}

\section{The \texttt{char} Datatype}

\begin{frame}
\frametitle{Learning Objectives: \texttt{char}}
\begin{itemize}
    \item Understand that \texttt{char} variables store single characters using single quotes.
    \item Recognize that characters are represented by numerical ASCII values.
    \item Identify common special characters (escape sequences) like \texttt{\\n} and \texttt{\\t}.
\end{itemize}
\end{frame}

\begin{frame}
\frametitle{Characters and ASCII}
\begin{alertblock}{Characters are Numbers!}
A \texttt{char} stores a single character like \texttt{'a'} or \texttt{'5'}. But behind the scenes, the computer stores it as an integer code. The most common system is \textbf{ASCII} (American Standard Code for Information Interchange).
\end{alertblock}
\begin{exampleblock}{Example}
The character \texttt{'A'} is stored as the number 65.
\newline
The character \texttt{'a'} is stored as the number 97.
\newline
This means we can perform math on characters! \texttt{'a' - 32} would result in \texttt{'A'}.
\end{exampleblock}
\end{frame}

\begin{frame}
\frametitle{Common Special Characters}
Some characters are not printable. We use an "escape sequence" (a backslash followed by a letter) to represent them.
\begin{center}
\begin{tabular}{|c|l|}
\hline
\textbf{Sequence} & \textbf{Meaning} \\ \hline
\texttt{\textbackslash n} & Newline \\
\texttt{\textbackslash t} & Horizontal Tab \\
\texttt{\textbackslash\textbackslash} & Backslash \\
\texttt{\textbackslash'} & Single quote \\
\texttt{\textbackslash"} & Double quote \\ \hline
\end{tabular}
\end{center}
\end{frame}

\begin{frame}[fragile]
\frametitle{We Do: Character Predictions}
Remember, characters are just numbers. What is the output here?
\begin{minted}[fontsize=\small, frame=lines, linenos, breaklines]{cpp}
#include <iostream>
using namespace std;
int main()
{
   int x;
   char c;

   // Question 7
   x = 'a'; // 'a' has ASCII value 97
   cout << "Question 7:\t" << x << endl;

   // Question 8
   x = 'z' - 'a'; // (122 - 97)
   cout << "Question 8:\t" << x << endl;

   // Question 9
   c = 100; // ASCII 100 is 'd'
   cout << "Question 9:\t" << c << endl;
   return 0;
}
\end{minted}
\end{frame}

\section{The \texttt{bool} Datatype}

\begin{frame}
\frametitle{Learning Objectives: \texttt{bool}}
\begin{itemize}
    \item Understand the purpose of the \texttt{bool} datatype for representing logical states.
    \item Know the relationship between \texttt{true}/\texttt{false} and their integer representations \texttt{1}/\texttt{0}.
\end{itemize}
\end{frame}

\begin{frame}
\frametitle{Boolean Logic}
\begin{block}{\texttt{bool}: True or False}
A Boolean variable can only hold one of two values: \texttt{true} or \texttt{false}. They are the foundation of decision-making in programs.
\end{block}

\begin{alertblock}{Booleans are also Numbers!}
In C++, \texttt{true} is represented by the integer \texttt{1}, and \texttt{false} is represented by the integer \texttt{0}.
\end{alertblock}

\begin{exampleblock}{Important Note}
When evaluating a condition, C++ considers any non-zero integer to be \texttt{true} and only \texttt{0} to be \texttt{false}.
\end{exampleblock}
\end{frame}

\begin{frame}[fragile]
\frametitle{We Do: Boolean Predictions}
What will the output be? Remember \texttt{true} is 1, \texttt{false} is 0.
\begin{minted}[fontsize=\small, frame=lines, linenos, breaklines]{cpp}
#include <iostream>
using namespace std;
int main()
{
   int x;

   // Question 4
   x = true;
   cout << "Question 4:\t" << x << endl;

   // Question 5
   x = false;
   cout << "Question 5:\t" << x << endl;

   // Question 6
   x = (99 == 99); // Is 99 equal to 99? This is true.
   cout << "Question 6:\t" << x << endl;

   return 0;
}
\end{minted}
\end{frame}

\section{Common Errors}

\begin{frame}
\frametitle{Common Errors and How to Fix Them}
\begin{alertblock}{Error: \texttt{expected ';' before '\}' token}}
\begin{itemize}
    \item \textbf{Meaning:} You missed a semicolon at the end of a line.
    \item \textbf{Fix:} Look at the line \textit{before} the error number and add a \texttt{;}
\end{itemize}
\end{alertblock}

\begin{alertblock}{Error: \texttt{redeclaration of 'int x'}}
\begin{itemize}
    \item \textbf{Meaning:} You declared the same variable twice.
    \item \textbf{Fix:} Declare the type only once. After that, just use the variable name.\\
    Correct: \texttt{int x = 5; x = 10;}
\end{itemize}
\end{alertblock}

\begin{alertblock}{Error: \texttt{'y' was not declared in this scope}}
\begin{itemize}
    \item \textbf{Meaning:} You tried to use a variable before creating it.
    \item \textbf{Fix:} Make sure you have a declaration line like \texttt{int y;} before you try to use \texttt{y}.
\end{itemize}
\end{alertblock}

\begin{alertblock}{Error: \texttt{invalid conversion from 'const char*' to 'char'}}
\begin{itemize}
    \item \textbf{Meaning:} You used double quotes (\texttt{"a"}) for a \texttt{char}.
    \item \textbf{Fix:} \texttt{char} types must use single quotes: \texttt{char c = 'a';}
\end{itemize}
\end{alertblock}
\end{frame}

\section{Homework and Challenge}

\begin{frame}
\frametitle{Homework}
\begin{block}{Complete the Exercises on Schoology}
\begin{itemize}
    \item I have included a short sample program on Schoology showing how you can organize your solutions.
    \item I am purposefully not giving you text you can copy/paste for the first few assignments.
    \item The goal is to improve your typing, especially finding the special characters like \texttt{\{\}}, \texttt{;}, \texttt{<}, \texttt{>}, etc. on the keyboard.
\end{itemize}
\end{block}
\end{frame}

\begin{frame}[fragile]
\frametitle{Challenge Question}
\begin{exampleblock}{Find the Range!}
Write C++ code to determine the range (smallest and largest values) for each of the following integer types:
\begin{enumerate}
    \item \texttt{unsigned int}
    \item \texttt{int}
    \item \texttt{short int}
\end{enumerate}
Here's a hint for the first one... what happens when you subtract 1 from 0 with an unsigned integer?
\begin{minted}[fontsize=\small, frame=lines]{cpp}
#include <iostream>
using namespace std;
int main()
{
   unsigned int a = 0;
   cout << "The max of an unsigned int is: " << a - 1 << endl;
   return 0;
}
\end{minted}
\end{exampleblock}
\end{frame}


\end{document}
  