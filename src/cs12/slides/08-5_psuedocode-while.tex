\documentclass{beamer}
% Use DS9 global theme (includes pgfplots for visualization)
\usepackage{../../../shared/templates/ds9_theme}
\usepackage{minted} % For code highlighting
% Title page configuration
\title[CSP Pseudocode]{Computer Science Principles Pseudocode}
\subtitle{For C++ Students}
\author[Mr. Gullo]{Mr. Gullo}
\date[Nov 2025]{November 2025}
\begin{document}
\frame{\titlepage}
\begin{frame}
\frametitle{Learning Objectives}
After this lesson, you will be able to:\pause
\begin{itemize}
\item Explain what pseudocode is and why it is used\pause
\item Read and understand CSP pseudocode format\pause
\item Convert simple C++ code to CSP pseudocode\pause
\item Convert CSP pseudocode to C++ code\pause
\item Identify key differences between pseudocode and C++ syntax
\end{itemize}
\end{frame}
\section{Introduction to Pseudocode}
\begin{frame}
\frametitle{What is Pseudocode?}
\textbf{Pseudocode} is a way to write algorithms using simple, plain language.\pause
\vspace{0.3cm}
\textbf{Key Points:}\pause
\begin{itemize}
\item Not a real programming language\pause
\item Cannot run on a computer\pause
\item Easy to read and understand\pause
\item Focuses on logic, not syntax
\end{itemize}
\end{frame}
\begin{frame}
\frametitle{Why Learn CSP Pseudocode?}
\textbf{Reasons:}\pause
\begin{itemize}
\item Language-independent (works for any programming language)\pause
\item Easier to communicate ideas\pause
\item Less strict than real code\pause
\item Helps focus on problem-solving
\end{itemize}\pause
\vspace{0.3cm}
\textbf{Good News:} If you know C++, learning pseudocode is easy!
\end{frame}
\section{Basic Syntax}
\begin{frame}[fragile]
\frametitle{Variables and Assignment}
\textbf{CSP Pseudocode:}
\begin{minted}[fontsize=\small]{text}
x <- 5
name <- "Alice"
isStudent <- true
\end{minted}
\pause
\textbf{C++ Equivalent:}
\begin{minted}[fontsize=\small]{cpp}
int x = 5;
string name = "Alice";
bool isStudent = true;
\end{minted}
\pause
\vspace{0.3cm}
\alert{Key Difference:} Use $\leftarrow$ instead of =
\end{frame}

\begin{frame}[fragile]
\frametitle{Display Output}
\textbf{CSP Pseudocode:}
\begin{minted}[fontsize=\small]{text}
DISPLAY("Hello World")
DISPLAY(x)
DISPLAY("Your score is: ", score)
\end{minted}
\pause
\textbf{C++ Equivalent:}
\begin{minted}[fontsize=\small]{cpp}
cout << "Hello World" << endl;
cout << x << endl;
cout << "Your score is: " << score << endl;
\end{minted}
\pause
\vspace{0.3cm}
\alert{Key Difference:} Use DISPLAY() instead of cout
\end{frame}
\begin{frame}[fragile]
\frametitle{Input from User}
\textbf{CSP Pseudocode:}
\begin{minted}[fontsize=\small]{text}
age <- INPUT()
name <- INPUT()
\end{minted}
\pause
\textbf{C++ Equivalent:}
\begin{minted}[fontsize=\small]{cpp}
int age;
cin >> age;
string name;
cin >> name;
\end{minted}
\pause
\vspace{0.3cm}
\alert{Key Difference:} Use INPUT() instead of cin
\end{frame}
\begin{frame}[fragile]
\frametitle{Comments}
\textbf{CSP Pseudocode:}
\begin{minted}[fontsize=\small]{text}
// This is a comment
x <- 10  // Set x to 10
\end{minted}
\pause
\textbf{C++ Equivalent:}
\begin{minted}[fontsize=\small]{cpp}
// This is a comment
int x = 10;  // Set x to 10
\end{minted}
\pause
\vspace{0.3cm}
\alert{Same in both!} Use // for comments
\end{frame}
\section{Control Structures}
\begin{frame}[fragile]
\frametitle{IF Statements}
\textbf{CSP Pseudocode:}
\begin{minted}[fontsize=\small]{text}
IF (age >= 18)
{
DISPLAY("You can vote")
}
\end{minted}
\pause
\textbf{C++ Equivalent:}
\begin{minted}[fontsize=\small]{cpp}
if (age >= 18)
{
cout << "You can vote" << endl;
}
\end{minted}
\pause
\vspace{0.3cm}
\alert{Key Difference:} IF is uppercase, parentheses optional
\end{frame}
\begin{frame}[fragile]
\frametitle{IF-ELSE Statements}
\textbf{CSP Pseudocode:}
\begin{minted}[fontsize=\small]{text}
IF (score >= 60)
{
DISPLAY("Pass")
}
ELSE
{
DISPLAY("Fail")
}
\end{minted}
\pause
\textbf{C++ Equivalent:}
\begin{minted}[fontsize=\small]{cpp}
if (score >= 60)
{
cout << "Pass" << endl;
}
else
{
cout << "Fail" << endl;
}
\end{minted}
\pause
\vspace{0.3cm}
\alert{Key Difference:} IF and ELSE are uppercase, no semicolons
\end{frame}
\begin{frame}[fragile]
\frametitle{Comparison Operators}
\textbf{Same in Both:}\pause
\begin{itemize}
\item = (equal to)\pause
\item $\neq$ (not equal to)\pause
\item $<$ (less than)\pause
\item $>$ (greater than)\pause
\item $\leq$ (less than or equal to)\pause
\item $\geq$ (greater than or equal to)
\end{itemize}\pause

\vspace{0.3cm}
\textbf{Example:}\pause
\begin{minted}[fontsize=\small]{text}
IF (x = 5)        // equal
IF (y != 0)       // not equal
IF (age >= 18)    // greater or equal
\end{minted}
\end{frame}
\begin{frame}[fragile]
\frametitle{REPEAT UNTIL Loop}
\textbf{CSP Pseudocode:}
\begin{minted}[fontsize=\small]{text}
count <- 1
REPEAT UNTIL (count > 5)
{
DISPLAY(count)
count <- count + 1
}
\end{minted}
\pause
\textbf{C++ Equivalent:}
\begin{minted}[fontsize=\small]{cpp}
int count = 1;
while (count <= 5)
{
cout << count << endl;
count = count + 1;
}
\end{minted}
\pause
\vspace{0.3cm}
\alert{Key Difference:} REPEAT UNTIL is like while loop
\end{frame}
\begin{frame}[fragile]
\frametitle{REPEAT n TIMES Loop}
\textbf{CSP Pseudocode:}
\begin{minted}[fontsize=\small]{text}
REPEAT 5 TIMES
{
DISPLAY("Hello")
}
\end{minted}
\pause
\textbf{C++ Equivalent:}
\begin{minted}[fontsize=\small]{cpp}
for (int i = 0; i < 5; i++)
{
cout << "Hello" << endl;
}
\end{minted}
\pause
\vspace{0.3cm}
\alert{Key Difference:} Simpler syntax, no counter variable needed
\end{frame}
\begin{frame}[fragile]
\frametitle{FOR EACH Loop}
\textbf{CSP Pseudocode:}
\begin{minted}[fontsize=\small]{text}
numbers <- [10, 20, 30, 40]
FOR EACH num IN numbers
{
DISPLAY(num)
}
\end{minted}
\pause
\textbf{C++ Equivalent:}
\begin{minted}[fontsize=\small]{cpp}
int numbers[] = {10, 20, 30, 40};
for (int num : numbers)
{
cout << num << endl;
}
\end{minted}
\pause
\vspace{0.3cm}
\alert{Key Difference:} Very similar to C++ range-based for loop
\end{frame}

\section{Practice Exercises}
\begin{frame}[fragile]
\frametitle{Exercise 1: Variable Assignment}
\textbf{Convert this C++ code to CSP pseudocode:}
\begin{minted}[fontsize=\small]{cpp}
int age = 16;
string name = "Maria";
bool isStudent = true;
cout << "Name: " << name << endl;
cout << "Age: " << age << endl;
\end{minted}
\pause
\textbf{Answer:}\pause
\begin{minted}[fontsize=\small]{text}
age <- 16
name <- "Maria"
isStudent <- true
DISPLAY("Name: ", name)
DISPLAY("Age: ", age)
\end{minted}
\end{frame}
\begin{frame}[fragile]
\frametitle{Exercise 2: IF-ELSE Statement}
\textbf{Convert this C++ code to CSP pseudocode:}
\begin{minted}[fontsize=\small]{cpp}
int temperature = 75;
if (temperature >= 70)
{
cout << "It's warm" << endl;
}
else
{
cout << "It's cold" << endl;
}
\end{minted}
\pause
\textbf{Answer:}\pause
\begin{minted}[fontsize=\small]{text}
temperature <- 75
IF (temperature >= 70)
{
DISPLAY("It's warm")
}
ELSE
{
DISPLAY("It's cold")
}
\end{minted}
\end{frame}
\begin{frame}[fragile]
\frametitle{Exercise 3: Loop Conversion}
\textbf{Convert this C++ code to CSP pseudocode:}
\begin{minted}[fontsize=\small]{cpp}
for (int i = 1; i <= 3; i++)
{
cout << "Count: " << i << endl;
}
\end{minted}
\pause
\textbf{Answer (Option 1):}\pause
\begin{minted}[fontsize=\small]{text}
REPEAT 3 TIMES
{
DISPLAY("Count: ", i)
}
\end{minted}
\pause
\textbf{Answer (Option 2):}\pause
\begin{minted}[fontsize=\small]{text}
i <- 1
REPEAT UNTIL (i > 3)
{
DISPLAY("Count: ", i)
i <- i + 1
}
\end{minted}
\end{frame}

\begin{frame}
\frametitle{Common Mistakes to Avoid}
\begin{enumerate}
\item Using = instead of $\leftarrow$ for assignment\pause
\item Forgetting that lists start at index 1, not 0\pause
\item Writing lowercase (if, else, display) instead of uppercase\pause
\item Using semicolons (not needed in pseudocode)\pause
\item Using cout or cin instead of DISPLAY or INPUT\pause
\item Forgetting parentheses around conditions
\end{enumerate}\pause

\vspace{0.3cm}
\textbf{Tip:} When converting between C++ and pseudocode, focus on the logic, not the exact syntax!
\end{frame}
\begin{frame}
\frametitle{Quick Reference Guide}
\begin{tabular}{|l|l|}
\hline
\textbf{C++} & \textbf{CSP Pseudocode} \\
\hline
==
= (assignment) & $\leftarrow$ \\
cout << & DISPLAY() \\
cin >> & INPUT() \\
if & IF \\
else & ELSE \\
while & REPEAT UNTIL \\
for (fixed count) & REPEAT n TIMES \\
for (range) & FOR EACH \\
vector & list \\
.push\_back() & APPEND() \\
.size() & LENGTH() \\
\hline
\end{tabular}
\end{frame}

\begin{frame}
\frametitle{Summary}
\textbf{Key Takeaways:}\pause
\begin{itemize}
\item Pseudocode is easier to read than real code\pause
\item CSP pseudocode uses uppercase keywords\pause
\item Use $\leftarrow$ for assignment, not =\pause
\item Lists start at index 1 (not 0 like C++)\pause
\item DISPLAY replaces cout, INPUT replaces cin\pause
\item Control structures are similar but simpler\pause
\item Focus on logic, not exact syntax
\end{itemize}\pause

\vspace{0.3cm}
\textbf{Next Steps:} Practice converting between C++ and pseudocode!
\end{frame}
\end{document}
