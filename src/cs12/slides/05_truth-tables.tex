\documentclass{beamer}
% Use DS9 global theme (includes pgfplots for visualization)
\usepackage{../../../shared/templates/ds9_theme}

% Title page configuration
\title[Truth Tables and Logic]{CS12 CH: Truth Tables}
\subtitle{Logical Expressions and Truth Tables in Excel}
\author[Mr. Gullo]{Mr. Gullo}
\date[Dec 2024]{December 2024}

\begin{document}
\frame{\titlepage}

\begin{frame}
\frametitle{Learning Objectives}
After this lesson, you will be able to:
\begin{itemize}
    \item Understand the eight standard logical operators: NOT, AND, OR, XOR, NAND, NOR, IF, and IFF\pause
    \item Construct truth tables for logical expressions\pause
    \item Evaluate complex logical expressions using truth tables\pause
    \item Apply logical operators to solve real-world problems (e.g., leap year calculations)\pause
    \item Create truth tables in Excel with proper formatting and formulas\pause
    \item Use conditional formatting to visualize Boolean values
\end{itemize}
\end{frame}

\section{Introduction}

\begin{frame}
\frametitle{Review: If-Else Statements}
Last class we examined C++ if-else statements.\pause

Today we focus on the \alert{conditional part} --- the logical expressions that determine which branch executes.\pause

\textbf{Key Question:} How do we systematically evaluate complex logical conditions?\pause

\vspace{0.3cm}
\textbf{Answer:} Truth tables provide the systematic approach!
\end{frame}

\begin{frame}
\frametitle{Motivating Example: Leap Year}
To test if a year $y$ is a leap year, one of the following must be true:
\begin{enumerate}
    \item $y$ is divisible by 400, \alert{or}
    \item $y$ is divisible by 4 \alert{but not} divisible by 100
\end{enumerate}\pause

\vspace{0.3cm}
This involves multiple logical conditions combined with AND, OR, and NOT operators.\pause

\vspace{0.3cm}
\textbf{Truth tables} help us understand and evaluate these complex expressions systematically.
\end{frame}

\section{The Eight Standard Logical Operators}

\begin{frame}
\frametitle{Overview of Logical Operators}
We will examine 8 standard truth tables:\pause

\begin{columns}
\column{0.5\textwidth}
\textbf{Basic Operators:}
\begin{itemize}
    \item NOT\pause
    \item AND\pause
    \item OR\pause
    \item XOR
\end{itemize}

\column{0.5\textwidth}
\textbf{Advanced Operators:}
\begin{itemize}
    \item NAND\pause
    \item NOR\pause
    \item IF (Implication)\pause
    \item IFF (Biconditional)
\end{itemize}
\end{columns}

\vspace{0.3cm}
\alert{Note:} Don't worry about memorizing --- you'll have these as reference material.
\end{frame}

\begin{frame}
\frametitle{Interactive Gates Demo}
\textbf{Explore Logic Gates with Tinkercad!}

\vspace{0.3cm}
Visit this interactive circuit simulator to see logic gates in action:

\vspace{0.3cm}
\begin{center}
\url{https://www.tinkercad.com/things/iq7BxA6CfDq-copy-of-gates/editel?sharecode=-bJi8AUPsx77y4DyGFV2Jsf5slVafWCLiGEDvuhlpKQ}
\end{center}

\vspace{0.3cm}
\textbf{What you'll find:}
\begin{itemize}
    \item Interactive AND, OR, NOT, XOR, NAND, and NOR gates
    \item Real-time input/output visualization
    \item Hands-on experimentation with different input combinations
\end{itemize}

\vspace{0.3cm}
\alert{Tip:} Try different input combinations to verify the truth tables we're studying!
\end{frame}

\begin{frame}
\frametitle{NOT Operator}
\textbf{Negation:} Reverses the truth value of an expression.\pause

\begin{itemize}
    \item TRUE becomes FALSE\pause
    \item FALSE becomes TRUE
\end{itemize}\pause

\vspace{0.3cm}
\begin{center}
\begin{tabular}{|c|c|}
\hline
\textbf{p} & \textbf{NOT p} \\
\hline
TRUE & FALSE \\
\hline
FALSE & TRUE \\
\hline
\end{tabular}
\end{center}\pause

\vspace{0.3cm}
\textbf{Excel Formula:} \texttt{=NOT(A1)}
\end{frame}

\begin{frame}
\frametitle{AND Operator}
\textbf{Conjunction:} TRUE if \alert{both} expressions are TRUE.\pause

\vspace{0.3cm}
\begin{center}
\begin{tabular}{|c|c|c|}
\hline
\textbf{p} & \textbf{q} & \textbf{p AND q} \\
\hline
TRUE & TRUE & TRUE \\
\hline
TRUE & FALSE & FALSE \\
\hline
FALSE & TRUE & FALSE \\
\hline
FALSE & FALSE & FALSE \\
\hline
\end{tabular}
\end{center}\pause

\vspace{0.3cm}
\textbf{Excel Formula:} \texttt{=AND(A1,A2)}
\end{frame}

\begin{frame}
\frametitle{OR Operator}
\textbf{Disjunction:} TRUE if \alert{either} expression is TRUE.\pause

\vspace{0.3cm}
\begin{center}
\begin{tabular}{|c|c|c|}
\hline
\textbf{p} & \textbf{q} & \textbf{p OR q} \\
\hline
TRUE & TRUE & TRUE \\
\hline
TRUE & FALSE & TRUE \\
\hline
FALSE & TRUE & TRUE \\
\hline
FALSE & FALSE & FALSE \\
\hline
\end{tabular}
\end{center}\pause

\vspace{0.3cm}
\textbf{Excel Formula:} \texttt{=OR(A1,A2)}
\end{frame}

\begin{frame}
\frametitle{XOR Operator}
\textbf{Exclusive OR:} TRUE if either is TRUE \alert{but not both}.\pause

\vspace{0.3cm}
\begin{center}
\begin{tabular}{|c|c|c|}
\hline
\textbf{p} & \textbf{q} & \textbf{p XOR q} \\
\hline
TRUE & TRUE & FALSE \\
\hline
TRUE & FALSE & TRUE \\
\hline
FALSE & TRUE & TRUE \\
\hline
FALSE & FALSE & FALSE \\
\hline
\end{tabular}
\end{center}\pause

\vspace{0.3cm}
\textbf{Excel Formula:} \texttt{=XOR(A1,A2)}
\end{frame}

\begin{frame}
\frametitle{NAND Operator}
\textbf{Negation of AND:} The opposite of AND.\pause

\vspace{0.3cm}
\begin{center}
\begin{tabular}{|c|c|c|}
\hline
\textbf{p} & \textbf{q} & \textbf{p NAND q} \\
\hline
TRUE & TRUE & FALSE \\
\hline
TRUE & FALSE & TRUE \\
\hline
FALSE & TRUE & TRUE \\
\hline
FALSE & FALSE & TRUE \\
\hline
\end{tabular}
\end{center}\pause

\vspace{0.3cm}
\textbf{Excel Formula:} \texttt{=NOT(AND(A1,A2))}
\end{frame}

\begin{frame}
\frametitle{NOR Operator}
\textbf{Negation of OR:} The opposite of OR.\pause

\vspace{0.3cm}
\begin{center}
\begin{tabular}{|c|c|c|}
\hline
\textbf{p} & \textbf{q} & \textbf{p NOR q} \\
\hline
TRUE & TRUE & FALSE \\
\hline
TRUE & FALSE & FALSE \\
\hline
FALSE & TRUE & FALSE \\
\hline
FALSE & FALSE & TRUE \\
\hline
\end{tabular}
\end{center}\pause

\vspace{0.3cm}
\textbf{Excel Formula:} \texttt{=NOT(OR(A1,A2))}\pause

\alert{Note:} NOR doesn't exist in Excel, so use NOT(OR(...))
\end{frame}

\begin{frame}
\frametitle{IF Operator (Implication)}
\textbf{Conditional:} ``If p then q'' or ``p implies q'' (symbolically: $p \rightarrow q$).\pause

FALSE only when TRUE $\rightarrow$ FALSE\pause

\vspace{0.3cm}
\begin{center}
\begin{tabular}{|c|c|c|}
\hline
\textbf{p} & \textbf{q} & \textbf{p $\rightarrow$ q} \\
\hline
TRUE & TRUE & TRUE \\
\hline
TRUE & FALSE & FALSE \\
\hline
FALSE & TRUE & TRUE \\
\hline
FALSE & FALSE & TRUE \\
\hline
\end{tabular}
\end{center}\pause

\vspace{0.3cm}
\textbf{Excel Formula:} \texttt{=IF(AND(A1=TRUE,A2=FALSE), FALSE, TRUE)}
\end{frame}

\begin{frame}
\frametitle{IFF Operator (Biconditional)}
\textbf{If and Only If:} Symbolically $p \leftrightarrow q$\pause

TRUE if both expressions have the \alert{same} truth value.\pause

\vspace{0.3cm}
\begin{center}
\begin{tabular}{|c|c|c|}
\hline
\textbf{p} & \textbf{q} & \textbf{p $\leftrightarrow$ q} \\
\hline
TRUE & TRUE & TRUE \\
\hline
TRUE & FALSE & FALSE \\
\hline
FALSE & TRUE & FALSE \\
\hline
FALSE & FALSE & TRUE \\
\hline
\end{tabular}
\end{center}\pause

\vspace{0.3cm}
\textbf{Excel Formula:} \texttt{=IF(A1=A2, TRUE, FALSE)} or simply \texttt{=A1=A2}
\end{frame}

\section{Creating Truth Tables in Excel}

\begin{frame}
\frametitle{Excel Conditional Formatting}
To make truth tables easier to read, we use \alert{conditional formatting} to color TRUE and FALSE values.\pause

\vspace{0.3cm}
\textbf{Steps:}
\begin{enumerate}
    \item Select the cell(s) you want to format\pause
    \item Click Conditional Formatting $\rightarrow$ Highlight Cells Rules $\rightarrow$ Equal To...\pause
    \item Enter FALSE and choose a color (e.g., red/pink)\pause
    \item Repeat for TRUE with a different color (e.g., green)\pause
    \item Use Format Painter to copy formatting to other cells
\end{enumerate}\pause

\vspace{0.3cm}
\alert{Tip:} Double-click Format Painter to apply multiple times. Press ESC to cancel.
\end{frame}

\begin{frame}
\frametitle{Excel Conditional Formatting --- Visual}
\alert{[Screenshot showing Excel conditional formatting menu and steps]}

\vspace{0.5cm}
After setting up once, you can use the Format Painter to copy the formatting throughout your spreadsheet.
\end{frame}

\section{Best Practices and Tips}

\begin{frame}
\frametitle{Important Note on Complexity}
\textbf{Don't worry if this feels challenging!}\pause

This lesson is meant as an \alert{introduction} to logical expressions and truth tables.\pause

\vspace{0.3cm}
We won't pursue this topic extensively, so:\pause
\begin{itemize}
    \item Focus on understanding the basic concepts\pause
    \item Do your best with the exercises\pause
    \item Don't stress if you don't grasp everything immediately
\end{itemize}\pause

\vspace{0.3cm}
\textbf{For those interested in deeper study:}
\begin{enumerate}
    \item Deductions\pause
    \item Boolean algebra
\end{enumerate}
\end{frame}

\begin{frame}
\frametitle{Criteria for Your Exercises}
When completing truth table exercises:\pause

\begin{itemize}
    \item All values (except input variables p, q, r, s) should be \alert{calculated using Excel formulas}\pause
    \item Use plenty of \alert{intermediate steps} --- break complex expressions into smaller parts\pause
    \item In exercises 12-13, it's okay to use \texttt{NOT(OR(...))} instead of \texttt{NOR(...)} since Excel doesn't have a built-in NOR function\pause
    \item Assign intermediate steps their own column/letter for clarity
\end{itemize}
\end{frame}

\begin{frame}
\frametitle{Simplification Tip}
\textbf{Use intermediate steps!}\pause

You can assign intermediate results their own letter and refer to them later.\pause

\vspace{0.3cm}
\textbf{Example:}
\begin{itemize}
    \item Column C: $p \wedge q$ (label as ``r'')\pause
    \item Column D: $\neg r$ (uses result from Column C)
\end{itemize}\pause

\vspace{0.3cm}
This makes complex expressions:
\begin{itemize}
    \item Easier to build\pause
    \item Easier to debug\pause
    \item Easier to understand
\end{itemize}
\end{frame}

\section{Assignment}

\begin{frame}
\frametitle{Your Assignment: Truth Tables in Excel}
\textbf{Complete the truth table exercises using the provided template.}\pause

\vspace{0.3cm}
\textbf{Template File:} \texttt{student-truth-tables-template.pdf}\pause

\vspace{0.3cm}
\textbf{Requirements:}
\begin{itemize}
    \item Create your truth tables in Excel\pause
    \item Use conditional formatting for TRUE/FALSE values\pause
    \item Use formulas for all calculated columns\pause
    \item Include intermediate steps where helpful\pause
    \item Save your completed work
\end{itemize}\pause

\vspace{0.3cm}
\textbf{Submission:}

Please submit your Excel file as: \alert{firstnameLastname\_truthtables.xlsx}
\end{frame}

\begin{frame}
\frametitle{Summary}
Today we covered:\pause

\begin{itemize}
    \item The eight standard logical operators: NOT, AND, OR, XOR, NAND, NOR, IF, and IFF\pause
    \item How to construct and read truth tables\pause
    \item Creating truth tables in Excel with formulas\pause
    \item Using conditional formatting to visualize Boolean values\pause
    \item Best practices: use intermediate steps and clear labeling
\end{itemize}\pause

\vspace{0.3cm}
\textbf{Remember:} Truth tables are a systematic way to evaluate logical expressions --- essential for understanding complex conditionals in programming!
\end{frame}

\end{document}