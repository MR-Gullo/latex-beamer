\documentclass{beamer}

% Package imports
\usepackage{amsmath}
\usepackage{listings}
\usepackage{xcolor}
\usepackage{graphicx}

% Define custom Star Trek DS9 colors
\definecolor{ds9blue}{RGB}{25,25,112}
\definecolor{ds9gold}{RGB}{218,165,32}
\definecolor{ds9grey}{RGB}{105,105,105}
\definecolor{ds9red}{RGB}{178,34,34}

% Setup theme with custom colors
\usetheme{Madrid}
\usecolortheme{default}

% Color customizations
\setbeamercolor{palette primary}{bg=ds9blue,fg=white}
\setbeamercolor{palette secondary}{bg=ds9grey,fg=white}
\setbeamercolor{palette tertiary}{bg=ds9gold,fg=black}
\setbeamercolor{palette quaternary}{bg=ds9red,fg=white}
\setbeamercolor{structure}{fg=ds9blue}
\setbeamercolor{title}{fg=ds9gold}
\setbeamercolor{subtitle}{fg=ds9gold}
\setbeamercolor{frametitle}{bg=ds9blue,fg=white}
\setbeamercolor{block title}{bg=ds9blue,fg=white}
\setbeamercolor{block body}{bg=ds9grey!20,fg=black}

% Configure code listings
\lstset{
  language=C++,
  basicstyle=\ttfamily\small,
  keywordstyle=\color{ds9blue}\bfseries,
  stringstyle=\color{ds9red},
  commentstyle=\color{ds9grey}\itshape,
  numbers=left,
  numberstyle=\tiny\color{ds9grey},
  breaklines=true,
  showstringspaces=false,
  frame=single,
  rulecolor=\color{ds9blue}
}

% Title page configuration
\title[C++ Strings]{CS12: C++ Strings}
\subtitle{Fundamentals and Implementation}
\author[Mr. Gullo]{Mr. Gullo}
\date[March 2025]{March, 2025}
\institute{Computer Science Department}

\begin{document}

\begin{frame}
    \titlepage
\end{frame}

\begin{frame}
    \frametitle{Table of Contents}
    \tableofcontents
\end{frame}

\section{Introduction}

\begin{frame}
    \frametitle{Learning Objectives}
    By the end of this lesson, you will be able to:
    \begin{itemize}
        \item Understand what strings are in C++ and how they are stored in memory
        \item Declare, initialize, and manipulate string objects
        \item Access and modify individual characters in strings
        \item Implement common string operations such as:
        \begin{itemize}
            \item Counting specific characters
            \item Converting case (uppercase/lowercase)
            \item Checking for palindromes
            \item Finding substrings
        \end{itemize}
        \item Write functions that process and transform strings
    \end{itemize}
\end{frame}

\begin{frame}
    \frametitle{Why Strings Matter}
    \begin{itemize}
        \item Strings are fundamental for handling text data
        \item Used in almost every real-world application:
        \begin{itemize}
            \item User interfaces
            \item File I/O
            \item Data processing
            \item Network communication
        \end{itemize}
        \item Understanding strings is essential for programming
        \item Many algorithms and problems involve string manipulation
    \end{itemize}
    
\end{frame}

\section{C++ String Basics}

\begin{frame}[fragile]
    \frametitle{String Declaration and Initialization}
    
    In C++, strings are objects of the \texttt{std::string} class:
    
    \begin{lstlisting}
#include <string>  // Required for using strings
using namespace std;  // Avoid in header files!

// Different ways to declare strings:
string str1;          // Empty string
string str2 = "Hello";  // Initialize with value
string str3("World");   // Constructor initialization
string str4 = str2;     // Copy from another string
    \end{lstlisting}
    
    \begin{block}{Important Notes}
        \begin{itemize}
            \item Always include the string header
            \item In header files, use \texttt{std::string} instead of \texttt{using namespace std}
            \item Strings are mutable (can be changed)
        \end{itemize}
    \end{block}
\end{frame}

\begin{frame}[fragile]
    \frametitle{Basic String Operations}
    
    \begin{lstlisting}
// Example 01: Basic string
string str1 = "Hello World\n";
cout << str1;  // Outputs: Hello World

// Example 02: Copying a string
string str2 = str1;
cout << str2;  // Outputs: Hello World

// Example 03: Concatenation
string str3 = str1 + ", happy Monday!" + "\n";
cout << str3;  // Outputs: Hello World, happy Monday!

// Example 04: String length
int str3Len = str3.size();  // or str3.length()
cout << "str3 length: " << str3Len << endl;
    \end{lstlisting}
\end{frame}

\begin{frame}[fragile]
    \frametitle{String Input}
    
    Two main ways to read strings:
    
    \begin{lstlisting}
// Using cin (stops at whitespace)
string name;
cout << "Enter your name: ";
cin >> name;  // Only reads until first space

// Using getline (reads entire line)
string fullText;
cout << "Enter a line of text: ";
getline(cin, fullText);  // Reads until newline

// Important: When mixing cin >> and getline()
cin >> name;
cin.ignore();  // Clear the newline character
getline(cin, fullText);
    \end{lstlisting}
    
    \begin{alertblock}{Common Pitfall}
        Always use \texttt{cin.ignore()} when switching from \texttt{cin >>} to \texttt{getline()}
    \end{alertblock}
\end{frame}

\section{String Manipulation}

\begin{frame}[fragile]
    \frametitle{Accessing Characters}
    
    Strings can be accessed character by character:
    
    \begin{lstlisting}
string text = "Hello";

// Access individual characters with indexing
char firstChar = text[0];  // 'H'
char lastChar = text[4];   // 'o'

// Modify characters
text[0] = 'J';  // Changes to "Jello"

// Iterate through all characters
for(int i = 0; i < text.size(); i++) {
    cout << text[i];  // Print each character
}
    \end{lstlisting}
    
    \begin{block}{Remember}
        \begin{itemize}
            \item Indexing starts at 0
            \item Be careful not to access beyond string length
        \end{itemize}
    \end{block}
\end{frame}

\begin{frame}[fragile]
    \frametitle{Understanding ASCII Values}
    
    Characters in C++ are represented by ASCII values:
    

        \begin{lstlisting}
char c = 'A';
int value = c;  // 65

// Uppercase letters: 65-90 (A-Z)
// Lowercase letters: 97-122 (a-z)
// Digits: 48-57 (0-9)

// Difference between cases
int diff = 'a' - 'A';  // 32
        \end{lstlisting}
  
    
    \begin{block}{Converting Case}
        \begin{itemize}
            \item To uppercase: subtract 32 from lowercase
            \item To lowercase: add 32 to uppercase
            \item Or use library functions: \texttt{toupper()}, \texttt{tolower()}
        \end{itemize}
    \end{block}
\end{frame}

\begin{frame}[fragile]
    \frametitle{Case Conversion Example}
    
    Implementation of case conversion functions:
    
    \begin{lstlisting}
// Convert to uppercase (returns new string)
string toUpper(string s) {
    int conversion = 'a' - 'A';
    for(int i = 0; i < s.length(); i++) {
        if(s[i] >= 'a' && s[i] <= 'z') {
            s[i] = s[i] - conversion;
        }    }
    return s;
}
// Convert to lowercase (modifies original)
void toLower(string &s) {
    int conversion = 'a' - 'A';
    for(int i = 0; i < s.length(); i++) {
        if(s[i] >= 'A' && s[i] <= 'Z') {
            s[i] = s[i] + conversion;
        }    }}
    \end{lstlisting}
\end{frame}

\section{Implementing String Functions}


\section{Advanced String Operations}

\begin{frame}[fragile]
    \frametitle{String Reversal}
    
    Implementing the \texttt{reverseString} function:
    
    \begin{lstlisting}
// Returns s, but in reverse order
string reverseString(string s) {
    string reversed = "";    
    // Method 1: Build a new string from back to front
    for(int i = s.length() - 1; i >= 0; i--) {
        reversed += s[i];
    }    
    /* Method 2: In-place reversal
    string reversed = s;
    int n = s.length();
    for(int i = 0; i < n/2; i++) {
        char temp = reversed[i];
        reversed[i] = reversed[n-1-i];
        reversed[n-1-i] = temp;
    }
    */    
    return reversed;
}
    \end{lstlisting}
\end{frame}

\begin{frame}[fragile]
    \frametitle{Finding Substrings}
    
    Implementing the \texttt{isSubstring} function:
    
    \begin{lstlisting}
// Returns true if s_full contains the string s_sub
bool isSubstring(string s_full, string s_sub) {
    // Method 1: Using string's find method
    if(s_full.find(s_sub) != string::npos) {
        return true;
    }
    return false;
        /* Method 2: Manual implementation
    for(int i = 0; i <= s_full.length() - s_sub.length(); i++) {
        bool match = true;
        for(int j = 0; j < s_sub.length(); j++) {
            if(s_full[i+j] != s_sub[j]) {
                match = false;
                break;
            }        }
        if(match) return true;
    }
    return false;
    */}
    \end{lstlisting}
\end{frame}

\begin{frame}[fragile]
    \frametitle{String Method Reference}
    
    Common string methods:
    
    \begin{columns}
        \column{0.5\textwidth}
        \begin{itemize}
            \item \texttt{s.length()} / \texttt{s.size()} - Returns string length
            \item \texttt{s.empty()} - Checks if string is empty
            \item \texttt{s.clear()} - Clears the string
            \item \texttt{s.substr(pos, len)} - Returns substring
            \item \texttt{s.find(str)} - Finds position of substring
        \end{itemize}
        
        \column{0.5\textwidth}
        \begin{itemize}
            \item \texttt{s.replace(pos, len, str)} - Replaces part of string
            \item \texttt{s.insert(pos, str)} - Inserts at position
            \item \texttt{s.erase(pos, len)} - Erases characters
            \item \texttt{s.append(str)} - Appends to end
            \item \texttt{s.compare(str)} - Compares strings
        \end{itemize}
    \end{columns}
    
    \begin{alertblock}{Note}
        The \texttt{string} class has many more methods. Check the C++ documentation for details.
    \end{alertblock}
\end{frame}

\section{Practical Exercise}

\begin{frame}[fragile]
    \frametitle{Capitalizing Words}
    
    Implementing the \texttt{capitalizeWords} function:
    
    \begin{lstlisting}
// Capitalizes the first letter of each word
string capitalizeWords(string s) {
    string result = s;    
    // Capitalize first character if it's a letter
    if(!result.empty() && isalpha(result[0])) {
        result[0] = toupper(result[0]);
    }    
    // Check each character
    for(int i = 1; i < result.length(); i++) {
        // If previous character is a space, hyphen, or period
        if(result[i-1] == ' ' || result[i-1] == '-' || 
           result[i-1] == '.') {
            if(isalpha(result[i])) {
                result[i] = toupper(result[i]);
            }        }}
        return result;}
    \end{lstlisting}
\end{frame}

\begin{frame}
    \frametitle{Summary}
    
    \begin{itemize}
        \item C++ strings are objects that store sequences of characters
        \item String operations include:
        \begin{itemize}
            \item Declaration and initialization
            \item Input/output
            \item Character access and manipulation
            \item Finding and modifying substrings
        \end{itemize}
        \item Implementing string functions requires:
        \begin{itemize}
            \item Understanding character representation (ASCII)
            \item Iteration through characters
            \item String manipulation techniques
        \end{itemize}
        \item Practice with the provided header file to strengthen your skills
    \end{itemize}
\end{frame}

\begin{frame}
    \frametitle{Practice Assignment}
    
    Implement all functions in the provided header file:
    
    \begin{itemize}
        \item \texttt{myName()} - Returns your name
        \item \texttt{countChar()} - Counts occurrences of a character
        \item \texttt{countVowels()} - Counts vowels in a string
        \item \texttt{countNumbers()} - Counts numeric characters
        \item \texttt{longestWord()} - Finds length of longest word
        \item \texttt{capitalizeWords()} - Capitalizes first letter of each word
        \item \texttt{changeCase()} - Inverts case of each letter
        \item \texttt{reverseString()} - Reverses a string
        \item \texttt{isPalindrome()} - Checks if string is a palindrome
        \item \texttt{isSubstring()} - Checks if string contains substring
    \end{itemize}
    
    \alert{[Submit through schoology by the due date]}
\end{frame}

\end{document}