\documentclass{beamer}
% Use DS9 global theme (includes pgfplots for visualization)
\usepackage{../../../../shared/templates/ds9_theme}

% Title page configuration
\title[Floats and Memory]{CS12 CH:Floats, Memory, and Input}
\subtitle{Data Types, Memory Size, and User Input}
\author[Mr. Gullo]{Mr. Gullo}
\date[Sep 15, 2025]{September 15, 2025}

\begin{document}
\frame{\titlepage}

\begin{frame}
\frametitle{Learning Objectives}
\begin{itemize}
    \item Understand and use the \texttt{float} data type for numbers with decimals.
    \item Differentiate between integer division and floating-point division.
    \item Define fundamental memory concepts: Bit and Byte.
    \item Use the \texttt{sizeof()} operator to determine the memory footprint of various data types.
    \item Understand binary (base-2) numbers and how to represent them in C++.
    \item Use \texttt{cin} to get input from a user via the console.
\end{itemize}
\end{frame}

\section{The float Data Type}

\begin{frame}
\frametitle{Key Concept: Floating-Point Numbers}
\begin{itemize}
    \item In programming, numbers with decimal parts are called \alert{floating-point numbers}.
    \item C++ provides the \texttt{float} data type to store these values.
    \item Declaration and initialization is similar to integers:
\end{itemize}
\begin{block}{Example Syntax}
\texttt{float pi = 3.14159;} \\
\texttt{float price = 0.95;}
\end{block}
\begin{itemize}
    \item Floats support standard arithmetic operations: addition, subtraction, multiplication, and division.
    \item \alert{Modulo division (\%)} is not supported for floating-point types.
\end{itemize}
\end{frame}

\begin{frame}
\frametitle{Essential Equations: Integer vs. Float Division}
The data type of your numbers dictates the type of division C++ performs.

\begin{itemize}
    \item \textbf{Integer Division}: If both operands are integers, the result is an integer. Any fractional part is \alert{truncated} (cut off).
    \begin{itemize}
        \item \[ \frac{5}{4} \rightarrow 1 \]
    \end{itemize}
    \pause
    \item \textbf{Floating-Point Division}: If at least one operand is a float, the result is a float, preserving the decimal.
    \begin{itemize}
        \item \[ \frac{5.0}{4} \rightarrow 1.25 \]
    \end{itemize}
\end{itemize}
\pause
\vspace{1em}
This is one of the most common sources of bugs for new programmers!
\end{frame}

\begin{frame}[fragile]
\frametitle{Code Demo: Division in Action}
\textbf{Demo File:} \texttt{03\_intDivision.cpp} (Interactive - comprehensive demo)
\newline
\vspace{1em}
Let's examine how C++ handles different division scenarios.

\begin{minted}[fontsize=\scriptsize, frame=lines, linenos, breaklines]{cpp}
#include <iostream>

using namespace std;

int main()
{
   float a = 5;
   float b = 4;
   float c = 5/4; // Integer division occurs *before* assignment!

   cout << "5/4 = " << 5/4 << endl;         // Integer division
   cout << "c = " << c << endl;             // Result of prior integer division
   cout << "5.0/4 = " << 5.0/4 << endl;     // Floating-point division
   cout << "5/4.0 = " << 5/4.0 << endl;     // Floating-point division
   cout << "a/b = " << a/b << endl;         // Floating-point division (vars)

   return 0;
}
\end{minted}
\end{frame}

\section{Memory and Data Types}

\begin{frame}
\frametitle{Key Concepts: Bits and Bytes}
All data in a computer is stored as binary digits, or bits.

\begin{itemize}
    \item \textbf{Bit}:
    \begin{itemize}
        \item The smallest unit of data in a computer.
        \item Can have a value of either \alert{0} or \alert{1}.
    \end{itemize}
    \pause
    \item \textbf{Byte}:
    \begin{itemize}
        \item A group of \alert{8 bits}.
        \item A common unit for measuring computer memory size.
        \item One byte can represent 256 different values (from 0 to 255).
    \end{itemize}
\end{itemize}
\end{frame}

\begin{frame}
\frametitle{Context: Visualizing a Byte}
The terms "bit" and "byte" can be abstract. To make this concrete, the next slide visualizes how 8 individual bits come together to form a single byte, the fundamental unit used to measure the size of data types like \texttt{int} and \texttt{char}.
\end{frame}

\begin{frame}
\frametitle{Visualization: 8 Bits in 1 Byte}
\centering
\begin{figure}
\centering
\begin{tikzpicture}[scale=0.9, transform shape]
    % Draw the outer Byte box
    \node[draw, thick, ds9blue, minimum width=8.4cm, minimum height=1.4cm, label={[ds9gold]above:\textbf{1 Byte}}] (byte) at (0,0) {};

    % Draw the 8 Bit boxes inside
    \foreach \i [count=\xi from 0] in {7,6,5,4,3,2,1,0} {
        \node[draw, thick, minimum size=1cm] (bit\i) at (-3.5+\xi, 0) {$2^\i$};
    }
\end{tikzpicture}
\caption{A byte is a sequence of 8 bits.}
\end{figure}
\end{frame}

\begin{frame}
\frametitle{Key Concept: The \texttt{sizeof()} Operator}
Different data types require different amounts of memory to store their values.

\begin{itemize}
    \item C++ has a built-in operator called \texttt{sizeof()} that tells you how much memory (in \alert{bytes}) a data type or variable occupies.
    \item This can vary slightly between computer architectures (e.g., 32-bit vs. 64-bit systems).
\end{itemize}
\pause
\begin{block}{Syntax Examples}
\texttt{sizeof(int)} \hfill \textit{// Returns the size of an integer} \\
\texttt{sizeof(myAge)} \hfill \textit{// Returns the size of the variable myAge}
\end{block}
\end{frame}

\begin{frame}
\frametitle{Context: Visualizing Data Type Sizes}
Running a program to see the output of \texttt{sizeof()} is useful, but a graph can help us instantly compare the memory footprint of different data types. The next slide shows a bar chart of common data types and their typical sizes in bytes on a 64-bit system.
\alert{
\textbf{File:} \texttt{03\_datatypesSizes.cpp} (Sizes)}
\end{frame}

\begin{frame}
\frametitle{Visualization: Typical Data Type Sizes}
\begin{figure}
\begin{tikzpicture}
\begin{axis}[
    ybar,
    bar width=20pt,
    width=\textwidth,
    height=0.7\textheight,
    enlarge x limits=0.2,
    xlabel={Data Type},
    ylabel={Size in Bytes},
    symbolic x coords={bool, char, int, float, double},
    xtick=data,
    nodes near coords,
    nodes near coords align={vertical},
    ymin=0,
    axis line style={-},
    tickwidth=0pt,
    label style={font=\bfseries},
    tick label style={font=\small},
    ymajorgrids=true,
    grid style=dashed,
]
\addplot[fill=ds9blue, draw=none] coordinates {
    (bool,1) (char,1) (int,4) (float,4) (double,8)
};
\end{axis}
\end{tikzpicture}
\end{figure}
\end{frame}


\section{Number Systems}

\begin{frame}
\frametitle{Key Concept: Binary Numbers}
\begin{itemize}
    \item We typically use the \textbf{decimal} (base-10) number system, which has ten digits (0-9).
    \item Computers use the \textbf{binary} (base-2) number system, which has only two digits (0 and 1).
    \item A number's base indicates how many digits are available.
    \begin{itemize}
        \item Decimal: $827_{10} = 8 \times 10^2 + 2 \times 10^1 + 7 \times 10^0$
        \item Binary: $101_2 = 1 \times 2^2 + 0 \times 2^1 + 1 \times 2^0 = 4 + 0 + 1 = 5_{10}$
    \end{itemize}
\end{itemize}
\end{frame}

\begin{frame}[fragile]
\frametitle{Concept: Binary Literals in C++}
\begin{itemize}
    \item You can write numbers in binary directly in your C++ code by using the \texttt{0b} prefix.
    \item When you print the number, C++ will automatically display it in its decimal (base-10) representation.
\end{itemize}
\vspace{1em}
\textbf{Code Example:}
\begin{minted}[fontsize=\small, frame=lines, linenos, breaklines]{cpp}
#include <iostream>

int main()
{
   std::cout << "0b1010011 = " << 0b1010011 << std::endl;
   return 0;
}
\end{minted}
\pause
\begin{block}{Terminal Output}
\texttt{0b1010011 = 83}
\end{block}
\end{frame}

\section{User Input}

\begin{frame}[fragile]
\frametitle{Key Concept: Getting Input with \texttt{cin}}
\begin{itemize}
    \item To make programs interactive, we need a way to get input from the user.
    \item In C++, we use the \texttt{cin} object (part of \texttt{<iostream>}) for this.
    \item The extraction operator \texttt{>>} is used to get data from the console and store it in a variable.
\end{itemize}
\pause
\begin{block}{Example: Reading an integer}
\begin{minted}[fontsize=\scriptsize, frame=lines, breaklines]{cpp}
int age; // Declare a variable to store the age

cout << "Please enter your age: "; // Prompt the user
cin >> age; // Read input from the keyboard into 'age'

cout << "You are " << age << " years old." << endl;
\end{minted}
\end{block}
\end{frame}

\section{Code Quality}

\begin{frame}
\frametitle{Why Does Code Formatting Matter?}
\begin{itemize}
    \item \textbf{Readability}: Code is read far more often than it is written. Consistent formatting makes it easier for you (and others) to understand what the code is doing.
    \pause
    \item \textbf{Collaboration}: When working in a team, a shared style guide prevents confusion and makes code reviews more efficient. Everyone is on the same page.
    \pause
    \item \textbf{Maintainability}: It's easier to find bugs and add new features to well-formatted code. Messy code hides problems.
    \pause
    \item \textbf{Professionalism}: Just like good grammar and spelling in an essay, good formatting is a sign of a careful and professional programmer.
\end{itemize}
\vspace{1em}
\begin{alertblock}{The Golden Rule}
Write your code as if the person who has to maintain it is a violent psychopath who knows where you live.
\end{alertblock}
\end{frame}

\begin{frame}[fragile]
\frametitle{Common C++ Formatting Rules}
While style guides vary, most agree on a few key principles:

\begin{columns}[T]
\begin{column}{0.5\textwidth}
    \textbf{Bad (Inconsistent)}
    \begin{minted}[fontsize=\tiny, frame=lines]{cpp}
#include <iostream>
int main(){
int x=5;int y=10;
if(x<y){
std::cout<<"x is smaller"<<std::endl;
}
return 0;}
    \end{minted}
\end{column}
\begin{column}{0.5\textwidth}
    \textbf{Good (Consistent)}
    \begin{minted}[fontsize=\tiny, frame=lines]{cpp}
#include <iostream>

int main() {
    int x = 5;
    int y = 10;

    if (x < y) {
        std::cout << "x is smaller" << std::endl;
    }

    return 0;
}
    \end{minted}
\end{column}
\end{columns}

\begin{itemize}
    \item \textbf{Indentation}: Use a consistent number of spaces (e.g., 4) for each level of nesting.
    \item \textbf{Spacing}: Use spaces around operators (`=`, `+`, `<`) to improve readability.
    \item \textbf{Brace Style}: Pick one style for your curly braces ("\{\}") and stick with it.
\end{itemize}
\end{frame}

\section{Practice with U-P-E-R}

\begin{frame}
\frametitle{The U-P-E-R Problem Solving Method}
\begin{block}{What is U-P-E-R?}
A structured approach to solving programming problems:
\begin{itemize}
    \item \textbf{U - Understand}: Analyze the problem, identify inputs/outputs, and work through examples
    \item \textbf{P - Plan}: Design the logic, identify variables, and create pseudocode
    \item \textbf{E - Execute}: Write the actual code based on your plan
    \item \textbf{R - Review}: Test your code, check for errors, and verify correctness
\end{itemize}
\end{block}
\pause
\begin{block}{Why Use U-P-E-R?}
\begin{itemize}
    \item Breaks complex problems into manageable steps
    \item Prevents jumping straight to coding without proper planning
    \item Encourages systematic testing and debugging
    \item Builds good programming habits for real-world development
\end{itemize}
\end{block}
\end{frame}

\begin{frame}
\frametitle{I Do: Grade Calculator - Understand}
\textbf{Problem:} Write a program that asks for the total possible score on a test, then calculates and displays the minimum score required to earn grades from A to F based on predefined percentages.

\vspace{1em}
\textbf{U - Understand the Problem}
\begin{itemize}
    \item \textbf{Goal:} Calculate grade cutoffs based on a total score.
    \item \textbf{Inputs:} One integer for the total possible score.
    \item \textbf{Outputs:} Three sentences, each stating the required score for a specific grade (A, B, C-).
    \item \textbf{Example:} If input is 100, output for an A (86\%) should be 86. If input is 200, output for an A should be 172.
\end{itemize}
\end{frame}

\begin{frame}
\frametitle{I Do: Grade Calculator - Plan}
\textbf{P - Plan the Logic}
\begin{itemize}
    \item \textbf{Variables:}
    \begin{itemize}
        \item \texttt{int totalScore;} to store user input.
        \item Use constants for percentages to avoid "magic numbers":
        \item \texttt{const int GRADE\_A = 86;}
        \item \texttt{const int GRADE\_B = 73;}
        \item \texttt{const int GRADE\_C\_MINUS = 50;}
    \end{itemize}
    \item \textbf{Steps (Pseudocode):}
    \begin{enumerate}
        \item Display a prompt asking for the total possible score.
        \item Read the user's input into the \texttt{totalScore} variable.
        \item Calculate the cutoff for an 'A': \texttt{totalScore * 86 / 100}.
        \item Print the result for 'A'.
        \item Repeat calculation and printing for 'B' (73\%) and 'C-' (50\%).
    \end{enumerate}
\end{itemize}
\end{frame}

\begin{frame}[fragile]
\frametitle{I Do: Grade Calculator - Execute \& Review}
\textbf{File:} \texttt{03\_grades.cpp} (Answer Key)
\vspace{1em}
\textbf{E - Execute (Write the Code)}
\begin{minted}[fontsize=\tiny, frame=lines, linenos]{cpp}
#include <iostream>
using namespace std;
// Grade cutoffs
const int GRADE_A = 86;
const int GRADE_B = 73;
const int GRADE_C_MINUS = 50;

int main() {
   int totalScore;
   cout << "Enter total possible score: ";
   cin >> totalScore;

   cout << "For an A, a mark of " << totalScore * GRADE_A / 100 << " is required." << endl;
   cout << "For a B, a mark of " << totalScore * GRADE_B / 100 << " is required." << endl;
   cout << "For a C-, a mark of " << totalScore * GRADE_C_MINUS / 100 << " is required." << endl;
   return 0;
}
\end{minted}
\pause
\textbf{R - Review and Test}
\begin{itemize}
    \item Compile and run. Does it build without errors?
    \item Test with example: Input 100. Output is 86, 73, 50. Correct.
    \item Test with another value: Input 200. Output is 172, 146, 100. Correct.
    \item What happens if we use floats? The result would be more precise, but here integer truncation is acceptable.
\end{itemize}
\end{frame}

\begin{frame}
\frametitle{We Do: Arithmetic Sequence - Understand \& Plan}
\textbf{Problem (Q5a):} Write a code chunk that prompts for $n$ and displays the $n^{th}$ number in the sequence: $11, 15, 19, 23, \dots$

\vspace{1em}
\textbf{U - Understand}
\begin{itemize}
    \item \textbf{Goal:} Find the value of a term in a sequence.
    \item \textbf{Inputs:} The term number, $n$.
    \item \textbf{Outputs:} A sentence showing the term and its value.
    \item \textbf{Example:} If $n=1$, output is 11. If $n=3$, output is 19.
\end{itemize}
\pause
\textbf{P - Plan}
\begin{itemize}
    \item \textbf{Variables:} \texttt{int n;} for input, \texttt{int termValue;} for result.
    \item \textbf{Formula:} The $n^{th}$ term of an arithmetic sequence is $a_n = a + (n-1)d$.
    \item Here, first term $a=11$ and common difference $d=4$.
\end{itemize}
\end{frame}

\begin{frame}[fragile]
\frametitle{We Do: Arithmetic Sequence - Execute \& Review}
\textbf{E - Execute the Plan} \\
Based on our plan, how do we translate the formula $a_n = 11 + (n-1) \times 4$ into C++?

\begin{minted}[fontsize=\small, frame=lines, breaklines]{cpp}
#include <iostream>
using namespace std;

int main() {
    int n;
    cout << "Enter the term number you want to find: ";
    cin >> n;

    // Calculate the nth term using the formula
    int termValue = _______________________; // What goes here?

    cout << "Term " << n << " is " << termValue << endl;
    return 0;
}
\end{minted}
\pause
\textbf{R - Review}
\begin{itemize}
    \item Once we fill in the blank, we'll test with our examples.
    \item If we input 3, does \texttt{11 + (3-1)*4} give us 19? Yes.
\end{itemize}
\end{frame}

\begin{frame}
\frametitle{You Do: Arithmetic Series Sum}
\textbf{Problem (Q5b):} Write a code chunk that prompts for $n$ and displays the \textit{sum} of the first $n$ terms in the series: $2 + 5 + 8 + 11 + \dots$

\vspace{1em}
\begin{alertblock}{Your Task: Use the U-P-E-R Method}
\begin{enumerate}
    \item \textbf{Understand}: What are the inputs/outputs? Work out an example for $n=3$ (sum should be $2+5+8=15$).
    \item \textbf{Plan}: What variables do you need? What is the formula for the sum of an arithmetic series? ($S_n = \frac{n}{2}(2a + (n-1)d)$).
    \item \textbf{Execute}: Translate your plan and formula into C++.
    \item \textbf{Review}: Test your code with your example case. Does it work?
\end{enumerate}
\end{alertblock}
\end{frame}




\begin{frame}
\frametitle{Homework Submission: 03 First C++ Calculations }

\begin{alertblock}{Instructions}
\begin{itemize}
    \item Submit your completed Jupyter Notebook file named: \\ \texttt{firstnameLastname\_floats.ipynb}.
    \pause
    \item Complete all questions in Parts 1 through 6 from the lesson notebook.
    \pause
    \item You are encouraged to collaborate with your peers, but the work you submit must be your own.
    \pause
    \item Please make an effort to organize and format your document similarly to the sample provided in class.
\end{itemize}
\end{alertblock}

\pause

\begin{block}{Grading Breakdown}
    \begin{itemize}
        \item Content Completion (Parts 1-6): \textbf{6 pts}
        \item Formatting and Structure: \textbf{1 pt}
    \end{itemize}
\end{block}

\end{frame}

\begin{frame}
\frametitle{Summary}
\begin{itemize}
    \item The \texttt{float} data type is used for numbers with decimal points.
    \item Division with two integers results in an \alert{integer} (truncation). If a \texttt{float} is involved, the result is a \texttt{float}.
    \item The \texttt{sizeof()} operator returns the memory size of a data type in \alert{bytes}.
    \item Computers store data using the \alert{binary} (base-2) system. In C++, you can denote a binary number with the \texttt{0b} prefix.
    \item \texttt{cin >> variable;} is the standard way to read user input from the console.
    \item The \alert{U-P-E-R} method provides a structured approach to solving programming problems.
\end{itemize}
\end{frame}

\end{document}