\documentclass{beamer}
% Use DS9 global theme (includes pgfplots for visualization)
\usepackage{../../../shared/templates/ds9_theme}
\usepackage{minted} % For code highlighting

% Title page configuration
\title[Floats and Memory]{PHYS11 CH:Lesson 3}
\subtitle{Floats, Byte-Size, and Type-Conversion}
\author[Mr. Gullo]{Mr. Gullo}
\date[Sep 10, 2025]{September 10, 2025}

\begin{document}
\frame{\titlepage}

\begin{frame}
\frametitle{Learning Objectives}
\begin{itemize}
    \item Understand and use the \texttt{float} data type for numbers with decimals.
    \item Differentiate between integer division and floating-point division.
    \item Define fundamental memory concepts: Bit and Byte.
    \item Use the \texttt{sizeof()} operator to determine the memory footprint of various data types.
    \item Understand binary (base-2) numbers and how to convert them to decimal (base-10).
    \item Use \texttt{cin} to get input from a user via the console.
\end{itemize}
\end{frame}

\section{The float Data Type}

\begin{frame}
\frametitle{Key Concept: Floating-Point Numbers}
\begin{itemize}
    \item In programming, numbers with decimal parts are called \alert{floating-point numbers}.
    \item C++ provides the \texttt{float} data type to store these values.
    \item Declaration and initialization is similar to integers:
\end{itemize}
\begin{block}{Example Syntax}
\texttt{float pi = 3.14159;} \\
\texttt{float price = 0.95;}
\end{block}
\begin{itemize}
    \item Floats support standard arithmetic operations: addition, subtraction, multiplication, and division.
    \item \alert{Modulo division (\%)} is generally not supported for floating-point types.
\end{itemize}
\end{frame}

\begin{frame}
\frametitle{Concept: Integer vs. Floating-Point Division}
When you perform division in C++, the type of the numbers involved matters greatly.
\begin{itemize}
    \item \textbf{Integer Division}: If you divide two integers, C++ performs integer division. This means the result is also an integer, and any fractional part is \alert{truncated} (cut off, not rounded).
    \begin{itemize}
        \item Example: \texttt{5 / 4} evaluates to \texttt{1}.
    \end{itemize}
    \pause
    \item \textbf{Floating-Point Division}: If at least one of the numbers in the division is a float, C++ performs floating-point division, and the result is a float.
    \begin{itemize}
        \item Example: \texttt{5.0 / 4} evaluates to \texttt{1.25}.
    \end{itemize}
\end{itemize}
\vspace{1em}
Next, we will look at a code demo to see this in action.
\end{frame}

\begin{frame}[fragile]
\frametitle{Visualization: Division in Action}
\textbf{Demo File:} \texttt{03\_intDivision.cpp} (Interactive - comprehensive demo)
\newline
\vspace{1em}
Let's examine how C++ handles different division scenarios.

\begin{minted}[fontsize=\scriptsize, frame=lines, linenos, breaklines]{cpp}
#include <iostream>

using namespace std;

int main()
{
   float a = 5;
   float b = 4;
   float c = 5/4; // What will c be?

   cout << "5/4 = " << 5/4 << endl;         // Integer division
   cout << "c = " << c << endl;             // Result of integer division
   cout << "5.0/4 = " << 5.0/4 << endl;     // Floating-point division
   cout << "5/4.0 = " << 5/4.0 << endl;     // Floating-point division
   cout << "a/b = " << a/b << endl;         // Floating-point division (vars)

   return 0;
}
\end{minted}
\pause
\begin{block}{Terminal Output}
\texttt{5/4 = 1} \newline
\texttt{c = 1} \newline
\texttt{5.0/4 = 1.25} \newline
\texttt{5/4.0 = 1.25} \newline
\texttt{a/b = 1.25}
\end{block}
\end{frame}

\section{Memory and Data Types}

\begin{frame}
\frametitle{Key Concepts: Bits and Bytes}
All data in a computer is stored as binary digits, or bits.

\begin{itemize}
    \item \textbf{Bit}:
    \begin{itemize}
        \item The smallest unit of data in a computer.
        \item Can have a value of either \alert{0} or \alert{1}.
    \end{itemize}
    \pause
    \item \textbf{Byte}:
    \begin{itemize}
        \item A group of \alert{8 bits}.
        \item A common unit for measuring computer memory size.
        \item One byte can represent 256 different values (from 0 to 255).
    \end{itemize}
\end{itemize}
\end{frame}

\begin{frame}
\frametitle{Key Concept: The \texttt{sizeof()} Operator}
Different data types require different amounts of memory to store their values.

\begin{itemize}
    \item C++ has a built-in operator called \texttt{sizeof()} that tells you how much memory (in bytes) a data type or variable occupies.
    \item This is useful for understanding memory usage and can vary slightly between different computer architectures (e.g., 32-bit vs. 64-bit systems).
\end{itemize}
\pause
\begin{block}{Syntax Examples}
\texttt{sizeof(int)} \hfill \textit{// Returns the size of an integer} \\
\texttt{sizeof(float)} \hfill \textit{// Returns the size of a float} \\
\texttt{int myAge = 30;} \\
\texttt{sizeof(myAge)} \hfill \textit{// Returns the size of the variable myAge}
\end{block}
\end{frame}

\begin{frame}[fragile]
\frametitle{Visualization: Data Type Sizes}
This program demonstrates how to use \texttt{sizeof()} to check the memory allocation for common data types.

\textbf{Demo File:} \texttt{03\_dataTypesSizes.cpp} (Demo, also Answer Key for Q3.d)
\begin{minted}[fontsize=\scriptsize, frame=lines, linenos, breaklines]{cpp}
#include <iostream>
using namespace std;

int main()
{
    cout << "Integers:\n";
    cout << "Data Type    Bytes\n"
        << "---------     -----" << endl
        << "int           " << sizeof(int) << endl
        << "char          " << sizeof(char) << endl
        << "bool          " << sizeof(bool) << endl;

    cout << "\nFloating-Point:\n";
    cout << "Data Type    Bytes\n"
        << "---------     -----" << endl
        << "float         " << sizeof(float) << endl
        << "double        " << sizeof(double) << endl;
  return 0;
}
\end{minted}
\pause
\begin{block}{Typical Terminal Output}
\texttt{Integers:} \newline
\texttt{Data Type    Bytes} \newline
\texttt{---------     -----} \newline
\texttt{int           4} \newline
\texttt{char          1} \newline
\texttt{bool          1} \newline
\newline
\texttt{Floating-Point:} \newline
\texttt{Data Type    Bytes} \newline
\texttt{---------     -----} \newline
\texttt{float         4} \newline
\texttt{double        8}
\end{block}
\end{frame}

\section{Number Systems}

\begin{frame}
\frametitle{Key Concept: Binary Numbers}
\begin{itemize}
    \item We typically use the \textbf{decimal} (base-10) number system, which has ten digits (0-9).
    \item Computers use the \textbf{binary} (base-2) number system, which has only two digits (0 and 1).
    \item A number's base indicates how many digits are available.
    \begin{itemize}
        \item Decimal: $827_{10} = 8 \times 10^2 + 2 \times 10^1 + 7 \times 10^0$
        \item Binary: $101_2 = 1 \times 2^2 + 0 \times 2^1 + 1 \times 2^0 = 4 + 0 + 1 = 5_{10}$
    \end{itemize}
\end{itemize}
\end{frame}

\begin{frame}[fragile]
\frametitle{Concept: Binary Literals in C++}
\begin{itemize}
    \item You can write numbers in binary directly in your C++ code by using the \texttt{0b} prefix.
    \item When you print the number, C++ will automatically display it in its decimal (base-10) representation.
\end{itemize}
\vspace{1em}
\textbf{Demo File Snippet from:} \texttt{03\_intDivision.cpp}
\begin{minted}[fontsize=\scriptsize, frame=lines, linenos, breaklines]{cpp}
#include <iostream>

int main()
{
   std::cout << "0b1010011 = " << 0b1010011 << std::endl;
   return 0;
}
\end{minted}
\pause
\begin{block}{Terminal Output}
\texttt{0b1010011 = 83}
\end{block}
\end{frame}

\section{User Input}

\begin{frame}
\frametitle{Key Concept: Getting Input with \texttt{cin}}
\begin{itemize}
    \item To make programs interactive, we need a way to get input from the user.
    \item In C++, we use the \texttt{cin} object (part of \texttt{<iostream>}) for this.
    \item The extraction operator \texttt{>>} is used to get data from the console and store it in a variable.
\end{itemize}
\pause
\begin{block}{Example: Reading an integer}
\begin{minted}[fontsize=\scriptsize, frame=lines, breaklines]{cpp}
int age; // Declare a variable to store the age

cout << "Please enter your age: "; // Prompt the user
cin >> age; // Read input from the keyboard into 'age'

cout << "You are " << age << " years old." << endl;
\end{minted}
\end{block}
\end{frame}

\section{Practice Problems}

\begin{frame}[fragile]
\frametitle{I Do: Grade Calculator}
\textbf{Problem:} Write a program that asks for the total possible score on a test, then calculates and displays the minimum score required to earn grades from A to F based on predefined percentages.
\vspace{1em}

\begin{columns}[T]
\begin{column}{0.4\textwidth}
    \textbf{G - Givens}
    \begin{itemize}
        \item Grade A = 86\%
        \item Grade B = 73\%
        \item Grade C+ = 67\%
        \item Grade C = 60\%
        \item Grade C- = 50\%
        \item User input for total score.
    \end{itemize}
    \textbf{U - Unknown}
    \begin{itemize}
        \item The score cutoff for each letter grade.
    \end{itemize}
    \textbf{E - Equation/Logic}
    \begin{itemize}
        \item \small{\texttt{cutoff = total * \% / 100}}
    \end{itemize}
\end{column}
\begin{column}{0.6\textwidth}
    \textbf{S - Substitute \& Solve}
    \newline
    \textbf{Demo File:} \texttt{03\_grades.cpp}
    \begin{minted}[fontsize=\tiny, frame=lines, linenos]{cpp}
#include <iostream>
using namespace std;

const int GRADE_A = 86;
const int GRADE_B = 73;
const int GRADE_C_MINUS = 50;

int main() {
   int totalScore;
   cout << "Enter total possible score: ";
   cin >> totalScore;

   cout << "For an A, a mark of " 
        << totalScore * GRADE_A / 100 
        << " is required."  << endl;
   cout << "For a B, a mark of " 
        << totalScore * GRADE_B / 100 
        << " is required."  << endl;
   cout << "For a C-, a mark of " 
        << totalScore * GRADE_C_MINUS / 100
        << " is required."  << endl;
   return 0;
}
    \end{minted}
\end{column}
\end{columns}
\end{frame}


\begin{frame}[fragile]
\frametitle{We Do: Arithmetic Sequence Term}
\textbf{Problem (Q5a):} Write a code chunk that prompts the user for $n$ and displays the $n^{th}$ number in the arithmetic sequence: $11, 15, 19, 23, \dots$
\vspace{1em}

\textbf{G - Givens}
\begin{itemize}
    \item First term, $a = 11$.
    \item Common difference, $d = 4$.
    \item The term number, $n$, will be provided by the user.
\end{itemize}
\pause
\textbf{U - Unknown}
\begin{itemize}
    \item The value of the $n^{th}$ term.
\end{itemize}
\pause
\textbf{E - Equation}
\begin{itemize}
    \item The formula for the $n^{th}$ term of an arithmetic sequence is:
    \[ a_n = a + (n-1)d \]
\end{itemize}
\pause
\textbf{S - Substitute \& Solve (Code)}
\begin{minted}[fontsize=\scriptsize, frame=lines, breaklines]{cpp}
int n;
cout << "Enter the term number you want to find: ";
cin >> n;

// Calculate the nth term using the formula
int term = _______________________; // What goes here?

cout << "Term " << n << " is " << term << endl;
\end{minted}
\end{frame}

\begin{frame}
\frametitle{You Do: Arithmetic Series Sum}
\textbf{Problem (Q5b):} Write a code chunk that prompts the user for $n$ and displays the \textit{sum} of the first $n$ terms in the following arithmetic series:
\[ 2 + 5 + 8 + 11 + \dots \]
\vspace{1em}

\begin{alertblock}{Your Task}
Use the GUESS method to plan your solution.
\begin{enumerate}
    \item Identify the Givens (first term, common difference).
    \item Identify the Unknown (the sum).
    \item Find the correct Equation for the sum of an arithmetic series.
    \item Substitute your values into the formula and implement it in C++.
\end{enumerate}
\end{alertblock}
\end{frame}

\begin{frame}
\frametitle{Summary}
\begin{itemize}
    \item The \texttt{float} data type is used for numbers with decimal points.
    \item Division with two integers results in an \alert{integer} (truncation). If a \texttt{float} is involved, the result is a \texttt{float}.
    \item The \texttt{sizeof()} operator returns the memory size of a data type in \alert{bytes}.
    \item Computers store data using the \alert{binary} (base-2) system. In C++, you can denote a binary number with the \texttt{0b} prefix.
    \item \texttt{cin >> variable;} is the standard way to read user input from the console.
\end{itemize}
\end{frame}

\end{document}