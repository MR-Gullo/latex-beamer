\documentclass{beamer}
% Required packages
\usepackage{amsmath}
\usepackage{physics}
\usepackage{graphicx}
\usepackage{siunitx}
\usepackage{xcolor}
% Define custom colors for DS9 theme
\definecolor{ds9blue}{RGB}{25,25,112}
\definecolor{ds9gold}{RGB}{218,165,32}
\definecolor{ds9grey}{RGB}{105,105,105}
\definecolor{ds9red}{RGB}{178,34,34}
% Set up the Madrid theme with custom colors
\usetheme{Madrid}
\usecolortheme{whale}
\setbeamercolor{palette primary}{bg=ds9blue,fg=white}
\setbeamercolor{palette secondary}{bg=ds9grey,fg=white}
\setbeamercolor{palette tertiary}{bg=ds9gold,fg=black}
\setbeamercolor{palette quaternary}{bg=ds9red,fg=white}
\setbeamercolor{structure}{fg=ds9blue}
\setbeamercolor{title}{fg=ds9gold}
\setbeamercolor{subtitle}{fg=ds9gold}
\setbeamercolor{frametitle}{bg=ds9blue,fg=white}
\setbeamercolor{block title}{bg=ds9blue,fg=white}
\setbeamercolor{block body}{bg=ds9grey!20,fg=black}

\title[Note Taking]{Note Taking and Asking Questions}
\subtitle{A Guide to Active Learning}
\author[Mr. Gullo]{Mr. Gullo}
\date[Sep 8, 2025]{September 8, 2025}

\begin{document}

\frame{\titlepage}

\section{Introduction: The Policy on Notes}

\begin{frame}
\frametitle{A Quick Note on My Notes Policy}
\begin{block}{My Policy}
    I do not provide copies of my lecture notes.
\end{block}

\begin{block}{Why?}
    Taking your own notes is one of the most important skills for learning physics (and any other subject!).
    \vspace{1cm}

    This presentation will explain why this skill is so important and give you tips to become a great note-taker.
\end{block}
\end{frame}

\section{Skill Building: Effective Note Taking}

\begin{frame}
\frametitle{Why Take Your Own Notes?}
\begin{itemize}
    \item \textbf{It keeps you awake and focused.} The act of writing keeps your brain active during the lesson.
    \pause
    \item \textbf{It helps you remember.} You are more likely to remember what you write down yourself.
    \pause
    \item \textbf{It helps you understand.} You have to process and simplify the information to write it down.
    \pause
    \item \textbf{You create your own study guide.} Your notes are made by you, for you. They will make sense to you when you study later.
\end{itemize}
\vfill

\end{frame}

\begin{frame}
\frametitle{How to Take Good Notes}
\begin{itemize}
    \item \textbf{Don't write everything.} Listen for the main ideas, definitions, and equations.
    \pause
    \item \textbf{Use short sentences.} You can use symbols and abbreviations to write faster (e.g., `w/` for with, `\&` for and).
    \pause
    \item \textbf{Copy all diagrams and examples.} Physics is a visual subject. Diagrams and example problems are very important.
    \pause
    \item \textbf{Leave space.} Leave empty space on your page to add more details or questions later.
    \pause
    \item \textbf{Review your notes soon after class.} Read your notes the same day. This helps you remember the information and fix any parts that are confusing.
\end{itemize}
\end{frame}

\section{Active Learning: The Power of Questions}

\begin{frame}
\frametitle{Asking Questions is Smart!}
Feeling confused is normal! Asking questions is the best way to learn.

\begin{block}{Please Ask Questions!}
\begin{itemize}
    \item If you have a question, it is very likely other students have the same question. You are helping everyone by asking!
    \pause
    \item There are no "stupid" questions in this class.
    \pause
    \item If you are shy, write your question down in your notes. You can ask me after class or during office hours.
\end{itemize}
\end{block}
\vfill
\end{frame}

\section{Summary}

\begin{frame}
\frametitle{Summary: Your Path to Success}
\begin{enumerate}
    \item \textbf{Come to class prepared to listen and write.}
    \item \textbf{Focus on writing down key ideas}, not every single word.
    \item \textbf{Always copy examples and diagrams.} They are the key to understanding physics.
    \item \textbf{Ask questions!} It is the fastest way to clear up confusion and learn more deeply.
\end{enumerate}
\end{frame}

\end{document}