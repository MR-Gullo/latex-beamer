\documentclass{beamer}
% Use DS9 global theme (includes pgfplots for visualization)
\usepackage{../../../shared/templates/ds9_theme}

% Title page configuration
\title[Number Systems and Math]{CS12 CH:4: Number Systems and Math}
\subtitle{Binary, Hexadecimal, and the Math Library}
\author[Mr. Gullo]{Mr. Gullo}
\date[Sep 19, 2025]{September 19, 2025}

\begin{document}
\frame{\titlepage}

\section{Introduction}

\begin{frame}
\frametitle{Learning Objectives}
\begin{itemize}
    \item Convert numbers between decimal (base-10), binary (base-2), and hexadecimal (base-16).
    \item Explain why hexadecimal is a convenient representation for binary data.
    \item Perform basic binary addition.
    \item Apply mathematical functions from the \texttt{<cmath>} library to solve complex problems.
\end{itemize}
\end{frame}

\section{Number Systems}

\begin{frame}
\frametitle{Review: Base-10 (Decimal) vs. Base-2 (Binary)}
\begin{columns}[T]
    \begin{column}{0.5\textwidth}
        \begin{block}{Decimal (Base-10)}
            \begin{itemize}
                \item The system we use every day.
                \item Uses ten digits (0-9).
                \item Each place value is a power of 10.
                \item Example: $827_{10}$ is\\ $8 \times 10^2 + 2 \times 10^1 + 7 \times 10^0$
            \end{itemize}
        \end{block}
    \end{column}
    \begin{column}{0.5\textwidth}
        \begin{block}{Binary (Base-2)}
            \begin{itemize}
                \item The system computers use.
                \item Uses two digits (0 and 1).
                \item Each place value is a power of 2.
                \item Example: $101_2$ is\\ $1 \times 2^2 + 0 \times 2^1 + 1 \times 2^0 = 5_{10}$
            \end{itemize}
        \end{block}
    \end{column}
\end{columns}
\end{frame}

\begin{frame}
\frametitle{Key Concept: Hexadecimal (Base-16)}
\begin{itemize}
    \item A base-16 number system used as a \alert{human-friendly representation} of binary.
    \item Uses 16 symbols: \textbf{0-9} and \textbf{A-F}.
    \item \textbf{A} represents 10, \textbf{B} is 11, ..., \textbf{F} is 15.
    \item Why use it? It's compact. One hex digit represents a group of four binary digits (a "nibble").
\end{itemize}

\begin{exampleblock}{Example}
It's much easier to read and write \texttt{0xAF} than \texttt{10101111}.
\end{exampleblock}

\begin{alertblock}{Hexadecimal in C++}
You can write a hexadecimal number in C++ by prefixing it with \texttt{0x}.
\end{alertblock}
\texttt{int myNum = 0x1A; // Same as int myNum = 26;}
\end{frame}

\begin{frame}
\frametitle{Context: Visualizing Binary to Hexadecimal}
\begin{itemize}
    \item The key to converting between binary and hexadecimal is to think in groups of four.
    \item A group of 4 bits is often called a \alert{nibble}.
    \item A nibble can represent $2^4 = 16$ different values (from 0 to 15).
    \item This maps perfectly to the 16 symbols in hexadecimal (0 to F).
    \item On the next slide, we will see a direct mapping from every possible 4-bit binary value to its corresponding hexadecimal digit.
\end{itemize}
\end{frame}

\begin{frame}
\frametitle{Visualization: Binary to Hexadecimal Mapping}
\begin{center}
\begin{tikzpicture}
\node[draw=ds9blue, thick, rounded corners, inner sep=10pt] {
\begin{tabular}{ccc|ccc}
\hline
\textbf{Hex} & \textbf{Decimal} & \textbf{Binary} & \textbf{Hex} & \textbf{Decimal} & \textbf{Binary} \\
\hline
\texttt{0} & 0  & \texttt{0000} & \texttt{8} & 8  & \texttt{1000} \\
\texttt{1} & 1  & \texttt{0001} & \texttt{9} & 9  & \texttt{1001} \\
\texttt{2} & 2  & \texttt{0010} & \texttt{A} & 10 & \texttt{1010} \\
\texttt{3} & 3  & \texttt{0011} & \texttt{B} & 11 & \texttt{1011} \\
\texttt{4} & 4  & \texttt{0100} & \texttt{C} & 12 & \texttt{1100} \\
\texttt{5} & 5  & \texttt{0101} & \texttt{D} & 13 & \texttt{1101} \\
\texttt{6} & 6  & \texttt{0110} & \texttt{E} & 14 & \texttt{1110} \\
\texttt{7} & 7  & \texttt{0111} & \texttt{F} & 15 & \texttt{1111} \\
\hline
\end{tabular}
};
\end{tikzpicture}
\end{center}
\end{frame}

\begin{frame}
\frametitle{Converting Binary to Decimal}
To convert a binary number to decimal, multiply each binary digit by its corresponding power of 2 and sum the results.

\begin{exampleblock}{Example: Convert $1011001_2$ to decimal}
\begin{align*}
1011001_2 &= (1 \times 2^6) + (0 \times 2^5) + (1 \times 2^4) \\
&\quad + (1 \times 2^3) + (0 \times 2^2) + (0 \times 2^1) + (1 \times 2^0) \\
&= (1 \times 64) + (0 \times 32) + (1 \times 16) \\
&\quad + (1 \times 8) + (0 \times 4) + (0 \times 2) + (1 \times 1) \\
&= 64 + 0 + 16 + 8 + 0 + 0 + 1 \\
&= 89_{10}
\end{align*}
\end{exampleblock}
\end{frame}

\begin{frame}
\frametitle{Binary in C++}
\begin{alertblock}{Binary in C++}
You can write a binary number in C++ by prefixing it with \texttt{0b}.
\end{alertblock}
\texttt{int myNum = 0b1011001; // Same as int myNum = 89;}

\begin{alertblock}{Hexadecimal in C++}
You can write a hexadecimal number in C++ by prefixing it with \texttt{0x}.
\end{alertblock}
\texttt{int myNum = 0x1A; // Same as int myNum = 26;}
\end{frame}

\begin{frame}
\frametitle{Converting Decimal to Binary}
To convert a decimal number to binary, repeatedly divide the decimal number by 2 and record the remainders. The binary result is the sequence of remainders read in \alert{reverse order}.

\begin{exampleblock}{Example: Convert $29_{10}$ to binary}
\begin{tabular}{r c l}
 $29 / 2$ & = 14 & Remainder \textbf{1} \\
 $14 / 2$ & = 7  & Remainder \textbf{0} \\
 $7 / 2$  & = 3  & Remainder \textbf{1} \\
 $3 / 2$  & = 1  & Remainder \textbf{1} \\
 $1 / 2$  & = 0  & Remainder \textbf{1} \\
\end{tabular}
\vfill
Reading the remainders from bottom to top gives: \alert{$11101_2$}.
\end{exampleblock}
\end{frame}

\begin{frame}
\frametitle{Number System Operations: Binary Addition}
Binary addition follows simple rules, similar to decimal addition.

\begin{block}{Rules for Binary Addition}
\begin{itemize}
    \item $0 + 0 = 0$
    \item $0 + 1 = 1$
    \item $1 + 0 = 1$
    \item $1 + 1 = 0$, carry the 1
    \item $1 + 1 + 1 = 1$, carry the 1
\end{itemize}
\end{block}

\begin{exampleblock}{Example: Add $1011_2 + 0110_2$}
\begin{center}
\begin{tabular}{r}
\texttt{  111  (carry bits)} \\
\texttt{  1011 (11 decimal)} \\
\texttt{+ 0110 ( 6 decimal)} \\
\texttt{------} \\
\texttt{ 10001 (17 decimal)}
\end{tabular}
\end{center}
\end{exampleblock}
\end{frame}

\section{C++ Math Library}

\begin{frame}
\frametitle{Advanced Math with \texttt{<cmath>}}
\begin{itemize}
    \item C++ provides a library for advanced mathematical functions.
    \item To use it, you must include the header file:
\end{itemize}
\begin{block}{C++ Include Directive}
\texttt{\#include <cmath>}
\end{block}
\begin{itemize}
    \item It provides many useful functions, including:
    \begin{itemize}
        \item \texttt{sqrt(x)} - Calculates the square root of \texttt{x}.
        \item \texttt{pow(base, exp)} - Raises \texttt{base} to the power of \texttt{exp}.
        \item \texttt{exp(x)} - Calculates the value of $e^x$.
        \item And many more like \texttt{sin()}, \texttt{cos()}, \texttt{log()}, etc.
    \end{itemize}
\end{itemize}
\end{frame}


\begin{frame}
\frametitle{Essential Equations}
Here are the key formulas for our upcoming problems.

\begin{block}{Distance Formula}
The distance between two points $(x_1, y_1)$ and $(x_2, y_2)$ is:
$$ d = \sqrt{(x_2 - x_1)^2 + (y_2 - y_1)^2} $$
\end{block}

\begin{block}{Radioactive Decay}
The amount of a radioactive isotope remaining after $t$ years is:
$$ remaining = (original) \times e^{-0.00012t} $$
where $e \approx 2.71828$. Note: In C++, you can use \texttt{exp(x)} for $e^x$.
\end{block}

\begin{block}{Fourth Root}
The fourth root of a number $x$ is $x^{1/4}$, which can be calculated in C++ using \texttt{pow(x, 0.25)}.
\end{block}
\end{frame}

\section{Problem Solving with U-P-E-R}

\begin{frame}
\frametitle{I Do: Binary to Decimal - Understand}
\textbf{Problem:} Convert the binary number \alert{$11010_2$} to its decimal equivalent.

\vfill
\textbf{U - Understand the Problem}
\begin{itemize}
    \item \textbf{Goal:} Find the base-10 value of a given base-2 number.
    \item \textbf{Inputs:} The binary number $11010_2$.
    \item \textbf{Outputs:} A single integer in base-10.
    \item \textbf{Example:} We know $10_2$ should be $2_{10}$.
\end{itemize}
\end{frame}

\begin{frame}
\frametitle{I Do: Binary to Decimal - Plan}
\textbf{P - Plan the Logic}
\begin{itemize}
    \item \textbf{Variables:} No variables needed for this manual calculation.
    \item \textbf{Steps (Algorithm):}
    \begin{enumerate}
        \item Write down the binary number: 1 1 0 1 0.
        \item Assign place values (powers of 2) from right to left, starting with $2^0$.
        \begin{center}
        $2^4 \quad 2^3 \quad 2^2 \quad 2^1 \quad 2^0$
        \end{center}
        \item Multiply each binary digit by its corresponding place value.
        \item Sum all the resulting products.
    \end{enumerate}
\end{itemize}
\end{frame}

\begin{frame}
\frametitle{I Do: Binary to Decimal - Execute}
\textbf{E - Execute the Plan}
\begin{align*}
11010_2 &= (1 \times 2^4) + (1 \times 2^3) + (0 \times 2^2) + (1 \times 2^1) + (0 \times 2^0) \\
\pause
&= (1 \times 16) + (1 \times 8) + (0 \times 4) + (1 \times 2) + (0 \times 1) \\
\pause
&= 16 + 8 + 0 + 2 + 0 \\
\pause
&= 26_{10}
\end{align*}
\end{frame}

\begin{frame}
\frametitle{I Do: Binary to Decimal - Review}
\textbf{R - Review and Test}
\begin{itemize}
    \item The final answer is \alert{$26_{10}$}.
    \item Does this seem reasonable?
    \item Let's check our place value calculations: $2^0=1, 2^1=2, 2^2=4, 2^3=8, 2^4=16$. Correct.
    \item Let's check the sum: $16 + 8 + 2 = 26$. Correct.
    \item The process was followed correctly and the arithmetic is verified.
\end{itemize}
\end{frame}

\begin{frame}
\frametitle{We Do (Concept): Decimal to Binary}
\textbf{Problem:} Convert the decimal number \alert{$42_{10}$} to its binary equivalent.

\vfill
\textbf{U - Understand the Problem}
\begin{itemize}
    \item \textbf{Goal:} Find the base-2 representation of $42_{10}$.
    \item \textbf{Inputs:} The integer 42.
    \item \textbf{Outputs:} A string of 1s and 0s.
\end{itemize}
\pause
\textbf{P - Plan the Logic}
\begin{itemize}
    \item \textbf{Steps (Algorithm):}
    \begin{enumerate}
        \item We will use the \alert{division-by-2} method.
        \item Divide the number by 2 and write down the remainder.
        \item Use the quotient from the previous step as the new number to divide.
        \item Repeat until the quotient is 0.
        \item Read the list of remainders \alert{backwards}.
    \end{enumerate}
\end{itemize}
\end{frame}

\begin{frame}
\frametitle{We Do (Concept): Decimal to Binary - Execute \& Review}
\textbf{E - Execute the Plan}
\begin{tabular}{r c l}
 $42 \div 2$ & = 21 & Remainder \alert{?} \pause \textbf{0} \\
 $21 \div 2$ & = 10 & Remainder \alert{?} \pause \textbf{1} \\
 $10 \div 2$ & = 5  & Remainder \alert{?} \pause \textbf{0} \\
 $5 \div 2$  & = 2  & Remainder \alert{?} \pause \textbf{1} \\
 $2 \div 2$  & = 1  & Remainder \alert{?} \pause \textbf{0} \\
 $1 \div 2$  & = 0  & Remainder \alert{?} \pause \textbf{1} \\
\end{tabular}

\vfill
\pause
\textbf{R - Review and Test}
\begin{itemize}
    \item Reading the remainders backwards, we get: \alert{$101010_2$}.
    \item Let's check by converting back:
    \item $1 \times 32 + 0 \times 16 + 1 \times 8 + 0 \times 4 + 1 \times 2 + 0 \times 1$
    \item $32 + 8 + 2 = 42$. The conversion is correct!
\end{itemize}
\end{frame}

\begin{frame}
\frametitle{We Do (Coding): Fourth Root - Understand}
\textbf{Problem:} Write a C++ program to find the fourth root of \alert{1728.896400}. You will need to use the \texttt{pow()} function from the \texttt{<cmath>} library.

\vfill
\textbf{U - Understand the Problem}
\begin{itemize}
    \item \textbf{Goal:} Find the fourth root of a specific number.
    \item \textbf{Inputs:} The number is given: \texttt{1728.896400}.
    \item \textbf{Outputs:} A sentence showing the result.
    \item \textbf{Example:} We know $6^4 = 1296$ and $7^4 = 2401$. Our answer should be between 6 and 7.
\end{itemize}
\end{frame}

\begin{frame}
\frametitle{We Do (Coding): Fourth Root - Plan}
\textbf{P - Plan the Logic}
\begin{itemize}
    \item \textbf{Variables:} What variables do we need? \pause
    \begin{itemize}
        \item \texttt{float number = 1728.896400;}
        \item \texttt{float fourthRoot;}
    \end{itemize} \pause
    \item \textbf{Steps (Pseudocode):}
    \begin{enumerate}
        \item Include libraries. Which ones? \alert{<iostream> and <cmath>} \pause
        \item Declare and initialize our number variable. \pause
        \item Declare a variable for the answer. \pause
        \item Calculate the fourth root. A fourth root is the same as raising to what power? \alert{The 1/4 or 0.25 power}. \pause
        \item Which C++ function do we use? \alert{\texttt{pow(base, exponent)}}. \pause
        \item Print the result.
    \end{enumerate}
\end{itemize}
\textbf{Answer Key:} \texttt{exercises/exercise1.cpp} (Hidden - general solution)
\end{frame}

\begin{frame}[fragile]
\frametitle{We Do (Coding): Fourth Root - Execute \& Review}
\textbf{E - Execute (Let's Write the Code)}
\begin{minted}[fontsize=\small, frame=lines, linenos, highlightlines={7}]{cpp}
#include <iostream>
#include <cmath>

int main() {
    float number = 1728.896400;
    float fourthRoot;
    // How do we write the next line?
\end{minted}
\pause
\begin{minted}[fontsize=\small, frame=lines, linenos, highlightlines={7}]{cpp}
    fourthRoot = pow(number, 0.25);

    std::cout << "The fourth root is: " << fourthRoot << std::endl;
    return 0;
}
\end{minted}
\end{frame}

\begin{frame}
\frametitle{You Do (Practice): Distance Between Two Points}
\textbf{Problem:} Write a C++ program to calculate the distance between two points with coordinates \alert{(7, 12)} and \alert{(3, 9)}. You must use the \texttt{sqrt()} and \texttt{pow()} functions from the \texttt{<cmath>} library.

\vfill

\begin{block}{Your Turn: Use the U-P-E-R Method}
\begin{itemize}
    \item \textbf{U - Understand}: What is the goal? What are your inputs and outputs? Calculate the expected answer by hand first: $d = \sqrt{(3-7)^2 + (9-12)^2} = \sqrt{16+9} = \sqrt{25} = 5$.
    \item \textbf{P - Plan}: What variables will you need for x1, y1, x2, y2, and the distance? Write down the steps in pseudocode.
    \item \textbf{E - Execute}: Translate your plan into a C++ program. Remember to \texttt{\#include <cmath>}.
    \item \textbf{R - Review}: Compile and run your code. Does the output match your hand-calculated answer of 5?
\end{itemize}
\end{block}
\vfill
\textbf{Answer Key:} \texttt{exercises/exercise2.cpp} (Hidden - general solution)
\end{frame}

\begin{frame}
\frametitle{You Do (Practice): Radioactive Decay}
\textbf{Problem:} Using the formula $remaining = (original)e^{-0.00012t}$, write a C++ program to determine the amount of radioactive material remaining after \alert{1000 years}, assuming an initial amount of \alert{100 grams}. Use the \texttt{exp()} function from \texttt{<cmath>}.

\vfill

\begin{block}{Your Turn: Use the U-P-E-R Method}
\begin{itemize}
    \item \textbf{U - Understand}: What is the goal? What are your inputs and outputs? Work a simple example (e.g., t=0).
    \item \textbf{P - Plan}: What variables will you need? Write down the steps in pseudocode. The formula is $remaining = 100 \times \exp(-0.00012 \times 1000)$.
    \item \textbf{E - Execute}: Translate your plan into a C++ program.
    \item \textbf{R - Review}: Compile and run your code. Does the answer make sense? (Should be less than 100).
\end{itemize}
\end{block}
\vfill
\textbf{Answer Key:} \texttt{exercises/exercise3.cpp} (Hidden - general solution)
\end{frame}

\section{Assignment}

\begin{frame}
\frametitle{Homework and Practice}

\begin{block}{Assignment: 04 Number Systems Exercises}
    \begin{itemize}
        \item \textbf{Task:} Complete Parts 0-3 of the Jupyter Notebook (`.ipynb`) file as demonstrated in class.
        \item \textbf{Submission:} Submit your completed file named in the format \alert{\texttt{firstnameLastname\_floats.ipynb}}.
        \item \textbf{Due Date:} \alert{Tuesday, September 23, 2025 at 11:59 pm}.
        \item \textbf{Collaboration:} I encourage you to work together, but the final document you submit must be your own.
    \end{itemize}
\end{block}

\vfill

\begin{alertblock}{Additional Practice}
For more practice with number systems, please complete the assigned modules on the \textbf{Delta Math} website. Focus on:
\begin{itemize}
    \item Binary Conversions
    \item Hexadecimal Conversions
\end{itemize}
\end{alertblock}
\end{frame}

\section{Summary}

\begin{frame}
\frametitle{Summary}
\begin{itemize}
    \item Computers operate using the \alert{binary} (base-2) number system.
    \item \alert{Hexadecimal} (base-16) is a compact, human-readable representation of binary, where 1 hex digit maps to 4 bits.
    \item We can convert between number systems using specific algorithms:
    \begin{itemize}
        \item Binary $\rightarrow$ Decimal: Sum of powers of 2.
        \item Decimal $\rightarrow$ Binary: Repeated division by 2.
    \end{itemize}
    \item The \texttt{<cmath>} library in C++ provides powerful functions for complex mathematical operations like square roots (\texttt{sqrt}) and powers (\texttt{pow}).
    \item The \alert{U-P-E-R method} is a reliable strategy for breaking down and solving both conceptual and programming problems.
\end{itemize}
\end{frame}

\end{document}
