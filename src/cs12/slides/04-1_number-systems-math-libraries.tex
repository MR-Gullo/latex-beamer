\documentclass{beamer}
% Use DS9 global theme (includes pgfplots for visualization)
\usepackage{../../../shared/templates/ds9_theme}
\usepackage{minted}

% Title page configuration
\title[Number Systems and Math]{CS12 CH: Number Systems and Math Library Functions}
\subtitle{Base Conversions and the \texttt{<cmath>} Library}
\author[Mr. Gullo]{Mr. Gullo}
\date[Sep 22, 2025]{September 22, 2025}

\begin{document}
\frame{\titlepage}

\section{Introduction}

\begin{frame}
\frametitle{Learning Objectives}
By the end of this lesson, you will be able to:
\begin{itemize}
    \item Describe the place-value systems for decimal, binary, and hexadecimal numbers.
    \item Use the repeated division algorithm to convert decimal numbers to binary.
    \item Explain why hexadecimal is a convenient, human-readable representation for binary data.
    \item Include and use the \texttt{<cmath>} library to perform advanced mathematical calculations in C++.
    \item Apply functions like \texttt{pow()}, \texttt{sqrt()}, and \texttt{exp()} to solve programming problems.
\end{itemize}
\end{frame}

\section{Number Systems}

\begin{frame}
\frametitle{Key Concepts: Decimal (Base-10)}
\begin{itemize}
    \item The decimal system is a base-10 system, using 10 symbols: 0, 1, 2, 3, 4, 5, 6, 7, 8, 9.
    \item Each digit's position represents a power of 10.
\end{itemize}
\begin{exampleblock}{Example}
The number 653\textsubscript{10} is expanded as:
\begin{align*}
    653_{10} &= (6 \times 10^2) + (5 \times 10^1) + (3 \times 10^0) \\
             &= (6 \times 100) + (5 \times 10) + (3 \times 1) \\
             &= 600 + 50 + 3
\end{align*}
\end{exampleblock}
\end{frame}

\begin{frame}
\frametitle{Key Concepts: Binary (Base-2)}
\begin{itemize}
    \item The binary system is a base-2 system, using 2 symbols: 0 and 1.
    \item This is the native language of computers.
    \item Each digit's position represents a power of 2.
\end{itemize}
\begin{exampleblock}{Example: Binary to Decimal}
The number 1101\textsubscript{2} is converted to decimal as:
\begin{align*}
    1101_{2} &= (1 \times 2^3) + (1 \times 2^2) + (0 \times 2^1) + (1 \times 2^0) \\
             &= (1 \times 8) + (1 \times 4) + (0 \times 2) + (1 \times 1) \\
             &= 8 + 4 + 0 + 1 = 13_{10}
\end{align*}
\end{exampleblock}
\end{frame}

\begin{frame}
\frametitle{Decimal to Binary Conversion: The Algorithm}
To convert a decimal number to binary, follow these steps:
\begin{enumerate}
    \item Divide the decimal number by 2.
    \item Get the integer quotient for the next step.
    \item Get the remainder (which will be 0 or 1) for the binary digit.
    \item Repeat the steps until the quotient is 0.
\end{enumerate}
\vfill
\begin{alertblock}{Important}
The binary result is the sequence of remainders read in \textbf{reverse} order (from bottom to top).
\end{alertblock}
\end{frame}

\begin{frame}
\frametitle{Decimal to Binary: Example 1}
\textbf{Problem:} Convert 13\textsubscript{10} to binary.

\begin{center}
\begin{tabular}{c|c|c}
\textbf{Division by 2} & \textbf{Quotient} & \textbf{Remainder} \\
\hline
13 / 2 & 6 & \alert{1} (LSB - Least Significant Bit) \\
6 / 2  & 3 & \alert{0} \\
3 / 2  & 1 & \alert{1} \\
1 / 2  & 0 & \alert{1} (MSB - Most Significant Bit) \\
\end{tabular}
\end{center}
\vfill
Reading the remainders from bottom to top, we get \alert{1101}.

\vfill
So, 13\textsubscript{10} = 1101\textsubscript{2}.
\end{frame}

\begin{frame}
\frametitle{Decimal to Binary: Example 2}
\textbf{Problem:} Convert 174\textsubscript{10} to binary.

\begin{center}
\begin{tabular}{c|c|c}
\textbf{Division by 2} & \textbf{Quotient} & \textbf{Remainder} \\
\hline
174 / 2 & 87 & \alert{0} \\
87 / 2  & 43 & \alert{1} \\
43 / 2  & 21 & \alert{1} \\
21 / 2  & 10 & \alert{1} \\
10 / 2  & 5  & \alert{0} \\
5 / 2   & 2  & \alert{1} \\
2 / 2   & 1  & \alert{0} \\
1 / 2   & 0  & \alert{1} \\
\end{tabular}
\end{center}
\vfill
Reading the remainders from bottom to top, we get \alert{10101110}.

\vfill
So, 174\textsubscript{10} = 10101110\textsubscript{2}.
\end{frame}

\section{The cmath Library}

\begin{frame}[fragile]
\frametitle{Key Concepts: The \texttt{<cmath>} Library}
\begin{itemize}
    \item For advanced math, C++ provides the \texttt{<cmath>} library.
    \item To use it, you must include it at the top of your program:
\end{itemize}
\begin{block}{\texttt{\#include <cmath>}}
\begin{minted}[fontsize=\small]{cpp}
#include <iostream>
#include <cmath>
using namespace std;

int main() {
    // Math functions like pow(), sqrt(), etc. are now available
}
\end{minted}
\end{block}
\end{frame}

\begin{frame}
\frametitle{Essential Formulas for Exercises}
\begin{block}{Slope of a Line (Exercise 0)}
$$ \text{slope} = \frac{y_2 - y_1}{x_2 - x_1} $$
\end{block}
\pause
\begin{block}{Fourth Root (Exercise 1)}
The fourth root of a number $x$ is $x^{1/4}$, calculated using \texttt{pow(x, 0.25)}.
\end{block}
\pause
\begin{block}{Distance Between Two Points (Exercise 2)}
$$ d = \sqrt{(x_2 - x_1)^2 + (y_2 - y_1)^2} $$
\end{block}
\pause
\begin{block}{Radioactive Decay (Exercise 3)}
$$ \text{remaining} = (\text{original}) \times e^{-0.00012t} $$
In C++, use \texttt{original * exp(-0.00012 * t)}.
\end{block}
\end{frame}

\section{In-Class Exercises}

\begin{frame}[fragile]
\frametitle{Exercise 0: Slope Calculator}
\textbf{Problem:} Write a C++ program to calculate the slope of the line connecting two points. Prompt the user for four values: $x_1, y_1, x_2, y_2$.

\begin{minted}[fontsize=\small, frame=lines, linenos]{cpp}
#include <iostream>
using namespace std;

int main() {
   float x_1, y_1, x_2, y_2;

   // TODO: Prompt the user for all four values and read
   // them into the variables declared above.

   // TODO: Calculate the slope using the formula and print
   // the result in a descriptive sentence.

   return 0;
}
\end{minted}
\vfill
\textbf{Answer Key:} \texttt{exercises/exercise0.cpp} (Hidden Answer Key)
\end{frame}

\begin{frame}[fragile]
\frametitle{Exercise 1: Fourth Root}
\textbf{Problem:} Write a C++ program that prompts for a number, calculates its fourth root, and displays the result. Test it with 81.0 and 1,728.896400.

\begin{minted}[fontsize=\small, frame=lines, linenos]{cpp}
#include <iostream>
#include <cmath>
using namespace std;

int main() {
   float x;

   // TODO: Prompt the user to enter a real number and
   // store it in the variable x.

   // TODO: Use the pow() function to calculate the fourth
   // root and print the result in a descriptive sentence.

   return 0;
}
\end{minted}
\vfill
\textbf{Answer Key:} \texttt{exercises/exercise1.cpp} (Hidden Answer Key)
\end{frame}

\begin{frame}[fragile]
\frametitle{Exercise 2: Distance Formula}
\textbf{Problem:} Write a C++ program to calculate the distance between two points. Manually verify your answer for the points (7, 12) and (3, 9), which should be 5.
\begin{minted}[fontsize=\small, frame=lines, linenos]{cpp}
#include <iostream>
#include <cmath>
using namespace std;

int main() {
   float x_1, y_1, x_2, y_2;

   // TODO: Prompt the user for the coordinates of two points.

   // TODO: Use sqrt() and pow() to calculate the distance
   // based on the formula and print the result.

   return 0;
}
\end{minted}
\vfill
\textbf{Answer Key:} \texttt{exercises/exercise2.cpp} (Hidden Answer Key)
\end{frame}

\begin{frame}[fragile]
\frametitle{Exercise 3: Radioactive Decay}
\textbf{Problem:} Write a C++ program to determine the amount of radioactive material remaining after a number of years, given a starting amount. Test with 100 grams and 1000 years.
\begin{minted}[fontsize=\small, frame=lines, linenos]{cpp}
#include <iostream>
#include <cmath>
using namespace std;

int main() {
   float startingAmount, numYears;

   // TODO: Prompt for starting amount and number of years.

   // TODO: Use the exp() function and the decay formula to
   // calculate the remaining material and print the result.
   // The exponent is (-0.00012 * numYears).

   return 0;
}
\end{minted}
\vfill
\textbf{Answer Key:} \texttt{exercises/exercise3.cpp} (Hidden Answer Key)
\end{frame}

\section{Assignment}
\begin{frame}
\frametitle{Assignment: 04 Number Systems Exercises}
\begin{block}{Programming Assignment}
    \begin{itemize}
        \item \textbf{Task:} Complete Parts 0-3 of the exercises as demonstrated in class.
        \item \textbf{Submission:} Submit your completed Jupyter Notebook file named in the format \alert{\texttt{firstnameLastname\_floats.ipynb}}.
    \end{itemize}
\end{block}
\vfill
\begin{alertblock}{Collaboration Policy}
I encourage you to work together, but the final document you submit must be your own.
\end{alertblock}
\end{frame}

\section{Summary}

\begin{frame}
\frametitle{Summary}
\begin{itemize}
    \item Number systems like \alert{binary} and \alert{decimal} are fundamental to computer science.
    \item We can convert between number systems using clear, step-by-step algorithms like \alert{repeated division}.
    \item The \texttt{<cmath>} library is a powerful tool in C++ for handling complex mathematical operations.
    \item Functions like \texttt{pow()}, \texttt{sqrt()}, and \texttt{exp()} allow us to translate mathematical formulas directly into code to solve scientific and engineering problems.
\end{itemize}
\end{frame}

\end{document}