\documentclass{beamer}
% Use DS9 global theme (includes pgfplots for visualization)
\usepackage{../../../shared/templates/ds9_theme}
\usepackage{minted}

% Title page configuration
\title[Number Systems and Math]{CS12 CH: Number Systems and Math Library Functions}
\subtitle{Base Conversions and the \texttt{<cmath>} Library}
\author[Mr. Gullo]{Mr. Gullo}
\date[Sep 22, 2025]{September 22, 2025}

\begin{document}
\frame{\titlepage}

\section{Introduction}

\begin{frame}
\frametitle{Learning Objectives}
By the end of this lesson, you will be able to:
\begin{itemize}
    \item Convert numbers between decimal (base-10), binary (base-2), and hexadecimal (base-16).
    \item Use the repeated division algorithm to convert decimal numbers to binary.
    \item Explain why hexadecimal is a convenient, human-readable representation for binary data.
    \item Include and use the \texttt{<cmath>} library to perform advanced mathematical calculations in C++.
    \item Apply functions like \texttt{pow()}, \texttt{sqrt()}, and \texttt{exp()} to solve programming problems using the U-P-E-R method.
\end{itemize}
\end{frame}

\section{Number Systems}

\begin{frame}
\frametitle{Key Concepts: Number Systems Review}
\begin{columns}[T]
    \begin{column}{0.5\textwidth}
        \begin{block}{Decimal (Base-10)}
            \begin{itemize}
                \item Uses ten digits (0-9).
                \item Each place value is a power of 10.
                \item Ex: $4095 = 4 \times 10^3 + 0 \times 10^2 + 9 \times 10^1 + 5 \times 10^0$
            \end{itemize}
        \end{block}
    \end{column}
    \begin{column}{0.5\textwidth}
        \begin{block}{Binary (Base-2)}
            \begin{itemize}
                \item Uses two digits (0 and 1).
                \item Each place value is a power of 2.
                \item Ex: $1010_2 = 1 \times 2^3 + 0 \times 2^2 + 1 \times 2^1 + 0 \times 2^0 = 10_{10}$
            \end{itemize}
        \end{block}
    \end{column}
\end{columns}
\end{frame}

\begin{frame}
\frametitle{Key Concepts: Hexadecimal (Base-16)}
\begin{itemize}
    \item A base-16 number system used as a \alert{human-friendly representation} of binary.
    \item Uses 16 symbols: \textbf{0-9} and \textbf{A-F} (where A=10, ..., F=15).
    \item \textbf{Why use it?} It's compact. One hex digit represents a group of four binary digits.
\end{itemize}
\begin{exampleblock}{Example}
It's much easier to read and write \texttt{0xE90B} than \texttt{0b1110100100001011}.
\end{exampleblock}
\vfill
\alert{[Table showing Decimal (0-15), Hex (0-F), and Binary (0000-1111) equivalents.]}
\end{frame}

\begin{frame}
\frametitle{Context: Decimal to Binary Conversion}
\begin{itemize}
    \item To convert a decimal number to binary, we use an algorithm called \alert{repeated division}.
    \item The process involves repeatedly dividing the decimal number by 2 and recording the remainder each time.
    \item The final binary number is formed by reading the sequence of remainders in \textbf{reverse order}.
    \item The next slide shows a diagram of this process for converting the decimal number 197.
\end{itemize}
\end{frame}

\begin{frame}
\frametitle{Visualization: The Repeated Division Algorithm}
\begin{center}
    \alert{[Diagram showing the repeated division of 197 by 2, with each step's quotient and remainder. An arrow indicates reading the remainders backwards to get the final binary number: 11000101.]}
\end{center}
\end{frame}

\section{The cmath Library}

\begin{frame}[fragile]
\frametitle{Key Concepts: The \texttt{<cmath>} Library}
\begin{itemize}
    \item C++ has a built-in library for common mathematical functions.
    \item To use these functions, you must add an include directive at the top of your program:
\end{itemize}
\begin{block}{\texttt{\#include <cmath>}}
\begin{minted}[fontsize=\small]{cpp}
#include <iostream>
#include <cmath> // This line gives you access to math functions
using namespace std;

int main() {
    // ... your code ...
}
\end{minted}
\end{block}
\end{frame}

\begin{frame}
\frametitle{Essential Equations for Exercises}
\begin{block}{Slope of a Line (Exercise 0)}
$$ \text{slope} = \frac{y_2 - y_1}{x_2 - x_1} $$
\end{block}
\pause
\begin{block}{Fourth Root (Exercise 1)}
The fourth root of a number $x$ is $x^{1/4}$. In C++, this is calculated using \texttt{pow(x, 0.25)}.
\end{block}
\pause
\begin{block}{Distance Between Two Points (Exercise 2)}
$$ d = \sqrt{(x_2 - x_1)^2 + (y_2 - y_1)^2} $$
\end{block}
\pause
\begin{block}{Radioactive Decay (Exercise 3)}
$$ \text{remaining} = (\text{original}) \times e^{-0.00012t} $$
In C++, use \texttt{original * exp(-0.00012 * t)}.
\end{block}
\end{frame}

\section{Problem Solving with U-P-E-R}

\begin{frame}
\frametitle{I Do: Slope Calculator - Understand}
\textbf{Problem (Exercise 0):} Write a C++ program to calculate the slope of the line connecting two points, $(x_1, y_1)$ and $(x_2, y_2)$.
\vfill
\textbf{U - Understand the Problem}
\begin{itemize}
    \item \textbf{Goal:} Calculate the slope using the formula $\frac{y_2 - y_1}{x_2 - x_1}$.
    \item \textbf{Inputs:} Four numbers from the user: \texttt{x\_1}, \texttt{y\_1}, \texttt{x\_2}, \texttt{y\_2}. These should be `float`s to allow for decimals.
    \item \textbf{Outputs:} A sentence displaying the calculated slope.
    \item \textbf{Example:} If the points are (2, 3) and (6, 5), the slope is $\frac{5 - 3}{6 - 2} = \frac{2}{4} = 0.5$.
\end{itemize}
\end{frame}

\begin{frame}
\frametitle{I Do: Slope Calculator - Plan}
\textbf{P - Plan the Logic}
\begin{itemize}
    \item \textbf{Variables:}
    \texttt{float x\_1, y\_1, x\_2, y\_2;}
    \item \textbf{Steps (Pseudocode):}
    \begin{enumerate}
        \item Ask the user to enter the first x-value and store it in \texttt{x\_1}.
        \item Ask the user for the first y-value and store it in \texttt{y\_1}.
        \item Ask for the second x-value and store it in \texttt{x\_2}.
        \item Ask for the second y-value and store it in \texttt{y\_2}.
        \item Calculate the slope: \texttt{(y\_2 - y\_1) / (x\_2 - x\_1)}.
        \item Print the result to the screen in a descriptive sentence.
    \end{enumerate}
\end{itemize}
\end{frame}

\begin{frame}[fragile]
\frametitle{I Do: Slope Calculator - Execute}
\textbf{E - Execute (Write the Code)}
\textbf{Demo File:} \texttt{exercises/exercise0.cpp} (Interactive Demo)
\begin{minted}[fontsize=\small, frame=lines, linenos]{cpp}
#include <iostream>
using namespace std;

int main(){
   float x_1, y_1, x_2, y_2;

   cout << "Enter the first x-value: ";
   cin >> x_1;
   cout << "Enter the first y-value: ";
   cin >> y_1;
   cout << "Enter the second x-value: ";
   cin >> x_2;
   cout << "Enter the second y-value: ";
   cin >> y_2;

   cout << "The slope between the two points is: "
        << (y_2 - y_1)/(x_2 - x_1)
        << endl;

   return 0;
}
\end{minted}
\end{frame}

\begin{frame}
\frametitle{I Do: Slope Calculator - Review}
\textbf{R - Review and Test}
\begin{itemize}
    \item \textbf{Compile:} Does the code compile without errors? Yes.
    \item \textbf{Test with Example:}
    \begin{itemize}
        \item Run the program.
        \item Input: `2`, `3`, `6`, `5`.
        \item Expected Output: "The slope...is: 0.5"
        \item Actual Output: Matches the expected output.
    \end{itemize}
    \item \textbf{Debug:} The prompt assumes valid input, but a robust program would need to handle division by zero if $x_2 = x_1$.
\end{itemize}
\end{frame}

\begin{frame}
\frametitle{We Do: Fourth Root - Understand}
\textbf{Problem (Exercise 1):} Write a program that calculates the fourth root of a number. First, test it with 81.0, then with 1,728.896400.
\vfill
\textbf{U - Understand the Problem}
\begin{itemize}
    \item \textbf{Goal:} Calculate $x^{1/4}$ for a given number $x$.
    \item \textbf{Inputs:} A single floating-point number from the user.
    \item \textbf{Outputs:} A sentence displaying the result.
    \item \textbf{Example:} For input 81.0, output should be "The fourth root... is 3".
\end{itemize}
\end{frame}

\begin{frame}
\frametitle{We Do: Fourth Root - Plan}
\textbf{P - Plan the Logic}
\begin{itemize}
    \item \textbf{Variables:} What variables will we need?
    \begin{itemize}
        \item \pause \texttt{float number;}
        \item \pause \texttt{float root;}
    \end{itemize}
    \item \textbf{Steps (Pseudocode):}
    \begin{enumerate}
        \item Include which two libraries? \pause \alert{<iostream> and <cmath>}
        \item Prompt the user for a number.
        \item Store it in the \texttt{number} variable.
        \item Calculate the fourth root. Which function do we use? \pause \alert{\texttt{pow(base, exponent)}}
        \item How would you call it? \pause \alert{\texttt{root = pow(number, 0.25);}}
        \item Print the final answer.
    \end{enumerate}
\end{itemize}
\end{frame}

\begin{frame}[fragile]
\frametitle{We Do: Fourth Root - Execute}
\textbf{E - Execute (Write the Code)}
\textbf{Answer Key:} \texttt{exercises/exercise1.cpp} (Hidden Answer Key)
\begin{minted}[fontsize=\small, frame=lines, linenos]{cpp}
#include <iostream>
#include <cmath>
using namespace std;

int main(){
   float x;

   cout << "Enter a real number: ";
   cin >> x;

   // How do we calculate the root and print it?
   // Let's complete the code together.


   return 0;
}
\end{minted}
\pause
\begin{alertblock}{Solution}
\begin{minted}[fontsize=\small]{cpp}
   cout << "The fourth root of " << x << " is "
        << pow(x, 0.25) << endl;
\end{minted}
\end{alertblock}
\end{frame}

\begin{frame}
\frametitle{We Do: Fourth Root - Review}
\textbf{R - Review and Test}
\begin{itemize}
    \item \textbf{Compile:} Does our completed code compile?
    \item \textbf{Test Case 1:}
    \begin{itemize}
        \item Input: \texttt{81.0}
        \item Does it output 3? \pause Yes.
    \end{itemize}
    \item \textbf{Test Case 2 (from problem):}
    \begin{itemize}
        \item Input: \texttt{1728.896400}
        \item What is the output? \pause Approx. 6.45. This seems correct.
    \end{itemize}
\end{itemize}
\end{frame}

\begin{frame}
\frametitle{You Do: Distance Formula - Understand}
\textbf{Problem (Exercise 2):} Write a program to calculate the distance between two points, given their coordinates $(x_1, y_1)$ and $(x_2, y_2)$.
\vfill
\textbf{U - Understand the Problem}
\begin{itemize}
    \item \textbf{Your Task:} On paper, write down the Goal, Inputs, Outputs, and an Example for this problem.
    \item \textbf{Hint:} For your example, manually calculate the distance between (7, 12) and (3, 9). The answer should be 5.
\end{itemize}
\end{frame}

\begin{frame}
\frametitle{You Do: Distance Formula - Plan}
\textbf{P - Plan the Logic}
\begin{itemize}
    \item \textbf{Your Task:} On paper, list the variables you will need and write the pseudocode steps to solve the problem.
    \item \textbf{Hint:} How can you combine `pow()` and `sqrt()` to implement the distance formula?
\end{itemize}
\textbf{Answer Key:} \texttt{exercises/exercise2.cpp} (Hidden Answer Key)
\end{frame}

\begin{frame}[fragile]
\frametitle{You Do: Distance Formula - Execute}
\textbf{E - Execute (Write the Code)}
\begin{itemize}
    \item \textbf{Your Task:} Translate your plan into a C++ program.
    \item \textbf{Hint:} Remember to include the necessary libraries and use `using namespace std;`.
\end{itemize}
\end{frame}

\begin{frame}
\frametitle{You Do: Distance Formula - Review}
\textbf{R - Review and Test}
\begin{itemize}
    \item \textbf{Your Task:} Compile your code. Once it's free of syntax errors, run it.
    \item \textbf{Test:} When you run the program, does the output match the answer (5) that you calculated by hand? If not, debug your logic.
\end{itemize}
\end{frame}


\section{Assignment}
\begin{frame}
\frametitle{Assignment: 04 Number Systems Exercises}
\begin{block}{Programming Assignment}
    \begin{itemize}
        \item \textbf{Task:} Complete Parts 0-3 of the exercises as demonstrated in class.
        \item \textbf{Submission:} Submit your completed Jupyter Notebook file named in the format \alert{\texttt{firstnameLastname\_floats.ipynb}}.
    \end{itemize}
\end{block}
\vfill
\begin{alertblock}{Collaboration Policy}
I encourage you to work together, but the final document you submit must be your own.
\end{alertblock}
\end{frame}

\section{Summary}

\begin{frame}
\frametitle{Summary}
\begin{itemize}
    \item Computers operate using \alert{binary} (base-2), while humans often use \alert{hexadecimal} (base-16) as a compact shorthand.
    \item We can convert between number systems using specific algorithms like \alert{repeated division by 2}.
    \item The \texttt{<cmath>} library in C++ provides powerful functions for complex math, such as \texttt{pow()}, \texttt{sqrt()}, and \texttt{exp()}.
    \item The \alert{U-P-E-R method} (Understand, Plan, Execute, Review) is a reliable strategy for breaking down and solving any programming problem.
\end{itemize}
\end{frame}

\end{document}