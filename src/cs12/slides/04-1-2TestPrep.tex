\documentclass{beamer}
% Use DS9 global theme (includes pgfplots for visualization)
\usepackage{../../../shared/templates/ds9_theme}

% Title page configuration
\title[Short Title]{CS12 CH:1-4 Review}
\subtitle{Program Structure, Data Types, and Math}
\author[Mr. Gullo]{Mr. Gullo}
\date[Oct 8, 2025]{October 8, 2025}

\begin{document}
\frame{\titlepage}

\begin{frame}
\frametitle{Learning Objectives}
\framesubtitle{Consolidated Review of Chapters 1-4}
\begin{itemize}
    \item Understand the basic 5-part structure of a C++ program.\pause
    \item Identify and use fundamental data types: \texttt{int}, \texttt{float}, \texttt{char}, and \texttt{bool}.\pause
    \item Differentiate between integer and floating-point division.\pause
    \item Explain fundamental memory concepts: Bit and Byte.\pause
    \item Use \texttt{cin} to get input from a user.\pause
    \item Include and use the \texttt{<cmath>} library for advanced math calculations.
\end{itemize}
\end{frame}

\section{C++ Program Fundamentals}

\begin{frame}[fragile]
\frametitle{The 5 Parts of a C++ Program}
\framesubtitle{A Quick Refresher}
\begin{columns}[T]
\begin{column}{0.5\textwidth}
    \begin{enumerate}
        \item \alert{Pre-processor Directives} \\ \texttt{\#include <...>}\pause
        \item \alert{Constant Definitions} \\ \texttt{const float pi = ...;}\pause
        \item \alert{Main Body Heading} \\ \texttt{int main() \{\}}\pause
        \item \alert{Variable Declarations} \\ \texttt{float diameter = ...;}\pause
        \item \alert{Main Body Statements} \\ \texttt{cout << ...;}
    \end{enumerate}
\end{column}\pause
\begin{column}{0.5\textwidth}
\begin{minted}[fontsize=\tiny, frame=lines, linenos]{cpp}
#include <iostream>
using namespace std;

const float pi = 3.14159;

int main()
{
    float diameter = 13.5;

    cout << "Area is "
         << pi * (diameter/2) * (diameter/2)
         << endl;

    return 0;
}
\end{minted}
\end{column}
\end{columns}
\end{frame}

\begin{frame}[fragile]
\frametitle{Code Comments}
\begin{block}{For Human Eyes Only}
Comments are text in your code that is completely ignored by the compiler. Use them to explain \textit{why} your code does something.
\end{block}
\pause

\begin{columns}[T]
\begin{column}{0.5\textwidth}
\begin{alertblock}{Single-Line Comments}
Start with \texttt{//}. The compiler ignores everything to the end of the line.
\begin{minted}[fontsize=\scriptsize]{cpp}
// Calculate the area
float area = l * w;
\end{minted}
\end{alertblock}
\end{column}
\begin{column}{0.5\textwidth}
\begin{exampleblock}{Multi-Line Comments}
Start with \texttt{/*} and end with \texttt{*/}. Can span multiple lines.
\begin{minted}[fontsize=\scriptsize]{cpp}
/* This function calculates
   the distance between two
   points in a 2D plane. */
\end{minted}
\end{exampleblock}
\end{column}
\end{columns}
\end{frame}

\section{Core Data Types \& Memory}

\begin{frame}
\frametitle{Fundamental Data Types}
\begin{columns}[T]
\begin{column}{0.5\textwidth}
    \begin{alertblock}{\texttt{int}}
        Stores positive and negative \alert{whole numbers}.\pause
        \begin{itemize}
            \item Examples: \texttt{-5}, \texttt{0}, \texttt{100}\pause
            \item Operations: \texttt{+}, \texttt{-}, \texttt{*}, \texttt{/}, \texttt{\%}
        \end{itemize}
    \end{alertblock}
    \pause
    \begin{exampleblock}{\texttt{char}}
        Stores a \alert{single character}. Internally, it's a number (ASCII).\pause
        \begin{itemize}
            \item Examples: \texttt{'A'}, \texttt{'?'}, \texttt{'\textbackslash n'}
        \end{itemize}
    \end{exampleblock}
\end{column}\pause
\begin{column}{0.5\textwidth}
    \begin{block}{\texttt{float}}
        Stores numbers with \alert{decimal points}.\pause
        \begin{itemize}
            \item Examples: \texttt{3.14}, \texttt{-0.05}\pause
            \item No modulo \texttt{\%} operator.
        \end{itemize}
    \end{block}
    \pause
    \begin{alertblock}{\texttt{bool}}
        Stores logical values: \alert{true} or \alert{false}.\pause
        \begin{itemize}
            \item Internally: \texttt{true} is 1, \texttt{false} is 0.
        \end{itemize}
    \end{alertblock}
\end{column}
\end{columns}
\end{frame}

\begin{frame}
\frametitle{Essential Concept: Integer vs. Float Division}
This is one of the most common sources of bugs for new programmers!
\pause
\begin{itemize}
    \item \textbf{Integer Division}: If \alert{both} numbers are integers, the result is an integer. The decimal part is \alert{truncated} (cut off).\pause
    \begin{itemize}
        \item \texttt{5 / 4} \hfill $\rightarrow$ \texttt{1}
    \end{itemize}
    \pause
    \item \textbf{Floating-Point Division}: If \alert{at least one} number is a float, the result is a float, preserving the decimal.\pause
    \begin{itemize}
        \item \texttt{5.0 / 4} \hfill $\rightarrow$ \texttt{1.25}\pause
        \item \texttt{5 / 4.0} \hfill $\rightarrow$ \texttt{1.25}
    \end{itemize}
\end{itemize}
\pause
\begin{alertblock}{Key Rule}
The division type is determined \textit{before} the result is assigned to a variable.
\end{alertblock}
\end{frame}


\begin{frame}
\frametitle{Context: Visualizing Memory}
All data types—integers, characters, floats—are stored in the computer's memory as sequences of binary digits (bits).\pause

To understand how much space they take, we need to understand the basic units of memory: bits and bytes.
\end{frame}


\section{User Interaction \& Math}

\begin{frame}[fragile]
\frametitle{Getting User Input with \texttt{cin}}
\begin{itemize}
    \item To make programs interactive, we read input from the user with \texttt{cin}.\pause
    \item \texttt{cin} is part of the \texttt{<iostream>} library.\pause
    \item It uses the extraction operator \texttt{>>} to "pull" data from the keyboard into a variable.
\end{itemize}
\pause
\begin{exampleblock}{Example: Reading an integer}
\begin{minted}[fontsize=\small, frame=lines, linenos, breaklines]{cpp}
#include <iostream>
using namespace std;

int main() {
    int age; // Variable to store input

    cout << "Please enter your age: "; // Prompt
    cin >> age; // Read keyboard input into 'age'

    cout << "You are " << age << " years old." << endl;

    return 0;
}
\end{minted}
\end{exampleblock}
\end{frame}

\begin{frame}[fragile]
\frametitle{The \texttt{<cmath>} Library}
\begin{itemize}
    \item For advanced math, C++ provides the \texttt{<cmath>} library.\pause
    \item You must include it at the top of your program to use its functions:
\end{itemize}
\begin{minted}[fontsize=\small]{cpp}
#include <cmath>
\end{minted}
\pause
\begin{alertblock}{Common Functions}
    \begin{itemize}
        \item \texttt{pow(base, exp)}: Calculates base to the power of exp. (\(x^y\))\pause
        \item \texttt{sqrt(x)}: Calculates the square root of x. (\(\sqrt{x}\))\pause
        \item \texttt{exp(x)}: Calculates the exponential function. (\(e^x\))
    \end{itemize}
\end{alertblock}
\end{frame}

\begin{frame}
\frametitle{Essential Equations Review}
\begin{block}{Distance Between Two Points}
$$ d = \sqrt{(x_2 - x_1)^2 + (y_2 - y_1)^2} $$
\end{block}
\pause
\begin{block}{Slope of a Line}
$$ \text{slope} = \frac{y_2 - y_1}{x_2 - x_1} $$
\end{block}
\pause
\begin{block}{Radioactive Decay}
$$ \text{remaining} = (\text{original}) \times e^{-0.00012t} $$
\end{block}
\end{frame}

\section{Review Exercises}

\begin{frame}[fragile]
\frametitle{Exercise 1: Combined Arithmetic}
\textbf{Exercise File:} \texttt{review_arithmetic.cpp}\pause

\textbf{Objective:} Test your understanding of integer vs. float division by calculating an average.\pause

\begin{minted}[fontsize=\scriptsize, frame=lines, linenos, breaklines]{cpp}
#include <iostream>
using namespace std;

int main() {
    int test1, test2, test3;

    // TODO 1: Prompt the user to enter three integer test scores.
    
    // TODO 2: Read the three scores into the variables test1, test2, and test3.
    
    // TODO 3: Calculate the average of the three scores.
    // BE CAREFUL! The result should be a float. How do you avoid
    // integer division here?
    
    // TODO 4: Print the average score to the console.

    return 0;
}
\end{minted}
\end{frame}

\begin{frame}[fragile]
\frametitle{Exercise 2: Distance Formula}
\textbf{Exercise File:} \texttt{review_distance.cpp}\pause

\textbf{Objective:} Combine user input, floating-point math, and the \texttt{<cmath>} library.\pause

\begin{minted}[fontsize=\scriptsize, frame=lines, linenos, breaklines]{cpp}
#include <iostream>
#include <cmath> // Don't forget this!
using namespace std;

int main() {
    float x1, y1, x2, y2;
    float distance;

    // TODO 1: Prompt the user for the coordinates of two points (x1, y1) and (x2, y2).
    
    // TODO 2: Read the four float values from the user.
    
    // TODO 3: Calculate the distance using the formula.
    // HINT: You will need pow() for squaring and sqrt() for the root.
    
    // TODO 4: Print the calculated distance.

    return 0;
}
\end{minted}
\end{frame}

\begin{frame}[fragile]
\frametitle{Exercise 3: ASCII Character Math}
\textbf{Exercise File:} \texttt{review_char_math.cpp}\pause

\textbf{Objective:} Reinforce that characters are numbers by converting character case.\pause

\begin{minted}[fontsize=\scriptsize, frame=lines, linenos, breaklines]{cpp}
#include <iostream>
using namespace std;

int main() {
    char uppercaseChar;
    char lowercaseChar;

    // TODO 1: Prompt the user to enter a single uppercase letter.
    
    // TODO 2: Read the character into the 'uppercaseChar' variable.
    
    // TODO 3: Calculate the corresponding lowercase letter.
    // HINT: The ASCII value for 'a' is 32 greater than 'A'.
    // Perform an addition operation on the char variable.
    
    // TODO 4: Print the original uppercase letter and the new lowercase letter.

    return 0;
}
\end{minted}
\end{frame}

\section{Summary}

\begin{frame}
\frametitle{Summary: Key Takeaways}
\begin{itemize}
    \item \textbf{Structure is Key}: All C++ programs follow a predictable 5-part structure, which makes them easier to read and write.
    \pause
    \item \textbf{Data Types Matter}: The type of a variable (\texttt{int}, \texttt{float}, etc.) determines what it can store and how it behaves in calculations, especially division.
    \pause
    \item \textbf{Memory is Finite}: Data is stored in memory as bits and bytes. \texttt{sizeof()} shows us that different types have different memory footprints.
    \pause
    \item \textbf{Programs are Interactive}: \texttt{cin} is our tool for getting input from the user, making our programs dynamic.
    \pause
    \item \textbf{Libraries Extend Power}: We don't have to reinvent the wheel. Libraries like \texttt{<cmath>} provide powerful, pre-built functions to solve complex problems.
\end{itemize}
\end{frame}


\end{document}