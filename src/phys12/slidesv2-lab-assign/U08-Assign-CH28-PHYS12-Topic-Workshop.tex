\documentclass[11pt, letterpaper]{article}
\usepackage[margin=0.8in]{geometry}
\usepackage{enumitem}
\usepackage{titling}
\usepackage{fancyhdr}
\usepackage{parskip}
\usepackage{amsmath}

% Set up headers
\pagestyle{fancy}
\fancyhf{}
\lhead{\textbf{Special Relativity Workshop}}
\rhead{Name: \underline{\hspace{5cm}}}
\cfoot{\thepage}

% Macro for fill-in-the-blank
\newcommand{\blank}[1]{\underline{\hspace{#1}}}

\begin{document}

\begin{center}
    \LARGE \textbf{Special Relativity: Watch \& Discuss}
\end{center}

\textbf{Instructions:} We will watch 6 short videos in order. After each video, we will pause for 4 minutes. During this time:
\begin{enumerate}
    \item Complete the \textbf{Quick Check} questions individually.
    \item Discuss the \textbf{Discussion Question} with your neighbor and write down your thoughts.
\end{enumerate}

\hrule
\vspace{0.5em}

% ---------------------------------------------------------
% VIDEO 1
% ---------------------------------------------------------
\section*{Video 1: History - The Michelson-Morley Experiment (SciShow)}
\textit{Context: The "failed" experiment that broke physics.}

\subsection*{A. Quick Check}
\begin{enumerate}
    \item In the 1800s, scientists believed an invisible substance called the \textbf{Luminiferous} \blank{3cm} filled the universe and carried light waves.
    \item The experiment found \textbf{no difference} in light speed, proving that the speed of light is \blank{3cm} in a vacuum, regardless of how fast the Earth is moving.
\end{enumerate}

\subsection*{B. Discussion (4 Minutes)}
\textit{Imagine you are a scientist in 1887. You just spent years building a machine to detect the "Aether Wind," and it found absolutely nothing. Why is a "failed" experiment sometimes more important to science than a successful one?}
\vspace{2cm}

% ---------------------------------------------------------
% VIDEO 2
% ---------------------------------------------------------
\hrule
\section*{Video 2: The Rules - Relative Velocities (MinutePhysics)}
\textit{Context: Establishing the "Speed Limit" of the universe.}

\subsection*{A. Quick Check}
\begin{enumerate}
    \item In Relativity, if you move at $0.6c$ and fire an object forward at $0.6c$, the total speed is \textbf{not} $1.2c$. It will be slightly \blank{3cm} than the speed of light ($c$).
    \item Velocities do not simply \blank{3cm} together like normal numbers when moving at relativistic speeds.
\end{enumerate}

\subsection*{B. Discussion (4 Minutes)}
\textit{In our daily life, speeds add up normally (if you throw a ball from a moving car, it goes faster). Why do we never notice this "universal speed limit" in our everyday lives?}
\vspace{2cm}

\newpage

% ---------------------------------------------------------
% VIDEO 3
% ---------------------------------------------------------
\section*{Video 3: The Hard Concept - Simultaneity (MinutePhysics)}
\textit{Context: Breaking the intuition of "Right Now."}

\subsection*{A. Quick Check}
\begin{enumerate}
    \item Spacetime diagrams plot \textbf{Time} on the vertical axis and \blank{3cm} on the horizontal axis.
    \item \textbf{Simultaneity} means two things happening at the exact same time. If you are moving, events that are simultaneous for a stationary person will become \textbf{out of} \blank{3cm} for you.
\end{enumerate}

\subsection*{B. Discussion (4 Minutes)}
\textit{If two people cannot agree on whether two events happened at the same time, does that mean "the present moment" (right now) doesn't actually exist for everyone? Discuss.}
\vspace{2.5cm}

\hrule

% ---------------------------------------------------------
% VIDEO 4
% ---------------------------------------------------------
\section*{Video 4: Puzzle 1 - The Ladder Paradox (Kyle Hill)}
\textit{Context: Applying simultaneity to physical objects.}

\subsection*{A. Quick Check}
\begin{enumerate}
    \item \textbf{Barn Perspective:} The ladder fits because length contraction makes the ladder \blank{3cm}.
    \item \textbf{Ladder Perspective:} The ladder does \textit{not} fit because the barn is length contracted and is too \blank{3cm}.
    \item Both are correct because the front and back of the ladder do not enter/exit at the same time. This is the \textbf{Relativity of} \blank{3cm}.
\end{enumerate}

\subsection*{B. Discussion (4 Minutes)}
\textit{How can \textbf{both} the Barn and the Ladder be correct? Try to explain to your partner how "Simultaneity" solves this paradox.}
\vspace{2.5cm}

\newpage

% ---------------------------------------------------------
% VIDEO 5
% ---------------------------------------------------------
\section*{Video 5: Puzzle 2 - The Twin Paradox (TED-Ed)}
\textit{Context: Applying Time Dilation to people.}

\subsection*{A. Quick Check}
\begin{enumerate}
    \item Twin \textbf{Terra} stays on Earth while Twin \textbf{Stella} travels in a rocket. Time moves \blank{3cm} for Stella on the ship compared to Terra.
    \item The situation is not symmetrical (and not a paradox) because Stella has to \blank{4cm} (change direction/accelerate) to come home.
    \item When they reunite, \textbf{Stella} (the astronaut) is \blank{3cm} (younger / older) than Terra.
\end{enumerate}

\subsection*{B. Discussion (4 Minutes)}
\textit{If you could travel at 99\% the speed of light for a few years, you would return to Earth to find that decades or centuries had passed. Would you do it? Why or why not?}
\vspace{2.5cm}

\hrule

% ---------------------------------------------------------
% VIDEO 6
% ---------------------------------------------------------
\section*{Video 6: The Proof - Real World Muons (MinutePhysics)}
\textit{Context: Moving from theory to reality.}

\subsection*{A. Quick Check}
\begin{enumerate}
    \item Muons are particles that should decay (die) in $2.2$ \blank{3cm}, which isn't long enough to reach the ground.
    \item \textbf{Earth's View:} The muons survive because their internal \textbf{time} runs \blank{3cm}.
    \item \textbf{Muon's View:} They survive because the \textbf{distance} to the ground has \blank{3cm}.
\end{enumerate}

\subsection*{B. Discussion (4 Minutes)}
\textit{The Muons prove that Length Contraction and Time Dilation are two sides of the same coin. Why is it important that we have physical proof (like muons) rather than just math equations?}
\vspace{2.5cm}

\end{document}