\documentclass{article}
\usepackage{amsmath}
\usepackage{physics}
\usepackage{graphicx}
\usepackage[margin=1in]{geometry}
\usepackage{siunitx}
\usepackage{enumitem}
\usepackage{tikz}
\usepackage{float}

\title{Ladder Problem Analysis Assignment}
\author{Student Workbook}
\date{\today}

\begin{document}

\maketitle

\section*{Problem Analysis Guide}

Given the ladder problem solution, complete the following analysis steps to deepen your understanding of the problem-solving process.

\section*{Part 1: System Visualization (20 points)}

\begin{enumerate}[label=\arabic*., leftmargin=*]
    \item Draw a detailed system diagram showing:
    \begin{itemize}
        \item The ladder's position relative to the house
        \item The person's position on the ladder
        \item All relevant dimensions
        \item The center of mass of the ladder
        \item System boundaries clearly defined 
    \end{itemize}
    
    \item Create a Free Body Diagram (FBD) showing:
    \begin{itemize}
        \item All forces acting on the ladder
        \item Proper force vectors with labels
        \item The angle $\theta$ between the ladder and the ground
        \item The coordinate system you're using
    \end{itemize}
    
    \item Identify and list all known quantities with their units:
    \begin{itemize}
        \item Mass of person = \rule{2cm}{0.15mm} \si{\kilogram}
        \item Mass of ladder = \rule{2cm}{0.15mm} \si{\kilogram}
        \item Length of ladder = \rule{2cm}{0.15mm} \si{\meter}
        \item Distance from wall = \rule{2cm}{0.15mm} \si{\meter}
        \item Height of person on ladder = \rule{2cm}{0.15mm} \si{\meter}
        \item Center of mass location = \rule{2cm}{0.15mm} \si{\meter}
    \end{itemize}
\end{enumerate}

\section*{Part 2: Mathematical Analysis (40 points)}

\begin{enumerate}[label=\arabic*., leftmargin=*, start=4]
    \item Show the step-by-step calculation of the angle $\theta$:
    \[\theta = \arccos\left(\frac{2}{6}\right) = \rule{3cm}{0.15mm}^\circ\]
    
    \item Write out the three equilibrium equations with proper units:
    \begin{itemize}
        \item Sum of forces in x-direction: \rule{8cm}{0.15mm}
        \item Sum of forces in y-direction: \rule{8cm}{0.15mm}
        \item Sum of torques about the bottom of the ladder: \rule{8cm}{0.15mm}
    \end{itemize}
    
    \item Show the complete algebraic steps to find $N$ starting from:
    \[f = \left(\frac{1}{2}w + \frac{1}{3}W\right)\sin \theta \cos \theta = (w + W - N)\tan \theta\]
    Include units in each step.
    
    \item Calculate the weights $w$ and $W$ with proper units:
    \begin{align*}
        w &= \rule{8cm}{0.15mm} \\
        W &= \rule{8cm}{0.15mm}
    \end{align*}
\end{enumerate}

\section*{Part 3: Final Calculations (40 points)}

\begin{enumerate}[label=\arabic*., leftmargin=*, start=8]
    \item Show the detailed calculation of $N$ with units:
    \begin{align*}
        N &= \left(1 - \frac{\cos^2 \theta}{2}\right)w + \left(1 - \frac{\cos^2 \theta}{3}\right)W \\
        &= \rule{12cm}{0.15mm} \\
        &= \rule{12cm}{0.15mm} \si{\newton}
    \end{align*}
    
    \item Calculate $f$ showing all steps and units:
    \begin{align*}
        f &= (w + W - N)\tan \theta \\
        &= \rule{12cm}{0.15mm} \\
        &= \rule{12cm}{0.15mm} \si{\newton}
    \end{align*}
    
    \item Calculate $N'$ showing all steps and units:
    \begin{align*}
        N' &= \frac{f}{\sin \theta} \\
        &= \rule{12cm}{0.15mm} \\
        &= \rule{12cm}{0.15mm} \si{\newton}
    \end{align*}
    
    \item Calculate the magnitude of the total force at the bottom:
    \begin{align*}
        F_{\text{bottom}} &= \sqrt{f^2 + N^2} \\
        &= \rule{12cm}{0.15mm} \\
        &= \rule{12cm}{0.15mm} \si{\newton}
    \end{align*}
\end{enumerate}

\section*{Analysis Questions (Bonus: 10 points)}

\begin{enumerate}[label=\alph*.]
    \item Why is the force at the top ($N'$) much smaller than the force at the bottom?
    \item How would the forces change if the angle $\theta$ were smaller?
    \item Why is it important that the rain gutter is assumed to be frictionless?
    \item What assumptions are we making about the ladder's structure in this problem?
\end{enumerate}

\end{document}