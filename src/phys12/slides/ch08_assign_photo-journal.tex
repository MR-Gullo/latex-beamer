\documentclass[11pt]{article}
\usepackage[margin=1in]{geometry}
\usepackage{enumitem}
\usepackage{tcolorbox}
\usepackage{fontawesome}
\usepackage{microtype}
\usepackage{xcolor}
\usepackage{titlesec}

% Custom colors
\definecolor{highlight}{RGB}{70,130,180}
\definecolor{boxcolor}{RGB}{240,248,255}

% Custom box style for key concepts
\newtcolorbox{conceptbox}{
    colback=boxcolor,
    colframe=highlight,
    boxrule=0.5pt,
    arc=2mm,
    beforeafter skip=10pt,
    enhanced,
    fontupper=\small
}

% Section styling
\titleformat{\section}
{\normalfont\Large\bfseries\color{highlight}}
{\thesection}{1em}{}

\titleformat{\subsection}
{\normalfont\large\bfseries}
{\thesubsection}{1em}{}

\begin{document}

\begin{center}
\LARGE\textbf{Creating Your Experimental Photo Journal}
\end{center}

\vspace{1em}

\begin{conceptbox}
\textbf{Photo Journal:} A visual narrative that captures the physical reality of scientific investigation, focusing on the tangible process of discovery rather than just its analytical outcomes.
\end{conceptbox}

The art of experimental photography lies in capturing the essence of scientific inquiry as it unfolds. Your photo journal should tell the story of your investigation through carefully composed images that reveal both technical precision and human curiosity.

\section*{Core Photography Sequence}

Frame your documentation around these key experimental moments:

\subsection*{1. Setting the Stage}

Capture the initial conditions that define your experimental space:
\begin{itemize}[leftmargin=*]
    \item Complete apparatus assembly in its environment
    \item Critical equipment alignments and calibrations
    \item Scale references that establish size relationships
    \item Safety equipment and protective measures in place
\end{itemize}

\begin{conceptbox}
\textbf{Environmental Context:} The physical space shapes the experiment as much as the equipment itself. Show how your setup exists within its environment.
\end{conceptbox}

\subsection*{2. The Investigation Unfolds}

Document the key transformative moments:
\begin{itemize}[leftmargin=*]
    \item Initial state of your experimental system
    \item Critical procedural transitions
    \item Unexpected observations or challenges
    \item Modified approaches or adaptations
\end{itemize}

\subsection*{3. Human Element}

Show the collaborative nature of scientific inquiry:
\begin{itemize}[leftmargin=*]
    \item Team members engaging with equipment
    \item Hand positions demonstrating technique
    \item Group problem-solving moments
    \item Safety protocols in action
\end{itemize}

\subsection*{4. Team Documentation}

\begin{conceptbox}
\textbf{Group Portrait:} A carefully composed image that captures your research team in the context of your experimental environment, serving as both historical record and testament to the collaborative nature of scientific inquiry.
\end{conceptbox}

Create a meaningful group photograph that:
\begin{itemize}[leftmargin=*]
    \item Shows all team members in their research environment
    \item Incorporates key experimental apparatus
    \item Demonstrates proper safety protocols
    \item Captures the collaborative spirit of your investigation
\end{itemize}

Consider composing your group photo to tell a story:
\begin{itemize}[leftmargin=*]
    \item Position team members naturally around the apparatus
    \item Show active engagement with the experiment
    \item Include relevant safety equipment
    \item Frame the image to include important contextual elements
\end{itemize}

\section*{Technical Essentials}

Keep these fundamental principles in mind:

\begin{conceptbox}
\textbf{Visual Clarity:} Every image should answer a specific question about your experimental process.
\end{conceptbox}

\subsection*{1. Frame Your Story}
\begin{itemize}[leftmargin=*]
    \item Use wide shots to establish context
    \item Capture close-ups for critical details
    \item Show relationships between components
    \item Include scale references naturally
\end{itemize}

\subsection*{2. Highlight Key Details}
\begin{itemize}[leftmargin=*]
    \item Focus on crucial mechanical alignments
    \item Document equipment modifications
    \item Show material properties and interactions
    \item Capture dynamic processes where possible
\end{itemize}

\subsection*{3. Create Visual Flow}
\begin{itemize}[leftmargin=*]
    \item Begin with complete system views
    \item Progress through procedural steps
    \item End with key observational moments
    \item Include transition points between states
\end{itemize}

\vspace{1em}
\noindent\emph{Remember: Your photo journal serves as both historical record and teaching tool. Each image should illuminate some essential aspect of the experimental process, creating a visual narrative that others can follow and learn from.}

\vspace{1em}
\noindent Focus on capturing the physical reality of your investigation rather than its analytical aftermath. Let the images tell the story of discovery as it happens.

\newpage
\section*{Proficiency Rubric}

\begin{conceptbox}
\textbf{Assessment Framework:} This rubric evaluates both technical execution and narrative coherence in experimental documentation, recognizing that effective scientific communication requires both precision and storytelling.
\end{conceptbox}

\subsection*{Emerging}
\textbf{Description:} Beginning to grasp fundamental concepts of experimental documentation, requiring significant guidance.

\noindent\textbf{Skills and Abilities:}
\begin{itemize}[leftmargin=*]
    \item Captures basic equipment photos and group portrait with minimal attention to composition
    \item Documents experimental steps with inconsistent detail or focus
    \item Shows limited awareness of scale references and environmental context
\end{itemize}

\subsection*{Developing}
\textbf{Description:} Shows growing understanding of documentation principles, but needs support in execution.

\noindent\textbf{Skills and Abilities:}
\begin{itemize}[leftmargin=*]
    \item Creates clear equipment photos and group portrait with basic compositional awareness
    \item Records major experimental transitions with adequate detail
    \item Includes basic scale references and some environmental context
\end{itemize}

\subsection*{Proficient}
\textbf{Description:} Demonstrates solid comprehension of documentation methods, working independently.

\noindent\textbf{Skills and Abilities:}
\begin{itemize}[leftmargin=*]
    \item Produces well-composed equipment photos and engaged group portrait
    \item Captures complete experimental progression with appropriate detail
    \item Effectively integrates scale references and environmental context
\end{itemize}

\subsection*{Extending}
\textbf{Description:} Shows advanced understanding and creative capability in documentation.

\noindent\textbf{Skills and Abilities:}
\begin{itemize}[leftmargin=*]
    \item Creates compelling visual narrative through thoughtful composition and sequencing
    \item Documents subtle experimental details and transitional moments
    \item Innovatively incorporates context and scale while maintaining technical precision
\end{itemize}

\end{document}