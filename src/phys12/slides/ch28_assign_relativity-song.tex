\documentclass[12pt]{article}
\usepackage[margin=1in]{geometry}
\usepackage{amsmath}
\usepackage{graphicx}
\usepackage{hyperref}
\usepackage{enumitem}
\usepackage{fancyhdr}
\usepackage{amssymb} % For checkboxes

\setlength{\headheight}{15pt}

\pagestyle{fancy}
\fancyhf{}
\rhead{Special Relativity Project}
\lhead{Physics 12}
\cfoot{\thepage}

\title{\textbf{Special Relativity: AI Song \& Music Video Project}}
\author{Physics 12}
\date{}

\begin{document}

\maketitle

\section{Overview}
You will work in groups of 5 to create an original song about a relativistic scenario using AI music generation tools, then produce a music video that demonstrates the physics through visuals.

\textbf{Crucial Note:} While the song and video are collaborative group efforts, the mathematical derivation and conceptual explanation will be assessed \textbf{individually} during an in-class analysis session. Your group's goal is to ensure \textit{every member} understands the physics deep enough to derive it on their own.

\section{Project Phases \& Checklist}
\textit{This project will span approximately 3 weeks. Specific due dates for each phase will be posted in class.}

\subsection*{Phase 1: Setup \& Derivation (Pre-Production)}
\begin{itemize}[label=$\square$]
    \item \textbf{Group Formation \& Topic Assignment:} Receive your specific relativistic scenario.
    \item \textbf{Initial AI Exploration:} Experiment with AI music tools (Suno, Udio) to find a genre and style.
    \item \textbf{Collaborative Calculation Session:} Work through the math \textbf{together}. 
    \begin{itemize}
        \item \textit{Goal:} Every member must be able to derive the $\gamma$ factor and relevant time/length changes on a whiteboard without looking at notes.
    \end{itemize}
    \item \textbf{Teacher Checkpoint 1:} Submit draft calculations for approval before filming.
\end{itemize}

\subsection*{Phase 2: Production \& Data Visualization}
\begin{itemize}[label=$\square$]
    \item \textbf{Song Finalization:} Generate the final audio track and transcribe lyrics.
    \item \textbf{Visual Storyboarding:} Plan how you will visualize the paradox (split screens, props, etc.).
    \item \textbf{Data Overlay Creation:} Design the graphics that will appear on screen showing variables (e.g., $v=0.9c$, $t'=2.3s$) changing in real-time.
    \item \textbf{Filming:} Record all video segments. Ensure all group members appear on camera.
\end{itemize}

\subsection*{Phase 3: Mastery \& Editing}
\begin{itemize}[label=$\square$]
    \item \textbf{Video Editing:} Assemble clips, sync audio, and add visual data overlays.
    \item \textbf{Group Study Session ("Teach the Teacher"):} 
    \begin{itemize}
        \item Quiz each other on the physics.
        \item If a group member cannot explain the paradox, stop editing and teach them. Their grade on Part 3 depends on this.
    \end{itemize}
    \item \textbf{Teacher Checkpoint 2:} Rough cut review (optional).
\end{itemize}

\subsection*{Phase 4: Assessment}
\begin{itemize}[label=$\square$]
    \item \textbf{Part 3: Individual In-Class Analysis:} A standalone pen-and-paper assessment (see details below).
    \item \textbf{Final Submission:} Upload Song, Lyrics, and Final Video files.
    \item \textbf{Showcase:} Class screening of music videos.
\end{itemize}

\section{Assignment Components}

\subsection{Part 1: AI-Generated Song (15 points)}
\textit{Group Component}

Use an AI music generation tool to create a 2--4 minute song in your chosen genre that tells the story of your relativistic scenario.

\subsubsection*{Requirements:}
\begin{itemize}[noitemsep]
    \item \textbf{Accuracy:} Lyrics must accurately describe the specific paradox or scenario.
    \item \textbf{Narrative:} Song should follow a structure: Setup $\rightarrow$ Conflict (Paradox) $\rightarrow$ Resolution.
    \item \textbf{Submission:} MP3/WAV file + Text file of lyrics.
\end{itemize}

\subsection{Part 2: Music Video (45 points)}
\textit{Group Component}

Create a 4--6 minute music video. Unlike previous years, you do \textbf{not} need to show every step of algebra on screen (that will be done in Part 3). Instead, use the video to visualize the \textit{results} and the \textit{concepts}.

\subsubsection*{Required Video Elements:}

\textbf{1. Visual Data Overlays (20 points)}
\begin{itemize}[noitemsep]
    \item Instead of solving equations on a whiteboard, use digital overlays to show the state of the system.
    \item \textbf{Example:} As the spaceship speeds up, an overlay shows: \\
      $v = 0.90c$ \quad $\gamma = 2.29$ \quad $t_{ship} = 10s$ \quad $t_{earth} = 22.9s$
    \item Data must update in sync with the song/narrative.
    \item Clearly label units and variables.
\end{itemize}

\textbf{2. Conceptual Demonstrations (15 points)}
\begin{itemize}[noitemsep]
    \item Visualise the paradox. Show us \textit{why} the observers disagree.
    \item Use split-screens to show Frame A vs. Frame B simultaneously.
    \item Creative props: Use rulers, clocks, or animations to represent length contraction and time dilation.
\end{itemize}

\textbf{3. Production \& Creativity (10 points)}
\begin{itemize}[noitemsep]
    \item Audio clarity (song must be audible).
    \item All group members must appear on camera.
    \item Editing coherence and creativity.
\end{itemize}

\subsection{Part 3: Individual In-Class Analysis (40 points)}
\textit{Individual Component -- Date TBD by Teacher}

To ensure academic integrity and individual understanding, the mathematical rigor will be assessed via an in-class write-up. You will not have access to your video or AI tools during this session.

\subsubsection*{The Format:}
\begin{itemize}[noitemsep]
    \item Time: 45--60 minutes.
    \item Materials Allowed: Scientific calculator, provided "Universal Formula Sheet" (standard constants and Lorentz formulas).
    \item Materials NOT Allowed: Pre-written notes, phones, AI.
\end{itemize}

\subsubsection*{The Task:}
You will be given the specific initial variables from your group's chosen topic (e.g., "Your group studied the Twin Paradox with $v=0.9c$"). You must:

\begin{enumerate}
    \item \textbf{Derive the Algebra:} Starting from the universal formulas, show the algebraic steps to solve for the specific values (time, length, velocity) relevant to your scenario.
    \item \textbf{Calculate Values:} Plug in the numbers to find the exact values used in your video.
    \item \textbf{Explain the Paradox:} In 1-2 paragraphs, explain \textit{why} the paradox occurs and how it is resolved, referencing the calculations you just performed.
    \item \textbf{Spacetime/Diagrams:} Draw a spacetime diagram or relevant sketch illustrating the reference frames.
\end{enumerate}

\section{Grading Breakdown (100 points)}

\begin{center}
\begin{tabular}{|p{0.7\textwidth}|c|}
\hline
\textbf{Component} & \textbf{Points} \\
\hline
\textbf{PART 1: SONG (Group)} & \textbf{15} \\
Physics accuracy of lyrics & 10 \\
Narrative structure and creativity & 5 \\
\hline
\textbf{PART 2: VIDEO (Group)} & \textbf{45} \\
Visual Data Overlays (Correct values, synced to action) & 20 \\
Conceptual Demo (Visualizing the paradox clearly) & 15 \\
Production Quality \& Participation & 10 \\
\hline
\textbf{PART 3: IN-CLASS ANALYSIS (Individual)} & \textbf{40} \\
Algebraic Derivation (Steps shown clearly) & 15 \\
Numerical Accuracy (Sig Figs, Units) & 10 \\
Conceptual Explanation (Written resolution of paradox) & 10 \\
Diagrams/Sketches & 5 \\
\hline
\textbf{TOTAL} & \textbf{100} \\
\hline
\end{tabular}
\end{center}

\section{Group Strategy: How to Prepare}

Since 40\% of your grade is individual, your group must function as a \textbf{study team}. You cannot assign "The Math Person" to do the math while "The Art Person" makes the video. If you do that, "The Art Person" will fail Part 3.

\textbf{Suggested Workflow:}
\begin{enumerate}
    \item \textbf{Collaborative Calculation:} Early in Phase 1, sit together and solve the problem. Do not move on until \textit{every} member can explain why $\gamma$ is calculated that way.
    \item \textbf{Peer Teaching:} During the video planning, quiz each other. "Hey, why did we put $t=4.5$ years on that overlay? Show me the math."
    \item \textbf{The "Whiteboard Test":} Before the individual assessment, have each group member go to a whiteboard and derive the solution from scratch while the others watch and correct them.
\end{enumerate}

\section{The Six Topics}

\textit{(Refer to the separate Topic Guide for specific variable values and scenarios)}

\begin{enumerate}
    \item \textbf{The Twin Paradox:} $v=0.90c$, $d=4.3$ ly.
    \item \textbf{Michelson-Morley:} Earth $v=30$ km/s, Arm $L=11$m.
    \item \textbf{Relativity of Simultaneity:} Train $v=0.60c$, $L=30$m.
    \item \textbf{Cosmic Ray Muons:} $v=0.998c$, Altitude 15km.
    \item \textbf{Velocity Addition:} Ship $v=0.90c$, Missile $u'=0.90c$.
    \item \textbf{The Ladder Paradox:} Runner $v=0.866c$, Ladder $L=10$m.
\end{enumerate}

\section{Academic Integrity and AI Policy}

\subsection{Green Light (Allowed):}
\begin{itemize}[noitemsep]
    \item Using AI (Suno/Udio) to generate the \textbf{audio} and \textbf{melody}.
    \item Using AI to help brainstorm rhymes or synonyms for physics terms.
    \item Using AI to generate static background images for the video.
\end{itemize}

\subsection{Red Light (Prohibited):}
\begin{itemize}[noitemsep]
    \item Using AI to solve the physics problems.
    \item Using AI to write the conceptual explanation for the In-Class Analysis.
    \item Bringing AI-generated notes into the Part 3 assessment.
\end{itemize}

\vspace{1cm}
\begin{center}
\textit{This project is designed to let you be creative with the "product" (the video) while ensuring you personally master the "process" (the physics). Work together, teach each other, and rock out!}
\end{center}

\end{document}