\documentclass[12pt]{article}
\usepackage[utf8]{inputenc}
\usepackage{amsmath}
\usepackage{amsthm}
\usepackage{amssymb}
\usepackage[margin=1in]{geometry}
\usepackage{fancyhdr}
\usepackage{boxedminipage}
\usepackage{framed}
\usepackage{tcolorbox}
\usepackage{xcolor}

% Define colors
\definecolor{titlecolor}{RGB}{70,130,180}
\definecolor{boxcolor}{RGB}{230,230,250}

% Configure page style
\pagestyle{fancy}
\fancyhf{}
\rhead{Physics of Magic}
\lhead{Terminal Velocity}
\rfoot{Page \thepage}

\begin{document}

\begin{center}
\begin{tcolorbox}[colback=titlecolor!10,colframe=titlecolor,width=\textwidth]
\begin{center}
\huge\textbf{The Flying Mammoth Problem} \\[0.3cm]
\large A Physics Tale of Terminal Velocity
\end{center}
\end{tcolorbox}
\end{center}

\begin{tcolorbox}[colback=boxcolor!40,colframe=boxcolor!60,title=\textbf{Scenario}]
In a magical realm, a satyr and a human wizard are ambushed by a dragon. Their ingenious plan to defeat the attacking dragon involves polymorphic transformation and precise timing:

\begin{itemize}
    \item The wizard will polymorph into a 6,000-kg woolly mammoth
    \item The satyr will polymorph into a 200-kg giant ape
    \item The ape will throw the mammoth at the dragon flying 300 feet overhead
\end{itemize}

However, being prudent adventurers, they wish to calculate the worst-case scenario: What happens if the dragon breaks the wizard's concentration mid-flight?
\end{tcolorbox}

\begin{tcolorbox}[colback=boxcolor!40,colframe=boxcolor!60,title=\textbf{Given Parameters}]
For the falling mammoth-wizard:
\begin{align*}
    m &= 6,000 \text{ kg} &&\text{(mass)} \\
    A &= 4.2 \text{ m}^2 &&\text{(cross-sectional area)} \\
    C &= 0.7 &&\text{(drag coefficient)} \\
    \rho &= 1.21 \text{ kg/m}^3 &&\text{(air density)} \\
    g &= 9.80 \text{ m/s}^2 &&\text{(gravitational acceleration)}
\end{align*}
\end{tcolorbox}

\begin{tcolorbox}[colback=boxcolor!40,colframe=boxcolor!60,title=\textbf{Physical Analysis}]
Terminal velocity occurs when:
\[ \text{Force of gravity } = \text{ Drag force} \]

Leading to the formula:
\[ v_t = \sqrt{\frac{2mg}{\rho CA}} \]

Where:
\begin{itemize}
    \item $v_t$ is the terminal velocity
    \item The numerator represents gravitational force factors
    \item The denominator represents air resistance factors
\end{itemize}
\end{tcolorbox}

\begin{tcolorbox}[colback=boxcolor!40,colframe=boxcolor!60,title=\textbf{Mathematical Solution}]
Substituting the values into our equation:
\begin{align*}
    v_t &= \sqrt{\frac{2 \times 6000\text{ kg} \times 9.80\text{ m/s}^2}{1.21\text{ kg/m}^3 \times 0.7 \times 4.2\text{ m}^2}} \\[0.3cm]
    v_t &= \sqrt{\frac{117,600}{3.5574}} \\[0.3cm]
    v_t &= \underline{98.8\text{ m/s}}
\end{align*}
\end{tcolorbox}

\begin{tcolorbox}[colback=boxcolor!40,colframe=boxcolor!60,title=\textbf{Conversion to Kilometers per Hour}]
\begin{align*}
    98.8\text{ m/s} \times \frac{1\text{ km}}{1000\text{ m}} \times \frac{3600\text{ s}}{1\text{ hr}} = \underline{355.7\text{ km/h}}
\end{align*}
\end{tcolorbox}

\begin{tcolorbox}[colback=boxcolor!40,colframe=boxcolor!60,title=\textbf{Conclusion}]
The mammoth-wizard would reach a terminal velocity of 98.8 m/s (355.7 km/h).

\vspace{0.3cm}
\textit{"Well," says the wizard, examining these calculations with concern, "let's hope the dragon doesn't burn us alive."}

\vspace{0.2cm}
The satyr nods grimly, noting that while this terminal velocity is impressive, it's also precisely why their timing must be perfect - there won't be a second chance if something goes wrong.
\end{tcolorbox}

\end{document}