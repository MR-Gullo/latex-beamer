\documentclass{beamer}
% Use DS9 global theme (includes pgfplots for visualization)
\usepackage{../../../../latex-beamer/shared/templates/ds9_theme}

% Title page configuration
\title[Relativity Part 1]{PHYS12 CH:28.1-28.3}
\subtitle{The Geometry of Spacetime}
\author[Mr. Gullo]{Mr. Gullo}
\date[Nov 21 2025]{November 21, 2025}

\begin{document}
\frame{\titlepage}

\begin{frame}
\frametitle{Learning Objectives}
By the end of this lesson, you will be able to:
\pause
\begin{itemize}
    \item State Einstein's two postulates of special relativity.
    \item Explain why simultaneity is relative to the observer.
    \item Calculate time dilation effects for moving objects using the Lorentz factor.
    \item Calculate length contraction for objects moving at relativistic speeds.
    \item Distinguish between proper time/length and relativistic time/length.
\end{itemize}
\end{frame}

\section{28.1 Einstein's Postulates}

\begin{frame}
\frametitle{28.1 Einstein's First Postulate}
\begin{block}{The Principle of Relativity}
The laws of physics are the same in all \textbf{inertial frames of reference}.
\end{block}
\pause
\begin{itemize}
    \item An inertial frame is one that is not accelerating (constant velocity).
    \item There is no "preferred" or "absolute" frame of reference.
    \item You cannot perform an experiment inside a smooth-moving train to determine if you are moving or standing still.
\end{itemize}
\end{frame}

\begin{frame}
\frametitle{28.1 Einstein's Second Postulate}
\begin{block}{The Constancy of the Speed of Light}
The speed of light in a vacuum ($c$) has the same value ($c = 3.00 \times 10^8$ m/s) in all inertial frames of reference, regardless of the motion of the light source or the observer.
\end{block}
\pause
\begin{itemize}
    \item This contradicts our everyday experience with relative velocities (like throwing a ball from a moving car).
    \item $c$ is the cosmic speed limit.
\end{itemize}
\end{frame}

\section{28.2 Simultaneity and Time Dilation}

\begin{frame}
\frametitle{28.2 The Relativity of Simultaneity}
\textbf{Simultaneity} refers to two events happening at the same time.
\pause
\begin{alertblock}{Key Insight}
Two events that are simultaneous in one frame of reference are \textbf{not necessarily simultaneous} in another frame that is moving relative to the first.
\end{alertblock}
\pause
\begin{itemize}
    \item This is not an optical illusion; it is a fundamental property of time.
    \item If observers cannot agree on "when" things happen, time itself must be relative.
\end{itemize}
\end{frame}

\begin{frame}
\frametitle{Concept Visualization: Simultaneity}
\begin{center}
		\includegraphics[width=0.6\linewidth]{pasted-images/ch28_slides_relativity_Part1-10-25-19.png}
	\end{center}
\pause
A thought experiment with a train car:
\begin{itemize}
    \item A light source is in the center of a moving train car.
    \item \textbf{Observer on train}: Sees light hit front and back walls simultaneously.
    \item \textbf{Observer on ground}: Sees light hit the back wall first (moving towards it) and front wall later (moving away from it).
\end{itemize}
\end{alertblock}
\end{frame}

\begin{frame}
\frametitle{28.2 Time Dilation}
\textbf{Time Dilation}: Moving clocks run slower as measured by an observer at rest.
\pause
\begin{itemize}
    \item $\Delta t$: Time interval measured by stationary observer (dilated time).
    \item $\Delta t_0$: \textbf{Proper time} (measured by observer moving with the event).
    \item $\gamma$: Lorentz factor (always $\ge 1$).
\end{itemize}
\end{frame}

\begin{frame}
\frametitle{Essential Equation: Time Dilation}
\begin{columns}
\column{0.5\textwidth}
\begin{block}{Time Dilation Formula}
$$ \Delta t = \gamma \Delta t_0 $$
or
$$ \Delta t = \frac{\Delta t_0}{\sqrt{1 - \frac{v^2}{c^2}}} $$
\end{block}

\column{0.5\textwidth}
\pause
\begin{itemize}
    \item $\Delta t > \Delta t_0$ (Moving clocks run slow).
    \item $v$: Relative velocity (m/s).
    \item $c$: Speed of light ($3.00 \times 10^8$ m/s).
\end{itemize}
\end{columns}
\end{frame}

\begin{frame}
\frametitle{The Lorentz Factor ($\gamma$)}
$$ \gamma = \frac{1}{\sqrt{1 - \frac{v^2}{c^2}}} $$
\pause
\begin{itemize}
    \item As $v \to 0$, $\gamma \to 1$ (Newtonian mechanics).
    \item As $v \to c$, $\gamma \to \infty$ (Relativistic effects dominate).
    \item Calculating $\gamma$ first often simplifies problems.
\end{itemize}
\end{frame}

\section{28.3 Length Contraction}

\begin{frame}
\frametitle{28.3 Length Contraction}
\textbf{Length Contraction}: Moving objects appear shorter in the direction of motion.
\pause
\begin{itemize}
    \item $L$: Length measured by stationary observer (contracted length).
    \item $L_0$: \textbf{Proper length} (measured by observer at rest relative to the object).
\end{itemize}
\end{frame}

\begin{frame}
\frametitle{Essential Equation: Length Contraction}
\begin{columns}
\column{0.5\textwidth}
\begin{block}{Length Contraction Formula}
$$ L = \frac{L_0}{\gamma} $$
or
$$ L = L_0 \sqrt{1 - \frac{v^2}{c^2}} $$
\end{block}

\column{0.5\textwidth}
\pause
\begin{itemize}
    \item $L < L_0$ (Moving objects shrink).
    \item Contraction happens \textbf{only} in the direction of motion.
    \item Width and height (perpendicular to motion) remain unchanged.
\end{itemize}
\end{columns}
\end{frame}

\begin{frame}
\frametitle{Example: I Do - Time Dilation}
\textbf{Problem}: A spaceship travels at $0.95c$ relative to Earth. An astronaut on board measures a trip to take 2.0 years. How long does the trip take according to Mission Control on Earth?
\end{frame}

% GUESS Frame 1: G and U
\begin{frame}
\frametitle{I Do: Time Dilation - G \& U}
\begin{columns}[T]
\column{0.48\textwidth}
\textbf{G - Givens}
\begin{itemize}
    \item $v = 0.95c$
    \item $\Delta t_0 = 2.0$ years (Proper time, measured on ship)
    \item Frame: Earth (stationary relative to motion)
\end{itemize}

\column{0.48\textwidth}
\pause
\textbf{U - Unknown}
\begin{itemize}
    \item $\Delta t = ?$ (Dilated time on Earth)
\end{itemize}
\end{columns}
\end{frame}

% GUESS Frame 2: E
\begin{frame}
\frametitle{I Do: Time Dilation - Equation}
\textbf{E - Equation}
\begin{itemize}
    \item First, calculate $\gamma$:
    $$ \gamma = \frac{1}{\sqrt{1 - \frac{v^2}{c^2}}} $$
    \pause
    \item Then use time dilation:
    $$ \Delta t = \gamma \Delta t_0 $$
\end{itemize}
\end{frame}

% GUESS Frame 3: S and S
\begin{frame}
\frametitle{I Do: Time Dilation - Substitute \& Solve}
\textbf{S - Substitute}
\begin{itemize}
    \item $\gamma = \frac{1}{\sqrt{1 - 0.95^2}} = \frac{1}{\sqrt{1 - 0.9025}} = \frac{1}{\sqrt{0.0975}}$
    \item $\gamma \approx 3.20$
    \item $\Delta t = (3.20)(2.0 \text{ years})$
\end{itemize}
\pause
\textbf{S - Solve}
\begin{itemize}
    \item $\Delta t = 6.4$ years
    \item \boxed{\Delta t = 6.4 \text{ years}}
    \item \textit{Earth observers wait longer than the astronaut ages.}
\end{itemize}
\end{frame}

\begin{frame}
\frametitle{We Do: Length Contraction}
\textbf{Problem}: A 100 m long spaceship (proper length) moves past a space station at $0.80c$. How long does the spaceship appear to an observer on the space station?
\pause
\vspace{1em}
\textbf{Class Discussion}:
\begin{enumerate}
    \item Who measures $L_0$? (The pilot or the station observer?)
    \item Who measures $L$?
    \item Will the answer be less than or greater than 100 m?
\end{enumerate}
\end{frame}

\begin{frame}
\frametitle{You Do: Practice}
\textbf{Problem}: Muons are unstable particles with a proper lifetime of $2.2 \times 10^{-6}$ s. If a muon travels at $0.99c$ relative to the lab:
\pause
\begin{enumerate}
    \item Calculate the Lorentz factor $\gamma$.
    \item How long does the muon live as measured by a scientist in the lab?
    \item Hint: Expect a longer time ($\Delta t > \Delta t_0$).
\end{enumerate}
\end{frame}

\begin{frame}
\frametitle{Reading Homework}
Before the next lecture on Part 2, please read:
\begin{itemize}
    \item Section 28.4: Relativistic Addition of Velocities
    \item Section 28.5: Relativistic Momentum
    \item Section 28.6: Relativistic Energy
\end{itemize}
\end{frame}

\begin{frame}
\frametitle{Summary}
\pause
\begin{itemize}
    \item \textbf{Postulate 1}: Laws of physics are invariant in inertial frames.
    \item \textbf{Postulate 2}: Speed of light $c$ is constant for all observers.
    \item \textbf{Simultaneity}: Relative to the observer's motion.
    \item \textbf{Time Dilation}: Moving clocks run slow ($\Delta t = \gamma \Delta t_0$).
    \item \textbf{Length Contraction}: Moving objects shorten ($L = L_0 / \gamma$).
    \item $\gamma$ becomes significant only as $v \to c$.
\end{itemize}
\end{frame}

\end{document}
