\documentclass{beamer}
% Use DS9 global theme (includes pgfplots for visualization)
\usepackage{../../../../latex-beamer/shared/templates/ds9_theme}
\usepackage[overridenote]{pdfpc}

% Title page configuration
\title[Relativity Part 1]{PHYS12 CH:28.1-28.3}
\subtitle{The Geometry of Spacetime}
\author[Mr. Gullo]{Mr. Gullo}
\date[Nov 21 2025]{November 21, 2025}

\begin{document}
\frame{\titlepage
\note{- Special relativity changed space/time understanding\\\\
- Time and space not fixed - relative to observer motion\\\\
- One of physics most fascinating topics}
}

\begin{frame}
\frametitle{Learning Objectives}
By the end of this lesson, you will be able to:
\pause
\begin{itemize}
    \item State Einstein's two postulates of special relativity.
    \pause
    \item Explain why simultaneity is relative to the observer.
    \pause
    \item Calculate time dilation effects for moving objects using the Lorentz factor.
    \pause
    \item Calculate length contraction for objects moving at relativistic speeds.
    \pause
    \item Distinguish between proper time/length and relativistic time/length.
\end{itemize}
\note{- Two postulates lead to mind-bending consequences\\\\
- Simultaneity relative, time dilates, lengths contract\\\\
- Key: proper vs relativistic measurements\\\\
- Always ask: who measures what, from which frame?}
\end{frame}

\section{28.1 Einstein's Postulates}

\begin{frame}
\frametitle{28.1 Einstein's First Postulate}
\begin{block}{The Principle of Relativity}
The laws of physics are the same in all \textbf{inertial frames of reference}.
\end{block}
\pause
\begin{itemize}
    \item An inertial frame is one that is not accelerating (constant velocity).
    \pause
    \item There is no "preferred" or "absolute" frame of reference.
    \pause
    \item You cannot perform an experiment inside a smooth-moving train to determine if you are moving or standing still.
\end{itemize}
\note{- Extension of Galilean relativity Newton used\\\\
- No experiment inside sealed box can detect constant velocity\\\\
- Smooth airplane: pour coffee, bounce ball - works normally\\\\
- Only acceleration detectable\\\\
- Physics same whether moving or not (constant v)}
\end{frame}

\begin{frame}
\frametitle{28.1 Einstein's Second Postulate}
\begin{block}{The Constancy of the Speed of Light}
The speed of light in a vacuum ($c$) has the same value ($c = 3.00 \times 10^8$ m/s) in all inertial frames of reference, regardless of the motion of the light source or the observer.
\end{block}
\pause
\begin{itemize}
    \item This contradicts our everyday experience with relative velocities (like throwing a ball from a moving car).
    \pause
    \item $c$ is the cosmic speed limit.
\end{itemize}
\note{- Revolutionary postulate\\\\
- Ball on train: 50+20=70 mph - normal velocity addition\\\\
- Light doesnt add! Shine flashlight on train: still c, not c+50\\\\
- Everyone measures same c regardless of motion\\\\
- Verified experimentally many times\\\\
- Resolving this paradox leads to all of special relativity}
\end{frame}

\section{28.2 Simultaneity and Time Dilation}

\begin{frame}
\frametitle{28.2 The Relativity of Simultaneity}
\textbf{Simultaneity} refers to two events happening at the same time.
\pause
\begin{alertblock}{Key Insight}
Two events that are simultaneous in one frame of reference are \textbf{not necessarily simultaneous} in another frame that is moving relative to the first.
\end{alertblock}
\pause
\begin{itemize}
    \item This is not an optical illusion; it is a fundamental property of time.
    \pause
    \item If observers cannot agree on "when" things happen, time itself must be relative.
\end{itemize}
\note{- Things get weird here\\\\
- We assume simultaneous for me = simultaneous for everyone\\\\
- Einstein: wrong!\\\\
- Not about delays in seeing - fundamental property of spacetime\\\\
- Accept constant c means simultaneity relative is unavoidable}
\end{frame}

\begin{frame}
\frametitle{Concept Visualization: Simultaneity}
\begin{center}
		\includegraphics[width=0.6\linewidth]{pasted-images/ch28_slides_relativity_Part1-10-25-19.png}
	\end{center}
\pause
A thought experiment with a train car:
\begin{itemize}
    \item A light source is in the center of a moving train car.
    \pause
    \item \textbf{Observer on train}: Sees light hit front and back walls simultaneously.
    \pause
    \item \textbf{Observer on ground}: Sees light hit the back wall first (moving towards it) and front wall later (moving away from it).
\end{itemize}
\note{- Light flashes in center of moving train\\\\
- Train observer: equal distances so simultaneous\\\\
- Ground observer: back wall rushes toward light, front away\\\\
- Light speed constant (not c +/- train speed) so back wall hit first\\\\
- Same events, different timing\\\\
- Neither wrong - both correct in own frame}
\end{frame}

\begin{frame}
\frametitle{28.2 Time Dilation}
\textbf{Time Dilation}: Moving clocks run slower as measured by an observer at rest.
\pause
\begin{itemize}
    \item $\Delta t$: Time interval measured by stationary observer (dilated time).
    \pause
    \item $\Delta t_0$: \textbf{Proper time} (measured by observer moving with the event).
    \pause
    \item $\gamma$: Lorentz factor (always $\ge 1$).
\end{itemize}
\note{- First calculable consequence\\\\
- t0 (proper time): observer moving with clock, events same location\\\\
- t (dilated): observer sees clock moving past\\\\
- gamma >= 1 always\\\\
- Moving clocks always run slow}
\end{frame}

\begin{frame}
\frametitle{Essential Equation: Time Dilation}
\begin{columns}
\column{0.5\textwidth}
\begin{block}{Time Dilation Formula}
$$ \Delta t = \gamma \Delta t_0 $$
or
$$ \Delta t = \frac{\Delta t_0}{\sqrt{1 - \frac{v^2}{c^2}}} $$
\end{block}

\column{0.5\textwidth}
\pause
\begin{itemize}
    \item $\Delta t > \Delta t_0$ (Moving clocks run slow).
    \pause
    \item $v$: Relative velocity (m/s).
    \pause
    \item $c$: Speed of light ($3.00 \times 10^8$ m/s).
\end{itemize}
\end{columns}
\note{- t = gamma * t0, so t >= t0 always\\\\
- v small: fraction near 0, gamma near 1, no dilation\\\\
- v approaches c: denominator approaches 0, gamma approaches infinity\\\\
- c = cosmic speed limit\\\\
- Infinite dilation at c means nothing with mass reaches c}
\end{frame}

\begin{frame}
\frametitle{The Lorentz Factor ($\gamma$)}
$$ \gamma = \frac{1}{\sqrt{1 - \frac{v^2}{c^2}}} $$
\pause
\begin{itemize}
    \item As $v \to 0$, $\gamma \to 1$ (Newtonian mechanics).
    \pause
    \item As $v \to c$, $\gamma \to \infty$ (Relativistic effects dominate).
    \pause
    \item Calculating $\gamma$ first often simplifies problems.
\end{itemize}
\note{- gamma = best friend in relativity. Calculate first!\\\\
- Everyday speeds: gamma near 1 (car, ISS - negligible)\\\\
- 0.1c: gamma=1.005. 0.5c: gamma=1.15. 0.9c: gamma=2.3\\\\
- 0.99c: gamma=7.1. 0.999c: gamma=22.4\\\\
- Effect grows dramatically near c}
\end{frame}

\section{28.3 Length Contraction}

\begin{frame}
\frametitle{28.3 Length Contraction}
\textbf{Length Contraction}: Moving objects appear shorter in the direction of motion.
\pause
\begin{itemize}
    \item $L$: Length measured by stationary observer (contracted length).
    \pause
    \item $L_0$: \textbf{Proper length} (measured by observer at rest relative to the object).
\end{itemize}
\note{- Spatial counterpart to time dilation\\\\
- L0 (proper length): at rest relative to object (pilot measures own ship)\\\\
- L (contracted): outside observer sees ship fly past\\\\
- Contraction ONLY in direction of motion\\\\
- Ship shorter, not thinner}
\end{frame}

\begin{frame}
\frametitle{Essential Equation: Length Contraction}
\begin{columns}
\column{0.5\textwidth}
\begin{block}{Length Contraction Formula}
$$ L = \frac{L_0}{\gamma} $$
or
$$ L = L_0 \sqrt{1 - \frac{v^2}{c^2}} $$
\end{block}

\column{0.5\textwidth}
\pause
\begin{itemize}
    \item $L < L_0$ (Moving objects shrink).
    \pause
    \item Contraction happens \textbf{only} in the direction of motion.
    \pause
    \item Width and height (perpendicular to motion) remain unchanged.
\end{itemize}
\end{columns}
\note{- L = L0/gamma, so L <= L0 always (moving objects shorter)\\\\
- Symmetry: time x gamma, length / gamma\\\\
- Space and time interconnected\\\\
- Object not physically compressed - spacetime transformation}
\end{frame}

\begin{frame}
\frametitle{Example: I Do - Time Dilation}
\textbf{Problem}: A spaceship travels at $0.95c$ relative to Earth. An astronaut on board measures a trip to take 2.0 years. How long does the trip take according to Mission Control on Earth?
\note{- Astronaut measures 2 yrs = proper time (moving with clock)\\\\
- Want: Earth observer time = dilated time\\\\
- v = 0.95c means definitely relativistic\\\\
- Use GUESS method}
\end{frame}

% GUESS Frame 1: G and U
\begin{frame}
\frametitle{I Do: Time Dilation - G \& U}
\begin{columns}[T]
\column{0.48\textwidth}
\textbf{G - Givens}
\begin{itemize}
    \item $v = 0.95c$
    \item $\Delta t_0 = 2.0$ years (Proper time, measured on ship)
    \item Frame: Earth (stationary relative to motion)
\end{itemize}

\column{0.48\textwidth}
\pause
\textbf{U - Unknown}
\begin{itemize}
    \item $\Delta t = ?$ (Dilated time on Earth)
\end{itemize}
\end{columns}
\note{- v = 0.95c, t0 = 2.0 yrs\\\\
- Astronaut = proper time (at rest relative to own clock)\\\\
- Events same location for astronaut (inside ship)\\\\
- Unknown: t (Earth)\\\\
- Always specify reference frame - critical!}
\end{frame}

% GUESS Frame 2: E
\begin{frame}
\frametitle{I Do: Time Dilation - Equation}
\textbf{E - Equation}
\begin{itemize}
    \item First, calculate $\gamma$:
    $$ \gamma = \frac{1}{\sqrt{1 - \frac{v^2}{c^2}}} $$
    \pause
    \item Then use time dilation:
    $$ \Delta t = \gamma \Delta t_0 $$
\end{itemize}
\note{- Two equations needed\\\\
- First: calculate gamma\\\\
- Then: t = gamma * t0\\\\
- Always calculate gamma first - cleaner, catches errors\\\\
- Often need gamma for multiple calculations}
\end{frame}

% GUESS Frame 3: S and S
\begin{frame}
\frametitle{I Do: Time Dilation - Substitute \& Solve}
\textbf{S - Substitute}
\begin{itemize}
    \item $\gamma = \frac{1}{\sqrt{1 - 0.95^2}} = \frac{1}{\sqrt{1 - 0.9025}} = \frac{1}{\sqrt{0.0975}}$
    \item $\gamma \approx 3.20$
    \item $\Delta t = (3.20)(2.0 \text{ years})$
\end{itemize}
\pause
\textbf{S - Solve}
\begin{itemize}
    \item $\Delta t = 6.4$ years
    \item \boxed{\Delta t = 6.4 \text{ years}}
    \item \textit{Earth observers wait longer than the astronaut ages.}
\end{itemize}
\note{- gamma = 1/sqrt(1-0.95^2) = 1/sqrt(0.0975) = 3.20\\\\
- t = 3.20 x 2.0 = 6.4 yrs\\\\
- Astronaut ages 2 yrs, Earth ages 6.4 yrs\\\\
- Twin paradox: twin on Earth 4.4 yrs older!\\\\
- Real! Verified with atomic clocks on planes}
\end{frame}

\begin{frame}
\frametitle{We Do: Length Contraction}
\textbf{Problem}: A 100 m long spaceship (proper length) moves past a space station at $0.80c$. How long does the spaceship appear to an observer on the space station?
\pause
\vspace{1em}
\textbf{Class Discussion}:
\begin{enumerate}
    \item Who measures $L_0$? (The pilot or the station observer?)
    \pause
    \item Who measures $L$?
    \pause
    \item Will the answer be less than or greater than 100 m?
\end{enumerate}
\note{- 30 sec: think about discussion questions\\\\
- L0: pilot (at rest relative to ship)\\\\
- L: station observer (sees ship moving)\\\\
- Answer less than 100m (moving objects contract)\\\\
- gamma = 1.67, L = 100/1.67 = 60m\\\\
- Ship appears only 60m to station!}
\end{frame}

\begin{frame}
\frametitle{You Do: Practice}
\textbf{Problem}: Muons are unstable particles with a proper lifetime of $2.2 \times 10^{-6}$ s. If a muon travels at $0.99c$ relative to the lab:
\pause
\begin{enumerate}
    \item Calculate the Lorentz factor $\gamma$.
    \pause
    \item How long does the muon live as measured by a scientist in the lab?
    \pause
    \item Hint: Expect a longer time ($\Delta t > \Delta t_0$).
\end{enumerate}
\note{- Real phenomenon! Muons from cosmic rays in upper atmosphere\\\\
- Should decay before reaching ground - but we detect them!\\\\
- Time dilation: muon clock runs slow from our perspective\\\\
- 5 min to work\\\\
- Answers: gamma about 7.1, dilated lifetime about 15.6 microseconds\\\\
- First experimental verification of time dilation}
\end{frame}

\begin{frame}
\frametitle{Reading Homework}
Before the next lecture on Part 2, please read:
\begin{itemize}
    \item Section 28.4: Relativistic Addition of Velocities
    \item Section 28.5: Relativistic Momentum
    \item Section 28.6: Relativistic Energy
\end{itemize}
\note{- Read for next class\\\\
- Velocity addition: doesnt just add like Newton!\\\\
- Relativistic momentum changes at high speed\\\\
- E = mc^2 derivation\\\\
- Come prepared with questions}
\end{frame}

\begin{frame}
\frametitle{Summary}
\pause
\begin{itemize}
    \item \textbf{Postulate 1}: Laws of physics are invariant in inertial frames.
    \pause
    \item \textbf{Postulate 2}: Speed of light $c$ is constant for all observers.
    \pause
    \item \textbf{Simultaneity}: Relative to the observer's motion.
    \pause
    \item \textbf{Time Dilation}: Moving clocks run slow ($\Delta t = \gamma \Delta t_0$).
    \pause
    \item \textbf{Length Contraction}: Moving objects shorten ($L = L_0 / \gamma$).
    \pause
    \item $\gamma$ becomes significant only as $v \to c$.
\end{itemize}
\note{- Two postulates lead to profound consequences\\\\
- Simultaneity relative: observers disagree on when\\\\
- Time dilation: moving clocks slow (x gamma)\\\\
- Length contraction: moving objects shorter (/ gamma)\\\\
- gamma significant only near c; Newton works at everyday speeds\\\\
- Questions?\\\\
- HW: practice problems 28.1-28.3}
\end{frame}

\end{document}
