\documentclass{beamer}
% Use DS9 global theme (includes pgfplots for visualization)
\usepackage{../../../shared/templates/ds9_theme}

% Title page configuration
\title[Short Title]{PHYS12 CH:2.1-2.8 and 3.1-3.5}
\subtitle{Kinematics in One and Two Dimensions}
\author[Mr. Gullo]{Mr. Gullo}
\date[Sep 12, 2025]{September 12, 2025}

\begin{document}

\frame{\titlepage}

\begin{frame}
\frametitle{Learning Objectives}
\begin{columns}[T]
    \begin{column}{0.5\textwidth}
        \textbf{1D Kinematics:}
        \begin{itemize}
            \item Define position, displacement, distance, velocity, speed, and acceleration.
            \item Distinguish between scalar and vector quantities.
            \item Interpret graphs of position, velocity, and acceleration vs. time.
            \item Use kinematic equations to solve problems for objects with constant acceleration.
            \item Describe the motion of objects in free fall.
        \end{itemize}
    \end{column}
    \begin{column}{0.5\textwidth}
        \textbf{2D Kinematics:}
        \begin{itemize}
            \item Understand the independence of horizontal and vertical motions.
            \item Add and subtract vectors graphically and analytically.
            \item Resolve vectors into perpendicular components.
            \item Apply kinematic equations to solve projectile motion problems.
            \item Use vector addition to solve relative velocity problems.
        \end{itemize}
    \end{column}
\end{columns}
\end{frame}

\begin{frame}
\frametitle{From Physics 11 to Physics 12}
\begin{block}{Building on Foundations}
This course builds directly upon the concepts you learned in Physics 11. Here's how we'll extend your understanding.
\end{block}

\begin{columns}[T]
    \begin{column}{0.5\textwidth}
        \alert{Review from Physics 11}
        \begin{itemize}
            \item<2-> Scalar vs. Vector quantities
            \item<3-> Graphical vector addition (head-to-tail)
            \item<4-> Basic 1D kinematics ($d = vt$, $a = \Delta v / \Delta t$)
            \item<5-> Foundational concept of projectile motion
        \end{itemize}
    \end{column}
    \begin{column}{0.5\textwidth}
        \alert{New Concepts \& Skills}
        \begin{itemize}
            \item<2-> Full derivation of all kinematic equations
            \item<3-> Analytical vector addition (components)
            \item<4-> Solving multi-stage and complex 1D/2D motion problems
            \item<5-> Formal study of \textbf{Relative Velocity}
        \end{itemize}
    \end{column}
\end{columns}
\end{frame}

\part{Part 1: One-Dimensional Kinematics}

\section{Position, Displacement, and Distance}

\begin{frame}
\frametitle{Key Concepts: Position, Displacement, Distance}
\begin{block}{Describing Motion}
To describe motion, we first need to define where an object is.
\end{block}
\begin{itemize}
    \item \textbf{Position ($x$)}: The location of an object at a particular time, measured relative to a \alert{reference frame}. The origin (x=0) is the reference point.
    \item \textbf{Displacement ($\Delta x$)}: The change in position of an object. It is a \alert{vector} quantity, meaning it has both magnitude and direction.
    \begin{itemize}
        \item Equation: $\Delta x = x_f - x_0$
        \item The sign (+ or -) indicates direction.
    \end{itemize}
    \item \textbf{Distance Traveled}: The total length of the path taken between two positions. It is a \alert{scalar} quantity (magnitude only, always positive).
\end{itemize}
\end{frame}

\begin{frame}
\frametitle{Concept Visualization: Displacement vs. Distance (Context)}
\begin{block}{Scenario: A Professor's Pace}
Imagine a professor lecturing. She starts at position $x_0 = 1.5$ m, walks to $x_f = 3.5$ m, and then walks back to $x_{final} = 0.5$ m.
\bigskip
Let's visualize the difference between her total distance traveled and her final displacement.
\end{block}
\alert{Key Question}: Are distance traveled and displacement the same?
\end{frame}

\begin{frame}
\frametitle{Concept Visualization: Displacement vs. Distance}
\begin{tikzpicture}
    % Axis
    \draw[->, thick] (-1,0) -- (6,0) node[below] {$x$ (m)};
    \foreach \x in {0,...,5} {
        \draw (\x, -0.2) -- (\x, 0.2) node[below=4pt] {\x};
    }

    % Initial and Final Positions
    \node[ds9blue, circle, fill, inner sep=2pt, label=above:{$x_0$}] at (1.5, 0) {};
    \node[ds9gold, circle, fill, inner sep=2pt, label=above:{$x_f$}] at (0.5, 0) {};

    % Path
    \draw[->, ds9red, very thick, dashed] (1.5, 0.5) -- (3.5, 0.5) node[midway, above] {Leg 1 (+2.0 m)};
    \draw[->, ds9red, very thick, dashed] (3.5, 1.0) -- (0.5, 1.0) node[midway, above] {Leg 2 (-3.0 m)};

    % Displacement Vector
    \draw[->, ds9blue, very thick] (1.5, -0.5) -- (0.5, -0.5) node[midway, below] {Displacement $\Delta x = -1.0$ m};
\end{tikzpicture}
\begin{itemize}
    \item \textbf{Distance Traveled}: $|+2.0 \text{ m}| + |-3.0 \text{ m}| = \alert{5.0 \text{ m}}$
    \item \textbf{Displacement}: $x_f - x_0 = 0.5 \text{ m} - 1.5 \text{ m} = \alert{-1.0 \text{ m}}$
\end{itemize}
\end{frame}

\section{Vectors, Scalars, Speed, and Velocity}

\begin{frame}
\frametitle{Key Concepts: Scalars and Vectors}
\begin{block}{Two Types of Physical Quantities}
In physics, every measurement is either a scalar or a vector.
\end{block}
\begin{columns}[T]
    \begin{column}{0.5\textwidth}
        \alert{Scalar}
        \begin{itemize}
            \item A quantity with \textbf{magnitude} (a numerical value) only.
            \item No direction.
            \item Examples:
            \begin{itemize}
                \item Distance (5.0 m)
                \item Speed (25 m/s)
                \item Time (10 s)
                \item Temperature (20°C)
            \end{itemize}
        \end{itemize}
    \end{column}
    \begin{column}{0.5\textwidth}
        \alert{Vector}
        \begin{itemize}
            \item A quantity with both \textbf{magnitude} and \textbf{direction}.
            \item In 1D, direction is given by a + or - sign.
            \item Examples:
            \begin{itemize}
                \item Displacement (-1.0 m)
                \item Velocity (-15 m/s)
                \item Acceleration (9.8 m/s²)
                \item Force (500 N, down)
            \end{itemize}
        \end{itemize}
    \end{column}
\end{columns}
\end{frame}

\begin{frame}
\frametitle{Key Concepts: Speed and Velocity}
\begin{block}{How Fast is an Object Moving?}
Speed and velocity both describe the rate of motion, but they are not the same.
\end{block}
\begin{itemize}
    \item \textbf{Average Speed} (Scalar): Total distance traveled divided by the elapsed time.
    \begin{itemize}
        \item $\text{Average Speed} = \frac{\text{Total Distance}}{\Delta t}$
    \end{itemize}
    \item \textbf{Average Velocity} ($\bar{v}$) (Vector): Displacement divided by elapsed time.
    \begin{itemize}
        \item $\bar{v} = \frac{\Delta x}{\Delta t} = \frac{x_f - x_0}{t_f - t_0}$
        \item The sign of velocity indicates the direction of motion.
    \end{itemize}
    \item \textbf{Instantaneous Velocity}: The velocity at a specific moment in time. This is what a car's speedometer shows (the magnitude part).
\end{itemize}
\end{frame}

\section{Acceleration}

\begin{frame}
\frametitle{Key Concepts: Acceleration}
\begin{block}{How Quickly is Velocity Changing?}
An object accelerates if its velocity is changing.
\end{block}
\begin{itemize}
    \item \textbf{Acceleration ($a$)}: The rate at which velocity changes. It is a \alert{vector} quantity.
    \begin{itemize}
        \item Equation: $\bar{a} = \frac{\Delta v}{\Delta t} = \frac{v_f - v_0}{t_f - t_0}$
        \item SI Units: meters per second squared (m/s²).
    \end{itemize}
    \item Acceleration can be a change in \textbf{speed}, a change in \textbf{direction}, or both.
    \item \textbf{Constant Acceleration}: Velocity changes by the same amount each second.
    \item When an object slows down, its acceleration is in the \alert{opposite direction} of its motion.
\end{itemize}
\end{frame}

\begin{frame}
\frametitle{Misconception Alert: Deceleration vs. Negative Acceleration}
\begin{block}{Important Distinction}
These two terms are often confused, but they have precise physical meanings.
\end{block}

\begin{itemize}
    \item \textbf{Deceleration}: Refers to acceleration in the direction \textit{opposite} to the velocity. Deceleration always causes a decrease in speed.
    \item \textbf{Negative Acceleration}: Refers to acceleration in the \textit{negative direction} as defined by your coordinate system. It may or may not be a deceleration.
\end{itemize}
\vfill
\alert{[Diagram based on Figure 2.14 showing four scenarios for a car's motion, illustrating the difference between positive/negative acceleration and speeding up/slowing down in both positive and negative directions.]}
\end{frame}

\section{Kinematic Equations}

\begin{frame}
\frametitle{The Kinematic Equations (for Constant Acceleration)}
\begin{block}{The Toolkit for Solving Motion Problems}
These equations relate the five main kinematic variables. They only work when acceleration \textbf{a} is constant.
\end{block}
\begin{enumerate}
    \item $v_f = v_0 + at$ \quad \small{(Doesn't use displacement)}
    \item $\Delta x = \frac{1}{2}(v_0 + v_f)t$ \quad \small{(Doesn't use acceleration)}
    \item $\Delta x = v_0 t + \frac{1}{2}at^2$ \quad \small{(Doesn't use final velocity)}
    \item $v_f^2 = v_0^2 + 2a\Delta x$ \quad \small{(Doesn't use time)}
\end{enumerate}
\vfill
\begin{columns}
    \begin{column}{0.5\textwidth}
        \textbf{Variables:}
        \begin{itemize}
            \item $\Delta x$: displacement
            \item $v_0$: initial velocity
            \item $v_f$: final velocity
        \end{itemize}
    \end{column}
    \begin{column}{0.5\textwidth}
        ~
        \begin{itemize}
            \item $a$: acceleration
            \item $t$: time interval
        \end{itemize}
    \end{column}
\end{columns}
\end{frame}

\section{Free Fall}

\begin{frame}
\frametitle{Key Concepts: Free Fall}
\begin{block}{Motion Under Gravity's Influence}
An object is in \textbf{free fall} if the only force acting on it is gravity. Air resistance is considered negligible.
\end{block}
\begin{itemize}
    \item All objects in free fall, regardless of their mass, have the same constant downward acceleration.
    \item This is called the \textbf{acceleration due to gravity ($g$)}.
    \item On Earth, the average value is $g = 9.80 \text{ m/s}^2$.
    \item \textbf{Sign Convention}: We will almost always define the \alert{upward} direction as positive.
    \begin{itemize}
        \item This means acceleration for any free-fall problem will be:
        \item $a = -g = -9.80 \text{ m/s}^2$
    \end{itemize}
\end{itemize}
\end{frame}

\begin{frame}
\frametitle{Concept Visualization: Object Thrown Upward (Context)}
\begin{block}{Scenario: A Rock Thrown from a Cliff}
A person throws a rock straight up into the air with an initial velocity of $+13.0$ m/s. It rises, slows down, stops momentarily at its peak, and then falls back down.

We will now visualize the rock's position, velocity, and acceleration over time using graphs.
\end{block}
\begin{center}
\alert{[Diagram of a person throwing a rock upwards from a cliff edge, similar to Figure 2.38]}
\end{center}
\end{frame}

\begin{frame}
\frametitle{Visualization: Position vs. Time}
\begin{figure}
\begin{tikzpicture}
\begin{axis}[
    width=\textwidth,
    height=0.7\textheight,
    axis lines=middle,
    xlabel={Time, $t$ (s)},
    ylabel={Vertical Position, $y$ (m)},
    xmin=0, xmax=4,
    ymin=-10, ymax=10,
    xtick={0,1,2,3,4},
    ytick={-10, -5, 0, 5, 10},
    legend pos=outer north east,
    grid=major,
    thick,
    ]
    \addplot[ds9blue, domain=0:3.5, samples=100, line width=1.5pt] {13*x - 0.5*9.8*x^2};
    \legend{$y(t) = v_0t + \frac{1}{2}at^2$}
\end{axis}
\end{tikzpicture}
\caption{The position-time graph is a parabola, showing the rock rises to a peak and then falls.}
\end{figure}
\end{frame}

\begin{frame}
\frametitle{Visualization: Velocity vs. Time}
\begin{figure}
\begin{tikzpicture}
\begin{axis}[
    width=\textwidth,
    height=0.7\textheight,
    axis lines=middle,
    xlabel={Time, $t$ (s)},
    ylabel={Velocity, $v_y$ (m/s)},
    xmin=0, xmax=4,
    ymin=-20, ymax=15,
    xtick={0,1,2,3,4},
    ytick={-20, -15, -10, -5, 0, 5, 10, 15},
    legend pos=outer north east,
    grid=major,
    thick,
    ]
    \addplot[ds9gold, domain=0:3.5, samples=2, line width=1.5pt] {13 - 9.8*x};
    \legend{$v(t) = v_0 + at$}
\end{axis}
\end{tikzpicture}
\caption{The velocity-time graph is a straight line with a constant negative slope, equal to $-g$.}
\end{figure}
\end{frame}

\begin{frame}
\frametitle{Visualization: Acceleration vs. Time}
\begin{figure}
\begin{tikzpicture}
\begin{axis}[
    width=\textwidth,
    height=0.7\textheight,
    axis lines=middle,
    xlabel={Time, $t$ (s)},
    ylabel={Acceleration, $a_y$ (m/s²)},
    xmin=0, xmax=4,
    ymin=-12, ymax=2,
    xtick={0,1,2,3,4},
    ytick={-12, -10, -8, -6, -4, -2, 0},
    legend pos=outer north east,
    grid=major,
    thick,
    ]
    \addplot[ds9red, domain=0:3.5, samples=2, line width=1.5pt] {-9.8};
    \legend{$a(t) = -g$}
\end{axis}
\end{tikzpicture}
\caption{The acceleration is constant and negative throughout the entire flight.}
\end{figure}
\end{frame}

\section{Graphical Analysis}

\begin{frame}
\frametitle{Graphical Analysis of Motion}
\begin{block}{A Picture is Worth a Thousand Numbers}
Graphs provide powerful insights into an object's motion.
\end{block}
\begin{itemize}
    \item \textbf{Position vs. Time Graph ($x$ vs. $t$)}
    \begin{itemize}
        \item The \alert{slope} of the line is the \alert{velocity}.
        \begin{itemize}
            \item Steeper slope $\rightarrow$ higher velocity.
            \item Horizontal line $\rightarrow$ zero velocity (at rest).
            \item Curved line $\rightarrow$ acceleration.
        \end{itemize}
    \end{itemize}
    \vfill
    \item \textbf{Velocity vs. Time Graph ($v$ vs. $t$)}
    \begin{itemize}
        \item The \alert{slope} of the line is the \alert{acceleration}.
        \item The \alert{area under the curve} is the \alert{displacement ($\Delta x$)}.
    \end{itemize}
\end{itemize}
\end{frame}

\begin{frame}
\frametitle{Concept Visualization: Graph Relationships (Context)}
\begin{block}{Scenario: A Trip to the Store}
Consider a simplified trip to a store 3 km away.
\begin{itemize}
    \item You travel at a constant speed for 0.25 hours.
    \item You immediately turn around and travel back at the same constant speed, arriving home after a total of 0.5 hours.
\end{itemize}
Let's see how the position, velocity, and speed graphs describe this trip.
\end{block}
\alert{Note:} This is a simplified model. We assume constant speed and no stopping time.
\end{frame}

\begin{frame}
\frametitle{Graph Relationships: Position vs. Time}
\begin{figure}
\begin{tikzpicture}
\begin{axis}[
    width=\textwidth, height=0.7\textheight,
    axis lines=middle, xlabel={Time (hours)}, ylabel={Position (km)},
    xmin=0, xmax=0.6, ymin=0, ymax=3.5,
    xtick={0,0.1,0.2,0.3,0.4,0.5,0.6},
    ytick={0,1,2,3},
    grid=major, thick,
    ]
    \addplot[ds9blue, mark=*, line width=1.5pt] coordinates {(0,0) (0.25,3) (0.5,0)};
\end{axis}
\end{tikzpicture}
\caption{The slope is positive on the way out, and negative on the way back.}
\end{figure}
\end{frame}

\begin{frame}
\frametitle{Graph Relationships: Velocity vs. Time}
\begin{figure}
\begin{tikzpicture}
\begin{axis}[
    width=\textwidth, height=0.7\textheight,
    axis lines=middle, xlabel={Time (hours)}, ylabel={Velocity (km/h)},
    xmin=0, xmax=0.6, ymin=-15, ymax=15,
    xtick={0,0.1,0.2,0.3,0.4,0.5,0.6},
    ytick={-15,-10,-5,0,5,10,15},
    grid=major, thick,
    ]
    \addplot[const plot, ds9gold, mark=*, line width=1.5pt] coordinates {(0,12) (0.25,12) (0.25,-12) (0.5,-12)};
\end{axis}
\end{tikzpicture}
\caption{Velocity is constant and positive, then constant and negative.}
\end{figure}
\end{frame}

\begin{frame}
\frametitle{Graph Relationships: Speed vs. Time}
\begin{figure}
\begin{tikzpicture}
\begin{axis}[
    width=\textwidth, height=0.7\textheight,
    axis lines=middle, xlabel={Time (hours)}, ylabel={Speed (km/h)},
    xmin=0, xmax=0.6, ymin=0, ymax=14,
    xtick={0,0.1,0.2,0.3,0.4,0.5,0.6},
    ytick={0,2,4,6,8,10,12,14},
    grid=major, thick,
    ]
    \addplot[const plot, ds9red, mark=*, line width=1.5pt] coordinates {(0,12) (0.5,12)};
\end{axis}
\end{tikzpicture}
\caption{Speed, a scalar, is the magnitude of velocity and is always positive.}
\end{figure}
\end{frame}

\section{1D Problem Solving}

\begin{frame}[fragile]
\frametitle{I do: Rock Thrown Upward (Ex. 2.14)}
\begin{block}{Problem}
A person on a cliff throws a rock straight up with an initial velocity of 13.0 m/s. Calculate the rock's position and velocity at t = 1.00 s. (Neglect air resistance, assume $y_0 = 0$).
\end{block}
\pause
\begin{minted}[fontsize=\small]{text}
G - Givens:
  v_0 = +13.0 m/s
  a = -9.80 m/s^2
  t = 1.00 s
  y_0 = 0 m

U - Unknowns:
  Position y_1 at t = 1.00 s
  Velocity v_1 at t = 1.00 s

E - Equations:
  (1) y = y_0 + v_0*t + (1/2)*a*t^2
  (2) v = v_0 + a*t

S - Substitute (Position):
  y_1 = 0 + (13.0 m/s)(1.00 s) + (1/2)(-9.80 m/s^2)(1.00 s)^2
S - Solve (Position):
  y_1 = 13.0 m - 4.90 m = +8.10 m

S - Substitute (Velocity):
  v_1 = 13.0 m/s + (-9.80 m/s^2)(1.00 s)
S - Solve (Velocity):
  v_1 = 13.0 m/s - 9.80 m/s = +3.20 m/s
\end{minted}
\end{frame}

\begin{frame}
\frametitle{We do: Rock Thrown Downward (Ex. 2.15)}
\begin{block}{Problem}
What if the person throws the rock straight down with a speed of 13.0 m/s? Calculate the velocity of the rock when it is 5.10 m below the starting point.
\end{block}

\textbf{G - Givens:}
\begin{itemize}
    \item $y_0 = 0$ m
    \item $v_0 = \alert{?}$
    \item $\Delta y = y_f - y_0 = \alert{?}$
    \item $a = \alert{?}$
\end{itemize}
\pause
\textbf{G - Givens (Filled):}
\begin{itemize}
    \item $y_0 = 0$ m
    \item $v_0 = \alert{-13.0 \text{ m/s}}$ (downward is negative)
    \item $\Delta y = y_f - y_0 = \alert{-5.10 \text{ m}}$
    \item $a = \alert{-9.80 \text{ m/s}^2}$
\end{itemize}
\pause
\textbf{U - Unknown:} Final velocity, $v_f$

\textbf{E - Equation:} Which kinematic equation should we use? (Hint: We don't know time).
\begin{center} \alert{$v_f^2 = v_0^2 + 2a\Delta y$} \end{center}
\pause
\textbf{S - Substitute \& Solve:} Your turn! Plug in the values and solve for $v_f$.
\end{frame}

\begin{frame}
\frametitle{You do: Falling Ice}
\begin{block}{Problem}
A chunk of ice breaks off a glacier and falls 30.0 meters before it hits the water. Assuming it falls freely (there is no air resistance), how long does it take to hit the water?
\end{block}
\vfill
\begin{center}
    Use the GUESS method to structure your solution.
\end{center}
\vfill
\begin{alertblock}{Hint}
    What is the initial velocity of the ice chunk? Which kinematic equation relates displacement, initial velocity, acceleration, and time?
\end{alertblock}
\end{frame}

\part{Part 2: Two-Dimensional Kinematics}
\section{2D Motion Concepts}

\begin{frame}
\frametitle{2D Motion: The Independence of Motion}
\begin{block}{The Most Important Concept in 2D Kinematics}
The horizontal and vertical components of two-dimensional motion are \textbf{independent} of each other.
\end{block}
\begin{itemize}
    \item Motion in the horizontal direction does not affect motion in the vertical direction, and vice versa.
    \item This allows us to break complex 2D problems into two simpler 1D problems: one for the x-direction and one for the y-direction.
    \item The only variable that connects the two separate motions is \alert{time ($t$)}.
\end{itemize}
\end{frame}

\begin{frame}
\frametitle{Concept Visualization: Independence of Motion (Context)}
\begin{block}{Scenario: The Two-Ball Drop}
Imagine two identical balls at the same height.
\begin{itemize}
    \item Ball 1 is dropped straight down.
    \item Ball 2 is launched horizontally at the exact same moment.
\end{itemize}
\textbf{Question:} Which ball hits the ground first?
\end{block}
Let's visualize their motion. The result demonstrates the independence of vertical and horizontal motion.
\end{frame}

\begin{frame}
\frametitle{Concept Visualization: Independence of Motion}
\begin{alertblock}{[Diagram based on Figure 3.6]}
A composite image showing the motion of two balls.
\begin{itemize}
    \item The red ball is dropped vertically from rest.
    \item The blue ball is projected horizontally with an initial velocity.
    \item Strobe flashes at equal time intervals show that both balls have the same vertical position at any given moment.
    \item This demonstrates that the horizontal motion of the blue ball does not affect its vertical motion due to gravity. They hit the ground at the same time.
\end{itemize}
\end{alertblock}
\end{frame}

\section{Vectors in 2D}

\begin{frame}
\frametitle{Vector Addition: Head-to-Tail Method}
\begin{block}{Graphical Method for Adding Vectors}
To add vectors graphically, we draw them one after another.
\end{block}
\begin{enumerate}
    \item Draw the first vector to scale and in the correct direction.
    \item Draw the second vector starting from the head (tip) of the first vector.
    \item Continue for all vectors.
    \item The \textbf{resultant vector} ($\vec{R}$) is the vector drawn from the tail (start) of the first vector to the head of the last vector.
\end{enumerate}
\vfill
\begin{alertblock}{[Diagram illustrating the head-to-tail method for adding vectors $\vec{A}$ and $\vec{B}$, showing the resultant vector $\vec{R}$. Based on Figure 3.10]}
\end{alertblock}
\end{frame}

\begin{frame}
\frametitle{Analytical Method: Vector Components}
\begin{block}{Using Trigonometry for Precision}
Any 2D vector can be broken down into two perpendicular components. We typically use x and y axes.
\end{block}
\begin{itemize}
    \item For a vector $\vec{A}$ with magnitude $A$ and at an angle $\theta$ (measured from the positive x-axis):
    \begin{itemize}
        \item The x-component is $A_x = A \cos \theta$
        \item The y-component is $A_y = A \sin \theta$
    \end{itemize}
    \item This process is called \textbf{resolving the vector}.
    \item To add vectors $\vec{A}$ and $\vec{B}$ to get $\vec{R}$:
    \begin{itemize}
        \item Add the x-components: $R_x = A_x + B_x$
        \item Add the y-components: $R_y = A_y + B_y$
    \end{itemize}
    \item Then, find the magnitude and direction of $\vec{R}$ using its components.
\end{itemize}
\end{frame}

\begin{frame}
\frametitle{Concept Visualization: Vector Components (Context)}
\begin{block}{Breaking a Vector Apart}
Let's visualize how a single vector $\vec{A}$ can be represented as the sum of its perpendicular components, $\vec{A}_x$ and $\vec{A}_y$.
\vfill
This is the reverse of adding vectors and is a crucial first step for solving almost any 2D physics problem.
\end{block}
\end{frame}

\begin{frame}
\frametitle{Concept Visualization: Vector Components}
\begin{figure}
\begin{tikzpicture}[scale=1.5, thick]
    % Axes
    \draw[->] (0,0) -- (4,0) node[below] {$x$};
    \draw[->] (0,0) -- (0,3) node[left] {$y$};

    % Vector A
    \draw[->, ds9blue, line width=1.5pt] (0,0) -- (3,2) node[midway, above left] {$\vec{A}$};
    \draw (0.5,0) arc (0:33.7:0.5) node[midway, right] {$\theta$};

    % Components
    \draw[->, ds9red, dashed, line width=1.2pt] (0,0) -- (3,0) node[midway, below] {$A_x = A \cos\theta$};
    \draw[->, ds9red, dashed, line width=1.2pt] (3,0) -- (3,2) node[midway, right] {$A_y = A \sin\theta$};

    % Right angle
    \draw (2.7,0) -- (2.7,0.3) -- (3,0.3);
\end{tikzpicture}
\caption{The vector $\vec{A}$ is the vector sum of its components: $\vec{A} = \vec{A}_x + \vec{A}_y$.}
\end{figure}
\end{frame}

\section{Projectile Motion}

\begin{frame}
\frametitle{Key Concepts: Projectile Motion}
\begin{block}{Applying 2D Kinematics}
A \textbf{projectile} is any object that is thrown or launched and then moves subject only to gravity.
\end{block}
\textbf{Analysis Steps:}
\begin{enumerate}
    \item Set up a coordinate system (usually origin at launch, +y is up).
    \item Resolve the initial velocity ($v_0$) into components:
    \begin{itemize}
        \item $v_{0x} = v_0 \cos \theta_0$
        \item $v_{0y} = v_0 \sin \theta_0$
    \end{itemize}
    \item Treat as two independent 1D motion problems:
    \begin{itemize}
        \item \textbf{Horizontal (x):} Constant velocity ($a_x = 0$)
        \item \textbf{Vertical (y):} Constant acceleration ($a_y = -g$)
    \end{itemize}
    \item Use the kinematic equations for each direction. Time ($t$) is the same for both.
\end{enumerate}
\end{frame}

\begin{frame}
\frametitle{Concept Visualization: Projectile Trajectory (Context)}
\begin{block}{Scenario: The Path of a Cannonball}
Let's trace the path of a projectile, paying close attention to its velocity vector.
\begin{itemize}
    \item How do the horizontal ($v_x$) and vertical ($v_y$) components of velocity change during the flight?
    \item What is the velocity at the highest point (the apex) of the trajectory?
\end{itemize}
This visualization is key to understanding why we separate the motion into two parts.
\end{block}
\end{frame}

\begin{frame}
\frametitle{Concept Visualization: Projectile Trajectory}
\begin{figure}
\begin{tikzpicture}[scale=1.2]
    \begin{axis}[
        axis lines=middle, xlabel={$x$}, ylabel={$y$},
        xmin=0, xmax=11, ymin=0, ymax=4,
        xtick=\empty, ytick=\empty,
        ]
        \addplot[ds9blue, domain=0:10, samples=100, line width=1.5pt] {-(1/5)*x*(x-10)};
        
        % Velocity vectors
        % Start
        \draw[->, ds9red, thick] (0,0) -- (1.5, 1.2) node[above right] {$\vec{v}_0$};
        \draw[->, gray, dashed] (0,0) -- (1.5, 0) node[below] {$v_{0x}$};
        \draw[->, gray, dashed] (0,0) -- (0, 1.2) node[left] {$v_{0y}$};

        % Mid-up
        \draw[->, ds9red, thick] (2.5, 3) -- (4, 3.6) node[above] {$\vec{v}$};
        \draw[->, gray, dashed] (2.5, 3) -- (4, 3) node[below] {$v_x$};
        \draw[->, gray, dashed] (2.5, 3) -- (2.5, 3.6);

        % Apex
        \draw[->, ds9red, thick] (5, 3.75) -- (6.5, 3.75) node[above] {$\vec{v}$};
        \node[below] at (5.75, 3.75) {$v_x$};
        \node[right] at (5, 3.75) {$v_y=0$};

        % Mid-down
        \draw[->, ds9red, thick] (7.5, 3) -- (9, 2.4) node[right] {$\vec{v}$};
        \draw[->, gray, dashed] (7.5, 3) -- (9, 3) node[above] {$v_x$};
        \draw[->, gray, dashed] (7.5, 3) -- (7.5, 2.4);

        % End
        \draw[->, ds9red, thick] (10,0) -- (11.5, -1.2) node[below right] {$\vec{v}_f$};
        \draw[->, gray, dashed] (10,0) -- (11.5, 0) node[below] {$v_{fx}$};
        \draw[->, gray, dashed] (10,0) -- (10, -1.2) node[left] {$v_{fy}$};
    \end{axis}
\end{tikzpicture}
\caption{$v_x$ is constant. $v_y$ decreases, becomes zero at the apex, and then increases in the negative direction.}
\end{figure}
\end{frame}

\section{Relative Velocity}

\begin{frame}
\frametitle{Key Concepts: Relative Velocity}
\begin{block}{Motion Depends on the Observer}
Velocity is always measured \textit{relative} to a frame of reference.
\end{block}
\begin{itemize}
    \item The velocity of an object can have different values when measured by different observers.
    \item We use vector addition to find the velocity of an object relative to a stationary observer (e.g., the ground).
    \item \textbf{Subscript Notation} is very helpful:
    \begin{itemize}
        \item $\vec{v}_{PG}$ = Velocity of the \textbf{P}lane relative to the \textbf{G}round.
        \item $\vec{v}_{PA}$ = Velocity of the \textbf{P}lane relative to the \textbf{A}ir.
        \item $\vec{v}_{AG}$ = Velocity of the \textbf{A}ir relative to the \textbf{G}round (i.e., the wind).
    \end{itemize}
    \item \textbf{Relative Velocity Equation}: $\vec{v}_{PG} = \vec{v}_{PA} + \vec{v}_{AG}$
\end{itemize}
\end{frame}

\begin{frame}
\frametitle{Concept Visualization: Boat in a River (Context)}
\begin{block}{Scenario: Crossing a Current}
A boat tries to travel straight across a river. However, the river's current carries the boat downstream.
\begin{itemize}
    \item $\vec{v}_{bw}$: Velocity of the \textbf{b}oat relative to the \textbf{w}ater.
    \item $\vec{v}_{wg}$: Velocity of the \textbf{w}ater relative to the \textbf{g}round (the current).
    \item $\vec{v}_{bg}$: Velocity of the \textbf{b}oat relative to the \textbf{g}round (its actual path).
\end{itemize}
The boat's actual velocity is the vector sum of its velocity in the water and the water's velocity.
\end{block}
\end{frame}

\begin{frame}
\frametitle{Concept Visualization: Boat in a River}
\begin{alertblock}{[Diagram based on Figure 3.40]}
A diagram showing a river with a current flowing to the right.
\begin{itemize}
    \item A vector labeled $\vec{v}_{bw}$ points straight across the river.
    \item A vector labeled $\vec{v}_{wg}$ points downstream, parallel to the banks.
    \item The resultant vector $\vec{v}_{bg} = \vec{v}_{bw} + \vec{v}_{wg}$ points diagonally downstream, showing the boat's true path relative to the ground.
\end{itemize}
\end{alertblock}
\end{frame}

\section{2D Problem Solving}

\begin{frame}[fragile]
\frametitle{I do: Fireworks Projectile (Ex. 3.4)}
\begin{block}{Problem}
A firework is shot with an initial speed of 70.0 m/s at an angle of 75.0° above the horizontal. (a) Calculate the height at which it explodes (its apex). (b) How much time passes until it explodes?
\end{block}
\pause
\begin{minted}[fontsize=\small]{text}
G - Givens:
  v_0 = 70.0 m/s,  theta_0 = 75.0 deg
  y_0 = 0 m, x_0 = 0 m
  a_y = -9.80 m/s^2, a_x = 0 m/s^2
  At apex, v_fy = 0 m/s

U - Unknowns:  (a) Max height y_f (or h), (b) Time to apex t

E - Equations:
  (1) Resolve v_0: v_0y = v_0*sin(theta_0)
  (2) For height (a): v_fy^2 = v_0y^2 + 2*a_y*Delta_y
  (3) For time (b): v_fy = v_0y + a_y*t

S - Substitute & Solve:
  (1) v_0y = (70.0 m/s)*sin(75.0) = 67.6 m/s
  
  (a) Height:
  (0 m/s)^2 = (67.6 m/s)^2 + 2*(-9.80 m/s^2)*(y_f - 0)
  0 = 4570 m^2/s^2 - (19.6 m/s^2)*y_f
  y_f = 4570 / 19.6 = 233 m
  
  (b) Time:
  0 m/s = 67.6 m/s + (-9.80 m/s^2)*t
  t = -67.6 / -9.80 = 6.90 s
\end{minted}
\end{frame}

\begin{frame}
\frametitle{We do: Hot Rock Projectile (Ex. 3.5)}
\begin{block}{Problem}
A rock is ejected from a volcano with speed 25.0 m/s at 35.0° above the horizontal. It strikes the side of the volcano 20.0 m \textit{lower} than its starting point. (a) Calculate the time it takes.
\end{block}
\textbf{G - Givens:}
\begin{itemize}
    \item $v_0 = 25.0$ m/s, $\theta_0 = 35.0^\circ$
    \item $y_0 = 0$ m, so $\Delta y = y_f - y_0 = \alert{-20.0 \text{ m}}$
    \item $a_y = -9.80$ m/s$^2$
\end{itemize}
\pause
\textbf{U - Unknown:} Time of flight, $t$.
\pause
\textbf{E - Equation:}
First, find initial vertical velocity: $v_{0y} = v_0 \sin\theta_0 = (25.0)\sin(35.0^\circ) = 14.3$ m/s.
\vfill
Now, which y-direction kinematic equation involves $\Delta y$, $v_{0y}$, $a_y$, and $t$?
\begin{center}\alert{$\Delta y = v_{0y}t + \frac{1}{2}a_y t^2$}\end{center}
\pause
\textbf{S - Substitute:}
\begin{center} $-20.0 = (14.3)t + \frac{1}{2}(-9.80)t^2 \implies 4.90t^2 - 14.3t - 20.0 = 0$ \end{center}
\textbf{S - Solve:} How do we solve this for $t$?
\end{frame}

\begin{frame}
\frametitle{You do: Projectile Launch}
\begin{block}{Problem (Ch.3, Q.25)}
A projectile is launched at ground level with an initial speed of 50.0 m/s at an angle of 30.0° above the horizontal. It strikes a target 3.00 seconds later.
\begin{enumerate}
    \item What is the horizontal distance ($x$) to the target?
    \item What is the vertical distance ($y$) to the target?
\end{enumerate}
\end{block}
\vfill
\begin{center}
Use the GUESS method. Remember to break the problem into x and y components.
\end{center}
\vfill
\begin{alertblock}{Hint}
First, resolve the initial velocity into $v_{0x}$ and $v_{0y}$. Then, solve the horizontal and vertical problems separately using $t = 3.00$ s.
\end{alertblock}
\end{frame}

\begin{frame}
\frametitle{Reading Homework}
\begin{block}{Practice and Deeper Understanding}
To solidify your understanding, please work through the following sections in your textbook:
\end{block}
\begin{itemize}
    \item \textbf{Chapter 2: 1D Kinematics}
    \begin{itemize}
        \item Conceptual Questions (Page 73)
        \item Problems \& Exercises (Page 82)
    \end{itemize}
    \vfill
    \item \textbf{Chapter 3: 2D Kinematics}
    \begin{itemize}
        \item Conceptual Questions (Page 156)
        \item Problems \& Exercises (Page 163)
    \end{itemize}
\end{itemize}
\end{frame}

\begin{frame}
\frametitle{Summary of Key Concepts}
\begin{itemize}
    \item \textbf{1D Motion}: We describe motion using scalars (distance, speed) and vectors (displacement, velocity, acceleration). The kinematic equations are our primary tool for solving problems with \alert{constant acceleration}.
    \vfill
    \item \textbf{Graphical Analysis}: The slope and area of motion graphs have physical meaning. (Slope of x-t is v, slope of v-t is a, area of v-t is $\Delta x$).
    \vfill
    \item \textbf{2D Motion}: The key is the \alert{independence of motion}. We break 2D problems into two 1D problems (horizontal and vertical) connected by time.
    \vfill
    \item \textbf{Projectile Motion}: A classic case of 2D motion where $a_x = 0$ (constant velocity) and $a_y = -g$ (constant acceleration).
    \vfill
    \item \textbf{Relative Velocity}: All velocities are relative to a reference frame. We use vector addition to find resultant velocities.
\end{itemize}
\end{frame}

\end{document}