\documentclass{beamer}
% Use DS9 global theme (includes pgfplots for visualization)
\usepackage{../../../../latex-beamer/shared/templates/ds9_theme}

% Title page configuration
\title[Reaction Time Lab]{PHYS12 CH: Reaction Time Lab}
\subtitle{Measuring Human Reaction Time with Kinematics}
\author[Mr. Gullo]{Mr. Gullo}
\date[Sep 27, 2025]{September 27, 2025}

\begin{document}

\frame{\titlepage}

\section{Introduction}

\begin{frame}
\frametitle{Learning Objectives}
\framesubtitle{What You Will Accomplish Today}
    \begin{itemize}
        \item \textbf{Apply Kinematics:} Use kinematic equations to calculate an unknown variable (time) from experimental data.
        \item \textbf{Connect Concepts:} Understand the connection between a physical measurement (distance) and a calculated physics quantity (reaction time).
        \item \textbf{Improve Lab Skills:} Practice making careful measurements, collecting multiple trials, and identifying sources of experimental uncertainty.
    \end{itemize}
\end{frame}

\begin{frame}
\frametitle{Video: The Science of Reaction Time}
    \vfill
    \begin{alertblock}{Media}
        Play the context video about reaction time in sports.
    \end{alertblock}
    \vfill
\end{frame}

\begin{frame}
\frametitle{Summary: What is Reaction Time?}
\framesubtitle{Key Ideas from the Video}
    \begin{block}{Definition}
        Reaction time is the total delay between detecting a stimulus and executing a physical response.
    \end{block}
    \pause
    \begin{itemize}
        \item The process involves a signal traveling from your senses (eyes) to your brain, and then from your brain to your muscles. \pause
        \item This delay, though short, is measurable and critical in high-speed situations. \pause
        \item In sports, even a few hundredths of a second can determine the outcome.
    \end{itemize}
\end{frame}


\begin{frame}
\frametitle{Video: The Science of Hitting a Fastball}
    \vfill
    \begin{alertblock}{Media}
        Play the video on hitting a fastball.
    \end{alertblock}
    \vfill
\end{frame}

\begin{frame}
\frametitle{Summary: Factors Affecting Reaction Time}
\framesubtitle{What Makes You Faster or Slower?}
    As the videos showed, your reaction time is not always the same. Several factors can influence it:
    \begin{columns}[T]
        \column{0.48\textwidth}
            \begin{block}{Physiological Factors}
                \begin{itemize}
                    \item \textbf{Alertness:} Tiredness slows down nerve signals.
                    \item \textbf{Stimulants:} Caffeine can temporarily shorten reaction time.
                \end{itemize}
            \end{block}
        
        \column{0.48\textwidth}
            \begin{block}{Cognitive Factors}
                \begin{itemize}
                    \item \textbf{Distractions:} Loud noises or other tasks increase the brain's processing time.
                    \item \textbf{Prediction:} Elite athletes "see into the future" by anticipating the trajectory of a ball based on the pitcher's body language.
                \end{itemize}
            \end{block}
    \end{columns}
    \vfill
    \begin{alertblock}{Think About This:}
    As you do the lab, consider which of these factors might be affecting your own results today.
    \end{alertblock}
\end{frame}

\begin{frame}
\frametitle{From Physics 11 to Physics 12}
\framesubtitle{Building on What You Know}
    \begin{columns}[T]
        \column{0.48\textwidth}
        \begin{alertblock}{In Physics 11...}
            \begin{itemize}
                \item You performed this lab to practice \textbf{measurement skills}.
                \item You took multiple trials and calculated an \textbf{average distance}.
                \item The goal was to understand measurement uncertainty.
            \end{itemize}
        \end{alertblock}

        \column{0.48\textwidth}
        \begin{block}{Now in Physics 12...}
            \begin{itemize}
                \item We will apply our knowledge of \textbf{kinematics}.
                \item We will use the measured distance to \textbf{calculate your reaction time}.
                \item The goal is to use a physics model to explain a real-world phenomenon.
            \end{itemize}
        \end{block}
    \end{columns}
\end{frame}

\section{The Physics of the Experiment}

\begin{frame}
\frametitle{The Physics of the Ruler Drop}
\framesubtitle{The Equation We Will Use}
    We will use the derivation from this kinematic equation:
    \[ \Delta x = v_0 t + \frac{1}{2} a t^2 \]
    \pause
    Since the ruler is dropped ($v_0=0$) and in free fall ($a=g$), it simplifies and we can solve for time ($t$).
    
    \begin{block}{Your Reaction Time Equation}
        \[ t = \sqrt{\frac{2 \Delta x}{g}} \]
    \end{block}
    \pause
    \vfill
    \begin{columns}[T]
        \column{0.5\textwidth}
            \textbf{Variables:}
            \begin{itemize}
                \item $t$ = reaction time
                \item $\Delta x$ = distance ruler fell
                \item $g = 9.8 \, \text{m/s}^2$
            \end{itemize}
        \column{0.5\textwidth}
            \textbf{Units:}
            \begin{itemize}
                \item $t$ in \textbf{seconds (s)}
                \item $\Delta x$ must be in \textbf{meters (m)!}
                \item $g$ in \textbf{m/s$^2$}
            \end{itemize}
    \end{columns}
\end{frame}

\begin{frame}
\frametitle{Lab Procedure: Setup}
\framesubtitle{Equipment and Safety}
    \begin{columns}[T]
        \column{0.48\textwidth}
        \textbf{Equipment:}
            \begin{itemize}
                \item One 30 cm ruler
                \item A partner
                \item Paper and pencil for recording data
            \end{itemize}
        
        \column{0.48\textwidth}
        \textbf{Safety First!}
            \begin{alertblock}{}
                Move slowly and carefully. Do not grab at the ruler too quickly or you might hit your partner's hand.
            \end{alertblock}
    \end{columns}
\end{frame}

\begin{frame}
\frametitle{Lab Procedure: Data Collection}
\framesubtitle{How to Do the Lab}
    \begin{enumerate}
        \item \textbf{Hold:} Your partner holds the ruler vertically. The 0 cm mark should be at the bottom.
        \item \textbf{Ready:} Place your thumb and index finger around the 0 cm mark, but \textit{do not touch} the ruler.
        \item \textbf{Drop:} Your partner will drop the ruler without warning.
        \item \textbf{Catch:} As soon as you see it fall, catch it!
        \item \textbf{Measure:} Read the distance in centimeters at the top edge of your fingers. This is $\Delta x$ for this trial.
        \item \textbf{Record:} Write this distance in your data table.
        \item \textbf{Repeat:} Perform a total of 5 trials to get a good average.
    \end{enumerate}
\end{frame}

\begin{frame}
\frametitle{Data Analysis Steps}
\framesubtitle{From Distance to Time}
    Once you have your 5 distance measurements:
    
    \begin{enumerate}
        \item \textbf{Find the Average:} Add your 5 distances together and divide by 5. This gives you an average distance in \textbf{cm}. \pause
        
        \item \alert{\textbf{Convert Units!}} Take your average distance in cm and divide by 100 to convert it to \textbf{meters (m)}. This is your final $\Delta x$. \pause
        
        \item \textbf{Calculate Time:} Plug your $\Delta x$ (in meters) into the equation to find your reaction time in seconds.
            \[ t = \sqrt{\frac{2 \times \Delta x}{9.8}} \]
    \end{enumerate}
\end{frame}

\begin{frame}
\frametitle{Homework: Analysis \& Discussion}
\framesubtitle{Thinking About Your Results}
    After you calculate your reaction time, answer these questions in your lab notebook.
    \begin{itemize}
        \item Are your measurements for each trial exactly the same? Why or why not? What does this tell you about measurement?
        \item What do you think were the biggest sources of error in this experiment? How could you make it more accurate?
        \item If you performed this experiment on Jupiter ($g \approx 25 \, \text{m/s}^2$), would the ruler fall a longer or shorter distance during the same reaction time? Explain your answer.
        \item How could you apply what you learned about reaction time to a real-world situation (e.g., driving, sports)?
    \end{itemize}
\end{frame}

\begin{frame}
\frametitle{Summary}
\framesubtitle{Key Takeaways}
    \begin{itemize}
        \item Human reaction time is a physical delay that can be measured using physics. \pause
        \item By measuring the \textbf{distance} an object falls in free fall, we can calculate the \textbf{time} it was falling. \pause
        \item We used the principles of kinematics for an object in \textbf{free fall} ($v_0 = 0$, $a=g$) to derive our key lab equation:
    \end{itemize}
    \begin{center}
    \begin{block}{}
        \[ t = \sqrt{\frac{2 \Delta x}{g}} \]
    \end{block}
    \end{center}
    \vfill
    \begin{alertblock}{}
        This lab is a perfect example of how we use physics models to understand and quantify the world around us.
    \end{alertblock}
\end{frame}

\end{document}