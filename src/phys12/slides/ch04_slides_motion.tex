\documentclass{beamer}
% Use DS9 global theme (includes pgfplots for visualization)
\usepackage{../../../../latex-beamer/shared/templates/ds9_theme}

% Title page configuration
\title[Newton's Laws and Systems]{PHYS12 CH: 4.1-4.4, 4.6, 4.7}
\subtitle{Force, Mass, and Defining the System}
\author[Mr. Gullo]{Mr. Gullo}
\date[Sep 2024]{September 17, 2024}

\begin{document}
\frame{\titlepage}

\section{Introduction}

\begin{frame}
\frametitle{Learning Objectives}
\begin{itemize}
    \item Understand the definition of \textbf{force} as a vector. \pause
    \item Define \textbf{mass} and \textbf{inertia}. \pause
    \item State and apply Newton's First, Second, and Third Laws of Motion. \pause
    \item Define a \textbf{system of interest} and distinguish between \textbf{external} and \textbf{internal} forces. \pause
    \item Draw and use \textbf{Free-Body Diagrams (FBDs)} to solve problems. \pause
    \item Apply Newton's laws to solve problems for both single objects and multi-body systems.
\end{itemize}
\end{frame}

\begin{frame}
\frametitle{From Physics 11 to Physics 12}
\begin{columns}[T]
\column{0.48\textwidth}
\begin{alertblock}{Review from Physics 11}
\begin{itemize}
    \item Newton's Laws for a \textbf{single object}.
    \item Drawing a Free-Body Diagram for one object.
    \item Identifying forces like gravity ($\vec{w}$), normal force ($\vec{N}$), and tension ($\vec{T}$).
    \item Solving for acceleration or force on one object using $\sum \vec{F} = m\vec{a}$.
\end{itemize}
\end{alertblock}

\column{0.48\textwidth}
\begin{exampleblock}{New in Physics 12}
\begin{itemize}
    \item Applying Newton's Laws to \textbf{systems of multiple objects}.
    \item Strategically choosing the "system of interest" to simplify problems.
    \item Understanding how internal forces cancel out within a system.
    \item Solving for forces between connected objects.
\end{itemize}
\end{exampleblock}
\end{columns}
\end{frame}

\section{Fundamental Concepts Review}

\begin{frame}
\frametitle{Concept: What is a Force?}
\begin{itemize}
    \item A force is fundamentally a \textbf{push} or a \textbf{pull}. \pause
    \item It is a \textbf{vector quantity}, meaning it has both \alert{magnitude} (how strong it is) and \alert{direction}. \pause
    \item The standard unit of force is the \textbf{Newton (N)}.
    \begin{itemize}
        \item 1 N = 1 kg $\cdot$ m/s$^2$
    \end{itemize} \pause
    \item Forces are added together using vector addition to find the \textbf{net force} ($\vec{F}_{net}$).
\end{itemize}
\end{frame}

\begin{frame}
\frametitle{Concept: The Free-Body Diagram (FBD)}
A Free-Body Diagram is a simplified drawing used to analyze the forces on an object or system.
\begin{itemize}
    \item The system of interest is represented by a single \textbf{dot}. \pause
    \item We draw vector arrows representing all \textbf{external forces} acting \textit{on} the system. \pause
    \item We do \textbf{not} draw internal forces or forces exerted \textit{by} the system. \pause
    \item The FBD is the most critical first step for solving almost any dynamics problem.
\end{itemize}
\end{frame}

\begin{frame}
\frametitle{Context: Visualizing Net Force}
\begin{itemize}
    \item Let's visualize how forces on an object combine to produce a net force using a Free-Body Diagram.
    \pause
    \item We will use the example of two ice skaters pushing a third skater from Figure 4.3 in your textbook.
    \pause
    \item The two pushes ($\vec{F}_1$ and $\vec{F}_2$) are individual external forces. The \textbf{net force} ($\vec{F}_{tot}$) is their vector sum, which determines the direction of acceleration.
\end{itemize}
\end{frame}

\begin{frame}
\frametitle{Visualization: Adding Forces on an FBD}
\begin{alertblock}{[Diagram based on Figure 4.3]}
\begin{columns}[T]
\column{0.5\textwidth}
    \textbf{Physical Situation:}
    \begin{itemize}
        \item An overhead view shows two skaters applying forces $\vec{F}_1$ and $\vec{F}_2$ to a third skater.
    \end{itemize}
    \alert{[Image of two skaters pushing a third skater]}
\column{0.5\textwidth}
    \textbf{Free-Body Diagram:}
    \begin{itemize}
        \item The third skater is a dot.
        \item $\vec{F}_1$ and $\vec{F}_2$ are drawn tail-to-dot.
        \item The resultant vector $\vec{F}_{tot}$ shows the net force.
    \end{itemize}
    \alert{[FBD showing two force vectors from a point]}
\end{columns}
\end{alertblock}
\end{frame}

\begin{frame}
\frametitle{Newton's First Law: The Law of Inertia}
\begin{itemize}
    \item "A body at rest remains at rest, or, if in motion, remains in motion at a constant velocity unless acted on by a \textbf{net external force}." \pause
    \item This means an object's velocity \textbf{will not change} if the net force on it is zero ($\vec{F}_{net} = 0$). \pause
    \item \textbf{Inertia} is the property of an object to resist changes in its state of motion. \pause
    \item \textbf{Mass (m)} is the quantitative measure of inertia. More mass means more inertia.
\end{itemize}
\end{frame}

\begin{frame}
\frametitle{Newton's Third Law: Action-Reaction}
\begin{itemize}
    \item "Whenever one body exerts a force on a second body, the first body experiences a force that is equal in magnitude and opposite in direction to the force that it exerts." \pause
    \item Mathematically: $\vec{F}_{A \text{ on } B} = -\vec{F}_{B \text{ on } A}$ \pause
    \item \alert{CRITICAL POINT}: The two forces in an action-reaction pair always act on \textbf{different objects}. \pause
    \begin{itemize}
        \item Therefore, they \textbf{never} cancel each other out when analyzing the motion of a single object.
    \end{itemize}
\end{itemize}
\end{frame}

\begin{frame}
\frametitle{Context: Visualizing Action-Reaction}
\begin{itemize}
    \item Let's visualize how action-reaction pairs work. The key is to see that the forces act on different systems. \pause
    \item We will look at a swimmer pushing off the wall of a pool (based on Figure 4.9). \pause
    \item The "action" is the swimmer pushing on the wall.
    \item The "reaction" is the wall pushing on the swimmer. Only the reaction force affects the swimmer's motion.
\end{itemize}
\end{frame}

\begin{frame}
\frametitle{Visualization: Swimmer at the Wall}
\begin{alertblock}{[Diagram based on Figure 4.9]}
    \begin{itemize}
        \item \textbf{Force 1 (Action):} The swimmer's feet exert a force $\vec{F}_{feet\_on\_wall}$ on the wall. This force acts ON THE WALL.
        \pause
        \item \textbf{Force 2 (Reaction):} The wall exerts an equal and opposite force $\vec{F}_{wall\_on\_feet}$ on the swimmer. This force acts ON THE SWIMMER.
        \pause
        \item The swimmer accelerates because the net external force on \textit{her} (from the wall) is not zero.
    \end{itemize}
    \alert{[Image showing swimmer pushing off a wall, with force vectors on both swimmer and wall]}
\end{alertblock}
\end{frame}

\section{The Concept of a System}

\begin{frame}
\frametitle{The "System": A Key Problem-Solving Tool}
\begin{itemize}
    \item In physics, a \textbf{system} is the object or collection of objects we choose to analyze. \pause
    \item \textbf{External forces} act on the system from the outside.
    \begin{itemize}
        \item These are the forces that cause the system to accelerate.
        \item They are the only forces shown on an FBD of the system.
    \end{itemize} \pause
    \item \textbf{Internal forces} are forces that objects within the system exert on each other.
    \begin{itemize}
        \item These forces always come in action-reaction pairs and cancel out, so they \textbf{do not affect the system's overall acceleration}.
    \end{itemize}
\end{itemize}
\end{frame}

\begin{frame}
\frametitle{Context: Choosing a System}
\begin{itemize}
    \item The choice of system is a strategic decision that can make a problem much easier to solve. \pause
    \item Let's see how changing the system changes which forces are external. \pause
    \item We'll use the example of a professor pushing a cart from Figure 4.10.
\end{itemize}
\end{frame}

\begin{frame}
\frametitle{Visualization: Professor and Cart Systems}
\begin{alertblock}{[Diagram based on Figure 4.10]}
\begin{columns}[T]
\column{0.48\textwidth}
    \textbf{System 1: (Professor + Cart)}
    \begin{itemize}
        \item External forces: Force from floor on feet, friction.
        \item Internal force: Professor pushing cart, cart pushing professor. These cancel.
    \end{itemize}
\column{0.48\textwidth}
    \textbf{System 2: (Cart Only)}
    \begin{itemize}
        \item External forces: Force from professor on cart, friction.
        \item No internal forces to consider.
    \end{itemize}
\end{columns}
\alert{[Image showing a professor pushing a cart, with boxes drawn around "System 1" and "System 2"]}
\end{alertblock}
\end{frame}

\begin{frame}
\frametitle{Newton's Second Law: The Law of Acceleration}
\begin{itemize}
    \item "The acceleration of a system is directly proportional to and in the same direction as the \textbf{net external force} acting on the system, and is inversely proportional to its total mass." \pause
    \item This is the central, quantitative law of dynamics. It connects force, mass, and motion.
\end{itemize}
\end{frame}

\begin{frame}
\frametitle{Essential Equations}
\begin{block}{Newton's Second Law}
\centering
$\vec{F}_{net} = m\vec{a}$
\begin{itemize}
    \item $\vec{F}_{net}$ is the vector sum of all \textit{external} forces on the system.
    \item $m$ is the total mass of the system.
    \item $\vec{a}$ is the acceleration of the system.
\end{itemize}
\end{block}
\pause
\begin{block}{Weight (Force of Gravity)}
\centering
$\vec{w} = m\vec{g}$
\begin{itemize}
    \item $\vec{w}$ is the force of gravity on an object.
    \item $m$ is the object's mass.
    \item $\vec{g}$ is the acceleration due to gravity (approx. 9.8 m/s$^2$ down on Earth).
\end{itemize}
\end{block}
\end{frame}

\section{Problem Solving with Systems}

\begin{frame}
\frametitle{I Do: Getting up to Speed (Example 4.3)}
\begin{block}{Problem}
A professor (65.0 kg) pushes a cart (12.0 kg) with equipment (7.0 kg). She exerts a 150 N backward force on the floor. All forces opposing the motion total 24.0 N. Calculate the acceleration.
\end{block}
\pause
\begin{alertblock}{System of Interest}
For this question, our system is the \textbf{professor + cart + equipment} because we want the acceleration of everything together.
\end{alertblock}
\end{frame}

\begin{frame}
\frametitle{I Do: GUESS Method (G \& U)}
\begin{columns}[T]
\column{0.48\textwidth}
\textbf{G - Givens}
\begin{itemize}
\item $m_{prof} = 65.0$ kg
\item $m_{cart} = 12.0$ kg
\item $m_{equip} = 7.0$ kg
\item Force on floor = 150 N
\item $\implies F_{floor\_on\_prof} = 150$ N [forward]
\item $f = 24.0$ N [backward]
\end{itemize}

\column{0.48\textwidth}
\textbf{U - Unknown}
\begin{itemize}
\item Acceleration, $a = ?$
\end{itemize}
\end{columns}
\end{frame}

\begin{frame}
\frametitle{I Do: GUESS Method (E)}
\textbf{E - Equation}
\begin{itemize}
    \item Start with Newton's Second Law for the whole system:
    \[ F_{net} = m_{total} a \] \pause
    \item The net external force is the forward force from the floor minus the backward friction:
    \[ F_{net} = F_{floor\_on\_prof} - f \] \pause
    \item The total mass is the sum of all parts:
    \[ m_{total} = m_{prof} + m_{cart} + m_{equip} \] \pause
    \item Rearrange for the unknown, $a$:
    \[ a = \frac{F_{net}}{m_{total}} = \frac{F_{floor\_on\_prof} - f}{m_{prof} + m_{cart} + m_{equip}} \]
\end{itemize}
\end{frame}

\begin{frame}
\frametitle{I Do: GUESS Method (S \& S)}
\textbf{S - Substitute}
\begin{itemize}
    \item First, calculate total mass:
    \[ m_{total} = 65.0 + 12.0 + 7.0 = 84.0 \text{ kg} \]
    \item Now substitute into the acceleration equation:
    \[ a = \frac{150 \text{ N} - 24.0 \text{ N}}{84.0 \text{ kg}} \]
\end{itemize}
\pause
\textbf{S - Solve}
\begin{itemize}
    \item Calculate the final value:
    \[ a = \frac{126 \text{ N}}{84.0 \text{ kg}} = 1.5 \text{ m/s}^2 \]
    \item \boxed{a = 1.5 \text{ m/s}^2 \text{ [forward]}}
\end{itemize}
\end{frame}

\begin{frame}
\frametitle{We Do: Force on the Cart (Example 4.4)}
\begin{block}{Problem}
Using the data from the previous problem, calculate the force the professor exerts on the cart.
\end{block}
\pause
\begin{alertblock}{New System of Interest}
Now, our system must be the \textbf{cart + equipment} because the force from the professor is an \textit{external force} on this new system.
\end{alertblock}
\end{frame}

\begin{frame}
\frametitle{We Do: GUESS Method (G \& U)}
\begin{columns}[T]
\column{0.48\textwidth}
\textbf{G - Givens}
\begin{itemize}
\item $m_{cart} = 12.0$ kg
\item $m_{equip} = 7.0$ kg
\item $m_{sys2} = 19.0$ kg
\item $a = 1.5$ m/s$^2$ (from "I do")
\item $f_{total} = 24.0$ N (The problem states this friction applies to cart wheels and air resistance, so it acts on the cart system).
\end{itemize}

\column{0.48\textwidth}
\textbf{U - Unknown}
\begin{itemize}
\item Force from professor on cart, $F_{prof} = ?$
\end{itemize}
\end{columns}
\end{frame}

\begin{frame}
\frametitle{We Do: GUESS Method (E)}
\textbf{E - Equation}
\begin{itemize}
    \item Apply Newton's Second Law to our new system (the cart + equipment):
    \[ F_{net} = m_{sys2} a \]
    \pause
    \item \textbf{Question:} What forces make up $F_{net}$ for this system?
    \begin{itemize}
        \item \textit{Answer: The forward push from the professor ($F_{prof}$) and the backward friction ($f$).}
        \[ F_{prof} - f = m_{sys2} a \]
    \end{itemize}
    \pause
    \item \textbf{Question:} How do we rearrange for the unknown, $F_{prof}$?
    \begin{itemize}
        \item \textit{Answer: Add friction $f$ to both sides.}
        \[ F_{prof} = m_{sys2} a + f \]
    \end{itemize}
\end{itemize}
\end{frame}

\begin{frame}
\frametitle{We Do: GUESS Method (S \& S)}
\textbf{S - Substitute}
\begin{itemize}
    \item Now we plug in the values for our system.
    \pause
    \item \textbf{Question:} What values should we use for $m_{sys2}$, $a$, and $f$?
    \begin{itemize}
        \item $m_{sys2} = 19.0$ kg
        \item $a = 1.5$ m/s$^2$
        \item $f = 24.0$ N
    \end{itemize}
    \[ F_{prof} = (19.0 \text{ kg})(1.5 \text{ m/s}^2) + 24.0 \text{ N} \]
\end{itemize}
\pause
\textbf{S - Solve}
\begin{itemize}
    \item Let's calculate the result.
    \[ F_{prof} = 28.5 \text{ N} + 24.0 \text{ N} = 52.5 \text{ N} \]
    \item \boxed{F_{prof} = 53 \text{ N}}
\end{itemize}
\end{frame}

\begin{frame}
\frametitle{You Do: Drag Force on a Barge (Example 4.7)}
\begin{block}{Problem}
Two tugboats push on a barge. Tugboat 1 exerts a force of $2.7 \times 10^5$ N in the x-direction. Tugboat 2 exerts a force of $3.6 \times 10^5$ N in the y-direction. The mass of the barge is $5.0 \times 10^6$ kg and its acceleration is observed to be $7.5 \times 10^{-2}$ m/s$^2$ in the direction of the net applied force from the tugboats.
\newline
\newline
What is the drag force of the water on the barge resisting the motion?
\end{block}

\begin{alertblock}{Hint}
1. Find the magnitude and direction of the total applied force from the tugboats.
2. Calculate the net force needed to cause the observed acceleration ($F_{net}=ma$).
3. The drag force is the difference between the applied force and the net force.
\end{alertblock}
\end{frame}

\section{Additional Worked Examples}

\begin{frame}
\frametitle{Example: Rugby Players (Problem 16)}
A rugby player (90.0 kg) is accelerating at $1.20 \mathrm{~m} / \mathrm{s}^{2}$ backward while being pushed by an opposing player exerting 800 N.
\begin{itemize}
    \item[(a)] What is the force of friction between the losing player's feet and the grass?
    \item[(b)] What force must the winning player (110 kg) exert on the ground to move forward at the same acceleration?
\end{itemize}
\end{frame}

\begin{frame}
\frametitle{Problem 16 - Solution (a)}
\begin{columns}[T]
    \column{0.5\textwidth}
        \begin{block}{System of Interest: Losing Player}
            \begin{itemize}
                \item $F_{\text{net}} = F_{push} - f = ma$
                \item $f = F_{push} - ma$
                \item $f = 800 \text{ N} - (90.0 \text{ kg})(1.20 \text{ m/s}^2)$
                \item $f = 800 \text{ N} - 108 \text{ N}$
                \item \boxed{f = 692 \text{ N}}
            \end{itemize}
        \end{block}
    \column{0.5\textwidth}
        \begin{figure}
            \centering
            \includegraphics[width=0.7\linewidth]{Screenshot 2024-10-18 111640.png}
        \end{figure}
\end{columns}
\end{frame}

\begin{frame}
\frametitle{Problem 16 - Solution (b)}
\begin{columns}[T]
    \column{0.5\textwidth}
        \begin{block}{System of Interest: Both Players}
            \begin{itemize}
                \item Let $F_{ground}$ be the force the winner exerts on the ground.
                \item $F_{net} = F_{ground} - f = (m_1+m_2)a$
                \item $F_{ground} = (m_1+m_2)a + f$
                \item $F_{ground} = (90.0+110)\text{kg}(1.20 \text{m/s}^2) + 692 \text{N}$
                \item $F_{ground} = 240 \text{N} + 692 \text{N}$
                \item \boxed{F_{ground} = 932 \text{N}}
            \end{itemize}
        \end{block}
    \column{0.5\textwidth}
        \begin{figure}
            \centering
            \includegraphics[width=0.7\linewidth]{Screenshot 2024-10-18 111809.png}
        \end{figure}
\end{columns}
\end{frame}

\section{Conclusion}

\begin{frame}
\frametitle{Reading Homework}
Please read the following sections from Chapter 4 in your textbook. They contain important examples of specific forces we will use in later chapters.
\begin{itemize}
    \item \textbf{Section 4.5: Normal, Tension, and Other Examples of Forces}
    \begin{itemize}
        \item Detailed examples of Normal Force on inclines and Tension.
    \end{itemize}
    \pause
    \item \textbf{Section 4.8: Extended Topic: The Four Basic Forces}
    \begin{itemize}
        \item An introduction to the fundamental forces of nature (Gravitational, Electromagnetic, Weak Nuclear, Strong Nuclear).
    \end{itemize}
\end{itemize}
\end{frame}

\begin{frame}
\frametitle{Summary}
\begin{itemize}
    \item \textbf{Newton's First Law} defines inertia and the condition for constant velocity ($\vec{F}_{net}=0$). \pause
    \item \textbf{Newton's Third Law} describes action-reaction pairs, which act on different objects. \pause
    \item \textbf{Newton's Second Law ($\vec{F}_{net}=m\vec{a}$)} is the core problem-solving tool that links forces to motion. \pause
    \item The key to solving complex dynamics problems is to correctly \alert{define the system of interest}. \pause
    \begin{itemize}
        \item This choice determines which forces are \textbf{external} (and included in $\vec{F}_{net}$) and which are \textbf{internal} (and ignored).
    \end{itemize} \pause
    \item Every analysis should begin with a \textbf{Free-Body Diagram} for your chosen system.
\end{itemize}
\end{frame}

\end{document}