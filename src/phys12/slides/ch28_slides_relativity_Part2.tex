\documentclass{beamer}
% Use DS9 global theme (includes pgfplots for visualization)
\usepackage{../../../../latex-beamer/shared/templates/ds9_theme}

% Title page configuration
\title[Relativity Part 2]{PHYS12 CH:28.4-28.6}
\subtitle{Relativistic Mechanics}
\author[Mr. Gullo]{Mr. Gullo}
\date[Nov 21 2025]{November 21, 2025}

\begin{document}
\frame{\titlepage}

\begin{frame}
\frametitle{Learning Objectives}
By the end of this lesson, you will be able to:
\pause
\begin{itemize}
    \item Apply the relativistic velocity addition formula.
    \item Calculate relativistic momentum and compare it to classical momentum.
    \item Define rest energy, total energy, and kinetic energy in relativistic terms.
    \item Solve problems involving mass-energy equivalence ($E=mc^2$).
\end{itemize}
\end{frame}

\begin{frame}
\frametitle{Progression and Recap}
\begin{alertblock}{From Part 1}
\begin{itemize}
    \item $\gamma = \frac{1}{\sqrt{1 - v^2/c^2}}$
    \item Time dilates ($\Delta t = \gamma \Delta t_0$)
    \item Length contracts ($L = L_0 / \gamma$)
\end{itemize}
\end{alertblock}
\pause
\begin{block}{Today: Dynamics}
\begin{itemize}
    \item What happens when we push objects near the speed of light?
    \item Does $F=ma$ still work? (Spoiler: Not simply)
    \item How do we add velocities correctly?
\end{itemize}
\end{block}
\end{frame}

\section{28.4 Relativistic Addition of Velocities}

\begin{frame}
\frametitle{28.4 The Limit of Galilean Relativity}
Classically, if you run at $u'$ on a train moving at $v$, your ground speed is simply:
$$ u = v + u' $$
\pause
\begin{alertblock}{Problem}
If a spaceship moves at $0.8c$ and fires a probe forward at $0.8c$, classical physics says the probe moves at $1.6c$.
\textbf{This violates the second postulate ($c$ is the limit).}
\end{alertblock}
\end{frame}

\begin{frame}
\frametitle{Essential Equation: Velocity Addition}
\begin{columns}
\column{0.5\textwidth}
\begin{block}{Relativistic Velocity Addition}
$$ u = \frac{v + u'}{1 + \frac{vu'}{c^2}} $$
\end{block}

\column{0.5\textwidth}
\pause
\begin{itemize}
    \item $u$: Velocity relative to stationary frame.
    \item $v$: Velocity of the moving frame (e.g., the ship).
    \item $u'$: Velocity relative to the moving frame (e.g., the probe).
\end{itemize}
\end{columns}
\pause
\vspace{1em}
Notice: If $v \ll c$ and $u' \ll c$, the denominator $\approx 1$, giving back $u \approx v + u'$.
\end{frame}

\begin{frame}
\frametitle{Example: I Do - Velocity Addition}
\textbf{Problem}: A spaceship travels away from Earth at $v = 0.5c$. It fires a missile forward at speed $u' = 0.5c$ relative to the ship. What is the missile's speed relative to Earth?
\end{frame}

% GUESS Frame 1: G and U
\begin{frame}
\frametitle{I Do: Velocity Addition - G \& U}
\begin{columns}[T]
\column{0.48\textwidth}
\textbf{G - Givens}
\begin{itemize}
    \item $v = 0.5c$ (Ship relative to Earth)
    \item $u' = 0.5c$ (Missile relative to Ship)
\end{itemize}

\column{0.48\textwidth}
\pause
\textbf{U - Unknown}
\begin{itemize}
    \item $u = ?$ (Missile relative to Earth)
\end{itemize}
\end{columns}
\end{frame}

% GUESS Frame 2: E
\begin{frame}
\frametitle{I Do: Velocity Addition - Equation}
\textbf{E - Equation}
\begin{itemize}
    \item Use the relativistic addition formula:
    $$ u = \frac{v + u'}{1 + \frac{vu'}{c^2}} $$
\end{itemize}
\end{frame}

% GUESS Frame 3: S and S
\begin{frame}
\frametitle{I Do: Velocity Addition - Substitute \& Solve}
\textbf{S - Substitute}
\begin{itemize}
    \item $u = \frac{0.5c + 0.5c}{1 + \frac{(0.5c)(0.5c)}{c^2}}$
    \item $u = \frac{1.0c}{1 + 0.25}$
\end{itemize}
\pause
\textbf{S - Solve}
\begin{itemize}
    \item $u = \frac{1.0c}{1.25} = 0.8c$
    \item \boxed{u = 0.8c}
    \item \textit{Result is less than $c$, as required.}
\end{itemize}
\end{frame}

\section{28.5 Relativistic Momentum}

\begin{frame}
\frametitle{28.5 Relativistic Momentum}
Newtonian momentum $p = mv$ is not conserved at high speeds. We must adjust the definition.
\pause
\begin{block}{Relativistic Momentum}
$$ p = \gamma mv = \frac{mv}{\sqrt{1 - \frac{v^2}{c^2}}} $$
\end{block}
\pause
\begin{itemize}
    \item As $v \to c$, $\gamma \to \infty$, so $p \to \infty$.
    \item This explains why a massive object cannot reach $c$: it would require infinite momentum (and infinite energy).
\end{itemize}
\end{frame}

\begin{frame}
\frametitle{Concept Visualization: Momentum Limit}
\begin{alertblock}{}
\begin{center}
		\includegraphics[width=0.5\linewidth]{pasted-images/ch28_slides_relativity_Part2-10-29-52.png}
	\end{center}
\pause
\begin{itemize}
    \item \textbf{Classical line}: Straight line ($p=mv$).
    \item \textbf{Relativistic curve}: Follows classical line at low speeds, then curves upward sharply, approaching a vertical asymptote at $v=c$.
\end{itemize}
\end{alertblock}
\end{frame}

\section{28.6 Relativistic Energy}

\begin{frame}
\frametitle{28.6 Rest Energy and Total Energy}
Einstein's most famous equation relates mass and energy.
\pause
\begin{block}{Rest Energy ($E_0$)}
The energy an object has simply because it has mass.
$$ E_0 = mc^2 $$
\end{block}
\pause
\begin{block}{Total Energy ($E$)}
The sum of rest energy and kinetic energy.
$$ E = \gamma mc^2 = \frac{mc^2}{\sqrt{1 - \frac{v^2}{c^2}}} $$
\end{block}
\end{frame}

\begin{frame}
\frametitle{Relativistic Kinetic Energy}
Kinetic Energy is the "extra" energy due to motion.
$$ E = E_0 + KE $$
\pause
Rearranging for KE:
$$ KE = E - E_0 = \gamma mc^2 - mc^2 $$
\pause
\begin{block}{Relativistic Kinetic Energy}
$$ KE = (\gamma - 1)mc^2 $$
\end{block}
\textit{Note: At low speeds, this simplifies to $\frac{1}{2}mv^2$.}
\end{frame}

\begin{frame}
\frametitle{We Do: Relativistic Energy}
\textbf{Problem}: An electron ($m = 9.11 \times 10^{-31}$ kg) is accelerated to $0.999c$. What is its total energy?
\pause
\vspace{1em}
\textbf{Steps}:
\begin{enumerate}
    \item Calculate $\gamma$ for $v = 0.999c$.
    \item Use $E = \gamma mc^2$.
    \item (Optional) Convert Joules to eV or MeV.
\end{enumerate}
\end{frame}

\begin{frame}
\frametitle{You Do: Practice}
\textbf{Problem}: A proton has a rest energy of 938 MeV. It is moving such that its total energy is 2000 MeV.
\pause
\begin{enumerate}
    \item What is its kinetic energy? ($KE = E - E_0$)
    \item What is the value of $\gamma$?
    \item How fast is it moving? (Solve $\gamma$ for $v$).
\end{enumerate}
\end{frame}

\begin{frame}
\frametitle{Summary}
\pause
\begin{itemize}
    \item \textbf{Velocity Addition}: Velocities do not add linearly; $c$ is the max speed.
    \item \textbf{Momentum}: $p = \gamma mv$. Infinite momentum is required to reach $c$.
    \item \textbf{Mass-Energy}: Mass is a form of energy ($E_0 = mc^2$).
    \item \textbf{Total Energy}: $E = \gamma mc^2$.
    \item \textbf{Kinetic Energy}: $KE = (\gamma - 1)mc^2$.
\end{itemize}
\end{frame}

\end{document}
