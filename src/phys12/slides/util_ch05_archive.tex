\documentclass{beamer}
\usepackage[utf8]{inputenc}
\usepackage{amsmath}
\usepackage{graphicx}
\usepackage{xcolor}
\usepackage{tikz}

\usetheme{Madrid}
\usecolortheme{default}

% Define custom colors inspired by Star Trek DS9
\definecolor{ds9blue}{RGB}{25,25,112} % Midnight Blue
\definecolor{ds9gold}{RGB}{218,165,32} % Goldenrod
\definecolor{ds9grey}{RGB}{105,105,105} % Dim Gray
\definecolor{ds9red}{RGB}{178,34,34} % Firebrick

% Customize the colors
\setbeamercolor{title}{fg=ds9gold}
\setbeamercolor{frametitle}{bg=ds9blue, fg=white}
\setbeamercolor{block title}{bg=ds9gold, fg=black}
\setbeamercolor{block body}{bg=ds9grey!20, fg=black}
\setbeamercolor{section in toc}{fg=ds9gold}
\setbeamercolor{subsection in toc}{fg=ds9gold!70}
\setbeamercolor{footline}{bg=ds9blue, fg=white}
\setbeamercolor{author in head/foot}{fg=white}
\setbeamercolor{date in head/foot}{fg=white}
\setbeamercolor{title in head/foot}{fg=white}

% Title page configuration
\title[ ]{PHYS12 CH5: Friction, Drag Forces, and Elasticity}
\subtitle{ }
\author[Mr. Gullo]{Mr. Gullo}
\date[Oct 2024]{October 2024}

% Add logo
\logo{\includegraphics[width=0.1\linewidth]{phys12-shared-cinec-logo.png}}

% Table of contents at the beginning of each section
\AtBeginSection[]
{
  \begin{frame}
    \frametitle{Table of Contents}
    \tableofcontents[currentsection]
  \end{frame}
}

\begin{document}

\frame{\titlepage}

\begin{frame}
\frametitle{Table of Contents}
\tableofcontents
\end{frame}
\begin{frame}
\frametitle{Key Equations: Friction Forces}
\begin{block}{Static and Kinetic Friction}
\begin{align*}
f_s &\leq \mu_s N \quad \text{(static friction)} \\
f_k &= \mu_k N \quad \text{(kinetic friction)} \\
N &= mg \quad \text{(normal force)}
\end{align*}
\end{block}


\begin{alertblock}{Important Notes}
\begin{itemize}
\item $\mu_s$ = coefficient of static friction (typically 0.3-0.8)
\item $\mu_k$ = coefficient of kinetic friction (typically less than $\mu_s$)
\item $N$ = normal force (perpendicular to surface)
\item $m$ = mass of object
\item $g$ = acceleration due to gravity (9.80 m/s²)
\end{itemize}
\end{alertblock}
\end{frame}

\begin{frame}
\frametitle{Key Equations: Drag Forces}
\begin{block}{Drag Force Equation}
\begin{align*}
F_D &= \frac{1}{2}\rho C A v^2
\end{align*}
where:
\begin{itemize}
\item $F_D$ = drag force
\item $\rho$ = fluid density (air: 1.20 kg/m³ at sea level)
\item $C$ = drag coefficient
\item $A$ = cross-sectional area
\item $v$ = velocity
\end{itemize}
\end{block}
\end{frame}

\begin{frame}
\begin{block}{Terminal Velocity}
\begin{align*}
v_t &\propto \sqrt{\frac{m}{A}} \\
F_D &= mg \quad \text{(at terminal velocity)}
\end{align*}
\end{block}

\begin{alertblock}{Terminal Velocity Notes}
\begin{itemize}
\item Terminal velocity occurs when drag force equals weight
\item Proportional to square root of mass-to-area ratio
\item Air resistance increases with velocity squared
\item Dependent on body position (affects area $A$)
\end{itemize}
\end{alertblock}
\end{frame}

\begin{frame}
\frametitle{Key Equations: Elasticity}
\begin{block}{Stress and Strain Relationships}
\begin{align*}
\text{Stress} &= \frac{F}{A} \\
\text{Strain} &= \frac{\Delta L}{L_0} \\
Y &= \frac{\text{Stress}}{\text{Strain}}
\end{align*}
\end{block}
\end{frame}

\begin{frame}
\begin{block}{Length Change}
\begin{align*}
\Delta L &= \frac{1}{Y}\frac{F}{A}L_0
\end{align*}
where:
\begin{itemize}
\item $Y$ = Young's modulus
\item $F$ = applied force
\item $A$ = cross-sectional area
\item $\Delta L$ = change in length
\item $L_0$ = original length
\end{itemize}
\end{block}

\begin{alertblock}{Material Properties}
\begin{itemize}
\item Young's modulus is material-specific
\item Bone: $Y \approx 9 \times 10^9$ N/m²
\item Valid only in elastic region
\item Assumes uniform cross-section
\end{itemize}
\end{alertblock}
\end{frame}

\section{5.1 Friction}

\begin{frame}
\frametitle{Problem 4: Friction Force}
\begin{block}{Problem Statement}
Suppose you have a 120-kg wooden crate resting on a wood floor.
\begin{itemize}
    \item[(a)] What maximum force can you exert horizontally on the crate without moving it?
    \item[(b)] If you continue to exert this force once the crate starts to slip, what will its acceleration then be?
\end{itemize}
\end{block}
\end{frame}

\begin{frame}
\frametitle{Problem 4: Solution}
\begin{block}{Part (a): Maximum Force}
\begin{align*}
f &\leq \mu_sN \\
f &= \mu_smg \\
f &= 0.5(120 \text{ kg})(9.80 \text{ m/s}^2) \\
f &= 588 \text{ N}
\end{align*}
\end{block}

\begin{block}{Part (b): Acceleration After Slipping}
\begin{align*}
f_k &= \mu_kmg = 0.3(120 \text{ kg})(9.80 \text{ m/s}^2) = 352.8 \text{ N} \\
F_{net} &= 588 \text{ N} - 352.8 \text{ N} = 235.2 \text{ N} \\
a &= \frac{F_{net}}{m} = \frac{235.2 \text{ N}}{120 \text{ kg}} = 1.96 \text{ m/s}^2
\end{align*}
\end{block}
\end{frame}

\section{5.2 Drag Forces}

\begin{frame}
\frametitle{Problem 21: Skydiver Terminal Velocity}
\begin{block}{Problem Statement}
A 60-kg and a 90-kg skydiver jump from an airplane at an altitude of 6000 m, both falling in a headfirst position.
Make some assumption on their frontal areas and calculate their terminal velocities. \\
How long will it take for each skydiver to reach the ground (assuming the time to reach terminal velocity is small)? \\
Assume all values are accurate to three significant digits.
\end{block}

\begin{block}{Terminal Velocity}
\begin{align*}
v_t &\propto \sqrt{\frac{m}{A}} \\
F_D &= mg \quad \text{(at terminal velocity)}
\end{align*}
\end{block}
\end{frame}

\begin{frame}
\frametitle{Problem 21: Solution}
\begin{block}{Terminal Velocities}
Assumption: frontal area of 0.7 m² for both.
\begin{align*}
v_1 &= \left(\frac{60}{80}\right)^{0.5} \times 115 = 100 \text{ m/s} \\
v_2 &= \left(\frac{90}{80}\right)^{0.5} \times 115 = 122 \text{ m/s}
\end{align*}
\end{block}

\begin{block}{Time to Ground}
\begin{align*}
t_1 &= \frac{6000}{100} = 60.0 \text{ s} \\
t_2 &= \frac{6000}{122} = 49.2 \text{ s}
\end{align*}
\end{block}
\end{frame}

\section{5.3 Elasticity: Stress and Strain}

\begin{frame}
\frametitle{Problem 30: Bone Compression}
\begin{block}{Problem Statement}
During a wrestling match, a 150 kg wrestler briefly stands on one hand during a maneuver. Calculate the shortening of the upper arm bone, given:
\begin{itemize}
    \item Length: 38.0 cm
    \item Radius: 2.10 cm
    \item Young's modulus for bone: $9 \times 10^9 \text{ N/m}^2$
\end{itemize}
\end{block}
\end{frame}

\begin{frame}
\begin{block}{Length Change}
\begin{align*}
\Delta L &= \frac{1}{Y}\frac{F}{A}L_0
\end{align*}
where:
\begin{itemize}
\item $Y$ = Young's modulus
\item $F$ = applied force
\item $A$ = cross-sectional area
\item $\Delta L$ = change in length
\item $L_0$ = original length
\end{itemize}
\end{block}

\begin{alertblock}{Material Properties}
\begin{itemize}
\item Young's modulus is material-specific
\item Bone: $Y \approx 9 \times 10^9$ N/m²
\item Valid only in elastic region
\item Assumes uniform cross-section
\end{itemize}
\end{alertblock}
\end{frame}

\begin{frame}
\frametitle{Problem 30: Solution}
\begin{block}{Step-by-step Solution}
\begin{align*}
F &= mg = 150 \text{ kg} \times 9.80 \text{ m/s}^2 = 1470 \text{ N} \\
A &= \pi r^2 = \pi(0.0210 \text{ m})^2 = 0.001385 \text{ m}^2 \\
\Delta L &= \frac{1}{Y}\frac{F}{A}L_0 \\
\Delta L &= \frac{1}{9 \times 10^9}\frac{1470}{0.001385}(0.380) \\
\Delta L &= 4.5 \times 10^{-5} \text{ m}
\end{align*}
\end{block}
\end{frame}

\begin{frame}
\frametitle{Summary}
\begin{itemize}
    \item Covered three main topics:
    \begin{itemize}
        \item Friction forces and acceleration
        \item Drag forces and terminal velocity
        \item Elasticity, stress, and strain
    \end{itemize}
    \item Solved problems involving:
    \begin{itemize}
        \item Static and kinetic friction
        \item Terminal velocity calculations
        \item Material deformation
    \end{itemize}
    \item Applied fundamental physics principles to real-world scenarios
\end{itemize}
\end{frame}

\end{document}

\end{document}