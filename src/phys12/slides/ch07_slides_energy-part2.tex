\documentclass{beamer}
% Use DS9 global theme
\usepackage{../../../../shared/templates/ds9_theme}

% Title page configuration
\title[PHYS12 CH7.4-7.6]{Work and Energy}
\subtitle{Chapter 7.1-7.3}
\author[Mr. Gullo]{Mr. Gullo}
\date[Nov 2024]{November 2024}

% Add logo
\logo{\includegraphics[width=0.1\linewidth]{phys12-shared-cinec-logo.png}}

% Table of contents at the beginning of each section
\AtBeginSection[]
{
  \begin{frame}
    \frametitle{Table of Contents}
    \tableofcontents[currentsection]
  \end{frame}
}

\begin{document}

\frame{\titlepage}

\section{7.4 Conservative Forces and Potential Energy}

\begin{frame}
\frametitle{Conservative Forces}
\begin{itemize}
    \item A force is conservative if the work done by it on a particle is independent of the path taken
    \item Examples of conservative forces:
    \begin{itemize}
        \item Gravitational force
        \item Elastic force (springs)
        \item Electrostatic force
    \end{itemize}
    \item Work done by conservative forces:
    \begin{itemize}
        \item Can be recovered
        \item Depends only on initial and final positions
        \item Path-independent
    \end{itemize}
\end{itemize}
\end{frame}

\begin{frame}
\frametitle{Potential Energy}
\begin{itemize}
    \item Potential energy: energy stored due to position or configuration
    \item For conservative forces:
    \[\Delta PE = -W_{\text{cons}}\]
    where:
    \begin{itemize}
        \item $\Delta PE$ is change in potential energy
        \item $W_{\text{cons}}$ is work done by conservative force
    \end{itemize}
    \item Types:
    \begin{itemize}
        \item Gravitational potential energy
        \item Elastic potential energy
    \end{itemize}
\end{itemize}
\end{frame}

\section{7.5 Conservation of Mechanical Energy}

\begin{frame}
\frametitle{Conservation of Mechanical Energy}
\begin{itemize}
    \item Mechanical energy = Kinetic energy + Potential energy
    \[E = K + PE\]
    \item When only conservative forces act:
    \[E_{\text{initial}} = E_{\text{final}}\]
    \item Or:
    \[K_i + PE_i = K_f + PE_f\]
    \item This principle helps solve many physics problems!
\end{itemize}
\end{frame}

\section{7.6 Nonconservative Forces}

\begin{frame}
\frametitle{Nonconservative Forces}
\begin{itemize}
    \item Work depends on path taken
    \item Examples:
    \begin{itemize}
        \item Friction
        \item Air resistance
        \item Tension in a rope
    \end{itemize}
    \item With nonconservative forces:
    \[\Delta E = W_{\text{nc}}\]
    where $W_{\text{nc}}$ is work done by nonconservative forces
\end{itemize}
\end{frame}

\section{Example Problems: I do, We do, You do}

\begin{frame}
\frametitle{I Do: Roller Coaster Problem}
A roller coaster car (mass 500 kg) starts from rest at height 40 m. What is its speed at height 15 m?
\end{frame}

\begin{frame}
\begin{block}{Solution}
Using conservation of mechanical energy:
\begin{align*}
E_i &= E_f \\
mgh_i + \frac{1}{2}mv_i^2 &= mgh_f + \frac{1}{2}mv_f^2 \\
(500)(9.8)(40) + 0 &= (500)(9.8)(15) + \frac{1}{2}(500)v_f^2 \\
196000 &= 73500 + 250v_f^2 \\
v_f &= 22.6 \text{ m/s}
\end{align*}
\end{block}
\end{frame}

\begin{frame}
\frametitle{We Do: Spring Problem}
Let's solve together: A 2 kg mass is attached to a spring (k = 100 N/m) and compressed 0.3 m. What height will it reach when released?
\end{frame}

\begin{frame}
\begin{block}{Step-by-Step}
1. Initial energy (compressed spring):
\[\frac{1}{2}kx^2 = \frac{1}{2}(100)(0.3)^2 = 4.5 \text{ J}\]
2. At maximum height:
\[mgh = 4.5 \text{ J}\]
3. Solve for h:
\[h = \frac{4.5}{(2)(9.8)} = 0.23 \text{ m}\]
\end{block}
\end{frame}

\begin{frame}
\frametitle{You Do: Practice Problem}
Now try this one:

\begin{block}{Problem}
A 0.5 kg ball is thrown upward with initial velocity 15 m/s. Calculate:
\begin{itemize}
    \item Maximum height reached
    \item Velocity when it returns to half the maximum height
\end{itemize}
Use conservation of energy principles!
\end{block}
\pause
\begin{block}{Hint}
Start with:
\[\frac{1}{2}mv_i^2 = mgh_{\text{max}}\]
Then use conservation of energy again for the second part.
\end{block}
\end{frame}

\begin{frame}
\frametitle{Key Takeaways}
\begin{itemize}
    \item Conservative forces:
    \begin{itemize}
        \item Path-independent work
        \item Enable potential energy definition
    \end{itemize}
    \item Conservation of mechanical energy:
    \begin{itemize}
        \item Powerful problem-solving tool
        \item Only valid for conservative forces
    \end{itemize}
    \item Nonconservative forces:
    \begin{itemize}
        \item Change total mechanical energy
        \item Require additional work calculations
    \end{itemize}
\end{itemize}
\end{frame}

\end{document}