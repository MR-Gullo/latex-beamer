\documentclass{beamer}
% Required packages
\usepackage{amsmath}
\usepackage{physics}
\usepackage{graphicx}
\usepackage{siunitx}
\usepackage{xcolor}
% Set image search paths
\graphicspath{{../images/}{../../shared/images/}}

% Define custom colors for DS9 theme
\definecolor{ds9blue}{RGB}{25,25,112}
\definecolor{ds9gold}{RGB}{218,165,32}
\definecolor{ds9grey}{RGB}{105,105,105}
\definecolor{ds9red}{RGB}{178,34,34}
% Set up the Madrid theme with custom colors
\usetheme{Madrid}
\usecolortheme{whale}
\setbeamercolor{palette primary}{bg=ds9blue,fg=white}
\setbeamercolor{palette secondary}{bg=ds9grey,fg=white}
\setbeamercolor{palette tertiary}{bg=ds9gold,fg=black}
\setbeamercolor{palette quaternary}{bg=ds9red,fg=white}
\setbeamercolor{structure}{fg=ds9blue}
\setbeamercolor{title}{fg=ds9gold}
\setbeamercolor{subtitle}{fg=ds9gold}
\setbeamercolor{frametitle}{bg=ds9blue,fg=white}
\setbeamercolor{block title}{bg=ds9blue,fg=white}
\setbeamercolor{block body}{bg=ds9grey!20,fg=black}

% Title page configuration
\title[Magnetism \& EM Induction]{PHYS11 CH: 22-23}
\subtitle{Magnetism and Electromagnetic Induction}
\author[Mr. Gullo]{Mr. Gullo}
\date[April 2025]{April 10, 2025}
\institute[Physics]{PHYS11 - College Physics}

\begin{document}

% Title slide
\frame{\titlepage}

% Outline slide
\begin{frame}
\frametitle{Outline}
\tableofcontents
\end{frame}

\section{Introduction to Magnetism}

\begin{frame}
\frametitle{Learning Objectives}
\begin{block}{By the end of this presentation, you will be able to:}
\begin{itemize}
    \item Describe the properties of magnets and magnetic fields
    \item Apply the right hand rules to determine directions of magnetic forces and fields
    \item Calculate magnetic forces on moving charges and current-carrying conductors
    \item Explain electromagnetic induction and apply Faraday's law
    \item Understand the operation of generators, motors, and transformers
    \item Calculate inductance and analyze RL and RLC circuits
\end{itemize}
\end{block}
\end{frame}

\begin{frame}
\frametitle{Magnets and Magnetic Fields}
\begin{columns}
\column{0.6\textwidth}
\begin{itemize}
    \item Magnets have north and south poles
    \item North pole: attracted to Earth's geographic north
    \item Like poles repel, unlike poles attract
    \item Poles always occur in pairs (no magnetic monopoles)
    \item All magnetism is created by electric current
\end{itemize}
\column{0.4\textwidth}
\alert{[Image of bar magnets showing attraction between opposite poles and repulsion between like poles]}
\end{columns}
\end{frame}

\begin{frame}
\frametitle{Magnetic Field Lines}
\begin{block}{Properties of Magnetic Field Lines}
\begin{enumerate}
    \item Field is tangent to the magnetic field line
    \item Field strength is proportional to line density
    \item Field lines cannot cross
    \item Field lines are continuous loops
\end{enumerate}
\end{block}
\alert{[Diagram showing magnetic field lines around a bar magnet]}
\end{frame}

\begin{frame}
\frametitle{Ferromagnets and Electromagnets}
\begin{columns}
\column{0.5\textwidth}
\textbf{Ferromagnetic Materials:}
\begin{itemize}
    \item Iron, cobalt, nickel, gadolinium
    \item Atoms act like small magnets
    \item Form domains where magnetic moments align
    \item Curie temperature: above this temperature, ferromagnetism disappears
\end{itemize}
\column{0.5\textwidth}
\textbf{Electromagnets:}
\begin{itemize}
    \item Current through a wire creates magnetic field
    \item Adding ferromagnetic core increases field strength
    \item Applications: motors, generators, MRI machines
\end{itemize}
\end{columns}
\alert{[Image showing magnetic domains in a ferromagnetic material]}
\end{frame}

\section{Magnetic Forces}

\begin{frame}
\frametitle{Force on a Moving Charge}
\begin{block}{Magnetic Force Formula}
\begin{equation}
\vec{F} = q\vec{v} \times \vec{B} \quad \text{or} \quad F = qvB\sin\theta
\end{equation}
\end{block}
\begin{itemize}
    \item $q$ = charge (C)
    \item $v$ = velocity (m/s)
    \item $B$ = magnetic field strength (T)
    \item $\theta$ = angle between $\vec{v}$ and $\vec{B}$
    \item Direction determined by Right Hand Rule 1 (RHR-1)
\end{itemize}
\alert{[Illustration of Right Hand Rule 1]}
\end{frame}

\begin{frame}
\frametitle{Right Hand Rule 1 (RHR-1)}
\begin{block}{RHR-1 for Force on a Moving Charge}
Point the thumb of your right hand in the direction of the velocity $\vec{v}$, fingers in the direction of the magnetic field $\vec{B}$, and the force $\vec{F}$ is perpendicular to the palm.
\end{block}
\begin{itemize}
    \item For negative charges, force is in opposite direction
    \item Force is always perpendicular to both $\vec{v}$ and $\vec{B}$
    \item When $\vec{v}$ and $\vec{B}$ are parallel, $\vec{F} = 0$
\end{itemize}
\end{frame}

\begin{frame}
\frametitle{Motion of Charged Particles in Magnetic Fields}
\begin{columns}
\column{0.6\textwidth}
\begin{itemize}
    \item When velocity perpendicular to field, particles move in circular path
    \item Radius of circular path: $r = \frac{mv}{qB}$
    \item Magnetic force provides centripetal force
    \item Period of revolution independent of speed
    \item Helical path when velocity has component parallel to field
\end{itemize}
\column{0.4\textwidth}
\alert{[Diagram showing circular motion of charged particle in magnetic field]}
\end{columns}
\end{frame}

\begin{frame}
\frametitle{Force on Current-Carrying Conductors}
\begin{block}{Magnetic Force on a Current-Carrying Conductor}
\begin{equation}
\vec{F} = I\vec{l} \times \vec{B} \quad \text{or} \quad F = IlB\sin\theta
\end{equation}
\end{block}
\begin{itemize}
    \item $I$ = current (A)
    \item $l$ = length of conductor (m)
    \item $B$ = magnetic field strength (T)
    \item $\theta$ = angle between $\vec{l}$ and $\vec{B}$
    \item Direction follows RHR-1 with thumb in direction of current
\end{itemize}
\alert{[Diagram showing force on a current-carrying wire in a magnetic field]}
\end{frame}

\begin{frame}
\frametitle{Torque on Current Loops}
\begin{block}{Torque on a Current Loop}
\begin{equation}
\tau = NIAB\sin\theta
\end{equation}
\end{block}

\begin{itemize}
    \item $N$ = number of turns
    \item $I$ = current (A)
    \item $A$ = area of loop (m$^2$)
    \item $B$ = magnetic field strength (T)
    \item $\theta$ = angle between loop normal and $\vec{B}$
\end{itemize}

\textbf{Applications:} Electric motors, meters
\alert{[Diagram of a current loop in a magnetic field showing torque]}
\end{frame}

\section{Magnetic Fields from Currents}

\begin{frame}
\frametitle{Magnetic Fields Produced by Currents}
\begin{block}{Magnetic Field from a Long Straight Wire}
\begin{equation}
B = \frac{\mu_0 I}{2\pi r}
\end{equation}
\end{block}

\begin{itemize}
    \item $\mu_0 = 4\pi \times 10^{-7}$ T$\cdot$m/A (permeability of free space)
    \item $I$ = current (A)
    \item $r$ = distance from wire (m)
    \item Direction determined by Right Hand Rule 2 (RHR-2)
\end{itemize}
\alert{[Diagram showing magnetic field lines around a current-carrying wire]}
\end{frame}

\begin{frame}
\frametitle{Right Hand Rule 2 (RHR-2)}
\begin{block}{RHR-2 for Magnetic Field from Current}
Point the thumb of your right hand in the direction of the current, and your fingers curl in the direction of the magnetic field lines.
\end{block}
\begin{itemize}
    \item Field lines form concentric circles around a straight wire
    \item Field strength decreases with distance from wire
    \item Total field from any current path is the vector sum of fields from all segments
\end{itemize}
\end{frame}

\begin{frame}
\frametitle{Magnetic Fields from Loops and Solenoids}
\begin{columns}
\column{0.5\textwidth}
\textbf{Circular Loop:}
\begin{equation}
B = \frac{\mu_0 I}{2R} \quad \text{(at center)}
\end{equation}
\begin{itemize}
    \item $R$ = radius of loop (m)
    \item Direction from RHR-2
\end{itemize}
\column{0.5\textwidth}
\textbf{Solenoid:}
\begin{equation}
B = \mu_0 n I \quad \text{(inside)}
\end{equation}
\begin{itemize}
    \item $n$ = number of turns per unit length
    \item Field inside is uniform
\end{itemize}
\end{columns}
\alert{[Diagram comparing field patterns of loop and solenoid]}
\end{frame}

\section{Electromagnetic Induction}

\begin{frame}
\frametitle{Magnetic Flux}
\begin{block}{Magnetic Flux Definition}
\begin{equation}
\Phi = BA\cos\theta
\end{equation}
\end{block}
\begin{itemize}
    \item $B$ = magnetic field strength (T)
    \item $A$ = area (m$^2$)
    \item $\theta$ = angle between $\vec{B}$ and area normal
    \item Units: T$\cdot$m$^2$ or weber (Wb)
    \item Flux measures the amount of magnetic field passing through an area
\end{itemize}
\alert{[Diagram illustrating magnetic flux through a loop]}
\end{frame}

\begin{frame}
\frametitle{Faraday's Law of Induction}
\begin{block}{Faraday's Law}
\begin{equation}
\text{emf} = -N\frac{\Delta\Phi}{\Delta t}
\end{equation}
\end{block}
\begin{itemize}
    \item emf = induced electromotive force (voltage)
    \item $N$ = number of turns in coil
    \item $\Delta\Phi/\Delta t$ = rate of change of magnetic flux
    \item Minus sign represents Lenz's law
    \item Electromagnetic induction: process of inducing emf through changing magnetic flux
\end{itemize}
\end{frame}

\begin{frame}
\frametitle{Lenz's Law}
\begin{block}{Lenz's Law}
The induced emf creates a current that produces a magnetic field that opposes the change in flux that produced it.
\end{block}
\begin{itemize}
    \item Conservation of energy principle
    \item Determines direction of induced current
    \item Represented by minus sign in Faraday's law
\end{itemize}
\alert{[Diagram showing Lenz's law - induced current creating opposing magnetic field]}
\end{frame}

\begin{frame}
\frametitle{Motional Emf}
\begin{block}{Motional Emf Formula}
\begin{equation}
\text{emf} = Blv \quad \text{(}B, l, \text{ and } v \text{ perpendicular)}
\end{equation}
\end{block}
\begin{itemize}
    \item $B$ = magnetic field strength (T)
    \item $l$ = length of conductor (m)
    \item $v$ = velocity (m/s)
    \item Special case of Faraday's law
    \item Induced when conductor moves through magnetic field
\end{itemize}
\alert{[Diagram of a conductor moving through a magnetic field]}
\end{frame}

\begin{frame}
\frametitle{Eddy Currents and Magnetic Damping}
\begin{itemize}
    \item \textbf{Eddy currents}: current loops induced in moving conductors
    \item Produced when conductors move through non-uniform magnetic fields
    \item \textbf{Magnetic damping}: drag force created by eddy currents
    \item Applications:
    \begin{itemize}
        \item Electromagnetic brakes
        \item Metal detectors
        \item Induction stoves
        \item Damping in moving-coil meters
    \end{itemize}
\end{itemize}
\alert{[Diagram showing eddy currents in a conductor moving through a magnetic field]}
\end{frame}

\section{Applications}

\begin{frame}
\frametitle{Electric Generators}
\begin{block}{Induced EMF in a Generator}
\begin{equation}
\text{emf} = NAB\omega\sin\omega t
\end{equation}
\end{block}
\begin{itemize}
    \item $N$ = number of turns
    \item $A$ = area of coil (m$^2$)
    \item $B$ = magnetic field strength (T)
    \item $\omega$ = angular velocity (rad/s)
    \item Peak emf: $\text{emf}_0 = NAB\omega$
    \item Converts mechanical energy to electrical energy
\end{itemize}
\alert{[Diagram of a simple AC generator]}
\end{frame}

\begin{frame}
\frametitle{Back EMF in Motors}
\begin{itemize}
    \item Motors are generators running in reverse
    \item Rotating coil in motor induces its own emf
    \item \textbf{Back emf}: induced emf that opposes the applied voltage
    \item Limits current through motor
    \item Proportional to motor's rotation speed
    \item Low back emf during startup → high current draw
\end{itemize}
\alert{[Diagram illustrating back emf in a motor]}
\end{frame}

\begin{frame}
\frametitle{Transformers}
\begin{block}{Transformer Equations}
\begin{align}
\frac{V_s}{V_p} &= \frac{N_s}{N_p} \\
\frac{I_s}{I_p} &= \frac{N_p}{N_s}
\end{align}
\end{block}
\begin{itemize}
    \item Use induction to change voltage levels
    \item $V_p$, $V_s$ = primary and secondary voltages
    \item $N_p$, $N_s$ = primary and secondary turns
    \item $I_p$, $I_s$ = primary and secondary currents
    \item Step-up: $N_s > N_p$, increases voltage
    \item Step-down: $N_s < N_p$, decreases voltage
    \item Power is conserved: $V_pI_p = V_sI_s$
\end{itemize}
\alert{[Diagram of transformer showing primary and secondary coils]}
\end{frame}

\begin{frame}
\frametitle{Electrical Safety Systems}
\begin{columns}
\column{0.5\textwidth}
\textbf{Three-Wire System:}
\begin{itemize}
    \item Live/hot wire
    \item Neutral wire
    \item Ground wire
    \item Grounds appliance case
\end{itemize}

\textbf{Circuit Breakers/Fuses:}
\begin{itemize}
    \item Interrupt excessive currents
    \item Prevent thermal hazards
\end{itemize}
\column{0.5\textwidth}
\textbf{GFI (Ground Fault Interrupter):}
\begin{itemize}
    \item Detects current imbalance
    \item Protects against shock
    \item Uses induction principles
\end{itemize}

\textbf{Isolation Transformer:}
\begin{itemize}
    \item Insulates device from source
    \item Prevents ground loops
\end{itemize}
\end{columns}
\end{frame}

\section{Inductance and AC Circuits}

\begin{frame}
\frametitle{Inductance}
\begin{block}{Self-Inductance}
\begin{equation}
\text{emf} = -L\frac{\Delta I}{\Delta t}
\end{equation}
\end{block}
\begin{itemize}
    \item \textbf{Inductance}: property describing how effectively a device induces emf
    \item $L$ = self-inductance (H, henry)
    \item $\Delta I/\Delta t$ = rate of current change (A/s)
    \item 1 H = 1 $\Omega\cdot$s
    \item Self-inductance of a solenoid: $L = \frac{\mu_0N^2A}{l}$
    \item Energy stored in inductor: $E_{ind} = \frac{1}{2}LI^2$
\end{itemize}
\end{frame}

\begin{frame}
\frametitle{RL Circuits}
\begin{columns}
\column{0.6\textwidth}
\textbf{Current when turning on:}
\begin{equation}
I = I_0(1 - e^{-t/\tau})
\end{equation}

\textbf{Current when turning off:}
\begin{equation}
I = I_0e^{-t/\tau}
\end{equation}

\textbf{Time constant:} $\tau = \frac{L}{R}$
\column{0.4\textwidth}
\begin{itemize}
    \item $I_0 = \frac{V}{R}$ is final current
    \item Current rises to $0.632I_0$ in first time constant
    \item Current falls to $0.368I_0$ in first time constant
\end{itemize}
\end{columns}
\alert{[Graph showing current vs. time for RL circuit turning on and off]}
\end{frame}

\begin{frame}
\frametitle{Reactance in AC Circuits}
\begin{columns}
\column{0.5\textwidth}
\textbf{Inductive Reactance:}
\begin{equation}
X_L = 2\pi fL
\end{equation}
\begin{itemize}
    \item In inductors, voltage leads current by 90°
    \item Reactance increases with frequency
\end{itemize}
\column{0.5\textwidth}
\textbf{Capacitive Reactance:}
\begin{equation}
X_C = \frac{1}{2\pi fC}
\end{equation}
\begin{itemize}
    \item In capacitors, voltage lags current by 90°
    \item Reactance decreases with frequency
\end{itemize}
\end{columns}
\alert{[Phasor diagrams showing phase relationships between voltage and current]}
\end{frame}

\begin{frame}
\frametitle{RLC Series Circuits}
\begin{block}{Impedance in RLC Circuit}
\begin{equation}
Z = \sqrt{R^2 + (X_L - X_C)^2}
\end{equation}
\end{block}
\begin{itemize}
    \item \textbf{Impedance}: AC equivalent of resistance
    \item \textbf{Resonant frequency}: $f_0 = \frac{1}{2\pi\sqrt{LC}}$
    \item At resonance: $X_L = X_C$ and $Z = R$
    \item \textbf{Phase angle}: $\cos\phi = \frac{R}{Z}$
    \item \textbf{Average power}: $P_{ave} = I_{rms}V_{rms}\cos\phi$
    \item \textbf{Power factor}: $\cos\phi$ (ranges from 0 to 1)
\end{itemize}
\alert{[Graph showing impedance vs. frequency with resonance peak]}
\end{frame}

\section{Example Problems}

\begin{frame}
\frametitle{I Do: Force on a Moving Charge}
\begin{block}{Example Problem}
An electron (charge $q = -1.6 \times 10^{-19}$ C) is moving at $v = 2.0 \times 10^7$ m/s perpendicular to a magnetic field of $B = 0.5$ T. Calculate the magnetic force on the electron.
\end{block}

\begin{align}
F &= |q|vB\sin\theta \\
&= (1.6 \times 10^{-19} \text{ C})(2.0 \times 10^7 \text{ m/s})(0.5 \text{ T})(\sin 90°) \\
&= (1.6 \times 10^{-19})(2.0 \times 10^7)(0.5)(1) \\
&= 1.6 \times 10^{-12} \text{ N}
\end{align}

Direction: Use RHR-1, but reverse direction since electron is negatively charged.
\end{frame}

\begin{frame}
\frametitle{We Do: Circular Motion in Magnetic Field}
\begin{block}{Example Problem}
A proton (mass $m = 1.67 \times 10^{-27}$ kg, charge $q = 1.6 \times 10^{-19}$ C) is moving at $v = 3.0 \times 10^6$ m/s perpendicular to a magnetic field of $B = 0.2$ T. Calculate the radius of its circular path.
\end{block}

\begin{align}
r &= \frac{mv}{qB} \\
&= \frac{(1.67 \times 10^{-27} \text{ kg})(3.0 \times 10^6 \text{ m/s})}{(1.6 \times 10^{-19} \text{ C})(0.2 \text{ T})} \\
&= ?
\end{align}

Let's work through this together...
\end{frame}

\begin{frame}
\frametitle{You Do: Faraday's Law Application}
\begin{block}{Example Problem}
A 200-turn circular coil with area $0.25 \text{ m}^2$ is in a magnetic field that changes from 0.5 T to 0.8 T perpendicular to the coil over a time of 0.1 s. Calculate the magnitude of the induced emf.
\end{block}

\begin{align}
\text{emf} &= -N\frac{\Delta\Phi}{\Delta t} \\
&= -N\frac{\Delta(BA\cos\theta)}{\Delta t} \\
&= ? \text{ V}
\end{align}

Try solving this problem yourself!
\end{frame}

\section{Summary}

\begin{frame}
\frametitle{Key Equations}
\begin{columns}
\column{0.5\textwidth}
\textbf{Magnetic Forces:}
\begin{align}
F &= qvB\sin\theta \\
F &= IlB\sin\theta \\
\tau &= NIAB\sin\theta
\end{align}

\textbf{Magnetic Fields:}
\begin{align}
B &= \frac{\mu_0I}{2\pi r} \text{ (straight wire)} \\
B &= \frac{\mu_0I}{2R} \text{ (loop center)} \\
B &= \mu_0nI \text{ (solenoid)}
\end{align}
\column{0.5\textwidth}
\textbf{Electromagnetic Induction:}
\begin{align}
\Phi &= BA\cos\theta \\
\text{emf} &= -N\frac{\Delta\Phi}{\Delta t} \\
\text{emf} &= Blv \text{ (motional)}
\end{align}

\textbf{AC Circuits:}
\begin{align}
X_L &= 2\pi fL \\
X_C &= \frac{1}{2\pi fC} \\
Z &= \sqrt{R^2 + (X_L - X_C)^2}
\end{align}
\end{columns}
\end{frame}

\begin{frame}
\frametitle{Key Concepts}
\begin{itemize}
    \item Magnetic fields exert forces on moving charges and current-carrying conductors
    \item The direction of magnetic forces and fields can be determined using right hand rules
    \item Changing magnetic fields induce emfs (electromagnetic induction)
    \item Lenz's law: induced effects oppose the change that caused them
    \item Applications include generators, motors, and transformers
    \item Inductors oppose changes in current
    \item In AC circuits, impedance combines resistance and reactance
    \item At resonance, inductive and capacitive reactances cancel
\end{itemize}
\end{frame}

\begin{frame}
\frametitle{Thank You!}
\begin{center}
\Large{Questions?}

\vspace{1cm}
\normalsize{Next class: Electromagnetic Waves}
\end{center}
\end{frame}

\end{document}