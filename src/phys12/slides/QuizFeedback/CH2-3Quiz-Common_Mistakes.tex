\documentclass[12pt]{article}
\usepackage{amsmath}
\usepackage{amssymb}
\usepackage{geometry}
\usepackage{fancyhdr}

\geometry{margin=1in}
\pagestyle{fancy}
\fancyhf{}
\rhead{Quiz 2-3 Solution Guide}
\lhead{Physics 12}
\cfoot{\thepage}

\title{CH 2-3 Quiz: Solution Guide for Common Mistakes}
\author{Physics 12}
\date{}

\begin{document}

\maketitle

\section{Target Questions for Review}

These are the two questions that showed the lowest performance on Quiz 2-3. Review these solutions carefully to understand the correct approach.

\subsection{Essential Kinematic Equations}

\begin{align}
y &= y_0 + v_0 t + \frac{1}{2}at^2 \tag{4}\\
v &= v_0 + at \tag{1}\\
v^2 &= v_0^2 + 2a\Delta y \tag{3}
\end{align}

Remember: $a_y = -g = -9.80$ m/s² for vertical motion

\section{Problem 1: Hot-Air Balloon Coin Drop}

\textbf{Common Error:} Many students forgot that the balloon is rising, giving the coin an initial upward velocity.

\textbf{G - Given:} $y_0 = 300$ m, $y = 0$ m, $v_0 = 10.0$ m/s, $a = -9.80$ m/s$^2$

\textbf{U - Unknown:} Time $t$

\textbf{E - Equation:} $y = y_0 + v_0 t + \frac{1}{2}at^2$

\textbf{S - Substitute and Solve:}
$$0 = 300 + 10.0t + \frac{1}{2}(-9.80)t^2$$
$$4.90t^2 - 10.0t - 300 = 0$$
$$t = \frac{10.0 \pm \sqrt{(10.0)^2 - 4(4.90)(-300)}}{2(4.90)} = \boxed{8.91 \text{ s}}$$

\textbf{Key Insight:} The initial upward velocity means the coin travels higher before falling, increasing total time.

\section{Problem 2: Distance vs. Displacement}

\textbf{Common Error:} Confusing distance (total path length) with displacement (straight-line from start to finish).

\textbf{Given:} 3 blocks north + 1 block east, each block = 120 m

\textbf{Distance:} Sum of actual path traveled
$$\text{Distance} = (3 \times 120) + (1 \times 120) = 480 \text{ m}$$

\textbf{Displacement:} Straight-line from start to finish
$$\text{Displacement} = \sqrt{(360)^2 + (120)^2} = 380 \text{ m}$$

\textbf{Key Difference:} Distance = 480 m (actual path), Displacement = 380 m (net result)

\section{Quick Review Tips}

\begin{itemize}
\item \textbf{Sign conventions:} Upward velocity = positive, gravity = negative
\item \textbf{Initial conditions:} Don't forget initial velocity when objects are dropped from moving platforms
\item \textbf{Distance vs. Displacement:} Distance = total path, Displacement = net change in position
\item \textbf{Check units:} Always include units in final answers
\end{itemize}

\end{document}