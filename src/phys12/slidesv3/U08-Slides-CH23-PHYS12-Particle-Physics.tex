\documentclass{beamer}
\usepackage{../../../shared/templates/ds9_theme}
\usepackage[overridenote]{pdfpc}
\graphicspath{{../images/}{../../shared/images/}}

\title[Inside the Atom]{PHYS12 CH:23 The Building Blocks of Reality}
\subtitle{From Quarks to the Universe}
\author[Mr. Gullo]{Mr. Gullo}
\date[December 2025]{December 2025}

\begin{document}

\frame{\titlepage
\note{[THE HOOK] Today we go deeper than anyone thought possible\\\\
- Beyond atoms, beyond nuclei, to the fundamental building blocks\\\\
- We'll discover quarks, antimatter, and forces that shaped the universe\\\\
[THE WONDER] By end of class, you'll understand what happened in first trillionth of second after Big Bang}
}

\begin{frame}
\frametitle{Outline}
\tableofcontents
\note{Three big questions:\\\\
- What forces rule the universe?\\\\
- What are the smallest pieces of matter?\\\\
- Were all forces once one?}
\end{frame}

\section{23.1 The Four Fundamental Forces}

\begin{frame}
\frametitle{Learning Objectives}
\begin{block}{By the end of this section, you will be able to:}
\begin{itemize}
\item \textbf{23.1:} Define and distinguish the four fundamental forces \pause
\item \textbf{23.1:} Describe carrier particles and force transmission \pause
\item \textbf{23.1:} Explain how particle accelerators probe nature
\end{itemize}
\end{block}
\note{[P0] Three objectives today\\\\
[P1] First: master the four forces that govern EVERYTHING\\\\
[P2] Second: understand how forces travel through particles\\\\
[P3] Third: see how we smash atoms to learn truth}
\end{frame}

\begin{frame}
\frametitle{23.1 The Mystery: How Many Forces Exist?}
\begin{center}
{\Large How many forces exist in the universe?}
\end{center}

\pause
\begin{exampleblock}{The Civilian's View}
Friction, gravity, tension, normal force, magnetic force, electric force, spring force, air resistance...
\end{exampleblock}

\pause
\begin{alertblock}{The Physicist's Truth}
Four. Just four fundamental forces explain EVERYTHING.
\end{alertblock}
\note{[P0] Think about all the forces you know\\\\
[P1] [THE CONNECTION - Kinetic] Athletes: friction on court, tension in rope\\\\
[P2] [THE REVELATION] All of those are just four forces in disguise\\\\
[THE WONDER] Four forces create galaxies, atoms, and YOU}
\end{frame}

\begin{frame}
\frametitle{23.1 The Four Forces That Run Everything}
\begin{block}{The Universal Forces}
\begin{enumerate}
\item \textbf{Gravity} - weakest, infinite range
\item \textbf{Electromagnetic} - charges and magnets, infinite range
\item \textbf{Weak Nuclear} - radioactive decay, tiny range
\item \textbf{Strong Nuclear} - binds nucleus, tiny range
\end{enumerate}
\end{block}

\pause
\begin{exampleblock}{The Nail Paradox}
Earth's entire mass pulls nail down. Small magnet lifts it up.
\end{exampleblock}
\note{[P0] Let's meet the four forces\\\\
- Gravity: you know it well, but it's the WEAKEST\\\\
- EM: friction, normal force - it's everywhere\\\\
- Weak: causes beta decay\\\\
- Strong: holds protons together despite repulsion\\\\
[P1] [THE HUMILITY] Gravity feels strong because Earth is MASSIVE\\\\
[THE WONDER] EM force is 10-to-36 times stronger}
\end{frame}

\begin{frame}
\frametitle{23.1 Gravity: The Cosmic Sculptor}
\begin{columns}
\begin{column}{0.5\textwidth}
\begin{itemize}
\item Acts on all mass
\item Always attractive
\item Infinite range
\item Weakest force
\item Shapes galaxies
\end{itemize}
\end{column}
\begin{column}{0.5\textwidth}
\begin{figure}
\centering
\includegraphics[width=\linewidth,height=0.55\textheight,keepaspectratio]{phys11-particle-physics-fig23-1.jpg}
\caption*{Large Hadron Collider}
\end{figure}
\end{column}
\end{columns}
\note{[Fig 23.1: LHC tubes, staircases showing massive scale] "Teaching hint: Use staircases for scale reference - this isn't 'lab equipment' but engineering marvel. Ask: 'Why so large?' (Higher energy requires stronger magnets, larger radius). Connect to gravity discussion - we need massive machines to probe tiniest scales."\\\\
[THE CONNECTION - Harmonic] Gravity is conductor of cosmic orchestra\\\\
[SAY] It dictates planetary motion, galactic structure\\\\
[THE PARADOX] Yet toy magnet defeats Earth's entire pull\\\\
[ANSWER] Gravity weakest but acts on EVERYTHING}
\end{frame}

\begin{frame}
\frametitle{23.1 Electromagnetic: The Force of Everyday Life}
\begin{block}{Hidden in Plain Sight}
\begin{itemize}
\item Acts on charged particles
\item Attractive AND repulsive
\item Infinite range (inverse square law)
\item Responsible for chemistry, friction, normal force
\end{itemize}
\end{block}

\pause
\begin{exampleblock}{The Mental Model}
When you sit in chair: electrons in your atoms repel electrons in chair. That's the "normal force."
\end{exampleblock}
\note{[P0] The EM force is most familiar stranger\\\\
[THE CONNECTION - Digital] Gamers: every pixel on screen, every circuit in console - EM force\\\\
[P1] [THE REVELATION] Friction isn't real - it's EM repulsion between atoms\\\\
[THE WONDER] You've never actually TOUCHED anything}
\end{frame}

\begin{frame}
\frametitle{23.1 The Nuclear Paradox}
\begin{alertblock}{Civilian View vs. Reality}
\textbf{Civilian:} "Protons stuck together in nucleus by gravity."\\[0.3cm]
\textbf{Physicist:} "Gravity too weak. Protons REPEL electromagnetically. Something else must hold them."
\end{alertblock}

\pause
\begin{block}{The Strong Nuclear Force}
\begin{itemize}
\item Strongest force at short range ($<10^{-15}$ m)
\item Acts on protons AND neutrons
\item Overcomes EM repulsion
\item Drops to zero beyond nuclear diameter
\end{itemize}
\end{block}
\note{[P0] [THE CONFLICT] Think about nucleus\\\\
- Protons have same positive charge\\\\
- They should FLY apart\\\\
- Yet nuclei exist\\\\
[P1] [THE REVELATION] A new force - the STRONG force\\\\
[ANSWER] Strongest at tiny distances, binds nucleons together}
\end{frame}

\begin{frame}
\frametitle{23.1 The Weak Nuclear Force: The Decay Master}
\begin{columns}
\begin{column}{0.6\textwidth}
\begin{block}{Nature's Transformer}
\begin{itemize}
\item Causes beta decay
\item Range: $<10^{-18}$ m
\item Weaker than strong and EM
\item Stronger than gravity
\item Acts on quarks and leptons
\end{itemize}
\end{block}
\end{column}
\begin{column}{0.4\textwidth}
Beta decay:
$${}_{Z}^{A}X_{N} \rightarrow {}_{Z+1}^{A}Y_{N-1} + e^- + \bar{\nu}$$
\end{column}
\end{columns}

\pause
\begin{exampleblock}{The Name Game}
It's called "weak" but it's stronger than gravity. Scientists named it before measuring carefully!
\end{exampleblock}
\note{[P0] The weak force - responsible for radioactive decay\\\\
- Neutron turns into proton, electron, antineutrino\\\\
[THE HUMILITY] Scientists named it weak before knowing gravity's true weakness\\\\
[P1] Weak is 10-to-25 stronger than gravity\\\\
[ANSWER] Range smaller than strong force}
\end{frame}

\begin{frame}
\frametitle{23.1 The Universal Law: Force Comparison}
\begin{center}
\small
\begin{tabular}{|l|c|c|c|}
\hline
\textbf{Force} & \textbf{Relative Strength} & \textbf{Range} & \textbf{Acts On} \\
\hline
Strong & 1 & $10^{-15}$ m & Nucleons \\
\hline
EM & $10^{-2}$ & Infinite & Charged \\
\hline
Weak & $10^{-13}$ & $10^{-18}$ m & Quarks/Leptons \\
\hline
Gravity & $10^{-39}$ & Infinite & All mass \\
\hline
\end{tabular}
\end{center}

\pause
\vspace{0.3cm}
\begin{block}{Nature's Source Code}
Four forces. That's it. They explain stars, atoms, chemistry, galaxies, YOU.
\end{block}
\note{[P0] Compare the four forces\\\\
- Strong is reference point: strength equals 1\\\\
- EM: one percent as strong, infinite range\\\\
- Weak: trillion times weaker, tiniest range\\\\
- Gravity: incomprehensibly weak\\\\
[P1] [THE WONDER] These four forces wrote story of universe}
\end{frame}

\begin{frame}
\frametitle{23.1 The Mystery of Action at a Distance}
\begin{center}
{\Large How does one proton "know" another proton exists?}
\end{center}

\pause
\begin{exampleblock}{Einstein's Dilemma}
Action at distance troubled Einstein. Fields helped, but particle physicists needed more.
\end{exampleblock}

\pause
\begin{block}{Yukawa's Solution (1935)}
Forces transmitted by \textbf{carrier particles} - real particles that carry force between objects.
\end{block}
\note{[P0] Fundamental question: how do forces work?\\\\
[THE CONFLICT] Two protons separated by space - how do they repel?\\\\
[P1] [THE HUMILITY] This bothered Einstein for decades\\\\
[P2] [THE REVELATION] Hideki Yukawa: forces are particles in motion\\\\
[ANSWER] Carrier particles transmit forces}
\end{frame}

\begin{frame}
\frametitle{23.1 Carrier Particles: Force Messengers}
\begin{figure}
\centering
\includegraphics[width=0.7\textwidth,height=0.5\textheight,keepaspectratio]{phys11-particle-physics-fig23-6.jpg}
\caption*{Pion exchange between proton and neutron}
\end{figure}

\pause
\begin{block}{Yukawa's Pion}
Proton emits pion $\rightarrow$ neutron absorbs it $\rightarrow$ strong force transmitted. Particle identities switch!
\end{block}
\note{[Fig 23.6: Proton/neutron exchange virtual pion] "Teaching hint: Use this diagram to show students how carrier particles literally transform particle identities - the proton becomes neutron by losing pion, neutron becomes proton by gaining it. Ask: 'What stays constant? What changes?' (Baryon number constant, identity changes)"\\\\
[P0] [THE CONNECTION - Digital] Like data packets in network\\\\
- Particles send messages via carrier particles\\\\
[P1] [SAY] Proton shoots pion to neutron\\\\
- Pion carries strong force\\\\
[THE REVELATION] Proton becomes neutron, neutron becomes proton\\\\
[ANSWER] Pion discovered in cosmic rays, 1947}
\end{frame}

\begin{frame}
\frametitle{23.1 Virtual Particles and Feynman Diagrams}
\begin{columns}
\begin{column}{0.5\textwidth}
\begin{figure}
\centering
\includegraphics[width=\linewidth,height=0.5\textheight,keepaspectratio]{phys11-particle-physics-fig23-5.jpg}
\caption*{Virtual photon exchange}
\end{figure}
\end{column}
\begin{column}{0.5\textwidth}
\begin{itemize}
\item Carrier particle is \textbf{virtual}
\item Cannot be directly observed
\item Exists briefly via uncertainty
\item Transmits force
\end{itemize}
\end{column}
\end{columns}

\pause
\begin{exampleblock}{Reading a Feynman Diagram}
Time flows UP. Particles move, exchange virtual particle, trajectories change.
\end{exampleblock}
\note{[Fig 23.5: Two positive charges exchange photon, repel] "Teaching hint: Trace the particle paths upward with finger - show how photon emission kicks left particle left, photon absorption kicks right particle right. Ask: 'How can EXCHANGING something cause REPULSION?' (Momentum transfer works both ways)"\\\\
[P0] Carrier particles are virtual - disturbances in space\\\\
- Can't see them directly without disrupting force\\\\
[THE CONNECTION - Harmonic] Like sound waves you can't catch\\\\
[P1] Feynman diagrams: time up, space sideways\\\\
- Two protons exchange photon, repel\\\\
[ANSWER] Richard Feynman won Nobel Prize, 1965}
\end{frame}

\begin{frame}
\frametitle{23.1 The Four Carrier Particles}
\begin{block}{Force Carriers}
\begin{itemize}
\item \textbf{Photon} - EM force, massless
\item \textbf{Gluon} - Strong force, massless (8 types)
\item \textbf{W$^+$, W$^-$, Z$^0$ bosons} - Weak force, very massive
\item \textbf{Graviton} - Gravity, not yet found (predicted massless)
\end{itemize}
\end{block}

\pause
\begin{alertblock}{Mass and Range Connection}
Massless carriers $\rightarrow$ infinite range (photon, graviton)\\
Massive carriers $\rightarrow$ short range (W, Z bosons)
\end{alertblock}
\note{[P0] Each force has carrier\\\\
- Photon: you know it - light particle, EM carrier\\\\
- Gluon: strong force, glues quarks together\\\\
- W and Z bosons: heavy, weak force carriers\\\\
- Graviton: hypothesized, never observed\\\\
[P1] [THE REVELATION] Mass determines range\\\\
[ANSWER] Heavy carriers have short range}
\end{frame}

\begin{frame}
\frametitle{23.1 Searching for the Graviton}
\begin{figure}
\centering
\includegraphics[width=0.7\textwidth,height=0.5\textheight,keepaspectratio]{phys11-particle-physics-fig23-7.jpg}
\caption*{LIGO - Laser Interferometer Gravitational-Wave Observatory}
\end{figure}

\pause
\begin{block}{The Missing Carrier}
Expected: massless, chargeless, spin-2 particle traveling at speed of light
\end{block}
\note{[Fig 23.7: LIGO Hanford control room with computers and spectra] "Teaching hint: Point out this is NOT the detector itself but the control room analyzing data. Show students that modern physics requires massive computational power - detecting graviton isn't just 'looking', it's filtering noise from signals."\\\\
[P0] [THE WONDER] Three carriers found, one remains\\\\
- LIGO searches for gravitational waves\\\\
[THE CONNECTION - Digital] Wave-particle duality: find wave, find particle\\\\
[P1] Binary stars may emit gravitational waves\\\\
- LHC looks for missing energy in collisions\\\\
[ANSWER] Not yet discovered, but actively searched}
\end{frame}

\begin{frame}
\frametitle{23.1 Particle Accelerators: Creating Matter from Energy}
\begin{block}{The Universal Equation}
\begin{center}
\Large $\boxed{E = mc^2}$
\end{center}
Energy converts to matter
\end{block}

\pause
\begin{itemize}
\item Accelerate known particles to high energy
\item Collide them with targets or each other
\item Create new particles from energy
\item Probe smaller distances = higher energies needed
\end{itemize}

\pause
\begin{exampleblock}{The Particle Physicist's Favorite Indoor Sport}
"Smash things together and see what comes out."
\end{exampleblock}
\note{[P0] [THE REVELATION] To study tiny things, we need huge machines\\\\
- E equals m c squared - energy becomes mass\\\\
[P1] Higher energy probes smaller distances\\\\
[THE CONNECTION - Kinetic] Like crashing cars to see engine parts\\\\
[P2] [THE WONDER] We create particles that haven't existed since Big Bang}
\end{frame}

\begin{frame}
\frametitle{23.1 Van de Graaff and Cyclotron}
\begin{figure}
\centering
\includegraphics[width=0.8\textwidth,height=0.6\textheight,keepaspectratio]{phys11-particle-physics-fig23-8.jpg}
\caption*{Van de Graaff (left) and Cyclotron (right)}
\end{figure}

\pause
\textbf{Van de Graaff:} Linear acceleration, up to 50 MV\\
\textbf{Cyclotron:} Spiral path, fixed frequency, higher energies
\note{[Fig 23.8: Van de Graaff photo and cyclotron diagram with alternating E-field] "Teaching hint: Use cyclotron diagram to show how alternating E-field must sync with particle rotation - if timing off, particle decelerates. Ask: 'What limits cyclotron energy?' (Relativistic effects change frequency at high speeds)"\\\\
[P0] [THE CONNECTION - All] You've seen Van de Graaff demos - hair standing up\\\\
- Same principle accelerates ions for experiments\\\\
[P1] Cyclotron: magnetic field bends path into spiral\\\\
- Voltage gap accelerates each loop\\\\
[ANSWER] Cyclotron achieves higher energies than Van de Graaff}
\end{frame}

\begin{frame}
\frametitle{23.1 Synchrotron: The Modern Workhorse}
\begin{figure}
\centering
\includegraphics[width=0.7\textwidth,height=0.5\textheight,keepaspectratio]{phys11-particle-physics-fig23-9.jpg}
\caption*{Synchrotron ring with accelerating tubes}
\end{figure}

\pause
\begin{itemize}
\item Particles travel fixed-radius ring
\item Magnetic field increases to keep radius constant
\item Voltage synchronized with particle speed
\item Very large for very high energies
\end{itemize}
\note{[Fig 23.9: Circular magnet ring with closeups showing E-field switching] "Teaching hint: Use the closeup panels to show HOW synchronization works - E-field reverses just as particle enters next tube. Ask: 'Why must field reverse?' (Particle must always see forward push, not backward)"\\\\
[P0] [SAY] Synchrotron is cyclotron's big brother\\\\
- Fixed ring, increasing magnetic field\\\\
[P1] LHC is synchrotron - 27 km circumference\\\\
[THE WONDER] Protons loop 11,000 times per second\\\\
[ANSWER] Size needed because high energy needs strong magnetic field}
\end{frame}

\begin{frame}
\frametitle{23.1 Colliding Beams: Maximum Energy}
\begin{figure}
\centering
\includegraphics[width=0.65\textwidth,height=0.45\textheight,keepaspectratio]{phys11-particle-physics-fig23-10.jpg}
\caption*{Fermilab's proton-antiproton collider}
\end{figure}

\pause
\begin{block}{Why Collide Head-On?}
Stationary target: much energy lost to recoil\\
Colliding beams: particles created with near-zero momentum
\end{block}
\note{[Fig 23.10: Circulating rings with proton/antiproton sources, collision detector] "Teaching hint: Trace particle paths with different colors - blue protons clockwise, red antiprotons counterclockwise. Ask: 'Why use antimatter?' (Opposite charge allows same magnets to bend both beams into collision course)"\\\\
[P0] [THE PARADOX] Hitting stationary target wastes energy\\\\
- Target recoils, carries away kinetic energy\\\\
[P1] [THE REVELATION] Head-on collisions: equal and opposite momentum\\\\
- Net momentum near zero\\\\
- All energy available for particle creation\\\\
[ANSWER] More massive particles can be created}
\end{frame}

\section{23.2 Quarks}

\begin{frame}
\frametitle{Learning Objectives}
\begin{block}{By the end of this section, you will be able to:}
\begin{itemize}
\item \textbf{23.2:} Describe quarks and their relationship to other particles \pause
\item \textbf{23.2:} Distinguish hadrons from leptons \pause
\item \textbf{23.2:} Distinguish matter from antimatter \pause
\item \textbf{23.2:} Describe the Standard Model \pause
\item \textbf{23.2:} Define Higgs boson and its importance
\end{itemize}
\end{block}
\note{[P0] Five big goals this section\\\\
[P1] Quarks - building blocks of protons\\\\
[P2] Hadrons vs leptons - two particle families\\\\
[P3] Antimatter - the mirror universe\\\\
[P4] Standard Model - periodic table of particles\\\\
[P5] Higgs boson - the mass giver}
\end{frame}

\begin{frame}
\frametitle{23.2 The Ancient Quest}
\begin{exampleblock}{Democritus, 460 BC}
"The first principles of universe are atoms and empty space. Everything else is merely thought to exist."
\end{exampleblock}

\pause
\begin{center}
{\Large The search for fundamental particles is nothing new.}
\end{center}

\pause
\begin{itemize}
\item 1930s: proton, neutron, electron discovered
\item Scientists thought: "We found smallest pieces!"
\item They were only partially correct...
\end{itemize}
\note{[P0] [THE CONNECTION - All] Humans always wondered: what is EVERYTHING made of?\\\\
- Ancient Greeks proposed atoms\\\\
[P1] Same question drives us today\\\\
[P2] [THE HUMILITY] Scientists in 1930s thought they finished search\\\\
[ANSWER] Proton and neutron have substructure - quarks}
\end{frame}

\begin{frame}
\frametitle{23.2 The Discovery That Shattered the Proton}
\textbf{1932:} Scientists thought they had it all - protons, neutrons, electrons

\pause
\vspace{0.3cm}

\textbf{1967:} SLAC experiment scatters high-energy electrons from protons

\pause
\vspace{0.3cm}

\begin{alertblock}{The Revelation}
Results showed three point-like charges \textit{inside} proton!
\end{alertblock}

\pause
Protons are NOT fundamental - they have substructure
\note{[P0] 1932: scientists thought they found fundamental particles\\\\
[P1] 1967: SLAC scattered 20 GeV electrons from protons\\\\
[P2] [THE REVELATION] Results showed three point-like charges inside proton\\\\
[P3] Protons are NOT fundamental\\\\
[THE CONNECTION] Like Rutherford's gold foil - scattering reveals hidden structure\\\\
[THE WONDER] We had to break the unbreakable}
\end{frame}

\begin{frame}
\frametitle{23.2 Electron Scattering Evidence}
\begin{figure}
\centering
\includegraphics[width=0.7\textwidth,height=0.5\textheight,keepaspectratio]{phys11-particle-physics-fig23-12.jpg}
\caption*{SLAC scattering experiment}
\end{figure}

\pause
Three point-like charges consistent with quark model
\note{[Fig 23.12: Large blue proton with three colored quarks, electrons scattering] "Teaching hint: Show how electrons passing through reveal nothing (proton looks 'smooth'), but electrons scattering at angles reveal three point charges. Compare to Rutherford's gold foil - same logic, smaller scale. Ask: 'Why use electrons not protons to probe?' (Electrons are point particles themselves)"\\\\
[P0] Electron scattering - like shining flashlight into proton\\\\
[P1] High-energy electrons have short wavelength - can see small details\\\\
- Three distinct scattering centers\\\\
- Each smaller than 10-to-minus-15 m\\\\
[THE REVELATION] This proved quarks exist inside protons}
\end{frame}

\begin{frame}
\frametitle{23.2 The Six Quark Flavors}
\begin{block}{The Quark Family}
\begin{center}
\small
\begin{tabular}{lll}
\textbf{Quark} & \textbf{Symbol} & \textbf{Charge} \\ \hline
Up & u & +$\frac{2}{3}$e \\
Down & d & -$\frac{1}{3}$e \\
Charm & c & +$\frac{2}{3}$e \\
Strange & s & -$\frac{1}{3}$e \\
Top & t & +$\frac{2}{3}$e \\
Bottom & b & -$\frac{1}{3}$e
\end{tabular}
\end{center}
\end{block}

\pause
\vspace{0.3cm}

\begin{alertblock}{The Illusion}
\textbf{Expected:} Charge is discrete (multiples of e)\\
\textbf{Reality:} Quarks have fractional charge!
\end{alertblock}
\note{[P0] Six quark flavors - discovered between 1967 and 1995\\\\
[P1] [THE CONFLICT] Charge thought to be discrete - multiples of elementary charge\\\\
- Quarks violate this - fractional charge\\\\
- But quarks never exist alone\\\\
- Combined quarks always give integer charge\\\\
[THE WONDER] Nature hides fractional charge}
\end{frame}

\begin{frame}
\frametitle{23.2 Color Charge: The Hidden Property}
\textbf{Quarks have three colors:} Red, Green, Blue

\pause
\vspace{0.3cm}

\textbf{Antiquarks have anticolors:} Cyan (anti-red), Magenta (anti-green), Yellow (anti-blue)

\pause
\vspace{0.3cm}

\begin{block}{The Universal Rule}
\begin{center}
All hadrons must have colors that sum to \textbf{white}
\end{center}
\end{block}

\pause
\textbf{Example:} Proton = red up + green up + blue down = white
\note{[P0] Quarks have color charge - not actual color, but property like electric charge\\\\
[P1] Three colors: red, green, blue\\\\
[P2] [THE REVELATION] All hadrons must have colors that sum to white\\\\
[P3] Proton: red up, green up, blue down - adds to white\\\\
[THE CONNECTION] Like color theory - red plus green plus blue equals white light\\\\
[THE WONDER] This rule called quantum chromodynamics}
\end{frame}

\begin{frame}
\frametitle{23.2 Gluon Exchange Between Quarks}
\begin{figure}
\centering
\includegraphics[width=0.7\textwidth,height=0.5\textheight,keepaspectratio]{phys11-particle-physics-fig23-13.jpg}
\caption*{Gluon changes quark color}
\end{figure}

\pause
Gluon carries strong force AND changes quark color

\pause
Quark flavor does NOT change, only color
\note{[Fig 23.13: Down and strange quarks exchange red-antigreen gluon] "Teaching hint: Emphasize gluon is labeled 'red-antigreen' - it carries color charge itself (unlike photons which are electrically neutral). Ask: 'What does gluon do to each quark?' (Gives red to strange, takes red from down, leaves anti-green). This makes gluons self-interact!"\\\\
[P0] Gluon exchange between quarks - carrier of strong force\\\\
[P1] Gluon carries force AND changes color\\\\
- Down quark changes from red to green\\\\
- Strange quark changes from green to red\\\\
[P2] Flavor does NOT change - still down and strange\\\\
[THE REVELATION] Strong force is color force}
\end{frame}

\begin{frame}
\frametitle{23.2 Building a Proton}
\begin{figure}
\centering
\includegraphics[width=0.7\textwidth,height=0.5\textheight,keepaspectratio]{phys11-particle-physics-fig23-14.jpg}
\caption*{Proton structure: uud}
\end{figure}

\pause
Proton = two up quarks + one down quark

\pause
Charge: $\frac{2}{3} + \frac{2}{3} - \frac{1}{3} = +1$ \checkmark

\pause
Color: red + green + blue = white \checkmark
\note{[Fig 23.14: Four hadrons with quark compositions and charge sums] "Teaching hint: Use this to show proton AND neutron side-by-side - highlight how swapping one quark changes particle. Then show pions (quark-antiquark pairs) to contrast baryons vs mesons. Walk through charge arithmetic for each. Ask: 'Why do pion colors cancel?' (Color + anticolor = white)"\\\\
[P0] Proton structure revealed\\\\
[P1] Two up quarks plus one down quark\\\\
[P2] Charge: two-thirds plus two-thirds minus one-third equals plus one\\\\
[P3] Color: red plus green plus blue equals white\\\\
[THE WONDER] Three fractional charges make one whole}
\end{frame}

\begin{frame}
\frametitle{23.2 Hadrons and Leptons}
\begin{columns}[T]
\column{0.48\textwidth}
\textbf{Hadrons}
\begin{itemize}
\item Feel strong force
\item Composed of quarks
\item Baryons: 3 quarks
\item Mesons: quark-antiquark
\item Examples: proton, neutron, pion
\end{itemize}

\pause
\column{0.48\textwidth}
\textbf{Leptons}
\begin{itemize}
\item Do NOT feel strong force
\item Fundamental particles
\item No substructure
\item Examples: electron, muon, neutrino
\end{itemize}
\end{columns}

\pause
\vspace{0.3cm}
\begin{exampleblock}{The Mental Model}
Hadrons are composite. Leptons are fundamental.
\end{exampleblock}
\note{[P0] Two major categories of matter particles\\\\
[P1] Hadrons feel strong force, composed of quarks\\\\
- Baryons have three quarks like proton\\\\
- Mesons have quark-antiquark pair\\\\
[P2] Leptons do NOT feel strong force - they are fundamental\\\\
- Electron is lepton\\\\
- Six leptons total\\\\
[P3] Hadrons composite, leptons fundamental}
\end{frame}

\begin{frame}
\frametitle{23.2 The Discovery of Antimatter}
\textbf{1932:} Carl Anderson discovers \textbf{positron} in cosmic rays

\pause
\vspace{0.3cm}

\begin{figure}
\centering
\includegraphics[width=0.6\textwidth,height=0.4\textheight,keepaspectratio]{phys11-particle-physics-fig23-16.jpg}
\caption*{Positron and electron tracks curve opposite directions}
\end{figure}

\pause
Same mass as electron, opposite charge = antielectron
\note{[Fig 23.16: Incident photon creates upward positron spiral, downward electron spiral] "Teaching hint: Show how magnetic field curves paths into spirals (not straight). Ask: 'Why opposite spirals?' (Opposite charges curve opposite directions). Then ask: 'Why spirals not circles?' (Particles lose energy via ionization). Connect to pair production energy threshold."\\\\
[P0] Carl Anderson 1932 - cloud chamber experiment\\\\
[P1] Magnetic field curves particles - some curve like negative charge, others like positive\\\\
- But positive curve had mass of electron, not proton\\\\
[P2] [THE REVELATION] Same mass as electron, opposite charge - the positron\\\\
[THE WONDER] For every particle, there exists antiparticle}
\end{frame}

\begin{frame}
\frametitle{23.2 Pair Production and Annihilation}
\begin{block}{Pair Production}
\begin{center}
Photon $\rightarrow$ electron + positron
\end{center}
Energy converts to matter
\end{block}

\pause
\vspace{0.3cm}

\begin{block}{Annihilation}
\begin{center}
electron + positron $\rightarrow$ photons
\end{center}
Matter converts to energy
\end{block}

\pause
\vspace{0.3cm}

Both mass-energy and charge conserved!
\note{[P0] Pair production - photon converts to electron and positron\\\\
- Energy becomes matter\\\\
[P1] Annihilation - electron and positron meet, convert to photons\\\\
- Matter becomes energy\\\\
[P2] Both mass-energy and charge conserved\\\\
- Photon has zero charge, electron plus positron has zero total charge\\\\
[THE WONDER] E equals m c squared in action}
\end{frame}

\begin{frame}
\frametitle{23.2 Why Antimatter Is Rare}
\begin{alertblock}{The Paradox}
If matter and antimatter created equally in Big Bang, where is all antimatter?
\end{alertblock}

\pause
\vspace{0.3cm}

When matter meets antimatter: \textbf{instant annihilation}

\pause
\vspace{0.3cm}

\textbf{Evidence:} Tiny excess of matter over antimatter in early universe

\pause
We are made of leftover matter!
\note{[P0] [THE CONFLICT] If Big Bang created equal matter and antimatter, where is antimatter?\\\\
[P1] When they meet: instant annihilation - both destroyed, converted to photons\\\\
[P2] Evidence shows tiny excess of matter over antimatter\\\\
[P3] [THE WONDER] We exist because of one-billionth excess of matter\\\\
[THE HUMILITY] We don't fully understand why this asymmetry exists}
\end{frame}

\begin{frame}
\frametitle{23.2 The Standard Model of Fundamental Particles}
\begin{figure}
\centering
\includegraphics[width=0.9\textwidth,height=0.7\textheight,keepaspectratio]{phys11-particle-physics-fig23-17.jpg}
\caption*{The Standard Model}
\end{figure}
\note{[Fig 23.17: 17 boxes - purple quarks, green leptons, red gauge bosons, yellow Higgs] "Teaching hint: Color-code categories - have students identify patterns (3 families, charge patterns within families). Ask: 'Why three families?' (We don't know - one of physics' open questions). Point out Higgs off to side because it's different category (scalar boson)."\\\\
[P0] Standard Model - organization of all fundamental particles\\\\
- Six quarks in two rows\\\\
- Six leptons in two rows\\\\
- Four gauge bosons (carrier particles)\\\\
- Higgs boson\\\\
[THE REVELATION] Seventeen fundamental particles explain everything\\\\
[THE WONDER] All matter in universe made from these}
\end{frame}

\begin{frame}
\frametitle{23.2 Reading the Standard Model}
\textbf{Three families of matter:}

\pause
\begin{enumerate}
\item \textbf{Family 1:} Normal matter (up, down, electron, neutrino) \pause
\item \textbf{Family 2:} More massive, less stable (charm, strange, muon) \pause
\item \textbf{Family 3:} Most massive, least stable (top, bottom, tau)
\end{enumerate}

\pause
\vspace{0.3cm}

\textbf{Pattern:} Mass increases left to right

\textbf{Trend:} Higher mass = less stable = faster decay
\note{[P0] Three families of matter\\\\
[P1] Family 1: normal matter - up, down, electron, electron neutrino\\\\
- This is what you're made of\\\\
[P2] Family 2: more massive - charm, strange, muon, muon neutrino\\\\
[P3] Family 3: most massive - top, bottom, tau, tau neutrino\\\\
[P4] Pattern: mass increases left to right\\\\
[THE WONDER] Why three families? We don't fully know}
\end{frame}

\begin{frame}
\frametitle{23.2 The Higgs Boson: The Mass Giver}
\textbf{The problem:} Why do W and Z bosons have mass, but photons and gluons don't?

\pause
\vspace{0.3cm}

\textbf{Peter Higgs (1960s):} All particles pass through \textbf{Higgs field}

\pause
\vspace{0.3cm}

\begin{exampleblock}{The Mental Model}
Higgs field is like water. Some particles swim through easily (photon), others slowed down (W, Z bosons).
\end{exampleblock}

\pause
The slowing creates mass!
\note{[P0] Why do some carrier particles have mass and others don't?\\\\
[P1] Peter Higgs 1960s: all particles pass through Higgs field\\\\
[P2] [THE CONNECTION] Like swimming through water - some particles pass easily, others slowed down\\\\
[P3] [THE REVELATION] The slowing transfers energy from motion to mass\\\\
- Photons and gluons unaffected\\\\
- W and Z bosons slowed significantly\\\\
[THE WONDER] The field gives particles mass}
\end{frame}

\begin{frame}
\frametitle{23.2 Discovering the Higgs Boson}
\textbf{July 4, 2012:} LHC announces discovery

\pause
\vspace{0.3cm}

\textbf{Method:} Proton-proton collisions at 7-8 TeV

\pause
\vspace{0.3cm}

\textbf{Evidence:} Particle with predicted mass, spin, and interactions

\pause
\vspace{0.3cm}

\textbf{March 13, 2013:} CERN confirms Higgs boson

\pause
\textbf{October 2013:} Peter Higgs wins Nobel Prize
\note{[P0] July 4, 2012 - historic announcement\\\\
[P1] Proton-proton collisions at 7-to-8 TeV - trillions of collisions\\\\
[P2] Found particle with predicted mass 125 GeV, spin zero, correct interactions\\\\
[P3] March 13, 2013 - CERN confirms Higgs boson\\\\
[P4] October 2013 - Peter Higgs and Francois Englert win Nobel Prize\\\\
[THE WONDER] Predicted in 1964, found 48 years later}
\end{frame}

\section{23.3 The Unification of Forces}

\begin{frame}
\frametitle{Learning Objectives}
\begin{block}{By the end of this section, you will be able to:}
\begin{itemize}
\item \textbf{23.3:} Define Grand Unified Theory \pause
\item \textbf{23.3:} Explain evolution of four forces from Big Bang \pause
\item \textbf{23.3:} Explain how unification theories can be tested
\end{itemize}
\end{block}
\note{[P0] Three objectives for force unification\\\\
[P1] First: Grand Unified Theory - unifying forces\\\\
[P2] Second: evolution of forces from Big Bang\\\\
[P3] Third: testing theories beyond our direct reach\\\\
- This connects particle physics to cosmology}
\end{frame}

\begin{frame}
\frametitle{23.3 The Dream of Unification}
\textbf{History of unification:}

\pause
\begin{itemize}
\item 1800s: Electric and magnetic forces unified $\rightarrow$ \textbf{Electromagnetic} \pause
\item 1960s: EM and weak nuclear unified $\rightarrow$ \textbf{Electroweak} \pause
\item Future: All four forces unified $\rightarrow$ \textbf{Theory of Everything}
\end{itemize}

\pause
\vspace{0.3cm}

\begin{block}{The Pattern}
At higher energies, forces become more similar
\end{block}
\note{[P0] History shows forces can be unified\\\\
[P1] 1800s: electricity and magnetism shown to be one force\\\\
[P2] 1960s: electromagnetic and weak nuclear unified\\\\
[P3] Future goal: unify all four forces\\\\
[P4] [THE REVELATION] At higher energies, forces become more similar\\\\
[THE WONDER] Four forces may be one superforce}
\end{frame}

\begin{frame}
\frametitle{23.3 Force Strength Versus Energy}
\begin{figure}
\centering
\includegraphics[width=0.8\textwidth,height=0.6\textheight,keepaspectratio]{phys11-particle-physics-fig23-20.jpg}
\caption*{Force strengths converge at high energy}
\end{figure}

\pause
At low energies: forces very different

At high energies: forces become similar!
\note{[Fig 23.20: Force strengths vs energy - lines converge at GUT, TOE] "Teaching hint: Trace each line with finger showing trends (gravity up, strong down, EM/weak converge at electroweak). Ask: 'What does convergence mean physically?' (Forces become indistinguishable at extreme energies). Point out GUT at 10^15 GeV and TOE at 10^19 GeV - emphasize scale beyond any accelerator."\\\\
[P0] Graph shows force strength versus energy\\\\
[P1] At low energies - everyday life - forces very different\\\\
- Gravity extremely weak\\\\
- Strong force very strong\\\\
- At high energies: forces converge\\\\
- At 100 GeV: EM and weak forces equal\\\\
- At 10-to-14 GeV: strong and electroweak equal\\\\
[THE WONDER] This suggests common origin}
\end{frame}

\begin{frame}
\frametitle{23.3 Electroweak Unification}
\textbf{Weinberg, Glashow, Salam (1960s):} EM and weak forces identical at high energies

\pause
\vspace{0.3cm}

\textbf{Prediction:} Three carrier particles: W$^+$, W$^-$, Z$^0$

Predicted masses: W = 81 GeV/c$^2$, Z = 90 GeV/c$^2$

\pause
\vspace{0.3cm}

\textbf{1983:} All three particles discovered at CERN with exact predicted masses!
\note{[P0] Weinberg, Glashow, Salam - electroweak theory\\\\
[P1] Predicted three particles with specific masses\\\\
- W-plus and W-minus: 81 GeV\\\\
- Z-zero: 90 GeV\\\\
[P2] [THE REVELATION] 1983: all three found at CERN with exact predicted masses\\\\
[THE WONDER] Theory predicted particles years before discovery}
\end{frame}

\begin{frame}
\frametitle{23.3 Grand Unified Theory (GUT)}
\textbf{Goal:} Unify strong, weak, and electromagnetic forces

\pause
\vspace{0.3cm}

\textbf{Energy required:} $10^{14}$ GeV

\pause
\vspace{0.3cm}

\begin{alertblock}{The Challenge}
\textbf{LHC maximum:} 14 TeV = $1.4 \times 10^4$ GeV\\
\textbf{GUT energy:} $10^{14}$ GeV\\
We're $10^{10}$ times too low!
\end{alertblock}

\pause
Cannot test directly with accelerators
\note{[P0] Grand Unified Theory - unify strong, weak, and EM forces\\\\
[P1] Energy required: 10-to-14 GeV\\\\
[P2] [THE CONFLICT] LHC maximum 14 TeV - we're 10 billion times too low\\\\
[P3] Cannot test directly\\\\
[THE HUMILITY] Beyond reach of any conceivable human-built accelerator\\\\
[THE WONDER] But we can test indirect consequences}
\end{frame}

\begin{frame}
\frametitle{23.3 Testing GUT Indirectly: Proton Decay}
\textbf{GUT prediction:} Protons should decay

Lifetime: $10^{31}$ years

\pause
\vspace{0.3cm}

\textbf{Test:} Super-Kamiokande in Japan - 50,000 tons of water

\pause
\vspace{0.3cm}

\textbf{Strategy:} If one proton decays in $10^{31}$ years, then $10^{31}$ protons will have one decay per year

\pause
\vspace{0.3cm}

\textbf{Result (2014):} No decay observed - proton lifetime $> 5.9 \times 10^{33}$ years
\note{[P0] Some GUTs predict proton decay - lifetime 10-to-31 years\\\\
[P1] Super-Kamiokande - 50,000 ton water tank in Japan\\\\
[P2] Clever use of proportional reasoning - if one proton decays in 10-to-31 years, then 10-to-31 protons will have one decay per year\\\\
[P3] Result: no decay found - proton lifetime greater than 5.9 times 10-to-33 years\\\\
[THE REVELATION] This rules out some GUTs but not all}
\end{frame}

\begin{frame}
\frametitle{23.3 The Big Bang and Force Evolution}
\begin{figure}
\centering
\includegraphics[width=0.9\textwidth,height=0.7\textheight,keepaspectratio]{phys11-particle-physics-fig23-22.jpg}
\caption*{Universe evolution from Big Bang}
\end{figure}
\note{[Fig 23.22: Timeline 10^-43s to 15x10^10 years showing eras and particles] "Teaching hint: Start at left (Planck time) and walk right through epochs. Emphasize logarithmic time scale - first microsecond has more 'epochs' than next billion years. Ask: 'Why does cooling universe cause force separation?' (Symmetry breaking - like water freezing breaks rotational symmetry)."\\\\
[P0] Universe evolution tied to particle physics\\\\
- Timeline from Big Bang forward\\\\
- As universe expands, energy decreases\\\\
- Forces separate one by one\\\\
[THE WONDER] Particle physics explains first seconds of universe}
\end{frame}

\begin{frame}
\frametitle{23.3 The First Trillionth of a Second}
\textbf{Planck Epoch} ($0 \to 10^{-43}$ s): All four forces unified as \textbf{superforce}

\pause
\vspace{0.3cm}

\textbf{Grand Unification Epoch} ($10^{-43} \to 10^{-36}$ s): Gravity separates

\pause
\vspace{0.3cm}

\textbf{Inflationary Epoch} ($10^{-36} \to 10^{-32}$ s): Strong force separates, universe inflates by $10^{50}$!

\pause
\vspace{0.3cm}

\textbf{Electroweak Epoch} ($10^{-32} \to 10^{-11}$ s): Strong force separated

\pause
\textbf{Quark Era} ($10^{-11} \to 10^{-6}$ s): All four forces separated, quarks form
\note{[P0] Planck Epoch - all four forces unified\\\\
[P1] Grand Unification Epoch - gravity separates\\\\
[P2] Inflationary Epoch - strong force separates, universe inflates by 10-to-50\\\\
- Expansion faster than light - space itself expanding\\\\
[P3] Electroweak Epoch - strong force separated\\\\
[P4] Quark Era - all four forces separated, quarks begin to form\\\\
[THE WONDER] We're reverse-engineering first trillionth of second}
\end{frame}

\begin{frame}
\frametitle{23.3 The Universe as Our Laboratory}
\begin{block}{The Connection}
\begin{center}
Particle accelerators recreate Big Bang conditions\\[0.2cm]
Cosmology tests particle physics theories\\[0.2cm]
The smallest and largest scales are connected
\end{center}
\end{block}

\pause
\vspace{0.3cm}

\begin{exampleblock}{The Cosmic Connection}
Understanding quarks helps us understand first seconds after Big Bang. Understanding Big Bang helps us understand quarks.
\end{exampleblock}
\note{[P0] [THE REVELATION] Particle accelerators recreate Big Bang conditions\\\\
[P1] [THE WONDER] The smallest and largest scales connected\\\\
- To understand universe, study particles\\\\
- To understand particles, study universe\\\\
[THE CONNECTION - Digital Archetype] Like debugging code - look at initial conditions to understand current state}
\end{frame}

\section{Summary}

\begin{frame}
\frametitle{What You Now Know}
\begin{block}{The Revelations}
\begin{enumerate}
\item Four fundamental forces govern all interactions \pause
\item Forces transmitted by carrier particles \pause
\item Protons made of three quarks with fractional charge \pause
\item Six quarks, six leptons - seventeen fundamental particles \pause
\item Antimatter exists and annihilates with matter \pause
\item Higgs field gives particles mass \pause
\item Forces unified at high energies \pause
\item Particle physics explains Big Bang evolution
\end{enumerate}
\end{block}
\note{[P0] Eight revelations today\\\\
[P1] Four fundamental forces\\\\
[P2] Forces transmitted by particles\\\\
[P3] Protons made of quarks\\\\
[P4] Seventeen fundamental particles\\\\
[P5] Antimatter exists\\\\
[P6] Higgs field gives mass\\\\
[P7] Forces unified at high energies\\\\
[P8] Particle physics explains universe\\\\
[THE WONDER] You now understand deepest level of reality we know}
\end{frame}

\begin{frame}
\frametitle{Key Concepts}
\textbf{Four Forces:} Gravity, EM, Weak nuclear, Strong nuclear

\vspace{0.2cm}

\textbf{Carrier Particles:} Graviton*, Photon, W/Z bosons, Gluon

\vspace{0.2cm}

\textbf{Quarks:} Six flavors, three colors, fractional charge

\vspace{0.2cm}

\textbf{Hadrons:} Baryons (3 quarks), Mesons (quark-antiquark)

\vspace{0.2cm}

\textbf{Leptons:} Fundamental particles (electron, muon, tau, neutrinos)

\vspace{0.2cm}

\textbf{Standard Model:} 6 quarks + 6 leptons + 4 carriers + Higgs = 17

\vspace{0.2cm}

\textbf{Unification:} Forces become similar at high energies
\note{Review key concepts\\\\
- Four forces with carrier particles\\\\
- Quarks build hadrons\\\\
- Leptons are fundamental\\\\
- Standard Model organizes everything\\\\
- Forces unify at high energies\\\\
- Questions before homework?}
\end{frame}

\begin{frame}
\frametitle{Homework}
\begin{center}
\Large
Complete the assigned problems\\[0.3cm]
posted on the LMS
\end{center}
\note{[SAY] Homework posted on LMS\\\\
[TIMING] Due date: check LMS\\\\
[CHECK] Questions before we end?\\\\
[THE WONDER] You now know what happened in first trillionth of second after Big Bang - that's remarkable\\\\
[TRANSITION] Next unit: we'll connect this to modern physics applications}
\end{frame}

\end{document}
