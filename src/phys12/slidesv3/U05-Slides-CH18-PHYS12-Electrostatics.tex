\documentclass{beamer}
\usepackage{../../../shared/templates/ds9_theme}
\usepackage{../../../shared/templates/semantic-physics-colors}
\usepackage[overridenote]{pdfpc}
\graphicspath{{../images/}{../../shared/images/}}

\title[Electrostatics]{PHYS12 CH18: The Invisible Forces Between Charges}
\subtitle{Coulomb's Law, Fields, Potential, and Capacitors}
\author[Mr. Gullo]{Mr. Gullo}
\date[December 2025]{December 2025}

\begin{document}

\frame{\titlepage
\note{[THE HOOK] Today we discover the invisible forces that run your phone, lightning, and your heart.\\\\
- From static shocks to touchscreens to defibrillators\\\\
- Four revelations: forces between charges, invisible fields, electrical pressure, energy storage\\\\
[THE WONDER] This is the force that holds atoms together and powers civilization.}
}

\begin{frame}
\frametitle{Outline}
\tableofcontents
\end{frame}

\section{Introduction}

\begin{frame}
\frametitle{The Mystery of the Invisible}
\begin{center}
\Large What if forces could act\\
\textit{without touching?}
\end{center}

\pause
\vspace{0.5cm}
From lightning bolts to the neurons in your brain...

\pause
\vspace{0.3cm}
\alert{Electric forces shape reality.}
\note{[P0] "What if forces could act without touching?"\\\\
[P1] "From lightning bolts to the neurons in your brain..."\\\\
[P2] [THE WONDER] "Electric forces shape reality. Every thought is electrical signals. Every heartbeat is triggered by electric pulses."\\\\
[THE CONNECTION - Digital Archetype] Your phone's touchscreen works because of electric forces}
\end{frame}

\section{Coulomb's Law}

\begin{frame}
\frametitle{Learning Objectives}
\begin{block}{By the end of this section, you will be able to:}
\begin{itemize}
\item \textbf{18.2:} Describe Coulomb's law verbally and mathematically \pause
\item \textbf{18.2:} Solve problems involving Coulomb's law
\end{itemize}
\end{block}
\note{[P0] "Two objectives for Coulomb's law"\\\\
[P1] "First: understand the mathematical relationship. Second: solve real problems"\\\\
- This is Newton's law for electric forces\\\\
- Assessment: problems on next assessment}
\end{frame}

\begin{frame}
\frametitle{18.2 The Force Between Charges}
\begin{exampleblock}{The Mental Model}
Like gravity pulls masses together, electric force acts between charges. But there's a twist: charges can attract OR repel.
\end{exampleblock}

\pause
\vspace{0.3cm}

\textbf{Two types of charge:}
\begin{itemize}
\item Positive (+) and Negative (-) \pause
\item Like charges repel \pause
\item Unlike charges attract
\end{itemize}
\note{[P0] "Like gravity pulls masses, electric force acts between charges - but with a twist"\\\\
[P1] "Two types: positive and negative"\\\\
[P2] "Like charges repel - two positives push apart"\\\\
[P3] "Unlike charges attract - positive and negative pull together"\\\\
[THE CONNECTION - Kinetic Archetype] Like magnets - same poles repel, opposite attract}
\end{frame}

\begin{frame}
\frametitle{18.2 Coulomb's Discovery}
\begin{figure}
\centering
\includegraphics[width=0.7\textwidth,height=0.5\textheight,keepaspectratio]{phys12-electrostatics-fig01.jpg}
\end{figure}

\pause
Charles-Augustin de Coulomb (1780s) used a torsion balance to measure forces between charged spheres.
\note{[Fig: Coulomb's torsion balance] "Ingenious device that measured invisible forces"\\\\
[P0] "Coulomb's torsion balance - ingenious device"\\\\
[P1] "Measured tiny forces by twisting a wire. Brought spheres closer together and measured how much twist increased"\\\\
- Discovered inverse-square law\\\\
- Same pattern as Newton's gravity\\\\
[TEACHING HINT] Emphasize experimental precision - how do you measure something invisible? Show connection to modern force sensors}
\end{frame}

\begin{frame}
\frametitle{18.2 The Source Code of Electric Force}
\begin{block}{Universal Law: Coulomb's Law}
\begin{center}
\Large $\boxed{\force{F} = \frac{\coulombk{k}\charge{q_1}\charge{q_2}}{\disp{r}^2}}$
\end{center}
\force{Force} between \charge{charges} equals Coulomb's constant times the product of \charge{charges} divided by \disp{distance} squared.
\end{block}

\pause
\vspace{0.3cm}

Where: $\coulombk{k} = 8.99 \times 10^9$ N$\cdot$m$^2$/C$^2$
\note{[P0] [THE REVELATION] "F equals k q-1 q-2 over r-squared"\\\\
- Force proportional to each charge\\\\
- Force inversely proportional to distance squared\\\\
[P1] "k is Coulomb's constant: 8.99 times 10 to the 9"\\\\
[THE WONDER] Same pattern as gravity, but k is HUGE compared to G - electric forces are incredibly strong}
\end{frame}

\begin{frame}
\frametitle{18.2 Reading the Signs}
\begin{alertblock}{The Paradox}
If both \charge{charges} are positive OR both negative: $\force{F} > 0$ (repulsive)\\
If \charge{charges} have opposite signs: $\force{F} < 0$ (attractive)
\end{alertblock}

\pause
\vspace{0.3cm}

\begin{figure}
\centering
\includegraphics[width=0.7\textwidth,height=0.4\textheight,keepaspectratio]{phys12-electrostatics-fig02.jpg}
\end{figure}
\note{[Fig 18.16: Force vectors between like and opposite charges] "Visual proof of attraction vs repulsion"\\\\
[P0] "The math tells you the direction"\\\\
- Positive result means repulsive\\\\
- Negative result means attractive\\\\
[P1] "Diagram shows both cases"\\\\
[THE CONFLICT] Math gives you sign, but you have to interpret it\\\\
- Always state whether attractive or repulsive in your answer\\\\
[TEACHING HINT] Use diagram to anchor sign convention - students struggle with negative meaning attractive. Connect to everyday magnets}
\end{frame}

\begin{frame}
\frametitle{18.2 Inverse-Square Law}
\textbf{What happens when you change the distance?}

\pause
\begin{itemize}
\item Double the \disp{distance} $(\disp{r} \to 2\disp{r})$: \force{Force} becomes $\frac{1}{4}$ as strong \pause
\item Triple the \disp{distance} $(\disp{r} \to 3\disp{r})$: \force{Force} becomes $\frac{1}{9}$ as strong \pause
\item Halve the \disp{distance} $(\disp{r} \to \frac{\disp{r}}{2})$: \force{Force} becomes $4$ times stronger
\end{itemize}

\pause
\vspace{0.3cm}

\begin{exampleblock}{The Mental Model}
Bring \charge{charges} twice as close: \force{force} quadruples. Move them twice as far: \force{force} drops to one-fourth.
\end{exampleblock}
\note{[P0] "What happens when you change the distance?"\\\\
[P1] "Double distance: force becomes one-fourth"\\\\
[P2] "Triple distance: force becomes one-ninth"\\\\
[P3] "Halve distance: force becomes 4 times stronger"\\\\
[P4] [THE CONNECTION - Kinetic Archetype] "Like sound getting quieter as you move away - but squared"\\\\
[THE WONDER] This is why atoms are so tightly bound - forces get huge at tiny distances}
\end{frame}

\begin{frame}
\frametitle{18.2 Coulomb vs Gravity}
\textbf{Similarities:}
\begin{itemize}
\item Both are inverse-square laws \pause
\item Both act at a distance
\end{itemize}

\pause
\vspace{0.3cm}

\textbf{Differences:}
\begin{itemize}
\item $\coulombk{k}$ is MUCH larger than $G$ (electric \force{forces} are stronger) \pause
\item Gravity only attracts; electric forces attract OR repel
\end{itemize}

\pause
\vspace{0.3cm}

\begin{alertblock}{Civilian vs Reality}
\textbf{Civilian:} "Gravity holds everything together."\\
\textbf{Physicist:} "Electric forces hold atoms and molecules together. Gravity holds planets and galaxies."
\end{alertblock}
\note{[P0] "Compare to Newton's gravity"\\\\
[P1] "Both inverse-square - same mathematical pattern"\\\\
[P2] "Both act at distance - no contact needed"\\\\
[P3] "k is much larger than G - electric forces dominate at small scales"\\\\
[P4] "Gravity only attracts. Electric can attract or repel"\\\\
[P5] [THE CONFLICT] "Civilians think gravity holds everything. Reality: electric forces hold you together, gravity holds you to Earth"}
\end{frame}

\begin{frame}
\frametitle{Attempt: Decoding Electric Force}
\begin{exampleblock}{The Challenge (3 min, silent)}
Two \charge{charges} $\charge{q_1} = +3 \times 10^{-9}$ C and $\charge{q_2} = -4 \times 10^{-9}$ C are separated by 3.0 cm.

\vspace{0.3cm}

\textbf{Given:}
\begin{itemize}
\item $\charge{q_1} = +3 \times 10^{-9}$ C
\item $\charge{q_2} = -4 \times 10^{-9}$ C
\item $\disp{r} = 3.0$ cm $= 0.030$ m
\end{itemize}

\textbf{Find:} Magnitude and direction of \force{force}

\vspace{0.3cm}

\textit{Can you calculate the force? Work silently.}
\end{exampleblock}
\note{[THE CHALLENGE] Can they use Coulomb's law?\\\\
[SAY] "Try this on your own. It's okay to get stuck."\\\\
[TIMING] 3-4 min SILENT individual work\\\\
[CIRCULATE] Note who converts units, who forgets absolute value\\\\
[WATCH FOR] Students confused by negative sign\\\\
[DON'T HELP] Let them struggle with units and signs}
\end{frame}

\begin{frame}
\frametitle{Compare: Force Calculation}
\textbf{Turn and talk (2 min):}

\vspace{0.3cm}

\begin{enumerate}
\item What did you substitute for $\charge{q_1}$ and $\charge{q_2}$?
\item Did you convert cm to m?
\item Is the force attractive or repulsive? How do you know?
\end{enumerate}

\vspace{0.5cm}

\pause
\alert{Name wheel:} One pair share your approach (not your answer).
\note{[TIMING] 2-3 min pair discussion\\\\
[CIRCULATE] Listen for common approaches\\\\
[CHECK] Name wheel: call a pair to share\\\\
[EXPECTED APPROACH] Plug in values, convert cm to m, calculate, then determine direction from signs\\\\
[COMMON ERROR] Forgetting to convert cm to m, confused about attractive vs repulsive}
\end{frame}

\begin{frame}
\frametitle{Reveal: The Electric Interaction}
\textbf{Self-correct in a different color:}

\vspace{0.3cm}

\textbf{Step 1:} $\force{F} = \frac{\coulombk{k}\charge{q_1}\charge{q_2}}{\disp{r}^2}$

\pause
\vspace{0.2cm}

\textbf{Step 2:} $\force{F} = \frac{(8.99 \times 10^9)(3 \times 10^{-9})(-4 \times 10^{-9})}{(0.030)^2}$

\pause
\vspace{0.2cm}

\textbf{Step 3:} $\force{F} = -1.2 \times 10^{-4}$ N

\pause
\vspace{0.2cm}

$$\boxed{|\force{F}| = 1.2 \times 10^{-4} \text{ N, attractive}}$$

\pause
\textbf{Check:} Opposite charges attract. Negative force confirms this!
\note{[P0] "Self-correct in a different color"\\\\
[P1] [ALGEBRA] "F equals k q-1 q-2 over r-squared"\\\\
[P2] "Substitute values - note r must be in meters"\\\\
[P3] "Result is negative 1.2 times 10 to the negative 4 N"\\\\
[P4] [ANSWER] "Magnitude is 1.2 times 10 to the negative 4 N, attractive"\\\\
[THE WONDER] Tiny charges, huge force - electric forces dominate at atomic scales}
\end{frame}

\section{Electric Field}

\begin{frame}
\frametitle{Learning Objectives}
\begin{block}{By the end of this section, you will be able to:}
\begin{itemize}
\item \textbf{18.3:} Calculate the strength of an electric field \pause
\item \textbf{18.3:} Create and interpret drawings of electric fields
\end{itemize}
\end{block}
\note{[P0] "Two objectives for electric fields"\\\\
[P1] "First: calculate field strength. Second: visualize fields with diagrams"\\\\
- Fields are how physicists think about forces at a distance\\\\
- Essential for understanding circuits and electronics}
\end{frame}

\begin{frame}
\frametitle{18.3 Force Fields in Physics}
\begin{center}
\Large What if space itself could push on charges?
\end{center}

\pause
\vspace{0.5cm}

\begin{exampleblock}{The Mental Model}
An electric field is an invisible map showing which way a positive charge would be pushed at every point in space.
\end{exampleblock}

\pause
\vspace{0.3cm}

Not science fiction - this is how physicists think about forces at a distance.
\note{[P0] "What if space itself could push on charges?"\\\\
[P1] "Field is a map - shows force direction everywhere"\\\\
[P2] "Not sci-fi - real physics concept"\\\\
[THE CONNECTION - Digital Archetype] "Like a game engine's collision map - defines forces at every coordinate"\\\\
[THE REVELATION] Fields are real - they carry energy and momentum}
\end{frame}

\begin{frame}
\frametitle{18.3 The Source Code of Fields}
\begin{block}{Universal Law: Electric Field}
\begin{center}
\Large $\boxed{\vec{\efield{E}} = \frac{\vec{\force{F}}}{\charge{q}_{\text{test}}}}$
\end{center}
\efield{Electric field} equals \force{force} per unit \charge{charge}.
\end{block}

\pause
\vspace{0.3cm}

For a point \charge{charge} $\charge{Q}$:
$$\efield{E} = \frac{\coulombk{k}|\charge{Q}|}{\disp{r}^2}$$

\pause
Units: N/C (newtons per coulomb)
\note{[P0] [THE REVELATION] "E equals F over q-test"\\\\
- Force per unit charge\\\\
- Vector field - has magnitude and direction\\\\
[P1] "For point charge: E equals k Q over r-squared"\\\\
[P2] "Units: newtons per coulomb"\\\\
[THE WONDER] Field exists even without test charge - space is modified by charges}
\end{frame}

\begin{frame}
\frametitle{18.3 Visualizing the Invisible}
\begin{figure}
\centering
\includegraphics[width=0.7\textwidth,height=0.55\textheight,keepaspectratio]{phys12-electrostatics-fig03.jpg}
\end{figure}

\pause
\textbf{Field lines show:}
\begin{itemize}
\item Direction of \force{force} on positive \charge{charge}
\item Strength (closer lines = stronger \efield{field})
\end{itemize}
\note{[Fig 18.17: Radial field lines from positive point charge] "Visualizing the invisible - force map in all directions"\\\\
[P0] "Field lines from positive charge radiate outward"\\\\
- Lines point away from positive charges\\\\
- Lines point toward negative charges\\\\
[P1] "Direction shows force on positive test charge. Density shows strength"\\\\
[THE CONNECTION - Harmonic Archetype] "Like ripples from a pebble in water - spreading from source"\\\\
[TEACHING HINT] Trace one field line with finger - ask "what force would a tiny + charge feel here?" This makes abstract field concrete}
\end{frame}

\begin{frame}
\frametitle{18.3 Field Line Rules}
\begin{enumerate}
\item Lines point \textbf{away} from positive \charge{charges} \pause
\item Lines point \textbf{toward} negative \charge{charges} \pause
\item Lines NEVER cross each other \pause
\item Denser lines = stronger \efield{field}
\end{enumerate}

\pause
\vspace{0.3cm}

\begin{alertblock}{The Paradox}
\textbf{Misconception:} "\efield{Field} lines are paths \charge{charges} follow."\\
\textbf{Reality:} \efield{Field} lines show \force{force} direction, but moving \charge{charges} have inertia - they curve gradually.
\end{alertblock}
\note{[P0] "Four rules for field lines"\\\\
[P1] "Lines away from positive"\\\\
[P2] "Lines toward negative"\\\\
[P3] "Never cross - only one field direction at each point"\\\\
[P4] "Denser lines mean stronger field"\\\\
[P5] [THE CONFLICT] "Students think charges follow field lines. Reality: inertia makes them curve"\\\\
- Like throwing a ball - doesn't fall straight down}
\end{frame}

\begin{frame}
\frametitle{18.3 Field Patterns}
\begin{columns}[T]
\column{0.48\textwidth}
\begin{center}
\includegraphics[width=\linewidth,height=0.5\textheight,keepaspectratio]{phys12-electrostatics-fig04.jpg}

\small Positive and negative
\end{center}

\pause
\column{0.48\textwidth}
\begin{center}
\includegraphics[width=\linewidth,height=0.5\textheight,keepaspectratio]{phys12-electrostatics-fig05.jpg}

\small Two negatives
\end{center}
\end{columns}

\vspace{0.3cm}

Field lines connect opposite charges, repel from like charges.
\note{[Fig 18.19: Field patterns for opposite and like charge pairs] "Geometry reveals interaction type"\\\\
[P0] "Left: opposite charges - lines connect them"\\\\
[P1] "Right: like charges - lines repel each other"\\\\
- Far away, field looks like single charge\\\\
- Close in, you see individual contributions\\\\
[THE WONDER] Superposition - fields add like vectors\\\\
[TEACHING HINT] Compare left/right - why do lines connect vs diverge? This visual builds intuition before math. Ask: where is field strongest in each case?}
\end{frame}

\begin{frame}
\frametitle{Attempt: Reading Field Maps}
\begin{exampleblock}{The Challenge (2 min, silent)}
Look at this field map. Three charges create these field lines.

\vspace{0.3cm}

\textbf{Questions:}
\begin{enumerate}
\item Which \charge{charges} are positive? Which are negative?
\item Which \charge{charge} has the largest magnitude?
\item Where is the \efield{field} strongest?
\end{enumerate}

\vspace{0.3cm}

\textit{Use field line density and direction to decode the charges.}
\end{exampleblock}
\note{[THE CHALLENGE] Can they interpret field diagrams?\\\\
[SAY] "Use what you know about field lines"\\\\
[TIMING] 2 min individual analysis\\\\
[CIRCULATE] See who counts lines, who looks at density\\\\
[WATCH FOR] Confusion about field strength vs charge magnitude}
\end{frame}

\begin{frame}
\frametitle{Compare: Field Interpretation}
\textbf{Turn and talk (2 min):}

\vspace{0.3cm}

\begin{enumerate}
\item How did you identify positive vs negative \charge{charges}?
\item How did you compare \charge{charge} magnitudes?
\item Where is the \efield{field} strongest, and how do you know?
\end{enumerate}

\vspace{0.5cm}

\pause
\alert{Name wheel:} One pair share your reasoning.
\note{[TIMING] 2 min pair discussion\\\\
[CIRCULATE] Listen for reasoning strategies\\\\
[CHECK] Name wheel: call a pair\\\\
[EXPECTED APPROACH] Lines out = positive, lines in = negative. Count lines for magnitude. Density for strength\\\\
[COMMON ERROR] Confusing field strength with charge size}
\end{frame}

\begin{frame}
\frametitle{Reveal: Decoding the Field}
\textbf{Self-correct in a different color:}

\vspace{0.3cm}

\textbf{Signs:}
\begin{itemize}
\item Lines OUT = positive \charge{charge}
\item Lines IN = negative \charge{charge}
\end{itemize}

\pause

\textbf{Magnitude:}
\begin{itemize}
\item More lines = larger \charge{charge}
\item Count \efield{field} lines touching each \charge{charge}
\end{itemize}

\pause

\textbf{\efield{Field} Strength:}
\begin{itemize}
\item Closest lines = strongest \efield{field}
\item Usually near \charge{charges}
\end{itemize}
\note{[P0] "Signs from direction"\\\\
[P1] "Magnitude from counting lines - proportional to charge"\\\\
[P2] "Strength from line density - closer together means stronger"\\\\
[THE WONDER] "One diagram encodes force information for EVERY point in space"}
\end{frame}

\section{Electric Potential}

\begin{frame}
\frametitle{Learning Objectives}
\begin{block}{By the end of this section, you will be able to:}
\begin{itemize}
\item \textbf{18.4:} Explain similarities and differences between electric and gravitational potential energy \pause
\item \textbf{18.4:} Calculate electric potential difference
\end{itemize}
\end{block}
\note{[P0] "Two objectives for electric potential"\\\\
[P1] "First: connect to gravity analogy. Second: calculate voltage"\\\\
- Voltage is just another word for potential difference\\\\
- This connects force to energy}
\end{frame}

\begin{frame}
\frametitle{18.4 The Universe's Pressure Gauge}
\begin{exampleblock}{The Mental Model}
Gravitational potential: height in a gravitational field\\
Electric potential: "height" in an electric field
\end{exampleblock}

\pause
\vspace{0.3cm}

\begin{figure}
\centering
\includegraphics[width=0.6\textwidth,height=0.4\textheight,keepaspectratio]{phys12-electrostatics-fig06.jpg}
\end{figure}

\pause
Both store energy that can be released to do work.
\note{[Fig 18.21: Parallel comparison of gravitational and electric potential energy] "Bridge from familiar (gravity) to abstract (electric)"\\\\
[P0] "Analogy: gravity and electric potential"\\\\
[P1] "Left: ball high in gravity has potential. Right: positive charge far from negative has potential"\\\\
[P2] "Both can do work when released"\\\\
[THE CONNECTION - Kinetic Archetype] "Like lifting a weight - you store energy that can be released"\\\\
[THE REVELATION] Potential energy is position in a field\\\\
[TEACHING HINT] Point to left panel first - activate prior knowledge. Then reveal right panel - same concept, different force. Critical analogy for understanding voltage}
\end{frame}

\begin{frame}
\frametitle{18.4 Potential Energy of Two Charges}
\begin{block}{Universal Law: Electric Potential Energy}
\begin{center}
\Large $\boxed{\energy{U_E} = \frac{\coulombk{k}\charge{q_1}\charge{q_2}}{\disp{r}}}$
\end{center}
\energy{Energy} stored in configuration of two \charge{charges}.
\end{block}

\pause
\vspace{0.3cm}

\textbf{Sign tells the story:}
\begin{itemize}
\item $\energy{U_E} > 0$: like \charge{charges} (they want to fly apart) \pause
\item $\energy{U_E} < 0$: opposite \charge{charges} (they want to come together)
\end{itemize}
\note{[P0] [THE REVELATION] "U-E equals k q-1 q-2 over r"\\\\
- Similar to Coulomb's law but NO r-squared\\\\
- Energy, not force\\\\
[P1] "Sign matters"\\\\
[P2] "Positive: like charges - repel, can do work by separating. Negative: opposite charges - attract, can do work by approaching"\\\\
[THE WONDER] This energy holds molecules together}
\end{frame}

\begin{frame}
\frametitle{18.4 Electric Potential (Voltage)}
\begin{block}{Universal Law: Electric Potential}
\begin{center}
\Large $\boxed{\voltage{V} = \frac{\energy{U_E}}{\charge{q}} = \frac{\coulombk{k}\charge{q}}{\disp{r}}}$
\end{center}
\energy{Potential energy} per unit \charge{charge}. Units: volts (V)
\end{block}

\pause
\vspace{0.3cm}

\begin{alertblock}{Civilian vs Reality}
\textbf{Civilian:} "\voltage{Voltage} is electricity flowing."\\
\textbf{Physicist:} "\voltage{Voltage} is electric pressure - \energy{potential energy} per \charge{charge}."
\end{alertblock}
\note{[P0] [THE REVELATION] "V equals U-E over q, which equals k q over r"\\\\
- Potential per unit charge\\\\
- Units: joules per coulomb = volts\\\\
[P1] [THE CONFLICT] "Civilians confuse voltage with current. Voltage is pressure, current is flow"\\\\
- Like water: pressure vs flow rate\\\\
[THE CONNECTION - Harmonic Archetype] "Voltage is like air pressure in instrument - drives the flow"}
\end{frame}

\begin{frame}
\frametitle{18.4 Potential Difference}
\textbf{What really matters: difference in potential}

\pause
\vspace{0.3cm}

In uniform \efield{field} $\efield{E}$:
$$\Delta \voltage{V} = -\efield{E}(\disp{x_f} - \disp{x_i})$$

\pause

\textbf{Rearranged:}
$$\efield{E} = \frac{\Delta \voltage{V}}{\disp{d}}$$

\pause
\efield{Electric field} units: V/m (volts per meter)
\note{[P0] "Usually we care about potential DIFFERENCE"\\\\
[P1] "In uniform field: delta-V equals negative E times distance"\\\\
[P2] "Rearranged: E equals delta-V over d"\\\\
[P3] "This gives alternate units for field: volts per meter"\\\\
[THE WONDER] Divide voltage by distance, get field strength - incredibly useful}
\end{frame}

\begin{frame}
\frametitle{18.4 The 9V Battery}
\begin{exampleblock}{Real-World: Battery Voltage}
A 9V battery creates 9V \voltage{potential difference} between terminals.

\vspace{0.3cm}

This means: moving 1 coulomb from - to + terminal requires 9 joules of \work{work}.
\end{exampleblock}

\pause
\vspace{0.3cm}

Battery converts chemical \energy{energy} to electric \energy{potential energy}.
\note{[P0] "9V battery example"\\\\
- 9 volts means 9 joules per coulomb\\\\
- Battery does work separating charges\\\\
[P1] "Chemical reactions push charges against electric field"\\\\
[THE CONNECTION - Digital Archetype] "Like charging a capacitor in your phone - storing energy in electric field"\\\\
[THE WONDER] All batteries are charge pumps}
\end{frame}

\begin{frame}
\frametitle{Attempt: Calculating Voltage}
\begin{exampleblock}{The Challenge (3 min, silent)}
A point \charge{charge} $\charge{Q} = +5 \times 10^{-9}$ C creates an electric \voltage{potential}.

\vspace{0.3cm}

\textbf{Given:}
\begin{itemize}
\item $\charge{Q} = +5 \times 10^{-9}$ C
\item $\disp{r} = 0.10$ m
\end{itemize}

\textbf{Find:} Electric \voltage{potential} at distance $\disp{r} = 0.10$ m

\vspace{0.3cm}

\textit{Can you calculate the voltage? Work silently.}
\end{exampleblock}
\note{[THE CHALLENGE] Can they use voltage formula?\\\\
[SAY] "Use the formula for potential from a point charge"\\\\
[TIMING] 3 min individual work\\\\
[CIRCULATE] Watch for correct formula choice\\\\
[WATCH FOR] Confusion between potential and potential energy}
\end{frame}

\begin{frame}
\frametitle{Compare: Voltage Calculation}
\textbf{Turn and talk (2 min):}

\vspace{0.3cm}

\begin{enumerate}
\item What formula did you use?
\item What did you substitute for each variable?
\item What are the units of your answer?
\end{enumerate}

\vspace{0.5cm}

\pause
\alert{Name wheel:} One pair share your approach.
\note{[TIMING] 2 min pair discussion\\\\
[CIRCULATE] Listen for formula choice\\\\
[CHECK] Name wheel\\\\
[EXPECTED APPROACH] V equals k Q over r, plug in values\\\\
[COMMON ERROR] Using potential energy formula instead of potential}
\end{frame}

\begin{frame}
\frametitle{Reveal: The Electric Pressure}
\textbf{Self-correct in a different color:}

\vspace{0.3cm}

\textbf{Formula:} $\voltage{V} = \frac{\coulombk{k}\charge{Q}}{\disp{r}}$

\pause
\vspace{0.2cm}

\textbf{Substitute:} $\voltage{V} = \frac{(8.99 \times 10^9)(5 \times 10^{-9})}{0.10}$

\pause
\vspace{0.2cm}

$$\boxed{\voltage{V} = 450 \text{ V}}$$

\pause
\textbf{Check:} 450 volts - much higher than a battery, but safe at this tiny \charge{charge}!
\note{[P0] "V equals k Q over r"\\\\
[P1] [ALGEBRA] "Substitute: 8.99 times 10 to the 9, times 5 times 10 to the negative 9, divided by 0.10"\\\\
[P2] [ANSWER] "450 volts"\\\\
[P3] "High voltage but tiny charge means low energy - won't hurt you"\\\\
[THE WONDER] Static shocks are thousands of volts but harmless - it's current that's dangerous}
\end{frame}

\section{Capacitors}

\begin{frame}
\frametitle{Learning Objectives}
\begin{block}{By the end of this section, you will be able to:}
\begin{itemize}
\item \textbf{18.5:} Calculate energy stored in a capacitor and capacitance \pause
\item \textbf{18.5:} Explain properties of capacitors and dielectrics
\end{itemize}
\end{block}
\note{[P0] "Two objectives for capacitors"\\\\
[P1] "First: calculate capacitance and energy. Second: understand how they work"\\\\
- Capacitors are in EVERY electronic device\\\\
- Essential for circuits, power supplies, touchscreens}
\end{frame}

\begin{frame}
\frametitle{18.5 Energy Storage Devices}
\begin{center}
\Large What if you could bottle electric fields?
\end{center}

\pause
\vspace{0.5cm}

\begin{figure}
\centering
\includegraphics[width=0.6\textwidth,height=0.4\textheight,keepaspectratio]{phys12-electrostatics-fig07.jpg}
\end{figure}

\pause
\capac{Capacitors} store \energy{energy} in \efield{electric fields} between charged plates.
\note{[Fig 18.29: Photograph of seven real capacitors in various shapes/sizes] "From theory to hardware - what they actually look like"\\\\
[P0] "What if you could bottle electric fields?"\\\\
[P1] "Capacitors - various types shown"\\\\
[P2] "Store energy in field between plates"\\\\
[THE CONNECTION - Digital Archetype] "Like RAM in computer - temporary energy storage"\\\\
[THE REVELATION] Field carries energy - it's physical, not abstract\\\\
[TEACHING HINT] Hold up actual capacitor if available. Point to photo: "These tiny components are in your phone, headphones, every circuit." Makes abstract concept tangible}
\end{frame}

\begin{frame}
\frametitle{18.5 The Parallel-Plate Capacitor}
\begin{figure}
\centering
\includegraphics[width=0.6\textwidth,height=0.45\textheight,keepaspectratio]{phys12-electrostatics-fig08.jpg}
\end{figure}

\pause
\textbf{Design:}
\begin{itemize}
\item Two metal plates separated by small \disp{distance}
\item One plate \charge{charged} +, other \charge{charged} -
\item \efield{Electric field} between plates is uniform
\end{itemize}
\note{[Fig 18.28: Parallel-plate capacitor connected to battery showing uniform field] "Canonical geometry - foundation for all capacitor understanding"\\\\
[P0] "Parallel-plate capacitor - simplest design"\\\\
[P1] "Two plates: positive and negative. Field between is uniform - straight lines"\\\\
- Battery pumps charge onto plates\\\\
- Field stores energy\\\\
[THE WONDER] Energy literally exists in empty space between plates\\\\
[TEACHING HINT] Trace charge flow from battery to plates. Ask: "Where exactly is the energy?" Answer: in the field lines themselves. This diagram makes field energy visible}
\end{frame}

\begin{frame}
\frametitle{18.5 The Source Code of Capacitance}
\begin{block}{Universal Law: Capacitance}
\begin{center}
\Large $\boxed{\capac{C} = \frac{\charge{Q}}{\voltage{V}}}$
\end{center}
\capac{Capacitance} equals \charge{charge} stored per volt applied. Units: farads (F)
\end{block}

\pause
\vspace{0.3cm}

For parallel plates:
$$\capac{C_0} = \pConst{\varepsilon_0} \frac{\area{A}}{\disp{d}}$$

where $\pConst{\varepsilon_0} = 8.85 \times 10^{-12}$ F/m
\note{[P0] [THE REVELATION] "C equals Q over V"\\\\
- Charge per volt\\\\
- Units: farads (extremely large unit)\\\\
[P1] "For parallel plates: C equals epsilon-zero A over d"\\\\
- Bigger area = more capacitance\\\\
- Smaller separation = more capacitance\\\\
[THE WONDER] Capacitance depends only on geometry, not on charge or voltage}
\end{frame}

\begin{frame}
\frametitle{18.5 Energy Storage}
\begin{block}{Universal Law: Energy in Capacitor}
\begin{center}
\Large $\boxed{\energy{U_E} = \frac{1}{2}\capac{C}\voltage{V}^2}$
\end{center}
\energy{Energy} stored equals half \capac{capacitance} times \voltage{voltage} squared.
\end{block}

\pause
\vspace{0.3cm}

\begin{exampleblock}{The Mental Model}
Like kinetic \energy{energy} $\kenergy{K} = \frac{1}{2}\mass{m}\vel{v}^2$, but for \efield{electric fields} instead of motion.
\end{exampleblock}
\note{[P0] [THE REVELATION] "U-E equals one-half C V-squared"\\\\
- Energy proportional to voltage squared\\\\
- Doubling voltage quadruples stored energy\\\\
[P1] "Similar form to kinetic energy formula"\\\\
[THE CONNECTION - Kinetic Archetype] "Like compressing a spring - energy stored that can be released"\\\\
[THE WONDER] Defibrillators use capacitors - release stored energy in milliseconds}
\end{frame}

\begin{frame}
\frametitle{18.5 Capacitance Factors}
\textbf{What determines \capac{capacitance}?}

\pause
$$\capac{C_0} = \pConst{\varepsilon_0} \frac{\area{A}}{\disp{d}}$$

\pause
\begin{itemize}
\item Increase plate \area{area} $\area{A}$: \capac{capacitance} increases \pause
\item Increase separation $\disp{d}$: \capac{capacitance} decreases \pause
\item Geometry only - not \charge{charge} or \voltage{voltage}!
\end{itemize}

\pause
\vspace{0.3cm}

\begin{alertblock}{The Paradox}
\textbf{Misconception:} "More \charge{charge} means more \capac{capacitance}."\\
\textbf{Reality:} \capac{Capacitance} is constant for given geometry. More \charge{charge} just means higher \voltage{voltage}.
\end{alertblock}
\note{[P0] "What determines capacitance?"\\\\
[P1] "Formula: epsilon-zero A over d"\\\\
[P2] "Larger plates store more charge at same voltage"\\\\
[P3] "Closer plates create stronger field"\\\\
[P4] "Geometry only - intrinsic property"\\\\
[P5] [THE CONFLICT] "Students think more charge increases capacitance. Reality: C is constant, Q and V vary together"\\\\
- Like a bucket - size doesn't change when you fill it}
\end{frame}

\begin{frame}
\frametitle{18.5 Dielectrics}
\begin{exampleblock}{Real-World Enhancement}
Insert insulating material (dielectric) between plates:
\begin{itemize}
\item \capac{Capacitance} increases
\item Can store more \energy{energy}
\item Prevents electrical breakdown
\end{itemize}
\end{exampleblock}

\pause
\vspace{0.3cm}

\textbf{Common dielectrics:}
\begin{itemize}
\item Paper, plastic, ceramic, air
\end{itemize}
\note{[P0] "Dielectrics increase capacitance"\\\\
- Insulating material between plates\\\\
- Polarizes in electric field\\\\
- Reduces field strength, allows more charge storage\\\\
[P1] "Common materials: paper, plastic, ceramic"\\\\
[THE CONNECTION - Digital Archetype] "Phone capacitors use thin ceramic dielectrics - maximizes capacitance in tiny space"}
\end{frame}

\begin{frame}
\frametitle{Attempt: Capacitor Design}
\begin{exampleblock}{The Challenge (3 min, silent)}
Design a parallel-plate capacitor with \capac{capacitance} $\capac{C} = 1.0 \times 10^{-9}$ F.

\vspace{0.3cm}

\textbf{Given:}
\begin{itemize}
\item $\capac{C} = 1.0 \times 10^{-9}$ F (1.0 nF)
\item Plate \area{area} $\area{A} = 0.010$ m$^2$
\item $\pConst{\varepsilon_0} = 8.85 \times 10^{-12}$ F/m
\end{itemize}

\textbf{Find:} Required plate separation $\disp{d}$

\vspace{0.3cm}

\textit{Can you find the spacing? Work silently.}
\end{exampleblock}
\note{[THE CHALLENGE] Can they rearrange and solve?\\\\
[SAY] "Design problem - find what separation you need"\\\\
[TIMING] 3 min individual work\\\\
[CIRCULATE] Watch for algebra errors\\\\
[WATCH FOR] Unit confusion, rearrangement mistakes}
\end{frame}

\begin{frame}
\frametitle{Compare: Design Strategy}
\textbf{Turn and talk (2 min):}

\vspace{0.3cm}

\begin{enumerate}
\item What formula did you start with?
\item How did you rearrange to solve for $\disp{d}$?
\item What units did you get for $\disp{d}$?
\end{enumerate}

\vspace{0.5cm}

\pause
\alert{Name wheel:} One pair share your approach.
\note{[TIMING] 2 min pair discussion\\\\
[CIRCULATE] Listen for rearrangement strategies\\\\
[CHECK] Name wheel\\\\
[EXPECTED APPROACH] Start with C equals epsilon-zero A over d, rearrange to d equals epsilon-zero A over C\\\\
[COMMON ERROR] Inverting wrong terms, unit errors}
\end{frame}

\begin{frame}
\frametitle{Reveal: The Design Solution}
\textbf{Self-correct in a different color:}

\vspace{0.3cm}

\textbf{Formula:} $\capac{C_0} = \pConst{\varepsilon_0} \frac{\area{A}}{\disp{d}}$

\pause
\vspace{0.2cm}

\textbf{Rearrange:} $\disp{d} = \pConst{\varepsilon_0} \frac{\area{A}}{\capac{C}}$

\pause
\vspace{0.2cm}

\textbf{Substitute:} $\disp{d} = (8.85 \times 10^{-12}) \frac{0.010}{1.0 \times 10^{-9}}$

\pause
\vspace{0.2cm}

$$\boxed{\disp{d} = 8.85 \times 10^{-5} \text{ m} = 0.089 \text{ mm}}$$

\pause
\textbf{Check:} Less than a tenth of a millimeter - capacitors need very small spacing!
\note{[P0] "Start with capacitance formula"\\\\
[P1] [ALGEBRA] "Rearrange: d equals epsilon-zero A over C"\\\\
[P2] "Substitute values"\\\\
[P3] [ANSWER] "0.089 mm - less than thickness of paper"\\\\
[P4] "This is why capacitors are fragile - tiny spacing"\\\\
[THE WONDER] Modern capacitors have nanometer spacing - that's atomic-scale engineering}
\end{frame}

\section{Summary}

\begin{frame}
\frametitle{The Four Revelations}
\begin{block}{What You Now Know}
\begin{enumerate}
\item Coulomb's Law: $\force{F} = \frac{\coulombk{k}\charge{q_1}\charge{q_2}}{\disp{r}^2}$ - \force{forces} between \charge{charges} \pause
\item \efield{Electric Field}: $\efield{E} = \frac{\force{F}}{\charge{q}}$ - \force{force} per unit \charge{charge} \pause
\item Electric \voltage{Potential}: $\voltage{V} = \frac{\coulombk{k}\charge{q}}{\disp{r}}$ - \energy{energy} per unit \charge{charge} \pause
\item \capac{Capacitance}: $\capac{C} = \frac{\charge{Q}}{\voltage{V}}$ - \charge{charge} storage capacity
\end{enumerate}
\end{block}
\note{[P0] "Four revelations today"\\\\
[P1] "Coulomb's law: force between charges"\\\\
[P2] "Electric field: map of forces in space"\\\\
[P3] "Potential: electric pressure, voltage"\\\\
[P4] "Capacitance: ability to store charge and energy"\\\\
[THE WONDER] These four concepts explain everything from atoms to lightning to your phone}
\end{frame}

\begin{frame}[shrink]
\frametitle{Key Equations}
\begin{align}
\force{F} &= \frac{\coulombk{k}\charge{q_1}\charge{q_2}}{\disp{r}^2} \quad \text{(Coulomb's law)}\\
\efield{E} &= \frac{\coulombk{k}|\charge{Q}|}{\disp{r}^2} \quad \text{(Electric field from point charge)}\\
\vec{\efield{E}} &= \frac{\vec{\force{F}}}{\charge{q}} \quad \text{(Field definition)}\\
\energy{U_E} &= \frac{\coulombk{k}\charge{q_1}\charge{q_2}}{\disp{r}} \quad \text{(Electric potential energy)}\\
\voltage{V} &= \frac{\coulombk{k}\charge{q}}{\disp{r}} \quad \text{(Electric potential from point charge)}\\
\capac{C} &= \frac{\charge{Q}}{\voltage{V}} \quad \text{(Capacitance)}\\
\capac{C_0} &= \pConst{\varepsilon_0} \frac{\area{A}}{\disp{d}} \quad \text{(Parallel-plate capacitor)}\\
\energy{U_E} &= \frac{1}{2}\capac{C}\voltage{V}^2 \quad \text{(Energy in capacitor)}
\end{align}
\note{- Eight essential equations\\\\
- First three: forces and fields\\\\
- Next two: energy and potential\\\\
- Last three: capacitors and energy storage\\\\
- Memorize these - you'll use them constantly}
\end{frame}

\begin{frame}
\frametitle{Constants to Remember}
\begin{center}
\begin{tabular}{ll}
\textbf{Constant} & \textbf{Value} \\ \hline
Coulomb's constant $\coulombk{k}$ & $8.99 \times 10^9$ N$\cdot$m$^2$/C$^2$ \\
Elementary \charge{charge} $\charge{e}$ & $1.602 \times 10^{-19}$ C \\
Permittivity $\pConst{\varepsilon_0}$ & $8.85 \times 10^{-12}$ F/m
\end{tabular}
\end{center}

\vspace{0.5cm}

These appear in every calculation!
\note{- Three fundamental constants\\\\
- k for Coulomb's law and fields\\\\
- e for elementary charge\\\\
- epsilon-zero for capacitance\\\\
- Keep these on formula sheet}
\end{frame}

\begin{frame}
\frametitle{From Theory to Reality}
\textbf{You now understand:}
\begin{itemize}
\item Why static shocks happen (\charge{charge} transfer) \pause
\item How touchscreens work (\capac{capacitance} sensing) \pause
\item Why lightning forms (\efield{electric field} breakdown) \pause
\item How defibrillators work (\capac{capacitor} discharge) \pause
\item Why atoms bond (electric \force{forces}) \pause
\item How computer memory works (\capac{capacitor} \charge{charge} storage)
\end{itemize}
\note{[P0] "Real-world applications"\\\\
[P1] "Static shocks: charge buildup and discharge"\\\\
[P2] "Touchscreens: finger changes capacitance"\\\\
[P3] "Lightning: field becomes too strong, air conducts"\\\\
[P4] "Defibrillators: rapid capacitor discharge through heart"\\\\
[P5] "Atoms: electric forces hold electrons to nucleus"\\\\
[P6] "Computer memory: billions of tiny capacitors store bits"\\\\
[THE WONDER] Electric forces shape everything from atoms to civilization}
\end{frame}

\begin{frame}
\frametitle{Homework}
\begin{center}
\Large
Complete the assigned problems\\[0.3cm]
posted on the LMS
\end{center}
\note{[SAY] "Homework posted on LMS"\\\\
[TIMING] Due date: check LMS\\\\
[CHECK] Questions before we end?\\\\
- Practice Coulomb's law, fields, potential, and capacitors\\\\
[TRANSITION] Next: Electric Current and Circuits}
\end{frame}

\end{document}
