\documentclass{beamer}
\usepackage{../../../shared/templates/ds9_theme}
\usepackage{../../../shared/templates/semantic-physics-colors}
\usepackage[overridenote]{pdfpc}
\graphicspath{{../images/}{../../shared/images/}}

\title[Falling Forever]{PHYS12 CH:6 The Art of Falling Forever}
\subtitle{Circular Motion and Rotation}
\author[Mr. Gullo]{Mr. Gullo}
\date[December 2025]{December 2025}

\begin{document}

\frame{\titlepage
\note{[THE HOOK] Today we decode the secret of satellites and race cars\\\\
- Same force keeps Moon circling Earth and cars on curved roads\\\\
- Three revelations: angles and spin, acceleration toward center, forces that turn\\\\
[THE WONDER] By end of class, you'll understand why astronauts float - they're falling forever\\\\
- This explains orbits, planets, and why you feel pushed outward on curves}
}

\begin{frame}
\frametitle{Outline}
\tableofcontents
\end{frame}

\section{Introduction}

\begin{frame}
\frametitle{The Mystery}
\begin{center}
\Large How do you move forward\\
\textit{while constantly turning?}
\end{center}

\pause
\vspace{0.5cm}
From Formula 1 cars screaming around curves to the Moon circling Earth...

\pause
\vspace{0.3cm}
\alert{All require a force toward the center.}
\note{[P0] "How do you move forward while constantly turning?"\\\\
[P1] "From Formula 1 cars to the Moon circling Earth..."\\\\
[P2] [THE WONDER] "All require a force toward the center. Newton's first law says objects go straight - so turning requires force"\\\\
[THE CONNECTION - Kinetic Archetype] "Athletes: every time you run a curve, friction pulls you inward"}
\end{frame}

\begin{frame}
\frametitle{Falling Forever}
\begin{figure}
\centering
\includegraphics[width=0.7\textwidth,height=0.5\textheight,keepaspectratio]{phys11-circular-motion-fig6-1.jpg}
\caption{Formula 1 car in circular motion}
\end{figure}

\pause
\begin{exampleblock}{The Mental Model}
A satellite in orbit is falling toward Earth but moving fast enough sideways to keep missing it.
\end{exampleblock}
\note{[Fig 6.1: Race car with large wheels on track] "Teaching hint: Point to the wheels - they spin AND move in circle, both rotational motions happening together"\\\\
[P0] "Formula 1 cars - wheels spin, car curves"\\\\
[P1] [THE REVELATION] "A satellite is falling toward Earth but moving sideways fast enough to keep missing"\\\\
[THE CONNECTION - Digital Archetype] "Like a game character running around a circular track"\\\\
[THE WONDER] Orbiting is controlled falling - you fall forever without hitting ground}
\end{frame}

\section{Angle of Rotation and Angular Velocity}

\begin{frame}
\frametitle{Learning Objectives}
\begin{block}{By the end of this section, you will be able to:}
\begin{itemize}
\item \textbf{6.1:} Describe the angle of rotation and relate it to its linear counterpart \pause
\item \textbf{6.1:} Describe angular velocity and relate it to its linear counterpart \pause
\item \textbf{6.1:} Solve problems involving angle of rotation and angular velocity
\end{itemize}
\end{block}
\note{[P0] "Three objectives for angles and rotation"\\\\
[P1] "First: understand angles in radians, not just degrees"\\\\
[P2] "Second: angular velocity - how fast something spins"\\\\
[P3] "Third: connect spinning motion to linear motion"\\\\
- We'll use these to understand everything from CDs to planets}
\end{frame}

\begin{frame}
\frametitle{6.1 Two Kinds of Rotation}
\textbf{Circular motion:} Object moves in a circular path (race car on track)

\pause
\vspace{0.3cm}

\textbf{Spin:} Object rotates about its own axis (Earth spinning)

\pause
\vspace{0.3cm}

\begin{exampleblock}{Real-World Examples}
\begin{itemize}
\item Earth spins on its axis (spin) AND orbits the Sun (circular motion)
\item Your car tire spins (spin) while the car follows a curve (circular motion)
\end{itemize}
\end{exampleblock}
\note{[P0] "Two kinds of rotation"\\\\
[P1] "Circular motion: object moves in circular path like race car"\\\\
[P2] "Spin: object rotates about own axis like Earth"\\\\
[P3] [THE CONNECTION] "Earth does both - spins on axis AND orbits Sun"\\\\
- Name wheel: give another example of both types}
\end{frame}

\begin{frame}
\frametitle{6.1 Angle of Rotation}
\begin{columns}[T]
\column{0.48\textwidth}
\begin{figure}
\centering
\includegraphics[width=\linewidth,height=0.55\textheight,keepaspectratio]{phys11-circular-motion-fig6-3.jpg}
\caption{Arc length and radius}
\end{figure}

\pause
\column{0.48\textwidth}
\begin{block}{Universal Law: Angle of Rotation}
$$\boxed{\Delta\angle{\theta} = \frac{\Delta \disp{s}}{\radius{r}}}$$
\angle{Angle} equals arc length divided by \radius{radius}
\end{block}

\pause
\vspace{0.3cm}
Measured in \textbf{radians} (rad)
\end{columns}
\note{[Fig 6.3: Circle showing radius r, arc change s from A to B, angle delta-theta, formula] "Teaching hint: Trace arc with finger - same angle theta works for any radius, but arc length scales with r"\\\\
[P0] "Points on a CD all rotate through same angle"\\\\
[P1] [THE REVELATION] "Angle equals arc length divided by radius"\\\\
[P2] "Measured in radians - ratio of two distances, so dimensionless"\\\\
[THE CONNECTION - Harmonic Archetype] "Musicians: radians are like beat divisions - natural units for rotation"}
\end{frame}

\begin{frame}
\frametitle{6.1 Radians vs Degrees}
\begin{block}{The Conversion}
$$\boxed{1 \text{ revolution} = 2\pi \text{ rad} = 360^\circ}$$
\end{block}

\pause
\vspace{0.3cm}

\textbf{Common conversions:}
\begin{itemize}
\item $\frac{\pi}{2}$ rad = $90^\circ$ \pause
\item $\pi$ rad = $180^\circ$ \pause
\item 1 rad $\approx$ $57.3^\circ$
\end{itemize}

\pause

\begin{alertblock}{Why Radians?}
Radians simplify equations in physics. Degrees are arbitrary - radians are natural.
\end{alertblock}
\note{[P0] "One full circle is 2 pi radians or 360 degrees"\\\\
[P1] "Pi over 2 radians equals 90 degrees"\\\\
[P2] "Pi radians equals 180 degrees"\\\\
[P3] "1 radian equals about 57 degrees"\\\\
[P4] [THE CONFLICT] "Degrees are arbitrary - Babylonians chose 360. Radians are natural - based on circle itself"\\\\
[THE WONDER] All physics equations work cleaner in radians}
\end{frame}

\begin{frame}
\frametitle{6.1 Angular Velocity}
\begin{block}{Universal Law: \angvel{Angular Velocity}}
\begin{center}
\Large $\boxed{\angvel{\omega} = \frac{\Delta\angle{\theta}}{\Delta \tvar{t}}}$
\end{center}
\angvel{Angular velocity} equals change in \angle{angle} divided by change in \tvar{time}
\end{block}

\pause
\vspace{0.3cm}

\textbf{Units:} radians per second (rad/s)

\pause

\textbf{Direction:}
\begin{itemize}
\item Counterclockwise: positive (out of page toward you)
\item Clockwise: negative (into page away from you)
\end{itemize}
\note{[P0] [THE REVELATION] "Omega equals delta theta over delta t - angular version of velocity"\\\\
[P1] "Units are radians per second"\\\\
[P2] "Direction: counterclockwise is positive, clockwise is negative"\\\\
[THE CONNECTION - Digital Archetype] "Like frame rate in animation - how many radians per second"}
\end{frame}

\begin{frame}
\frametitle{6.1 Connecting Spinning to Moving}
\begin{block}{The Bridge Equation}
\begin{center}
\Large $\boxed{\vel{v} = \radius{r}\angvel{\omega}}$
\end{center}
Tangential \vel{velocity} equals \radius{radius} times \angvel{angular velocity}
\end{block}

\pause
\vspace{0.3cm}

\begin{exampleblock}{The Mental Model}
Points farther from the center move faster linearly, but all points have the same \angvel{angular velocity}.
\end{exampleblock}

\pause

\textbf{Example:} CD spinning - outer edge moves faster than inner part, but both complete one revolution in same \tvar{time}.
\note{[P0] [THE REVELATION] "v equals r omega - connects linear and angular motion"\\\\
[P1] "Points farther from center move faster linearly but same angular velocity"\\\\
[P2] "CD example: outer edge travels farther in same time"\\\\
[THE HUMILITY] This feels weird - same rotation, different speeds\\\\
[THE WONDER] Same equation works for Earth's rotation, car tires, galaxies}
\end{frame}

\begin{frame}
\frametitle{6.1 Why Car Tires Matter}
\begin{figure}
\centering
\includegraphics[width=0.7\textwidth,height=0.5\textheight,keepaspectratio]{phys11-circular-motion-fig6-5.jpg}
\caption{Car tire rolling}
\end{figure}

\pause

Large $\angvel{\omega}$ means large $\vel{v}$ because $\vel{v} = \radius{r}\angvel{\omega}$

\pause

Larger \radius{radius} tire at same $\angvel{\omega}$ produces greater $\vel{v}$
\note{[Fig 6.5: Car front showing wheel with omega arrow clockwise, velocity arrow forward, radius r, equation v=r omega] "Teaching hint: Point to equation on diagram - shows why bigger wheels move car faster at same tire rotation"\\\\
[P0] "Tire spinning - how does car move forward?"\\\\
[P1] "Large omega means large v"\\\\
[P2] "Larger radius tire produces greater linear velocity"\\\\
[THE CONNECTION - Kinetic Archetype] "Cyclists: why bigger wheels go farther per pedal stroke"\\\\
[THE WONDER] Your car speedometer measures tire rotation - if you change tire size, it reads wrong}
\end{frame}

\begin{frame}
\frametitle{Attempt: Clock Tower Angle}
\begin{exampleblock}{The Challenge (3 min, silent)}
A clock tower has a radius of 1.0 m. The hour hand moves from 12 p.m. to 3 p.m.

\vspace{0.3cm}

\textbf{Given:}
\begin{itemize}
\item \radius{Radius} $\radius{r} = 1.0$ m
\item \tvar{Time}: 12 to 3 (quarter rotation)
\end{itemize}

\textbf{Find:}
\begin{enumerate}
\item \angle{Angle} of rotation in radians
\item Arc length along outer edge
\end{enumerate}

\vspace{0.3cm}

\textit{Can you decode this rotation? Work silently.}
\end{exampleblock}
\note{[THE CHALLENGE] Can they connect fractions of circle to radians?\\\\
[SAY] "Try this on your own. 3 minutes silent work."\\\\
[TIMING] 3-4 min SILENT individual work\\\\
[CIRCULATE] Note who knows full circle is 2 pi radians\\\\
[WATCH FOR] Students using 90 degrees instead of converting to radians\\\\
[DON'T HELP] Let them struggle - they learn more in Compare phase}
\end{frame}

\begin{frame}
\frametitle{Compare: Clock Tower}
\textbf{Turn and talk (2 min):}

\vspace{0.3cm}

\begin{enumerate}
\item What fraction of a full rotation does the hour hand make from 12 to 3?
\item How many radians in a full circle?
\item What equation connects arc length to angle?
\end{enumerate}

\vspace{0.5cm}

\pause
\alert{Name wheel:} One pair share your approach (not your answer).
\note{[TIMING] 2-3 min pair discussion\\\\
[CIRCULATE] Listen for common approaches\\\\
[CHECK] Name wheel: call a pair to share\\\\
[EXPECTED APPROACH] Quarter circle equals pi over 2 radians, then use delta s equals r delta theta\\\\
[COMMON ERROR] Using 90 degrees directly without converting to radians}
\end{frame}

\begin{frame}
\frametitle{Reveal: The Geometry of Time}
\textbf{Self-correct in a different color:}

\vspace{0.3cm}

\textbf{Part (a):} From 12 to 3 is $\frac{1}{4}$ of full rotation

\pause

Full rotation = $2\pi$ rad, so \angle{angle} = $\frac{1}{4} \times 2\pi = \boxed{\frac{\pi}{2} \text{ rad}}$

\pause
\vspace{0.3cm}

\textbf{Part (b):} Use $\Delta \disp{s} = \radius{r}\Delta\angle{\theta}$

\pause

$$\Delta \disp{s} = (1.0 \text{ m})\left(\frac{\pi}{2} \text{ rad}\right) = \boxed{1.6 \text{ m}}$$

\pause

\textbf{Check:} Arc length is less than circumference ($2\pi r \approx 6.3$ m). Reasonable!
\note{[P0] "Self-correct in different color"\\\\
[P1] [ALGEBRA] "From 12 to 3 is one-quarter rotation"\\\\
[P2] "Quarter of 2 pi equals pi over 2 radians"\\\\
[P3] [ALGEBRA] "Arc length equals r delta theta"\\\\
[P4] [ANSWER] "1.6 meters - about 5 feet along outer edge"\\\\
[THE WONDER] Same geometry works for planet orbits - angles and arcs connect everywhere}
\end{frame}

\begin{frame}
\frametitle{Attempt: Spinning Car Tire}
\begin{exampleblock}{The Challenge (3 min, silent)}
A car tire has radius 0.300 m and the car travels at 15.0 m/s (about 54 km/h).

\vspace{0.3cm}

\textbf{Given:}
\begin{itemize}
\item \radius{Radius} $\radius{r} = 0.300$ m
\item Tangential \vel{velocity} $\vel{v} = 15.0$ m/s
\end{itemize}

\textbf{Find:} \angvel{Angular velocity} $\angvel{\omega}$ of the tire in rad/s

\vspace{0.3cm}

\textit{How fast is the tire spinning?}
\end{exampleblock}
\note{[THE CHALLENGE] Can they rearrange the bridge equation?\\\\
[SAY] "How fast does the tire spin to move the car at 15 m/s?"\\\\
[TIMING] 3 min SILENT work\\\\
[CIRCULATE] Watch for students using v equals r omega\\\\
[WATCH FOR] Students multiplying instead of dividing\\\\
[DON'T HELP] Productive struggle builds understanding}
\end{frame}

\begin{frame}
\frametitle{Compare: Tire Speed}
\textbf{Turn and talk (2 min):}

\vspace{0.3cm}

\begin{enumerate}
\item What equation connects linear and \angvel{angular velocity}?
\item How did you rearrange it to solve for $\angvel{\omega}$?
\item What are the units of your answer?
\end{enumerate}

\vspace{0.5cm}

\pause
\alert{Name wheel:} One pair share your rearrangement strategy.
\note{[TIMING] 2 min pair discussion\\\\
[CIRCULATE] Listen for equation choices\\\\
[CHECK] Name wheel: call pair to explain rearrangement\\\\
[EXPECTED APPROACH] v equals r omega, so omega equals v over r\\\\
[COMMON ERROR] Forgetting to include rad in units}
\end{frame}

\begin{frame}
\frametitle{Reveal: The Spinning Wheel}
\textbf{Self-correct in a different color:}

\vspace{0.3cm}

\textbf{Equation:} $\vel{v} = \radius{r}\angvel{\omega}$, so $\angvel{\omega} = \frac{\vel{v}}{\radius{r}}$

\pause

\textbf{Substitute:}
$$\angvel{\omega} = \frac{15.0 \text{ m/s}}{0.300 \text{ m}}$$

\pause

$$\boxed{\angvel{\omega} = 50.0 \text{ rad/s}}$$

\pause

\textbf{Check:} About 8 revolutions per second (since $2\pi$ rad = 1 rev). Fast but reasonable for highway speed!
\note{[P0] "Self-correct in different color"\\\\
[P1] [ALGEBRA] "v equals r omega, so omega equals v over r"\\\\
[P2] "15 divided by 0.3 equals 50"\\\\
[P3] [ANSWER] "50 radians per second - about 8 revolutions per second"\\\\
[THE WONDER] Larger tire would spin slower for same car speed - monster trucks have slow tire rotation}
\end{frame}

\section{Uniform Circular Motion}

\begin{frame}
\frametitle{Learning Objectives}
\begin{block}{By the end of this section, you will be able to:}
\begin{itemize}
\item \textbf{6.2:} Describe centripetal acceleration and relate it to linear acceleration \pause
\item \textbf{6.2:} Describe centripetal force and relate it to linear force \pause
\item \textbf{6.2:} Solve problems involving centripetal acceleration and centripetal force
\end{itemize}
\end{block}
\note{[P0] "Three objectives for circular motion"\\\\
[P1] "First: acceleration toward center even at constant speed"\\\\
[P2] "Second: force required to turn - centripetal force"\\\\
[P3] "Third: calculate forces on curves and in orbits"\\\\
- This explains why you feel pushed outward in turning car}
\end{frame}

\begin{frame}
\frametitle{6.2 The Paradox of Constant Speed}
\textbf{Uniform circular motion:} Object travels circular path at constant \vel{speed}

\pause

\begin{alertblock}{Civilian View vs. Reality}
\textbf{Civilian:} "Constant \vel{speed} means no \accel{acceleration}."\\
\textbf{Physicist:} "\vel{Velocity} is changing direction, so there IS \accel{acceleration}."
\end{alertblock}

\pause
\vspace{0.3cm}

\accel{Acceleration} is a change in \vel{velocity} - magnitude OR direction!
\note{[P0] "Uniform circular motion - constant speed on circular path"\\\\
[P1] [THE CONFLICT] "Civilians think constant speed means no acceleration"\\\\
[P2] "Physicists know velocity includes direction - changing direction IS acceleration"\\\\
[THE HUMILITY] This confused scientists for centuries before Newton\\\\
[THE WONDER] Moon accelerates toward Earth constantly but never gets closer}
\end{frame}

\begin{frame}
\frametitle{6.2 The Illusion of Being Flung}
\begin{exampleblock}{The Mental Model}
When you turn in a car, you feel pushed outward. But no \force{force} pushes you out - your body wants to go straight (Newton's first law) while the car turns.
\end{exampleblock}

\pause
\vspace{0.3cm}

\begin{alertblock}{The Fictional Force}
\textbf{Centrifugal force} is not real - it's the illusion created by your inertia resisting the turn.
\end{alertblock}

\pause

The real \force{force} is \textbf{centripetal} - pulling you inward toward the center!
\note{[P0] "When car turns, you feel pushed outward"\\\\
[P1] [THE CONFLICT] "No force pushes you out - your body wants to go straight, car turns"\\\\
[P2] "Centrifugal force is fictional - centripetal force is real and pulls inward"\\\\
[THE CONNECTION - Kinetic Archetype] "Athletes: when you cut on basketball court, your shoes push inward"\\\\
[THE WONDER] Astronauts in space station feel weightless not because there's no gravity but because they're in constant freefall}
\end{frame}

\begin{frame}
\frametitle{6.2 Centripetal Acceleration}
\begin{figure}
\centering
\includegraphics[width=0.7\textwidth,height=0.45\textheight,keepaspectratio]{phys11-circular-motion-fig6-7.jpg}
\caption{Velocity changes direction, acceleration points toward center}
\end{figure}

\pause

\begin{block}{Universal Law: Centripetal \accel{Acceleration}}
$$\boxed{\accel{a_c} = \frac{\vel{v}^2}{\radius{r}}} \quad \text{or} \quad \boxed{\accel{a_c} = \radius{r}\angvel{\omega}^2}$$
\end{block}
\note{[Fig 6.7: Velocity triangle showing delta-v = v2 - v1, circle with arc delta-s, velocity vectors v1 at B and v2 at C] "Teaching hint: Show velocity vectors are same length but different direction - change in direction means acceleration even at constant speed"\\\\
[P0] "Velocity arrows change direction - acceleration points inward"\\\\
[P1] [THE REVELATION] "a-c equals v-squared over r - acceleration toward center"\\\\
[THE HUMILITY] Proportional to speed squared surprises people\\\\
- Doubling speed quadruples acceleration\\\\
[THE WONDER] Tighter curves require more acceleration - why sharp turns feel intense}
\end{frame}

\begin{frame}
\frametitle{6.2 Why Speed Squared Matters}
$$\accel{a_c} = \frac{\vel{v}^2}{\radius{r}}$$

\pause
\vspace{0.3cm}

\textbf{Proportional to $\vel{v}^2$:} Doubling \vel{speed} means 4 times the \accel{acceleration}!

\pause

\textbf{Example:}
\begin{itemize}
\item Curve at 50 km/h: moderate \accel{acceleration}
\item Same curve at 100 km/h: \alert{4 times} the \accel{acceleration}
\end{itemize}

\pause

\begin{alertblock}{The Warning}
This is why \vel{speed} limits are lower on curves - small \vel{speed} increase creates huge \accel{acceleration} increase.
\end{alertblock}
\note{[P0] "Centripetal acceleration proportional to speed squared"\\\\
[P1] "Doubling speed quadruples acceleration"\\\\
[P2] "Curve at 50 versus 100 km/h - 4 times acceleration"\\\\
[P3] [THE CONNECTION] "Why speed limits drop on curves - engineers calculated safe acceleration"\\\\
[THE WONDER] Race car drivers feel multiple g-forces in tight turns at high speed}
\end{frame}

\begin{frame}
\frametitle{6.2 Centripetal Force}
Newton's second law: $\force{F}_{\text{net}} = \mass{m}\accel{a}$

\pause

For circular motion: $\force{F}_{\text{net}} = \mass{m}\accel{a_c}$

\pause

\begin{block}{Universal Law: Centripetal \force{Force}}
$$\boxed{\force{F_c} = \mass{m}\frac{\vel{v}^2}{\radius{r}}} \quad \text{or} \quad \boxed{\force{F_c} = \mass{m}\radius{r}\angvel{\omega}^2}$$
\end{block}

\pause

\textbf{Direction:} Always toward the center of rotation
\note{[P0] "Newton's second law - net force equals mass times acceleration"\\\\
[P1] "For circular motion - net force equals m times a-c"\\\\
[P2] [THE REVELATION] "F-c equals m v-squared over r - centripetal force formula"\\\\
[P3] "Direction always toward center"\\\\
[THE WONDER] Without centripetal force, object flies off in straight line - Newton's first law}
\end{frame}

\begin{frame}
\frametitle{6.2 Sources of Centripetal Force}
\textbf{Centripetal \force{force} can be provided by:}

\begin{itemize}
\item \textbf{Friction:} Car tires on road \pause
\item \textbf{Tension:} String on tetherball \pause
\item \textbf{\grav{Gravity}:} Moon orbiting Earth \pause
\item \textbf{Normal force:} Roller coaster on loop
\end{itemize}

\pause
\vspace{0.3cm}

\begin{exampleblock}{The Mental Model}
Centripetal \force{force} isn't a new kind of \force{force} - it's whatever \force{force} points toward the center and causes circular motion.
\end{exampleblock}
\note{[P0] "Centripetal force provided by different forces"\\\\
[P1] "Friction: car tires gripping road in turn"\\\\
[P2] "Tension: string holding tetherball in circle"\\\\
[P3] "Gravity: Moon orbiting Earth"\\\\
[P4] "Normal force: roller coaster on loop"\\\\
[P5] [THE REVELATION] "Centripetal force is not new type - it's the role a force plays"\\\\
- Name wheel: give another example}
\end{frame}

\begin{frame}
\frametitle{Attempt: Car on Curve}
\begin{exampleblock}{The Challenge (3 min, silent)}
A 900 kg car rounds a curve with radius 600 m at speed 25.0 m/s.

\vspace{0.3cm}

\textbf{Given:}
\begin{itemize}
\item \mass{Mass} $\mass{m} = 900$ kg
\item \radius{Radius} $\radius{r} = 600$ m
\item \vel{Speed} $\vel{v} = 25.0$ m/s
\end{itemize}

\textbf{Find:} Centripetal \force{force} required to keep car on curve

\vspace{0.3cm}

\textit{How much force do the tires provide?}
\end{exampleblock}
\note{[THE CHALLENGE] Can they apply centripetal force formula?\\\\
[SAY] "Calculate the force friction must provide. 3 minutes silent."\\\\
[TIMING] 3-4 min SILENT work\\\\
[CIRCULATE] Note who uses F-c equals m v-squared over r\\\\
[WATCH FOR] Students forgetting to square the velocity\\\\
[DON'T HELP] They'll catch errors in Compare}
\end{frame}

\begin{frame}
\frametitle{Compare: Car Force}
\textbf{Turn and talk (2 min):}

\vspace{0.3cm}

\begin{enumerate}
\item What equation did you use for centripetal \force{force}?
\item Did you remember to square the \vel{velocity}?
\item What \force{force} provides the centripetal \force{force} for a car?
\end{enumerate}

\vspace{0.5cm}

\pause
\alert{Name wheel:} One pair share your equation and reasoning.
\note{[TIMING] 2 min pair discussion\\\\
[CIRCULATE] Listen for equation choices\\\\
[CHECK] Name wheel: which equation did you use?\\\\
[EXPECTED APPROACH] F-c equals m v-squared over r\\\\
[COMMON ERROR] Using v instead of v-squared\\\\
[INSIGHT] Friction between tires and road provides centripetal force}
\end{frame}

\begin{frame}
\frametitle{Reveal: The Force That Turns}
\textbf{Self-correct in a different color:}

\vspace{0.3cm}

\textbf{Equation:} $\force{F_c} = \mass{m}\frac{\vel{v}^2}{\radius{r}}$

\pause

\textbf{Substitute:}
$$\force{F_c} = \frac{(900 \text{ kg})(25.0 \text{ m/s})^2}{600 \text{ m}}$$

\pause

$$\force{F_c} = \frac{(900)(625)}{600} = \boxed{938 \text{ N}}$$

\pause

\textbf{Check:} About 940 N - this is the friction \force{force} between tires and road. Without it, car slides straight!
\note{[P0] "Self-correct in different color"\\\\
[P1] [ALGEBRA] "F-c equals m v-squared over r"\\\\
[P2] "900 times 625 divided by 600"\\\\
[P3] [ANSWER] "938 Newtons - friction between tires and road"\\\\
[THE WONDER] On ice, friction drops - car can't generate enough centripetal force, slides straight off curve}
\end{frame}

\begin{frame}
\frametitle{Attempt: Acceleration Comparison}
\begin{exampleblock}{The Challenge (3 min, silent)}
A car follows a curve of radius 500 m at speed 25.0 m/s.

\vspace{0.3cm}

\textbf{Given:}
\begin{itemize}
\item \radius{Radius} $\radius{r} = 500$ m
\item \vel{Speed} $\vel{v} = 25.0$ m/s
\item $\grav{g} = 9.80$ m/s$^2$
\end{itemize}

\textbf{Find:}
\begin{enumerate}
\item Centripetal \accel{acceleration}
\item Express as fraction of $\grav{g}$
\end{enumerate}

\vspace{0.3cm}

\textit{How does turning compare to falling?}
\end{exampleblock}
\note{[THE CHALLENGE] Can they calculate and compare accelerations?\\\\
[SAY] "Compare centripetal acceleration to gravity. 3 minutes."\\\\
[TIMING] 3 min SILENT work\\\\
[CIRCULATE] Watch for calculation then ratio\\\\
[WATCH FOR] Students forgetting to take ratio with g\\\\
[DON'T HELP] Comparison part is key insight}
\end{frame}

\begin{frame}
\frametitle{Compare: Acceleration Scale}
\textbf{Turn and talk (2 min):}

\vspace{0.3cm}

\begin{enumerate}
\item What equation did you use for centripetal \accel{acceleration}?
\item How did you express it as a fraction of $\grav{g}$?
\item Is the \accel{acceleration} large or small compared to \grav{gravity}?
\end{enumerate}

\vspace{0.5cm}

\pause
\alert{Name wheel:} One pair share your comparison method.
\note{[TIMING] 2 min pair discussion\\\\
[CIRCULATE] Listen for ratio calculation\\\\
[CHECK] Name wheel: how did you compare to g?\\\\
[EXPECTED APPROACH] Calculate a-c, then divide by g\\\\
[INSIGHT] Highway curves designed for comfortable acceleration - fraction of g}
\end{frame}

\begin{frame}
\frametitle{Reveal: Comparing to Gravity}
\textbf{Self-correct in a different color:}

\vspace{0.3cm}

\textbf{Calculate $\accel{a_c}$:}
$$\accel{a_c} = \frac{\vel{v}^2}{\radius{r}} = \frac{(25.0 \text{ m/s})^2}{500 \text{ m}}$$

\pause

$$\accel{a_c} = \frac{625}{500} = \boxed{1.25 \text{ m/s}^2}$$

\pause

\textbf{Compare to $\grav{g}$:}
$$\frac{\accel{a_c}}{\grav{g}} = \frac{1.25}{9.80} = 0.128 \quad \Rightarrow \quad \boxed{\accel{a_c} = 0.13\grav{g}}$$

\pause

\textbf{Revelation:} Gentle highway curve at moderate \vel{speed} produces about 1/10th the \accel{acceleration} of \grav{gravity}!
\note{[P0] "Self-correct in different color"\\\\
[P1] [ALGEBRA] "a-c equals v-squared over r equals 625 over 500"\\\\
[P2] [ANSWER] "1.25 meters per second squared"\\\\
[P3] "Divide by 9.8 to get 0.13 - about one-tenth g"\\\\
[THE WONDER] Race cars pull 3 to 4 g's in tight turns - fighter pilots can pull 9 g's before blacking out}
\end{frame}

\section{Rotational Motion}

\begin{frame}
\frametitle{Learning Objectives}
\begin{block}{By the end of this section, you will be able to:}
\begin{itemize}
\item \textbf{6.3:} Describe rotational kinematic variables and relate them to linear counterparts \pause
\item \textbf{6.3:} Describe torque and lever arm \pause
\item \textbf{6.3:} Solve problems involving torque and rotational kinematics
\end{itemize}
\end{block}
\note{[P0] "Three objectives for rotation"\\\\
[P1] "First: angular acceleration - how spin changes"\\\\
[P2] "Second: torque - the rotational version of force"\\\\
[P3] "Third: predict how objects spin up and slow down"\\\\
- This explains everything from figure skaters to fishing reels}
\end{frame}

\begin{frame}
\frametitle{6.3 When Spin Changes}
So far: constant angular velocity

\pause

\textbf{But what if spin changes?}
\begin{itemize}
\item Figure skater pulls arms in - spins faster \pause
\item Child pushes merry-go-round - starts rotating \pause
\item CD player stops - disc slows to halt
\end{itemize}

\pause

\begin{block}{Universal Law: \angacc{Angular Acceleration}}
$$\boxed{\angacc{\alpha} = \frac{\Delta\angvel{\omega}}{\Delta \tvar{t}}}$$
Rate of change of \angvel{angular velocity}
\end{block}
\note{[P0] "So far constant angular velocity"\\\\
[P1] "But what if spin changes?"\\\\
[P2] "Figure skater pulls arms in - spins faster"\\\\
[P3] "Child pushes merry-go-round"\\\\
[P4] "CD slows to stop"\\\\
[P5] [THE REVELATION] "Alpha equals delta omega over delta t - angular acceleration"\\\\
[THE WONDER] Same pattern - linear has a, rotational has alpha}
\end{frame}

\begin{frame}
\frametitle{6.3 Connecting Linear and Angular Acceleration}
\begin{block}{The Bridge}
$$\boxed{\accel{a} = \radius{r}\angacc{\alpha}} \quad \text{or} \quad \boxed{\angacc{\alpha} = \frac{\accel{a}}{\radius{r}}}$$
\end{block}

\pause
\vspace{0.3cm}

\textbf{Tangential \accel{acceleration}:} Linear \accel{acceleration} along the circle's edge

\pause

\begin{exampleblock}{The Mental Model}
Greater \angacc{angular acceleration} means greater tangential \accel{acceleration}. Points farther from center have larger tangential \accel{acceleration} for same $\angacc{\alpha}$.
\end{exampleblock}
\note{[P0] [THE REVELATION] "a equals r alpha - connects linear and angular acceleration"\\\\
[P1] "Tangential acceleration - linear acceleration along circle's edge"\\\\
[P2] "Greater alpha means greater a. Farther from center means larger a"\\\\
[THE CONNECTION - Kinetic Archetype] "Dancers: spin with arms out, bring them in - you accelerate"\\\\
[THE WONDER] Conservation of angular momentum - skaters use this to spin faster}
\end{frame}

\begin{frame}
\frametitle{6.3 Rotational Kinematics Equations}
\begin{center}
\small
\begin{tabular}{ll}
\textbf{Linear} & \textbf{Rotational} \\ \hline
$\vel{v} = \vel{v_0} + \accel{a}\tvar{t}$ & $\angvel{\omega} = \angvel{\omega_0} + \angacc{\alpha} \tvar{t}$ \\
$\disp{x} = \disp{x_0} + \vel{v_0} \tvar{t} + \frac{1}{2}\accel{a}\tvar{t}^2$ & $\angle{\theta} = \angle{\theta_0} + \angvel{\omega_0} \tvar{t} + \frac{1}{2}\angacc{\alpha} \tvar{t}^2$ \\
$\vel{v}^2 = \vel{v_0}^2 + 2\accel{a}\Delta \disp{x}$ & $\angvel{\omega}^2 = \angvel{\omega_0}^2 + 2\angacc{\alpha}\Delta\angle{\theta}$
\end{tabular}
\end{center}

\pause
\vspace{0.3cm}

\begin{alertblock}{The Pattern}
Every linear kinematics equation has a rotational analog. Just swap $\disp{x} \rightarrow \angle{\theta}$, $\vel{v} \rightarrow \angvel{\omega}$, $\accel{a} \rightarrow \angacc{\alpha}$!
\end{alertblock}
\note{[P0] "Three rotational kinematics equations - exact analogs of linear"\\\\
[P1] [THE REVELATION] "Swap x for theta, v for omega, a for alpha"\\\\
[THE HUMILITY] Same math structure - physics has beautiful symmetry\\\\
[THE WONDER] Universe reuses patterns - rotational motion mirrors linear motion}
\end{frame}

\begin{frame}
\frametitle{6.3 The Rotational Version of Force}
\textbf{\force{Force} causes linear \accel{acceleration}}

What causes \angacc{angular acceleration}?

\pause

\begin{block}{Universal Law: \torque{Torque}}
$$\boxed{\torque{\tau} = \radius{r}\force{F}\sin\angle{\theta}}$$
\torque{Torque} equals lever arm times \force{force} times sine of \angle{angle}
\end{block}

\pause
\vspace{0.3cm}

\textbf{Units:} N$\cdot$m (Newton-meters)

\textbf{Direction:} Same as the \angacc{angular acceleration} it produces
\note{[P0] "Force causes linear acceleration - what causes angular?"\\\\
[P1] [THE REVELATION] "Torque - tau equals r F sine theta"\\\\
[P2] "Units are Newton-meters. Direction same as angular acceleration"\\\\
[THE CONNECTION - Kinetic Archetype] "Athletes: opening a door - push far from hinge for easier opening"\\\\
[THE WONDER] Torque explains why wrenches have long handles}
\end{frame}

\begin{frame}
\frametitle{6.3 Maximizing Torque}
$$\torque{\tau} = \radius{r}\force{F}\sin\angle{\theta}$$

\pause

\textbf{To maximize \torque{torque}:}
\begin{itemize}
\item Apply \force{force} far from pivot (large $\radius{r}$) \pause
\item Apply \force{force} perpendicular to lever arm ($\angle{\theta} = 90^\circ$, so $\sin\angle{\theta} = 1$) \pause
\item Apply larger \force{force} (large $\force{F}$)
\end{itemize}

\pause

\begin{exampleblock}{Real-World Applications}
\begin{itemize}
\item Door handle placed far from hinges
\item Wrench with long handle
\item Teeter-totter balanced by distance and weight
\end{itemize}
\end{exampleblock}
\note{[P0] "Three ways to maximize torque"\\\\
[P1] "Large r - apply force far from pivot"\\\\
[P2] "Theta equals 90 degrees - perpendicular force"\\\\
[P3] "Large F - stronger push"\\\\
[P4] [THE CONNECTION] "Door handle far from hinge - long wrench loosens tight bolt"\\\\
[THE WONDER] Archimedes: give me lever long enough and I'll move the world}
\end{frame}

\begin{frame}
\frametitle{Attempt: Fishing Reel}
\begin{exampleblock}{The Challenge (3 min, silent)}
A fishing reel spins at $\omega_0 = 220$ rad/s. Fisherman applies brake creating angular acceleration $\alpha = -300$ rad/s$^2$.

\vspace{0.3cm}

\textbf{Given:}
\begin{itemize}
\item Initial $\angvel{\omega_0} = 220$ rad/s
\item Final $\angvel{\omega} = 0$ (stops)
\item $\angacc{\alpha} = -300$ rad/s$^2$
\end{itemize}

\textbf{Find:} \tvar{Time} $\tvar{t}$ for reel to stop

\vspace{0.3cm}

\textit{How long does it take?}
\end{exampleblock}
\note{[THE CHALLENGE] Can they use rotational kinematics?\\\\
[SAY] "Big fish pulls line - how long to stop reel? 3 minutes."\\\\
[TIMING] 3 min SILENT work\\\\
[CIRCULATE] Watch for omega equals omega-zero plus alpha t\\\\
[WATCH FOR] Sign errors with negative alpha\\\\
[DON'T HELP] Let them wrestle with algebra}
\end{frame}

\begin{frame}
\frametitle{Compare: Fishing Reel}
\textbf{Turn and talk (2 min):}

\vspace{0.3cm}

\begin{enumerate}
\item Which rotational kinematics equation did you choose?
\item How did you solve for time $t$?
\item Why is $\alpha$ negative?
\end{enumerate}

\vspace{0.5cm}

\pause
\alert{Name wheel:} One pair share your equation choice and reasoning.
\note{[TIMING] 2 min pair discussion\\\\
[CIRCULATE] Listen for equation selection\\\\
[CHECK] Name wheel: which equation and why?\\\\
[EXPECTED APPROACH] omega equals omega-zero plus alpha t, solve for t\\\\
[COMMON ERROR] Sign confusion with negative alpha\\\\
[INSIGHT] Alpha is negative because reel is slowing down}
\end{frame}

\begin{frame}
\frametitle{Reveal: Stopping the Spin}
\textbf{Self-correct in a different color:}

\vspace{0.3cm}

\textbf{Equation:} $\angvel{\omega} = \angvel{\omega_0} + \angacc{\alpha} \tvar{t}$

\pause

\textbf{Solve for $\tvar{t}$:}
$$\tvar{t} = \frac{\angvel{\omega} - \angvel{\omega_0}}{\angacc{\alpha}}$$

\pause

\textbf{Substitute:}
$$\tvar{t} = \frac{0 - 220 \text{ rad/s}}{-300 \text{ rad/s}^2} = \frac{-220}{-300}$$

\pause

$$\boxed{\tvar{t} = 0.733 \text{ s}}$$

\pause

\textbf{Insight:} Less than one second because the angular acceleration is quite large!
\note{[P0] "Self-correct in different color"\\\\
[P1] [ALGEBRA] "omega equals omega-zero plus alpha t, solve for t"\\\\
[P2] "t equals omega minus omega-zero over alpha"\\\\
[P3] "Negative 220 divided by negative 300"\\\\
[P4] [ANSWER] "0.733 seconds - less than one second"\\\\
[THE WONDER] Fishermen let fish swim first - tired fish needs less force to stop, less likely to snap line}
\end{frame}

\begin{frame}
\frametitle{Attempt: Merry-Go-Round Torque}
\begin{exampleblock}{The Challenge (3 min, silent)}
A man pushes a merry-go-round with force 250 N at the edge, perpendicular to the radius of 1.50 m.

\vspace{0.3cm}

\textbf{Given:}
\begin{itemize}
\item \force{Force} $\force{F} = 250$ N
\item Lever arm $\radius{r} = 1.50$ m
\item \angle{Angle} $\angle{\theta} = 90^\circ$ (perpendicular)
\end{itemize}

\textbf{Find:} \torque{Torque} $\torque{\tau}$ produced

\vspace{0.3cm}

\textit{How effective is his push?}
\end{exampleblock}
\note{[THE CHALLENGE] Can they calculate torque with perpendicular force?\\\\
[SAY] "Calculate the torque. 3 minutes silent."\\\\
[TIMING] 3 min SILENT work\\\\
[CIRCULATE] Watch for tau equals r F sine theta\\\\
[WATCH FOR] Students forgetting sine 90 equals 1\\\\
[DON'T HELP] They'll recognize perpendicular simplifies in Compare}
\end{frame}

\begin{frame}
\frametitle{Compare: Torque Calculation}
\textbf{Turn and talk (2 min):}

\vspace{0.3cm}

\begin{enumerate}
\item What is the value of $\sin 90^\circ$?
\item How does this simplify the torque equation?
\item Why did the man push at the edge and perpendicular?
\end{enumerate}

\vspace{0.5cm}

\pause
\alert{Name wheel:} One pair explain why perpendicular force maximizes torque.
\note{[TIMING] 2 min pair discussion\\\\
[CIRCULATE] Listen for sine 90 equals 1\\\\
[CHECK] Name wheel: why perpendicular?\\\\
[EXPECTED APPROACH] tau equals r F, sine 90 equals 1\\\\
[INSIGHT] Perpendicular and at edge both maximize torque}
\end{frame}

\begin{frame}
\frametitle{Reveal: The Power of Position}
\textbf{Self-correct in a different color:}

\vspace{0.3cm}

\textbf{Equation:} $\torque{\tau} = \radius{r}\force{F}\sin\angle{\theta}$

\pause

\textbf{Substitute:} $\angle{\theta} = 90^\circ$, so $\sin 90^\circ = 1$

\pause

$$\torque{\tau} = (1.50 \text{ m})(250 \text{ N})(1)$$

\pause

$$\boxed{\torque{\tau} = 375 \text{ N}\cdot\text{m}}$$

\pause

\textbf{Strategy:} Man maximized \torque{torque} by pushing perpendicular at the outer edge!
\note{[P0] "Self-correct in different color"\\\\
[P1] [ALGEBRA] "tau equals r F sine theta"\\\\
[P2] "sine 90 degrees equals 1, so tau equals r F"\\\\
[P3] "1.5 times 250 equals 375"\\\\
[P4] [ANSWER] "375 Newton-meters"\\\\
[THE WONDER] Same force near center would produce much less torque - position matters as much as strength}
\end{frame}

\section{Summary}

\begin{frame}
\frametitle{What You Now Know}
\begin{block}{The Revelations}
\begin{enumerate}
\item Radians are natural units - \angle{angles} from circle geometry \pause
\item $\vel{v} = \radius{r}\angvel{\omega}$ connects spinning to moving \pause
\item Circular motion requires \accel{acceleration} toward center \pause
\item Centripetal \force{force} $\force{F_c} = \frac{\mass{m}\vel{v}^2}{\radius{r}}$ keeps objects turning \pause
\item Rotational kinematics mirrors linear kinematics \pause
\item \torque{Torque} $\torque{\tau} = \radius{r}\force{F}\sin\angle{\theta}$ is the rotational \force{force}
\end{enumerate}
\end{block}
\note{[P0] "Six revelations today"\\\\
[P1] "Radians are natural - based on circle itself"\\\\
[P2] "v equals r omega connects spinning to moving"\\\\
[P3] "Circular motion requires inward acceleration"\\\\
[P4] "Centripetal force keeps objects turning"\\\\
[P5] "Rotational kinematics mirrors linear"\\\\
[P6] "Torque is rotational force"\\\\
[THE WONDER] Same principles govern spinning galaxies, orbiting planets, turning cars\\\\
- Name wheel: which was most surprising?}
\end{frame}

\begin{frame}[shrink]
\frametitle{Key Equations}
\small
\begin{align}
\Delta\angle{\theta} &= \frac{\Delta \disp{s}}{\radius{r}} \quad \text{(angle of rotation)} \\
\angvel{\omega} &= \frac{\Delta\angle{\theta}}{\Delta \tvar{t}} \quad \text{(angular velocity)} \\
\vel{v} &= \radius{r}\angvel{\omega} \quad \text{(tangential velocity)} \\
\accel{a_c} &= \frac{\vel{v}^2}{\radius{r}} = \radius{r}\angvel{\omega}^2 \quad \text{(centripetal acceleration)} \\
\force{F_c} &= \mass{m}\frac{\vel{v}^2}{\radius{r}} = \mass{m}\radius{r}\angvel{\omega}^2 \quad \text{(centripetal force)} \\
\angacc{\alpha} &= \frac{\Delta\angvel{\omega}}{\Delta \tvar{t}} \quad \text{(angular acceleration)} \\
\accel{a} &= \radius{r}\angacc{\alpha} \quad \text{(tangential acceleration)} \\
\torque{\tau} &= \radius{r}\force{F}\sin\angle{\theta} \quad \text{(torque)}
\end{align}
\note{- Eight fundamental equations for circular and rotational motion\\\\
- First three: angles and spinning\\\\
- Next two: acceleration and force toward center\\\\
- Last three: changing rotation and torque\\\\
- Know when to use each\\\\
- Questions before homework?}
\end{frame}

\begin{frame}
\frametitle{Homework}
\begin{center}
\Large
Complete the assigned problems\\[0.3cm]
posted on the LMS
\end{center}
\note{[SAY] "Homework posted on LMS"\\\\
[TIMING] Due date: check LMS\\\\
[CHECK] Questions before we end?\\\\
[TRANSITION] Next class: Chapter 7 Newton's Law of Universal Gravitation\\\\
- We'll use circular motion to understand planetary orbits}
\end{frame}

\end{document}
