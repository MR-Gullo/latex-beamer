\documentclass{beamer}
\usepackage{../../../shared/templates/ds9_theme}
\usepackage[overridenote]{pdfpc}
\graphicspath{{../images/}{../../shared/images/}}

\title[Gravitation]{PHYS12 CH:7 The Force That Shapes Galaxies}
\subtitle{From Falling Apples to Orbiting Planets}
\author[Mr. Gullo]{Mr. Gullo}
\date[December 2025]{December 2025}

\begin{document}

\frame{\titlepage
\note{[THE HOOK] Today: discover invisible force governing everything from falling apples to galaxies\\\\
- Same force holds you to Earth AND guides planets around sun\\\\
- Two revelations: Kepler's patterns, Newton's explanation\\\\
[THE WONDER] By end of class, calculate orbital mechanics like NASA\\\\
- Foundation of astrophysics}
}

\begin{frame}
\frametitle{Outline}
\tableofcontents
\end{frame}

\section{Introduction}

\begin{frame}
\frametitle{The Mystery}
\begin{center}
\Large What do a falling apple and the moon\\
\textit{have in common?}
\end{center}

\pause
\vspace{0.5cm}
The same invisible force governs both...

\pause
\vspace{0.3cm}
\alert{Gravity shapes everything.}
\note{[P0] "What do falling apple and moon have in common?"\\\\
[P1] "Same invisible force governs both"\\\\
[P2] [THE WONDER] "Gravity shapes everything - pencil hitting floor to galaxies colliding"\\\\
[THE REVELATION] Newton connected terrestrial and celestial motion}
\end{frame}

\begin{frame}
\frametitle{Kepler and Newton}
\begin{figure}
\centering
\includegraphics[width=0.8\textwidth,height=0.6\textheight,keepaspectratio]{phys12-gravitation-fig01.jpg}
\caption{Kepler found patterns. Newton found cause.}
\end{figure}
\note{- Kepler: German astronomer, 1571-1630\\\\
- Crunched 20 years of data by hand\\\\
- Newton: English physicist, 1643-1727\\\\
- Found underlying force explaining Kepler's patterns\\\\
[THE CONNECTION - Digital Archetype] "Finding source code behind patterns"}
\end{frame}

\section{Kepler's Laws of Planetary Motion}

\begin{frame}
\frametitle{Learning Objectives}
\begin{block}{By the end of this section, you will be able to:}
\begin{itemize}
\item \textbf{7.1:} Explain Kepler's three laws of planetary motion \pause
\item \textbf{7.1:} Apply Kepler's laws to calculate characteristics of orbits
\end{itemize}
\end{block}
\note{[P0] "Two objectives for Kepler's laws"\\\\
[P1] "First: understand three laws. Second: use them to solve orbit problems"\\\\
- Laws describe ALL orbits: moons, planets, satellites\\\\
- Assessment: homework problems on orbital calculations}
\end{frame}

\begin{frame}
\frametitle{7.1 The Rules of Orbits}
\textbf{Kepler studied planetary motion for 20 years}

\pause
\vspace{0.3cm}

\textbf{Three laws describing all orbits:}
\begin{enumerate}
\item Planets orbit in ellipses with sun at one focus \pause
\item Equal areas in equal times \pause
\item Period squared proportional to radius cubed
\end{enumerate}

\pause
\begin{exampleblock}{The Mental Model}
Kepler found patterns in mountains of data. Described WHAT happens, not WHY.
\end{exampleblock}
\note{[P0] "Kepler studied planetary motion 20 years - by hand!"\\\\
[P1] "Three laws describe all orbits"\\\\
[P2] "First: ellipses with sun at one focus"\\\\
[P3] "Second: equal areas in equal times"\\\\
[P4] "Third: period squared proportional to radius cubed"\\\\
[P5] [THE HUMILITY] "Kepler found patterns but couldn't explain WHY. Took Newton."\\\\
[THE CONNECTION - Digital Archetype] "Kepler found repeating pattern in nature's code"}
\end{frame}

\begin{frame}
\frametitle{7.1 Kepler's First Law}
\begin{block}{The Law of Elliptical Orbits}
Orbit of each planet is ellipse with sun at one focus.
\end{block}

\pause
\begin{figure}
\centering
\includegraphics[width=0.7\textwidth,height=0.5\textheight,keepaspectratio]{phys12-gravitation-fig02.jpg}
\end{figure}

\pause
\textbf{Perihelion:} Closest approach to sun\\
\textbf{Aphelion:} Farthest distance from sun
\note{[P0] "First law: orbits are ellipses, not circles"\\\\
[P1] "Ellipse has two foci - sun at one focus"\\\\
[P2] "Perihelion: closest. Aphelion: farthest"\\\\
[THE CONFLICT] For centuries, everyone thought orbits were perfect circles\\\\
[THE REVELATION] Kepler proved they're ellipses - slightly squashed circles\\\\
- Earth's orbit nearly circular, comets highly elliptical}
\end{frame}

\begin{frame}
\frametitle{7.1 Anatomy of an Ellipse}
\begin{figure}
\centering
\includegraphics[width=0.6\textwidth,height=0.45\textheight,keepaspectratio]{phys12-gravitation-fig03.jpg}
\end{figure}

\pause
\textbf{Key property:} Sum of distances from any point to both foci is constant.

\pause
\vspace{0.3cm}

\textbf{Drawing an ellipse:} Pins, string, pencil method!
\note{[P0] "Ellipse has special property"\\\\
[P1] "Sum of distances to both foci constant - that's definition"\\\\
[P2] "Draw ellipse with pins, string, pencil"\\\\
- Two pins are foci\\\\
- String length stays constant\\\\
- Pencil traces ellipse\\\\
[THE CONNECTION - Kinetic Archetype] "Like tracing path planet must follow"}
\end{frame}

\begin{frame}
\frametitle{7.1 Ellipse Formulas}
\textbf{If you know perihelion $r_p$ and aphelion $r_a$:}

\pause
\begin{align}
\text{Semi-major axis: } a &= \frac{r_a + r_p}{2} \\
\text{Semi-minor axis: } b &= \sqrt{r_a r_p}
\end{align}

\pause
\vspace{0.3cm}

\textbf{Eccentricity} (elongation measure):
$$e = \frac{f}{a}$$

\pause
When $e = 0$, ellipse is circle!
\note{[P0] "Formulas to calculate ellipse dimensions"\\\\
[P1] "Semi-major axis: average of closest and farthest distances"\\\\
[P2] "Semi-minor axis: geometric mean"\\\\
[P3] "Eccentricity: zero means perfect circle, higher means more elongated"\\\\
- Earth eccentricity: 0.017 (nearly circular)\\\\
- Comet eccentricity: 0.9 or higher (very elongated)\\\\
[THE CONNECTION - Digital Archetype] "Parameters defining orbit's shape"}
\end{frame}

\begin{frame}
\frametitle{7.1 Kepler's Second Law}
\begin{block}{The Law of Equal Areas}
Line from sun to planet sweeps equal areas in equal times.
\end{block}

\pause
\begin{figure}
\centering
\includegraphics[width=0.7\textwidth,height=0.5\textheight,keepaspectratio]{phys12-gravitation-fig04.jpg}
\end{figure}

\pause
\begin{alertblock}{The Paradox}
\textbf{Result:} Planets move FASTER when closer to sun!
\end{alertblock}
\note{[P0] "Second law: equal areas in equal times"\\\\
[P1] "Shaded regions equal area, planet takes same time traverse each"\\\\
[P2] [THE CONFLICT] "Planet speeds up near sun, slows far away"\\\\
- Why? To sweep equal areas in equal times\\\\
- Distance shorter near sun, speed must increase\\\\
[THE CONNECTION - Kinetic Archetype] "Like skater pulling arms in to spin faster"}
\end{frame}

\begin{frame}
\frametitle{7.1 Kepler's Third Law}
\begin{block}{The Law of Periods}
Ratio of periods squared equals ratio of orbital radii cubed.
\end{block}

\pause
\begin{center}
\Large
$$\boxed{\frac{T_1^2}{T_2^2} = \frac{r_1^3}{r_2^3}}$$
\end{center}

\pause
\vspace{0.3cm}

\textbf{Where:}
\begin{itemize}
\item $T$ = period (time for one orbit)
\item $r$ = average orbital radius
\end{itemize}

\pause
\textbf{Critical:} Only works for satellites orbiting SAME parent body!
\note{[P0] "Third law: period squared over radius cubed is constant"\\\\
[P1] "Mathematical form: T-1 squared over T-2 squared equals r-1 cubed over r-2 cubed"\\\\
[P2] "T is period in years or days, r orbital radius"\\\\
[P3] "Only works comparing two objects orbiting same parent"\\\\
[THE REVELATION] Purely descriptive - no explanation WHY\\\\
[THE WONDER] Same ratio for Earth-Moon and satellites orbiting Earth}
\end{frame}

\begin{frame}
\frametitle{7.1 Ptolemy vs Copernicus}
\begin{columns}[T]
\column{0.48\textwidth}
\textbf{Ptolemaic (Geocentric)}
\begin{itemize}
\item Earth at center
\item Complex circular paths
\item Different rule each planet
\item Purely descriptive
\end{itemize}

\pause
\column{0.48\textwidth}
\textbf{Copernican (Heliocentric)}
\begin{itemize}
\item Sun at center
\item Simple elliptical paths
\item Same laws for all
\item Causal explanation
\end{itemize}
\end{columns}

\pause
\vspace{0.3cm}

\begin{alertblock}{The Conflict}
For 1400 years, everyone believed Earth center. Copernicus terrified to publish.
\end{alertblock}
\note{[P0] "Two competing solar system models"\\\\
[P1] "Copernican model: sun at center, much simpler"\\\\
[P2] [THE CONFLICT] "Copernicus waited 30 years publish - feared ridicule and persecution"\\\\
- Galileo put under house arrest for supporting this\\\\
- Took 300 years everyone accept\\\\
[THE HUMILITY] Being right doesn't mean believed\\\\
[THE WONDER] Truth won in end}
\end{frame}

\begin{frame}
\frametitle{Attempt: Moon and Satellite Orbits}
\begin{exampleblock}{The Challenge (3 min, silent)}
Moon orbits Earth every 27.3 days at average distance $3.84 \times 10^8$ m from Earth's center.

\vspace{0.3cm}

\textbf{Given:}
\begin{itemize}
\item Moon: $T_1 = 27.3$ d, $r_1 = 3.84 \times 10^8$ m
\item Satellite: altitude = 1500 km above surface
\item Earth radius = 6380 km
\end{itemize}

\textbf{Find:} Period of satellite

\vspace{0.3cm}

\textit{Can you predict how long this orbit takes? Work silently.}
\end{exampleblock}
\note{[THE CHALLENGE] Calculate satellite orbit like NASA?\\\\
[SAY] "Try on your own. Okay to get stuck."\\\\
[TIMING] 3-4 min SILENT individual work\\\\
[CIRCULATE] Note who converts altitude to orbital radius correctly\\\\
[WATCH FOR] Forgetting add Earth's radius to altitude\\\\
[DON'T HELP] Let struggle with unit conversions}
\end{frame}

\begin{frame}
\frametitle{Compare: Orbital Calculations}
\textbf{Turn and talk (2 min):}

\vspace{0.3cm}

\begin{enumerate}
\item What is $r_2$ for satellite? Did you add Earth's radius?
\item Which version Kepler's third law use?
\item Did you solve for $T_2$ correctly?
\end{enumerate}

\vspace{0.5cm}

\pause
\alert{Name wheel:} One pair share approach (not answer).
\note{[TIMING] 2-3 min pair discussion\\\\
[CIRCULATE] Listen common approaches\\\\
[CHECK] Name wheel: call pair share approach\\\\
[EXPECTED APPROACH] r-2 equals 1500 km plus 6380 km equals 7880 km, then use Kepler's third\\\\
[COMMON ERROR] Using altitude instead orbital radius from center}
\end{frame}

\begin{frame}
\frametitle{Reveal: Satellite Period}
\textbf{Self-correct in different color:}

\vspace{0.3cm}

\textbf{Step 1:} Find orbital radius: $r_2 = 1500 \text{ km} + 6380 \text{ km} = 7880 \text{ km}$

\pause
\vspace{0.2cm}

\textbf{Step 2:} Use Kepler's third law: $T_2 = T_1 \left(\frac{r_2}{r_1}\right)^{3/2}$

\pause
\vspace{0.2cm}

\textbf{Step 3:} Substitute and solve:
$$T_2 = (27.3 \text{ d}) \left(\frac{24.0 \text{ h}}{\text{d}}\right) \left(\frac{7880 \text{ km}}{3.84 \times 10^5 \text{ km}}\right)^{3/2}$$

\pause
$$\boxed{T_2 = 1.93 \text{ h}}$$

\pause
\textbf{Check:} Closer orbit = shorter period. Makes sense!
\note{[P0] "Self-correct in different color"\\\\
[P1] [ALGEBRA] "r-2 equals altitude plus Earth radius equals 7880 km"\\\\
[P2] "T-2 equals T-1 times r-2 over r-1 to three-halves power"\\\\
[P3] "Convert days to hours, then calculate"\\\\
[P4] [ANSWER] "1.93 hours - about 2 hours per orbit"\\\\
[THE WONDER] GPS satellites use exact math - just did orbital mechanics}
\end{frame}

\section{Newton's Law of Universal Gravitation}

\begin{frame}
\frametitle{Learning Objectives}
\begin{block}{By the end of this section, you will be able to:}
\begin{itemize}
\item \textbf{7.2:} Explain Newton's law of universal gravitation \pause
\item \textbf{7.2:} Perform calculations using Newton's law \pause
\item \textbf{7.2:} Compare Newton's theory to Einstein's general relativity
\end{itemize}
\end{block}
\note{[P0] "Three objectives universal gravitation"\\\\
[P1] "First: understand Newton's law"\\\\
[P2] "Second: calculate gravitational forces"\\\\
[P3] "Third: understand Einstein's improvement"\\\\
- Newton explained WHY Kepler's patterns exist}
\end{frame}

\begin{frame}
\frametitle{7.2 The Apple and Moon}
\begin{figure}
\centering
\includegraphics[width=0.6\textwidth,height=0.45\textheight,keepaspectratio]{phys12-gravitation-fig07.jpg}
\end{figure}

\pause
\begin{exampleblock}{Newton's Insight}
Why do apples fall straight down? What if same force pulling apples also pulls moon?
\end{exampleblock}

\pause
\textbf{Revolutionary idea:} Terrestrial and celestial motion have SAME cause!
\note{[P0] "Legendary apple story has truth"\\\\
[P1] [THE HOOK] "Newton asked: why apples fall straight down? What pulls them?"\\\\
[P2] [THE REVELATION] "Same force pulls apples AND keeps moon in orbit"\\\\
[THE WONDER] One force explains everything from weight to galaxies\\\\
- Before Newton: people thought heavenly bodies followed different rules\\\\
[THE CONNECTION - Digital Archetype] "One physics engine runs entire universe"}
\end{frame}

\begin{frame}
\frametitle{7.2 Newton's Law of Universal Gravitation}
\begin{block}{The Universal Law}
\begin{center}
\Large
$$\boxed{F = G\frac{mM}{r^2}}$$
\end{center}
Every mass attracts every other mass with force proportional to product of masses and inversely proportional to distance squared.
\end{block}

\pause
\vspace{0.3cm}

\textbf{Where:}
\begin{itemize}
\item $F$ = gravitational force (N)
\item $G = 6.67 \times 10^{-11}$ N·m²/kg² (universal constant)
\item $m, M$ = masses (kg)
\item $r$ = distance between centers (m)
\end{itemize}
\note{[P0] "Newton's law universal gravitation - source code"\\\\
[P1] "F equals G times m M over r squared"\\\\
- G gravitational constant - same everywhere universe\\\\
- Force ALWAYS attractive (never repulsive)\\\\
- Acts along line joining centers\\\\
[THE REVELATION] One equation explains falling objects AND planetary orbits\\\\
[THE WONDER] Universal means works everywhere - Earth, Mars, distant galaxies}
\end{frame}

\begin{frame}
\frametitle{7.2 The Gravitational Constant}
\begin{center}
\includegraphics[width=0.6\textwidth,height=0.4\textheight,keepaspectratio]{phys12-gravitation-fig09.jpg}

\small Cavendish's torsion balance (1798)
\end{center}

\pause
\textbf{Challenge:} Measuring tiny force between ordinary masses

\pause
\vspace{0.3cm}

\textbf{Result:} $G = 6.67 \times 10^{-11}$ N·m²/kg²

\pause
Cavendish's value differs less than 1\% from modern value!
\note{[P0] "G measured by Henry Cavendish 1798"\\\\
[P1] "Challenge: gravitational force between small masses incredibly weak"\\\\
[P2] "Used torsion balance measure tiny twist in fiber"\\\\
[P3] "Remarkably accurate - within 1 percent modern value"\\\\
[THE HUMILITY] Took over 100 years after Newton measure G\\\\
[THE WONDER] Determines strength one of four fundamental forces}
\end{frame}

\begin{frame}
\frametitle{7.2 Mass vs Weight}
\begin{alertblock}{Civilian View vs Reality}
\textbf{Civilian:} "Mass and weight same thing."\\
\textbf{Physicist:} "Mass constant. Weight varies with gravity."
\end{alertblock}

\pause
\vspace{0.3cm}

\textbf{Mass:} Amount of matter (kg) - constant everywhere

\pause
\textbf{Weight:} Gravitational force (N) - varies by location
$$W = mg$$

\pause
\vspace{0.3cm}

Your mass on moon same as Earth. Your weight 1/6 as much!
\note{[P0] [THE CONFLICT] "People confuse mass and weight"\\\\
[P1] "Mass amount matter - doesn't change"\\\\
[P2] "Weight force of gravity - changes with g"\\\\
[P3] "60 kg person weighs 588 N Earth, only 98 N moon"\\\\
[THE CONNECTION - Kinetic Archetype] "Athletes: jump 6 times higher moon - same leg strength, less weight"\\\\
[THE REVELATION] English units confuse this - pounds measure force, not mass}
\end{frame}

\begin{frame}
\frametitle{7.2 Connecting to Kepler}
\textbf{Newton derived Kepler's third law from universal gravitation!}

\pause
\vspace{0.3cm}

\textbf{Result:}
$$\frac{r^3}{T^2} = \frac{GM}{4\pi^2}$$

\pause
\vspace{0.3cm}

\begin{exampleblock}{The Revelation}
Kepler said WHAT (pattern). Newton said WHY (gravitational force causes it).
\end{exampleblock}

\pause
\textbf{Power:} If you know $r$ and $T$, can calculate mass $M$ of parent body!
\note{[P0] "Newton proved Kepler's laws follow from universal gravitation"\\\\
[P1] "Derived constant r-cubed over T-squared"\\\\
[P2] [THE REVELATION] "Kepler described pattern. Newton explained cause"\\\\
[P3] "Application: measure orbital radius and period, calculate mass planet or star"\\\\
[THE WONDER] Measured masses distant planets this way\\\\
- How we know mass sun, planets, even black holes}
\end{frame}

\begin{frame}
\frametitle{Attempt: Weight on Mars}
\begin{exampleblock}{The Challenge (3 min, silent)}
Value of $g$ on Mars is 3.71 m/s². You have mass 60.0 kg on Earth.

\vspace{0.3cm}

\textbf{Given:}
\begin{itemize}
\item Your mass: $m = 60.0$ kg
\item Mars gravity: $g_M = 3.71$ m/s²
\item Earth gravity: $g_E = 9.80$ m/s²
\end{itemize}

\textbf{Find:}
\begin{enumerate}
\item What is your mass on Mars?
\item What is your weight on Mars?
\end{enumerate}

\vspace{0.3cm}

\textit{How much would you weigh on Mars? Work silently.}
\end{exampleblock}
\note{[THE CHALLENGE] Calculate weight on another planet?\\\\
[SAY] "Think carefully mass versus weight"\\\\
[TIMING] 3-4 min SILENT individual work\\\\
[CIRCULATE] Note who remembers mass constant\\\\
[WATCH FOR] Students changing mass on Mars\\\\
[DON'T HELP] Let work through confusion}
\end{frame}

\begin{frame}
\frametitle{Compare: Mass and Weight}
\textbf{Turn and talk (2 min):}

\vspace{0.3cm}

\begin{enumerate}
\item Does mass change when go Mars? Why or why not?
\item What formula use for weight on Mars?
\item How Mars weight compare Earth weight?
\end{enumerate}

\vspace{0.5cm}

\pause
\alert{Name wheel:} One pair share reasoning.
\note{[TIMING] 2-3 min pair discussion\\\\
[CIRCULATE] Listen understanding mass vs weight\\\\
[CHECK] Name wheel: call pair share\\\\
[EXPECTED APPROACH] Mass stays 60 kg, weight equals m times g-M\\\\
[COMMON ERROR] Thinking mass changes different planets}
\end{frame}

\begin{frame}
\frametitle{Reveal: Mars Weight}
\textbf{Self-correct in different color:}

\vspace{0.3cm}

\textbf{Part 1:} Mass on Mars = \textbf{60.0 kg} (mass constant!)

\pause
\vspace{0.3cm}

\textbf{Part 2:} Weight on Mars: $W_M = mg_M$

\pause
$$W_M = (60.0 \text{ kg})(3.71 \text{ m/s}^2)$$

\pause
$$\boxed{W_M = 223 \text{ N}}$$

\pause
\vspace{0.3cm}

\textbf{Compare:} Earth weight = $(60.0)(9.80) = 588$ N

\pause
Mars weight about 38\% Earth weight!
\note{[P0] "Self-correct in different color"\\\\
[P1] "Part 1: mass constant - still 60 kg"\\\\
[P2] [ALGEBRA] "W-M equals m times g-M"\\\\
[P3] "60 kg times 3.71 m/s squared"\\\\
[P4] [ANSWER] "223 N on Mars"\\\\
[P5] "Compare: 588 N Earth, so Mars about 38 percent"\\\\
[THE WONDER] Mars weigh less but mass wouldn't change\\\\
[THE CONNECTION - Kinetic Archetype] "Athletes: jump 2.6 times higher Mars"}
\end{frame}

\begin{frame}
\frametitle{7.2 Einstein's General Relativity}
\textbf{Newton's theory works perfectly most situations...}

\pause
\vspace{0.3cm}

\textbf{But Einstein showed gravity more than just force:}

\pause
\begin{itemize}
\item Gravity warps space and time \pause
\item Light bends in gravitational fields \pause
\item Massive objects drag space-time with them
\end{itemize}

\pause
\vspace{0.3cm}

\begin{exampleblock}{The Mental Model}
Newton: Gravity tug of war between masses.\\
Einstein: Gravity bending of space-time itself.
\end{exampleblock}
\note{[P0] "Newton's theory works everyday situations"\\\\
[P1] "But Einstein discovered gravity even stranger"\\\\
[P2] "Gravity warps space and time - not separate"\\\\
[P3] "Light bends around massive objects"\\\\
[P4] "Spinning objects drag space-time with them"\\\\
[P5] [THE REVELATION] "Newton: force between masses. Einstein: curved space-time"\\\\
[THE WONDER] GPS satellites must correct general relativity to work\\\\
[THE HUMILITY] Even Newton's brilliant theory needed refinement}
\end{frame}

\begin{frame}
\frametitle{7.2 Light Bending Around Sun}
\begin{figure}
\centering
\includegraphics[width=0.7\textwidth,height=0.5\textheight,keepaspectratio]{phys12-gravitation-fig11.jpg}
\end{figure}

\pause
\textbf{1919 solar eclipse:} Starlight bent exactly as Einstein predicted!

\pause
Made Einstein scientific celebrity overnight.
\note{[P0] "Einstein's first major test: 1919 solar eclipse"\\\\
[P1] "Starlight passing near sun bent - stars appeared shifted"\\\\
[P2] "Amount bending matched Einstein's prediction perfectly"\\\\
[THE REVELATION] Light follows curved space-time around massive objects\\\\
- Revolutionary - light affected by gravity\\\\
- Made Einstein world famous\\\\
[THE WONDER] Use this effect detect distant planets and dark matter}
\end{frame}

\begin{frame}
\frametitle{Attempt: Earth's Gravity at Moon Distance}
\begin{exampleblock}{The Challenge (3 min, silent)}
Find acceleration due Earth's gravity at moon's distance.

\vspace{0.3cm}

\textbf{Given:}
\begin{itemize}
\item $G = 6.67 \times 10^{-11}$ N·m²/kg²
\item Earth mass: $M = 5.98 \times 10^{24}$ kg
\item Earth-moon distance: $r = 3.84 \times 10^8$ m
\end{itemize}

\textbf{Find:} Value $g$ at moon's distance from Earth

\vspace{0.3cm}

\textit{How weak is Earth's pull at that distance? Work silently.}
\end{exampleblock}
\note{[THE CHALLENGE] Calculate how gravity weakens with distance?\\\\
[SAY] "Use g equals G M over r squared"\\\\
[TIMING] 3-4 min SILENT individual work\\\\
[CIRCULATE] Note who remembers formula\\\\
[WATCH FOR] Unit conversions and scientific notation\\\\
[DON'T HELP] Let struggle with exponents}
\end{frame}

\begin{frame}
\frametitle{Compare: Gravity Calculation}
\textbf{Turn and talk (2 min):}

\vspace{0.3cm}

\begin{enumerate}
\item What formula use for $g$?
\item How handle scientific notation?
\item How this $g$ compare Earth's surface $g = 9.8$ m/s²?
\end{enumerate}

\vspace{0.5cm}

\pause
\alert{Name wheel:} One pair share approach.
\note{[TIMING] 2-3 min pair discussion\\\\
[CIRCULATE] Listen formula g equals G M over r squared\\\\
[CHECK] Name wheel: call pair share\\\\
[EXPECTED APPROACH] Substitute values, calculate scientific notation\\\\
[COMMON ERROR] Forgetting square distance}
\end{frame}

\begin{frame}
\frametitle{Reveal: Gravity at Moon Distance}
\textbf{Self-correct in different color:}

\vspace{0.3cm}

\textbf{Formula:} $g = G\frac{M}{r^2}$

\pause
\vspace{0.2cm}

\textbf{Substitute:}
$$g = \left(6.67 \times 10^{-11} \frac{\text{N·m}^2}{\text{kg}^2}\right) \left(\frac{5.98 \times 10^{24} \text{ kg}}{(3.84 \times 10^8 \text{ m})^2}\right)$$

\pause
\vspace{0.2cm}

$$g = \frac{3.99 \times 10^{14}}{1.47 \times 10^{17}}$$

\pause
$$\boxed{g = 2.70 \times 10^{-3} \text{ m/s}^2}$$

\pause
\textbf{Compare:} About 0.03\% Earth's surface gravity!
\note{[P0] "Self-correct in different color"\\\\
[P1] [ALGEBRA] "g equals G M over r squared"\\\\
[P2] "Substitute values - watch exponents"\\\\
[P3] "Divide: 3.99 times 10 to 14 over 1.47 times 10 to 17"\\\\
[P4] [ANSWER] "2.70 times 10 to negative 3 m/s squared"\\\\
[P5] "Only 0.03 percent surface gravity"\\\\
[THE WONDER] Earth's pull weak moon distance, but still there\\\\
- Moon's own gravity (1.62 m/s²) dominates at surface}
\end{frame}

\section{Summary}

\begin{frame}
\frametitle{What You Now Know}
\begin{block}{The Revelations}
\begin{enumerate}
\item Kepler found three laws describing orbital motion \pause
\item Newton discovered gravitational force explains Kepler's laws \pause
\item Universal gravitation: $F = Gm_1m_2/r^2$ \pause
\item Mass constant, weight varies with $g$ \pause
\item Einstein showed gravity warps space-time itself \pause
\item Same force governs falling apples and orbiting galaxies
\end{enumerate}
\end{block}
\note{[P0] "Six revelations today"\\\\
[P1] "Kepler found patterns planetary motion"\\\\
[P2] "Newton found cause: gravitational force"\\\\
[P3] "Universal gravitation works everywhere"\\\\
[P4] "Mass constant, weight varies"\\\\
[P5] "Einstein refined Newton's theory"\\\\
[P6] "One force shapes cosmos"\\\\
[THE WONDER] Now understand force holds galaxies together\\\\
- Name wheel: which most mind-bending?}
\end{frame}

\begin{frame}[shrink]
\frametitle{Key Equations}
\begin{align}
\text{Kepler's Third Law: } \frac{T_1^2}{T_2^2} &= \frac{r_1^3}{r_2^3} \\
\text{Universal Gravitation: } F &= G\frac{mM}{r^2} \\
\text{Gravitational Constant: } G &= 6.67 \times 10^{-11} \text{ N·m}^2/\text{kg}^2 \\
\text{Weight: } W &= mg \\
\text{Surface Gravity: } g &= G\frac{M}{r^2} \\
\text{Orbital Constant: } \frac{r^3}{T^2} &= \frac{GM}{4\pi^2}
\end{align}
\note{- Six key equations this chapter\\\\
- Kepler's third law: compare orbital periods\\\\
- Universal gravitation: force between any two masses\\\\
- Weight: force gravity on object\\\\
- Surface gravity: calculate g any planet\\\\
- Orbital constant: derive from Newton's law\\\\
- Know when use each equation}
\end{frame}

\begin{frame}
\frametitle{Homework}
\begin{center}
\Large
Complete assigned problems\\[0.3cm]
posted on LMS
\end{center}
\note{[SAY] "Homework posted LMS"\\\\
[TIMING] Due date: check LMS\\\\
[CHECK] Questions before end?\\\\
[TRANSITION] Next class: Applications gravitation - satellites and tides\\\\
[THE WONDER] Can now calculate orbits like NASA engineers}
\end{frame}

\end{document}
