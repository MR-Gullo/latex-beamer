\documentclass{beamer}
\usepackage{../../../shared/templates/ds9_theme}
\usepackage{../../../shared/templates/semantic-physics-colors}
\usepackage[overridenote]{pdfpc}
\graphicspath{{../images/}{../../shared/images/}}

\title[Momentum]{PHYS12 CH:8 The Physics of Collisions}
\subtitle{Why Crashes Change Everything}
\author[Mr. Gullo]{Mr. Gullo}
\date[January 2025]{January 2025}

\begin{document}

\frame{\titlepage
\note{[THE HOOK] Today we decode the physics of crashes\\\\
- Football tackles, car accidents, rocket launches\\\\
- One concept explains them all: momentum\\\\
[THE WONDER] By end of class, you'll predict what happens when things collide\\\\
- This explains why airbags save lives}
}

\begin{frame}
\frametitle{Outline}
\tableofcontents
\end{frame}

\section{Introduction}

\begin{frame}
\frametitle{The Mystery of Collisions}
\begin{center}
\Large What if you could predict\\
exactly what happens when objects collide?
\end{center}

\pause
\vspace{0.5cm}
Football tackles. Car crashes. Rocket launches.

\pause
\vspace{0.3cm}
\alert{The same law governs them all.}
\note{[P0] "What if you could predict exactly what happens when objects collide?"\\\\
[P1] "Football tackles, car crashes, rocket launches"\\\\
[P2] [THE WONDER] "Same law governs them all. Today: the law of momentum conservation"\\\\
[THE CONNECTION - Kinetic Archetype] Athletes use this instinctively}
\end{frame}

\begin{frame}
\frametitle{The Great Exchange}
\begin{figure}
\centering
\includegraphics[width=0.8\textwidth,height=0.6\textheight,keepaspectratio]{phys12-momentum-fig01.jpg}
\caption{NFC defensive backs gang tackle AFC running back during 2006 Pro Bowl}
\end{figure}
\note{[Fig 8.1: Football tackle with 4 defenders on 1 runner] "Use this to show vector addition - multiple momentum vectors converging on one point. Ask: Why do multiple defenders tackle together? Because momentum adds as vectors."\\\\
- Players collide with tremendous force\\\\
- Momentum transfers from one to another\\\\
- Mass and velocity both matter\\\\
- This image shows the physics we'll decode today}
\end{frame}

\section{Linear Momentum and Impulse}

\begin{frame}
\frametitle{Learning Objectives}
\begin{block}{By the end of this section, you will be able to:}
\begin{itemize}
\item \textbf{8.1:} Describe momentum, impulse, and the impulse-momentum theorem \pause
\item \textbf{8.1:} Express Newton's second law in terms of momentum \pause
\item \textbf{8.1:} Solve problems using the impulse-momentum theorem
\end{itemize}
\end{block}
\note{[P0] "Three objectives for momentum and impulse"\\\\
[P1] "First: what is momentum and how does it change"\\\\
[P2] "Second: connect to Newton's second law"\\\\
[P3] "Third: apply to real collisions"\\\\
- This is foundational for everything else}
\end{frame}

\begin{frame}
\frametitle{8.1 Mass in Motion}
\begin{block}{The Universal Law: \mom{Momentum}}
\begin{center}
\Large $\boxed{\vec{\mom{p}} = \mass{m}\vec{\vel{v}}}$
\end{center}
\mom{Momentum} equals \mass{mass} times \vel{velocity}. The quantity of motion.
\end{block}

\pause
\vspace{0.3cm}

\textbf{Key insights:}
\begin{itemize}
\item Directly proportional to \mass{mass} and \vel{velocity} \pause
\item Vector - same direction as \vel{velocity} \pause
\item SI unit: $\text{kg}\cdot\text{m/s}$
\end{itemize}
\note{[P0] [THE REVELATION] "Momentum is mass in motion - p equals m v"\\\\
[P1] "Greater mass or greater speed means greater momentum"\\\\
[P2] "Vector quantity - direction matters"\\\\
[P3] "Units: kilograms times meters per second"\\\\
[THE WONDER] Newton called this the quantity of motion}
\end{frame}

\begin{frame}
\frametitle{8.1 The Civilian's Mistake}
\begin{alertblock}{Civilian View vs. Reality}
\textbf{Civilian:} "\vel{Speed} is all that matters in a collision."\\
\textbf{Physicist:} "\mass{Mass} matters just as much as \vel{velocity}."
\end{alertblock}

\pause
\vspace{0.3cm}

\begin{exampleblock}{The Mental Model}
A slow-moving truck can have more \mom{momentum} than a fast-moving bicycle.\\
Why? \mass{Mass} wins.
\end{exampleblock}
\note{[P0] [THE CONFLICT] "Civilians think only speed matters"\\\\
[P1] [THE CONNECTION - Kinetic Archetype] "Think about getting hit by a slow truck versus a fast tennis ball"\\\\
- Truck wins every time\\\\
- Both mass AND velocity matter\\\\
[THE HUMILITY] Our intuition misleads us}
\end{frame}

\begin{frame}
\frametitle{8.1 Newton's Hidden Truth}
\begin{block}{The Original Second Law}
\begin{center}
\Large $\boxed{\vec{\force{F}}_{\text{net}} = \frac{\Delta\vec{\mom{p}}}{\Delta \tvar{t}}}$
\end{center}
Net \force{force} equals the rate of change of \mom{momentum}.
\end{block}

\pause
\vspace{0.3cm}

This is actually MORE fundamental than $\force{F} = \mass{m}\accel{a}$.

\pause
\vspace{0.3cm}

\textbf{Why?} This version works even when \mass{mass} changes (like rockets burning fuel).
\note{[P0] [THE REVELATION] "Newton actually wrote his second law in terms of momentum"\\\\
[P1] "F equals m a is just a special case for constant mass"\\\\
[P2] "This version works for rockets, which lose mass as they burn fuel"\\\\
[THE WONDER] The universe speaks in momentum, not acceleration}
\end{frame}

\begin{frame}
\frametitle{8.1 The Force of Time}
\begin{block}{The Universal Law: Impulse}
\begin{center}
\Large $\boxed{\mom{J} = \vec{\force{F}}_{\text{net}}\Delta \tvar{t} = \Delta\vec{\mom{p}}}$
\end{center}
Impulse equals \force{force} times \tvar{time} equals change in \mom{momentum}.
\end{block}

\pause
\vspace{0.3cm}

\textbf{Key insight:} A small \force{force} over long \tvar{time} $=$ Large \force{force} over short \tvar{time}

\pause
\vspace{0.3cm}

\begin{exampleblock}{In the Real World}
Airbags increase collision \tvar{time} $\rightarrow$ decrease \force{force} on your body.
\end{exampleblock}
\note{[P0] [THE REVELATION] "Impulse: J equals F times delta t equals change in momentum"\\\\
[P1] "Same momentum change can happen with small force over long time or large force over short time"\\\\
[P2] "Airbags work by increasing the collision time, which decreases the force"\\\\
[THE CONNECTION - All students] This is why airbags save lives}
\end{frame}

\begin{frame}
\frametitle{8.1 Engineering Life-Savers}
\begin{figure}
\centering
\includegraphics[width=0.7\textwidth,height=0.5\textheight,keepaspectratio]{phys12-momentum-fig02.jpg}
\caption{Airbags and seat belts installed in vehicles}
\end{figure}

\pause
\textbf{Physics:} $\vec{\force{F}}_{\text{net}} = \frac{\Delta\vec{\mom{p}}}{\Delta \tvar{t}}$

Increase $\Delta \tvar{t}$ $\rightarrow$ Decrease $\vec{\force{F}}_{\text{net}}$
\note{[Fig 8.2: Deployed airbag with person in seatbelt] "Critical teaching moment: Airbag extends collision time from ~5ms (dashboard) to ~30ms. Ask students to calculate force reduction ratio (6x safer). Connect to F=Δp/Δt algebraically."\\\\
[P0] "Airbags and crumple zones: pure impulse-momentum theorem"\\\\
[P1] "Momentum change is the same either way. But by increasing the time, we decrease the force"\\\\
- Rearrange equation: F equals delta p over delta t\\\\
- Bigger delta t means smaller F\\\\
[THE WONDER] Physics saves thousands of lives every year}
\end{frame}

\begin{frame}
\frametitle{8.1 Why Bend Your Knees?}
\begin{alertblock}{The Challenge}
Your friend dares you to jump off a bench without bending your knees.\\
Why is this foolish?
\end{alertblock}

\pause
\vspace{0.3cm}

\textbf{Answer:} Bending knees increases $\Delta \tvar{t}$, which decreases $\force{F}$.

\pause
\vspace{0.3cm}

\begin{exampleblock}{The Mental Model}
Stiff legs: short collision \tvar{time} $\rightarrow$ HUGE \force{force} on bones.\\
Bent knees: long collision \tvar{time} $\rightarrow$ smaller \force{force}, no injury.
\end{exampleblock}
\note{[P0] [THE CHALLENGE] "Why bend your knees when landing?"\\\\
[P1] "Bending knees increases the time of impact"\\\\
[P2] [THE CONNECTION - Kinetic Archetype] "Athletes know this instinctively - gymnasts, basketball players"\\\\
- Same momentum change, but spread over longer time\\\\
- Smaller force means no broken bones}
\end{frame}

\begin{frame}
\frametitle{Attempt: The Football Player}
\begin{exampleblock}{The Challenge (3 min, silent)}
A 110 kg football player runs at $8$ m/s.

\vspace{0.3cm}

\textbf{Given:}
\begin{itemize}
\item $\mass{m} = 110$ kg
\item $\vel{v} = 8$ m/s
\end{itemize}

\textbf{Find:} \mom{Momentum} $\vec{\mom{p}}$

\vspace{0.3cm}

\textit{Can you calculate the quantity of motion? Work silently.}
\end{exampleblock}
\note{[THE CHALLENGE] Can you calculate momentum?\\\\
[SAY] "Try this on your own. It's okay to get stuck."\\\\
[TIMING] 3 min SILENT individual work\\\\
[CIRCULATE] Watch for unit errors\\\\
[DON'T HELP] Let them struggle with the equation}
\end{frame}

\begin{frame}
\frametitle{Compare: Momentum Strategy}
\textbf{Turn and talk (2 min):}

\vspace{0.3cm}

\begin{enumerate}
\item What equation did you use?
\item Did you include units?
\item Is \mom{momentum} a scalar or vector?
\end{enumerate}

\vspace{0.5cm}

\pause
\alert{Name wheel:} One pair share your approach (not your answer).
\note{[TIMING] 2-3 min pair discussion\\\\
[CIRCULATE] Listen for p equals m v\\\\
[CHECK] Name wheel: call a pair to share\\\\
[EXPECTED APPROACH] p equals m times v, plug in 110 and 8\\\\
[COMMON ERROR] Forgetting units or direction}
\end{frame}

\begin{frame}
\frametitle{Reveal: The Quantity of Motion}
\textbf{Self-correct in a different color:}

\vspace{0.3cm}

\textbf{Equation:} $\vec{\mom{p}} = \mass{m}\vec{\vel{v}}$

\pause
\vspace{0.2cm}

\textbf{Substitute:} $\vec{\mom{p}} = (110 \text{ kg})(8 \text{ m/s})$

\pause
\vspace{0.2cm}

$$\boxed{\vec{\mom{p}} = 880 \text{ kg}\cdot\text{m/s}}$$

\pause
\textbf{Check:} Large \mass{mass}, moderate \vel{speed} $\rightarrow$ large \mom{momentum}. Reasonable!
\note{[P0] "Self-correct in a different color"\\\\
[P1] [ALGEBRA] "p equals m times v"\\\\
[P2] "110 kilograms times 8 meters per second"\\\\
[P3] [ANSWER] "880 kilogram meters per second - this is the player's momentum"\\\\
[THE WONDER] This much momentum takes a lot of force to stop}
\end{frame}

\begin{frame}
\frametitle{Attempt: Venus Williams' Serve}
\begin{exampleblock}{The Challenge (4 min, silent)}
Venus Williams hits a 0.057 kg tennis ball. It accelerates from rest to $58$ m/s in $5$ ms.

\vspace{0.3cm}

\textbf{Given:}
\begin{itemize}
\item $\mass{m} = 0.057$ kg
\item $\vel{v_i} = 0$ m/s, $\vel{v_f} = 58$ m/s
\item $\Delta \tvar{t} = 5 \times 10^{-3}$ s
\end{itemize}

\textbf{Find:} Average \force{force} on ball

\textit{Can you decode the power of this serve?}
\end{exampleblock}
\note{[THE CHALLENGE] Can you find the force from the racquet?\\\\
[SAY] "This is harder. Use impulse-momentum theorem."\\\\
[TIMING] 4 min SILENT work\\\\
[CIRCULATE] Watch for delta p calculation first\\\\
[WATCH FOR] Students trying to use F equals m a without finding a first}
\end{frame}

\begin{frame}
\frametitle{Compare: Impulse Strategy}
\textbf{Turn and talk (2 min):}

\vspace{0.3cm}

\begin{enumerate}
\item Did you find $\Delta \mom{p}$ first or jump straight to \force{force}?
\item What's the relationship between impulse and \mom{momentum}?
\item How did you handle the milliseconds?
\end{enumerate}

\vspace{0.5cm}

\pause
\alert{Name wheel:} One pair share your approach.
\note{[TIMING] 2-3 min pair discussion\\\\
[CIRCULATE] Listen for two-step approach\\\\
[EXPECTED APPROACH] First find delta p equals m times delta v, then F equals delta p over delta t\\\\
[COMMON ERROR] Not converting milliseconds to seconds}
\end{frame}

\begin{frame}
\frametitle{Reveal: The Power Serve}
\textbf{Self-correct in a different color:}

\textbf{Step 1 - Change in \mom{momentum}:}
$$\Delta\vec{\mom{p}} = \mass{m}(\vec{\vel{v}}_f - \vec{\vel{v}}_i) = (0.057)(58 - 0) = 3.3 \text{ kg}\cdot\text{m/s}$$

\pause

\textbf{Step 2 - \force{Force} from impulse:}
$$\vec{\force{F}}_{\text{net}} = \frac{\Delta\vec{\mom{p}}}{\Delta \tvar{t}} = \frac{3.3}{5 \times 10^{-3}}$$

\pause

$$\boxed{\vec{\force{F}}_{\text{net}} = 660 \text{ N}}$$

\pause
\textbf{Check:} About 150 pounds of force - that's a powerful serve!
\note{[P0] "Two-step solution"\\\\
[P1] [ALGEBRA] "Delta p equals m times delta v equals 3.3"\\\\
[P2] "F equals delta p over delta t"\\\\
[P3] [ANSWER] "660 Newtons - about 150 pounds of force"\\\\
[THE WONDER] Professional athletes generate forces we can barely imagine}
\end{frame}

\section{Conservation of Momentum}

\begin{frame}
\frametitle{Learning Objectives}
\begin{block}{By the end of this section, you will be able to:}
\begin{itemize}
\item \textbf{8.2:} Describe the law of conservation of momentum
\end{itemize}
\end{block}
\note{- One objective but it's the most important law in collisions\\\\
- This law lets us predict what happens when things crash\\\\
- Works for football, billiards, rocket launches\\\\
- Foundation for next section on elastic and inelastic collisions}
\end{frame}

\begin{frame}
\frametitle{8.2 The Universe's Accounting System}
\begin{block}{The Universal Law: Conservation of \mom{Momentum}}
For an isolated system:
\begin{center}
\Large $\boxed{\vec{\mom{p}}_{\text{tot}} = \text{constant}}$
\end{center}
or
\begin{center}
\Large $\boxed{\vec{\mom{p}}_{\text{before}} = \vec{\mom{p}}_{\text{after}}}$
\end{center}
Total \mom{momentum} is conserved.
\end{block}

\pause
\textbf{Isolated system:} Net external \force{force} is zero.
\note{[P0] [THE REVELATION] "Conservation of momentum: total momentum is constant"\\\\
[P1] "Works only for isolated systems - no net external force"\\\\
[THE WONDER] This is one of the deepest laws in physics\\\\
- Momentum can transfer between objects\\\\
- But the total never changes}
\end{frame}

\begin{frame}
\frametitle{8.2 Two Cars Colliding}
\begin{figure}
\centering
\includegraphics[width=0.8\textwidth,height=0.5\textheight,keepaspectratio]{phys12-momentum-fig03.jpg}
\caption{Car $m_1$ bumps into car $m_2$. Total momentum conserved.}
\end{figure}

\pause
\textbf{Before:} $\vec{\mom{p}}_1 + \vec{\mom{p}}_2 = \mass{m_1}\vec{\vel{v}}_1 + \mass{m_2}\vec{\vel{v}}_2$

\textbf{After:} $\vec{\mom{p}}'_1 + \vec{\mom{p}}'_2 = \mass{m_1}\vec{\vel{v}}'_1 + \mass{m_2}\vec{\vel{v}}'_2$

\textbf{Conservation:} $\mass{m_1}\vec{\vel{v}}_1 + \mass{m_2}\vec{\vel{v}}_2 = \mass{m_1}\vec{\vel{v}}'_1 + \mass{m_2}\vec{\vel{v}}'_2$
\note{[Fig 8.4: Before/after diagram of two cars, showing velocity vectors v₁, v₂, v₁', v₂'] "Use this to introduce vector subscript notation systematically. Trace momentum transfer visually: car 1's lost momentum = car 2's gained momentum. Have students identify which car slows/speeds by comparing vector lengths."\\\\
- Car 1 slows down, loses momentum\\\\
- Car 2 speeds up, gains momentum\\\\
- Total momentum stays the same\\\\
- This assumes friction is negligible\\\\
- Works because forces are internal to the two-car system}
\end{frame}

\begin{frame}
\frametitle{8.2 Why Momentum Seems to Vanish}
\begin{alertblock}{The Illusion}
A football player runs into the goalpost and bounces backward.\\
His \mom{momentum} changed! Is \mom{momentum} conserved?
\end{alertblock}

\pause
\vspace{0.3cm}

\textbf{Answer:} Expand the system to include Earth.

\pause
\vspace{0.3cm}

\begin{exampleblock}{The Mental Model}
Earth recoils backward (imperceptibly) when you push the goalpost.\\
Player's \mom{momentum} transfers to the entire planet.
\end{exampleblock}
\note{[P0] [THE CONFLICT] "Momentum seems to disappear when player hits goalpost"\\\\
[P1] "Include Earth in the system - momentum is conserved!"\\\\
[P2] "Earth recoils backward, but it's so massive we don't notice"\\\\
[THE HUMILITY] It's always possible to find a larger system where momentum is conserved}
\end{frame}

\begin{frame}
\frametitle{8.2 Figure Skating and Angular Momentum}
\begin{figure}
\centering
\includegraphics[width=0.7\textwidth,height=0.5\textheight,keepaspectratio]{phys12-momentum-fig04.jpg}
\caption{Ice skater spinning faster by pulling arms in}
\end{figure}

\pause
\textbf{Angular momentum:} $\vec{L} = I\vec{\angvel{\omega}}$

\textbf{Conservation:} $I_1\angvel{\omega_1} = I_2\angvel{\omega_2}$

Decrease $I$ (pull arms in) $\rightarrow$ Increase $\angvel{\omega}$ (spin faster)
\note{[Fig 8.5: Ice skater in two positions - arms extended (slow spin) vs arms pulled in (fast spin), showing L=Iω equations] "Kinesthetic learning opportunity: Have students stand and spin with arms out, then pull in. They'll feel the speed increase. Connect sensation to I₁ω₁=I₂ω₂. This builds intuition before algebra."\\\\
[P0] "Angular momentum is the rotational version of linear momentum"\\\\
[P1] "L equals I omega - moment of inertia times angular velocity"\\\\
- When skater pulls arms in, moment of inertia decreases\\\\
- So angular velocity must increase to conserve L\\\\
[THE WONDER] Same principle as linear momentum, just for rotation}
\end{frame}

\section{Elastic and Inelastic Collisions}

\begin{frame}
\frametitle{Learning Objectives}
\begin{block}{By the end of this section, you will be able to:}
\begin{itemize}
\item \textbf{8.3:} Distinguish between elastic and inelastic collisions \pause
\item \textbf{8.3:} Solve collision problems using conservation of momentum
\end{itemize}
\end{block}
\note{[P0] "Two objectives for collisions"\\\\
[P1] "First: understand the two types of collisions"\\\\
[P2] "Second: use momentum conservation to solve collision problems"\\\\
- This connects everything from this chapter}
\end{frame}

\begin{frame}
\frametitle{8.3 Two Types of Crashes}
\begin{block}{Elastic Collision}
Objects bounce off each other. \kenergy{Kinetic energy} is conserved.
\end{block}

\pause

\begin{block}{Inelastic Collision}
Objects stick together. \kenergy{Kinetic energy} is NOT conserved (converted to heat).
\end{block}

\pause
\vspace{0.3cm}

\begin{alertblock}{Key Insight}
\mom{Momentum} is ALWAYS conserved (if isolated).\\
\kenergy{Kinetic energy} is conserved ONLY in elastic collisions.
\end{alertblock}
\note{[P0] "Elastic: objects bounce, KE conserved"\\\\
[P1] "Inelastic: objects stick, KE lost as heat"\\\\
[P2] [THE REVELATION] "Momentum is always conserved. Energy conservation depends on collision type"\\\\
- Perfectly elastic collisions are rare\\\\
- Most real collisions are somewhere in between}
\end{frame}

\begin{frame}
\frametitle{8.3 Elastic Collision}
\begin{figure}
\centering
\includegraphics[width=0.8\textwidth,height=0.5\textheight,keepaspectratio]{phys12-momentum-fig05.jpg}
\caption{One-dimensional elastic collision}
\end{figure}

\pause
\textbf{Conservation of \mom{momentum}:}
$$\mass{m_1}\vec{\vel{v}}_1 + \mass{m_2}\vec{\vel{v}}_2 = \mass{m_1}\vec{\vel{v}}'_1 + \mass{m_2}\vec{\vel{v}}'_2$$

\pause
\textbf{Conservation of \kenergy{kinetic energy}:}
$$\frac{1}{2}\mass{m_1}\vel{v_1}^2 + \frac{1}{2}\mass{m_2}\vel{v_2}^2 = \frac{1}{2}\mass{m_1}\vel{v'_1}^2 + \frac{1}{2}\mass{m_2}\vel{v'_2}^2$$
\note{[Fig 8.6: Before/after elastic collision with two boxes on frictionless surface, showing v₁, v₂, v₁', v₂' vectors] "Emphasize frictionless surface label - this is WHY system is isolated. Point to velocity vectors reversing direction (classic elastic bounce). Use this to contrast with next slide's inelastic collision where vectors don't reverse."\\\\
[P0] "Elastic collision diagram"\\\\
[P1] "Momentum conserved: sum of momenta before equals sum after"\\\\
[P2] "Kinetic energy also conserved in elastic collision"\\\\
- Two equations, two unknowns\\\\
- Can solve for final velocities}
\end{frame}

\begin{frame}
\frametitle{8.3 Inelastic Collision}
\begin{figure}
\centering
\includegraphics[width=0.8\textwidth,height=0.5\textheight,keepaspectratio]{phys12-momentum-fig06.jpg}
\caption{Perfectly inelastic collision - objects stick together}
\end{figure}

\pause
\textbf{Conservation of \mom{momentum}:}
$$\mass{m_1}\vec{\vel{v}}_1 + \mass{m_2}\vec{\vel{v}}_2 = (\mass{m_1} + \mass{m_2})\vec{\vel{v}}'$$

\pause
\textbf{Key:} Final \vel{velocity} $\vec{\vel{v}}'$ is the same for both objects.
\note{[Fig 8.7: Before/after perfectly inelastic collision, equal-mass boxes stick together, v₁=-v₂ initially, v'=0 finally] "Perfect teaching example: equal masses, equal-opposite velocities → complete stop. Have students predict final velocity BEFORE showing equation. Special case makes conservation visceral: all KE converted to deformation."\\\\
[P0] "Inelastic collision: objects stick together"\\\\
[P1] "Momentum still conserved"\\\\
[P2] "Simplification: both objects move together with same final velocity"\\\\
- Easier equation than elastic\\\\
- Some KE converted to heat, sound, deformation}
\end{frame}

\begin{frame}
\frametitle{8.3 Memory Trick}
\begin{exampleblock}{Remember This}
\textbf{Elastic} materials are bouncy.\\
$\rightarrow$ Elastic collisions: objects bounce off each other.

\vspace{0.3cm}

\textbf{Inelastic} materials are sticky.\\
$\rightarrow$ Inelastic collisions: objects stick together.
\end{exampleblock}

\pause
\vspace{0.3cm}

\begin{alertblock}{Common Mistake}
Don't assume \kenergy{kinetic energy} is conserved! Only in elastic collisions.
\end{alertblock}
\note{- Memory trick helps with terminology\\\\
- Elastic = bouncy = separate after collision\\\\
- Inelastic = sticky = stick together\\\\
- Always check: is this elastic or inelastic?\\\\
- Determines which equations to use}
\end{frame}

\begin{frame}
\frametitle{Attempt: Hockey Goalie Catch}
\begin{exampleblock}{The Challenge (4 min, silent)}
A 70 kg goalie catches a 0.150 kg puck traveling at $35$ m/s. The goalie is initially at rest.

\vspace{0.3cm}

\textbf{Given:}
\begin{itemize}
\item $\mass{m_1} = 0.150$ kg, $\vel{v_1} = 35$ m/s
\item $\mass{m_2} = 70$ kg, $\vel{v_2} = 0$ m/s
\end{itemize}

\textbf{Find:} Final \vel{velocity} $\vel{v}'$ of goalie-plus-puck

\textit{Can you predict the recoil?}
\end{exampleblock}
\note{[THE CHALLENGE] Inelastic collision - goalie catches puck\\\\
[SAY] "This is an inelastic collision. What equation should you use?"\\\\
[TIMING] 4 min SILENT work\\\\
[CIRCULATE] Watch for correct equation choice\\\\
[WATCH FOR] Students using elastic collision equations}
\end{frame}

\begin{frame}
\frametitle{Compare: Collision Type}
\textbf{Turn and talk (2 min):}

\vspace{0.3cm}

\begin{enumerate}
\item Is this elastic or inelastic? How do you know?
\item Which conservation equation did you use?
\item What simplification happens because they stick together?
\end{enumerate}

\vspace{0.5cm}

\pause
\alert{Name wheel:} One pair share your reasoning.
\note{[TIMING] 2-3 min pair discussion\\\\
[CIRCULATE] Listen for inelastic identification\\\\
[EXPECTED APPROACH] Inelastic because they stick. Use m1 v1 plus m2 v2 equals (m1 plus m2) v-prime\\\\
[KEY INSIGHT] Both move together after collision}
\end{frame}

\begin{frame}
\frametitle{Reveal: The Recoil}
\textbf{Self-correct in a different color:}

\textbf{Conservation of \mom{momentum} (inelastic):}
$$\mass{m_1}\vel{v_1} + \mass{m_2}\vel{v_2} = (\mass{m_1} + \mass{m_2})\vel{v}'$$

\pause

\textbf{Substitute ($\vel{v_2} = 0$):}
$$\vel{v}' = \frac{\mass{m_1}\vel{v_1}}{\mass{m_1} + \mass{m_2}} = \frac{(0.150)(35)}{0.150 + 70}$$

\pause

$$\boxed{\vel{v}' = 0.075 \text{ m/s}}$$

\pause
\textbf{Check:} Tiny recoil - goalie is much more \mass{massive} than puck!
\note{[P0] "Inelastic collision equation"\\\\
[P1] [ALGEBRA] "v-prime equals m1 v1 over (m1 plus m2)"\\\\
[P2] "Plug in values"\\\\
[P3] [ANSWER] "0.075 meters per second - about 7 centimeters per second"\\\\
[THE WONDER] Goalie barely moves because his mass is 460 times the puck's mass}
\end{frame}

\begin{frame}
\frametitle{Attempt: Elastic Cart Collision}
\begin{exampleblock}{The Challenge (5 min, silent)}
Cart 1 (0.350 kg) moving at $2$ m/s collides with cart 2 (0.500 kg) moving at $-0.5$ m/s. After collision, cart 1 recoils at $-4$ m/s.

\vspace{0.3cm}

\textbf{Given:}
\begin{itemize}
\item $\mass{m_1} = 0.350$ kg, $\vel{v_1} = 2$ m/s, $\vel{v'_1} = -4$ m/s
\item $\mass{m_2} = 0.500$ kg, $\vel{v_2} = -0.5$ m/s
\end{itemize}

\textbf{Find:} Final \vel{velocity} $\vel{v'_2}$ of cart 2
\end{exampleblock}
\note{[THE CHALLENGE] Elastic collision - solve for unknown final velocity\\\\
[SAY] "Use conservation of momentum. This is the hardest one yet."\\\\
[TIMING] 5 min SILENT work\\\\
[CIRCULATE] Watch for sign errors with negative velocities}
\end{frame}

\begin{frame}
\frametitle{Compare: Momentum Algebra}
\textbf{Turn and talk (2 min):}

\vspace{0.3cm}

\begin{enumerate}
\item How did you handle the negative velocities?
\item Which momentum equation did you use?
\item What did you solve for?
\end{enumerate}

\vspace{0.5cm}

\pause
\alert{Name wheel:} One pair share your approach.
\note{[TIMING] 2-3 min pair discussion\\\\
[CIRCULATE] Listen for correct handling of signs\\\\
[EXPECTED APPROACH] m1 v1 plus m2 v2 equals m1 v1-prime plus m2 v2-prime, solve for v2-prime\\\\
[COMMON ERROR] Sign mistakes with negative velocities}
\end{frame}

\begin{frame}
\frametitle{Reveal: The Ricochet}
\textbf{Self-correct in a different color:}

\textbf{Conservation of \mom{momentum}:}
$$\mass{m_1}\vel{v_1} + \mass{m_2}\vel{v_2} = \mass{m_1}\vel{v'_1} + \mass{m_2}\vel{v'_2}$$

\pause

\textbf{Solve for $\vel{v'_2}$:}
$$\vel{v'_2} = \frac{\mass{m_1}\vel{v_1} + \mass{m_2}\vel{v_2} - \mass{m_1}\vel{v'_1}}{\mass{m_2}}$$

\pause

\textbf{Substitute:}
$$\vel{v'_2} = \frac{(0.350)(2) + (0.500)(-0.5) - (0.350)(-4)}{0.500} = 3.7 \text{ m/s}$$

\pause
\textbf{Check:} Positive \vel{velocity} means cart 2 moves to the right after collision.
\note{[P0] "Standard momentum conservation"\\\\
[P1] [ALGEBRA] "Rearrange to solve for v2-prime"\\\\
[P2] "Careful with signs: negative 0.5 and negative 4"\\\\
[P3] [ANSWER] "3.7 meters per second to the right"\\\\
[THE WONDER] You just predicted the outcome of a collision using only algebra}
\end{frame}

\begin{frame}
\frametitle{8.3 Two-Dimensional Collisions}
\begin{figure}
\centering
\includegraphics[width=0.7\textwidth,height=0.5\textheight,keepaspectratio]{phys12-momentum-fig07.jpg}
\caption{2D collision with $m_2$ initially at rest}
\end{figure}

\pause
\textbf{Strategy:} Break into components.

\textbf{$x$-direction:} $\mass{m_1}\vel{v_1} = \mass{m_1}\vel{v'_1}\cos\angle{\theta_1} + \mass{m_2}\vel{v'_2}\cos\angle{\theta_2}$

\textbf{$y$-direction:} $0 = \mass{m_1}\vel{v'_1}\sin\angle{\theta_1} + \mass{m_2}\vel{v'_2}\sin\angle{\theta_2}$
\note{[Fig 8.8: 2D collision with m₁ hitting stationary m₂, both scatter at angles θ₁ and θ₂] "Component method scaffolding: Draw x/y axes on board, have students identify which velocity vectors contribute to x-direction (cosine terms) vs y-direction (sine terms). Note m₂ initially at rest simplifies y-equation to zero."\\\\
[P0] "2D collisions are more complex"\\\\
[P1] "Break momentum into x and y components"\\\\
- Choose coordinate system wisely\\\\
- Momentum conserved in both directions\\\\
- Two equations, can solve for two unknowns}
\end{frame}

\section{Summary}

\begin{frame}[shrink]
\frametitle{What You Now Know}
\begin{block}{The Revelations}
\begin{enumerate}
\item \mom{Momentum} $\vec{\mom{p}} = \mass{m}\vec{\vel{v}}$ - \mass{mass} in motion \pause
\item Impulse $\mom{J} = \vec{\force{F}}\Delta \tvar{t} = \Delta\vec{\mom{p}}$ - \force{force} over \tvar{time} changes \mom{momentum} \pause
\item Newton's second law: $\vec{\force{F}} = \Delta\vec{\mom{p}}/\Delta \tvar{t}$ - the original form \pause
\item Conservation: $\vec{\mom{p}}_{\text{tot}} = \text{constant}$ in isolated systems \pause
\item Elastic collisions: objects bounce, \kenergy{KE} conserved \pause
\item Inelastic collisions: objects stick, \kenergy{KE} lost \pause
\item \mom{Momentum} ALWAYS conserved (if isolated)
\end{enumerate}
\end{block}
\note{[P0] "Seven revelations today"\\\\
[P1] "Momentum is mass in motion"\\\\
[P2] "Impulse connects force and momentum change"\\\\
[P3] "Newton actually wrote F equals delta p over delta t"\\\\
[P4] "Total momentum conserved in isolated systems"\\\\
[P5] "Elastic: bounce and KE conserved"\\\\
[P6] "Inelastic: stick and KE lost"\\\\
[P7] [THE WONDER] "Momentum conservation is one of the deepest laws in physics - explains everything from billiards to rocket science"}
\end{frame}

\begin{frame}
\frametitle{Key Equations}
\begin{align}
\vec{\mom{p}} &= \mass{m}\vec{\vel{v}} \\
\vec{\force{F}}_{\text{net}} &= \frac{\Delta\vec{\mom{p}}}{\Delta \tvar{t}} \\
\mom{J} &= \vec{\force{F}}_{\text{net}}\Delta \tvar{t} = \Delta\vec{\mom{p}} \\
\vec{\mom{p}}_{\text{tot}} &= \text{constant (isolated system)} \\
\mass{m_1}\vel{v_1} + \mass{m_2}\vel{v_2} &= \mass{m_1}\vel{v'_1} + \mass{m_2}\vel{v'_2} \text{ (elastic)} \\
\mass{m_1}\vel{v_1} + \mass{m_2}\vel{v_2} &= (\mass{m_1} + \mass{m_2})\vel{v}' \text{ (inelastic)}
\end{align}
\note{- Six equations to master\\\\
- First three: momentum, force, impulse\\\\
- Last three: conservation for different collision types\\\\
- Know when to use each\\\\
- Practice identifying elastic vs inelastic}
\end{frame}

\begin{frame}
\frametitle{Homework}
\begin{center}
\Large
Complete the assigned problems\\[0.3cm]
posted on the LMS
\end{center}
\note{[SAY] "Homework is posted on the LMS"\\\\
[TIMING] Due date: check LMS\\\\
[CHECK] Questions before we end?\\\\
[TRANSITION] Next class: Chapter 9 Work and Energy}
\end{frame}

\end{document}
