\documentclass[11pt]{article}
\usepackage[margin=0.7in]{geometry}
\usepackage{booktabs}
\usepackage{array}
\usepackage{tabularx}
\usepackage{enumitem}
\usepackage{titlesec}
\usepackage{hyperref}
\usepackage{xcolor}
\usepackage{colortbl}
\usepackage{tcolorbox}
\usepackage{microtype}
\usepackage{parskip}
\usepackage{fontawesome5}
\usepackage{graphicx}
\usepackage{multicol}
\usepackage{longtable}
\usepackage{background}
\usepackage{amssymb}

% Background image setup
\backgroundsetup{
    scale=0.4,
    opacity=0.08,
    angle=0,
    position=current page.center,
    contents={\includegraphics{media/overview_image1.jpeg}}
}

% Colors - MR-GULLO Brand
\definecolor{darkgray}{RGB}{37, 36, 34}
\definecolor{mediumgray}{RGB}{64, 61, 57}
\definecolor{offwhite}{RGB}{255, 252, 242}
\definecolor{accentorange}{RGB}{235, 94, 40}
\definecolor{lightgray}{RGB}{204, 197, 185}
\definecolor{headerblue}{RGB}{37, 36, 34}
% Row colors removed for transparency - background shows through tables
\definecolor{bcblue}{RGB}{0, 51, 102}

% tcolorbox styles
\tcbset{
    sharp corners,
    boxrule=0pt,
    left=8pt,
    right=8pt,
    top=6pt,
    bottom=6pt
}

% Section formatting
\titleformat{\section}
    {\large\bfseries\color{darkgray}}
    {}
    {0pt}
    {\raisebox{0pt}[0pt][0pt]{\textcolor{accentorange}{\rule[-2pt]{3pt}{14pt}}}\hspace{8pt}}
\titleformat{\subsection}{\normalsize\bfseries\color{darkgray}}{\thesubsection}{0.5em}{}
\titlespacing*{\section}{0pt}{2ex plus 1ex minus .2ex}{1ex plus .2ex}
\titlespacing*{\subsection}{0pt}{1.2ex plus 1ex minus .2ex}{0.5ex plus .2ex}

\setcounter{secnumdepth}{0}

% List styling
\setlist{nosep, leftmargin=1.5em}
\setlist[itemize]{label=\textcolor{lightgray}{\scriptsize$\blacktriangleright$}}
\setlist[enumerate]{label=\textcolor{accentorange}{\arabic*.}}

% Table styling
\renewcommand{\arraystretch}{1.4}
\newcolumntype{L}[1]{>{\raggedright\arraybackslash}p{#1}}
\newcolumntype{C}[1]{>{\centering\arraybackslash}p{#1}}
\newcolumntype{Y}{>{\raggedright\arraybackslash}X}

\hypersetup{colorlinks=true, linkcolor=accentorange, urlcolor=accentorange}

\begin{document}

% Title Header with Logos
\begin{tcolorbox}[colback=headerblue, colframe=headerblue, width=\textwidth, arc=0mm]
    \centering
    \begin{minipage}{0.2\textwidth}
        \centering
        \includegraphics[height=1.2cm]{media/overview_image3.png}
    \end{minipage}%
    \begin{minipage}{0.6\textwidth}
        \centering
        {\color{white}\Large\bfseries Shanghai Nanyang Model Private School}\\[0.2em]
        {\color{white!80}\small (BC OFFSHORE PROGRAM)}\\[0.5em]
        {\color{accentorange}\LARGE\bfseries PHYSICS 12 Annual Plan}\\[0.3em]
        {\color{white}\large 2025--2026}
    \end{minipage}%
    \begin{minipage}{0.2\textwidth}
        \centering
        \includegraphics[height=1.2cm]{media/overview_image4.jpeg}
    \end{minipage}
\end{tcolorbox}

\vspace{0.5em}
\small
\textbf{Link to Curriculum:} \url{https://curriculum.gov.bc.ca/sites/curriculum.gov.bc.ca/files/curriculum/science/en_science_12_physics_elab.pdf}

% Course Synopsis
\begin{tcolorbox}[colback=white, colframe=accentorange, leftrule=4pt, rightrule=0pt, toprule=0pt, bottomrule=0pt, opacityback=0.85]
\textbf{\faBook\hspace{0.5em}Course Synopsis}\\[0.3em]
The course starts from where Physics 11 left off with the study of motion. Students will explore dynamics, circular motion, energy, electricity and magnetism. The course requires a strong foundation in mathematics. The course emphasizes problem-solving and exploration.
\end{tcolorbox}

% Big Ideas
\section{Big Ideas}
\begin{tabularx}{\textwidth}{|X|X|X|X|}
\hline
\rowcolor{headerblue}\multicolumn{4}{|c|}{\textcolor{white}{\textbf{Big Ideas}}} \\
\hline
Measurement of motion depends on our frame of reference. &
Forces can cause linear and circular motion. &
Forces and energy interactions occur within fields. &
Momentum is conserved within a closed and isolated system. \\
\hline
\end{tabularx}

% Core Competencies, Curricular Competencies, Content
\section{Competencies \& Content}

\begin{tabularx}{\textwidth}{|L{0.28\textwidth}|L{0.38\textwidth}|L{0.28\textwidth}|}
\hline
\rowcolor{headerblue}
\textcolor{white}{\textbf{Core Competencies}} &
\textcolor{white}{\textbf{Curricular Competencies}} &
\textcolor{white}{\textbf{Content}} \\
\hline
\textbf{Communication}
\begin{itemize}[leftmargin=1em, nosep]
\item Connect and engage with others
\item Acquire, interpret, and present information
\item Collaborate to plan, carry out, and review activities
\item Explain/recount and reflect on experiences
\end{itemize}

\textbf{Creative Thinking}
\begin{itemize}[leftmargin=1em, nosep]
\item Novelty and value
\item Generating ideas
\item Developing ideas
\end{itemize}

\textbf{Critical Thinking}
\begin{itemize}[leftmargin=1em, nosep]
\item Analyze and critique
\item Question and investigate
\item Develop and design
\end{itemize}

\textbf{Personal \& Cultural Identity}
\begin{itemize}[leftmargin=1em, nosep]
\item Relationship and cultural contexts
\item Personal values and choice
\item Personal strengths and abilities
\end{itemize}

\textbf{Personal Awareness \& Responsibility}
\begin{itemize}[leftmargin=1em, nosep]
\item Self-determination
\item Self-regulation
\item Well-being
\end{itemize}

\textbf{Social Responsibility}
\begin{itemize}[leftmargin=1em, nosep]
\item Contributing to community
\item Solving problems peacefully
\item Valuing diversity
\item Building relationships
\end{itemize}
&
\textbf{Questioning and predicting}
\begin{itemize}[leftmargin=1em, nosep]
\item Demonstrate sustained intellectual curiosity
\item Make observations to identify questions
\item Formulate multiple hypotheses
\end{itemize}

\textbf{Planning and conducting}
\begin{itemize}[leftmargin=1em, nosep]
\item Plan and use appropriate investigation methods
\item Assess risks and address ethical issues
\item Use appropriate SI units and equipment
\item Apply accuracy and precision concepts
\end{itemize}

\textbf{Processing and analyzing}
\begin{itemize}[leftmargin=1em, nosep]
\item Experience and interpret local environment
\item Apply First Peoples perspectives
\item Seek patterns, trends, and connections
\item Construct, analyze, and interpret graphs
\item Draw evidence-based conclusions
\item Analyze cause-and-effect relationships
\end{itemize}

\textbf{Evaluating}
\begin{itemize}[leftmargin=1em, nosep]
\item Evaluate methods and identify sources of error
\item Describe ways to improve investigations
\item Evaluate validity and limitations of models
\item Demonstrate awareness of assumptions and bias
\item Connect scientific explorations to careers
\end{itemize}

\textbf{Applying and innovating}
\begin{itemize}[leftmargin=1em, nosep]
\item Contribute to care for self, others, community
\item Transfer and apply learning to new situations
\item Generate new ideas when problem solving
\end{itemize}

\textbf{Communicating}
\begin{itemize}[leftmargin=1em, nosep]
\item Formulate theoretical models
\item Communicate scientific ideas with evidence
\item Reflect on experiences and worldviews
\end{itemize}
&
\textbf{Students are expected to know:}
\begin{itemize}[leftmargin=1em, nosep]
\item Frames of reference
\item Relative motion within a stationary reference frame
\item Postulates of special relativity
\item Relativistic effects within a moving reference frame
\item Static equilibrium
\item Uniform circular motion: centripetal force and acceleration
\item Changes to apparent weight
\item First Peoples knowledge and applications of forces
\item Gravitational field and Newton's law of universal gravitation
\item Gravitational potential energy
\item Gravitational dynamics and energy relationships
\item Electric field and Coulomb's law
\item Electric potential energy, potential, and potential difference
\item Electrostatic dynamics and energy relationships
\item Magnetic field and magnetic force
\item Electromagnetic induction
\item Applications of electromagnetic induction
\item Impulse and momentum
\item Conservation of momentum and energy in collisions
\item Graphical methods in physics
\end{itemize}
\\
\hline
\end{tabularx}

\newpage

% English Language Strategies & Indigenous Learning
\section{English Language Strategies, Indigenous Learning, Timeline}

\begin{tabularx}{\textwidth}{|L{0.22\textwidth}|X|}
\hline
\textbf{English Language Strategies} &
\begin{itemize}[leftmargin=1em, nosep, topsep=2pt]
\item Vocabulary words highlighted and practiced
\item Assessments include vocabulary and language components
\item Large projects scaffolded with checkpoints
\item Oral speaking through discussions, think-pair-share, group work
\item Materials supported by high quality visuals
\item Lecture notes (animated PowerPoints) available online
\item Foster open and safe environment for speaking
\end{itemize}
\\
\hline
\textcolor{darkgray}{\textbf{Indigenous Learning}} &
\begin{itemize}[leftmargin=1em, nosep, topsep=2pt]
\item First People's principles embedded throughout course
\item Learning process: holistic, reflexive, reflective, experiential, relational
\item Focused on connectedness, reciprocal relationships, sense of place
\item Learning involves patience and time; learning is different for everyone
\end{itemize}
\\
\hline
\end{tabularx}

\vspace{1em}

% Timeline
\section{Timeline}
\begin{tabularx}{\textwidth}{|C{1.2cm}|X|C{2cm}|}
\hline
\rowcolor{headerblue}
\textcolor{white}{\textbf{Unit}} & \textcolor{white}{\textbf{Title}} & \textcolor{white}{\textbf{Month}} \\
\hline
1 & Introduction to Physics and Kinematics & September \\
\hline
2 & Dynamics: Forces and Newton's Laws & October \\
\hline
3 & Work, Energy and Power & November \\
\hline
4 & Momentum and Conservation Laws & December \\
\hline
5 & Circular Motion and Gravitation & February \\
\hline
6 & Electrostatics and Electric Fields & March \\
\hline
7 & Electric Circuits and Electromotive Force & April \\
\hline
8 & Magnetism and Electromagnetic Induction & May \\
\hline
9 & Special Relativity & June \\
\hline
\end{tabularx}

\vspace{1em}

% Summary of Assessment
\section{Summary of Assessment}
\begin{tabularx}{\textwidth}{|X|X|X|}
\hline
\rowcolor{headerblue}
\textcolor{white}{\textbf{Formative Assessments}} &
\textcolor{white}{\textbf{Self Evaluations}} &
\textcolor{white}{\textbf{Summative Assessments}} \\
\hline
\begin{itemize}[leftmargin=1em, nosep]
\item Circulating during conceptual questions
\item Gauging needs based on common homework questions
\item Verbal checks for understanding
\item Vocabulary: classroom challenge questions
\item Homework checks as needed
\item Demos/conversations
\end{itemize}
&
\begin{itemize}[leftmargin=1em, nosep]
\item During lectures: students try questions before teacher
\item Answer keys for pre-tests and tests for self-corrections
\item Core competency self reflections
\end{itemize}
&
\begin{itemize}[leftmargin=1em, nosep]
\item Unit tests
\item Midterm and Final Exams
\item Student submissions for activities and projects
\item Labs
\end{itemize}
\\
\hline
\end{tabularx}

\vspace{1em}

% Assessment Weighting
\section{Assessment Weighting}
\begin{center}
\begin{tabular}{|l|c|}
\hline
\rowcolor{headerblue}
\textcolor{white}{\textbf{Category}} & \textcolor{white}{\textbf{Weight}} \\
\hline
Quizzes & 15\% \\
\hline
Unit Tests & 30\% \\
\hline
Labs and Activities & 15\% \\
\hline
Homework & 10\% \\
\hline
Midterm & 10\% \\
\hline
Final Exam & 20\% \\
\hline
\end{tabular}
\end{center}

\newpage

% Unit Overviews
\section{Unit Overviews}

% Unit 1
\begin{tcolorbox}[colback=headerblue, colframe=headerblue, width=\textwidth, arc=0mm]
\centering{\color{white}\large\bfseries Unit 1: Kinematics}
\end{tcolorbox}
\vspace{-0.5em}
\begin{tabularx}{\textwidth}{|L{0.18\textwidth}|L{0.32\textwidth}|L{0.18\textwidth}|L{0.26\textwidth}|}
\hline
\cellcolor{headerblue}\textcolor{white}{\textbf{Big Idea(s)}} &
\cellcolor{headerblue}\textcolor{white}{\textbf{Core Competencies}} &
\cellcolor{headerblue}\textcolor{white}{\textbf{Content}} &
\cellcolor{headerblue}\textcolor{white}{\textbf{Activities}} \\
\hline
How can uniform motion and uniform acceleration be modelled?\newline
When are measurements considered to be relative?\newline
How is vector addition different from scalar addition? &
\textbf{Communication:} Lab report writing\newline
\textbf{Personal Awareness:} Set realistic goals, persevere\newline
\textbf{Creativity:} Design experiment to plot distance vs height\newline
\textbf{Critical Thinking:} Limits of scientific models &
\textbf{Vector/scalar:}\newline
- Addition and subtraction\newline
- Right-angle triangle\newline
\textbf{Uniform/accelerated motion:}\newline
- Graphical and quantitative\newline
\textbf{Projectile motion:}\newline
- Vertical, horizontal, angled launch &
\textbf{Assessments:} Quiz, unit tests, project, lab\newline
\textbf{ESL:} Review vocabulary\newline
\textbf{Indigenous:} As outlined above\newline
\textbf{Lab:} Water balloon launcher \\
\hline
\end{tabularx}

\vspace{1em}

% Unit 2
\begin{tcolorbox}[colback=headerblue, colframe=headerblue, width=\textwidth, arc=0mm]
\centering{\color{white}\large\bfseries Unit 2: Newton's Laws}
\end{tcolorbox}
\vspace{-0.5em}
\begin{tabularx}{\textwidth}{|L{0.18\textwidth}|L{0.32\textwidth}|L{0.18\textwidth}|L{0.26\textwidth}|}
\hline
\cellcolor{headerblue}\textcolor{white}{\textbf{Big Idea(s)}} &
\cellcolor{headerblue}\textcolor{white}{\textbf{Core Competencies}} &
\cellcolor{headerblue}\textcolor{white}{\textbf{Content}} &
\cellcolor{headerblue}\textcolor{white}{\textbf{Activities}} \\
\hline
How can forces change motion?\newline
How can Newton's laws explain changes in motion? &
\textbf{Communication:} Lab report writing\newline
\textbf{Personal Awareness:} Set realistic goals, persevere\newline
\textbf{Creativity:} Design water balloon launch experiment\newline
\textbf{Critical Thinking:} Consider alternative approaches &
\textbf{Contact forces:} normal, spring, tension, friction\newline
\textbf{Newton's laws of motion}\newline
\textbf{Forces in systems:}\newline
- One-body and multi-body systems\newline
- Inclined planes\newline
- Angled forces\newline
- Elevators &
\textbf{Assessments:} Quiz, unit tests, project\newline
\textbf{ESL:} Review vocabulary\newline
\textbf{Indigenous:} As outlined above\newline
\textbf{Lab:} Elevator acceleration experiment with scale \\
\hline
\end{tabularx}

\vspace{1em}

% Unit 3
\begin{tcolorbox}[colback=headerblue, colframe=headerblue, width=\textwidth, arc=0mm]
\centering{\color{white}\large\bfseries Unit 3: Equilibrium}
\end{tcolorbox}
\vspace{-0.5em}
\begin{tabularx}{\textwidth}{|L{0.18\textwidth}|L{0.32\textwidth}|L{0.18\textwidth}|L{0.26\textwidth}|}
\hline
\cellcolor{headerblue}\textcolor{white}{\textbf{Big Idea(s)}} &
\cellcolor{headerblue}\textcolor{white}{\textbf{Core Competencies}} &
\cellcolor{headerblue}\textcolor{white}{\textbf{Content}} &
\cellcolor{headerblue}\textcolor{white}{\textbf{Activities}} \\
\hline
When is an object in equilibrium?\newline
What are the implications for building structures? &
\textbf{Communication:} Lab report writing\newline
\textbf{Personal Awareness:} Set realistic goals, persevere\newline
\textbf{Creativity:} Design experiment to plot distance vs height\newline
\textbf{Critical Thinking:} Limits of scientific models &
\textbf{Static equilibrium:}\newline
- Translational: sum of all forces equals zero (vertical and horizontal)\newline
- Rotational: sum of all torques equals zero, location of centre of gravity of a uniform body &
\textbf{Assessments:} Quiz, unit tests, project, lab\newline
\textbf{ESL:} Review Grade 10 vocabulary\newline
\textbf{Indigenous:} As outlined above\newline
\textbf{Lab:} Torque lab \\
\hline
\end{tabularx}

\vspace{1em}

% Unit 4
\begin{tcolorbox}[colback=headerblue, colframe=headerblue, width=\textwidth, arc=0mm]
\centering{\color{white}\large\bfseries Unit 4: Uniform Circular Motion}
\end{tcolorbox}
\vspace{-0.5em}
\begin{tabularx}{\textwidth}{|L{0.18\textwidth}|L{0.32\textwidth}|L{0.18\textwidth}|L{0.26\textwidth}|}
\hline
\cellcolor{headerblue}\textcolor{white}{\textbf{Big Idea(s)}} &
\cellcolor{headerblue}\textcolor{white}{\textbf{Core Competencies}} &
\cellcolor{headerblue}\textcolor{white}{\textbf{Content}} &
\cellcolor{headerblue}\textcolor{white}{\textbf{Activities}} \\
\hline
Why do you feel a sideways sliding motion when you speed around a corner?\newline
Why must the ``orbiting electron'' model of the atom be false? &
\textbf{Communication:} Lab report writing\newline
\textbf{Personal Awareness:} Set realistic goals, persevere\newline
\textbf{Creativity:} Design a pendulum experiment\newline
\textbf{Critical Thinking:} Limits of scientific models &
\textbf{Uniform circular motion:} both horizontal and vertical circles\newline
\textbf{Changes to apparent weight:} vertical and horizontal circles &
\textbf{Assessments:} Quiz, unit tests, projects\newline
\textbf{ESL:} Review vocabulary\newline
\textbf{Indigenous:} As outlined above\newline
\textbf{Lab:} Pendulum Lab \\
\hline
\end{tabularx}

\newpage

% Unit 5
\begin{tcolorbox}[colback=headerblue, colframe=headerblue, width=\textwidth, arc=0mm]
\centering{\color{white}\large\bfseries Unit 5: Fields and Forces}
\end{tcolorbox}
\vspace{-0.5em}
\begin{tabularx}{\textwidth}{|L{0.18\textwidth}|L{0.32\textwidth}|L{0.18\textwidth}|L{0.26\textwidth}|}
\hline
\cellcolor{headerblue}\textcolor{white}{\textbf{Big Idea(s)}} &
\cellcolor{headerblue}\textcolor{white}{\textbf{Core Competencies}} &
\cellcolor{headerblue}\textcolor{white}{\textbf{Content}} &
\cellcolor{headerblue}\textcolor{white}{\textbf{Activities}} \\
\hline
Why is gravity considered a fundamental force?\newline
Explain similarities and differences between electrostatic and gravitational force.\newline
How are electric fields similar to magnetic and gravitational fields?\newline
What is the relationship between the moon orbiting Earth and an apple falling? &
\textbf{Communication:} Lab report writing\newline
\textbf{Personal Awareness:} Set realistic goals, persevere\newline
\textbf{Creativity:} Build skills to make ideas work\newline
\textbf{Critical Thinking:} Consider alternative approaches &
\textbf{Gravitational field:}\newline
- Vector field, interacts with mass\newline
- Attractive only\newline
- Gravitational dynamics: satellite motion, orbits, escape velocity\newline
\textbf{Electric field:}\newline
- Vector field, interacts with charge\newline
- Attractive or repulsive\newline
- Point charges and parallel plates\newline
\textbf{Magnetic field:}\newline
- Induced by moving charges\newline
- Permanent magnets, wires, solenoids\newline
\textbf{Magnetic force:}\newline
- On moving charge or current-carrying wire\newline
- Right-hand rules &
\textbf{Assessments:} Quiz, unit test, project\newline
\textbf{ESL:} Review vocabulary\newline
\textbf{Indigenous:} As outlined above\newline
\textbf{Activity:} 7E Phases for Science Fair topics \\
\hline
\end{tabularx}

\vspace{1em}

% Unit 6
\begin{tcolorbox}[colback=headerblue, colframe=headerblue, width=\textwidth, arc=0mm]
\centering{\color{white}\large\bfseries Unit 6: Energy in Fields and Interactions}
\end{tcolorbox}
\vspace{-0.5em}
\begin{tabularx}{\textwidth}{|L{0.18\textwidth}|L{0.32\textwidth}|L{0.18\textwidth}|L{0.26\textwidth}|}
\hline
\cellcolor{headerblue}\textcolor{white}{\textbf{Big Idea(s)}} &
\cellcolor{headerblue}\textcolor{white}{\textbf{Core Competencies}} &
\cellcolor{headerblue}\textcolor{white}{\textbf{Content}} &
\cellcolor{headerblue}\textcolor{white}{\textbf{Activities}} \\
\hline
How can a conductor and a magnet be used to generate electricity?\newline
How are electric fields similar to gravitational fields? &
\textbf{Communication:} Lab report writing\newline
\textbf{Personal Awareness:} Set realistic goals, persevere\newline
\textbf{Creativity:} Build skills to make ideas work\newline
\textbf{Critical Thinking:} Consider alternative approaches &
\textbf{Gravitational dynamics:} satellite motion, orbit changes, launch/escape velocity\newline
\textbf{Electrostatic dynamics:}\newline
- Force, charge, distance relationships\newline
- 1D and 2D with other charges\newline
- In orbits, between parallel plates\newline
- Conservation of energy applications (CRT, mass spectrometer, particle accelerator)\newline
\textbf{Electromagnetic induction:}\newline
- Faraday's law, Lenz's law\newline
- Current induced by changing magnetic flux\newline
\textbf{Applications:} back EMF, DC motors, generators, transformers &
\textbf{Assessments:} Quizzes, unit test, Electricity Lab\newline
\textbf{ESL:} Review vocabulary\newline
\textbf{Indigenous:} As outlined above\newline
\textbf{Activity:} 7E Phases for Science Fair \\
\hline
\end{tabularx}

\vspace{1em}

% Unit 8
\begin{tcolorbox}[colback=headerblue, colframe=headerblue, width=\textwidth, arc=0mm]
\centering{\color{white}\large\bfseries Unit 8: Special Relativity}
\end{tcolorbox}
\vspace{-0.5em}
\begin{tabularx}{\textwidth}{|L{0.18\textwidth}|L{0.32\textwidth}|L{0.18\textwidth}|L{0.26\textwidth}|}
\hline
\cellcolor{headerblue}\textcolor{white}{\textbf{Big Idea(s)}} &
\cellcolor{headerblue}\textcolor{white}{\textbf{Core Competencies}} &
\cellcolor{headerblue}\textcolor{white}{\textbf{Content}} &
\cellcolor{headerblue}\textcolor{white}{\textbf{Activities}} \\
\hline
What are the implications of the theory of special relativity? &
\textbf{Communication:} Research project\newline
\textbf{Personal Awareness:} Set realistic goals, persevere\newline
\textbf{Creativity:} Build skills to make ideas work\newline
\textbf{Critical Thinking:} Consider alternative approaches &
\textbf{Relativistic effects:}\newline
- Changes in time\newline
- Changes in length\newline
- Changes in mass &
\textbf{Assessments:} Quiz, unit tests\newline
\textbf{ESL:} Review Grade 10 vocabulary\newline
\textbf{Indigenous:} As outlined above\newline
\textbf{Activity:} Research project \\
\hline
\end{tabularx}

\end{document}
