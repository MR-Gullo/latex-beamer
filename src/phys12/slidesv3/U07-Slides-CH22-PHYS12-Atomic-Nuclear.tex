\documentclass{beamer}
\usepackage{../../../shared/templates/ds9_theme}
\usepackage[overridenote]{pdfpc}
\graphicspath{{../images/}{../../shared/images/}}

% Title configuration
\title[Ch22: Atomic and Nuclear Physics]{Chapter 22: Atomic and Nuclear Physics}
\subtitle{The Structure of Matter and Nuclear Processes}
\author[Mr. Gullo]{Mr. Gullo}
\institute[NANMO]{NANMO Physics 12}
\date{Winter 2025}

\begin{document}

% Title page
\frame{\titlepage}

% Table of Contents
\begin{frame}
\frametitle{Outline}
\tableofcontents
\note{[P0] [THE HOOK] "Today we dive into the invisible architecture of reality"\\\\
- "We'll explore atoms smaller than light wavelengths"\\\\
- "And nuclear forces millions of times stronger than chemistry"\\\\
[THE CONNECTION - Digital] "Game engines simulate this physics"\\\\
[TIMING] Brief overview, 1 min}
\end{frame}

\section{22.1 The Structure of the Atom}

% Learning Objectives
\begin{frame}
\frametitle{Learning Objectives: 22.1}
\begin{block}{By the end of this section, you will be able to:}
\begin{itemize}
\item Describe Rutherford's experiment and his model of the atom \pause
\item Describe emission and absorption spectra of atoms \pause
\item Describe the Bohr model of the atom \pause
\item Calculate the energy of electrons when they change energy levels \pause
\item Calculate the frequency and wavelength of emitted photons \pause
\item Describe the quantum model of the atom
\end{itemize}
\end{block}
\note{[P0] "Six objectives for understanding atomic structure"\\\\
[P1] "First: Rutherford's gold foil shock"\\\\
[P2] "Spectra: atomic fingerprints"\\\\
[P3] "Bohr: quantized orbits"\\\\
[P4] "Energy calculations"\\\\
[P5] "Light from atoms"\\\\
[P6] "Quantum reality: probability clouds"\\\\
[THE WONDER] "From spectra to quantum mechanics in one section"}
\end{frame}

% Rutherford's Gold Foil Experiment
\begin{frame}
\frametitle{22.1 Rutherford's Gold Foil Experiment}
\begin{columns}
\column{0.5\textwidth}
\begin{figure}
\centering
\includegraphics[width=\linewidth,height=0.55\textheight,keepaspectratio]{phys12-atomic-nuclear-fig05-1.jpg}
\end{figure}

\column{0.5\textwidth}
\textbf{The Setup (1909):}
\begin{itemize}
\item Radioactive source shoots alpha particles
\item Thin gold foil target
\item Phosphorescent screen detects scattered particles
\end{itemize}
\pause

\textbf{Expected:} Slight deflection (plum pudding model)

\pause
\textbf{Observed:} Some particles bounced straight back!
\end{columns}
\note{[P0] [THE HOOK] "Imagine firing bullets at tissue paper and having them bounce back"\\\\
- "That's what shocked Rutherford"\\\\
[THE CONNECTION - Kinetic] "Like throwing a tennis ball at a curtain and having it ricochet"\\\\
[P1] "Expected: particles sail through"\\\\
- "Plum pudding: diffuse positive charge"\\\\
[P2] [THE CONFLICT] "1 in 8,000 bounced backward"\\\\
- "Rutherford: 'Most incredible event of my life'"\\\\
[THE WONDER] "This proved atoms are mostly empty space"}
\end{frame}

% Rutherford's Discovery
\begin{frame}
\frametitle{22.1 The Nuclear Atom}
\begin{block}{Rutherford's Revolutionary Conclusion}
\begin{center}
The atom has a tiny, dense nucleus\\
surrounded by mostly empty space
\end{center}
\end{block}

\pause
\begin{columns}
\column{0.5\textwidth}
\begin{figure}
\centering
\includegraphics[width=\linewidth,height=0.45\textheight,keepaspectratio]{phys12-atomic-nuclear-fig06-1.jpg}
\end{figure}

\column{0.5\textwidth}
\textbf{Key Facts:}
\begin{itemize}
\item Nucleus: $\sim 10^{-15}$ m
\item Atom: $\sim 10^{-10}$ m
\item Nucleus is 100,000 times smaller!
\item Density: $10^{15}$ g/cm$^3$
\end{itemize}
\end{columns}
\note{[P0] [THE REVELATION] "Most of the atom is vacuum"\\\\
- "If nucleus were a marble, atom would be football stadium"\\\\
[P1] "Scale comparison"\\\\
[ALGEBRA] "ten to the negative fifteen meters"\\\\
- "Nucleus: femtometer scale"\\\\
- "Atom: angstrom scale"\\\\
[THE WONDER] "You are 99.9999999999999 percent empty space"\\\\
[THE CONNECTION - Digital] "This emptiness is why wifi signals pass through you"}
\end{frame}

% Planetary Model
\begin{frame}
\frametitle{22.1 The Planetary Model}
\begin{figure}
\centering
\includegraphics[width=0.6\textwidth,height=0.5\textheight,keepaspectratio]{phys12-atomic-nuclear-fig07-1.jpg}
\end{figure}

\begin{exampleblock}{The Analogy}
Low-mass electrons orbit massive nucleus\\
like planets orbit the Sun
\end{exampleblock}

\pause
\begin{alertblock}{The Problem}
Classical physics predicts orbiting electrons should radiate energy\\
and spiral into the nucleus in $10^{-10}$ seconds!
\end{alertblock}
\note{[P0] "Rutherford proposed planetary model"\\\\
- "Coulomb force replaces gravity"\\\\
- "Beautiful analogy"\\\\
[P1] [THE CONFLICT] "But classical EM says this is impossible"\\\\
- "Accelerating charges radiate"\\\\
- "Atoms should collapse instantly"\\\\
[THE HUMILITY] "Rutherford knew his model was incomplete"\\\\
[THE WONDER] "The solution required quantum mechanics"}
\end{frame}

% Emission Spectra
\begin{frame}
\frametitle{22.1 Emission and Absorption Spectra}
\begin{columns}
\column{0.6\textwidth}
\begin{figure}
\centering
\includegraphics[width=\linewidth,height=0.4\textheight,keepaspectratio]{phys12-atomic-nuclear-fig09-1.jpg}
\end{figure}

\column{0.4\textwidth}
\textbf{The Mystery:}
\begin{itemize}
\item Heat a gas
\item It emits discrete wavelengths
\item Not continuous spectrum!
\end{itemize}
\end{columns}

\pause
\vspace{0.3cm}
\begin{block}{Each Element Has a Unique Spectral Fingerprint}
Iron, hydrogen, helium - all produce different line patterns
\end{block}
\note{[P0] "Spectroscopy: the quantum detective work"\\\\
- "Pass light through diffraction grating"\\\\
- "See discrete lines, not rainbow"\\\\
[THE CONNECTION - Harmonic] "Like a guitar string only playing certain notes"\\\\
[P1] [THE REVELATION] "Each element sings its own song"\\\\
- "Astronomers use this to identify stars"\\\\
[THE WONDER] "We know what distant galaxies are made of because of spectroscopy"}
\end{frame}

% Hydrogen Spectrum
\begin{frame}
\frametitle{22.1 The Hydrogen Spectrum}
\begin{figure}
\centering
\includegraphics[width=0.8\textwidth,height=0.4\textheight,keepaspectratio]{phys12-atomic-nuclear-fig10-3.jpg}
\end{figure}

\textbf{Three Series:}
\begin{itemize}
\item \textbf{Lyman series:} Ultraviolet (electrons drop to $n=1$)
\item \textbf{Balmer series:} Visible light (electrons drop to $n=2$)
\item \textbf{Paschen series:} Infrared (electrons drop to $n=3$)
\end{itemize}

\pause
\begin{exampleblock}{The Pattern}
The wavelengths follow precise mathematical relationships.\\
But why?
\end{exampleblock}
\note{[P0] "Hydrogen: simplest atom, richest spectrum"\\\\
- "UV, visible, IR emissions"\\\\
- "Named after discoverers"\\\\
[THE CONNECTION - Digital] "Fiber optics use specific wavelengths"\\\\
[P1] [THE CONFLICT] "Pattern was known for decades"\\\\
- "But no one could explain WHY"\\\\
- "Needed Bohr's genius"\\\\
[THE WONDER] "A single proton and electron create this complexity"}
\end{frame}

% Bohr Model
\begin{frame}
\frametitle{22.1 Bohr's Quantum Atom (1913)}
\begin{columns}
\column{0.5\textwidth}
\begin{figure}
\centering
\includegraphics[width=\linewidth,height=0.45\textheight,keepaspectratio]{phys12-atomic-nuclear-fig11-1.jpg}
\end{figure}

\column{0.5\textwidth}
\begin{block}{Bohr's Radical Ideas}
\begin{enumerate}
\item Only certain orbits allowed (quantized!)
\item Electrons don't radiate while in orbit
\item Energy emitted/absorbed when electron jumps
\end{enumerate}
\end{block}
\end{columns}

\pause
\vspace{0.3cm}
\begin{block}{Energy Levels for Hydrogen}
\Large
$$E_n = -\frac{13.6 \text{ eV}}{n^2} \quad (n=1,2,3,\ldots)$$
\end{block}
\note{[P0] [THE REVELATION] "Bohr said: orbits are quantized"\\\\
- "Like rungs on a ladder"\\\\
- "You can be on rung 1 or 2, but not 1.5"\\\\
[THE CONNECTION - Harmonic] "Standing waves in a loop - only certain frequencies fit"\\\\
[P1] "Ground state: n equals 1"\\\\
[ALGEBRA] "E-one equals negative 13.6 electron-volts"\\\\
- "Negative because electron is bound"\\\\
[THE WONDER] "This equation predicted hydrogen spectrum perfectly"}
\end{frame}

% Energy Level Diagram
\begin{frame}
\frametitle{22.1 Energy Level Diagram}
\begin{figure}
\centering
\includegraphics[width=0.6\textwidth,height=0.6\textheight,keepaspectratio]{phys12-atomic-nuclear-fig13-1.jpg}
\end{figure}

\begin{exampleblock}{Reading the Diagram}
\begin{itemize}
\item Arrow down: photon emitted
\item Arrow up: photon absorbed
\item Longer arrow: higher energy photon
\end{itemize}
\end{exampleblock}
\note{[P0] "Vertical axis: energy"\\\\
- "Horizontal lines: allowed states"\\\\
- "Ground state at bottom"\\\\
- "Zero energy: electron is free"\\\\
[THE CONNECTION - Digital] "LEDs work by electron transitions"\\\\
- "Different materials give different colors"\\\\
[THE WONDER] "Every photon tells a story of an electron's journey"}
\end{frame}

% Attempt: Energy Calculation
\begin{frame}
\frametitle{Attempt: Electron Energy Transition}
\textbf{Try this on your own (3 min, silent):}

A hydrogen atom absorbs a photon. The electron jumps from the ground state ($n=1$) to the third energy level ($n=3$).

\textbf{Given:}
\begin{itemize}
\item Initial state: $n_i = 1$
\item Final state: $n_f = 3$
\item $E_n = -\frac{13.6 \text{ eV}}{n^2}$
\end{itemize}

\textbf{Find:} How much energy must the photon have?

\vspace{0.3cm}
\textit{Work individually. Show your GUESS steps.}
\note{[THE CHALLENGE] "Can you find the photon energy?"\\\\
[TIMING] 3-4 min SILENT work\\\\
- NO HINTS - let them struggle\\\\
[CIRCULATE] Note who uses delta-E correctly\\\\
- Watch for sign errors\\\\
- Common mistake: forgetting it's absorption (positive delta-E)\\\\
[EXPECTED APPROACH] Find E-initial, E-final, then delta-E}
\end{frame}

% Compare: Energy Calculation
\begin{frame}
\frametitle{Compare: Energy Transition Approach}
\textbf{Turn and talk (2 min):}

\begin{enumerate}
\item What equation did you use first?
\item Did you get a positive or negative result?
\item What does the sign mean physically?
\end{enumerate}

\pause
\vspace{0.3cm}
\alert{Name wheel:} One pair share your approach (not your answer).
\note{[P0] "Pair discussion"\\\\
- Listen for: Do they find each E separately?\\\\
- Listen for: Sign confusion\\\\
[P1] [THE CONFLICT] "Some will get negative delta-E"\\\\
- "That means emission, not absorption!"\\\\
[CIRCULATE] Find a pair with correct approach\\\\
- And one with sign error\\\\
[EXPECTED APPROACH] E-final minus E-initial equals delta-E\\\\
- Positive: absorption\\\\
- Negative: emission}
\end{frame}

% Reveal: Energy Calculation
\begin{frame}
\frametitle{Reveal: Solution}
\textbf{Self-correct in a different color:}

\textbf{U - Unknown:} $\Delta E = ?$

\pause
\textbf{E - Equations:}
$$E_n = -\frac{13.6 \text{ eV}}{n^2}, \quad \Delta E = E_f - E_i$$

\pause
\textbf{S - Substitute:}
$$E_i = -\frac{13.6}{1^2} = -13.6 \text{ eV}$$
$$E_f = -\frac{13.6}{3^2} = -1.51 \text{ eV}$$

\pause
$$\Delta E = (-1.51) - (-13.6) = +12.09 \text{ eV}$$

\pause
$$\boxed{\Delta E = 12.1 \text{ eV}}$$

\textbf{Check:} Positive means absorption. Photon must provide energy!
\note{[P0] [ALGEBRA] "Unknown: delta-E"\\\\
[P1] "Two equations needed"\\\\
[P2] "E-initial: negative 13.6 eV"\\\\
- "E-final: negative 1.51 eV"\\\\
[P3] "Subtract: negative 1.51 minus negative 13.6"\\\\
- "Watch those signs!"\\\\
[P4] [ANSWER] 12.1 electron-volts\\\\
[THE WONDER] "This exact energy creates one specific UV wavelength"\\\\
- "Hydrogen's unique signature"}
\end{frame}

% Wavelength and Energy
\begin{frame}
\frametitle{22.1 Photon Energy and Wavelength}
\begin{block}{Energy-Frequency Relationship}
\Large
$$E = hf = \frac{hc}{\lambda}$$
\end{block}

\pause
\textbf{Where:}
\begin{itemize}
\item $h = 6.626 \times 10^{-34}$ J$\cdot$s (Planck's constant)
\item $c = 3.00 \times 10^8$ m/s (speed of light)
\item $f$ = frequency (Hz)
\item $\lambda$ = wavelength (m)
\end{itemize}

\pause
\begin{exampleblock}{The Connection}
Higher energy transition $\rightarrow$ shorter wavelength photon
\end{exampleblock}
\note{[P0] "Photon energy connects to wavelength"\\\\
[ALGEBRA] "E equals h-f equals h-c over lambda"\\\\
[P1] "Planck's constant: quantum of action"\\\\
- "Smallest possible angular momentum"\\\\
[P2] [THE REVELATION] "Energy and wavelength inversely related"\\\\
- "Blue photons more energetic than red"\\\\
[THE WONDER] "Light is both particle and wave"}
\end{frame}

% Rydberg Formula
\begin{frame}
\frametitle{22.1 The Rydberg Formula}
\begin{block}{Wavelength of Emitted Light}
\Large
$$\frac{1}{\lambda} = R\left(\frac{1}{n_f^2} - \frac{1}{n_i^2}\right)$$
\end{block}

\textbf{Where:}
\begin{itemize}
\item $R = 1.097 \times 10^7$ m$^{-1}$ (Rydberg constant)
\item $n_i$ = initial quantum number (higher)
\item $n_f$ = final quantum number (lower)
\end{itemize}

\pause
\begin{alertblock}{Historical Note}
Rydberg discovered this formula empirically in 1888.\\
Bohr explained why it works in 1913!
\end{alertblock}
\note{[P0] "This formula was magic until Bohr"\\\\
[ALGEBRA] "One over lambda equals R times difference of inverse squares"\\\\
- "Works perfectly for hydrogen"\\\\
[THE HUMILITY] "Rydberg found pattern but didn't know why"\\\\
[P1] [THE REVELATION] "Bohr derived it from first principles"\\\\
- "Proof that quantization is real"\\\\
[THE WONDER] "Math reveals nature's hidden rules"}
\end{frame}

% De Broglie Waves
\begin{frame}
\frametitle{22.1 De Broglie's Matter Waves}
\begin{columns}
\column{0.5\textwidth}
\begin{figure}
\centering
\includegraphics[width=\linewidth,height=0.5\textheight,keepaspectratio]{phys12-atomic-nuclear-fig22-1.jpg}
\end{figure}

\column{0.5\textwidth}
\begin{block}{De Broglie (1924)}
If light can be particles,\\
can particles be waves?
\end{block}

\pause
\textbf{Yes!}
$$\lambda = \frac{h}{p}$$

Only certain wavelengths fit in orbit $\rightarrow$ quantization!
\end{columns}
\note{[P0] [THE HOOK] "Matter waves: the ultimate twist"\\\\
- "Electron is both particle and wave"\\\\
[THE CONNECTION - Harmonic] "Standing wave on circular string"\\\\
- "Only integer wavelengths fit"\\\\
[P1] "Lambda equals h over p"\\\\
- "Momentum determines wavelength"\\\\
[THE WONDER] "This is WHY orbits are quantized"\\\\
- "Constructive interference only for certain orbits"}
\end{frame}

% Heisenberg Uncertainty
\begin{frame}
\frametitle{22.1 Heisenberg Uncertainty Principle}
\begin{block}{The Fundamental Limit}
\Large
$$\Delta x \Delta p \geq \frac{h}{4\pi}$$
\end{block}

\pause
\begin{alertblock}{What It Means}
You cannot simultaneously know exact position AND exact momentum.\\
This is not a measurement problem - it's reality!
\end{alertblock}

\pause
\begin{exampleblock}{Implication for Atoms}
Electrons don't have well-defined orbits.\\
They exist as probability clouds.
\end{exampleblock}
\note{[P0] [THE CONFLICT] "Your intuition says: know position and velocity"\\\\
- "Quantum says: impossible!"\\\\
[ALGEBRA] "Delta-x times delta-p greater-than-or-equal h over 4-pi"\\\\
[P1] "Not experimental limitation"\\\\
[THE HUMILITY] "Einstein fought this. He was wrong."\\\\
[P2] [THE REVELATION] "This destroys planetary model"\\\\
- "Replaces it with probability cloud"\\\\
[THE WONDER] "The universe is fundamentally uncertain"}
\end{frame}

% Quantum Model
\begin{frame}
\frametitle{22.1 The Quantum Model}
\begin{figure}
\centering
\includegraphics[width=0.5\textwidth,height=0.5\textheight,keepaspectratio]{phys12-atomic-nuclear-fig26-1.jpg}
\end{figure}

\begin{block}{Electron Cloud}
\begin{itemize}
\item Each dot: one position measurement
\item Darker region: higher probability
\item No defined orbit - only probability distribution
\end{itemize}
\end{block}
\note{[P0] "Modern view: electron cloud"\\\\
- "Probability density function"\\\\
- "Darker means more likely"\\\\
[THE CONNECTION - Digital] "Computer graphics use probability rendering"\\\\
[THE WONDER] "This is what atoms actually look like"\\\\
- "Not tiny solar systems"\\\\
- "Fuzzy clouds of possibility"}
\end{frame}

\section{22.2 Nuclear Forces and Radioactivity}

% Learning Objectives 22.2
\begin{frame}
\frametitle{Learning Objectives: 22.2}
\begin{block}{By the end of this section, you will be able to:}
\begin{itemize}
\item Describe the structure and forces present within the nucleus \pause
\item Explain the three types of radiation \pause
\item Write nuclear equations associated with radioactive decay
\end{itemize}
\end{block}
\note{[P0] "Now we zoom into the nucleus"\\\\
- "Ten thousand times smaller than atom"\\\\
[P1] "Strong force: most powerful in nature"\\\\
[P2] "Alpha, beta, gamma radiation"\\\\
[P3] "Nuclear equations conserve mass and charge"\\\\
[THE WONDER] "The energy here dwarfs chemistry"}
\end{frame}

% Nuclear Structure
\begin{frame}
\frametitle{22.2 The Nucleus}
\begin{columns}
\column{0.5\textwidth}
\begin{figure}
\centering
\includegraphics[width=\linewidth,height=0.5\textheight,keepaspectratio]{phys12-atomic-nuclear-fig31-1.jpg}
\end{figure}

\column{0.5\textwidth}
\textbf{Nucleons:}
\begin{itemize}
\item Protons: +1e charge
\item Neutrons: neutral
\item Mass $\approx$ 1 u (atomic mass unit)
\end{itemize}

\pause
\textbf{Notation:}
$$^A_Z X_N$$
\begin{itemize}
\item $Z$ = atomic number (protons)
\item $A$ = mass number (protons + neutrons)
\item $N = A - Z$ (neutrons)
\end{itemize}
\end{columns}
\note{[P0] "Nucleus: protons and neutrons"\\\\
- "Collectively: nucleons"\\\\
- "Nearly equal mass"\\\\
[THE CONNECTION - Kinetic] "Nucleus like tightly packed ball of marbles"\\\\
[P1] "Nuclide notation"\\\\
[ALGEBRA] "Lithium-7: three protons, four neutrons"\\\\
- "Mass number: A equals 7"\\\\
- "Atomic number: Z equals 3"\\\\
[THE WONDER] "Carbon-12 defines the atomic mass unit"}
\end{frame}

% Isotopes
\begin{frame}
\frametitle{22.2 Isotopes}
\begin{block}{Same Element, Different Neutrons}
Isotopes have same $Z$ (same element) but different $A$
\end{block}

\pause
\textbf{Hydrogen Isotopes:}
\begin{itemize}
\item $^1$H: 1 proton, 0 neutrons (hydrogen)
\item $^2$H: 1 proton, 1 neutron (deuterium)
\item $^3$H: 1 proton, 2 neutrons (tritium - radioactive)
\end{itemize}

\pause
\begin{exampleblock}{Chemistry vs. Physics}
\textbf{Chemistry:} Isotopes behave nearly identically (same electrons)\\
\textbf{Physics:} Isotopes have very different nuclear stability
\end{exampleblock}
\note{[P0] "Isotopes: same chemistry, different nucleus"\\\\
- "All have 1 proton: all are hydrogen"\\\\
[P1] "Deuterium: heavy water"\\\\
- "Tritium: radioactive, rare"\\\\
[P2] [THE REVELATION] "Neutrons don't affect chemistry"\\\\
- "But dramatically affect stability"\\\\
[THE WONDER] "Most elements have multiple stable isotopes"\\\\
- "Carbon-12 and carbon-13 both common"}
\end{frame}

% Nuclear Forces
\begin{frame}
\frametitle{22.2 The Strong Nuclear Force}
\begin{columns}
\column{0.5\textwidth}
\begin{block}{The Problem}
Protons repel via Coulomb force.\\
Why doesn't nucleus explode?
\end{block}

\pause
\begin{block}{The Solution}
\textbf{Strong nuclear force:}
\begin{itemize}
\item Attractive between nucleons
\item Much stronger than EM force
\item Very short range ($\sim 10^{-15}$ m)
\end{itemize}
\end{block}

\column{0.5\textwidth}
\begin{figure}
\centering
\includegraphics[width=\linewidth,height=0.5\textheight,keepaspectratio]{phys12-atomic-nuclear-fig33-1.jpg}
\end{figure}
\end{columns}
\note{[P0] [THE CONFLICT] "Protons hate each other"\\\\
- "Coulomb repulsion is enormous"\\\\
- "At femtometer distances: trillions of newtons"\\\\
[P1] [THE REVELATION] "Strong force is the glue"\\\\
- "100 times stronger than EM"\\\\
- "But only at contact"\\\\
[THE WONDER] "Without strong force: no atoms, no chemistry, no life"}
\end{frame}

% Radioactivity Discovery
\begin{frame}
\frametitle{22.2 Discovery of Radioactivity (1896)}
\begin{block}{Becquerel's Accident}
\begin{itemize}
\item Uranium ore placed on wrapped photographic plate
\item Plate darkened - even in complete darkness!
\item Uranium emitting invisible, penetrating rays
\end{itemize}
\end{block}

\pause
\begin{alertblock}{The Shock}
Energy emerging from matter without any input!\\
Apparent violation of energy conservation!
\end{alertblock}

\pause
\begin{exampleblock}{The Resolution}
Einstein's $E = mc^2$ explains it:\\
Mass converts to energy in nuclear reactions
\end{exampleblock}
\note{[P0] [THE HOOK] "Becquerel didn't plan this discovery"\\\\
- "Cloudy day, left ore in drawer"\\\\
- "Developed plate anyway: fogged!"\\\\
[P1] [THE CONFLICT] "Energy from nowhere?"\\\\
- "Violated known physics"\\\\
[THE HUMILITY] "Took years to understand"\\\\
[P2] [THE REVELATION] "Mass IS energy"\\\\
[THE WONDER] "Uranium slowly converting mass to energy for billions of years"}
\end{frame}

% Three Types of Radiation
\begin{frame}
\frametitle{22.2 Three Types of Nuclear Radiation}
\begin{block}{Alpha ($\alpha$) Radiation}
2 protons + 2 neutrons (helium nucleus)\\
Charge: +2e, Penetration: cm in air, stopped by paper
\end{block}

\pause
\begin{block}{Beta ($\beta$) Radiation}
Electron (or positron) from nucleus\\
Charge: $-$e (or +e), Penetration: m in air, stopped by aluminum
\end{block}

\pause
\begin{block}{Gamma ($\gamma$) Radiation}
High-energy photon\\
Charge: 0, Penetration: km in air, reduced by lead
\end{block}
\note{[P0] "Three flavors of radioactive decay"\\\\
[ALGEBRA] "Alpha: helium-4 nucleus"\\\\
- "Massive, highly charged"\\\\
- "Interacts strongly: short range"\\\\
[P1] "Beta: electron from neutron decay"\\\\
- "Less massive, less charged"\\\\
- "Penetrates further"\\\\
[P2] "Gamma: pure energy"\\\\
- "No charge, no mass"\\\\
- "Hardest to stop"\\\\
[THE WONDER] "Same nucleus can emit all three at different times"}
\end{frame}

% Alpha Decay
\begin{frame}
\frametitle{22.2 Alpha Decay}
\begin{columns}
\column{0.5\textwidth}
\begin{figure}
\centering
\includegraphics[width=\linewidth,height=0.5\textheight,keepaspectratio]{phys12-atomic-nuclear-fig35-1.jpg}
\end{figure}

\column{0.5\textwidth}
\begin{block}{Nuclear Equation}
$$^A_Z X_N \rightarrow ^{A-4}_{Z-2} Y_{N-2} + ^4_2 He_2$$
\end{block}

\pause
\textbf{Example:} Uranium-238
$$^{238}_{92} U \rightarrow ^{234}_{90} Th + ^4_2 He$$

\pause
\textbf{Conservation:}
\begin{itemize}
\item Mass number: 238 = 234 + 4 \checkmark
\item Charge: 92 = 90 + 2 \checkmark
\end{itemize}
\end{columns}
\note{[P0] "Alpha: eject helium nucleus"\\\\
- "Reduces A by 4"\\\\
- "Reduces Z by 2"\\\\
[THE CONNECTION - Digital] "Smoke detectors use alpha decay"\\\\
[P1] "Uranium to thorium transmutation"\\\\
- "Element changes!"\\\\
[P2] "Conservation laws sacred"\\\\
[ALGEBRA] "238 equals 234 plus 4"\\\\
[THE WONDER] "Alchemist's dream: elements changing"}
\end{frame}

% Beta Decay
\begin{frame}
\frametitle{22.2 Beta Decay}
\begin{columns}
\column{0.5\textwidth}
\begin{figure}
\centering
\includegraphics[width=\linewidth,height=0.5\textheight,keepaspectratio]{phys12-atomic-nuclear-fig36-1.jpg}
\end{figure}

\column{0.5\textwidth}
\begin{block}{$\beta^-$ Decay (most common)}
Neutron $\rightarrow$ proton + electron + antineutrino
$$^A_Z X_N \rightarrow ^A_{Z+1} Y_{N-1} + e^- + \bar{\nu}_e$$
\end{block}

\pause
\textbf{Example:} Carbon-14
$$^{14}_6 C \rightarrow ^{14}_7 N + e^- + \bar{\nu}_e$$
\end{columns}
\note{[P0] "Beta: neutron transforms"\\\\
- "Becomes proton, ejects electron"\\\\
- "Also releases neutrino (ghostly particle)"\\\\
[P1] "Carbon-14 to nitrogen-14"\\\\
- "Basis of radiocarbon dating"\\\\
[ALGEBRA] "Mass number unchanged: 14 equals 14"\\\\
- "Charge increases: 6 to 7"\\\\
[THE WONDER] "Fundamental particles can change identity"\\\\
- "Weak nuclear force at work"}
\end{frame}

% Gamma Decay
\begin{frame}
\frametitle{22.2 Gamma Decay}
\begin{columns}
\column{0.5\textwidth}
\begin{figure}
\centering
\includegraphics[width=\linewidth,height=0.5\textheight,keepaspectratio]{phys12-atomic-nuclear-fig37-1.jpg}
\end{figure}

\column{0.5\textwidth}
\begin{block}{Excited Nucleus}
Nucleus drops from excited state to ground state
$$^A_Z X^* \rightarrow ^A_Z X + \gamma$$
\end{block}

\pause
\textbf{Key Points:}
\begin{itemize}
\item $A$ and $Z$ unchanged
\item Pure energy release
\item Often follows $\alpha$ or $\beta$ decay
\item MeV energies (vs. eV for atoms)
\end{itemize}
\end{columns}
\note{[P0] "Gamma: nucleus de-excites"\\\\
- "Like electron dropping energy levels"\\\\
- "But million times more energetic"\\\\
[P1] "Element doesn't change"\\\\
- "Just releases energy"\\\\
[THE CONNECTION - Harmonic] "Nucleus vibrating, releases photon"\\\\
[THE WONDER] "Gamma rays from space tell us about supernova"}
\end{frame}

% Attempt: Nuclear Equation
\begin{frame}
\frametitle{Attempt: Write Nuclear Equation}
\textbf{Try this on your own (3 min, silent):}

Plutonium-239 undergoes alpha decay.

\textbf{Given:}
\begin{itemize}
\item Parent nucleus: $^{239}_{94}Pu$
\item Type: alpha decay
\item Periodic table for atomic numbers
\end{itemize}

\textbf{Find:}
\begin{enumerate}
\item Complete nuclear equation
\item Identity of daughter nucleus
\end{enumerate}

\vspace{0.3cm}
\textit{Remember to conserve mass number and charge!}
\note{[THE CHALLENGE] "Can you find the daughter nucleus?"\\\\
[TIMING] 3 min SILENT work\\\\
[CIRCULATE] Watch for:\\\\
- Do they subtract 4 from A?\\\\
- Do they subtract 2 from Z?\\\\
- Do they identify element Z equals 92?\\\\
[EXPECTED APPROACH] A: 239 minus 4 equals 235\\\\
- Z: 94 minus 2 equals 92\\\\
- Look up Z equals 92 on table}
\end{frame}

% Compare: Nuclear Equation
\begin{frame}
\frametitle{Compare: Nuclear Equation Strategy}
\textbf{Turn and talk (2 min):}

\begin{enumerate}
\item What did you subtract from $A$ and $Z$?
\item How did you identify the daughter element?
\item Did you check conservation?
\end{enumerate}

\pause
\vspace{0.3cm}
\alert{Name wheel:} One pair share their method.
\note{[P0] "Pair discussion"\\\\
[CIRCULATE] Listen for correct logic\\\\
- A decreases by 4\\\\
- Z decreases by 2\\\\
[P1] "Most should get Z equals 92"\\\\
- "But did they identify uranium?"\\\\
[EXPECTED APPROACH] Subtract alpha particle\\\\
- Use periodic table\\\\
- Verify conservation}
\end{frame}

% Reveal: Nuclear Equation
\begin{frame}
\frametitle{Reveal: Alpha Decay Solution}
\textbf{Self-correct in a different color:}

\textbf{General form:}
$$^A_Z X \rightarrow ^{A-4}_{Z-2} Y + ^4_2 He$$

\pause
\textbf{Apply to Pu-239:}
$$A_{daughter} = 239 - 4 = 235$$
$$Z_{daughter} = 94 - 2 = 92$$

\pause
\textbf{Identify element:} $Z = 92$ is Uranium (U)

\pause
$$\boxed{^{239}_{94}Pu \rightarrow ^{235}_{92}U + ^4_2He}$$

\pause
\textbf{Check:} 239 = 235 + 4 \checkmark, \quad 94 = 92 + 2 \checkmark
\note{[P0] [ALGEBRA] "A decreases by 4"\\\\
[P1] "239 minus 4 equals 235"\\\\
[P2] "Z equals 92: uranium"\\\\
[P3] [ANSWER] Plutonium to uranium plus helium\\\\
[P4] "Conservation verified"\\\\
[THE WONDER] "This is how plutonium bombs work"\\\\
- "Chain reaction of fission, not just decay"}
\end{frame}

\section{22.3 Half Life and Radiometric Dating}

% Learning Objectives 22.3
\begin{frame}
\frametitle{Learning Objectives: 22.3}
\begin{block}{By the end of this section, you will be able to:}
\begin{itemize}
\item Explain radioactive half-life and its role in radiometric dating \pause
\item Calculate radioactive half-life and solve problems associated with radiometric dating
\end{itemize}
\end{block}
\note{[P0] "Half-life: nuclear clock"\\\\
[P1] "Carbon dating, uranium dating"\\\\
[THE WONDER] "We can determine age of Earth, fossils, ancient artifacts"}
\end{frame}

% Half-Life Concept
\begin{frame}
\frametitle{22.3 What is Half-Life?}
\begin{figure}
\centering
\includegraphics[width=0.7\textwidth,height=0.5\textheight,keepaspectratio]{phys12-atomic-nuclear-fig45-1.jpg}
\end{figure}

\begin{block}{Definition}
\textbf{Half-life ($t_{1/2}$):} Time for half the nuclei to decay
\end{block}

\pause
\begin{itemize}
\item After $t_{1/2}$: $N \rightarrow N/2$
\item After $2t_{1/2}$: $N \rightarrow N/4$
\item After $3t_{1/2}$: $N \rightarrow N/8$
\end{itemize}
\note{[P0] [THE HOOK] "Exponential decay: nature's countdown"\\\\
- "Every half-life: divide by two"\\\\
[THE CONNECTION - Digital] "Battery life, game cooldowns use exponential decay"\\\\
[P1] "Geometric series"\\\\
- "Never reaches zero"\\\\
- "But becomes negligible"\\\\
[THE WONDER] "Some isotopes: microseconds. Uranium-238: 4.5 billion years"}
\end{frame}

% Exponential Decay
\begin{frame}
\frametitle{22.3 Exponential Decay Law}
\begin{block}{Number of Nuclei vs. Time}
\Large
$$N(t) = N_0 e^{-\lambda t}$$
\end{block}

\textbf{Where:}
\begin{itemize}
\item $N_0$ = initial number of nuclei
\item $N(t)$ = number remaining at time $t$
\item $\lambda$ = decay constant
\item $e = 2.71828\ldots$
\end{itemize}

\pause
\begin{block}{Relationship to Half-Life}
$$\lambda = \frac{\ln(2)}{t_{1/2}} \approx \frac{0.693}{t_{1/2}}$$
\end{block}
\note{[P0] "Exponential function: fundamental decay"\\\\
[ALGEBRA] "N-of-t equals N-zero e to the negative lambda-t"\\\\
- "Lambda: decay constant"\\\\
- "Large lambda: fast decay, short half-life"\\\\
[P1] "Decay constant from half-life"\\\\
- "Natural log of 2 over t-half"\\\\
[THE WONDER] "Same math describes drug elimination, atmospheric pressure"}
\end{frame}

% Activity
\begin{frame}
\frametitle{22.3 Activity (Rate of Decay)}
\begin{block}{Definition}
\textbf{Activity $R$:} Number of decays per unit time
$$R = \frac{\Delta N}{\Delta t} = \lambda N$$
\end{block}

\pause
\textbf{Units:}
\begin{itemize}
\item Becquerel (Bq): 1 decay/second (SI unit)
\item Curie (Ci): $3.7 \times 10^{10}$ decays/second (traditional)
\end{itemize}

\pause
\begin{exampleblock}{Physical Meaning}
More radioactive material $\rightarrow$ higher activity\\
As sample decays $\rightarrow$ activity decreases
\end{exampleblock}
\note{[P0] "Activity: how radioactive is it RIGHT NOW?"\\\\
[ALGEBRA] "R equals lambda times N"\\\\
- "Proportional to amount remaining"\\\\
[P1] "Becquerel: one decay per second"\\\\
- "Curie: named for Marie Curie"\\\\
- "Based on 1 gram of radium-226"\\\\
[THE WONDER] "Geiger counter clicks measure activity"}
\end{frame}

% Carbon Dating
\begin{frame}
\frametitle{22.3 Carbon-14 Dating}
\begin{block}{The Method}
\begin{enumerate}
\item Cosmic rays create $^{14}$C in atmosphere
\item Living organisms maintain constant $^{14}$C ratio
\item After death: no new $^{14}$C absorbed
\item $^{14}$C decays with $t_{1/2} = 5,730$ years
\item Measure remaining $^{14}$C to find age
\end{enumerate}
\end{block}

\pause
\begin{exampleblock}{Range}
Effective for ages: 100 to 50,000 years\\
(about 10 half-lives maximum)
\end{exampleblock}
\note{[P0] [THE HOOK] "How old is this bone?"\\\\
- "Carbon dating answers"\\\\
[THE CONNECTION - Kinetic] "Every breath you take contains C-14"\\\\
- "You are slightly radioactive!"\\\\
[P1] "While alive: C-14 replenished"\\\\
- "Death: clock starts"\\\\
[THE WONDER] "Shroud of Turin dated 1300 CE, not 33 CE"}
\end{frame}

% Shroud of Turin
\begin{frame}
\frametitle{22.3 Case Study: Shroud of Turin}
\begin{columns}
\column{0.5\textwidth}
\begin{figure}
\centering
\includegraphics[width=\linewidth,height=0.5\textheight,keepaspectratio]{phys12-atomic-nuclear-fig48-1.jpg}
\end{figure}

\column{0.5\textwidth}
\textbf{The Claim:}
Burial shroud of Jesus (33 CE)

\pause
\textbf{The Test (1988):}
\begin{itemize}
\item Three independent labs
\item Found 92\% of living $^{14}$C
\item Calculate age...
\end{itemize}

\pause
\textbf{The Result:}
Dated to $1320 \pm 60$ CE\\
Medieval, not ancient!
\end{columns}
\note{[P0] "Famous controversy"\\\\
- "Appeared in 1354 CE"\\\\
- "Claimed to be 1300 years older"\\\\
[P1] "Triple-blind test"\\\\
- "Three labs, didn't know which sample was shroud"\\\\
[P2] [THE REVELATION] "92 percent C-14 remaining"\\\\
- "Corresponds to ~700 years"\\\\
[THE WONDER] "Physics settled 700-year debate"}
\end{frame}

% Attempt: Half-Life Problem
\begin{frame}
\frametitle{Attempt: Half-Life Calculation}
\textbf{Try this on your own (3 min, silent):}

A radioactive sample has a half-life of 10 days. You start with 80 g of material.

\textbf{Given:}
\begin{itemize}
\item $N_0 = 80$ g
\item $t_{1/2} = 10$ days
\item Time elapsed: $t = 30$ days
\end{itemize}

\textbf{Find:} How much material remains after 30 days?

\vspace{0.3cm}
\textit{Think: How many half-lives have passed?}
\note{[THE CHALLENGE] "Can you think in half-lives?"\\\\
[TIMING] 3 min SILENT work\\\\
[CIRCULATE] Watch for:\\\\
- Do they divide 30 by 10?\\\\
- Do they repeatedly halve?\\\\
- Or use exponential formula?\\\\
[EXPECTED APPROACH] 30 days = 3 half-lives\\\\
- After 1: 40 g\\\\
- After 2: 20 g\\\\
- After 3: 10 g}
\end{frame}

% Compare: Half-Life Problem
\begin{frame}
\frametitle{Compare: Half-Life Strategy}
\textbf{Turn and talk (2 min):}

\begin{enumerate}
\item How many half-lives occurred?
\item Did you use repeated halving or a formula?
\item What's your final answer?
\end{enumerate}

\pause
\vspace{0.3cm}
\alert{Name wheel:} Share your approach.
\note{[P0] "Pair discussion"\\\\
[CIRCULATE] Two methods should emerge:\\\\
- Intuitive: divide by 2 three times\\\\
- Formula: N equals N-zero over 2 to the n\\\\
[P1] "Both valid!"\\\\
[EXPECTED APPROACH] Recognize 3 half-lives\\\\
- Divide by 8}
\end{frame}

% Reveal: Half-Life Problem
\begin{frame}
\frametitle{Reveal: Half-Life Solution}
\textbf{Self-correct in a different color:}

\textbf{Method 1 - Counting Half-Lives:}

Number of half-lives: $n = \frac{t}{t_{1/2}} = \frac{30}{10} = 3$

\pause
After each half-life:
\begin{itemize}
\item $t = 10$ days: $80 \div 2 = 40$ g
\item $t = 20$ days: $40 \div 2 = 20$ g
\item $t = 30$ days: $20 \div 2 = 10$ g
\end{itemize}

\pause
\textbf{Method 2 - Formula:}
$$N = N_0 \left(\frac{1}{2}\right)^n = 80 \left(\frac{1}{2}\right)^3 = 80 \times \frac{1}{8} = 10 \text{ g}$$

\pause
$$\boxed{N = 10 \text{ g}}$$
\note{[P0] "Two paths to same answer"\\\\
[P1] [ALGEBRA] "Method 1: intuitive"\\\\
- "Halve three times"\\\\
[P2] "Method 2: formula"\\\\
- "One-half to the n"\\\\
[P3] [ANSWER] 10 grams remaining\\\\
[THE WONDER] "After 10 half-lives: less than 0.1 percent left"}
\end{frame}

% Other Dating Methods
\begin{frame}
\frametitle{22.3 Other Radiometric Dating}
\begin{block}{Uranium-Lead Dating}
$^{238}$U $\rightarrow$ $^{206}$Pb, \quad $t_{1/2} = 4.5 \times 10^9$ years\\
Used for ancient rocks (oldest Earth rocks: 3.5 billion years)
\end{block}

\pause
\begin{block}{Potassium-Argon Dating}
$^{40}$K $\rightarrow$ $^{40}$Ar, \quad $t_{1/2} = 1.25 \times 10^9$ years\\
Used for volcanic rocks, human fossils
\end{block}

\pause
\begin{exampleblock}{The Power}
Different isotopes cover different time scales:\\
Years, millennia, millions of years, billions of years
\end{exampleblock}
\note{[P0] "Different isotopes for different timescales"\\\\
- "C-14: thousands of years"\\\\
- "U-238: billions of years"\\\\
[P1] "Uranium dating: oldest tool"\\\\
- "Age of Earth: 4.54 billion years"\\\\
[P2] "Potassium dating: human evolution"\\\\
[THE WONDER] "We know Earth's age to eight significant figures"}
\end{frame}

\section{22.4 Nuclear Fission and Fusion}

% Learning Objectives 22.4
\begin{frame}
\frametitle{Learning Objectives: 22.4}
\begin{block}{By the end of this section, you will be able to:}
\begin{itemize}
\item Explain nuclear fission \pause
\item Explain nuclear fusion \pause
\item Describe how fission and fusion work in weapons and power generation
\end{itemize}
\end{block}
\note{[P0] "Nuclear energy: million times more powerful than chemical"\\\\
[P1] "Fission: splitting heavy nuclei"\\\\
[P2] "Fusion: combining light nuclei"\\\\
[P3] "Both power civilization and threaten it"\\\\
[THE WONDER] "Stars are fusion reactors. We're trying to build our own"}
\end{frame}

% Binding Energy Curve
\begin{frame}
\frametitle{22.4 Nuclear Binding Energy}
\begin{figure}
\centering
\includegraphics[width=0.7\textwidth,height=0.5\textheight,keepaspectratio]{phys12-atomic-nuclear-fig58-1.jpg}
\end{figure}

\begin{block}{The Key Insight}
Iron-56 has highest binding energy per nucleon\\
$\rightarrow$ Most stable nucleus
\end{block}

\pause
\begin{itemize}
\item Heavy nuclei (right of Fe): release energy by \textbf{fission}
\item Light nuclei (left of Fe): release energy by \textbf{fusion}
\end{itemize}
\note{[P0] [THE REVELATION] "This curve explains everything"\\\\
- "Why stars shine"\\\\
- "Why bombs explode"\\\\
- "Why iron is endpoint"\\\\
[P1] "Moving UP the curve releases energy"\\\\
- "Left side: fusion climbs"\\\\
- "Right side: fission climbs"\\\\
[THE WONDER] "Iron is nuclear ash - can't be burned for energy"}
\end{frame}

% Nuclear Fission
\begin{frame}
\frametitle{22.4 Nuclear Fission}
\begin{columns}
\column{0.5\textwidth}
\begin{figure}
\centering
\includegraphics[width=\linewidth,height=0.5\textheight,keepaspectratio]{phys12-atomic-nuclear-fig53-1.jpg}
\end{figure}

\column{0.5\textwidth}
\begin{block}{The Process}
\begin{enumerate}
\item Neutron strikes heavy nucleus
\item Nucleus elongates
\item EM repulsion overcomes strong force
\item Nucleus splits
\item Releases energy + more neutrons
\end{enumerate}
\end{block}
\end{columns}

\pause
\textbf{Typical energy:} ~200 MeV per fission\\
(vs. ~1 eV per chemical reaction)
\note{[P0] [THE HOOK] "Splitting the unsplittable"\\\\
[THE CONNECTION - Digital] "Nuclear power plants run game servers"\\\\
[P1] "Liquid drop model"\\\\
- "Nucleus wobbles like water drop"\\\\
- "Stretches, narrows, breaks"\\\\
[THE WONDER] "One fission: 200 million electron-volts"\\\\
- "Million times more than breaking molecular bond"}
\end{frame}

% Chain Reaction
\begin{frame}
\frametitle{22.4 Chain Reaction}
\begin{figure}
\centering
\includegraphics[width=0.6\textwidth,height=0.5\textheight,keepaspectratio]{phys12-atomic-nuclear-fig54-1.jpg}
\end{figure}

\begin{block}{Self-Sustaining Fission}
Each fission releases 2-3 neutrons\\
$\rightarrow$ Those neutrons cause more fissions\\
$\rightarrow$ Exponential growth
\end{block}

\pause
\begin{alertblock}{Critical Mass}
Minimum amount of fissile material for self-sustaining reaction\\
$^{235}$U: ~52 kg, \quad $^{239}$Pu: ~10 kg
\end{alertblock}
\note{[P0] "Chain reaction: nuclear avalanche"\\\\
- "1 becomes 2, becomes 4, becomes 8..."\\\\
- "Generation time: microseconds"\\\\
[P1] [THE CONFLICT] "Below critical mass: fizzles"\\\\
- "Above: explodes"\\\\
[THE WONDER] "Hiroshima bomb: 64 kg U-235, only 0.7 kg fissioned"\\\\
- "Rest scattered without fissioning"}
\end{frame}

% Nuclear Reactor
\begin{frame}
\frametitle{22.4 Nuclear Fission Reactor}
\begin{figure}
\centering
\includegraphics[width=0.7\textwidth,height=0.55\textheight,keepaspectratio]{phys12-atomic-nuclear-fig55-1.jpg}
\end{figure}

\textbf{Key Components:}
\begin{itemize}
\item Fuel rods: enriched $^{235}$U
\item Moderator: water (slows neutrons)
\item Control rods: absorb excess neutrons
\item Heat exchanger: produces steam $\rightarrow$ turbine $\rightarrow$ electricity
\end{itemize}
\note{[P0] "Controlled chain reaction"\\\\
- "Balance: exactly one neutron causes next fission"\\\\
[THE CONNECTION - Digital] "Nuclear plants provide baseload power for internet"\\\\
[THE WONDER] "20 percent of US electricity from ~90 nuclear plants"\\\\
- "France: 70 percent nuclear"}
\end{frame}

% Mass-Energy in Fission
\begin{frame}
\frametitle{22.4 Mass-Energy Conversion}
\begin{block}{Einstein's Equation}
\Large
$$E = mc^2$$
\end{block}

\pause
\textbf{In fission:}
\begin{itemize}
\item Products have less mass than reactants
\item Missing mass $\rightarrow$ converted to energy
\item $\Delta m \approx 0.1\%$ of original mass
\item Still liberates enormous energy
\end{itemize}

\pause
\begin{exampleblock}{Example}
1 kg of $^{235}$U fully fissioned:\\
$E = 8.2 \times 10^{13}$ J $\approx$ 14,000 barrels of oil!
\end{exampleblock}
\note{[P0] [THE REVELATION] "Mass IS energy"\\\\
- "Not converts TO energy"\\\\
- "Mass IS a form of energy"\\\\
[ALGEBRA] "E equals m c-squared"\\\\
- "c-squared: huge number"\\\\
[P1] "Tiny mass loss: gigantic energy"\\\\
[P2] [THE WONDER] "1 gram equals energy to power car 270 years"}
\end{frame}

% Nuclear Fusion
\begin{frame}
\frametitle{22.4 Nuclear Fusion}
\begin{columns}
\column{0.5\textwidth}
\begin{figure}
\centering
\includegraphics[width=\linewidth,height=0.5\textheight,keepaspectratio]{phys12-atomic-nuclear-fig59-1.jpg}
\end{figure}

\column{0.5\textwidth}
\begin{block}{The Process}
Combine light nuclei\\
$\rightarrow$ Overcome Coulomb repulsion\\
$\rightarrow$ Strong force binds them\\
$\rightarrow$ Release energy
\end{block}

\pause
\textbf{Requirements:}
\begin{itemize}
\item Extreme temperature ($\sim 10^7$ K)
\item Extreme pressure
\item $\rightarrow$ Found in star cores!
\end{itemize}
\end{columns}
\note{[P0] [THE HOOK] "How does the Sun shine?"\\\\
- "Fusion: nature's power source"\\\\
[P1] "Barrier: positive charges repel"\\\\
- "Need high kinetic energy to overcome"\\\\
[THE WONDER] "Sun fuses 620 million tons H per second"\\\\
- "Has done so for 4.6 billion years"\\\\
- "Will continue 5 billion more"}
\end{frame}

% Proton-Proton Chain
\begin{frame}
\frametitle{22.4 The Proton-Proton Cycle}
\begin{block}{How the Sun Fuses Hydrogen}
\begin{align*}
^1\text{H} + ^1\text{H} &\rightarrow ^2\text{H} + e^+ + \nu_e \quad (0.42 \text{ MeV})\\
^1\text{H} + ^2\text{H} &\rightarrow ^3\text{He} + \gamma \quad (5.49 \text{ MeV})\\
^3\text{He} + ^3\text{He} &\rightarrow ^4\text{He} + ^1\text{H} + ^1\text{H} \quad (12.86 \text{ MeV})
\end{align*}
\end{block}

\pause
\textbf{Net result:}
$$4\,^1\text{H} \rightarrow\, ^4\text{He} + 2e^+ + 2\nu_e + 6\gamma \quad (26.7 \text{ MeV})$$

\pause
\begin{exampleblock}{The Wonder}
This energy took 32,000 years to reach Sun's surface\\
Then 8 minutes to reach Earth!
\end{exampleblock}
\note{[P0] "Three-step dance"\\\\
[ALGEBRA] "First: two protons make deuterium"\\\\
- "Releases positron and neutrino"\\\\
[P1] "Second: deuterium captures proton"\\\\
- "Makes helium-3, releases gamma"\\\\
[P2] "Third: two helium-3s fuse"\\\\
- "Makes helium-4, returns two protons"\\\\
[THE WONDER] "Net: 4 protons become 1 helium"\\\\
- "0.7 percent mass becomes energy"}
\end{frame}

% Fusion Energy
\begin{frame}
\frametitle{22.4 Fusion Energy Potential}
\begin{figure}
\centering
\includegraphics[width=0.6\textwidth,height=0.4\textheight,keepaspectratio]{phys12-atomic-nuclear-fig67-1.jpg}
\end{figure}

\begin{block}{The Promise}
\begin{itemize}
\item Fuel: deuterium from seawater (virtually unlimited)
\item No chain reaction $\rightarrow$ inherently safe
\item No long-lived radioactive waste
\item 4x more energy per kg than fission
\end{itemize}
\end{block}

\pause
\begin{alertblock}{The Challenge}
Confine 100-million-degree plasma long enough for net energy gain\\
\textbf{Progress:} ITER reactor under construction in France
\end{alertblock}
\note{[P0] [THE HOOK] "Energy of the future - for 70 years"\\\\
- "Always 20 years away"\\\\
[P1] "Tokamak: magnetic bottle"\\\\
- "Plasma can't touch walls"\\\\
- "Would instantly cool"\\\\
[THE HUMILITY] "Fusion is hard. Sun has gravity. We don't."\\\\
[THE WONDER] "If we succeed: energy problem solved forever"}
\end{frame}

% Fission vs Fusion Summary
\begin{frame}
\frametitle{22.4 Fission vs. Fusion Summary}
\begin{columns}[t]
\column{0.5\textwidth}
\begin{block}{Fission}
\textbf{Split} heavy nuclei\\[0.3cm]
\textbf{Fuel:} $^{235}$U, $^{239}$Pu (rare)\\[0.3cm]
\textbf{Products:} Radioactive waste\\[0.3cm]
\textbf{Status:} Mature technology\\[0.3cm]
\textbf{Use:} Power plants, weapons
\end{block}

\column{0.5\textwidth}
\begin{block}{Fusion}
\textbf{Combine} light nuclei\\[0.3cm]
\textbf{Fuel:} H isotopes (abundant)\\[0.3cm]
\textbf{Products:} Helium (inert)\\[0.3cm]
\textbf{Status:} Experimental\\[0.3cm]
\textbf{Use:} Stars, H-bombs, (future power?)
\end{block}
\end{columns}
\note{[P0] "Two paths to nuclear energy"\\\\
- "Both tap binding energy curve"\\\\
[THE REVELATION] "Fission: we've mastered"\\\\
- "Fusion: still learning"\\\\
[THE WONDER] "Both release energy by becoming more like iron"}
\end{frame}

% Section Summary
\begin{frame}
\frametitle{Summary: Chapter 22 (Sections 1-4)}
\begin{block}{22.1 Atomic Structure}
Rutherford $\rightarrow$ Bohr $\rightarrow$ Quantum model\\
Discrete energy levels explain emission spectra
\end{block}

\begin{block}{22.2 Nuclear Forces and Radioactivity}
Strong force holds nucleus together\\
Alpha, beta, gamma decay restore stability
\end{block}

\begin{block}{22.3 Half-Life}
Exponential decay enables radiometric dating\\
Carbon-14, U-238 date different timescales
\end{block}

\begin{block}{22.4 Fission and Fusion}
Binding energy curve: both release enormous energy\\
Fission mature, fusion promising
\end{block}
\note{[P0] "Journey from atoms to stars"\\\\
[THE WONDER] "Physics reveals unity:"\\\\
- "Atoms and nuclei obey same quantum rules"\\\\
- "Energy and mass are interchangeable"\\\\
- "Stars and bombs use same fusion"\\\\
"Understanding this changes civilization"}
\end{frame}

% Key Equations
\begin{frame}[shrink]
\frametitle{Key Equations: Chapter 22}
\begin{block}{Atomic Structure}
$$E_n = -\frac{13.6 \text{ eV}}{n^2}, \quad \Delta E = E_f - E_i, \quad E = hf = \frac{hc}{\lambda}$$
$$\frac{1}{\lambda} = R\left(\frac{1}{n_f^2} - \frac{1}{n_i^2}\right), \quad \Delta x \Delta p \geq \frac{h}{4\pi}$$
\end{block}

\begin{block}{Nuclear Physics}
$$E = mc^2$$
\end{block}

\begin{block}{Radioactive Decay}
$$N(t) = N_0 e^{-\lambda t}, \quad \lambda = \frac{0.693}{t_{1/2}}, \quad R = \lambda N$$
\end{block}

\begin{block}{Nuclear Notation}
$$^A_Z X_N \quad (A = Z + N)$$
\end{block}
\note{[P0] "Seven essential equations"\\\\
- "Bohr energy, photon energy, Rydberg"\\\\
- "Heisenberg uncertainty"\\\\
- "Einstein mass-energy"\\\\
- "Exponential decay, half-life, activity"\\\\
- "Nuclide notation"\\\\
[THE WONDER] "These equations describe reality from atoms to stars"}
\end{frame}

% Homework
\begin{frame}
\frametitle{Homework}
\begin{center}
\Large
Complete the assigned problems\\[0.3cm]
posted on the LMS
\end{center}
\note{[P0] "Practice these concepts"\\\\
- "Energy level transitions"\\\\
- "Half-life calculations"\\\\
- "Nuclear equations"\\\\
[TIMING] "Due date on LMS"\\\\
"Questions before we end?"}
\end{frame}

\end{document}
