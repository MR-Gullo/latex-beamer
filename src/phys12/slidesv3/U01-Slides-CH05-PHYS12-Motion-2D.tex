\documentclass{beamer}
\usepackage{../../../shared/templates/ds9_theme}
\usepackage[overridenote]{pdfpc}
\graphicspath{{../images/}{../../shared/images/}}

\title[Motion in 2D]{PHYS12 CH:5 Motion in Two Dimensions}
\subtitle{Vectors and Projectiles}
\author[Mr. Gullo]{Mr. Gullo}
\date[December 2025]{December 2025}

\begin{document}

\frame{\titlepage
\note{[THE HOOK] Today we unlock motion in two dimensions.\\\\
- First dimension was simple - just forward and back\\\\
- Real world is 2D and 3D - basketballs, planes, rockets\\\\
[THE WONDER] By end of class, you'll predict where a projectile lands\\\\
- Same math NASA uses for spacecraft trajectories}
}

\begin{frame}
\frametitle{Outline}
\tableofcontents
\end{frame}

\section{Introduction}

\begin{frame}
\frametitle{The Mystery}
\begin{center}
\Large How do you describe motion\\
\textit{when objects move in two directions at once?}
\end{center}

\pause
\vspace{0.5cm}
A football spiraling downfield...

\pause
\vspace{0.3cm}
A drone navigating through space...

\pause
\vspace{0.3cm}
\alert{Motion happens in multiple dimensions simultaneously.}
\note{[P0] "How do you describe motion when objects move in two directions at once?"\\\\
[P1] "A football spiraling downfield - horizontal and vertical motion together"\\\\
[P2] "A drone navigating through space - three dimensional control"\\\\
[P3] [THE WONDER] "Motion happens in multiple dimensions simultaneously"\\\\
[THE REVELATION] We break complex 2D motion into simple 1D components}
\end{frame}

\section{Vector Addition: Graphical Methods}

\begin{frame}
\frametitle{Learning Objectives}
\begin{block}{By the end of this section, you will be able to:}
\begin{itemize}
\item \textbf{5.1:} Describe the graphical method of vector addition and subtraction \pause
\item \textbf{5.1:} Use the graphical method to solve physics problems
\end{itemize}
\end{block}
\note{[P0] "Two objectives for graphical vector addition"\\\\
[P1] "Second: use this method to solve real physics problems"\\\\
- This is foundation for all 2D motion}
\end{frame}

\begin{frame}
\frametitle{5.1 What Is a Vector?}
\begin{block}{The Dual Nature}
A vector is a quantity with both magnitude AND direction.
\end{block}

\pause
\vspace{0.3cm}

\textbf{Vectors in physics:}
\begin{itemize}
\item Displacement, velocity, acceleration \pause
\item Force, momentum \pause
\item Boldface notation: $\mathbf{F}$, $\mathbf{v}$, $\mathbf{a}$
\end{itemize}

\pause
\begin{exampleblock}{The Mental Model}
Velocity is like a speedometer with a compass attached.
Speed tells how fast. Velocity adds where.
\end{exampleblock}
\note{[P0] "A vector has both magnitude AND direction"\\\\
[P1] "Displacement, velocity, acceleration - all vectors"\\\\
[P2] "Force, momentum - also vectors"\\\\
[P3] "Boldface notation shows it's a vector"\\\\
[P4] [THE CONNECTION - Digital Archetype] "Like a game character - position vector tells x,y coordinates"\\\\
[THE WONDER] Direction matters as much as size}
\end{frame}

\begin{frame}
\frametitle{5.1 The Journey to the Destination}
\begin{figure}
\centering
\includegraphics[width=0.7\textwidth,height=0.5\textheight,keepaspectratio]{phys12-motion-2d-fig04.jpg}
\end{figure}

\pause
\textbf{Path:} 9 blocks east + 5 blocks north = 14 blocks walked

\textbf{Displacement:} 10.3 blocks at $29.1^\circ$ north of east

\pause
\begin{alertblock}{Civilian View vs. Reality}
\textbf{Civilian:} "I walked 14 blocks."\\
\textbf{Physicist:} "Displacement was 10.3 blocks."
\end{alertblock}
\note{[P0] [Fig 5.2: City grid navigation] "Map showing 9 blocks east, 5 blocks north with displacement vector. Teaching hint: Have students trace path with finger, then draw straight line - which is shorter? This builds geometric intuition before calculation."\\\\
[P1] "Path: 14 blocks total. Displacement: 10.3 blocks straight line"\\\\
[P2] [THE CONFLICT] "Civilians measure path. Physicists measure displacement"\\\\
[THE REVELATION] Displacement is shortest straight-line distance\\\\
[THE CONNECTION - Kinetic Archetype] "Like GPS navigation - shows straight-line distance, not road distance"}
\end{frame}

\begin{frame}
\frametitle{5.1 The Head-to-Tail Method}
\textbf{How to add vectors graphically:}
\begin{enumerate}
\item Draw first vector to scale with protractor \pause
\item Place tail of second vector at head of first \pause
\item Continue for all vectors \pause
\item Draw resultant from tail of first to head of last \pause
\item Measure magnitude with ruler, direction with protractor
\end{enumerate}

\pause
\vspace{0.3cm}
\textbf{Key insight:} Order doesn't matter! $\mathbf{A} + \mathbf{B} = \mathbf{B} + \mathbf{A}$
\note{[P0] "Five steps to add vectors graphically"\\\\
[P1] "Draw first vector to scale"\\\\
[P2] "Place tail of second at head of first"\\\\
[P3] "Continue for all vectors"\\\\
[P4] "Draw resultant from start to finish"\\\\
[P5] "Measure with ruler and protractor"\\\\
[P6] [THE REVELATION] "Order doesn't matter - commutative property"\\\\
[THE HUMILITY] Graphical method is approximate - limited by drawing precision}
\end{frame}

\begin{frame}
\frametitle{5.1 Building the Resultant}
\begin{figure}
\centering
\includegraphics[width=0.7\textwidth,height=0.6\textheight,keepaspectratio]{phys12-motion-2d-fig05.jpg}
\end{figure}

\pause
The resultant vector connects start to finish.
\note{[P0] [Fig 5.5: Head-to-tail construction] "Shows resultant forming right triangle with 10.3 units magnitude at 29.1 degrees. Teaching hint: Emphasize connecting tail-to-head creates a geometric path - the resultant 'closes the loop' from start to final position."\\\\
[P1] "Resultant vector is the sum - connects start to finish"\\\\
- Use ruler and protractor for precision\\\\
- This is how we solved problems before calculators\\\\
[THE WONDER] Same method works for forces, velocities, any vectors}
\end{frame}

\begin{frame}
\frametitle{5.1 Vector Subtraction}
\textbf{To subtract:} Add the negative vector

\vspace{0.3cm}
$\mathbf{A} - \mathbf{B} = \mathbf{A} + (-\mathbf{B})$

\pause
\vspace{0.3cm}

\begin{figure}
\centering
\includegraphics[width=0.6\textwidth,height=0.4\textheight,keepaspectratio]{phys12-motion-2d-fig06.jpg}
\end{figure}

\pause
\textbf{The negative vector:} Same magnitude, opposite direction
\note{[P0] [Fig 5.6: Vector negation] "Shows vector B pointing upward and -B pointing downward, parallel but opposite. Teaching hint: Ask students to point their pencil in a direction, then flip 180 degrees - that's the negative vector. Physical gesture reinforces concept."\\\\
[P1] "Negative vector shown - opposite direction, same magnitude"\\\\
[P2] "Just flip it 180 degrees"\\\\
- Same head-to-tail method applies\\\\
[THE CONNECTION - Digital Archetype] "Like reversing velocity in a game"}
\end{frame}

\begin{frame}
\frametitle{5.1 Forces on Ice}
\begin{figure}
\centering
\includegraphics[width=0.7\textwidth,height=0.5\textheight,keepaspectratio]{phys12-motion-2d-fig08.jpg}
\end{figure}

\pause
Two skaters push with 400 N each at right angles.

\pause
\textbf{Resultant force:} $F_{tot} = \sqrt{(400)^2 + (400)^2} = 566$ N

\pause
\alert{Pythagorean theorem works when vectors are perpendicular!}
\note{[P0] [Fig 5.7: Perpendicular force vectors] "Overhead view: two skaters pushing third skater with 400N forces at right angles. Teaching hint: Draw on board - when vectors meet at 90 degrees, right triangle emerges. This is WHY Pythagorean theorem applies - geometric proof, not just formula."\\\\
[P1] "Each pushes with 400 N"\\\\
[P2] "Use Pythagorean theorem: 566 N resultant"\\\\
[P3] [THE REVELATION] "Pythagorean theorem only for right angles"\\\\
[THE CONNECTION - Kinetic Archetype] "Hockey players use this instinctively"\\\\
[THE WONDER] Vector addition is geometric, not just arithmetic}
\end{frame}

\section{Vector Addition: Analytical Methods}

\begin{frame}
\frametitle{Learning Objectives}
\begin{block}{By the end of this section, you will be able to:}
\begin{itemize}
\item \textbf{5.2:} Define components of vectors \pause
\item \textbf{5.2:} Describe the analytical method of vector addition \pause
\item \textbf{5.2:} Use the analytical method to solve problems
\end{itemize}
\end{block}
\note{[P0] "Three objectives for analytical method"\\\\
[P1] "First: break vectors into x and y components"\\\\
[P2] "Second: learn the analytical method - more precise than graphical"\\\\
[P3] "Third: solve real problems with this method"\\\\
- This is the method we'll use most often}
\end{frame}

\begin{frame}
\frametitle{5.2 The Power of Trigonometry}
\begin{figure}
\centering
\includegraphics[width=0.6\textwidth,height=0.4\textheight,keepaspectratio]{phys12-motion-2d-fig19.jpg}
\end{figure}

\pause
\textbf{Refresher:}
\begin{itemize}
\item $\sin\theta = \frac{\text{opposite}}{\text{hypotenuse}}$ \pause
\item $\cos\theta = \frac{\text{adjacent}}{\text{hypotenuse}}$ \pause
\item $\tan\theta = \frac{\text{opposite}}{\text{adjacent}}$
\end{itemize}
\note{[P0] [Fig 5.17: Right triangle with trig ratios] "Shows labeled sides h (hypotenuse), x (adjacent), y (opposite) with formulas. Teaching hint: Draw triangle on board, have students chant 'SOH-CAH-TOA' - kinesthetic + auditory reinforcement of abstract formulas."\\\\
[P1] "Sine: opposite over hypotenuse"\\\\
[P2] "Cosine: adjacent over hypotenuse"\\\\
[P3] "Tangent: opposite over adjacent"\\\\
[THE HUMILITY] "If trig feels rusty, that's normal - we'll practice"\\\\
[THE WONDER] Ancient Greeks discovered these ratios 2000 years ago}
\end{frame}

\begin{frame}
\frametitle{5.2 Breaking Vectors Into Components}
\begin{block}{The Source Code}
\begin{center}
\Large $\boxed{A_x = A \cos\theta}$

\Large $\boxed{A_y = A \sin\theta}$
\end{center}
Every 2D vector can be split into x and y components.
\end{block}

\pause
\begin{figure}
\centering
\includegraphics[width=0.5\textwidth,height=0.35\textheight,keepaspectratio]{phys12-motion-2d-fig20.jpg}
\end{figure}
\note{[P0] [Fig 5.18: Vector component decomposition] "Vector A with tail at origin splits into Ax (horizontal) and Ay (vertical) forming right triangle. Teaching hint: Trace the vector, then 'shadow cast' onto x-axis (cosine) and y-axis (sine) - visual metaphor helps students remember which trig function to use."\\\\
- Cosine gives x-component (horizontal)\\\\
- Sine gives y-component (vertical)\\\\
[P1] "Diagram shows vector A and its components"\\\\
[THE CONNECTION - Digital Archetype] "Like pixel coordinates - every position is x,y"\\\\
[THE WONDER] This transforms geometry into algebra}
\end{frame}

\begin{frame}
\frametitle{5.2 Example: City Walk Components}
Displacement: $A = 10.3$ blocks at $\theta = 29.1^\circ$ north of east

\pause
\vspace{0.3cm}

\textbf{East component (x):}
\begin{align*}
A_x &= A \cos\theta \\
&= (10.3)(\cos 29.1^\circ) = 9.0 \text{ blocks}
\end{align*}

\pause

\textbf{North component (y):}
\begin{align*}
A_y &= A \sin\theta \\
&= (10.3)(\sin 29.1^\circ) = 5.0 \text{ blocks}
\end{align*}

\pause
\alert{Components match the original path: 9 east + 5 north!}
\note{[P0] "Apply component formulas to city walk"\\\\
[P1] [ALGEBRA] "A-x equals A times cosine theta"\\\\
[P2] [ALGEBRA] "A-y equals A times sine theta"\\\\
[P3] [THE REVELATION] "Components match the original path - 9 east, 5 north"\\\\
[ANSWER] x = 9.0 blocks, y = 5.0 blocks\\\\
[THE WONDER] Trig recovers the information we started with}
\end{frame}

\begin{frame}
\frametitle{5.2 Reverse: From Components to Vector}
\begin{block}{Reconstruction Formulas}
\begin{center}
\Large $\boxed{A = \sqrt{A_x^2 + A_y^2}}$

\Large $\boxed{\theta = \tan^{-1}\left(\frac{A_y}{A_x}\right)}$
\end{center}
Pythagorean theorem gives magnitude, inverse tangent gives direction.
\end{block}

\pause
\begin{alertblock}{The Paradox}
\textbf{Students confuse:} $\mathbf{A}_x + \mathbf{A}_y = \mathbf{A}$ (vector addition)\\
with $A = \sqrt{A_x^2 + A_y^2}$ (magnitude calculation)
\end{alertblock}
\note{[P0] [THE REVELATION] "Pythagorean theorem gives magnitude, inverse tangent gives direction"\\\\
- Square root of sum of squares for magnitude\\\\
- Arctangent of y over x for angle\\\\
[P1] [THE CONFLICT] "Common mistake: adding magnitudes directly"\\\\
- Vector addition is NOT scalar addition\\\\
[THE HUMILITY] This confuses everyone at first\\\\
[THE WONDER] Two numbers (components) completely describe a vector}
\end{frame}

\begin{frame}
\frametitle{5.2 The Analytical Method}
\textbf{To add vectors $\mathbf{A}$ and $\mathbf{B}$ analytically:}

\begin{enumerate}
\item Find components: $A_x, A_y, B_x, B_y$ \pause
\item Add x-components: $R_x = A_x + B_x$ \pause
\item Add y-components: $R_y = A_y + B_y$ \pause
\item Find magnitude: $R = \sqrt{R_x^2 + R_y^2}$ \pause
\item Find direction: $\theta = \tan^{-1}(R_y/R_x)$
\end{enumerate}

\pause
\vspace{0.3cm}
\alert{More accurate than graphical method - not limited by drawing precision!}
\note{[P0] "Five steps to analytical vector addition"\\\\
[P1] "Find all components using sine and cosine"\\\\
[P2] "Add x-components separately"\\\\
[P3] "Add y-components separately"\\\\
[P4] "Pythagorean theorem for magnitude"\\\\
[P5] "Inverse tangent for direction"\\\\
[P6] [THE REVELATION] "Much more accurate than ruler and protractor"\\\\
[THE WONDER] Algebra is more precise than geometry}
\end{frame}

\begin{frame}
\frametitle{Attempt: Two-Leg Journey}
\begin{exampleblock}{The Challenge (3 min, silent)}
A person walks 53.0 m at $20.0^\circ$ north of east, then 34.0 m at $63.0^\circ$ north of east.

\vspace{0.3cm}

\textbf{Given:}
\begin{itemize}
\item $A = 53.0$ m, $\theta_A = 20.0^\circ$
\item $B = 34.0$ m, $\theta_B = 63.0^\circ$
\end{itemize}

\textbf{Find:} Total displacement magnitude and direction

\vspace{0.3cm}

\textit{Can you decode this journey? Work silently.}
\end{exampleblock}
\note{[THE CHALLENGE] Can they navigate this two-leg journey?\\\\
[SAY] "Try this on your own. It's okay to get stuck."\\\\
[TIMING] 3-4 min SILENT individual work\\\\
[CIRCULATE] Note who breaks into components correctly\\\\
[WATCH FOR] Confusion about which angle to use\\\\
[DON'T HELP] Productive struggle builds understanding}
\end{frame}

\begin{frame}
\frametitle{Compare: Vector Addition Strategy}
\textbf{Turn and talk (2 min):}

\vspace{0.3cm}

\begin{enumerate}
\item What's the first step - find components or add vectors?
\item How many components do you need to calculate?
\item Do you add x and y together, or keep them separate?
\end{enumerate}

\vspace{0.5cm}

\pause
\alert{Name wheel:} One pair share your approach (not your answer).
\note{[TIMING] 2-3 min pair discussion\\\\
[CIRCULATE] Listen for understanding of component method\\\\
[CHECK] Name wheel: call a pair to share approach\\\\
[EXPECTED APPROACH] Find A-x, A-y, B-x, B-y, then add separately\\\\
[COMMON ERROR] Adding magnitudes directly without components}
\end{frame}

\begin{frame}
\frametitle{Reveal: The Navigation Solution}
\textbf{Self-correct in a different color:}

\textbf{Step 1:} Find components of $\mathbf{A}$
\begin{align*}
A_x &= (53.0)(\cos 20.0^\circ) = 49.8 \text{ m} \\
A_y &= (53.0)(\sin 20.0^\circ) = 18.1 \text{ m}
\end{align*}

\pause

\textbf{Step 2:} Find components of $\mathbf{B}$
\begin{align*}
B_x &= (34.0)(\cos 63.0^\circ) = 15.4 \text{ m} \\
B_y &= (34.0)(\sin 63.0^\circ) = 30.3 \text{ m}
\end{align*}
\note{[P0] [ALGEBRA] "A-x equals 53 times cosine 20 degrees equals 49.8 meters"\\\\
- "A-y equals 53 times sine 20 degrees equals 18.1 meters"\\\\
[P1] [ALGEBRA] "B-x equals 34 times cosine 63 degrees equals 15.4 meters"\\\\
- "B-y equals 34 times sine 63 degrees equals 30.3 meters"\\\\
[THE CONNECTION - Digital Archetype] "Like calculating sprite positions in a game"}
\end{frame}

\begin{frame}
\frametitle{Reveal: Combining Components}
\textbf{Step 3:} Add components
\begin{align*}
R_x &= A_x + B_x = 49.8 + 15.4 = 65.2 \text{ m} \\
R_y &= A_y + B_y = 18.1 + 30.3 = 48.4 \text{ m}
\end{align*}

\pause

\textbf{Step 4:} Find magnitude
$$R = \sqrt{(65.2)^2 + (48.4)^2} = \boxed{81.2 \text{ m}}$$

\pause

\textbf{Step 5:} Find direction
$$\theta = \tan^{-1}(48.4/65.2) = \boxed{36.6^\circ \text{ north of east}}$$
\note{[P0] [ALGEBRA] "R-x equals 65.2 meters, R-y equals 48.4 meters"\\\\
[P1] [ALGEBRA] "R equals square root of 65.2 squared plus 48.4 squared"\\\\
[ANSWER] Magnitude: 81.2 meters\\\\
[P2] [ALGEBRA] "Theta equals inverse tangent of 48.4 over 65.2"\\\\
[ANSWER] Direction: 36.6 degrees north of east\\\\
[THE WONDER] You just navigated like a GPS satellite does}
\end{frame}

\section{Projectile Motion}

\begin{frame}
\frametitle{Learning Objectives}
\begin{block}{By the end of this section, you will be able to:}
\begin{itemize}
\item \textbf{5.3:} Describe the properties of projectile motion \pause
\item \textbf{5.3:} Apply kinematic equations and vectors to solve projectile problems
\end{itemize}
\end{block}
\note{[P0] "Two objectives for projectile motion"\\\\
[P1] "Second: use kinematics to predict trajectories"\\\\
- This is where vectors meet motion\\\\
[THE WONDER] Same equations that land spacecraft on Mars}
\end{frame}

\begin{frame}
\frametitle{5.3 The Great Separation}
\begin{block}{Nature's Rule for Projectiles}
\begin{center}
\Large Horizontal and vertical motions\\
\Large \alert{are independent}
\end{center}
They don't influence each other.
\end{block}

\pause
\begin{figure}
\centering
\includegraphics[width=0.7\textwidth,height=0.45\textheight,keepaspectratio]{phys11-motion-2d-fig5-27.jpg}
\end{figure}
\note{[P0] [THE REVELATION] "Horizontal and vertical motions are independent"\\\\
- Most important concept in projectile motion\\\\
- They happen simultaneously but separately\\\\
[P1] [Fig 5.27: Cannonball trajectories] "Three paths shown: vertical drop (A), curved projectile (B), horizontal launch (C). Teaching hint: Ask students which hits ground first if dropped vs launched horizontally - same time! Vertical motion independent of horizontal."\\\\
[THE WONDER] One idea unlocks all projectile motion}
\end{frame}

\begin{frame}
\frametitle{5.3 What Is a Projectile?}
\textbf{Definition:} An object launched into the air that moves under gravity alone

\pause
\vspace{0.3cm}

\textbf{Key terms:}
\begin{itemize}
\item \textbf{Projectile:} The moving object \pause
\item \textbf{Trajectory:} Its curved path \pause
\item \textbf{Range:} Horizontal distance traveled \pause
\item \textbf{Maximum height:} Peak altitude
\end{itemize}

\pause
\vspace{0.3cm}

\alert{We ignore air resistance in introductory physics.}
\note{[P0] "Projectile moves under gravity alone after launch"\\\\
[P1] "Projectile is the object"\\\\
[P2] "Trajectory is the curved path"\\\\
[P3] "Range is horizontal distance"\\\\
[P4] "Maximum height is peak altitude"\\\\
[P5] [THE HUMILITY] "Air resistance matters in reality, but calculation is complex"\\\\
[THE CONNECTION - Kinetic Archetype] "Every ball you throw is a projectile"}
\end{frame}

\begin{frame}
\frametitle{5.3 The Independence Principle}
\textbf{Horizontal motion:}
\begin{itemize}
\item No acceleration: $a_x = 0$ \pause
\item Constant velocity: $v_x = v_{0x}$ \pause
\item Position: $x = x_0 + v_x t$
\end{itemize}

\pause

\textbf{Vertical motion:}
\begin{itemize}
\item Constant acceleration: $a_y = -g = -9.8$ m/s$^2$ \pause
\item Velocity: $v_y = v_{0y} - gt$ \pause
\item Position: $y = y_0 + v_{0y}t - \frac{1}{2}gt^2$
\end{itemize}

\pause
\alert{Time $t$ is the only variable connecting the two motions!}
\note{[P0] "Horizontal: no acceleration, constant velocity"\\\\
[P1] "Velocity stays constant horizontally"\\\\
[P2] "Position is velocity times time"\\\\
[P3] "Vertical: gravity pulls down at 9.8 m/s squared"\\\\
[P4] "Velocity decreases going up, increases coming down"\\\\
[P5] "Position uses kinematic equation with gravity"\\\\
[P6] [THE REVELATION] "Time connects the two - same t for both"\\\\
[THE WONDER] Two 1D problems instead of one 2D problem}
\end{frame}

\begin{frame}
\frametitle{5.3 The Trajectory}
\begin{figure}
\centering
\includegraphics[width=0.7\textwidth,height=0.6\textheight,keepaspectratio]{phys12-motion-2d-fig34.jpg}
\end{figure}

\pause
Ball kicked at angle $\theta$ follows parabolic path.
\note{[P0] [Fig 5.28: Soccer ball trajectory] "Boy kicking ball with curved parabolic path shown. Right triangle overlaid with displacement components x (horizontal), y (vertical), d (hypotenuse). Teaching hint: Ask students to sketch ball path in air - they'll draw parabola intuitively. Then reveal: 'Your brain already knows physics!'"\\\\
[P1] "Displacement has x and y components"\\\\
- Horizontal component from constant velocity\\\\
- Vertical component from accelerated motion\\\\
[THE CONNECTION - Kinetic Archetype] "Every soccer player knows this curve instinctively"\\\\
[THE WONDER] Parabola is the natural shape of projectile motion}
\end{frame}

\begin{frame}
\frametitle{5.3 The Four-Step Method}
\textbf{To solve projectile problems:}

\begin{enumerate}
\item Separate into x and y components \pause
\item Treat as two independent 1D motions \pause
\item Solve for unknowns (use $t$ to connect them) \pause
\item Recombine to find total displacement and velocity
\end{enumerate}

\pause
\vspace{0.3cm}

\textbf{Recombination formulas:}
$$d = \sqrt{x^2 + y^2}, \quad \theta = \tan^{-1}(y/x)$$
$$v = \sqrt{v_x^2 + v_y^2}, \quad \theta_v = \tan^{-1}(v_y/v_x)$$
\note{[P0] "Four-step method for projectile problems"\\\\
[P1] "Separate into x and y - use sine and cosine"\\\\
[P2] "Treat as independent 1D motions"\\\\
[P3] "Solve separately, time connects them"\\\\
[P4] "Recombine using Pythagorean theorem and inverse tangent"\\\\
[P5] [THE REVELATION] "Same component formulas we learned earlier"\\\\
[THE WONDER] All 2D motion reduces to two 1D problems}
\end{frame}

\begin{frame}
\frametitle{5.3 Maximum Height Formula}
\begin{block}{The Peak}
\begin{center}
\Large $\boxed{h = \frac{v_{0y}^2}{2g}}$
\end{center}
Maximum height depends only on initial vertical velocity.
\end{block}

\pause
\vspace{0.3cm}

\textbf{Key insight:} At maximum height, $v_y = 0$

\pause
\vspace{0.3cm}

\begin{alertblock}{What Your Brain Gets Wrong}
\textbf{Intuition:} "Faster launch means higher peak."\\
\textbf{Reality:} "Only the \textit{vertical} component matters."
\end{alertblock}
\note{[P0] [THE REVELATION] "Max height depends only on vertical velocity"\\\\
- Horizontal velocity doesn't affect height\\\\
[P1] "At peak, vertical velocity is zero - turning point"\\\\
[P2] [THE CONFLICT] "Intuition says total speed matters"\\\\
- Reality: only v-zero-y matters for height\\\\
[THE WONDER] You can throw horizontally fast and barely go up}
\end{frame}

\begin{frame}
\frametitle{Attempt: Fireworks Launch}
\begin{exampleblock}{The Challenge (3 min, silent)}
A fireworks shell is launched at 70.0 m/s at $75^\circ$ above horizontal.

\vspace{0.3cm}

\textbf{Given:}
\begin{itemize}
\item $v_0 = 70.0$ m/s
\item $\theta = 75^\circ$
\end{itemize}

\textbf{Find:} Maximum height

\vspace{0.3cm}

\textit{Can you predict where it explodes? Work silently.}
\end{exampleblock}
\note{[THE CHALLENGE] Can they predict the explosion height?\\\\
[SAY] "Try this on your own. This is real pyrotechnics physics."\\\\
[TIMING] 3-4 min SILENT individual work\\\\
[CIRCULATE] Note who finds vertical component first\\\\
[WATCH FOR] Using total velocity instead of v-zero-y\\\\
[DON'T HELP] Let them discover the vertical component insight}
\end{frame}

\begin{frame}
\frametitle{Compare: Projectile Strategy}
\textbf{Turn and talk (2 min):}

\vspace{0.3cm}

\begin{enumerate}
\item What's the first step - find components or use a formula?
\item Which velocity matters - total or vertical?
\item Which kinematic equation did you choose?
\end{enumerate}

\vspace{0.5cm}

\pause
\alert{Name wheel:} One pair share your approach (not your answer).
\note{[TIMING] 2-3 min pair discussion\\\\
[CIRCULATE] Listen for understanding of vertical component\\\\
[CHECK] Name wheel: call a pair to share\\\\
[EXPECTED APPROACH] Find v-zero-y using sine, then use max height formula\\\\
[COMMON ERROR] Using total velocity in height formula}
\end{frame}

\begin{frame}
\frametitle{Reveal: Reaching the Sky}
\textbf{Self-correct in a different color:}

\textbf{Step 1:} Find vertical velocity component
\begin{align*}
v_{0y} &= v_0 \sin\theta \\
&= (70.0)(\sin 75^\circ) = 67.6 \text{ m/s}
\end{align*}

\pause

\textbf{Step 2:} Use maximum height formula
\begin{align*}
h &= \frac{v_{0y}^2}{2g} \\
&= \frac{(67.6)^2}{2(9.8)} = \boxed{233 \text{ m}}
\end{align*}

\pause
\textbf{Check:} About 765 feet - reasonable for large fireworks!
\note{[P0] [ALGEBRA] "v-zero-y equals v-zero times sine theta"\\\\
- "equals 70 times sine 75 degrees equals 67.6 m/s"\\\\
[P1] [ALGEBRA] "h equals v-zero-y squared over 2g"\\\\
- "equals 67.6 squared over 2 times 9.8"\\\\
[ANSWER] Maximum height: 233 meters\\\\
[P2] [THE WONDER] "That's 765 feet - taller than a 70-story building"\\\\
- Professional fireworks reach these heights\\\\
[THE CONNECTION - Harmonic Archetype] "The boom echoes across the city"}
\end{frame}

\begin{frame}
\frametitle{5.3 The Range Equation}
\begin{block}{The Distance Formula}
\begin{center}
\Large $\boxed{R = \frac{v_0^2 \sin 2\theta}{g}}$
\end{center}
Range depends on initial speed and launch angle.
\end{block}

\pause
\vspace{0.3cm}

\textbf{Key insights:}
\begin{itemize}
\item Maximum range at $\theta = 45^\circ$ \pause
\item Same range for complementary angles: $30^\circ$ and $60^\circ$ \pause
\item Only valid when initial and final heights are equal
\end{itemize}
\note{[P0] [THE REVELATION] "Range formula for level ground"\\\\
- v-zero squared times sine of twice theta, over g\\\\
[P1] "Maximum range at 45 degrees - perfect balance"\\\\
[P2] "Complementary angles give same range"\\\\
- 30 and 60, or 20 and 70, etc.\\\\
[P3] "Only works if you land at same height you launched"\\\\
[THE WONDER] Artillery officers memorized this 500 years ago}
\end{frame}

\begin{frame}
\frametitle{Attempt: Volcanic Rock}
\begin{exampleblock}{The Challenge (3 min, silent)}
A rock is ejected from a volcano at 25.0 m/s at $35^\circ$ above horizontal. It lands 20.0 m below its starting point.

\vspace{0.3cm}

\textbf{Given:}
\begin{itemize}
\item $v_0 = 25.0$ m/s, $\theta = 35^\circ$
\item $y = -20.0$ m
\end{itemize}

\textbf{Find:} Time of flight

\vspace{0.3cm}

\textit{Can you predict the flight time? Work silently.}
\end{exampleblock}
\note{[THE CHALLENGE] Can they handle vertical displacement?\\\\
[SAY] "This one is trickier - quadratic equation ahead"\\\\
[TIMING] 3-4 min SILENT individual work\\\\
[CIRCULATE] Note who sets up kinematic equation correctly\\\\
[WATCH FOR] Sign errors with negative displacement\\\\
[DON'T HELP] Quadratic struggle is valuable learning}
\end{frame}

\begin{frame}
\frametitle{Compare: Quadratic Strategy}
\textbf{Turn and talk (2 min):}

\vspace{0.3cm}

\begin{enumerate}
\item Which kinematic equation has both $y$ and $t$?
\item What values do you substitute for $y_0$ and $y$?
\item How do you solve a quadratic equation?
\end{enumerate}

\vspace{0.5cm}

\pause
\alert{Name wheel:} One pair share your approach (not your answer).
\note{[TIMING] 2-3 min pair discussion\\\\
[CIRCULATE] Listen for quadratic equation recognition\\\\
[CHECK] Name wheel: call a pair to share\\\\
[EXPECTED APPROACH] Use y equals y-zero plus v-zero-y t minus half g t squared\\\\
[COMMON ERROR] Forgetting negative sign for downward displacement}
\end{frame}

\begin{frame}
\frametitle{Reveal: Time of Flight}
\textbf{Self-correct in a different color:}

\textbf{Step 1:} Find $v_{0y} = (25.0)(\sin 35^\circ) = 14.3$ m/s

\pause

\textbf{Step 2:} Use kinematic equation with $y_0 = 0$, $y = -20.0$ m
$$-20.0 = (14.3)t - (4.90)t^2$$

\pause

\textbf{Step 3:} Rearrange to standard form
$$(4.90)t^2 - (14.3)t - (20.0) = 0$$

\pause

\textbf{Step 4:} Quadratic formula
$$t = \frac{14.3 \pm \sqrt{(14.3)^2 - 4(4.90)(-20.0)}}{2(4.90)}$$
\note{[P0] [ALGEBRA] "v-zero-y equals 14.3 m/s"\\\\
[P1] "Substitute into y equals v-zero-y t minus half g t squared"\\\\
[P2] "Rearrange to standard quadratic form"\\\\
[P3] [ALGEBRA] "Quadratic formula: t equals negative b plus or minus square root..."\\\\
[THE HUMILITY] "Quadratic formula is tedious but reliable"\\\\
[THE CONNECTION - Digital Archetype] "Calculators use this exact method"}
\end{frame}

\begin{frame}
\frametitle{Reveal: Solving the Quadratic}
\textbf{Step 5:} Calculate discriminant
$$\sqrt{(14.3)^2 + 4(4.90)(20.0)} = \sqrt{596.5} = 24.4$$

\pause

\textbf{Step 6:} Two solutions
$$t = \frac{14.3 + 24.4}{9.8} = 3.96 \text{ s} \quad \text{(physical)}$$
$$t = \frac{14.3 - 24.4}{9.8} = -1.03 \text{ s} \quad \text{(reject)}$$

\pause

$$\boxed{t = 3.96 \text{ s}}$$

\pause
\alert{Negative time means before launch - physically impossible!}
\note{[P0] [ALGEBRA] "Calculate discriminant: 24.4"\\\\
[P1] "Two solutions: 3.96 seconds and negative 1.03 seconds"\\\\
[P2] [ANSWER] Time of flight: 3.96 seconds\\\\
[P3] [THE REVELATION] "Negative time is before launch - reject it"\\\\
[THE WONDER] Math gives two answers, physics picks the right one\\\\
- Quadratics always give two roots\\\\
- Physical reasoning eliminates impossible solution}
\end{frame}

\section{Summary}

\begin{frame}
\frametitle{What You Now Know}
\begin{block}{The Revelations}
\begin{enumerate}
\item Vectors have magnitude AND direction \pause
\item Graphical addition: head-to-tail method \pause
\item Analytical method: break into components, recombine \pause
\item Projectile motion: horizontal and vertical are independent \pause
\item Time connects the two dimensions \pause
\item Maximum height depends only on $v_{0y}$
\end{enumerate}
\end{block}
\note{[P0] "Six revelations from motion in 2D"\\\\
[P1] "Vectors encode two pieces of information"\\\\
[P2] "Head-to-tail for graphical addition"\\\\
[P3] "Components for analytical precision"\\\\
[P4] "Independence principle for projectiles"\\\\
[P5] "Time is the bridge between dimensions"\\\\
[P6] "Vertical component determines height"\\\\
[THE WONDER] You now predict trajectories like NASA engineers\\\\
- Name wheel: which concept was most surprising?}
\end{frame}

\begin{frame}[shrink]
\frametitle{Key Equations}
\textbf{Vector Components:}
\begin{align}
A_x &= A \cos\theta \\
A_y &= A \sin\theta
\end{align}

\textbf{Magnitude and Direction:}
\begin{align}
A &= \sqrt{A_x^2 + A_y^2} \\
\theta &= \tan^{-1}(A_y/A_x)
\end{align}

\textbf{Projectile Motion:}
\begin{align}
\text{Horizontal: } x &= x_0 + v_x t \\
\text{Vertical: } y &= y_0 + v_{0y}t - \frac{1}{2}gt^2 \\
\text{Max height: } h &= \frac{v_{0y}^2}{2g}
\end{align}
\note{- Seven fundamental equations for 2D motion\\\\
- Components split vectors into x and y\\\\
- Pythagorean and inverse tangent recombine\\\\
- Horizontal motion has constant velocity\\\\
- Vertical motion has constant acceleration\\\\
- Max height from vertical component only\\\\
[THE WONDER] Seven equations unlock all 2D trajectories}
\end{frame}

\begin{frame}
\frametitle{Homework}
\begin{center}
\Large
Complete the assigned problems\\[0.3cm]
posted on the LMS
\end{center}
\note{[SAY] "Homework is posted on the LMS"\\\\
[TIMING] Due date: check LMS\\\\
[CHECK] Questions before we end?\\\\
[TRANSITION] Next class: Chapter 6 - Circular Motion and Gravitation\\\\
[THE WONDER] We've conquered 2D. Next: objects moving in circles.}
\end{frame}

\end{document}
