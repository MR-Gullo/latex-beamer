\documentclass{beamer}
\usepackage{../../../shared/templates/ds9_theme}
\usepackage{../../../shared/templates/semantic-physics-colors}
\usepackage[overridenote]{pdfpc}
\graphicspath{{../images/}{../../shared/images/}}

\title[When the Universe Gets Weird]{PHYS12 CH:10 When the Universe Gets Weird}
\subtitle{Einstein's Revolution}
\author[Mr. Gullo]{Mr. Gullo}
\date[December 2025]{December 2025}

\begin{document}

\frame{\titlepage
\note{[THE HOOK] Today we discover why the universe breaks at extreme speeds.\\\\
- Einstein rewrote the rules: time slows, space shrinks, mass becomes energy\\\\
- Same laws that prevent us from reaching the stars\\\\
[THE WONDER] By end of class, you'll see reality differently - nothing moves the same way near light speed}
}

\begin{frame}
\frametitle{Outline}
\tableofcontents
\end{frame}

\section{Introduction}

\begin{frame}
\frametitle{The Dream}
\begin{center}
\Large Have you ever dreamed of traveling\\
\textit{to other star systems?}
\end{center}

\pause
\vspace{0.5cm}
Just fly fast enough, right?

\pause
\vspace{0.3cm}
\alert{Wrong. Physics has other plans.}
\note{[P0] "Have you ever dreamed of traveling to other star systems?"\\\\
[P1] "Just fly fast enough, right?"\\\\
[P2] [THE CONFLICT] "Wrong. Physics has other plans"\\\\
[THE HOOK] There's a speed limit to the universe - and it changes everything\\\\
- Time, space, mass, energy - all different at high speeds}
\end{frame}

\begin{frame}
\frametitle{Distant Worlds}
\begin{figure}
\centering
\includegraphics[width=0.8\textwidth,height=0.6\textheight,keepaspectratio]{phys11-relativity-fig10-1.jpg}
\caption{The Orion Nebula - home to distant star systems}
\end{figure}

\pause
\begin{alertblock}{The Barrier}
Special relativity explains why we can't reach these stars with current technology.
\end{alertblock}
\note{[Fig 10.1: Orion Nebula star system] "Use this to ground relativity in real astronomical wonder - students need to see what's at stake before accepting speed limits."\\\\
[P0] "The Orion Nebula - thousands of light years away"\\\\
[P1] [THE REVELATION] "Special relativity explains why we can't reach these stars"\\\\
[THE CONNECTION - Digital Archetype] "Like the universe has a maximum frame rate"\\\\
[THE WONDER] The speed of light is the ultimate speed limit}
\end{frame}

\begin{frame}
\frametitle{Before Einstein}
\textbf{Newton and Galileo were right... mostly.}

\pause
\vspace{0.3cm}

Classical relativity worked for centuries:
\begin{itemize}
\item Motion is relative to your frame of reference \pause
\item Velocities add together \pause
\item Time flows the same for everyone
\end{itemize}

\pause
\vspace{0.3cm}

\alert{But at extreme speeds, everything breaks down.}
\note{[P0] "Newton and Galileo were right for 300 years"\\\\
[P1] "Motion is relative to your frame of reference"\\\\
[P2] "Velocities add together"\\\\
[P3] "Time flows the same for everyone"\\\\
[P4] [THE CONFLICT] "But at extreme speeds, everything breaks down"\\\\
[THE HUMILITY] Einstein didn't replace Newton - he extended him}
\end{frame}

\section{Postulates of Special Relativity}

\begin{frame}
\frametitle{Learning Objectives}
\begin{block}{By the end of this section, you will be able to:}
\begin{itemize}
\item \textbf{10.1:} Describe the experiments that led Einstein to special relativity \pause
\item \textbf{10.1:} Understand the two postulates on which the theory is based \pause
\item \textbf{10.1:} Explain why simultaneity depends on frame of reference
\end{itemize}
\end{block}
\note{[P0] "Three objectives today"\\\\
[P1] "First: what experiments led to special relativity"\\\\
[P2] "Second: Einstein's two simple postulates"\\\\
[P3] "Third: why observers disagree on simultaneity"\\\\
- Assessment: problems next week}
\end{frame}

\begin{frame}
\frametitle{10.1 The Phantom Medium}
\textbf{19th century belief:} Light must travel through a medium

\pause
\vspace{0.3cm}

Like sound travels through air, light travels through... \textit{ether}?

\pause
\vspace{0.3cm}

\begin{exampleblock}{The Mental Model}
Scientists believed space was filled with an invisible fluid called the ether.
\end{exampleblock}

\pause
\alert{Problem: The ether doesn't exist.}
\note{[P0] "19th century: light must travel through a medium"\\\\
[P1] "Like sound travels through air, light travels through ether"\\\\
[P2] "Scientists believed space was filled with invisible fluid"\\\\
[P3] [THE CONFLICT] "Problem: The ether doesn't exist"\\\\
[THE HUMILITY] Even brilliant scientists can be wrong about fundamental assumptions}
\end{frame}

\begin{frame}
\frametitle{10.1 The Most Famous Failed Experiment}
\begin{figure}
\centering
\includegraphics[width=0.7\textwidth,height=0.5\textheight,keepaspectratio]{phys11-relativity-fig10-2.jpg}
\caption{Michelson-Morley interferometer (1887)}
\end{figure}

\pause
\textbf{Goal:} Measure Earth's speed through the ether

\pause
\textbf{Result:} No ether detected. Light speed is constant.
\note{[Fig 10.2: Interferometer light path diagram] "Point out the semi-silvered mirror and perpendicular paths - this shows students how null results can falsify theories. Emphasize that experimental precision drives conceptual revolution."\\\\
[P0] "Michelson and Morley built an interferometer in 1887"\\\\
[P1] "Goal: measure Earth's speed through the ether"\\\\
[P2] [THE REVELATION] "Result: no ether detected - light speed is constant"\\\\
[THE WONDER] This "failed" experiment changed physics forever\\\\
- Sometimes the best discoveries come from experiments that don't work as expected}
\end{frame}

\begin{frame}
\frametitle{10.1 What They Expected}
\textbf{The swimmer analogy:}

\vspace{0.3cm}

Two swimmers leave a moving platform:
\begin{itemize}
\item One swims with and against the current \pause
\item One swims perpendicular to the current \pause
\item Even at same speed, they arrive at different times
\end{itemize}

\pause
\vspace{0.3cm}

\alert{But light beams don't behave like swimmers.}
\note{[P0] "Imagine two swimmers leaving a moving platform"\\\\
[P1] "One swims with and against the current"\\\\
[P2] "One swims perpendicular to the current"\\\\
[P3] [THE CONFLICT] "But light beams don't behave like swimmers"\\\\
[THE REVELATION] Light speed is the same in all directions - no matter how you move}
\end{frame}

\begin{frame}
\frametitle{10.1 Enter Einstein}
\begin{figure}
\centering
\includegraphics[width=0.5\textwidth,height=0.4\textheight,keepaspectratio]{phys11-relativity-fig10-3.jpg}
\caption{Albert Einstein (1879-1955)}
\end{figure}

\pause
\textbf{1905:} Einstein proposes special relativity

\pause
Based on two simple postulates...
\note{[Fig 10.3: Einstein portrait] "Use Einstein's image to humanize genius - mention patent clerk background to show students that revolutionary thinking doesn't require credentials, just courage to question assumptions."\\\\
[P0] "Albert Einstein - 1879 to 1955"\\\\
[P1] "1905: Einstein proposes special relativity at age 26"\\\\
[P2] "Based on two simple postulates"\\\\
[THE HUMILITY] Einstein was working as a patent clerk when he changed physics\\\\
[THE WONDER] Sometimes the biggest ideas come from the simplest assumptions}
\end{frame}

\begin{frame}
\frametitle{10.1 The Two Postulates}
\begin{block}{Postulate 1: Universal Laws}
The laws of physics are the same in all inertial reference frames.
\end{block}

\pause
\vspace{0.3cm}

\begin{block}{Postulate 2: The Cosmic Speed Limit}
\begin{center}
\Large $\boxed{\vel{c} = 3.00 \times 10^8 \text{ m/s}}$
\end{center}
The \vel{speed of light} is the same in all inertial frames and is NOT affected by the speed of its source.
\end{block}
\note{[P0] [THE REVELATION] "Postulate 1: physics works the same everywhere"\\\\
- No special frame of reference\\\\
[P1] "Postulate 2: light speed is constant"\\\\
[ALGEBRA] "c equals 3.00 times 10 to the 8 meters per second"\\\\
[THE WONDER] These two simple ideas - when combined - shatter our intuition about reality}
\end{frame}

\begin{frame}
\frametitle{10.1 The Speed of Light}
\begin{center}
\Large $\vel{c} = 299{,}792{,}458 \text{ m/s}$
\end{center}

\pause
\vspace{0.3cm}

For most purposes: $\vel{c} \approx 3.00 \times 10^8 \text{ m/s}$

\pause
\vspace{0.3cm}

\begin{alertblock}{Civilian View vs. Reality}
\textbf{Civilian:} "Light from a speeding car goes faster."\\
\textbf{Physicist:} "Light always travels at $\vel{c}$, regardless of source speed."
\end{alertblock}
\note{[P0] [ALGEBRA] "c equals 299,792,458 meters per second - exact"\\\\
[P1] "For calculations, we round to 3.00 times 10 to the 8"\\\\
[P2] [THE CONFLICT] "Civilian thinks light from speeding car goes faster"\\\\
[THE REVELATION] "Physicist knows light always travels at c"\\\\
[THE WONDER] This is the most carefully measured constant in physics}
\end{frame}

\begin{frame}
\frametitle{10.1 Inertial Reference Frame}
\begin{block}{Definition}
A reference frame where objects follow Newton's First Law: Objects at rest stay at rest, objects in motion stay in motion at constant velocity, unless acted upon by external force.
\end{block}

\pause
\vspace{0.3cm}

\textbf{Examples:}
\begin{itemize}
\item Inside a car moving at constant velocity \pause
\item Inside a stationary house \pause
\item Inside a spacecraft coasting through space
\end{itemize}
\note{[P0] "Inertial reference frame: where Newton's First Law holds"\\\\
[P1] "Inside a car moving at constant velocity"\\\\
[P2] "Inside a stationary house"\\\\
[P3] "Inside a spacecraft coasting through space"\\\\
- NOT accelerating frames - those need general relativity}
\end{frame}

\begin{frame}
\frametitle{10.1 The Paradox of Velocities}
\textbf{Newtonian mechanics:} Velocities add

\pause
\vspace{0.3cm}

\begin{exampleblock}{The Mental Model}
You run at 3 m/s and throw a ball forward at 10 m/s.\\
Ball \vel{speed}: $3 + 10 = 13$ m/s
\end{exampleblock}

\pause
\vspace{0.3cm}

\textbf{But what about light?}

\pause
\vspace{0.3cm}

\begin{alertblock}{The Illusion}
Airliner traveling at 200 m/s emits light forward.\\
\textbf{Your brain says:} Light \vel{speed} = $\vel{c} + 200$ m/s\\
\textbf{Reality:} Light \vel{speed} = $\vel{c}$ (always)
\end{alertblock}
\note{[P0] "Newtonian mechanics: velocities add together"\\\\
[P1] [THE CONNECTION - Kinetic Archetype] "You run at 3 m/s and throw at 10 - ball goes 13 m/s"\\\\
[P2] "But what about light?"\\\\
[P3] [THE CONFLICT] "Your brain says light should go c plus 200"\\\\
[THE REVELATION] "Reality: light always goes c - no matter what"\\\\
[THE HUMILITY] This feels impossible - but experiments prove it true}
\end{frame}

\begin{frame}
\frametitle{10.1 Simultaneity Is Relative}
\begin{figure}
\centering
\includegraphics[width=0.8\textwidth,height=0.5\textheight,keepaspectratio]{phys11-relativity-fig10-4.jpg}
\caption{Two flash lamps on a moving train}
\end{figure}

\pause
\textbf{Observer A (on train):} Flashes simultaneous

\pause
\textbf{Observer B (on platform):} Flashes NOT simultaneous
\note{[Fig 10.4: Train thought experiment with observers] "Walk students through both perspectives slowly - have them identify with train observer first (symmetric), then switch to platform observer (asymmetric). This cognitive shift is where simultaneity breaks."\\\\
[P0] "Observer A sits in center of moving train - two flash lamps fire"\\\\
[P1] "Observer A sees flashes simultaneously"\\\\
[P2] [THE CONFLICT] "Observer B on platform sees right flash first"\\\\
[THE REVELATION] Simultaneity depends on frame of reference\\\\
[THE WONDER] Two observers, same event, different realities - both correct}
\end{frame}

\begin{frame}
\frametitle{10.1 Why Simultaneity Breaks}
\textbf{Key insight:} $\tvar{t} = \frac{\disp{d}}{\vel{v}}$

\pause
\vspace{0.3cm}

If light \vel{speed} $\vel{c}$ is constant for all observers...

\pause
\vspace{0.3cm}

...and \disp{distance} $\disp{d}$ appears different to different observers...

\pause
\vspace{0.3cm}

\alert{...then \tvar{time} $\tvar{t}$ must also be different!}

\pause
\vspace{0.3cm}

\begin{block}{The Universal Law}
Two events are simultaneous only if an observer measures them as occurring at the same time. Two events are NOT necessarily simultaneous to all observers.
\end{block}
\note{[P0] [ALGEBRA] "Time equals distance over velocity"\\\\
[P1] "If light speed c is constant for all observers"\\\\
[P2] "And distance d appears different to different observers"\\\\
[P3] [THE REVELATION] "Then time t must also be different"\\\\
[P4] "Two events simultaneous to one observer might not be to another"\\\\
[THE WONDER] Time itself is relative - not absolute}
\end{frame}

\begin{frame}
\frametitle{Attempt: Light Travel Time}
\begin{exampleblock}{The Challenge (3 min, silent)}
The sun is $1.50 \times 10^8$ km from Earth. How long does it take light to travel from the sun to Earth?

\vspace{0.3cm}

\textbf{Given:}
\begin{itemize}
\item \disp{Distance} $\disp{d} = 1.50 \times 10^8$ km
\item \vel{Speed of light} $\vel{c} = 3.00 \times 10^8$ m/s
\end{itemize}

\textbf{Find:} Time in seconds and minutes

\vspace{0.3cm}

\textit{Work silently. Convert units carefully.}
\end{exampleblock}
\note{[THE CHALLENGE] Can they calculate light travel time?\\\\
[SAY] "Try this on your own. Watch your units."\\\\
[TIMING] 3-4 min SILENT individual work\\\\
[CIRCULATE] Note who converts km to m correctly\\\\
[WATCH FOR] Students forgetting unit conversion\\\\
[DON'T HELP] Let them struggle with conversion}
\end{frame}

\begin{frame}
\frametitle{Compare: Light Travel Time}
\textbf{Turn and talk (2 min):}

\vspace{0.3cm}

\begin{enumerate}
\item What equation did you use?
\item How did you handle the units (km vs m)?
\item What did you get for your answer?
\end{enumerate}

\vspace{0.5cm}

\pause
\alert{Name wheel:} One pair share your approach (not your answer).
\note{[TIMING] 2-3 min pair discussion\\\\
[CIRCULATE] Listen for unit conversion strategies\\\\
[CHECK] Name wheel: call a pair to share approach\\\\
[EXPECTED APPROACH] Use v equals d over t, solve for t, convert km to m\\\\
[COMMON ERROR] Forgetting to convert kilometers to meters}
\end{frame}

\begin{frame}
\frametitle{Reveal: Light From the Sun}
\textbf{Self-correct in a different color:}

\vspace{0.3cm}

\textbf{Equation:} $\vel{v} = \frac{\disp{d}}{\tvar{t}}$ so $\tvar{t} = \frac{\disp{d}}{\vel{v}}$

\pause
\vspace{0.2cm}

\textbf{Convert units:} $1.50 \times 10^8 \text{ km} \times \frac{10^3 \text{ m}}{1 \text{ km}} = 1.50 \times 10^{11} \text{ m}$

\pause
\vspace{0.2cm}

\textbf{Substitute:} $\tvar{t} = \frac{1.50 \times 10^{11} \text{ m}}{3.00 \times 10^8 \text{ m/s}}$

\pause
\vspace{0.2cm}

$$\boxed{\tvar{t} = 500 \text{ s} = 8 \text{ min } 20 \text{ s}}$$

\pause
\textbf{Check:} Sunlight takes 8 minutes to reach Earth. When you see a sunspot, it happened 8 minutes ago!
\note{[P0] [ALGEBRA] "Time equals distance over velocity"\\\\
[P1] "Convert: 1.50 times 10 to the 8 km times 10 to the 3 equals 1.50 times 10 to the 11 m"\\\\
[P2] "Substitute into equation"\\\\
[P3] [ANSWER] "500 seconds equals 8 minutes 20 seconds"\\\\
[P4] [THE WONDER] "When you see a sunspot, it happened 8 minutes ago"\\\\
- Light from nearest star takes 4.2 years}
\end{frame}

\section{Consequences of Special Relativity}

\begin{frame}
\frametitle{Learning Objectives}
\begin{block}{By the end of this section, you will be able to:}
\begin{itemize}
\item \textbf{10.2:} Describe time dilation, length contraction, and relativistic momentum \pause
\item \textbf{10.2:} Explain mass-energy equivalence \pause
\item \textbf{10.2:} Perform calculations involving relativistic effects
\end{itemize}
\end{block}
\note{[P0] "Three objectives for consequences"\\\\
[P1] "First: time dilation, length contraction, relativistic momentum"\\\\
[P2] "Second: E equals m c squared"\\\\
[P3] "Third: calculations using gamma factor"\\\\
- These are the mind-bending results}
\end{frame}

\begin{frame}
\frametitle{10.2 The Relativistic Factor}
\begin{block}{The Universal Factor}
\begin{center}
\Large $\boxed{\gamma = \frac{1}{\sqrt{1 - \frac{\vel{v}^2}{\vel{c}^2}}}}$
\end{center}
$\gamma$ (gamma) is the relativistic factor that appears in ALL relativistic effects.
\end{block}

\pause
\vspace{0.3cm}

\textbf{When $\vel{v} \ll \vel{c}$:} $\gamma \approx 1$ (classical physics works)

\pause
\textbf{When $\vel{v} \approx \vel{c}$:} $\gamma \gg 1$ (relativistic effects dominate)
\note{[P0] [THE REVELATION] "Gamma is the master equation of relativity"\\\\
[ALGEBRA] "Gamma equals 1 over square root of 1 minus v squared over c squared"\\\\
[P1] "When v much less than c: gamma approximately 1 - classical physics works"\\\\
[P2] "When v approaches c: gamma much greater than 1 - weird stuff happens"\\\\
[THE WONDER] One equation controls time, space, and momentum}
\end{frame}

\begin{frame}
\frametitle{10.2 Testing the Relativistic Factor}
\textbf{Try some values for $\vel{v}$:}

\pause
\vspace{0.3cm}

\begin{itemize}
\item Airliner: $\vel{v} = 225$ m/s $\rightarrow$ $\gamma = 1.0000000003$ \pause
\item Earth's orbit: $\vel{v} = 2.98 \times 10^4$ m/s $\rightarrow$ $\gamma = 1.000000005$ \pause
\item Particle accelerator: $\vel{v} = 2.99 \times 10^8$ m/s $\rightarrow$ $\gamma = 7.1$
\end{itemize}

\pause
\vspace{0.3cm}

\alert{Relativistic effects only matter near light speed!}
\note{[P0] "Let's test gamma with real speeds"\\\\
[P1] "Airliner: gamma equals 1.0000000003 - negligible"\\\\
[P2] "Earth orbit: gamma equals 1.000000005 - still negligible"\\\\
[P3] "Particle accelerator at 99.7 percent c: gamma equals 7.1 - huge!"\\\\
[P4] [THE REVELATION] "Relativistic effects only matter near light speed"\\\\
[THE HUMILITY] That's why humans didn't notice for thousands of years}
\end{frame}

\begin{frame}
\frametitle{10.2 Time Dilation}
\begin{block}{The Law of Time}
\begin{center}
\Large $\boxed{\Delta \tvar{t} = \gamma \Delta \tvar{t_0}}$
\end{center}
\tvar{Time} passes MORE SLOWLY for an observer moving relative to you.
\end{block}

\pause
\vspace{0.3cm}

\begin{itemize}
\item $\Delta \tvar{t_0}$ = proper \tvar{time} (measured by moving observer) \pause
\item $\Delta \tvar{t}$ = dilated \tvar{time} (measured by stationary observer) \pause
\item $\Delta \tvar{t} > \Delta \tvar{t_0}$ always
\end{itemize}
\note{[P0] [THE REVELATION] "Time dilation: time passes more slowly for moving observer"\\\\
[ALGEBRA] "Delta t equals gamma times delta t-zero"\\\\
[P1] "Delta t-zero: proper time measured by moving observer"\\\\
[P2] "Delta t: dilated time measured by stationary observer"\\\\
[P3] "Delta t is always greater than delta t-zero"\\\\
[THE WONDER] Moving clocks run slow - proven by atomic clocks on satellites}
\end{frame}

\begin{frame}
\frametitle{10.2 The Astronaut's Clock}
\begin{figure}
\centering
\includegraphics[width=0.8\textwidth,height=0.5\textheight,keepaspectratio]{phys11-relativity-fig10-5.jpg}
\caption{Light crossing a moving spacecraft}
\end{figure}

\pause
\textbf{Astronaut measures:} \tvar{Time} $\Delta \tvar{t_0}$ (shorter path)

\pause
\textbf{Earth observer measures:} \tvar{Time} $\Delta \tvar{t}$ (longer path)

\pause
\alert{Same light, different distances, different times!}
\note{[Fig 10.5: Dual-perspective frames showing light path] "Use geometry here - sketch the paths on board if needed. Show how Pythagorean theorem connects vertical vs diagonal paths. This visual proof makes time dilation inevitable, not mysterious."\\\\
[P0] "Astronaut measures light crossing her ship - vertical path"\\\\
[P1] "Astronaut measures shorter time delta t-zero"\\\\
[P2] "Earth observer sees diagonal path - longer distance"\\\\
[P3] [THE CONFLICT] "Same light speed, different distances, so different times"\\\\
[THE REVELATION] Distance affects time - they're connected}
\end{frame}

\begin{frame}
\frametitle{10.2 The Twin Paradox}
\textbf{Thought experiment:}

\vspace{0.3cm}

Twin A travels to a distant star at near light speed

\pause
Twin B stays on Earth

\pause
\vspace{0.3cm}

When Twin A returns...

\pause
\alert{Twin B has aged much more than Twin A!}

\pause
\vspace{0.3cm}

\begin{exampleblock}{Real-World Confirmation}
Atomic clocks on GPS satellites run slower than Earth clocks. GPS must correct for time dilation to give accurate positioning.
\end{exampleblock}
\note{[P0] "Twin paradox: one twin travels at near light speed"\\\\
[P1] "Other twin stays on Earth"\\\\
[P2] "When traveling twin returns..."\\\\
[P3] [THE REVELATION] "Earth twin has aged much more - time dilation is real"\\\\
[P4] [THE CONNECTION - Digital Archetype] "GPS satellites must correct for this"\\\\
[THE WONDER] Time travel to the future is possible - just travel fast enough}
\end{frame}

\begin{frame}
\frametitle{10.2 Length Contraction}
\begin{block}{The Law of Length}
\begin{center}
\Large $\boxed{\disp{L} = \frac{\disp{L_0}}{\gamma} = \disp{L_0}\sqrt{1 - \frac{\vel{v}^2}{\vel{c}^2}}}$
\end{center}
Objects appear SHORTER when moving relative to you.
\end{block}

\pause
\vspace{0.3cm}

\begin{itemize}
\item $\disp{L_0}$ = proper \disp{length} (measured at rest) \pause
\item $\disp{L}$ = contracted \disp{length} (measured by moving observer) \pause
\item $\disp{L} < \disp{L_0}$ always
\end{itemize}
\note{[P0] [THE REVELATION] "Length contraction: moving objects appear shorter"\\\\
[ALGEBRA] "L equals L-zero over gamma equals L-zero times square root of 1 minus v squared over c squared"\\\\
[P1] "L-zero: proper length measured at rest"\\\\
[P2] "L: contracted length measured by moving observer"\\\\
[P3] "L is always less than L-zero"\\\\
[THE CONFLICT] Space itself shrinks at high speeds}
\end{frame}

\begin{frame}
\frametitle{10.2 The Road Ahead}
\begin{figure}
\centering
\includegraphics[width=0.7\textwidth,height=0.5\textheight,keepaspectratio]{phys11-relativity-fig10-6.jpg}
\caption{The road ahead}
\end{figure}

\pause
At everyday speeds: You both measure the same distance

\pause
At relativistic speeds: You measure different distances!

\pause
\alert{Because $\vel{v} = \frac{\disp{d}}{\tvar{t}}$ and you agree on $\vel{v}$ but not on $\tvar{t}$, you must also disagree on $\disp{d}$!}
\note{[Fig 10.6: Empty highway in desert] "Use this familiar scene to anchor abstract concept - ask students to imagine measuring this road from a spaceship at 0.9c. Length contraction becomes visceral when applied to known objects."\\\\
[P0] "At everyday speeds: you both measure same distance"\\\\
[P1] "At relativistic speeds: different distances measured"\\\\
[P2] [ALGEBRA] "v equals d over t"\\\\
[P3] [THE REVELATION] "You agree on speed but not time, so you must disagree on distance"\\\\
[THE WONDER] Time dilation and length contraction are two sides of same coin}
\end{frame}

\begin{frame}
\frametitle{Attempt: The Alien Spaceship}
\begin{exampleblock}{The Challenge (3 min, silent)}
An alien spaceship is 50 m long and travels at 95\% of the speed of light. What is the ship's length as measured from Earth?

\vspace{0.3cm}

\textbf{Given:}
\begin{itemize}
\item Proper \disp{length} $\disp{L_0} = 50$ m
\item \vel{Velocity} $\vel{v} = 0.95\vel{c}$
\end{itemize}

\textbf{Find:} Contracted \disp{length} $\disp{L}$

\vspace{0.3cm}

\textit{Use the length contraction formula. Work silently.}
\end{exampleblock}
\note{[THE CHALLENGE] Can they calculate relativistic length contraction?\\\\
[SAY] "Try this on your own. Remember to square the velocity."\\\\
[TIMING] 3-4 min SILENT individual work\\\\
[CIRCULATE] Note who uses gamma vs direct formula\\\\
[WATCH FOR] Students forgetting to square 0.95\\\\
[DON'T HELP] Let them figure out the algebra}
\end{frame}

\begin{frame}
\frametitle{Compare: Spaceship Length}
\textbf{Turn and talk (2 min):}

\vspace{0.3cm}

\begin{enumerate}
\item What formula did you use?
\item Did you calculate $\gamma$ first or use the combined formula?
\item How did you handle $v = 0.95c$?
\end{enumerate}

\vspace{0.5cm}

\pause
\alert{Name wheel:} One pair share your approach (not your answer).
\note{[TIMING] 2-3 min pair discussion\\\\
[CIRCULATE] Listen for different approaches\\\\
[CHECK] Name wheel: call a pair to share approach\\\\
[EXPECTED APPROACH] Use L equals L-zero times square root of 1 minus v squared over c squared\\\\
[COMMON ERROR] Forgetting to square 0.95 before subtracting from 1}
\end{frame}

\begin{frame}
\frametitle{Reveal: The Contracted Spaceship}
\textbf{Self-correct in a different color:}

\vspace{0.3cm}

\textbf{Equation:} $\disp{L} = \disp{L_0}\sqrt{1 - \frac{\vel{v}^2}{\vel{c}^2}}$

\pause
\vspace{0.2cm}

\textbf{Substitute:} $\disp{L} = 50 \text{ m}\sqrt{1 - \frac{(0.95\vel{c})^2}{\vel{c}^2}}$

\pause
\vspace{0.2cm}

\textbf{Simplify:} $\disp{L} = 50 \text{ m}\sqrt{1 - (0.95)^2}$

\pause
\vspace{0.2cm}

$$\disp{L} = 50 \text{ m}\sqrt{1 - 0.9025} = 50 \text{ m}\sqrt{0.0975}$$

\pause
$$\boxed{\disp{L} = 16 \text{ m}}$$

\pause
\textbf{Check:} Ship contracted from 50 m to 16 m - only 32\% of original length!
\note{[P0] [ALGEBRA] "L equals L-zero times square root of 1 minus v squared over c squared"\\\\
[P1] "Substitute: 50 m times square root of 1 minus 0.95 c squared over c squared"\\\\
[P2] "c squared cancels: square root of 1 minus 0.95 squared"\\\\
[P3] "1 minus 0.9025 equals 0.0975"\\\\
[P4] [ANSWER] "L equals 16 meters"\\\\
[P5] [THE WONDER] "Ship contracted to 32 percent of original length - aliens still measure 50 m from inside"}
\end{frame}

\begin{frame}
\frametitle{10.2 Relativistic Momentum}
\begin{block}{The Law of Momentum}
\begin{center}
\Large $\boxed{\mom{p} = \gamma \mass{m}\vel{u}}$
\end{center}
\mom{Momentum} increases without limit as \vel{velocity} approaches $\vel{c}$.
\end{block}

\pause
\vspace{0.3cm}

\begin{itemize}
\item $\mass{m}$ = rest \mass{mass} \pause
\item $\vel{u}$ = \vel{velocity} of object \pause
\item As $\vel{u} \rightarrow \vel{c}$, $\gamma \rightarrow \infty$, so $\mom{p} \rightarrow \infty$
\end{itemize}
\note{[P0] [THE REVELATION] "Relativistic momentum: classical momentum times gamma"\\\\
[ALGEBRA] "p equals gamma m u"\\\\
[P1] "m is rest mass measured when object is at rest"\\\\
[P2] "u is velocity of object"\\\\
[P3] "As u approaches c, gamma approaches infinity, so p approaches infinity"\\\\
[THE WONDER] This is why objects with mass cannot reach light speed - would require infinite momentum}
\end{frame}

\begin{frame}
\frametitle{10.2 The Momentum Barrier}
\begin{figure}
\centering
\includegraphics[width=0.7\textwidth,height=0.5\textheight,keepaspectratio]{phys11-relativity-fig10-7.jpg}
\caption{Relativistic momentum approaches infinity}
\end{figure}

\pause
\alert{As $\vel{v} \rightarrow \vel{c}$, \mom{momentum} $\mom{p} \rightarrow \infty$}

\pause
This is why you can't reach the speed of light!
\note{[Fig 10.7: Asymptotic momentum vs speed graph] "Trace the curve with your finger - show how momentum stays manageable until ~0.6c then explodes. Connect to their calculator work: gamma factor makes this inevitable. Graph proves light speed is physically impossible for massive objects."\\\\
[P0] "Graph shows relativistic momentum vs velocity"\\\\
[P1] [THE REVELATION] "As v approaches c, momentum approaches infinity"\\\\
[P2] [THE CONFLICT] "This is why you can't reach light speed"\\\\
[THE WONDER] Would require infinite energy to accelerate to c\\\\
- Light can travel at c because photons have zero rest mass}
\end{frame}

\begin{frame}
\frametitle{10.2 Mass-Energy Equivalence}
\begin{block}{The Source Code of Energy}
\begin{center}
\Huge $\boxed{\energy{E} = \mass{m}\vel{c}^2}$
\end{center}
\mass{Mass} and \energy{energy} are interchangeable. Matter IS \energy{energy}.
\end{block}

\pause
\vspace{0.3cm}

\begin{itemize}
\item $\energy{E}$ = rest \energy{energy} (joules) \pause
\item $\mass{m}$ = rest \mass{mass} (kg) \pause
\item $\vel{c}$ = \vel{speed of light} ($3.00 \times 10^8$ m/s)
\end{itemize}
\note{[P0] [THE REVELATION] "Einstein's most famous equation"\\\\
[ALGEBRA] "E equals m c squared"\\\\
[P1] "E is rest energy in joules"\\\\
[P2] "m is rest mass in kilograms"\\\\
[P3] "c is speed of light"\\\\
[THE WONDER] The original source of all energy we use is conversion of mass to energy}
\end{frame}

\begin{frame}
\frametitle{10.2 The Power of $\vel{c}^2$}
\textbf{How much \energy{energy} in 1 gram of matter?}

\pause
\vspace{0.3cm}

$\energy{E} = \mass{m}\vel{c}^2 = (0.001 \text{ kg})(3.00 \times 10^8 \text{ m/s})^2$

\pause
$\energy{E} = 9.0 \times 10^{13}$ J

\pause
\vspace{0.3cm}

\alert{That's enough to power 750,000 homes for one hour!}

\pause
\vspace{0.3cm}

\begin{exampleblock}{Comparison}
Burning 1 gram of coal: 24 J\\
Converting 1 gram of \mass{mass} to \energy{energy}: $9.0 \times 10^{13}$ J
\end{exampleblock}
\note{[P0] "How much energy in 1 gram of matter?"\\\\
[P1] [ALGEBRA] "E equals 0.001 kg times 3.00 times 10 to the 8 m/s squared"\\\\
[P2] "E equals 9.0 times 10 to the 13 joules"\\\\
[P3] [THE WONDER] "Enough to power 750,000 homes for one hour"\\\\
[P4] [THE CONNECTION - Digital Archetype] "Burning coal: 24 J. Converting mass: 90 trillion J"\\\\
[THE HUMILITY] c squared is a VERY large number}
\end{frame}

\begin{frame}
\frametitle{10.2 Where Mass Becomes Energy}
\begin{figure}
\centering
\includegraphics[width=0.8\textwidth,height=0.6\textheight,keepaspectratio]{phys11-relativity-fig10-8.jpg}
\caption{The Sun (fusion) and nuclear power plant (fission)}
\end{figure}

\pause
Both convert mass into energy through nuclear reactions.
\note{[Fig 10.8: Sun and power plant side-by-side] "Point out scale difference - sun does fusion (hydrogen to helium), plant does fission (uranium splits). Both lose tiny mass, release enormous energy via E=mc². Connect to binding energy calculation students just did."\\\\
[P0] "The Sun - fusion converts hydrogen to helium"\\\\
[P1] "Nuclear power plant - fission splits uranium"\\\\
[P2] [THE REVELATION] "Both convert mass into energy through E equals m c squared"\\\\
[THE WONDER] Every photon from the Sun comes from matter being converted to energy}
\end{frame}

\begin{frame}
\frametitle{10.2 Nuclear Binding Energy}
\textbf{Example:} Helium nucleus

\pause
\vspace{0.3cm}

Made of: 2 protons + 2 neutrons = 4.0330 u

\pause
Actual mass: 4.0003 u

\pause
\vspace{0.3cm}

\alert{\mass{Mass} defect: 0.0327 u}

\pause
\vspace{0.3cm}

This "missing" \mass{mass} became binding \energy{energy} when the nucleus formed:

$\energy{E} = (5.04 \times 10^{-30} \text{ kg})(3.00 \times 10^8 \text{ m/s})^2 = 4.54 \times 10^{-12}$ J
\note{[P0] "Helium nucleus: 2 protons plus 2 neutrons"\\\\
[P1] "Sum of parts: 4.0330 atomic mass units"\\\\
[P2] "Actual mass: 4.0003 u"\\\\
[P3] "Mass defect: 0.0327 u - where did it go?"\\\\
[P4] [THE REVELATION] "Missing mass became binding energy that holds nucleus together"\\\\
[ALGEBRA] "E equals m c squared gives 4.54 times 10 to the negative 12 joules"\\\\
[THE WONDER] For one gram of helium: 683 billion joules released}
\end{frame}

\begin{frame}
\frametitle{Attempt: Positron-Electron Annihilation}
\begin{exampleblock}{The Challenge (3 min, silent)}
When a positron and electron collide, they annihilate and convert completely to energy. How much energy is released?

\vspace{0.3cm}

\textbf{Given:}
\begin{itemize}
\item Both particles have rest \mass{mass} $\mass{m} = 9.11 \times 10^{-31}$ kg
\item Total \mass{mass}: $2 \times 9.11 \times 10^{-31}$ kg
\end{itemize}

\textbf{Find:} \energy{Energy} $\energy{E}$ in joules

\vspace{0.3cm}

\textit{Use $E = mc^2$. Work silently.}
\end{exampleblock}
\note{[THE CHALLENGE] Can they calculate matter-antimatter annihilation energy?\\\\
[SAY] "Try this on your own. Total annihilation - all mass becomes energy."\\\\
[TIMING] 3-4 min SILENT individual work\\\\
[CIRCULATE] Note who multiplies by 2 for both particles\\\\
[WATCH FOR] Students forgetting to account for both particles\\\\
[DON'T HELP] Let them work through the calculation}
\end{frame}

\begin{frame}
\frametitle{Compare: Annihilation Energy}
\textbf{Turn and talk (2 min):}

\vspace{0.3cm}

\begin{enumerate}
\item Did you account for both particles?
\item What value did you use for $c$?
\item What units did you get for your answer?
\end{enumerate}

\vspace{0.5cm}

\pause
\alert{Name wheel:} One pair share your approach (not your answer).
\note{[TIMING] 2-3 min pair discussion\\\\
[CIRCULATE] Listen for how they handled two particles\\\\
[CHECK] Name wheel: call a pair to share approach\\\\
[EXPECTED APPROACH] Multiply mass by 2, then use E equals m c squared\\\\
[COMMON ERROR] Forgetting to multiply by 2 for both particles}
\end{frame}

\begin{frame}
\frametitle{Reveal: Total Annihilation}
\textbf{Self-correct in a different color:}

\vspace{0.3cm}

\textbf{Equation:} $\energy{E} = \mass{m}\vel{c}^2$

\pause
\vspace{0.2cm}

\textbf{Total \mass{mass}:} $\mass{m} = 2(9.11 \times 10^{-31} \text{ kg}) = 1.822 \times 10^{-30} \text{ kg}$

\pause
\vspace{0.2cm}

\textbf{Substitute:} $\energy{E} = (1.822 \times 10^{-30} \text{ kg})(3.00 \times 10^8 \text{ m/s})^2$

\pause
\vspace{0.2cm}

$$\energy{E} = (1.822 \times 10^{-30})(9.00 \times 10^{16})$$

\pause
$$\boxed{\energy{E} = 1.64 \times 10^{-13} \text{ J}}$$

\pause
\textbf{Check:} Tiny particles, but enormous energy density. This becomes gamma rays!
\note{[P0] [ALGEBRA] "E equals m c squared"\\\\
[P1] "Total mass: 2 times 9.11 times 10 to the negative 31 kg"\\\\
[P2] "Substitute into E equals m c squared"\\\\
[P3] "Multiply: 1.822 times 10 to the negative 30 times 9.00 times 10 to the 16"\\\\
[P4] [ANSWER] "E equals 1.64 times 10 to the negative 13 joules"\\\\
[P5] [THE WONDER] "Complete conversion of matter to energy - becomes gamma rays"\\\\
- This is antimatter annihilation - most efficient energy conversion possible}
\end{frame}

\begin{frame}
\frametitle{10.2 The RHIC Collider}
\begin{figure}
\centering
\includegraphics[width=0.7\textwidth,height=0.5\textheight,keepaspectratio]{phys11-relativity-fig10-9.jpg}
\caption{Brookhaven National Laboratory RHIC}
\end{figure}

\pause
\textbf{Speed:} 99.7\% of light speed $\rightarrow$ $\gamma = 12.9$

\pause
\textbf{Result:} Time dilates by factor of 13, length contracts by factor 13

\pause
\textbf{Goal:} Recreate conditions from the Big Bang!
\note{[Fig 10.9: Aerial view of RHIC facility] "Show students that relativity isn't just theory - this building complex accelerates particles to verify every equation they've learned. Real-world application: testing physics at universe's birth conditions."\\\\
[P0] "RHIC: Relativistic Heavy Ion Collider at Brookhaven"\\\\
[P1] "Accelerates gold nuclei to 99.7 percent light speed - gamma equals 12.9"\\\\
[P2] "Time dilates by factor 13, length contracts by factor 13"\\\\
[P3] [THE WONDER] "Goal: recreate Big Bang conditions - 4 trillion degrees"\\\\
- Smashes protons and neutrons into quarks and gluons\\\\
- Creates quark-gluon plasma - matter at universe's birth}
\end{frame}

\begin{frame}
\frametitle{10.2 Summary of Relativistic Effects}
\begin{block}{The Three Laws}
\begin{align*}
\text{Time Dilation:} \quad & \Delta \tvar{t} = \gamma \Delta \tvar{t_0} \\
\text{Length Contraction:} \quad & \disp{L} = \frac{\disp{L_0}}{\gamma} \\
\text{Mass-Energy:} \quad & \energy{E} = \mass{m}\vel{c}^2
\end{align*}
\end{block}

\pause
\vspace{0.3cm}

\textbf{Where:} $\gamma = \frac{1}{\sqrt{1 - \frac{\vel{v}^2}{\vel{c}^2}}}$

\pause
\vspace{0.3cm}

All controlled by the relativistic factor $\gamma$!
\note{[P0] "Three key equations of special relativity"\\\\
[ALGEBRA] "Delta t equals gamma times delta t-zero, L equals L-zero over gamma, E equals m c squared"\\\\
[P1] "Where gamma equals 1 over square root of 1 minus v squared over c squared"\\\\
[P2] [THE REVELATION] "All controlled by gamma - the master equation"\\\\
[THE WONDER] Three simple equations that shattered our understanding of reality}
\end{frame}

\section{Summary}

\begin{frame}
\frametitle{What You Now Know}
\begin{block}{The Revelations}
\begin{enumerate}
\item The ether doesn't exist - light speed is constant \pause
\item Two postulates: physics is universal, $\vel{c}$ is constant \pause
\item Simultaneity is relative - depends on frame of reference \pause
\item Time dilates: moving clocks run slow \pause
\item Length contracts: moving objects shrink \pause
\item $\energy{E} = \mass{m}\vel{c}^2$: \mass{mass} and \energy{energy} are equivalent
\end{enumerate}
\end{block}
\note{[P0] "Six revelations today"\\\\
[P1] "Ether doesn't exist - light speed is constant"\\\\
[P2] "Two postulates: physics universal, c constant"\\\\
[P3] "Simultaneity is relative"\\\\
[P4] "Time dilates for moving observers"\\\\
[P5] "Length contracts for moving objects"\\\\
[P6] "E equals m c squared - mass is energy"\\\\
[THE WONDER] Einstein changed everything with two simple postulates\\\\
- Name wheel: which was most mind-bending?}
\end{frame}

\begin{frame}[shrink]
\frametitle{Key Equations}
\begin{align}
\gamma &= \frac{1}{\sqrt{1 - \frac{\vel{v}^2}{\vel{c}^2}}} \\
\Delta \tvar{t} &= \gamma \Delta \tvar{t_0} \\
\disp{L} &= \frac{\disp{L_0}}{\gamma} = \disp{L_0}\sqrt{1 - \frac{\vel{v}^2}{\vel{c}^2}} \\
\mom{p} &= \gamma \mass{m}\vel{u} \\
\energy{E} &= \mass{m}\vel{c}^2 \\
\vel{c} &= 3.00 \times 10^8 \text{ m/s}
\end{align}
\note{- These are the foundational equations of special relativity\\\\
- Gamma controls everything: time, length, momentum\\\\
- E equals m c squared: most famous equation in physics\\\\
- Know when to use each equation\\\\
- Questions before we end?}
\end{frame}

\begin{frame}
\frametitle{Why We Can't Reach the Stars}
\textbf{The barrier:}

\vspace{0.3cm}

As $\vel{v} \rightarrow \vel{c}$:
\begin{itemize}
\item $\gamma \rightarrow \infty$ \pause
\item \mom{Momentum} $\rightarrow \infty$ \pause
\item \energy{Energy} required $\rightarrow \infty$
\end{itemize}

\pause
\vspace{0.3cm}

\alert{You can never reach light speed. The universe has a speed limit.}

\pause
\vspace{0.3cm}

\begin{exampleblock}{The Mental Model}
The faster you go, the more energy you need. At light speed, you'd need infinite energy. Impossible.
\end{exampleblock}
\note{[P0] "As v approaches c, gamma approaches infinity"\\\\
[P1] "Momentum approaches infinity"\\\\
[P2] "Energy required approaches infinity"\\\\
[P3] [THE REVELATION] "You can never reach light speed - universe has speed limit"\\\\
[P4] [THE CONNECTION - Digital Archetype] "Like trying to reach 100 percent on asymptotic curve"\\\\
[THE WONDER] This is why we can't travel to distant star systems - not yet anyway}
\end{frame}

\begin{frame}
\frametitle{Homework}
\begin{center}
\Large
Complete the assigned problems\\[0.3cm]
posted on the LMS
\end{center}
\note{[SAY] "Homework is posted on the LMS"\\\\
[TIMING] Due date: check LMS\\\\
[CHECK] Questions before we end?\\\\
[TRANSITION] Next class: we'll explore quantum mechanics - where things get even weirder}
\end{frame}

\end{document}
