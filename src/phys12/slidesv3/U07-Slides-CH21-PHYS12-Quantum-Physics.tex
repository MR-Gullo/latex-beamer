\documentclass{beamer}
\usepackage{../../../shared/templates/ds9_theme}
\usepackage{../../../shared/templates/semantic-physics-colors}
\usepackage[overridenote]{pdfpc}
\graphicspath{{../images/}{../../shared/images/}}

\title[When Light Breaks the Rules]{PHYS12 CH:21 When Light Breaks the Rules}
\subtitle{The Quantum Revolution}
\author[Mr. Gullo]{Mr. Gullo}
\date[December 2025]{December 2025}

\begin{document}

\frame{\titlepage
\note{[THE HOOK] Today light stops being a wave.\\\\
- For 200 years, we thought we understood light\\\\
- Then Planck and Einstein discovered light breaks its own rules\\\\
[THE WONDER] This discovery launched the electronics revolution - from transistors to solar panels\\\\
- Everything in your phone depends on today's lesson}
}

\begin{frame}
\frametitle{Outline}
\tableofcontents
\end{frame}

\section{Introduction}

\begin{frame}
\frametitle{Alice's Rabbit Hole}
\begin{figure}
\centering
\includegraphics[width=0.7\textwidth,height=0.5\textheight,keepaspectratio]{phys11-quantum-fig21-1.jpg}
\end{figure}

\pause
\begin{center}
\Large In Wonderland, nothing behaves as expected...\\
\textit{but there's still an underlying logic.}
\end{center}
\note{[Fig 21.1: Alice at Mad Hatter's tea party] "Wonderland archetype - familiar setting, alien rules"\\\\
[P0] "Alice follows a rabbit into a world where the rules change"\\\\
[P1] [THE CONNECTION - Digital Archetype] "Like Alice, we're about to discover reality has hidden rules"\\\\
- "Down the quantum rabbit hole, light acts like particles, particles act like waves"\\\\
[THE HOOK] Quantum mechanics is the Wonderland of physics\\\\
[THE WONDER] But there's still beautiful logic underneath - and it built your phone}
\end{frame}

\begin{frame}
\frametitle{The Illusion of Understanding}
\begin{alertblock}{What Your Brain Gets Wrong}
\textbf{You think:} Light is a wave. Waves are continuous.\\
\textbf{Reality:} Light arrives in discrete chunks called photons.
\end{alertblock}

\pause
\vspace{0.3cm}

This isn't intuition failing at extreme speeds or sizes.

\pause
\vspace{0.3cm}

\alert{This is everyday light fooling us constantly.}
\note{[P0] "You've been lied to about light"\\\\
[P1] "Light is a wave - we've proven this with interference patterns"\\\\
[P2] [THE CONFLICT] "But light also arrives in discrete packets like bullets"\\\\
[THE HUMILITY] This confused the greatest minds for decades\\\\
[THE WONDER] Today you'll learn how Planck and Einstein solved the mystery}
\end{frame}

\section{Planck and Quantization}

\begin{frame}
\frametitle{Learning Objectives}
\begin{block}{By the end of this section, you will be able to:}
\begin{itemize}
\item \textbf{21.1:} Describe blackbody radiation and the ultraviolet catastrophe \pause
\item \textbf{21.1:} Define quantum states and calculate photon energies \pause
\item \textbf{21.1:} Explain how photon energies vary across the EM spectrum
\end{itemize}
\end{block}
\note{[P0] "Three objectives for Planck and quantization"\\\\
[P1] "First: what is a blackbody and why did it break classical physics"\\\\
[P2] "Second: quantum states - energy comes in chunks"\\\\
[P3] "Third: photon energy across the spectrum"\\\\
- Assessment: problem set on LMS}
\end{frame}

\begin{frame}
\frametitle{21.1 The T-Shirt Mystery}
\begin{exampleblock}{The Mental Model}
Why is wearing a black T-shirt on a hot day unbearable,\\
while a white shirt keeps you cool?
\end{exampleblock}

\pause
\vspace{0.3cm}

Black shirts absorb and re-emit more radiation.

\pause
\vspace{0.3cm}

A perfect black object that absorbs ALL radiation is called a \textbf{blackbody}.
\note{[P0] "You know this from experience"\\\\
[P1] "Black absorbs heat, white reflects it"\\\\
[P2] "A blackbody absorbs 100 percent of incident radiation"\\\\
[THE CONNECTION - Kinetic Archetype] "Athletes know to wear white in summer games"\\\\
[THE REVELATION] Studying blackbodies led to quantum mechanics}
\end{frame}

\begin{frame}
\frametitle{21.1 The Blackbody Spectrum}
\begin{figure}
\centering
\includegraphics[width=0.8\textwidth,height=0.6\textheight,keepaspectratio]{phys11-quantum-fig21-2.jpg}
\end{figure}

\pause
\textbf{Key observations:}
\begin{itemize}
\item Hotter objects radiate MORE total energy (area under curve)
\item Peak wavelength SHIFTS leftward as temperature increases
\end{itemize}
\note{[Fig 21.2: Radiation intensity vs wavelength for 3 temps] "Three curves shift left as temp rises - peak moves to shorter wavelengths"\\\\
[P0] "Three temperature curves - study them carefully"\\\\
[P1] "At 3000 K: peaks in infrared around 1000 nm"\\\\
- "At 6000 K: peaks in visible - white hot, like the Sun"\\\\
- "Area under curve: total energy radiated increases with temperature"\\\\
[THE WONDER] Same pattern for all objects - your body, the sun, distant stars\\\\
- Turn and talk: what trends do you see?}
\end{frame}

\begin{frame}
\frametitle{21.1 The Ultraviolet Catastrophe}
\begin{figure}
\centering
\includegraphics[width=0.7\textwidth,height=0.5\textheight,keepaspectratio]{phys11-quantum-fig21-3.jpg}
\end{figure}

\pause
\begin{alertblock}{The Classical Prediction}
Theory predicted INFINITE energy at short wavelengths.\\
Reality: energy drops back down. \textbf{Physics was broken.}
\end{alertblock}
\note{[Fig 21.3: Classical theory curve goes to infinity at short wavelengths] "Dashed line predicts catastrophic UV output - reality curves back down"\\\\
[P0] "Classical physics prediction: dashed line shooting to infinity"\\\\
[P1] [THE CONFLICT] "Theory said infinite energy in UV - the ultraviolet catastrophe"\\\\
- "But experiments showed energy drops at short wavelengths"\\\\
- "Your coffee mug should kill you with UV radiation - but it doesn't"\\\\
[THE HUMILITY] The best physicists couldn't explain this disconnect\\\\
[THE REVELATION] Max Planck solved it with a radical idea}
\end{frame}

\begin{frame}
\frametitle{21.1 Planck's Revolution}
\begin{block}{The Source Code: Quantized Energy}
\begin{center}
\Large $\boxed{\energy{E} = \particles{n}\pConst{h}\freq{f}}$
\end{center}
\energy{Energy} is not continuous. \energy{Energy} exists only in discrete chunks.
\end{block}

\pause
\vspace{0.3cm}

\begin{itemize}
\item $\particles{n}$ = integer (0, 1, 2, 3...)
\item $\pConst{h} = 6.626 \times 10^{-34}$ J·s (Planck's constant)
\item $\freq{f}$ = frequency
\end{itemize}

\pause
\vspace{0.3cm}

\textbf{Key insight:} Energy comes in quantum packets, not continuous streams.
\note{[P0] "Planck's radical idea: energy is quantized"\\\\
[P1] [THE REVELATION] "E equals n h f - energy comes in discrete packets"\\\\
[P2] "Like digital steps on a staircase, not a smooth ramp"\\\\
[THE CONNECTION - Digital Archetype] "Digital clock jumps 11:14 to 11:15 - no 11:14.5"\\\\
[THE WONDER] This equation won Planck the 1918 Nobel Prize}
\end{frame}

\begin{frame}
\frametitle{21.1 The Coin Analogy}
\begin{exampleblock}{Imagine This...}
Pour coins through a funnel.\\
More pennies pass through than quarters.\\
\textit{Why? Quarters have higher value AND larger size.}
\end{exampleblock}

\pause
\vspace{0.3cm}

Same with photons:
\begin{itemize}
\item High frequency = high energy = less probable at low temperatures
\item Low frequency = low energy = more probable at low temperatures
\end{itemize}

\pause
\vspace{0.3cm}

This \textit{probability factor} explains why the blackbody curve drops at short wavelengths.
\note{[P0] "Feynman simplicity - concrete anchor"\\\\
[P1] "High frequency photons are like quarters - bigger energy packets"\\\\
- "Low frequency photons are like pennies - smaller energy packets"\\\\
[P2] "At fixed temperature, more low-energy photons can exist"\\\\
[THE REVELATION] Probability factor prevents ultraviolet catastrophe\\\\
[THE WONDER] Interplay between classical wave model and quantum probability creates the curve}
\end{frame}

\begin{frame}
\frametitle{21.1 Quantization Everywhere}
\textbf{Quantum states exist in:}
\begin{itemize}
\item Standing waves on a string (harmonics) \pause
\item Steps on a staircase (discrete heights) \pause
\item Atoms (you can't have 0.5 atoms) \pause
\item Electric charge (multiples of $e$)
\end{itemize}

\pause
\vspace{0.3cm}

\begin{alertblock}{The Shock}
But energy was thought to be \textit{continuous}.\\
Planck proved otherwise.
\end{alertblock}
\note{[P0] "Quantization isn't new on macroscopic scale"\\\\
[P1] "Guitar strings have discrete harmonics - first, second, third mode"\\\\
[P2] "Stairs have discrete steps - you can't stand on step 2.5"\\\\
[P3] "You can't have half an atom or 1.37 cents"\\\\
[P4] "Charge comes in multiples of electron charge"\\\\
[P5] [THE CONFLICT] "But energy? We thought it was smooth and continuous"\\\\
[THE REVELATION] Planck showed energy is quantized at atomic scales too}
\end{frame}

\begin{frame}
\frametitle{21.1 Max Planck}
\begin{figure}
\centering
\includegraphics[width=0.5\textwidth,height=0.5\textheight,keepaspectratio]{phys11-quantum-fig21-4.jpg}
\end{figure}

\pause
\begin{center}
\textit{"I was ready to sacrifice any of my previous convictions about physics."}\\
- Max Planck, on quantization
\end{center}
\note{[Fig 21.4: Max Planck at desk reading] "Human face of revolution - shows scientist as scholar, not celebrity"\\\\
[P0] "Max Planck revolutionized physics by questioning everything"\\\\
[P1] "Born 1858, Germany - studied thermodynamics"\\\\
- "Solved blackbody problem in 1900 - reluctantly accepted quantization"\\\\
- "1918 Nobel Prize - father of quantum mechanics"\\\\
[THE HUMILITY] "He was afraid of his own idea - it was too radical"\\\\
[THE WONDER] He unlocked a door he was afraid to walk through - but the world followed}
\end{frame}

\begin{frame}
\frametitle{21.1 The Photon Energy Equation}
\begin{block}{Nature's Rule for Light}
\begin{center}
\Large $\boxed{\energy{E} = \pConst{h}\freq{f} = \frac{\pConst{h}\vel{c}}{\wavelen{\lambda}}}$
\end{center}
\end{block}

\pause
\vspace{0.3cm}

\textbf{Key insights:}
\begin{itemize}
\item Higher \freq{frequency} $\rightarrow$ higher \energy{energy} per photon
\item Shorter \wavelen{wavelength} $\rightarrow$ higher \energy{energy} per photon
\item Planck's constant $\pConst{h}$ links \energy{energy} to \freq{frequency}
\end{itemize}
\note{[P0] "Two forms of photon energy equation"\\\\
[P1] "Higher frequency means higher energy per photon"\\\\
- "Shorter wavelength means higher energy per photon"\\\\
- "h is the conversion factor between frequency and energy"\\\\
[THE WONDER] Same equation for radio waves and gamma rays\\\\
- Universal law governing all electromagnetic radiation}
\end{frame}

\begin{frame}
\frametitle{Attempt: Photon Energy of a Light Bulb}
\begin{exampleblock}{The Challenge (3 min, silent)}
A 100-W light bulb converts 10\% of its power to visible light with average wavelength 580 nm. Calculate the number of visible photons emitted per second.

\vspace{0.3cm}

\textbf{Given:}
\begin{itemize}
\item Power in visible light: 10.0 J/s
\item Wavelength: $\wavelen{\lambda} = 580$ nm
\item $\pConst{h} = 6.626 \times 10^{-34}$ J·s, $\vel{c} = 3.00 \times 10^8$ m/s
\end{itemize}

\textbf{Find:} Photons per second

\vspace{0.3cm}

\textit{Work silently. It's okay to get stuck.}
\end{exampleblock}
\note{[THE CHALLENGE] Can you count individual particles of light?\\\\
[SAY] "Try this on your own. Choose your equation carefully."\\\\
[TIMING] 3-4 min SILENT individual work\\\\
[CIRCULATE] Note who uses E equals h f vs E equals h c over lambda\\\\
[WATCH FOR] Unit conversion errors with nanometers, strategy for finding photon count\\\\
[DON'T HELP] Let them struggle with multi-step problem}
\end{frame}

\begin{frame}
\frametitle{Compare: Light Bulb Strategy}
\textbf{Turn and talk (2 min):}

\vspace{0.3cm}

\begin{enumerate}
\item Which form of photon energy equation did you use?
\item How did you find energy per photon?
\item What's your strategy for finding number of photons?
\end{enumerate}

\vspace{0.5cm}

\pause
\alert{Name wheel:} One pair share your approach (not your answer).
\note{[TIMING] 2-3 min pair discussion\\\\
[CIRCULATE] Listen for E equals h c over lambda, then divide total power by E per photon\\\\
[CHECK] Name wheel: call a pair to share\\\\
[EXPECTED APPROACH] Find E per photon using wavelength, then divide 10 J/s by E\\\\
[COMMON ERROR] Forgetting to convert nm to m, or not recognizing this is a two-step problem}
\end{frame}

\begin{frame}
\frametitle{Reveal: Counting Photons}
\textbf{Self-correct in a different color:}

\vspace{0.3cm}

\textbf{Step 1 - \energy{Energy} per photon:}
$$\energy{E} = \frac{\pConst{h}\vel{c}}{\wavelen{\lambda}} = \frac{(6.626 \times 10^{-34})(3.00 \times 10^8)}{580 \times 10^{-9}} = 3.43 \times 10^{-19} \text{ J}$$

\pause
\vspace{0.2cm}

\textbf{Step 2 - Photons per second:}
$$\frac{\text{photons}}{s} = \frac{10.0 \text{ J/s}}{3.43 \times 10^{-19} \text{ J/photon}} = \boxed{2.92 \times 10^{19} \text{ photons/s}}$$

\pause
\textbf{Check:} About 29 billion billion photons every second - no wonder light seems continuous!
\note{[P0] "Self-correct in a different color"\\\\
[P1] [ALGEBRA] "E equals h c over lambda - watch units cancel to get joules"\\\\
[P2] [ALGEBRA] "Divide total power by energy per photon to get photon rate"\\\\
[ANSWER] 2.92 times 10 to the 19 photons per second\\\\
[THE WONDER] Individual photons are invisible to us - but your eye can detect single photons in darkness}
\end{frame}

\begin{frame}
\frametitle{21.1 Blue Fire vs Red Fire}
\begin{exampleblock}{Real-World: Danger by Color}
A blue flame is more dangerous than a red flame.\\
\textit{Why?}
\end{exampleblock}

\pause
\vspace{0.3cm}

Blue light: shorter \wavelen{wavelength} $\rightarrow$ higher \freq{frequency} $\rightarrow$ \alert{higher \energy{energy} per photon}

\pause
Red light: longer \wavelen{wavelength} $\rightarrow$ lower \freq{frequency} $\rightarrow$ lower \energy{energy} per photon

\pause
\vspace{0.3cm}

Each blue photon delivers more energy than each red photon.
\note{[P0] "Turn and talk: why is blue flame more dangerous?"\\\\
[P1] "Blue: wavelength around 450 nm, higher frequency"\\\\
[P2] "Red: wavelength around 700 nm, lower frequency"\\\\
[P3] [ANSWER] "Each blue photon carries MORE energy than each red photon"\\\\
[THE CONNECTION - Kinetic Archetype] "Welders know blue sparks are hotter than red"\\\\
[THE WONDER] Color directly reveals temperature and energy}
\end{frame}

\begin{frame}
\frametitle{21.1 Energy Across the Spectrum}
\begin{center}
\small
\begin{tabular}{lll}
\textbf{Radiation} & \textbf{Wavelength} & \textbf{Energy/photon} \\ \hline
Radio & 1 m & $2 \times 10^{-25}$ J \\
Infrared & 1000 nm & $2 \times 10^{-19}$ J \\
Visible & 500 nm & $4 \times 10^{-19}$ J \\
UV & 100 nm & $2 \times 10^{-18}$ J \\
X-ray & 1 nm & $2 \times 10^{-16}$ J \\
Gamma & 0.01 nm & $2 \times 10^{-14}$ J
\end{tabular}
\end{center}

\pause
\vspace{0.3cm}

\begin{alertblock}{UV Hazard}
UV photons carry enough energy to break DNA bonds ($\approx 1$ eV $= 1.6 \times 10^{-19}$ J).\\
\textbf{That's why UV causes sunburn and cancer.}
\end{alertblock}
\note{[P0] "Photon energy increases dramatically from radio to gamma"\\\\
[P1] [THE REVELATION] "UV and above: enough energy to damage molecules"\\\\
- "DNA breaks with about 1 eV - UV photons exceed this threshold"\\\\
- "Visible light photons are mostly harmless - below DNA damage threshold"\\\\
[THE CONFLICT] Sunlight feels warm and bright but contains invisible UV danger\\\\
[THE WONDER] Your skin knows the difference - sunburn is quantum mechanics in action}
\end{frame}

\section{Photoelectric Effect}

\begin{frame}
\frametitle{Learning Objectives}
\begin{block}{By the end of this section, you will be able to:}
\begin{itemize}
\item \textbf{21.2:} Describe the photoelectric effect and Einstein's explanation \pause
\item \textbf{21.2:} Explain why classical physics couldn't explain the effect \pause
\item \textbf{21.2:} Calculate photoelectron kinetic energy \pause
\item \textbf{21.2:} Describe applications in solar cells and technology
\end{itemize}
\end{block}
\note{[P0] "Four objectives for photoelectric effect"\\\\
[P1] "First: what is the photoelectric effect and Einstein's photon explanation"\\\\
[P2] "Second: why classical wave theory utterly failed"\\\\
[P3] "Third: calculating electron energies using Einstein's equation"\\\\
[P4] "Fourth: real-world applications - solar panels, sensors, cameras"\\\\
[THE REVELATION] This section won Einstein his Nobel Prize - not relativity!}
\end{frame}

\begin{frame}
\frametitle{Light as a Battering Ram}
\begin{center}
\Large Light can knock electrons out of metal.\\
\textit{How does a wave punch like a particle?}
\end{center}

\pause
\begin{alertblock}{The Paradox}
Classical physics: Light is a wave - spread out, continuous energy.\\
Reality: Light ejects electrons instantly, one photon at a time.
\end{alertblock}
\note{[P0] [THE HOOK] "Shine light on metal. Electrons fly out. This is the photoelectric effect."\\\\
[P1] [THE CONFLICT] "Waves don't hit like hammers. Particles do."\\\\
- "Classical theory predicts gradual energy accumulation"\\\\
- "Reality shows instant ejection"\\\\
[THE HUMILITY] This baffled the best minds of the early 1900s\\\\
[THE WONDER] Einstein solved it in 1905 - his miracle year}
\end{frame}

\begin{frame}
\frametitle{21.2 The Photoelectric Setup}
\begin{figure}
\centering
\includegraphics[width=0.7\textwidth,height=0.5\textheight,keepaspectratio]{phys11-quantum-fig21-7.jpg}
\end{figure}

\pause
\textbf{The experiment:}
\begin{itemize}
\item Light strikes metal surface
\item Electrons ejected (photoelectrons)
\item Current measured - proves electrons are moving
\end{itemize}
\note{[Fig 21.7: Incoming radiation striking metal, electrons ejected] "Schematic shows photon arrows striking electron sea - two electrons shown escaping"\\\\
[P0] "Classic apparatus - evacuated tube with metal plate"\\\\
[P1] "Photons hit metal, kick out electrons"\\\\
- "Collector wire measures current - more electrons means more current"\\\\
[THE CONNECTION - Digital Archetype] "This is how digital cameras work - light hits sensor, electrons flow, image captured"\\\\
[ANSWER] Photoelectric effect = light produces electricity}
\end{frame}

\begin{frame}
\frametitle{21.2 Five Mysteries}
\textbf{Classical wave theory predictions:}
\begin{enumerate}
\item Any frequency should eject electrons (just wait long enough) \pause
\item Brighter light should eject higher-energy electrons \pause
\item Electrons should take time to accumulate energy
\end{enumerate}

\pause
\vspace{0.3cm}

\begin{alertblock}{Experimental Reality}
\begin{enumerate}
\item Threshold frequency exists - below it, NO electrons (ever!)
\item Brighter light ejects MORE electrons, not faster ones
\item Ejection is INSTANTANEOUS
\end{enumerate}
\end{alertblock}
\note{[P0] "Classical predictions from wave theory"\\\\
[P1] "Prediction 1: any frequency works if you wait - waves accumulate energy"\\\\
[P2] "Prediction 2: brighter means more energy per electron"\\\\
[P3] "Prediction 3: takes minutes to accumulate enough wave energy"\\\\
[P4] [THE CONFLICT] "Reality violated ALL THREE predictions"\\\\
- "Threshold frequency - some colors never work, no matter how long you wait"\\\\
- "Brightness doesn't change electron speed, just number ejected"\\\\
- "Instant ejection - no delay"\\\\
[THE HUMILITY] This baffled physicists for years - wave theory was wrong\\\\
[THE REVELATION] Einstein solved it with photons}
\end{frame}

\begin{frame}
\frametitle{21.2 Einstein's Photon Model}
\begin{figure}
\centering
\includegraphics[width=0.7\textwidth,height=0.5\textheight,keepaspectratio]{phys11-quantum-fig21-6.jpg}
\end{figure}

\pause
\textbf{Key insight:} Light is not a continuous wave.\\
Light is a stream of \textbf{photons} - discrete packets of energy.

\pause
\vspace{0.3cm}

Each photon energy: $E = hf$
\note{[Fig 21.6: Flashlight beam with ovals showing photon wavelengths, equations E=hf] "Discrete packets visible in beam - different frequencies labeled E=hf and E'=hf'"\\\\
[P0] "Flashlight beam as stream of individual photons"\\\\
[P1] [THE REVELATION] "Light arrives in discrete packets called photons"\\\\
[P2] "Each photon carries energy E equals h f"\\\\
- "Higher frequency photons carry more energy"\\\\
[THE WONDER] This idea won Einstein the 1921 Nobel Prize - not for relativity!\\\\
[THE CONNECTION - Digital Archetype] "Like data packets streaming through internet, not continuous flow"}
\end{frame}

\begin{frame}
\frametitle{21.2 One Photon, One Electron}
\begin{exampleblock}{The Mental Model}
Each photon interacts with ONE electron.\\
If photon energy $< $ binding energy: electron stays bound.\\
If photon energy $\geq $ binding energy: electron escapes!
\end{exampleblock}

\pause
\vspace{0.3cm}

This explains ALL five observations:
\begin{itemize}
\item \textbf{Threshold:} $hf_0 = BE$ (minimum energy to escape)
\item \textbf{Instant:} one photon, one collision
\item \textbf{Brightness:} more photons = more electrons (not faster electrons)
\end{itemize}
\note{[P0] "One-on-one collision: one photon, one electron"\\\\
[P1] "Explains threshold: photon must have minimum energy h f zero equals BE"\\\\
- "Explains instant ejection: single collision event, no accumulation needed"\\\\
- "Explains brightness: more photons means more collisions, more electrons ejected"\\\\
[THE REVELATION] Like pool balls colliding - discrete interactions\\\\
[THE WONDER] Light acts like particles when it interacts with matter}
\end{frame}

\begin{frame}
\frametitle{21.2 The Photoelectric Equation}
\begin{block}{The Law of Energy Exchange}
\begin{center}
\Large $\boxed{\kenergy{KE_e} = \pConst{h}\freq{f} - \energy{BE}}$
\end{center}
\end{block}

\pause
\vspace{0.3cm}

\begin{itemize}
\item $\kenergy{KE_e}$ = kinetic energy of ejected electron
\item $\pConst{h}\freq{f}$ = photon energy
\item $\energy{BE}$ = binding energy (work function)
\end{itemize}

\pause
\vspace{0.3cm}

\textbf{Photon energy budget:}
\begin{enumerate}
\item Pay the escape cost (binding energy $BE$)
\item Leftover becomes kinetic energy
\end{enumerate}
\note{[P0] "Energy conservation in photoelectric effect"\\\\
[P1] "KE equals h f minus BE - Nobel Prize equation"\\\\
- "KE: kinetic energy of freed electron"\\\\
- "h f: incoming photon energy"\\\\
- "BE: binding energy needed to break electron free from metal"\\\\
[P2] "Think of it as an energy budget"\\\\
- "Photon arrives with total energy h f"\\\\
- "First expense: pay binding energy to escape metal"\\\\
- "Remainder: kinetic energy of ejected electron"\\\\
[THE REVELATION] Energy is conserved in quantum collisions}
\end{frame}

\begin{frame}
\frametitle{21.2 Kinetic Energy vs Frequency}
\begin{figure}
\centering
\includegraphics[width=0.7\textwidth,height=0.5\textheight,keepaspectratio]{phys11-quantum-fig21-8.jpg}
\end{figure}

\pause
\textbf{Key features:}
\begin{itemize}
\item Below $\freq{f_0}$: NO electrons ejected ($\kenergy{KE} = 0$)
\item Above $\freq{f_0}$: $\kenergy{KE}$ increases linearly with $\freq{f}$
\item Slope = $\pConst{h}$ (Planck's constant!)
\end{itemize}
\note{[Fig 21.8: KE vs frequency graph, diagonal line starting at f0, labeled KE=hf-BE] "Linear relationship proves photon model - slope equals h, x-intercept equals BE/h"\\\\
[P0] "Graph of electron kinetic energy vs incident light frequency"\\\\
[P1] "Below threshold frequency f-zero: zero kinetic energy - no ejection"\\\\
- "Above threshold: KE increases linearly with frequency"\\\\
- "Slope of line is Planck's constant h"\\\\
[THE REVELATION] This graph let scientists measure h experimentally\\\\
- "Einstein predicted this graph in 1905"\\\\
- "Millikan confirmed it experimentally in 1916"\\\\
[THE WONDER] Quantum mechanics makes testable predictions}
\end{frame}

\begin{frame}
\frametitle{Attempt: Violet Light and Calcium}
\begin{exampleblock}{The Challenge (3 min, silent)}
Violet light with wavelength 420 nm strikes calcium metal.\\
Binding energy of calcium: BE = 2.71 eV

\vspace{0.3cm}

\textbf{Given:}
\begin{itemize}
\item $\wavelen{\lambda} = 420$ nm $= 4.20 \times 10^{-7}$ m
\item $\energy{BE} = 2.71$ eV
\item $\pConst{h} = 6.626 \times 10^{-34}$ J·s, $\vel{c} = 3.00 \times 10^8$ m/s
\item Conversion: $1 \text{ eV} = 1.60 \times 10^{-19}$ J
\end{itemize}

\textbf{Find:} (a) Photon energy in eV\\
\hspace{1.7cm} (b) Maximum kinetic energy of ejected electron in eV

\vspace{0.3cm}

\textit{Work silently. It's okay to get stuck.}
\end{exampleblock}
\note{[THE CHALLENGE] Can you predict electron energy from light wavelength?\\\\
[SAY] "Two-part problem. Start with photon energy, then apply Einstein's equation."\\\\
[TIMING] 3-4 min SILENT individual work\\\\
[CIRCULATE] Note who converts J to eV correctly, who remembers two-step process\\\\
[WATCH FOR] Unit consistency - mixing J and eV is common error\\\\
[DON'T HELP] Let them work through conversion and energy budget}
\end{frame}

\begin{frame}
\frametitle{Compare: Photoelectric Strategy}
\textbf{Turn and talk (2 min):}

\vspace{0.3cm}

\begin{enumerate}
\item How did you find photon energy from wavelength?
\item How did you convert joules to electron volts?
\item Which equation did you use for part (b)?
\end{enumerate}

\vspace{0.5cm}

\pause
\alert{Name wheel:} One pair share your approach (not your answer).
\note{[TIMING] 2-3 min pair discussion\\\\
[CIRCULATE] Listen for E equals h c over lambda, then divide by 1.6 times 10 to the negative 19\\\\
[CHECK] Name wheel: call a pair\\\\
[EXPECTED APPROACH] Calculate photon energy in J, convert to eV, then use KE equals h f minus BE\\\\
[COMMON ERROR] Forgetting to convert J to eV before using Einstein's equation, or mixing units}
\end{frame}

\begin{frame}
\frametitle{Reveal: Photoelectric Calculation}
\textbf{Self-correct in a different color:}

\vspace{0.3cm}

\textbf{Part (a) - Photon \energy{energy}:}
$$\energy{E} = \frac{\pConst{h}\vel{c}}{\wavelen{\lambda}} = \frac{(6.626 \times 10^{-34})(3.00 \times 10^8)}{4.20 \times 10^{-7}} = 4.74 \times 10^{-19} \text{ J}$$

\pause
Convert to eV: $\energy{E} = \frac{4.74 \times 10^{-19}}{1.60 \times 10^{-19}} = \boxed{2.96 \text{ eV}}$

\pause
\vspace{0.3cm}

\textbf{Part (b) - Electron \kenergy{kinetic energy}:}
$$\kenergy{KE_e} = \pConst{h}\freq{f} - \energy{BE} = 2.96 \text{ eV} - 2.71 \text{ eV} = \boxed{0.25 \text{ eV}}$$

\pause
\textbf{Check:} Small KE - photon energy barely exceeds threshold!
\note{[P0] "Self-correct in a different color"\\\\
[P1] [ALGEBRA] "E equals h c over lambda - substitute and calculate"\\\\
[P2] "Convert J to eV: divide by 1.6 times 10 to the negative 19"\\\\
[ANSWER] "Photon energy is 2.96 eV"\\\\
[P3] [ALGEBRA] "KE equals photon energy minus binding energy"\\\\
[ANSWER] "Kinetic energy is 0.25 eV"\\\\
[P4] "Small number - electron barely escapes - 420 nm is near threshold for calcium"\\\\
[THE WONDER] This 0.25 eV is how solar cells work - one photon at a time creating current}
\end{frame}

\begin{frame}
\frametitle{21.2 Solar Cells: Photons to Power}
\begin{figure}
\centering
\includegraphics[width=0.7\textwidth,height=0.5\textheight,keepaspectratio]{phys11-quantum-fig21-9.jpg}
\end{figure}

\pause
\textbf{How photovoltaic cells work:}
\begin{itemize}
\item Photons strike semiconductor (usually silicon)
\item Electrons freed via photoelectric effect
\item Electrons flow as electric current
\item Current powers devices or charges batteries
\end{itemize}
\note{[Fig 21.9: Blue square solar cell with diagonal corner notches] "Physical appearance - students recognize real solar panels from rooftops"\\\\
[P0] "Solar cell: photoelectric effect in action"\\\\
[P1] "Sunlight photons hit silicon semiconductor"\\\\
- "Photons with energy greater than silicon's bandgap free electrons"\\\\
- "Freed electrons create electric current"\\\\
- "Current powers devices directly or charges battery for later use"\\\\
[THE CONNECTION - Digital Archetype] "Solar calculators, satellites, International Space Station - all powered by photoelectric effect"\\\\
[THE WONDER] Earth receives enough sunlight each hour to power the globe for a year - quantum mechanics can help us capture it}
\end{frame}

\begin{frame}
\frametitle{21.2 Photoelectric Devices Everywhere}
\textbf{Applications you use daily:}
\begin{itemize}
\item \textbf{Digital cameras:} Light hits sensor pixels, electrons counted as brightness \pause
\item \textbf{Motion sensors:} Automatic doors, elevators - beam broken triggers response \pause
\item \textbf{Light meters:} Camera exposure control adjusts aperture automatically \pause
\item \textbf{Streetlights:} Photocell detects darkness, switches on at dusk
\end{itemize}

\pause
\vspace{0.3cm}

All rely on one quantum principle: \alert{photons create electric current}
\note{[P0] "Photoelectric effect is everywhere in modern technology"\\\\
[P1] "Digital cameras: each pixel is a tiny solar cell counting photons"\\\\
[P2] "Automatic doors: photocell detects when beam is blocked by person"\\\\
[P3] "Camera light meters: photoelectric current adjusts aperture for correct exposure"\\\\
[P4] "Streetlights: photocell switches circuit when photon rate drops at sunset"\\\\
[THE WONDER] Quantum mechanics runs your daily life without you noticing}
\end{frame}

\section{Dual Nature of Light}

\begin{frame}
\frametitle{Learning Objectives}
\begin{block}{By the end of this section, you will be able to:}
\begin{itemize}
\item \textbf{21.3:} Describe particle-wave duality \pause
\item \textbf{21.3:} Calculate photon momentum using $p = h/\lambda$ \pause
\item \textbf{21.3:} Explain Compton scattering as photon-electron collisions
\end{itemize}
\end{block}
\note{[P0] "Final section: the ultimate quantum paradox"\\\\
[P1] "Light is both wave AND particle - simultaneously, not sequentially"\\\\
[P2] "Photons carry momentum despite having zero mass"\\\\
[P3] "Compton scattering: photons collide like billiard balls"\\\\
[THE WONDER] This duality isn't just light - electrons, atoms, even you have wavelike properties}
\end{frame}

\begin{frame}
\frametitle{The Great Contradiction}
\begin{center}
\Large Light is a wave.\\
\textit{Light is a particle.}\\
\vspace{0.3cm}
Both statements are true.
\end{center}

\pause
\begin{alertblock}{Civilian View vs Reality}
\textbf{Civilian:} "Something is either a wave or a particle. Pick one."\\
\textbf{Physicist:} "Light is both. It depends on how you look at it."
\end{alertblock}
\note{[P0] [THE HOOK] "Welcome to the deepest weirdness of quantum mechanics"\\\\
- "Light diffracts and interferes - pure wave behavior"\\\\
- "Light ejects electrons and carries momentum - pure particle behavior"\\\\
[P1] [THE CONFLICT] "Our brains evolved in a macroscopic world where things are clearly wave or particle"\\\\
- "Quantum reality refuses our categories"\\\\
[THE HUMILITY] "Einstein himself struggled with this paradox his entire life"\\\\
[THE WONDER] This isn't a flaw in our understanding - it's the actual nature of reality}
\end{frame}

\begin{frame}
\frametitle{21.3 Two Faces of Light}
\begin{columns}
\column{0.5\textwidth}
\textbf{Wave Evidence:}
\begin{itemize}
\item Diffraction through slits
\item Interference patterns
\item Polarization
\item Wavelength and frequency
\end{itemize}

\column{0.5\textwidth}
\textbf{Particle Evidence:}
\begin{itemize}
\item Photoelectric effect
\item Discrete energy $E = hf$
\item Photon momentum
\item Compton scattering
\end{itemize}
\end{columns}

\vspace{0.3cm}
\pause
\begin{block}{The Resolution}
Light exhibits \textbf{wave-particle duality}. Which aspect you observe depends on your experiment.
\end{block}
\note{[P0] "Two columns - same phenomenon, two faces"\\\\
- "Left side: 200 years of wave experiments - interference, diffraction"\\\\
- "Right side: quantum experiments from 1900s - photons, discrete energy"\\\\
[P1] [THE REVELATION] "Both are true. Light is neither pure wave nor pure particle"\\\\
- "It's something deeper we don't have everyday language for"\\\\
[ANSWER] Duality is fundamental - not our confusion, but nature's design\\\\
[THE WONDER] Every electromagnetic wave from radio to gamma rays shows this duality}
\end{frame}

\begin{frame}
\frametitle{21.3 Massless Momentum}
\begin{block}{The Source Code: Photon Momentum}
\begin{center}
\Large $\boxed{\mom{p} = \frac{\pConst{h}}{\wavelen{\lambda}}}$
\end{center}
Photons carry \mom{momentum} despite having zero \mass{mass}.
\end{block}

\pause
\vspace{0.3cm}

\textbf{Key insights:}
\begin{itemize}
\item Shorter \wavelen{wavelength} = higher \mom{momentum}
\item Blue photons push harder than red photons
\item Light exerts radiation pressure
\end{itemize}

\pause
\vspace{0.3cm}

\textbf{Real effects:}
\begin{itemize}
\item Comet tails point away from Sun (photon pressure)
\item Solar sails for spacecraft propulsion
\item Optical tweezers manipulate cells using light
\end{itemize}
\note{[P0] [THE REVELATION] "Momentum without mass - this shatters classical physics"\\\\
- "Classical: p equals m v. Photon has m equals zero, so p should be zero. Wrong."\\\\
- "Quantum: p equals h over lambda. Inversely proportional to wavelength."\\\\
[P1] "Shorter wavelength means higher momentum"\\\\
- "UV photons push harder than infrared photons"\\\\
[P2] "Real observable effects in everyday cosmos"\\\\
[THE CONNECTION - Kinetic Archetype] "Like wind in a sail, but the wind is light"\\\\
[THE WONDER] Future spacecraft may sail on sunlight across the solar system}
\end{frame}

\begin{frame}
\frametitle{21.3 Photons as Billiard Balls}
\textbf{Compton Effect (1923):} X-ray photons collide with electrons.

\pause
\vspace{0.3cm}

\textbf{Observations:}
\begin{itemize}
\item Photon scatters at angle (like billiard ball)
\item Electron recoils with kinetic energy
\item Scattered photon has \textit{longer wavelength} (lost energy to electron)
\item Conservation of momentum and energy apply perfectly
\end{itemize}

\pause
\vspace{0.3cm}

\begin{block}{The Significance}
Photons behave like classical particles in collisions.\\
Final proof of photon particle nature.
\end{block}
\note{[P0] "Arthur Compton, 1923 - fired X-rays at electrons"\\\\
[P1] "Photon and electron collide like pool balls"\\\\
- "Photon transfers some energy and momentum to electron"\\\\
- "Photon wavelength increases - energy decreased"\\\\
- "Electron flies off with gained energy"\\\\
[P2] [THE REVELATION] "Perfect confirmation: photons have momentum and energy like particles"\\\\
- "Conservation laws work in quantum collisions"\\\\
[THE HUMILITY] "This won Compton the 1927 Nobel Prize"\\\\
[THE WONDER] Same effect used in medical imaging - Compton scattering reveals your bones}
\end{frame}

\begin{frame}
\frametitle{21.3 The Ultimate Twist}
\begin{alertblock}{The Final Paradox}
If light (a wave) can act like a particle...\\
Can matter (a particle) act like a wave?
\end{alertblock}

\pause
\textbf{Answer:} Yes. Louis de Broglie (1924) proposed:
$$\boxed{\wavelen{\lambda} = \frac{\pConst{h}}{\mom{p}}}$$

\pause
\textbf{Consequence:} Electrons, atoms, molecules - \textit{everything} has a wavelength.

\pause
\vspace{0.3cm}

\textbf{Experimental proof:}
\begin{itemize}
\item Electron diffraction creates interference patterns (1927)
\item Electron microscopes use electron waves to image atoms
\item Matter-wave duality is universal
\end{itemize}
\note{[P0] [THE CONFLICT] "De Broglie asked the mirror-image question"\\\\
- "Light has duality. Why not matter?"\\\\
[P1] [THE REVELATION] "He proposed matter waves - earned 1929 Nobel Prize for the idea"\\\\
[P2] "Everything in the universe has a wavelength - even you"\\\\
[P3] "Electron diffraction experiments confirmed it in 1927"\\\\
- "Electrons create interference patterns just like light"\\\\
[THE HUMILITY] "Your wavelength is h over m v - unmeasurably tiny for macroscopic objects"\\\\
[THE WONDER] Electron microscopes exploit matter waves to see atoms - wavelike electrons image particle-like atoms}
\end{frame}

\section{Summary}

\begin{frame}
\frametitle{The Quantum Revolution}
\begin{block}{What You Now Know}
\begin{enumerate}
\item \energy{Energy} is quantized: $\energy{E} = \particles{n}\pConst{h}\freq{f}$ (Planck) \pause
\item Photons are discrete packets: $\energy{E} = \pConst{h}\freq{f}$ (Planck/Einstein) \pause
\item Photoelectric effect proves light is quantized: $\kenergy{KE_e} = \pConst{h}\freq{f} - \energy{BE}$ (Einstein) \pause
\item Photons carry \mom{momentum}: $\mom{p} = \pConst{h}/\wavelen{\lambda}$ (Einstein/Compton) \pause
\item Light exhibits wave-particle duality \pause
\item All matter exhibits wave-particle duality (de Broglie) \pause
\item Quantum mechanics governs the universe at atomic scales
\end{enumerate}
\end{block}
\note{[P0] "Seven revelations today - each one revolutionary"\\\\
[P1] "Planck: Energy is quantized - discrete packets, not continuous"\\\\
[P2] "Photons are real particles with energy h f"\\\\
[P3] "Einstein: Photoelectric effect proves light is quantized - Nobel Prize 1921"\\\\
[P4] "Photons carry momentum despite zero mass"\\\\
[P5] "Light is both wave and particle - depends on experiment"\\\\
[P6] "de Broglie: Matter has wavelength too - universal duality"\\\\
[P7] "Quantum mechanics is the operating system of reality at small scales"\\\\
[THE WONDER] This revolution created transistors, lasers, computers, MRI, solar cells - your entire technological world}
\end{frame}

\begin{frame}
\frametitle{Key Equations: Chapter 21}
\begin{align}
\energy{E} &= \particles{n}\pConst{h}\freq{f} \quad \text{(quantized energy - Planck)} \\
\energy{E} &= \pConst{h}\freq{f} = \frac{\pConst{h}\vel{c}}{\wavelen{\lambda}} \quad \text{(photon energy)} \\
\kenergy{KE_e} &= \pConst{h}\freq{f} - \energy{BE} \quad \text{(photoelectric equation - Einstein)} \\
\mom{p} &= \frac{\pConst{h}}{\wavelen{\lambda}} \quad \text{(photon momentum)}
\end{align}

\vspace{0.3cm}

\textbf{Key constant:}
$$\pConst{h} = 6.626 \times 10^{-34} \text{ J·s (Planck's constant)}$$
\note{- Four fundamental equations of quantum physics\\\\
- Planck's constant h appears in all of them - the quantum of action\\\\
- Know when to use each equation - wavelength given use h c over lambda, frequency given use h f\\\\
- Photoelectric uses energy budget: incoming photon energy minus binding energy equals kinetic energy\\\\
- Questions before we end?}
\end{frame}

\begin{frame}
\frametitle{Homework}
\begin{center}
\Large
Complete the assigned problems\\[0.3cm]
posted on the LMS
\end{center}
\note{[SAY] "Homework posted on LMS - Chapter 21 problems"\\\\
[TIMING] Due date: check LMS\\\\
- "Focus on photon energy calculations, photoelectric effect, and duality"\\\\
[CHECK] Questions before we end?\\\\
[TRANSITION] Next class: Chapter 22 - we'll explore atomic structure and how quantum mechanics explains the periodic table\\\\
[THE WONDER] You've just learned the physics that won FIVE Nobel Prizes - Planck, Einstein, Compton, de Broglie, Millikan}
\end{frame}

\end{document}
