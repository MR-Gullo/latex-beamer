\documentclass{beamer}
\usepackage{../../../shared/templates/ds9_theme}
\usepackage[overridenote]{pdfpc}
\graphicspath{{../images/}{../../shared/images/}}

\title[Simple Machines]{PHYS12 CH:9.3 How Ancient Humans Moved Mountains}
\subtitle{The Six Tools That Built Civilization}
\author[Mr. Gullo]{Mr. Gullo}
\date[December 2025]{December 2025}

\begin{document}

\frame{\titlepage
\note{[THE HOOK] How did humans build the pyramids without engines?\\\\
- Six simple machines changed everything\\\\
- Same principles used today in construction, engineering, medicine\\\\
[THE WONDER] You'll learn how to multiply force using only geometry\\\\
- This is ancient wisdom that still powers modern world}
}

\begin{frame}
\frametitle{Outline}
\tableofcontents
\end{frame}

\section{Introduction}

\begin{frame}
\frametitle{The Mystery}
\begin{center}
\Large How did ancient humans move\\
\textit{20-ton stone blocks}\\
without engines or electricity?
\end{center}

\pause
\vspace{0.5cm}
They discovered six simple machines that multiply force.

\pause
\vspace{0.3cm}
\alert{Same tools built the pyramids and launch rockets today.}
\note{[P0] "How did ancient humans move 20-ton stone blocks without engines?"\\\\
[P1] "They discovered six simple machines that multiply force"\\\\
[P2] [THE WONDER] "Same tools built the pyramids and launch rockets today"\\\\
[THE CONNECTION - Kinetic Archetype] Athletes use levers every time they swing a bat\\\\
[THE REVELATION] Physics hasn't changed in 5,000 years}
\end{frame}

\section{Simple Machines and Mechanical Advantage}

\begin{frame}
\frametitle{Learning Objectives}
\begin{block}{By the end of this lesson, you will be able to:}
\begin{itemize}
\item \textbf{9.3:} Describe simple and complex machines \pause
\item \textbf{9.3:} Calculate mechanical advantage of simple machines \pause
\item \textbf{9.3:} Calculate efficiency of simple and complex machines
\end{itemize}
\end{block}
\note{[P0] "Three objectives for simple machines"\\\\
[P1] "First: understand what simple machines are and how they work"\\\\
[P2] "Second: calculate mechanical advantage - how much they multiply force"\\\\
[P3] "Third: calculate efficiency - accounting for friction"\\\\
- Assessment: problem set due next week}
\end{frame}

\begin{frame}
\frametitle{9.3 The Conservation Trade-Off}
\begin{block}{Nature's Accounting System}
In closed systems, total energy is conserved.\\
Machines cannot create energy.
\end{block}

\pause
\vspace{0.3cm}

\textbf{What simple machines DO:}
\begin{itemize}
\item Reduce force required \pause
\item Increase distance over which force acts \pause
\item Product $f \times d$ stays constant
\end{itemize}

\pause
\begin{alertblock}{The Illusion}
Machines make work "easier" but don't reduce total work.
\end{alertblock}
\note{[P0] [THE REVELATION] "In closed systems, total energy is conserved. Machines cannot create energy"\\\\
[P1] "What simple machines DO: reduce force required"\\\\
[P2] "Increase distance over which force acts"\\\\
[P3] "Product f times d stays constant"\\\\
[P4] [THE CONFLICT] "Machines make work feel easier but don't reduce total work"\\\\
[THE HUMILITY] This confused people for centuries until conservation of energy was discovered\\\\
[THE WONDER] W equals f d - simple equation, profound consequence}
\end{frame}

\begin{frame}
\frametitle{9.3 The Trade-Off Equation}
\begin{block}{Universal Law of Simple Machines}
\begin{center}
\Large $\boxed{W_i = W_o}$
\end{center}
Work input equals work output (in ideal case)
\end{block}

\pause
\vspace{0.3cm}

Expanding this:
$$F_e \times d_e = F_r \times d_r$$

\pause
\vspace{0.2cm}

\begin{exampleblock}{The Mental Model}
If effort force $F_e$ is less than resistance force $F_r$,\\
then effort distance $d_e$ must be greater than resistance distance $d_r$.
\end{exampleblock}
\note{[P0] [THE REVELATION] "Work input equals work output in ideal case"\\\\
[P1] "Expanding this: F-e times d-e equals F-r times d-r"\\\\
[P2] "If effort force is less than resistance force, effort distance must be greater"\\\\
[THE CONNECTION - Digital Archetype] "Like lossless compression: rearrange but don't lose information"\\\\
[THE WONDER] Same principle applies to all six simple machines}
\end{frame}

\section{The Six Simple Machines}

\begin{frame}
\frametitle{9.3 The Lever: Humanity's First Force Multiplier}
\begin{figure}
\centering
\includegraphics[width=0.7\textwidth,height=0.5\textheight,keepaspectratio]{phys12-simple-machines-fig01.jpg}
\end{figure}

\pause
\textbf{Components:}
\begin{itemize}
\item Fulcrum: the pivot point
\item Effort arm $L_e$: distance from fulcrum to effort force
\item Resistance arm $L_r$: distance from fulcrum to load
\end{itemize}
\note{[P0] "The lever: a pry bar removing a nail"\\\\
[P1] "Three components: fulcrum, effort arm, resistance arm"\\\\
[THE CONNECTION - Kinetic Archetype] "Every bat, golf club, hammer is a lever"\\\\
[THE REVELATION] Archimedes said 'Give me a place to stand and I will move the Earth'\\\\
[THE WONDER] 2,000-year-old quote about leverage still true today}
\end{frame}

\begin{frame}
\frametitle{9.3 Mechanical Advantage}
\begin{block}{Definition: Ideal Mechanical Advantage (IMA)}
The number of times a machine multiplies the effort force.
\end{block}

\pause
\vspace{0.3cm}

\textbf{For a lever:}
$$IMA = \frac{L_e}{L_r}$$

\pause
\vspace{0.2cm}

\textbf{General formula:}
$$IMA = \frac{F_r}{F_e} = \frac{d_e}{d_r}$$

\pause
\begin{exampleblock}{The Mental Model}
IMA of 4 means you lift 400 N with only 100 N effort.\\
The catch: you pull 4 times the distance.
\end{exampleblock}
\note{[P0] "IMA: the number of times a machine multiplies effort force"\\\\
[P1] "For a lever: IMA equals effort arm over resistance arm"\\\\
[P2] "General formula: resistance force over effort force"\\\\
[P3] "IMA of 4 means lift 400 N with 100 N effort, but pull 4 times the distance"\\\\
[THE HUMILITY] This seems like cheating but conservation of energy says otherwise\\\\
[THE WONDER] You can multiply force but never multiply work}
\end{frame}

\begin{frame}
\frametitle{9.3 Lever Classes}
\textbf{Three classes based on fulcrum position:}

\pause
\begin{itemize}
\item \textbf{Class 1:} Fulcrum between effort and load\\
Examples: seesaw, pry bar, scissors \pause
\item \textbf{Class 2:} Load between fulcrum and effort\\
Examples: wheelbarrow, bottle opener, nutcracker \pause
\item \textbf{Class 3:} Effort between fulcrum and load\\
Examples: baseball bat, hammer, golf club
\end{itemize}

\pause
\vspace{0.3cm}
\begin{alertblock}{The Paradox}
Class 3 levers have IMA less than 1. They reduce force but increase speed!
\end{alertblock}
\note{[P0] "Three classes based on fulcrum position"\\\\
[P1] "Class 1: fulcrum between effort and load - seesaw, pry bar"\\\\
[P2] "Class 2: load between fulcrum and effort - wheelbarrow"\\\\
[P3] "Class 3: effort between fulcrum and load - baseball bat, hammer"\\\\
[P4] [THE CONFLICT] "Class 3 has IMA less than 1 - reduces force but increases speed"\\\\
[THE CONNECTION - Kinetic Archetype] Your forearm is a Class 3 lever when you lift something\\\\
[THE WONDER] Nature chose speed over force for human body design}
\end{frame}

\begin{frame}
\frametitle{9.3 Wheel and Axle}
\begin{figure}
\centering
\includegraphics[width=0.6\textwidth,height=0.45\textheight,keepaspectratio]{phys12-simple-machines-fig02.jpg}
\end{figure}

\pause
\textbf{IMA formula:}
$$IMA = \frac{r_{wheel}}{r_{axle}}$$

\pause
\begin{exampleblock}{Real-World Examples}
Steering wheel, door knob, windlass, screwdriver handle
\end{exampleblock}
\note{[P0] "Wheel and axle: really a rotating lever"\\\\
[P1] "IMA equals radius of wheel over radius of axle"\\\\
[P2] "Examples: steering wheel, door knob, windlass"\\\\
[THE CONNECTION] Why were old car steering wheels so big? No power steering\\\\
[THE REVELATION] Larger wheel radius means greater mechanical advantage\\\\
[THE WONDER] Same physics in medieval wells and modern surgery tools}
\end{frame}

\begin{frame}
\frametitle{9.3 Inclined Plane}
\begin{figure}
\centering
\includegraphics[width=0.7\textwidth,height=0.5\textheight,keepaspectratio]{phys12-simple-machines-fig03.jpg}
\end{figure}

\pause
\textbf{IMA formula:}
$$IMA = \frac{L}{h}$$

where $L$ is length of ramp and $h$ is height
\note{[P0] "Inclined plane: a ramp"\\\\
[P1] "IMA equals length over height"\\\\
[THE CONNECTION - Kinetic Archetype] "Wheelchair ramps, loading docks, mountain roads"\\\\
[THE REVELATION] Ancient Egyptians used ramps to build pyramids\\\\
[THE WONDER] Longer ramp means less force but more distance to push}
\end{frame}

\begin{frame}
\frametitle{9.3 The Wedge}
\begin{figure}
\centering
\includegraphics[width=0.6\textwidth,height=0.4\textheight,keepaspectratio]{phys12-simple-machines-fig04.jpg}
\end{figure}

\pause
\textbf{What it is:}
Two inclined planes back to back

\pause
\vspace{0.3cm}

\textbf{IMA formula:}
$$IMA = \frac{L}{h}$$

\pause
\begin{exampleblock}{Real-World Examples}
Knife, axe, chisel, nail, zipper teeth
\end{exampleblock}
\note{[P0] "Wedge: two inclined planes back to back"\\\\
[P1] "What it is: two inclined planes joined"\\\\
[P2] "IMA equals length over height"\\\\
[P3] "Examples: knife, axe, chisel, zipper teeth"\\\\
[THE REVELATION] Difference from incline: load stays still, machine moves\\\\
[THE WONDER] A sharp knife is just a very thin wedge}
\end{frame}

\begin{frame}
\frametitle{9.3 The Screw}
\begin{figure}
\centering
\includegraphics[width=0.5\textwidth,height=0.4\textheight,keepaspectratio]{phys12-simple-machines-fig05.jpg}
\end{figure}

\pause
\textbf{What it is:}
An inclined plane wrapped around a cylinder

\pause
\vspace{0.3cm}

\textbf{IMA formula:}
$$IMA = \frac{2\pi L}{P}$$

where $L$ is handle length and $P$ is pitch (distance between threads)
\note{[P0] "Screw: inclined plane wrapped around cylinder"\\\\
[P1] "What it is: inclined plane wrapped in spiral"\\\\
[P2] "IMA equals 2 pi L over P - handle length over thread spacing"\\\\
[THE REVELATION] Screw can have enormous mechanical advantage\\\\
[THE CONNECTION] Car jacks, wood screws, bottle caps\\\\
[THE WONDER] Devices like this can lift entire houses}
\end{frame}

\begin{frame}
\frametitle{9.3 The Pulley}
\begin{figure}
\centering
\includegraphics[width=0.8\textwidth,height=0.5\textheight,keepaspectratio]{phys12-simple-machines-fig06.jpg}
\end{figure}

\pause
\textbf{IMA formula (easiest!):}
$$IMA = N$$

where $N$ is number of ropes supporting the load
\note{[P0] "Pulley systems: three examples shown"\\\\
[P1] "IMA formula is easiest: just count ropes supporting load"\\\\
[THE CONNECTION] "Flagpoles, window blinds, construction cranes"\\\\
[THE REVELATION] Pulleys were essential on sailing ships\\\\
[THE WONDER] To raise load 1 m with 4 ropes, you pull 4 m of rope}
\end{frame}

\begin{frame}
\frametitle{9.3 Complex Machines}
\begin{figure}
\centering
\includegraphics[width=0.6\textwidth,height=0.4\textheight,keepaspectratio]{phys12-simple-machines-fig07.jpg}
\end{figure}

\pause
\begin{block}{Definition: Complex Machine}
A combination of two or more simple machines
\end{block}

\pause
\vspace{0.3cm}

\textbf{Wire cutters combine:}
\begin{itemize}
\item Two levers
\item Two wedges
\end{itemize}

\pause
\begin{exampleblock}{More Examples}
Bicycle: wheel and axle, levers, screws, pulleys\\
Car: hundreds of simple machines combined
\end{exampleblock}
\note{[P0] "Wire cutters: complex machine"\\\\
[P1] "Complex machine: combination of two or more simple machines"\\\\
[P2] "Wire cutters: two levers and two wedges"\\\\
[P3] "Bicycle has wheel and axle, levers, screws, pulleys"\\\\
[THE REVELATION] IMAs of component machines usually multiply\\\\
[THE WONDER] Output force of one becomes input force of next}
\end{frame}

\section{Efficiency and Real Machines}

\begin{frame}
\frametitle{9.3 The Reality of Friction}
\begin{alertblock}{Civilian View vs. Reality}
\textbf{Ideal machine:} $W_i = W_o$ (100\% efficiency)\\
\textbf{Real machine:} $W_o < W_i$ (always less than 100\%)
\end{alertblock}

\pause
\vspace{0.3cm}

\textbf{Why the difference?}
\begin{itemize}
\item Friction between moving parts \pause
\item Air resistance \pause
\item Heat generation \pause
\item Deformation of materials
\end{itemize}

\pause
\vspace{0.3cm}
Energy is conserved, but some becomes unavailable heat.
\note{[P0] [THE CONFLICT] "Ideal machine has 100 percent efficiency, real machines always less"\\\\
[P1] "Why? Friction between moving parts"\\\\
[P2] "Air resistance"\\\\
[P3] "Heat generation"\\\\
[P4] "Deformation of materials"\\\\
[P5] "Energy is conserved but some becomes unavailable heat"\\\\
[THE HUMILITY] Engineers constantly fight friction\\\\
[THE WONDER] Lubrication can improve efficiency but never reach 100 percent}
\end{frame}

\begin{frame}
\frametitle{9.3 Calculating Efficiency}
\begin{block}{Definition: Efficiency}
The ratio of output work to input work, expressed as percentage
\end{block}

\pause
\vspace{0.3cm}

\textbf{Efficiency formula:}
$$\text{Efficiency} = \frac{W_o}{W_i} \times 100\%$$

\pause
\vspace{0.3cm}

Since $W = F \times d$:
$$\text{Efficiency} = \frac{F_o \times d_o}{F_i \times d_i} \times 100\%$$
\note{[P0] "Efficiency: ratio of output work to input work"\\\\
[P1] "Formula: W-o over W-i times 100 percent"\\\\
[P2] "Since W equals F times d, can expand to forces and distances"\\\\
[THE REVELATION] Efficiency tells you how much work is lost to friction\\\\
[THE CONNECTION] Higher efficiency means less wasted energy\\\\
[THE WONDER] Even 91 percent efficiency means 9 percent lost to heat}
\end{frame}

\begin{frame}
\frametitle{9.3 Which Machines Are Most Efficient?}
\textbf{Think about friction:}

\pause
\begin{itemize}
\item Depends on smoothness of surfaces \pause
\item Depends on area of surfaces in contact \pause
\item Depends on types of materials
\end{itemize}

\pause
\vspace{0.3cm}

\begin{exampleblock}{Ranking by Efficiency (typical)}
\begin{enumerate}
\item Pulleys (95-98\%)
\item Wheel and axle (90-95\%)
\item Levers (90-95\%)
\item Inclined plane (60-90\%)
\item Wedge (50-80\%)
\item Screw (30-70\%)
\end{enumerate}
\end{exampleblock}
\note{[P0] "Think about friction - what affects it?"\\\\
[P1] "Smoothness of surfaces"\\\\
[P2] "Area of surfaces in contact"\\\\
[P3] "Types of materials"\\\\
[P4] "Ranking: pulleys most efficient, screws least efficient"\\\\
[THE REVELATION] More surface contact means more friction\\\\
[THE CONNECTION] Lubrication can boost efficiency by 10-20 percent}
\end{frame}

\section{Problem Solving}

\begin{frame}
\frametitle{Attempt: Lever Efficiency}
\begin{exampleblock}{The Challenge (3 min, silent)}
An input force of 11 N acting on the effort arm of a lever moves 0.4 m.\\
This lifts a 40 N weight on the resistance arm a distance of 0.1 m.

\vspace{0.3cm}

\textbf{Given:}
\begin{itemize}
\item $F_i = 11$ N, $d_i = 0.4$ m
\item $F_o = 40$ N, $d_o = 0.1$ m
\end{itemize}

\textbf{Find:} Efficiency of the lever

\vspace{0.3cm}

\textit{Can you calculate how much work is lost to friction?}
\end{exampleblock}
\note{[THE CHALLENGE] Can they calculate efficiency from forces and distances?\\\\
[SAY] "Try this on your own. It's okay to get stuck."\\\\
[TIMING] 3-4 min SILENT individual work\\\\
[CIRCULATE] Note who calculates W-i and W-o first, who goes straight to formula\\\\
[WATCH FOR] Students forgetting to multiply by 100\\\\
[DON'T HELP] Let them struggle - learning happens in Compare}
\end{frame}

\begin{frame}
\frametitle{Compare: Efficiency Strategy}
\textbf{Turn and talk (2 min):}

\vspace{0.3cm}

\begin{enumerate}
\item What equation did you use for efficiency?
\item Did you calculate $W_i$ and $W_o$ separately?
\item What units should efficiency have?
\end{enumerate}

\vspace{0.5cm}

\pause
\alert{Name wheel:} One pair share your approach (not your answer).
\note{[TIMING] 2-3 min pair discussion\\\\
[CIRCULATE] Listen for common approaches\\\\
[CHECK] Name wheel: call a pair to share approach\\\\
[EXPECTED APPROACH] Calculate W-i equals F-i times d-i, W-o equals F-o times d-o, then divide and multiply by 100\\\\
[COMMON ERROR] Forgetting to multiply by 100 for percentage}
\end{frame}

\begin{frame}
\frametitle{Reveal: Lever Efficiency Solution}
\textbf{Self-correct in a different color:}

\vspace{0.3cm}

\textbf{Step 1:} Calculate input work
$$W_i = F_i \times d_i = 11 \times 0.4 = 4.4 \text{ J}$$

\pause
\textbf{Step 2:} Calculate output work
$$W_o = F_o \times d_o = 40 \times 0.1 = 4.0 \text{ J}$$

\pause
\textbf{Step 3:} Calculate efficiency
$$\text{Efficiency} = \frac{W_o}{W_i} \times 100\% = \frac{4.0}{4.4} \times 100\% = \boxed{91\%}$$

\pause
\textbf{Check:} 0.4 J lost to friction out of 4.4 J input - reasonable!
\note{[P0] "Self-correct in a different color"\\\\
[P1] "Step 1: W-i equals 11 times 0.4 equals 4.4 joules"\\\\
[P2] "Step 2: W-o equals 40 times 0.1 equals 4.0 joules"\\\\
[P3] "Step 3: Efficiency equals 4.0 over 4.4 times 100 equals 91 percent"\\\\
[P4] [ANSWER] "91 percent efficiency - 0.4 J lost to friction"\\\\
[THE WONDER] Even a simple lever loses 9 percent to friction}
\end{frame}

\begin{frame}
\frametitle{Attempt: Inclined Plane IMA}
\begin{exampleblock}{The Challenge (3 min, silent)}
An inclined plane is 5.0 m long and 2.0 m high.

\vspace{0.3cm}

\textbf{Given:}
\begin{itemize}
\item Length $L = 5.0$ m
\item Height $h = 2.0$ m
\end{itemize}

\textbf{Find:} Ideal mechanical advantage (IMA)

\vspace{0.3cm}

\textit{How much does this ramp multiply your force?}
\end{exampleblock}
\note{[THE CHALLENGE] Can they identify correct formula for inclined plane?\\\\
[SAY] "Try this on your own. It's okay to get stuck."\\\\
[TIMING] 3 min SILENT individual work\\\\
[CIRCULATE] Note who uses correct formula, who inverts the fraction\\\\
[WATCH FOR] Students putting height over length instead of length over height\\\\
[DON'T HELP] Let them discover the right formula}
\end{frame}

\begin{frame}
\frametitle{Compare: IMA Formula}
\textbf{Turn and talk (2 min):}

\vspace{0.3cm}

\begin{enumerate}
\item What is the formula for IMA of an inclined plane?
\item Which value goes in numerator and which in denominator?
\item Should IMA have units?
\end{enumerate}

\vspace{0.5cm}

\pause
\alert{Name wheel:} One pair share your reasoning.
\note{[TIMING] 2 min pair discussion\\\\
[CIRCULATE] Listen for correct formula\\\\
[CHECK] Name wheel: call a pair to share\\\\
[EXPECTED APPROACH] IMA equals L over h\\\\
[COMMON ERROR] Inverting to h over L\\\\
[KEY INSIGHT] IMA is dimensionless - units cancel}
\end{frame}

\begin{frame}
\frametitle{Reveal: Inclined Plane IMA Solution}
\textbf{Self-correct in a different color:}

\vspace{0.3cm}

\textbf{Formula for inclined plane:}
$$IMA = \frac{L}{h}$$

\pause
\textbf{Substitute values:}
$$IMA = \frac{5.0 \text{ m}}{2.0 \text{ m}} = \boxed{2.5}$$

\pause
\vspace{0.3cm}

\textbf{Interpretation:} This ramp multiplies your force by 2.5.\\
You can push a 250 N load with only 100 N effort.\\
The trade-off: you push 2.5 times the distance.

\note{[P0] "Self-correct in a different color"\\\\
[P1] "IMA equals L over h for inclined plane"\\\\
[P2] "5.0 over 2.0 equals 2.5"\\\\
[P3] "Interpretation: ramp multiplies force by 2.5"\\\\
[ANSWER] IMA equals 2.5 - dimensionless\\\\
[THE WONDER] Push 250 N load with 100 N, but over 2.5 times the distance}
\end{frame}

\begin{frame}
\frametitle{Attempt: Pulley System}
\begin{exampleblock}{The Challenge (3 min, silent)}
A pulley system lifts a 200 N load with an effort force of 52 N.\\
The system has an efficiency of almost 100\%.

\vspace{0.3cm}

\textbf{Given:}
\begin{itemize}
\item $F_r = 200$ N (load)
\item $F_e = 52$ N (effort)
\item Efficiency $\approx 100\%$
\end{itemize}

\textbf{Find:} Number of ropes supporting the load

\vspace{0.3cm}

\textit{How many ropes does this system need?}
\end{exampleblock}
\note{[THE CHALLENGE] Can they connect IMA to number of ropes?\\\\
[SAY] "Try this on your own. It's okay to get stuck."\\\\
[TIMING] 3 min SILENT individual work\\\\
[CIRCULATE] Note who calculates actual MA first, who knows pulley rule\\\\
[WATCH FOR] Students calculating MA then knowing N equals MA\\\\
[DON'T HELP] Let them connect force ratio to rope count}
\end{frame}

\begin{frame}
\frametitle{Compare: Pulley Logic}
\textbf{Turn and talk (2 min):}

\vspace{0.3cm}

\begin{enumerate}
\item What is mechanical advantage for pulleys?
\item How do you find actual MA from given forces?
\item For pulleys, what does IMA equal?
\end{enumerate}

\vspace{0.5cm}

\pause
\alert{Name wheel:} One pair share your reasoning.
\note{[TIMING] 2 min pair discussion\\\\
[CIRCULATE] Listen for connection between MA and rope count\\\\
[CHECK] Name wheel: call a pair to share\\\\
[EXPECTED APPROACH] MA equals F-r over F-e, and for pulleys IMA equals N\\\\
[KEY INSIGHT] Since efficiency is 100 percent, actual MA equals IMA equals N}
\end{frame}

\begin{frame}
\frametitle{Reveal: Pulley System Solution}
\textbf{Self-correct in a different color:}

\vspace{0.3cm}

\textbf{Step 1:} Calculate actual mechanical advantage
$$MA = \frac{F_r}{F_e} = \frac{200 \text{ N}}{52 \text{ N}} = 3.85$$

\pause
\textbf{Step 2:} For pulleys, $IMA = N$ (number of ropes)

\pause
\textbf{Step 3:} Since efficiency $\approx 100\%$, actual MA $\approx$ IMA
$$N \approx 3.85 \approx \boxed{4 \text{ ropes}}$$

\pause
\textbf{Check:} 4 ropes supporting load makes sense.\\
Each rope carries roughly 200/4 = 50 N.
\note{[P0] "Self-correct in a different color"\\\\
[P1] "MA equals 200 over 52 equals 3.85"\\\\
[P2] "For pulleys, IMA equals N, number of ropes"\\\\
[P3] "Since efficiency near 100 percent, N equals 4 ropes"\\\\
[P4] [ANSWER] "4 ropes - each carries about 50 N"\\\\
[THE WONDER] Count ropes in real cranes - that's the mechanical advantage}
\end{frame}

\section{Summary}

\begin{frame}
\frametitle{The Six Force Multipliers}
\begin{block}{Simple Machines Summary}
\begin{enumerate}
\item \textbf{Lever:} $IMA = \frac{L_e}{L_r}$ \pause
\item \textbf{Wheel and Axle:} $IMA = \frac{r_{wheel}}{r_{axle}}$ \pause
\item \textbf{Inclined Plane:} $IMA = \frac{L}{h}$ \pause
\item \textbf{Wedge:} $IMA = \frac{L}{h}$ \pause
\item \textbf{Screw:} $IMA = \frac{2\pi L}{P}$ \pause
\item \textbf{Pulley:} $IMA = N$ (number of ropes)
\end{enumerate}
\end{block}
\note{[P0] "Six simple machines - six force multipliers"\\\\
[P1] "Lever: effort arm over resistance arm"\\\\
[P2] "Wheel and axle: wheel radius over axle radius"\\\\
[P3] "Inclined plane: length over height"\\\\
[P4] "Wedge: same as incline"\\\\
[P5] "Screw: 2 pi L over pitch"\\\\
[P6] "Pulley: just count the ropes"\\\\
[THE WONDER] Six tools, thousands of years, still unchanged}
\end{frame}

\begin{frame}
\frametitle{Key Equations}
\begin{align}
IMA &= \frac{F_r}{F_e} = \frac{d_e}{d_r} \\
W_i &= W_o \quad \text{(ideal machine)} \\
F_e \times d_e &= F_r \times d_r \\
\text{Efficiency} &= \frac{W_o}{W_i} \times 100\%
\end{align}
\note{- Four fundamental equations for simple machines\\\\
- IMA: resistance over effort or distances inverted\\\\
- Conservation: work in equals work out\\\\
- Force-distance product stays constant\\\\
- Efficiency accounts for friction\\\\
- Questions before we end?}
\end{frame}

\begin{frame}
\frametitle{What You Now Know}
\begin{block}{The Revelations}
\begin{enumerate}
\item Simple machines trade force for distance \pause
\item Machines cannot create energy - conservation rules \pause
\item Six types: lever, wheel-axle, incline, wedge, screw, pulley \pause
\item IMA tells you how much force is multiplied \pause
\item Efficiency measures work lost to friction \pause
\item Complex machines combine simple machines
\end{enumerate}
\end{block}
\note{[P0] "Six revelations about simple machines"\\\\
[P1] "Simple machines trade force for distance"\\\\
[P2] "Cannot create energy - conservation rules"\\\\
[P3] "Six types that built civilization"\\\\
[P4] "IMA tells force multiplication"\\\\
[P5] "Efficiency measures friction loss"\\\\
[P6] "Complex machines are just combinations"\\\\
[THE WONDER] Ancient wisdom, modern applications, same physics\\\\
- Name wheel: which machine surprised you most?}
\end{frame}

\begin{frame}
\frametitle{Homework}
\begin{center}
\Large
Complete the assigned problems\\[0.3cm]
posted on the LMS
\end{center}
\note{[SAY] "Homework is posted on the LMS"\\\\
[TIMING] Due date: check LMS\\\\
[CHECK] Questions before we end?\\\\
[TRANSITION] Next class: we'll apply these concepts to real engineering problems}
\end{frame}

\end{document}
