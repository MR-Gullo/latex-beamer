\documentclass{beamer}
\usepackage{../../../shared/templates/ds9_theme}
\usepackage[overridenote]{pdfpc}
\graphicspath{{../images/}{../../shared/images/}}

\title[Forces That Shape Technology]{PHYS12 CH:20 The Invisible Force That Powers Your Life}
\subtitle{From Aurora to Electric Motors}
\author[Mr. Gullo]{Mr. Gullo}
\date[December 2025]{December 2025}

\begin{document}

% Title slide
\begin{frame}
    \titlepage
\end{frame}

% Outline slide
\begin{frame}
    \frametitle{Outline}
    \tableofcontents
\end{frame}

\section{Introduction}

\begin{frame}
    \frametitle{The Aurora's Secret}

    \begin{figure}
        \centering
        \includegraphics[width=0.8\textwidth,height=0.6\textheight,keepaspectratio]{phys12-magnetism-fig01-1.jpg}
    \end{figure}

    \pause

    \begin{exampleblock}{The Connection}
    \textit{Same physics that creates northern lights powers your phone, lights your home, and makes your car move.}
    \end{exampleblock}

    \note{[P0] [THE HOOK] "Aurora Borealis - one of nature's most spectacular shows"\\\\
    - "Charged particles from Sun spiral along Earth's magnetic field"\\\\
    - "Light show hundreds of kilometers high"\\\\
    [P1] [THE CONNECTION - Digital Archetype]\\\\
    - "Every device you own uses electromagnetic induction"\\\\
    - "Your laptop charger is a transformer"\\\\
    - "Electric motors in hard drives, fans, vibration motors"\\\\
    [THE WONDER] "We're about to learn how to transform energy with invisible fields"}
\end{frame}

\begin{frame}
    \frametitle{Learning Objectives}

    \begin{block}{By the end of this lesson, you will be able to:}
        \begin{itemize}
            \item \textbf{20.1:} Calculate magnetic force on moving charges and current-carrying wires \pause
            \item \textbf{20.2:} Explain how electric motors, generators, and transformers work \pause
            \item \textbf{20.2:} Describe how commercial electric power is produced and transmitted \pause
            \item \textbf{20.3:} Explain how changing magnetic fields produce current \pause
            \item \textbf{20.3:} Calculate induced electromotive force and current
        \end{itemize}
    \end{block}

    \note{[P0] "Five big ideas today"\\\\
    [P1] "First - magnetic force on moving charges"\\\\
    [P2] "Second - motors turn electrical energy into motion"\\\\
    [P3] "Third - power grid delivers energy to millions"\\\\
    [P4] "Fourth - Faraday's discovery that changed everything"\\\\
    [P5] "Fifth - actual calculations engineers use daily"\\\\
    [THE CONNECTION - Kinetic] "Athletes: Your treadmill uses a motor. Gamers: Your console cooling fan too."}
\end{frame}

\section{20.1 Magnetic Fields and Force}

\begin{frame}
    \frametitle{20.1 Magnetic Poles}

    \textbf{Every magnet has two poles:}
    \begin{itemize}
        \item North pole (points toward geographic North) \pause
        \item South pole (points toward geographic South)
    \end{itemize}

    \pause
    \vspace{0.3cm}

    \begin{alertblock}{The Paradox}
        \textbf{Civilian:} "Opposites attract, same poles repel."\\
        \textbf{Physicist:} "True, but you can NEVER isolate a single pole."
    \end{alertblock}

    \note{[P0] "Every magnet has two poles"\\\\
    [P1] "North and south - named for how they orient with Earth"\\\\
    [P2] [THE CONFLICT] "Unlike electric charge, you cannot isolate a magnetic pole"\\\\
    - "Cut a magnet in half, you get two smaller magnets with two poles each"\\\\
    [THE HUMILITY] "Scientists searched for magnetic monopoles for centuries - never found one"\\\\
    [THE WONDER] "Magnetism is fundamentally different from electricity"}
\end{frame}

\begin{frame}
    \frametitle{20.1 The Universal Law: Magnetic Force}

    \begin{block}{Nature's Rule for Moving Charges}
        \begin{center}
            \Large $\boxed{F = qvB \sin \theta}$
        \end{center}
        Force on a charge moving through a magnetic field.
    \end{block}

    \pause
    \vspace{0.3cm}

    \textbf{Variables:}
    \begin{itemize}
        \item $F$ = force (N)
        \item $q$ = charge (C)
        \item $v$ = velocity (m/s)
        \item $B$ = magnetic field strength (T, tesla)
        \item $\theta$ = angle between $\vec{v}$ and $\vec{B}$
    \end{itemize}

    \note{[P0] [THE REVELATION] "F equals q v B sine theta - force on moving charge"\\\\
    - "Four factors: charge, velocity, field strength, angle"\\\\
    [P1] "Tesla is the unit of magnetic field - named after Nikola Tesla"\\\\
    - "Earth's field is about 50 microtesla"\\\\
    - "MRI machines use 1 to 3 tesla"\\\\
    [THE CONNECTION - Kinetic Archetype] "Athletes: think of curve balls - spin creates force perpendicular to motion"\\\\
    [THE WONDER] "This force is ALWAYS perpendicular to velocity - charges spiral in magnetic fields"}
\end{frame}

\begin{frame}
    \frametitle{20.1 Right-Hand Rule for Force}

    \textbf{To find force direction on positive charge:}
    \begin{enumerate}
        \item Point fingers in direction of velocity $\vec{v}$ \pause
        \item Curl fingers toward magnetic field $\vec{B}$ \pause
        \item Thumb points in direction of force $\vec{F}$
    \end{enumerate}

    \pause
    \vspace{0.3cm}

    \begin{exampleblock}{Key Insight}
        For negative charges, force is OPPOSITE the thumb direction.
    \end{exampleblock}

    \note{[P0] "Right-hand rule determines force direction"\\\\
    [P1] "Point fingers along velocity"\\\\
    [P2] "Curl toward magnetic field"\\\\
    [P3] "Thumb shows force for positive charge"\\\\
    - "Negative charge: force is opposite"\\\\
    - "This is a tool you'll use constantly"\\\\
    [THE CONNECTION - Kinetic] "Like steering - direction depends on how you turn the wheel"}
\end{frame}

\begin{frame}
    \frametitle{20.1 Maximum Force Condition}

    \begin{alertblock}{When is Force Maximum?}
        $\sin \theta$ is maximum when $\theta = 90°$
    \end{alertblock}

    \pause
    \vspace{0.3cm}

    \textbf{Maximum force:} $F_{\text{max}} = qvB$ (when $v \perp B$)

    \pause
    \vspace{0.3cm}

    \textbf{Zero force:} $F = 0$ (when $v \parallel B$, $\theta = 0°$ or $180°$)

    \pause
    \vspace{0.3cm}

    \begin{exampleblock}{The Mental Model}
        Charge moving parallel to field lines feels no force. Moving perpendicular to field lines feels maximum force.
    \end{exampleblock}

    \note{[P0] "When is force maximum?"\\\\
    [P1] "Maximum when velocity perpendicular to field - sine 90 equals 1"\\\\
    [P2] "Zero when velocity parallel to field - sine 0 equals 0"\\\\
    [P3] [THE CONNECTION - Digital Archetype] "Like game physics: collision force depends on angle of impact"\\\\
    [THE WONDER] "Particles spiral along field lines because parallel motion feels no force"}
\end{frame}

\section{20.2 Motors, Generators, and Transformers}

\begin{frame}
    \frametitle{20.2 The Force on Current in Magnetic Field}

    \begin{block}{Universal Law: Magnetic Force on Wire}
        \begin{center}
            \Large $\boxed{F = I \ell B \sin \theta}$
        \end{center}
        \vspace{0.2cm}
        Current-carrying wire in magnetic field experiences force.
    \end{block}

    \pause

    \begin{exampleblock}{The Mental Model}
        Think of electricity as water flowing through a hose. Put that hose in a magnetic field, and the field \textit{pushes} the hose sideways.
    \end{exampleblock}

    \note{[P0] "Foundation for everything today"\\\\
    [ALGEBRA] "F equals I times ell times B times sine theta"\\\\
    - "Current I in amps, length ell in meters, field B in tesla"\\\\
    - "Theta is angle between wire and field"\\\\
    [P1] [THE CONNECTION - Harmonic]\\\\
    - "Musicians: Speaker cone moves because current in coil feels magnetic force"\\\\
    - "Same physics makes your headphones work"\\\\
    [THE WONDER] "Every electric motor on Earth exploits this force"}
\end{frame}

\begin{frame}
    \frametitle{20.2 Electric Motor: Converting Energy}

    \begin{figure}
        \centering
        \includegraphics[width=0.7\textwidth,height=0.5\textheight,keepaspectratio]{phys12-magnetism-fig03-1.jpg}
    \end{figure}

    \pause

    \begin{alertblock}{The Paradox}
        \textbf{Civilian:} "Motors are complicated machines."\\
        \textbf{Physicist:} "Just a current loop in a magnetic field."
    \end{alertblock}

    \note{[P0] "Schematic of simple DC motor"\\\\
    - "Wire loop in horizontal magnetic field attached to vertical shaft"\\\\
    - "Pass current through loop, magnetic field exerts torque"\\\\
    - "Shaft rotates - electrical energy becomes mechanical"\\\\
    [P1] [THE CONFLICT]\\\\
    - "Looks complex but principle is elegantly simple"\\\\
    - "Current perpendicular to field creates maximum force"\\\\
    - "Force on opposite sides of loop creates rotation"\\\\
    [THE CONNECTION - Kinetic] "Ceiling fan, power tools, electric cars - same design"}
\end{frame}

\begin{frame}
    \frametitle{20.2 Motor Torque Analysis}

    \begin{figure}
        \centering
        \includegraphics[width=0.6\textwidth,height=0.4\textheight,keepaspectratio]{phys12-magnetism-fig05-2.jpg}
    \end{figure}

    \pause

    \begin{block}{Universal Law: Torque on Current Loop}
        \begin{center}
            \Large $\boxed{\tau = NIAB \sin \theta}$
        \end{center}
        Where $N$ = turns, $I$ = current, $A$ = loop area, $B$ = magnetic field
    \end{block}

    \note{[P0] "Top view shows forces on vertical wire segments"\\\\
    - "Current flows into page on one side, out on other"\\\\
    - "Magnetic field pushes both wires, creating torque"\\\\
    - "Distance from pivot is w over 2"\\\\
    [P1] [ALGEBRA] "Tau equals N I A B sine theta"\\\\
    - "More loops N means more torque"\\\\
    - "More current I means more torque"\\\\
    - "Bigger area A means more torque"\\\\
    [THE WONDER] "This equation designed every motor you've ever used"}
\end{frame}

\begin{frame}
    \frametitle{20.2 Keeping the Motor Spinning}

    \begin{figure}
        \centering
        \includegraphics[width=0.7\textwidth,height=0.45\textheight,keepaspectratio]{phys12-magnetism-fig05-3.jpg}
    \end{figure}

    \pause

    \begin{alertblock}{The Problem}
        Torque reverses every half turn. Without \textbf{brushes} to reverse current, motor oscillates instead of rotating.
    \end{alertblock}

    \note{[P0] "The torque problem"\\\\
    - "At theta equals zero, torque is zero"\\\\
    - "Maximum torque at theta equals 90 degrees"\\\\
    - "After 180 degrees, torque reverses direction"\\\\
    - "Loop would just rock back and forth"\\\\
    [P1] [THE REVELATION]\\\\
    - "Brushes are automatic switches"\\\\
    - "Reverse current every half revolution"\\\\
    - "Torque stays clockwise continuously"\\\\
    [THE CONNECTION - Digital] "DC motors in drones use electronic switching instead"}
\end{frame}

\begin{frame}
    \frametitle{20.2 Run Motor in Reverse: Generator}

    \begin{figure}
        \centering
        \includegraphics[width=0.65\textwidth,height=0.45\textheight,keepaspectratio]{phys12-magnetism-fig06-1.jpg}
    \end{figure}

    \pause

    \begin{exampleblock}{The Symmetry}
        \textbf{Motor:} Electrical energy $\rightarrow$ Mechanical energy\\
        \textbf{Generator:} Mechanical energy $\rightarrow$ Electrical energy
    \end{exampleblock}

    \note{[P0] "Motor physics reversed"\\\\
    - "Attach handle to shaft, force coil to rotate"\\\\
    - "Charges in wire moving through magnetic field"\\\\
    - "Experience magnetic force parallel to wire"\\\\
    - "Force pushes charges creating current"\\\\
    [P1] [THE CONNECTION - Kinetic]\\\\
    - "Bicycles: Your legs turn generator, powers headlight"\\\\
    - "Regenerative braking: Wheels turn motor backwards, charges battery"\\\\
    [THE WONDER] "Every power plant on Earth uses this principle"}
\end{frame}

\begin{frame}
    \frametitle{20.2 Generator: The Math}

    Velocity of wire makes angle $\theta$ with magnetic field:

    \begin{figure}
        \centering
        \includegraphics[width=0.5\textwidth,height=0.3\textheight,keepaspectratio]{phys12-magnetism-fig07-1.jpg}
    \end{figure}

    \pause

    \begin{block}{Universal Law: Generator EMF}
        \begin{center}
            \Large $\boxed{\varepsilon = NAB\omega \sin \omega t}$
        \end{center}
        Peak emf: $\varepsilon_0 = NAB\omega$
    \end{block}

    \note{[P0] "Deriving generator output"\\\\
    - "Velocity component perpendicular to B is v sine theta"\\\\
    - "EMF on each wire is B ell v sine theta"\\\\
    - "Two vertical wires contribute"\\\\
    [P1] [ALGEBRA] "Epsilon equals N A B omega sine omega t"\\\\
    - "N is number of turns in coil"\\\\
    - "Omega is angular velocity in radians per second"\\\\
    - "Output oscillates sinusoidally"\\\\
    [THE WONDER] "This is alternating current - AC power in your walls"}
\end{frame}

\begin{frame}
    \frametitle{20.2 AC Power from Generator}

    \begin{figure}
        \centering
        \includegraphics[width=0.7\textwidth,height=0.5\textheight,keepaspectratio]{phys12-magnetism-fig08-1.jpg}
    \end{figure}

    \pause

    \begin{alertblock}{Civilian View vs. Reality}
        \textbf{Civilian:} "Why don't lights flicker 120 times per second?"\\
        \textbf{Physicist:} "Faster than eye refresh rate. We don't notice."
    \end{alertblock}

    \note{[P0] "Generator connected to light bulb"\\\\
    - "Graph shows EMF versus time"\\\\
    - "Oscillates from plus epsilon-zero to minus epsilon-zero"\\\\
    - "Zero crossings mean zero current momentarily"\\\\
    - "In US, 60 Hertz frequency means 120 zero crossings per second"\\\\
    [P1] [THE HUMILITY]\\\\
    - "Your eyes can't detect flicker above about 60 Hertz"\\\\
    - "Plus light bulbs don't instantly turn off"\\\\
    [THE CONNECTION - Harmonic] "Musicians: 60 Hz is between B-flat and B below middle C"}
\end{frame}

\begin{frame}
    \frametitle{20.2 Real Generators: Steam Turbines}

    \begin{figure}
        \centering
        \includegraphics[width=0.7\textwidth,height=0.55\textheight,keepaspectratio]{phys12-magnetism-fig09-1.jpg}
    \end{figure}

    \begin{exampleblock}{Energy Chain}
        Coal/Nuclear/Gas $\rightarrow$ Heat $\rightarrow$ Steam $\rightarrow$ Turbine $\rightarrow$ Generator $\rightarrow$ Electricity
    \end{exampleblock}

    \note{[P0] "Cutaway of steam turbine generator"\\\\
    - "Steam produced by burning coal or nuclear reactor"\\\\
    - "High pressure steam impacts turbine blades"\\\\
    - "Blades connected to shaft inside generator"\\\\
    - "Rotating coil in magnetic field produces electricity"\\\\
    [THE CONNECTION - All]\\\\
    - "Hydroelectric: Falling water turns turbine"\\\\
    - "Wind power: Wind turns turbine"\\\\
    - "Same generator design, different energy source"\\\\
    [THE WONDER] "Powers entire cities from rotating coils"}
\end{frame}

\begin{frame}
    \frametitle{20.2 Transformers: Changing Voltage}

    \begin{columns}
        \begin{column}{0.5\textwidth}
            \begin{figure}
                \centering
                \includegraphics[width=\linewidth,height=0.4\textheight,keepaspectratio]{phys12-magnetism-fig10-3.jpg}
            \end{figure}
        \end{column}
        \begin{column}{0.5\textwidth}
            \pause
            \begin{block}{What Transformers Do}
                Change AC voltage from one value to another\\
                \vspace{0.2cm}
                \textit{Phone chargers, laptop adapters, power tools}
            \end{block}
        \end{column}
    \end{columns}

    \note{[P0] "Two types shown"\\\\
    - "Left: Laminated core transformer - common in power transmission"\\\\
    - "Right: Toroidal transformer - doughnut shaped"\\\\
    - "Notice visible wire coils on both"\\\\
    [P1] [THE CONNECTION - Digital]\\\\
    - "Every device you plug in has a transformer"\\\\
    - "Wall outlet 120V becomes 5V for USB, 19V for laptop"\\\\
    - "Transform voltage to what device needs"\\\\
    [THE WONDER] "Pure electromagnetic induction - no moving parts"}
\end{frame}

\begin{frame}
    \frametitle{20.2 How Transformers Work}

    \begin{figure}
        \centering
        \includegraphics[width=0.65\textwidth,height=0.45\textheight,keepaspectratio]{phys12-magnetism-fig10-2.jpg}
    \end{figure}

    \pause

    \begin{block}{The Principle}
        \begin{enumerate}
            \item AC current in primary coil creates changing magnetic field
            \item Iron core traps and amplifies magnetic field
            \item Changing field passes through secondary coil
            \item Induces AC voltage in secondary coil
        \end{enumerate}
    \end{block}

    \note{[P0] "Transformer construction"\\\\
    - "Two wire coils wound on ferromagnetic iron core"\\\\
    - "Primary coil receives input AC voltage"\\\\
    - "Secondary coil produces output voltage"\\\\
    - "Based on Faraday's law of induction"\\\\
    [P1] "Why AC only?"\\\\
    - "DC produces constant magnetic field"\\\\
    - "No change means no induction"\\\\
    - "AC constantly changing creates constantly changing field"\\\\
    [THE WONDER] "Faraday discovered this in 1831 - still essential technology"}
\end{frame}

\begin{frame}
    \frametitle{20.2 Transformer Equation}

    \begin{block}{Universal Law: Voltage Transformation}
        \begin{center}
            \Large $\boxed{\frac{V_S}{V_P} = \frac{N_S}{N_P}}$
        \end{center}
        \vspace{0.2cm}
        Secondary voltage / Primary voltage = Turns ratio
    \end{block}

    \pause

    \begin{exampleblock}{Step-Up vs. Step-Down}
        \textbf{Step-Up:} $N_S > N_P$ $\rightarrow$ Increases voltage (power transmission)\\
        \textbf{Step-Down:} $N_S < N_P$ $\rightarrow$ Decreases voltage (home delivery)
    \end{exampleblock}

    \note{[P0] [ALGEBRA] "V-S over V-P equals N-S over N-P"\\\\
    - "Ratio of output voltage to input voltage"\\\\
    - "Equals ratio of coil turns"\\\\
    - "More turns on secondary gives higher voltage"\\\\
    [P1] "Two types"\\\\
    - "Step-up: Power plant uses to boost voltage before transmission"\\\\
    - "Step-down: Reduces voltage for safe home use"\\\\
    [THE CONNECTION - Digital] "Your laptop brick is a step-down transformer: 120V to 19V"}
\end{frame}

\begin{frame}
    \frametitle{20.2 Power Transmission: Why High Voltage?}

    \begin{block}{Power Transmitted}
        $$P_{\text{transmitted}} = I_{\text{transmitted}} V_{\text{transmitted}}$$
    \end{block}

    \pause

    \begin{alertblock}{The Problem: Joule Losses}
        $$P_{\text{lost}} = I_{\text{transmitted}}^2 R_{\text{wire}}$$
        Power lost as heat proportional to \textbf{current squared}
    \end{alertblock}

    \pause

    \begin{exampleblock}{The Solution}
        Increase voltage $\rightarrow$ Decrease current $\rightarrow$ Minimize losses
    \end{exampleblock}

    \note{[P0] [ALGEBRA] "P-transmitted equals I-transmitted times V-transmitted"\\\\
    - "To transmit same power, can use high voltage and low current"\\\\
    - "Or low voltage and high current"\\\\
    [P1] [THE CONFLICT]\\\\
    - "P-lost equals I-transmitted squared times R-wire"\\\\
    - "Heating in transmission wires depends on current squared"\\\\
    - "Double current means four times heat loss"\\\\
    [P2] [THE REVELATION]\\\\
    - "Step up voltage massively, current drops proportionally"\\\\
    - "Same power delivered with far less loss"\\\\
    [THE WONDER] "Transmission lines carry 120,000 to 700,000 volts"}
\end{frame}

\begin{frame}
    \frametitle{20.2 The Power Grid}

    \begin{figure}
        \centering
        \includegraphics[width=0.9\textwidth,height=0.65\textheight,keepaspectratio]{phys12-magnetism-fig13-1.jpg}
    \end{figure}

    \note{[P0] "Journey of electricity from plant to home"\\\\
    - "Generated at power station: greater than 10 kilovolts"\\\\
    - "Step-up transformer: 120 to 700 kilovolts for transmission"\\\\
    - "Long distance high-voltage lines minimize current and losses"\\\\
    - "Substation step-down: 5 to 13 kilovolts for local distribution"\\\\
    - "Final step-down at home: 120, 240, or 480 volts"\\\\
    [THE CONNECTION - All]\\\\
    - "Every transformer uses Faraday's law"\\\\
    - "Without transformers, we'd need power plant every few kilometers"\\\\
    [THE WONDER] "Queen Elizabeth asked Faraday what use was electricity - look at us now"}
\end{frame}

\section{20.3 Electromagnetic Induction}

\begin{frame}
    \frametitle{20.3 Nature's Symmetry}

    \begin{block}{We Already Know}
        Electric current creates magnetic field (electromagnet)
    \end{block}

    \pause

    \begin{block}{Faraday's Question (1831)}
        Can magnetic field create electric current?
    \end{block}

    \pause

    \begin{exampleblock}{The Answer}
        \textbf{Yes} - but only when magnetic field \textit{changes}
    \end{exampleblock}

    \note{[P0] "Physics seeks symmetry"\\\\
    - "We know current makes magnetic field"\\\\
    - "Faraday and Henry asked: Does field make current?"\\\\
    [P1] [THE CONNECTION - All]\\\\
    - "Took 12 years after electromagnet discovery"\\\\
    - "1831: Both scientists independently found answer"\\\\
    [P2] [THE REVELATION]\\\\
    - "Not just any field - must be changing"\\\\
    - "Static magnet near wire: nothing"\\\\
    - "Moving magnet near wire: current flows"\\\\
    [THE WONDER] "This discovery powers modern civilization"}
\end{frame}

\begin{frame}
    \frametitle{20.3 Faraday's Experiment}

    \begin{figure}
        \centering
        \includegraphics[width=0.75\textwidth,height=0.6\textheight,keepaspectratio]{phys12-magnetism-fig16-1.jpg}
    \end{figure}

    \note{[P0] "Simple but revolutionary experiment"\\\\
    - "Bar magnet moved through wire coil"\\\\
    - "Galvanometer measures current in coil"\\\\
    - "(a) Magnet stationary: Zero current"\\\\
    - "(b) Magnet moving toward coil: Current flows one direction"\\\\
    - "(c) Magnet moving away: Current flows opposite direction"\\\\
    - "(d) Reverse magnet poles: Current reverses"\\\\
    [THE HUMILITY]\\\\
    - "Sometimes simplest experiments reveal deepest truths"\\\\
    [THE WONDER] "Queen Victoria asked Faraday what good is electricity - he replied: What good is a baby?"}
\end{frame}

\begin{frame}
    \frametitle{20.3 What is EMF?}

    \begin{alertblock}{Terrible Name}
        \textbf{Electromotive Force} is NOT a force\\
        It's a \textit{potential difference} (voltage)
    \end{alertblock}

    \pause

    \begin{block}{Universal Law: EMF Definition}
        \begin{center}
            Energy added per unit charge by source\\
            \vspace{0.2cm}
            Symbol: $\varepsilon$ \quad Units: Volts (V)
        \end{center}
    \end{block}

    \note{[P0] [THE CONFLICT]\\\\
    - "Name stuck from 1800s - causes confusion"\\\\
    - "Faraday thought it was a force pushing charges"\\\\
    - "Actually, external source adds energy to charges"\\\\
    - "Energy per charge has units of volts"\\\\
    [P1] "Key distinction"\\\\
    - "EMF is energy added by source per unit charge"\\\\
    - "Voltage is energy released per unit charge flowing through circuit"\\\\
    - "We use abbreviation EMF, symbol epsilon"\\\\
    [THE CONNECTION - Digital] "Battery EMF is the voltage it provides to circuit"}
\end{frame}

\begin{frame}
    \frametitle{20.3 Understanding Magnetic Flux}

    \begin{figure}
        \centering
        \includegraphics[width=0.65\textwidth,height=0.4\textheight,keepaspectratio]{phys12-magnetism-fig18-2.jpg}
    \end{figure}

    \pause

    \begin{block}{Universal Law: Magnetic Flux}
        \begin{center}
            \Large $\boxed{\Phi = BA \cos \theta}$
        \end{center}
        \vspace{0.2cm}
        Number of field lines perpendicular through area $A$\\
        Unit: Weber (Wb) = T·m$^2$ = V·s
    \end{block}

    \note{[P0] "What creates EMF?"\\\\
    - "Moving magnet changes number of field lines through loop"\\\\
    - "Left: Seven field lines through loop"\\\\
    - "Right: After time delta-t, only five lines"\\\\
    - "Change in field lines induces EMF"\\\\
    [P1] [ALGEBRA] "Phi equals B A cosine theta"\\\\
    - "B is magnetic field strength in tesla"\\\\
    - "A is area of loop"\\\\
    - "Theta is angle between field and perpendicular to loop"\\\\
    [THE CONNECTION - Harmonic] "Like sound intensity through window - depends on angle"}
\end{frame}

\begin{frame}
    \frametitle{20.3 Flux and Loop Orientation}

    \begin{figure}
        \centering
        \includegraphics[width=0.75\textwidth,height=0.5\textheight,keepaspectratio]{phys12-magnetism-fig18-3.jpg}
    \end{figure}

    \pause

    \begin{exampleblock}{The Sail Analogy}
        \textbf{Loop = Sail, Magnetic Field = Wind}\\
        Maximum flux when perpendicular ($\theta = 0°$)\\
        Zero flux when parallel ($\theta = 90°$)
    \end{exampleblock}

    \note{[P0] "Angle matters"\\\\
    - "Left loop: Field parallel to plane, perpendicular to area vector"\\\\
    - "Theta equals 90 degrees, cosine is zero, no flux"\\\\
    - "Right loop: Field perpendicular to plane, parallel to area vector"\\\\
    - "Theta equals zero, cosine is one, maximum flux"\\\\
    [P1] [THE CONNECTION - Kinetic]\\\\
    - "Sailors: Sail perpendicular to wind catches maximum force"\\\\
    - "Sail parallel to wind catches nothing"\\\\
    - "Same with magnetic field through loop"\\\\
    [THE WONDER] "Rotate loop from left to right, flux increases, EMF induced"}
\end{frame}

\begin{frame}
    \frametitle{20.3 Faraday's Law of Induction}

    \begin{block}{Universal Law: Faraday's Law}
        \begin{center}
            \Large $\boxed{\varepsilon = -N \frac{\Delta \Phi}{\Delta t}}$
        \end{center}
        \vspace{0.2cm}
        EMF induced equals rate of change of magnetic flux
    \end{block}

    \pause

    \begin{exampleblock}{Three Ways to Induce EMF}
        \begin{enumerate}
            \item Change magnetic field strength $B$
            \item Change loop area $A$
            \item Change orientation angle $\theta$
        \end{enumerate}
    \end{exampleblock}

    \note{[P0] [THE REVELATION]\\\\
    [ALGEBRA] "Epsilon equals negative N times delta-Phi over delta-t"\\\\
    - "N is number of loops in coil"\\\\
    - "Delta-Phi over delta-t is rate of change of flux"\\\\
    - "Minus sign explained by Lenz's law"\\\\
    [P1] "Three mechanisms"\\\\
    - "Stronger or weaker magnet changes B"\\\\
    - "Expand or shrink loop changes A"\\\\
    - "Rotate loop changes theta"\\\\
    - "Any change in product B-A-cosine-theta induces EMF"\\\\
    [THE WONDER] "This single equation designed every generator and transformer on Earth"}
\end{frame}

\begin{frame}
    \frametitle{20.3 Lenz's Law: The Minus Sign}

    \begin{block}{Universal Law: Lenz's Law}
        Induced current flows in direction that \textit{opposes} the change in flux
    \end{block}

    \pause

    \begin{alertblock}{Nature Resists Change}
        Flux increasing? $\rightarrow$ Induced field opposes increase\\
        Flux decreasing? $\rightarrow$ Induced field opposes decrease
    \end{alertblock}

    \note{[P0] "Why the minus sign?"\\\\
    - "Russian scientist Heinrich Lenz explained it"\\\\
    - "Induced current creates magnetic field"\\\\
    - "That field tries to keep flux constant"\\\\
    - "Nature's way of resisting change"\\\\
    [P1] [THE CONNECTION - All]\\\\
    - "Like inertia for magnetism"\\\\
    - "Objects resist change in motion (Newton's First Law)"\\\\
    - "Circuits resist change in magnetic flux (Lenz's Law)"\\\\
    [THE HUMILITY]\\\\
    - "Took years to understand the direction"\\\\
    - "Right-hand rule helps us predict current flow"}
\end{frame}

\begin{frame}
    \frametitle{20.3 Applying Lenz's Law}

    \begin{figure}
        \centering
        \includegraphics[width=0.8\textwidth,height=0.65\textheight,keepaspectratio]{phys12-magnetism-fig21-1.jpg}
    \end{figure}

    \note{[P0] "Three scenarios with right-hand rule"\\\\
    - "(a) North pole approaches: B-mag pointing right increases"\\\\
    - "Induced B-coil points left to oppose increase"\\\\
    - "Curl fingers counterclockwise, thumb points left"\\\\
    - "(b) North pole retreats: B-mag pointing right decreases"\\\\
    - "Induced B-coil points right to oppose decrease"\\\\
    - "Curl fingers clockwise, thumb points right"\\\\
    - "(c) South pole approaches: B-mag pointing left increases"\\\\
    - "Induced B-coil points right to oppose increase"\\\\
    [THE CONNECTION - Kinetic] "Like pushing someone on swing - they push back"}
\end{frame}

\begin{frame}
    \frametitle{Attempt: EMF in Moving Coil}

    \textbf{Try this on your own (3 min, silent):}

    \vspace{0.3cm}

    A magnetic field passes through a 16-turn coil with diameter 2.0 cm. The magnetic field decreases from 0.020 T to 0.010 T in 34 s. The coil has resistance 0.1 $\Omega$.

    \vspace{0.3cm}

    \textbf{Given:}
    \begin{itemize}
        \item $N = 16$ turns
        \item $d = 0.020$ m
        \item $\Delta B = -0.010$ T
        \item $\Delta t = 34$ s
        \item $R = 0.1$ $\Omega$
    \end{itemize}

    \textbf{Find:} Magnitude and direction of induced current

    \vspace{0.2cm}

    \textit{Work individually. It's okay to get stuck.}

    \note{[THE CHALLENGE] "Can you find the induced current?"\\\\
    [TIMING] 3-4 min SILENT work\\\\
    [CIRCULATE] Note who uses Faraday's law correctly\\\\
    - "Watch for area calculation using diameter"\\\\
    - "Watch for Ohm's law application"\\\\
    - "Don't help yet - productive struggle"}
\end{frame}

\begin{frame}
    \frametitle{Compare: EMF in Moving Coil}

    \textbf{Turn and talk (2 min):}

    \begin{enumerate}
        \item What equation did you use for EMF?
        \item How did you calculate the magnetic flux?
        \item How did you find current from EMF?
        \item What direction does current flow?
    \end{enumerate}

    \pause
    \vspace{0.3cm}
    \alert{Name wheel:} One pair share your approach (not your answer).

    \note{[TIMING] 2-3 min pair discussion\\\\
    [CIRCULATE] Listen for approach\\\\
    - "Should use Faraday's law for EMF"\\\\
    - "Flux uses area equals pi d-squared over 4"\\\\
    - "Ohm's law gives current from EMF"\\\\
    - "Lenz's law determines direction"\\\\
    [NAME WHEEL] Call a pair\\\\
    - "Listen for equation choice"\\\\
    [EXPECTED APPROACH] Use epsilon equals negative N delta-Phi over delta-t, then I equals epsilon over R}
\end{frame}

\begin{frame}
    \frametitle{Reveal: Solution}

    \textbf{Self-correct in a different color:}

    \textbf{G - Given:} See problem

    \pause

    \textbf{U - Unknown:} $I = ?$

    \pause

    \textbf{E - Equations:}
    \begin{align*}
        \varepsilon &= -N \frac{\Delta \Phi}{\Delta t} = -N \frac{\Delta B \pi d^2}{4 \Delta t}\\
        I &= \frac{\varepsilon}{R}
    \end{align*}

    \pause

    \textbf{S - Substitute:}
    $$I = -16 \frac{(-0.010\text{ T}) \pi (0.020\text{ m})^2}{4(0.10\, \Omega)(34\text{ s})} = 15\, \mu\text{A}$$

    \pause

    \textbf{S - Statement:} $\boxed{I = 15\, \mu\text{A}}$ to the right (opposes decrease in field)

    \note{[ALGEBRA] Step by step\\\\
    [P1] "Unknown: Current in microamps"\\\\
    [P2] "Combine Faraday's law with flux equation"\\\\
    - "Area of circle: pi d-squared over 4"\\\\
    [P3] "Substitute numbers"\\\\
    - "Note: Delta-B is negative 0.010 tesla"\\\\
    - "Negatives cancel giving positive current"\\\\
    [P4] [ANSWER] Current is 15 microamps flowing to right\\\\
    - "Lenz says current opposes decrease, so produces field to right"\\\\
    [THE WONDER] "You just calculated what powers every generator"}
\end{frame}

\begin{frame}
    \frametitle{Attempt: Sliding Rod Circuit}

    \textbf{Try this on your own (3 min, silent):}

    \vspace{0.2cm}

    A U-shaped wire with a 20 $\Omega$ resistor has a conducting rod sliding on it at 0.50 m/s. The circuit is in a constant 0.010 T magnetic field pointing into the page. The rod is 0.10 m long.

    \begin{figure}
        \centering
        \includegraphics[width=0.5\textwidth,height=0.25\textheight,keepaspectratio]{phys12-magnetism-fig25-1.jpg}
    \end{figure}

    \textbf{Given:} $B = 0.010$ T, $v = 0.50$ m/s, $\ell = 0.10$ m, $R = 20\, \Omega$

    \textbf{Find:} Current magnitude and direction

    \vspace{0.2cm}

    \textit{Work individually. It's okay to get stuck.}

    \note{[THE CHALLENGE] "Rod slides, area changes, flux changes, current flows"\\\\
    [TIMING] 3-4 min SILENT work\\\\
    [CIRCULATE] Watch for area change rate\\\\
    - "Some will struggle with changing area instead of changing field"\\\\
    - "That's good - different application of same law"\\\\
    - "Don't help yet"}
\end{frame}

\begin{frame}
    \frametitle{Compare: Sliding Rod Circuit}

    \textbf{Turn and talk (2 min):}

    \begin{enumerate}
        \item What changes: $B$, $A$, or $\theta$?
        \item How fast does area change?
        \item What's the rate of flux change?
        \item Direction of induced current?
    \end{enumerate}

    \pause
    \vspace{0.3cm}
    \alert{Name wheel:} One pair share your approach (not your answer).

    \note{[TIMING] 2-3 min pair discussion\\\\
    [CIRCULATE] Listen for recognition\\\\
    - "Area changes, field constant"\\\\
    - "Rate of area change is delta-A over delta-t equals v times ell"\\\\
    - "Flux change rate is B times v times ell"\\\\
    [NAME WHEEL] Call a pair\\\\
    [EXPECTED APPROACH] Area increasing at rate v-ell, so flux increasing at rate B-v-ell}
\end{frame}

\begin{frame}
    \frametitle{Reveal: Sliding Rod Solution}

    \textbf{Self-correct in a different color:}

    \textbf{E - Equation:}
    $$\frac{\Delta \Phi}{\Delta t} = B \frac{\Delta A}{\Delta t} = B v \ell$$

    \pause

    $$\varepsilon = -B v \ell$$

    \pause

    \textbf{S - Substitute:}
    $$I = \frac{\varepsilon}{R} = \frac{B v \ell}{R} = \frac{(0.010\text{ T})(0.50\text{ m/s})(0.10\text{ m})}{20\,\Omega}$$

    \pause

    $$\boxed{I = 25\,\mu\text{A}}$$ flowing \textbf{clockwise}

    \pause

    \textbf{Check:} Lenz's law - flux into page increasing, so induced field out of page (counterclockwise current would point out, but we have sign... current is clockwise)

    \note{[P1] [ALGEBRA] "Rate of area change is v times ell"\\\\
    - "EMF equals negative B v ell"\\\\
    [P2] "Ohm's law"\\\\
    [P3] "Substitute numbers"\\\\
    - "0.010 times 0.50 times 0.10 over 20"\\\\
    [P4] [ANSWER] Current is 25 microamps\\\\
    - "Direction: Flux into page increasing"\\\\
    - "Induced field must point out of page to oppose"\\\\
    - "Right-hand rule: Clockwise current makes field out"\\\\
    [P5] [THE WONDER] "External force pulling rod does work, circuit dissipates power - energy conserved"}
\end{frame}

\section{Summary}

\begin{frame}
    \frametitle{Key Equations Summary}

    \begin{block}{Motors and Generators}
        \begin{align*}
            F &= I\ell B \sin\theta \quad \text{(Force on wire)}\\
            \tau &= NIAB\sin\theta \quad \text{(Motor torque)}\\
            \varepsilon &= NAB\omega\sin\omega t \quad \text{(Generator EMF)}
        \end{align*}
    \end{block}

    \begin{block}{Transformers and Power}
        \begin{align*}
            \frac{V_S}{V_P} &= \frac{N_S}{N_P} \quad \text{(Transformer equation)}\\
            P_{\text{lost}} &= I^2 R \quad \text{(Joule heating)}
        \end{align*}
    \end{block}

    \note{[P0] "Five essential equations for motors, generators, transformers"\\\\
    - "Force on current-carrying wire: F equals I ell B sine theta"\\\\
    - "Torque on motor loop: tau equals N I A B sine theta"\\\\
    - "Generator output: epsilon equals N A B omega sine omega t"\\\\
    - "Transformer voltage ratio: V-S over V-P equals turns ratio"\\\\
    - "Power loss in transmission: I-squared R - why we use high voltage"}
\end{frame}

\begin{frame}[shrink]
    \frametitle{Key Equations Summary (continued)}

    \begin{block}{Electromagnetic Induction}
        \begin{align*}
            \Phi &= BA\cos\theta \quad \text{(Magnetic flux)}\\
            \varepsilon &= -N\frac{\Delta\Phi}{\Delta t} \quad \text{(Faraday's Law)}\\
            \varepsilon &= B\ell v \quad \text{(Motional EMF)}
        \end{align*}
    \end{block}

    \vspace{0.3cm}

    \begin{exampleblock}{Lenz's Law (Direction)}
        Induced current opposes the change in magnetic flux
    \end{exampleblock}

    \note{"Three more for electromagnetic induction"\\\\
    - "Magnetic flux: Phi equals B A cosine theta"\\\\
    - "Faraday's law: epsilon equals negative N delta-Phi over delta-t"\\\\
    - "Motional EMF for moving wire: epsilon equals B ell v"\\\\
    - "Lenz's law determines current direction"\\\\
    [THE WONDER] "Eight equations that power modern civilization"}
\end{frame}

\begin{frame}
    \frametitle{The Big Picture}

    \begin{exampleblock}{Symmetry of Electromagnetism}
        \begin{center}
            Electricity $\leftrightarrow$ Magnetism
        \end{center}
    \end{exampleblock}

    \pause

    \begin{itemize}
        \item Electric current creates magnetic field (electromagnet)
        \pause
        \item Changing magnetic field creates electric current (induction)
        \pause
        \item Motors convert electrical $\rightarrow$ mechanical energy
        \pause
        \item Generators convert mechanical $\rightarrow$ electrical energy
        \pause
        \item Transformers change voltage using induction
        \pause
        \item Power grid uses transformers to transmit energy efficiently
    \end{itemize}

    \note{[P0] "Beautiful symmetry"\\\\
    [P1] "We knew current makes field"\\\\
    [P2] "Faraday showed field makes current"\\\\
    [P3] "Motors: Electricity pushes wires in magnetic field"\\\\
    [P4] "Generators: Motion creates changing flux, induces current"\\\\
    [P5] "Transformers: Changing current in one coil induces voltage in another"\\\\
    [P6] "Power grid: Step up, transmit, step down"\\\\
    [THE WONDER] "From aurora to smartphone charger - all electromagnetic induction"}
\end{frame}

\begin{frame}
    \frametitle{Homework}

    \begin{center}
        \Large
        Complete the assigned problems\\[0.3cm]
        posted on the LMS
    \end{center}

    \note{Homework posted on LMS\\\\
    Due date: check LMS\\\\
    Questions before we end?\\\\
    [THE CONNECTION - All]\\\\
    - "Tonight when you plug in your phone, remember the transformer"\\\\
    - "When you turn on a light, thank Faraday and his law"\\\\
    [THE WONDER] "You now understand the physics that powers the world"}
\end{frame}

\end{document}
