% Options for packages loaded elsewhere
\PassOptionsToPackage{unicode}{hyperref}
\PassOptionsToPackage{hyphens}{url}
\documentclass[]{article}
\usepackage{xcolor}
\usepackage{graphicx}
\graphicspath{{docx-images/}{../images/}{../../shared/images/}}
\usepackage[margin=1in]{geometry}
\usepackage{amsmath,amssymb}
\setcounter{secnumdepth}{-\maxdimen} % remove section numbering
\usepackage{iftex}
\ifPDFTeX
  \usepackage[T1]{fontenc}
  \usepackage[utf8]{inputenc}
  \usepackage{textcomp} % provide euro and other symbols
\else % if luatex or xetex
  \usepackage{unicode-math} % this also loads fontspec
  \defaultfontfeatures{Scale=MatchLowercase}
  \defaultfontfeatures[\rmfamily]{Ligatures=TeX,Scale=1}
\fi
\usepackage{lmodern}
\ifPDFTeX\else
  % xetex/luatex font selection
\fi
% Use upquote if available, for straight quotes in verbatim environments
\IfFileExists{upquote.sty}{\usepackage{upquote}}{}
\IfFileExists{microtype.sty}{% use microtype if available
  \usepackage[]{microtype}
  \UseMicrotypeSet[protrusion]{basicmath} % disable protrusion for tt fonts
}{}
\makeatletter
\@ifundefined{KOMAClassName}{% if non-KOMA class
  \IfFileExists{parskip.sty}{%
    \usepackage{parskip}
  }{% else
    \setlength{\parindent}{0pt}
    \setlength{\parskip}{6pt plus 2pt minus 1pt}}
}{% if KOMA class
  \KOMAoptions{parskip=half}}
\makeatother
\usepackage{longtable,booktabs,array}
\newcounter{none} % for unnumbered tables
\usepackage{calc} % for calculating minipage widths
% Correct order of tables after \paragraph or \subparagraph
\usepackage{etoolbox}
\makeatletter
\patchcmd\longtable{\par}{\if@noskipsec\mbox{}\fi\par}{}{}
\makeatother
% Allow footnotes in longtable head/foot
\IfFileExists{footnotehyper.sty}{\usepackage{footnotehyper}}{\usepackage{footnote}}
\makesavenoteenv{longtable}
\setlength{\emergencystretch}{3em} % prevent overfull lines
\providecommand{\tightlist}{%
  \setlength{\itemsep}{0pt}\setlength{\parskip}{0pt}}
\usepackage{bookmark}
\IfFileExists{xurl.sty}{\usepackage{xurl}}{} % add URL line breaks if available
\urlstyle{same}
\hypersetup{
  hidelinks,
  pdfcreator={LaTeX via pandoc}}

\author{}
\date{}

\begin{document}

\textbf{Physics Lab: Graphs and Motion Analysis}

Objective

To develop an intuitive understanding of the relationships between
position, velocity, and acceleration graphs for one-dimensional motion
using the Graphs and Tracks simulator.

Materials

- Computer with internet access

- Graphs and Tracks simulator (https://graphsandtracks.com)

Introduction

In this lab, you will use the Graphs and Tracks simulator to explore how
changes in a ball\textquotesingle s motion are represented in position,
velocity, and acceleration graphs. You will complete a series of
challenges that require you to match given motion graphs by adjusting
the initial conditions and track setup.

Procedure

Part 1: Tutorial and Familiarization

1. Open the Graphs and Tracks simulator.

2. Complete the built-in tutorial, paying close attention to:

- How to read and interpret position, velocity, and acceleration graphs

- How to adjust the initial position and velocity of the ball

- How to modify the track setup by changing post heights

Part 2: Challenges

3. Navigate to the Challenges section of the simulator.

4. Complete all 15 challenges in order, from Easy to Hard difficulty.
For each challenge:

- Analyze the given graphs (dashed lines) for position, velocity, and
acceleration.

- Adjust the initial position and velocity of the ball.

- Modify the track setup to match the given graphs.

- Use the "Roll Ball" button to test your setup and compare your graphs
(solid lines) to the challenge graphs.

- Continue refining your setup until all three graphs match the
challenge graphs.

5. For each challenge, \textbf{record in a .docx file}:

- The challenge number and difficulty level

- The initial position and velocity you used

- A sketch/screenshot or description of your final track setup

- The number of attempts it took to solve the challenge

- Any key insights or strategies you developed

Part 3: Analysis and Reflection

\begin{enumerate}
\def\labelenumi{\arabic{enumi}.}
\setcounter{enumi}{5}
\item
  After completing all 15 challenges, answer the following questions:
\end{enumerate}

\begin{enumerate}
\def\labelenumi{\alph{enumi})}
\item
  How does the slope of a position-time graph relate to velocity?
\item
  What does a horizontal line on a velocity-time graph represent?
\item
  How can you determine if acceleration is positive, negative, or zero
  from a velocity-time graph?
\item
  Describe the relationship between the shape of a track segment and the
  corresponding acceleration graph.
\item
  Explain how you can use the area under a velocity-time graph to
  determine displacement.
\item
  What was the most challenging aspect of matching the graphs, and how
  did you overcome it?
\end{enumerate}

Extension (Optional)

Create your own challenge using the simulator\textquotesingle s editor
mode. Design a set of graphs that represent an interesting motion
scenario, and challenge a classmate to solve it. Explain the physical
situation your graphs represent.

{\def\LTcaptype{none} % do not increment counter
\begin{longtable}[]{@{}
  >{\raggedright\arraybackslash}p{(\linewidth - 8\tabcolsep) * \real{0.1943}}
  >{\raggedright\arraybackslash}p{(\linewidth - 8\tabcolsep) * \real{0.1972}}
  >{\raggedright\arraybackslash}p{(\linewidth - 8\tabcolsep) * \real{0.1980}}
  >{\raggedright\arraybackslash}p{(\linewidth - 8\tabcolsep) * \real{0.2076}}
  >{\raggedright\arraybackslash}p{(\linewidth - 8\tabcolsep) * \real{0.2029}}@{}}
\toprule\noalign{}
\endhead
\bottomrule\noalign{}
\endlastfoot
Criteria & Excellent (4) & Good (3) & Satisfactory (2) & Needs
Improvment(1) \\
Completion, Documentation, and Analysis & All 15 challenges completed
with detailed records. Well-organized documentation. Thoughtful,
detailed answers to all reflection questions. & 12-14 challenges
completed with adequate records. Mostly organized documentation. Good
answers to most reflection questions with some depth. & 9-11 challenges
completed with some records. Somewhat disorganized documentation. Basic
answers to reflection questions with limited depth. & Less than 9
challenges completed. Poorly organized documentation. Incomplete or
superficial answers to reflection questions. \\
\end{longtable}
}

\end{document}
