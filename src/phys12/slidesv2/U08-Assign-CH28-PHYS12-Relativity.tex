\documentclass[12pt]{article}
\usepackage[margin=1in]{geometry}
\usepackage{amsmath}
\usepackage{graphicx}
\usepackage{hyperref}
\usepackage{enumitem}
\usepackage{fancyhdr}

\setlength{\headheight}{15pt}

\pagestyle{fancy}
\fancyhf{}
\rhead{Special Relativity Project}
\lhead{Physics 12}
\cfoot{\thepage}

\title{\textbf{Special Relativity: Socratic Dialogue \& Video Project}}
\author{Physics 12}
\date{}

\begin{document}

\maketitle

\section{Overview}
You will work in groups of 5 to create a written dialogue and video presentation exploring one of six relativistic scenarios. Your dialogue will feature two characters with different perspectives on a relativistic situation, and your video will demonstrate the key physics concepts with calculations and visual aids.

\section{Your Assignment Has Two Parts}

\subsection{Part 1: Written Dialogue (60\% of grade)}
Write a 4--5 page script-format dialogue between two characters who have different reference frames or perspectives on a relativistic situation.

\subsubsection*{Format Requirements:}
\begin{itemize}[noitemsep]
    \item Use script format with character names in \textbf{bold}
    \item Include stage directions in \textit{italics} (e.g., \textit{Character A draws diagram on board})
    \item Double-spaced
    \item Include at least 3 calculation blocks showing the algebraic work
    \item \textbf{Must be submitted as PDF only}
\end{itemize}

\subsubsection*{Content Requirements:}
\begin{itemize}[noitemsep]
    \item Set up the scenario clearly at the beginning
    \item Both characters must present their perspective with supporting calculations
    \item Show an apparent paradox or contradiction
    \item Resolve the paradox using special relativity concepts
    \item End with both characters understanding why both perspectives are valid
\end{itemize}

\subsubsection*{Your dialogue must include:}
\begin{itemize}[noitemsep]
    \item All required calculations for your topic (see topic details below)
    \item Algebraic work shown step-by-step before plugging in numbers
    \item Correct use of units throughout
    \item Reference to at least one real-world application or experiment
    \item Discussion of what each observer actually measures
\end{itemize}

\newpage

\subsection{Part 2: Video Presentation (40\% of grade)}
Create a 5--7 minute video that explains and demonstrates the physics from your dialogue.

\subsubsection*{Technical Requirements:}
\begin{itemize}[noitemsep]
    \item 5--7 minutes long (videos outside this range will lose points)
    \item Minimum 720p resolution, maximum 1080p resolution
    \item Clear audio (test before final recording!)
    \item \textbf{Submit as MP4 or MOV file only -- no YouTube links, no Google Drive links}
    \item All group members must appear on camera at some point
\end{itemize}

\subsubsection*{Required Video Segments:}

\textbf{Segment 1: Calculations (2 minutes)}
\begin{itemize}[noitemsep]
    \item At least one group member visible on camera
    \item Show handwritten calculations on whiteboard, paper, or tablet
    \item Write clearly and explain each step
    \item \textbf{Show algebraic manipulation first, then plug in numbers at the end}
    \item Must include all required calculations for your topic
\end{itemize}

\textbf{Segment 2: Visual Demonstration (2--3 minutes)}
\begin{itemize}[noitemsep]
    \item Create at least one original diagram, graph, or spacetime diagram
    \item Can be hand-drawn or created digitally, but must show it being explained
    \item Use props, animations, or demonstrations to illustrate the concept
    \item Make it visual -- don't just talk!
\end{itemize}

\textbf{Segment 3: Dialogue Performance (1--2 minutes)}
\begin{itemize}[noitemsep]
    \item Act out a key section of your written dialogue
    \item Can use two group members, or one person with props/voices
    \item Should show the paradox or the resolution moment
    \item Be creative -- use stuffed animals, action figures, puppets, or costumes if you want!
\end{itemize}

\textbf{Segment 4: Real-World Connection (1 minute)}
\begin{itemize}[noitemsep]
    \item Explain one real-world application or experimental evidence
    \item Connect it to your calculations
    \item Examples: GPS satellites, particle accelerators, cosmic rays, atomic clocks on airplanes
\end{itemize}

\subsubsection*{What NOT to do:}
\begin{itemize}[noitemsep]
    \item Don't just read your dialogue on camera
    \item Don't use AI-generated voices
    \item Don't show only typed text or PowerPoint slides
    \item Don't use other people's animations without creating your own explanations
    \item Don't plug numbers into formulas without showing the algebraic steps first
\end{itemize}

\newpage

\section{The Six Topics}

\subsection{Topic 1: The Twin Paradox}

\textbf{Characters:} Twin A (stays on Earth) and Twin B (travels to Alpha Centauri and back)

\subsubsection*{Scenario:}
\begin{itemize}[noitemsep]
    \item Twin B travels to Alpha Centauri (4.3 light-years away)
    \item Spaceship travels at $v = 0.90c$
    \item Twin B immediately returns at the same speed
\end{itemize}

\subsubsection*{The Paradox:}
\begin{itemize}[noitemsep]
    \item Twin A says: ``You're moving, so YOUR clock is slow. You'll age less.''
    \item Twin B says: ``No, YOU'RE moving relative to me. YOU should age less.''
    \item But when they reunite, they must agree on their ages!
\end{itemize}

\subsubsection*{Your dialogue must address:}
\begin{itemize}[noitemsep]
    \item Why both twins initially think the other will age less
    \item How much each twin actually ages (with calculations)
    \item What breaks the symmetry (why isn't it a true paradox)
    \item The role of acceleration/deceleration
    \item Why both observers' measurements are valid in their own frames
\end{itemize}

\subsubsection*{Required calculations:}
\begin{itemize}[noitemsep]
    \item $\gamma$ for $v = 0.90c$
    \item Earth time for journey (both ways)
    \item Twin B's proper time for journey
    \item Age difference when twins reunite
\end{itemize}

\textbf{Key concept:} Proper time vs. coordinate time, asymmetry of acceleration

\subsection{Topic 2: The Michelson-Morley Experiment}

\textbf{Characters:} Pre-1905 physicist (believes in ether) and Post-1905 physicist (accepts Einstein's postulates)

\subsubsection*{Scenario:}
\begin{itemize}[noitemsep]
    \item Discussion of why the Michelson-Morley experiment was expected to detect Earth's motion through the ether
    \item The shocking null result
    \item How Einstein's postulates explain the result
\end{itemize}

\subsubsection*{The Paradox:}
\begin{itemize}[noitemsep]
    \item Light should travel at different speeds depending on Earth's motion through ether
    \item Experiment should detect this as a fringe shift
    \item But NO fringe shift was detected!
\end{itemize}

\subsubsection*{Your dialogue must address:}
\begin{itemize}[noitemsep]
    \item What the ether theory predicted
    \item How the interferometer experiment works
    \item Why no fringe shift was detected
    \item How Einstein's second postulate (constant speed of light) eliminates the problem
    \item How length contraction and time dilation make constant $c$ possible
\end{itemize}

\subsubsection*{Required calculations:}
\begin{itemize}[noitemsep]
    \item Expected time difference for light traveling parallel vs. perpendicular to Earth's motion
    \item Use Earth's orbital velocity: $v \approx 30$ km/s
    \item Show why the expected effect would be measurable
    \item Calculate why constant $c$ eliminates the fringe shift
\end{itemize}

\textbf{Key concept:} Speed of light is constant in all inertial frames, experimental basis for relativity

\newpage

\subsection{Topic 3: Relativity of Simultaneity}

\textbf{Characters:} Observer A (on train platform) and Observer B (on moving train)

\subsubsection*{Scenario:}
\begin{itemize}[noitemsep]
    \item Train moves at $v = 0.60c$ past the platform
    \item Lightning strikes both ends of a 30-meter train car
    \item Observer A (on platform, midway between strike points) sees both flashes simultaneously
    \item Observer B (on train, at midpoint of train car) sees the front flash first
\end{itemize}

\subsubsection*{The Paradox:}
\begin{itemize}[noitemsep]
    \item Observer A says: ``The strikes were simultaneous''
    \item Observer B says: ``No! The front strike happened first!''
    \item Both are at the midpoint of the measured distance
    \item Who is right?
\end{itemize}

\subsubsection*{Your dialogue must address:}
\begin{itemize}[noitemsep]
    \item Why Observer A concludes the strikes were simultaneous
    \item Why Observer B concludes they were NOT simultaneous
    \item How length contraction affects what each observer measures
    \item Why both observers are correct in their own reference frames
    \item What ``simultaneous'' really means
\end{itemize}

\subsubsection*{Required calculations:}
\begin{itemize}[noitemsep]
    \item $\gamma$ for $v = 0.60c$
    \item Contracted length of train (in platform frame)
    \item Time for light to reach each observer
    \item Time difference between the strikes (in Observer B's frame)
\end{itemize}

\textbf{Key concept:} Simultaneity is relative, events simultaneous in one frame are not simultaneous in another

\subsection{Topic 4: Cosmic Ray Muons}

\textbf{Characters:} A muon (traveling from upper atmosphere) and an Earth-bound observer

\subsubsection*{Scenario:}
\begin{itemize}[noitemsep]
    \item Muons are created at 15 km altitude
    \item Muons have a half-life of 2.2 $\mu$s when at rest
    \item Muons travel at $v = 0.998c$ toward Earth
    \item Classical physics predicts they should decay before reaching the ground
    \item But we detect thousands at sea level!
\end{itemize}

\subsubsection*{The Paradox:}
\begin{itemize}[noitemsep]
    \item Earth observer says: ``The muon's clock is slow due to time dilation, so it lives longer''
    \item Muon says: ``I'm at rest in my own frame. My clock shows my proper half-life. How can I travel 15 km in just 2.2 $\mu$s?''
\end{itemize}

\subsubsection*{Your dialogue must address:}
\begin{itemize}[noitemsep]
    \item Classical calculation showing muons shouldn't reach the ground
    \item Earth observer's explanation (time dilation)
    \item Muon's perspective (length contraction of the atmosphere)
    \item Why both perspectives give the same result
    \item Experimental evidence (actual detection rates)
\end{itemize}

\subsubsection*{Required calculations:}
\begin{itemize}[noitemsep]
    \item $\gamma$ for $v = 0.998c$
    \item Classical distance muon can travel before decay
    \item Dilated half-life (Earth observer's frame)
    \item Contracted atmosphere height (muon's frame)
    \item Time to reach ground (muon's frame)
\end{itemize}

\textbf{Key concept:} Time dilation and length contraction are two perspectives on the same phenomenon

\newpage

\subsection{Topic 5: Relativistic Velocity Addition}

\textbf{Characters:} Engineer (proposes exceeding light speed) and Physicist (explains why it's impossible)

\subsubsection*{Scenario:}
\begin{itemize}[noitemsep]
    \item Spaceship travels at $v = 0.90c$ relative to Earth
    \item Spaceship fires a missile forward at $u' = 0.90c$ (in the ship's frame)
    \item Engineer claims: ``$0.90c + 0.90c = 1.80c$! We've exceeded light speed!''
    \item Physicist explains why this is wrong
\end{itemize}

\subsubsection*{The Paradox:}
\begin{itemize}[noitemsep]
    \item Velocities seem like they should add simply
    \item Classical addition gives speeds greater than $c$
    \item But the second postulate says nothing can exceed $c$
    \item How do we reconcile this?
\end{itemize}

\subsubsection*{Your dialogue must address:}
\begin{itemize}[noitemsep]
    \item Classical velocity addition and why it seems obvious
    \item What an Earth observer actually measures
    \item The correct relativistic velocity addition formula
    \item Why you can never reach $c$ by adding velocities
    \item What happens when you add any velocity to the speed of light
\end{itemize}

\subsubsection*{Required calculations:}
\begin{itemize}[noitemsep]
    \item Classical velocity addition result
    \item Relativistic velocity addition: $u = \frac{v + u'}{1 + \frac{vu'}{c^2}}$
    \item Try multiple cases: different values of $v$ and $u'$
    \item Special case: show that adding any $v$ to $c$ still gives $c$
    \item $\gamma$ values to compare kinetic energies
\end{itemize}

\textbf{Key concept:} Velocity addition is not simple arithmetic at relativistic speeds

\subsection{Topic 6: The Ladder Paradox}

\textbf{Characters:} Runner (carrying ladder) and Observer (with garage)

\subsubsection*{Scenario:}
\begin{itemize}[noitemsep]
    \item Runner carries a 10-meter ladder at $v = 0.866c$
    \item Observer has a 5-meter garage
    \item Observer says: ``The ladder contracts to 5 meters -- it fits in my garage! I can close both doors at once.''
    \item Runner says: ``My ladder is 10 meters. Your garage contracts to 2.5 meters. The ladder can NEVER fit!''
\end{itemize}

\subsubsection*{The Paradox:}
\begin{itemize}[noitemsep]
    \item Observer sees ladder fit in garage
    \item Runner sees ladder much longer than garage
    \item Can both doors close simultaneously or not?
    \item Physical reality must be the same for both!
\end{itemize}

\subsubsection*{Your dialogue must address:}
\begin{itemize}[noitemsep]
    \item Length contraction from both perspectives
    \item What ``simultaneously'' means to each observer
    \item When each door closes in each reference frame
    \item The role of relativity of simultaneity in the resolution
    \item What actually happens (does the ladder get damaged or not?)
\end{itemize}

\subsubsection*{Required calculations:}
\begin{itemize}[noitemsep]
    \item $\gamma$ for $v = 0.866c$ (this equals 2.0 exactly)
    \item Contracted ladder length (observer's frame)
    \item Contracted garage length (runner's frame)
    \item Time difference between door closings (runner's frame)
\end{itemize}

\textbf{Key concept:} Length contraction and relativity of simultaneity work together

\newpage

\section{Grading Breakdown}

\subsection{Written Dialogue (60 points)}
\begin{itemize}[noitemsep]
    \item Scenario setup and clarity (5 points)
    \item Accurate calculations with algebraic work shown before numbers (20 points)
    \item Both characters' perspectives presented convincingly (15 points)
    \item Paradox clearly stated (5 points)
    \item Resolution using special relativity (10 points)
    \item Real-world connection included (5 points)
\end{itemize}

\subsection{Video Presentation (40 points)}
\begin{itemize}[noitemsep]
    \item Length: 5--7 minutes (5 points)
    \item Technical quality: clear video and audio, correct format (5 points)
    \item Calculations shown clearly on camera with algebra first (10 points)
    \item Original visual demonstration (10 points)
    \item Dialogue performance (5 points)
    \item Real-world application explained (5 points)
\end{itemize}

\section{Timeline}

\textbf{Week 1:}
\begin{itemize}[noitemsep]
    \item Monday: Topics assigned, groups formed
    \item Wednesday: Group work time -- outline dialogues
    \item Friday: Draft calculations due for review
\end{itemize}

\textbf{Week 2:}
\begin{itemize}[noitemsep]
    \item Monday: Written dialogues due
    \item Wednesday: Video filming work time
    \item Friday: Peer review (optional -- watch another group's rough cut)
\end{itemize}

\textbf{Week 3:}
\begin{itemize}[noitemsep]
    \item Monday: Final videos due
    \item Wednesday--Friday: Video presentations in class (2 per day)
\end{itemize}

\section{Submission Requirements}

\textbf{Written Dialogue:}
\begin{itemize}[noitemsep]
    \item Submit to Google Classroom as \textbf{PDF only}
    \item File name: ``Period\_TopicNumber\_GroupNames.pdf''
    \item Example: ``Period3\_Topic1\_Smith\_Jones\_Lee\_Kim\_Chen.pdf''
\end{itemize}

\textbf{Video:}
\begin{itemize}[noitemsep]
    \item \textbf{MP4 or MOV format only}
    \item \textbf{720p minimum resolution, 1080p maximum resolution}
    \item Submit video file directly to Google Classroom (or USB drive if file is too large)
    \item Include in submission: list of who did what in the project
\end{itemize}

\newpage

\section{Tips for Success}

\subsection{For the Dialogue:}
\begin{itemize}[noitemsep]
    \item Write like people actually talk -- make it conversational
    \item Have both characters draw diagrams or write calculations
    \item Show confusion before understanding
    \item Make sure each character's logic makes sense from their perspective
    \item Read your dialogue out loud to check if it sounds natural
    \item \textbf{Show all algebraic steps before plugging in numbers}
\end{itemize}

\subsection{For the Video:}
\begin{itemize}[noitemsep]
    \item Test your recording setup before you film everything
    \item Write big when showing calculations -- will it be readable on a phone screen?
    \item Practice your dialogue performance at least once
    \item Don't rush through the math -- explain each step
    \item \textbf{Derive the formula algebraically, THEN substitute values}
    \item Use props! A meter stick can be a ``ladder,'' toy cars can be ``spaceships''
    \item Edit out long pauses or mistakes (free tools: iMovie, OpenShot, CapCut)
\end{itemize}

\subsection{For Calculations:}
\begin{itemize}[noitemsep]
    \item \textbf{Always show algebraic manipulation first}
    \item \textbf{Only plug in numbers at the very end}
    \item Show your work step by step
    \item Always include units
    \item Check if your answer makes sense (is it bigger than the speed of light? That's wrong!)
    \item Verify: Does the classical limit work? (If $v$ is very small, do you get the classical answer?)
    \item Use enough decimal places -- relativistic effects can be small!
\end{itemize}

\section{Frequently Asked Questions}

\textbf{Q: Can we use special effects or animations?}

A: Yes! As long as you also show original calculations and diagrams. Using animations to help explain is great.

\vspace{0.3cm}

\textbf{Q: Do all 5 group members need to be on camera?}

A: Yes, everyone must appear at least briefly. You can divide responsibilities (one person does calculations, another does dialogue, another creates diagrams, etc.)

\vspace{0.3cm}

\textbf{Q: What if we can't all meet to film together?}

A: You can film segments separately and edit them together. But the video should look cohesive.

\vspace{0.3cm}

\textbf{Q: Can we add music or sound effects?}

A: Yes, but keep it appropriate and make sure we can still hear your explanations clearly.

\vspace{0.3cm}

\textbf{Q: What if our dialogue is shorter than 4 pages?}

A: Make sure you've included enough detail in the calculations and character discussion. If it's comprehensive and includes everything required, 3.5 pages might be okay -- check with me.

\vspace{0.3cm}

\textbf{Q: Can we make it funny?}

A: Absolutely! Humor is great as long as the physics is correct.

\vspace{0.3cm}

\textbf{Q: What should we do if our calculations don't match the numbers in examples online?}

A: Use the formulas and numbers given in this assignment. Different sources might use slightly different scenarios or round differently.

\vspace{0.3cm}

\textbf{Q: Why do we need to show algebra before plugging in numbers?}

A: This demonstrates you understand the physics and the relationships between variables. It also makes it much easier to check your work and catch errors. This is standard practice in all physics.

\newpage

\section{Academic Integrity}

\begin{itemize}[noitemsep]
    \item All dialogue must be your group's original writing
    \item All calculations must be your group's original work
    \item \textbf{You may NOT use AI tools to help with calculations or math}
    \item You may use online resources to understand concepts, but cite any specific information taken from outside sources
    \item All video content must be created by your group
\end{itemize}

\subsection{Prohibited Use of AI:}
\begin{itemize}[noitemsep]
    \item Using AI to check your math
    \item Using AI to solve problems
    \item Using AI to write your dialogue
    \item Using AI to generate explanations
\end{itemize}

\textbf{Using AI for any part of this assignment constitutes academic dishonesty and will result in a zero for the assignment.}

\vspace{1cm}

\begin{center}
\textbf{Good luck! This project will help you deeply understand some of the most mind-bending ideas in physics. Have fun with it!}
\end{center}

\newpage

\begin{center}
\section*{APPENDIX: Calculation Reference}
\end{center}

\subsection*{Universal Formulas}

All topics will use these fundamental formulas from special relativity:

\subsubsection*{Lorentz Factor:}
\begin{equation}
\gamma = \frac{1}{\sqrt{1-\frac{v^2}{c^2}}}
\end{equation}

\subsubsection*{Time Dilation:}
\begin{equation}
\Delta t = \gamma \Delta t_0
\end{equation}
where:
\begin{itemize}[noitemsep]
    \item $\Delta t_0$ = proper time (time measured in rest frame of the event)
    \item $\Delta t$ = time measured by observer moving relative to the event
\end{itemize}

\subsubsection*{Length Contraction:}
\begin{equation}
L = \frac{L_0}{\gamma}
\end{equation}
where:
\begin{itemize}[noitemsep]
    \item $L_0$ = proper length (length measured in rest frame of object)
    \item $L$ = length measured by observer moving relative to object
\end{itemize}

\subsubsection*{Relativistic Velocity Addition:}
\begin{equation}
u = \frac{v + u'}{1 + \frac{vu'}{c^2}}
\end{equation}
where:
\begin{itemize}[noitemsep]
    \item $v$ = velocity of object A relative to observer
    \item $u'$ = velocity of object B relative to object A
    \item $u$ = velocity of object B relative to observer
\end{itemize}

\subsection*{Constants}
\begin{itemize}[noitemsep]
    \item Speed of light: $c = 3.00 \times 10^8$ m/s
    \item 1 light-year (ly) = $9.46 \times 10^{15}$ m
    \item Earth's orbital velocity $\approx 30$ km/s = $3.0 \times 10^4$ m/s
\end{itemize}

\newpage

\subsection*{Topic 1: Twin Paradox -- Algebraic Steps}

\textbf{Given values:}
\begin{itemize}[noitemsep]
    \item Distance to Alpha Centauri: $d = 4.3$ ly
    \item Spaceship velocity: $v = 0.90c$
\end{itemize}

\subsubsection*{Step 1: Calculate Lorentz factor}
\begin{align}
\gamma &= \frac{1}{\sqrt{1-\frac{v^2}{c^2}}} \\
&= \frac{1}{\sqrt{1-\frac{(0.90c)^2}{c^2}}} \\
&= \frac{1}{\sqrt{1-0.81}} \\
&= \frac{1}{\sqrt{0.19}}
\end{align}

\textit{Now calculate the numerical value.}

\subsubsection*{Step 2: Earth time for one-way journey}
\begin{equation}
\Delta t = \frac{d}{v}
\end{equation}

\textit{Substitute values and calculate. Note: if distance is in light-years and velocity is in terms of $c$, time comes out in years.}

\subsubsection*{Step 3: Total Earth time (round trip)}
\begin{equation}
\Delta t_{\text{total}} = 2\Delta t
\end{equation}

\subsubsection*{Step 4: Twin B's proper time (one way)}
\begin{equation}
\Delta t_0 = \frac{\Delta t}{\gamma}
\end{equation}

\subsubsection*{Step 5: Twin B's total proper time}
\begin{equation}
\Delta t_{0,\text{total}} = 2\Delta t_0
\end{equation}

\subsubsection*{Step 6: Age difference}
\begin{equation}
\Delta(\text{age}) = \Delta t_{\text{total}} - \Delta t_{0,\text{total}}
\end{equation}

\newpage

\subsection*{Topic 2: Michelson-Morley -- Algebraic Steps}

\textbf{Given values:}
\begin{itemize}[noitemsep]
    \item Earth's orbital velocity: $v = 30$ km/s = $3.0 \times 10^4$ m/s
    \item Speed of light: $c = 3.0 \times 10^8$ m/s
    \item Interferometer arm length: $L = 11$ m
\end{itemize}

\subsubsection*{Step 1: Time for light parallel to motion (forward)}
\begin{equation}
t_{\parallel,\text{forward}} = \frac{L}{c-v}
\end{equation}

\subsubsection*{Step 2: Time for light parallel to motion (backward)}
\begin{equation}
t_{\parallel,\text{backward}} = \frac{L}{c+v}
\end{equation}

\subsubsection*{Step 3: Total time parallel}
\begin{align}
t_{\parallel} &= t_{\parallel,\text{forward}} + t_{\parallel,\text{backward}} \\
&= \frac{L}{c-v} + \frac{L}{c+v} \\
&= L\left(\frac{1}{c-v} + \frac{1}{c+v}\right) \\
&= L\left(\frac{c+v+c-v}{(c-v)(c+v)}\right) \\
&= L\left(\frac{2c}{c^2-v^2}\right) \\
&= \frac{2Lc}{c^2-v^2}
\end{align}

\subsubsection*{Step 4: Time for light perpendicular to motion}

Light must travel diagonally. Using Pythagorean theorem:
\begin{equation}
(ct_{\perp}/2)^2 = L^2 + (vt_{\perp}/2)^2
\end{equation}

Solving for $t_{\perp}$:
\begin{align}
c^2(t_{\perp}/2)^2 &= L^2 + v^2(t_{\perp}/2)^2 \\
(t_{\perp}/2)^2(c^2 - v^2) &= L^2 \\
t_{\perp}/2 &= \frac{L}{\sqrt{c^2-v^2}} \\
t_{\perp} &= \frac{2L}{\sqrt{c^2-v^2}}
\end{align}

\subsubsection*{Step 5: Time difference}
\begin{align}
\Delta t &= t_{\parallel} - t_{\perp} \\
&= \frac{2Lc}{c^2-v^2} - \frac{2L}{\sqrt{c^2-v^2}}
\end{align}

For $v \ll c$, this simplifies to:
\begin{equation}
\Delta t \approx \frac{Lv^2}{c^3}
\end{equation}

\textit{Now substitute values and calculate.}

\subsubsection*{Step 6: With constant $c$}

If $c$ is constant in all directions:
\begin{equation}
t_{\parallel} = t_{\perp} = \frac{2L}{c}
\end{equation}

Therefore $\Delta t = 0$ (no fringe shift).

\newpage

\subsection*{Topic 3: Relativity of Simultaneity -- Algebraic Steps}

\textbf{Given values:}
\begin{itemize}[noitemsep]
    \item Train velocity: $v = 0.60c$
    \item Train proper length: $L_0 = 30$ m
\end{itemize}

\subsubsection*{Step 1: Calculate Lorentz factor}
\begin{align}
\gamma &= \frac{1}{\sqrt{1-\frac{v^2}{c^2}}} \\
&= \frac{1}{\sqrt{1-\frac{(0.60c)^2}{c^2}}} \\
&= \frac{1}{\sqrt{1-0.36}} \\
&= \frac{1}{\sqrt{0.64}}
\end{align}

\textit{Calculate the numerical value.}

\subsubsection*{Step 2: Contracted train length (platform frame)}
\begin{equation}
L = \frac{L_0}{\gamma}
\end{equation}

\textit{Substitute your $\gamma$ value and calculate.}

\subsubsection*{Step 3: Time for light to reach Observer A}

Observer A is at the midpoint, so light travels $L/2$ from each strike:
\begin{equation}
t_A = \frac{L/2}{c}
\end{equation}

\subsubsection*{Step 4: Time difference between strikes (Observer B's frame)}

This is given by the relativity of simultaneity formula:
\begin{equation}
\Delta t' = \frac{v L_0}{c^2}
\end{equation}

\textit{Substitute all known values and calculate.}

This shows that in Observer B's frame, the front lightning strike happened $\Delta t'$ earlier than the rear strike.

\newpage

\subsection*{Topic 4: Cosmic Ray Muons -- Algebraic Steps}

\textbf{Given values:}
\begin{itemize}[noitemsep]
    \item Muon velocity: $v = 0.998c$
    \item Muon rest half-life: $\tau_0 = 2.2 \times 10^{-6}$ s = $2.2$ $\mu$s
    \item Creation altitude: $h_0 = 15$ km = $15{,}000$ m
\end{itemize}

\subsubsection*{Step 1: Calculate Lorentz factor}
\begin{align}
\gamma &= \frac{1}{\sqrt{1-\frac{v^2}{c^2}}} \\
&= \frac{1}{\sqrt{1-\frac{(0.998c)^2}{c^2}}} \\
&= \frac{1}{\sqrt{1-0.996004}} \\
&= \frac{1}{\sqrt{0.003996}}
\end{align}

\textit{Calculate the numerical value.}

\subsubsection*{Step 2: Classical distance before decay}
\begin{equation}
d_{\text{classical}} = v\tau_0
\end{equation}

\textit{Substitute values: $v = 0.998c = 0.998(3.0 \times 10^8 \text{ m/s})$ and $\tau_0 = 2.2 \times 10^{-6}$ s, then calculate.}

\subsubsection*{Step 3: Dilated half-life (Earth observer's frame)}
\begin{equation}
\tau = \gamma \tau_0
\end{equation}

\subsubsection*{Step 4: Distance muon travels (Earth frame)}
\begin{equation}
d = v\tau = v(\gamma \tau_0)
\end{equation}

\subsubsection*{Step 5: Contracted atmosphere height (muon's frame)}
\begin{equation}
h = \frac{h_0}{\gamma}
\end{equation}

\subsubsection*{Step 6: Time to reach ground (muon's frame)}
\begin{equation}
t = \frac{h}{v} = \frac{h_0}{\gamma v}
\end{equation}

\textit{Note: Both perspectives should give consistent results about whether muons reach the ground!}

\newpage

\subsection*{Topic 5: Relativistic Velocity Addition -- Algebraic Steps}

\textbf{Given values:}
\begin{itemize}[noitemsep]
    \item Spaceship velocity: $v = 0.90c$
    \item Missile velocity (in ship frame): $u' = 0.90c$
\end{itemize}

\subsubsection*{Step 1: Classical addition (for comparison)}
\begin{equation}
u_{\text{classical}} = v + u'
\end{equation}

\textit{This gives the wrong answer! Calculate it anyway to show the problem.}

\subsubsection*{Step 2: Relativistic velocity addition}
\begin{equation}
u = \frac{v + u'}{1 + \frac{vu'}{c^2}}
\end{equation}

Substituting:
\begin{align}
u &= \frac{0.90c + 0.90c}{1 + \frac{(0.90c)(0.90c)}{c^2}} \\
&= \frac{1.80c}{1 + \frac{0.81c^2}{c^2}} \\
&= \frac{1.80c}{1 + 0.81} \\
&= \frac{1.80c}{1.81}
\end{align}

\textit{Now calculate the numerical value. Note it's less than $c$!}

\subsubsection*{Step 3: Extreme case}

Try $v = 0.99c$ and $u' = 0.99c$:
\begin{align}
u &= \frac{0.99c + 0.99c}{1 + \frac{(0.99c)(0.99c)}{c^2}} \\
&= \frac{1.98c}{1 + 0.9801} \\
&= \frac{1.98c}{1.9801}
\end{align}

\textit{Calculate this value. Still less than $c$!}

\subsubsection*{Step 4: Special case -- adding to light speed}

Let $u' = c$:
\begin{align}
u &= \frac{v + c}{1 + \frac{v \cdot c}{c^2}} \\
&= \frac{v + c}{1 + \frac{v}{c}} \\
&= \frac{v + c}{\frac{c + v}{c}} \\
&= \frac{(v + c) \cdot c}{c + v} \\
&= c
\end{align}

This proves that adding any velocity to $c$ gives $c$!

\subsubsection*{Step 5: Lorentz factors (optional, for energy comparison)}

For $v = 0.90c$:
\begin{equation}
\gamma_v = \frac{1}{\sqrt{1-(0.90)^2}} = \frac{1}{\sqrt{0.19}}
\end{equation}

For $u = 0.9945c$ (from Step 2):
\begin{equation}
\gamma_u = \frac{1}{\sqrt{1-(0.9945)^2}}
\end{equation}

\textit{Calculate both and compare. Notice how much larger $\gamma_u$ is!}

\newpage

\subsection*{Topic 6: Ladder Paradox -- Algebraic Steps}

\textbf{Given values:}
\begin{itemize}[noitemsep]
    \item Ladder proper length: $L_0 = 10$ m
    \item Garage proper length: $D_0 = 5$ m
    \item Relative velocity: $v = 0.866c$
\end{itemize}

\subsubsection*{Step 1: Calculate Lorentz factor}
\begin{align}
\gamma &= \frac{1}{\sqrt{1-\frac{v^2}{c^2}}} \\
&= \frac{1}{\sqrt{1-\frac{(0.866c)^2}{c^2}}} \\
&= \frac{1}{\sqrt{1-0.749956}} \\
&= \frac{1}{\sqrt{0.250044}}
\end{align}

\textit{This should give $\gamma = 2.0$ (the velocity was chosen for this!)}

\subsubsection*{Step 2: Contracted ladder length (garage frame)}
\begin{equation}
L = \frac{L_0}{\gamma}
\end{equation}

\textit{The ladder contracts to exactly 5 m -- it fits!}

\subsubsection*{Step 3: Contracted garage length (ladder frame)}
\begin{equation}
D = \frac{D_0}{\gamma}
\end{equation}

\textit{The garage contracts to 2.5 m -- much shorter than the ladder!}

\subsubsection*{Step 4: Time difference between door closings (ladder frame)}

Using the relativity of simultaneity formula:
\begin{equation}
\Delta t' = \frac{v L_0}{c^2}
\end{equation}

Substituting:
\begin{equation}
\Delta t' = \frac{(0.866c)(10 \text{ m})}{c^2} = \frac{8.66c \cdot \text{m}}{c^2} = \frac{8.66 \text{ m}}{c}
\end{equation}

Now substitute $c = 3.0 \times 10^8$ m/s:
\begin{equation}
\Delta t' = \frac{8.66 \text{ m}}{3.0 \times 10^8 \text{ m/s}}
\end{equation}

\textit{Calculate this time in seconds (or nanoseconds).}

\vspace{0.5cm}

This time difference is the key! In the garage frame, both doors close simultaneously. In the ladder frame, the front door closes first, then opens, then the ladder passes through, then the back door closes. Both observers agree on the physical outcome: the ladder passes through undamaged.

\end{document}