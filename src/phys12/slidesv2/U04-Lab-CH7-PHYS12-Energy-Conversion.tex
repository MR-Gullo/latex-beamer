% U04-Lab-CH7-PHYS12-Energy-Conversion.tex
% Consolidated from Energy Conversion and Friction Lab + Student Template
\PassOptionsToPackage{unicode}{hyperref}
\PassOptionsToPackage{hyphens}{url}
\documentclass[]{article}
\usepackage{xcolor}
\usepackage[margin=1in]{geometry}
\usepackage{amsmath,amssymb}
\usepackage{graphicx}
\graphicspath{{docx-images/}{../images/}{../../shared/images/}}
\setcounter{secnumdepth}{-\maxdimen}
\usepackage{iftex}
\ifPDFTeX
  \usepackage[T1]{fontenc}
  \usepackage[utf8]{inputenc}
  \usepackage{textcomp}
\else
  \usepackage{unicode-math}
  \defaultfontfeatures{Scale=MatchLowercase}
  \defaultfontfeatures[\rmfamily]{Ligatures=TeX,Scale=1}
\fi
\usepackage{lmodern}
\IfFileExists{upquote.sty}{\usepackage{upquote}}{}
\IfFileExists{microtype.sty}{%
  \usepackage[]{microtype}
  \UseMicrotypeSet[protrusion]{basicmath}
}{}
\makeatletter
\@ifundefined{KOMAClassName}{%
  \IfFileExists{parskip.sty}{%
    \usepackage{parskip}
  }{%
    \setlength{\parindent}{0pt}
    \setlength{\parskip}{6pt plus 2pt minus 1pt}}
}{\KOMAoptions{parskip=half}}
\makeatother
\usepackage{longtable,booktabs,array}
\newcounter{none}
\usepackage{calc}
\usepackage{etoolbox}
\makeatletter
\patchcmd\longtable{\par}{\if@noskipsec\mbox{}\fi\par}{}{}
\makeatother
\IfFileExists{footnotehyper.sty}{\usepackage{footnotehyper}}{\usepackage{footnote}}
\makesavenoteenv{longtable}
\makeatletter
\newsavebox\pandoc@box
\newcommand*\pandocbounded[1]{%
  \sbox\pandoc@box{#1}%
  \Gscale@div\@tempa{\textheight}{\dimexpr\ht\pandoc@box+\dp\pandoc@box\relax}%
  \Gscale@div\@tempb{\linewidth}{\wd\pandoc@box}%
  \ifdim\@tempb\p@<\@tempa\p@\let\@tempa\@tempb\fi
  \ifdim\@tempa\p@<\p@\scalebox{\@tempa}{\usebox\pandoc@box}%
  \else\usebox{\pandoc@box}%
  \fi%
}
\def\fps@figure{htbp}
\makeatother
\setlength{\emergencystretch}{3em}
\providecommand{\tightlist}{%
  \setlength{\itemsep}{0pt}\setlength{\parskip}{0pt}}
\usepackage{bookmark}
\IfFileExists{xurl.sty}{\usepackage{xurl}}{}
\urlstyle{same}
\hypersetup{hidelinks,pdfcreator={LaTeX via pandoc}}

\title{Energy Conversion and Friction\\A Two-Part Investigation Using Bouncy Balls}
\author{Physics 12}
\date{}

\begin{document}
\maketitle

In this two-part activity, you will first study gravitational potential
energy converting to kinetic energy, then investigate how friction
converts kinetic energy to thermal energy.

\section{PART 1: Converting Potential to Kinetic Energy}

\subsection{Materials Needed}
\begin{itemize}
\item One ruler with groove OR open book with groove
\item One book for incline
\item One rubber bouncy ball
\item One steel ball bearing
\item One timer
\item One smooth, level surface
\item Meter stick or measuring tape
\end{itemize}

\subsection{Procedure}
\begin{enumerate}
\item Set up an incline using the ruler/book groove and book on a smooth, level surface.
\item Place a rubber bouncy ball at the 10-cm position and let it roll down.
\item When the ball hits the level surface, measure the time it takes to roll one meter.
\item Repeat with the ball at the 20-cm and 30-cm positions.
\item Calculate the velocity of the ball for all three positions.
\item Plot velocity squared versus the distance traveled by the ball.
\item Describe the shape of the plot. If linear, it shows the ball's kinetic energy at the bottom is proportional to its potential energy at the release point.
\end{enumerate}

\begin{center}
\includegraphics[width=0.85\textwidth]{energy-conv-image1.png}
\end{center}

\section{PART 2: Determining Friction from the Stopping Distance}

\subsection{Materials Needed}
\begin{itemize}
\item Same incline setup from Part 1
\item Rubber bouncy ball from Part 1
\item Steel ball bearing from Part 1
\item Foam cup with a small hole in the side
\item Meter stick or measuring tape
\item Scale (to measure masses of balls and cup)
\end{itemize}

\subsection{Procedure}
\begin{enumerate}
\item Position the foam cup at the bottom of the incline so the ball will roll into it through the small hole.
\item Place the rubber bouncy ball at the 10-cm position and let it roll down into the cup.
\item Measure the distance the cup moves before stopping.
\item Repeat at the 20-cm and 30-cm positions.
\item Plot stopping distance versus initial ball position. Is the relationship linear?
\item Repeat steps 2-5 using the steel ball bearing.
\item Using the masses of the balls and cup, calculate the coefficient of kinetic friction of the cup on the table.
\end{enumerate}

\begin{center}
\includegraphics[width=0.85\textwidth]{energy-conv-image2.png}
\end{center}

\textbf{Note:} The force of friction on the cup is $f_k = \mu_k N$, where the
normal force is the weight of the cup plus the ball. The work done by
friction is $W = f_k \times d$.

\newpage
%% ===== STUDENT SUBMISSION TEMPLATE =====

\section*{Student Submission Template}

Name: \rule{5cm}{0.4pt} \hfill Partner(s): \rule{5cm}{0.4pt}

Date: \rule{5cm}{0.4pt} \hfill Class/Period: \rule{3cm}{0.4pt}

\subsection*{Pre-Lab Checklist (Complete before starting)}
\begin{itemize}
\item[$\square$] Ruler with groove OR open book with groove obtained
\item[$\square$] Book for incline selected
\item[$\square$] Rubber bouncy ball and steel ball bearing obtained
\item[$\square$] Timer available
\item[$\square$] Foam cup with small hole prepared
\item[$\square$] Scale to measure masses obtained
\item[$\square$] Smooth, level surface identified
\item[$\square$] Meter stick or measuring tape available
\item[$\square$] Camera/phone available for photos
\item[$\square$] Graph paper or digital graphing tool ready
\end{itemize}

\subsection*{Mass Measurements (Required for Part 2)}
Mass of rubber bouncy ball: \rule{3cm}{0.4pt} kg\\
Mass of steel ball bearing: \rule{3cm}{0.4pt} kg\\
Mass of foam cup: \rule{3cm}{0.4pt} kg

\subsection*{Part 1 Data: Rubber Bouncy Ball}

\begin{center}
\begin{tabular}{|c|c|c|c|c|}
\hline
\textbf{Release Position} & \textbf{Trial} & \textbf{Time for 1 m (s)} & \textbf{Velocity (m/s)} & \textbf{Velocity$^2$ (m$^2$/s$^2$)} \\
\hline
10 cm (0.10 m) & 1 & & & \\
& 2 & & & \\
& 3 & & & \\
& Avg & & & \\
\hline
20 cm (0.20 m) & 1 & & & \\
& 2 & & & \\
& 3 & & & \\
& Avg & & & \\
\hline
30 cm (0.30 m) & 1 & & & \\
& 2 & & & \\
& 3 & & & \\
& Avg & & & \\
\hline
\end{tabular}
\end{center}

\subsection*{Sample Calculation: Part 1}
Show your complete work for calculating velocity and velocity squared for one trial:\\[2ex]
Given: Time to roll 1.0 m = \rule{2cm}{0.4pt} s \quad Release position = \rule{2cm}{0.4pt} m\\[1ex]
Velocity ($v$) = distance/time = 1.0 m / \rule{2cm}{0.4pt} s = \rule{2cm}{0.4pt} m/s\\[1ex]
Velocity squared ($v^2$) = (\rule{2cm}{0.4pt} m/s)$^2$ = \rule{2cm}{0.4pt} m$^2$/s$^2$

\subsection*{Part 2 Data: Cup Stopping Distance}

\textbf{Rubber Bouncy Ball:}
\begin{center}
\begin{tabular}{|c|c|c|c|}
\hline
\textbf{Release Position} & \textbf{Trial} & \textbf{Stopping Distance (m)} & \textbf{Avg Distance (m)} \\
\hline
10 cm & 1, 2, 3 & & \\
\hline
20 cm & 1, 2, 3 & & \\
\hline
30 cm & 1, 2, 3 & & \\
\hline
\end{tabular}
\end{center}

\textbf{Steel Ball Bearing:}
\begin{center}
\begin{tabular}{|c|c|c|c|}
\hline
\textbf{Release Position} & \textbf{Trial} & \textbf{Stopping Distance (m)} & \textbf{Avg Distance (m)} \\
\hline
10 cm & 1, 2, 3 & & \\
\hline
20 cm & 1, 2, 3 & & \\
\hline
30 cm & 1, 2, 3 & & \\
\hline
\end{tabular}
\end{center}

\subsection*{Sample Calculation: Part 2 (Coefficient of Friction)}
Release position = \rule{2cm}{0.4pt} m \quad Average stopping distance ($d$) = \rule{2cm}{0.4pt} m\\
Mass of ball = \rule{2cm}{0.4pt} kg \quad Mass of cup = \rule{2cm}{0.4pt} kg \quad $g$ = 9.8 m/s$^2$\\[1ex]
Normal force ($N$) = (mass of cup + mass of ball) $\times g$ = \rule{3cm}{0.4pt} N\\
Work done by friction = $\Delta KE$ = \rule{3cm}{0.4pt} J\\
Coefficient of friction $\mu_k$ = Work / ($N \times d$) = \rule{3cm}{0.4pt}

\subsection*{Analysis Questions}
\begin{enumerate}
\item What is the shape of your velocity squared versus distance plot? What does this indicate about the relationship between potential energy at release and kinetic energy at the bottom?
\vspace{2cm}
\item Is the relationship between stopping distance and initial ball position linear? Explain what this means for the energy conversion process.
\vspace{2cm}
\item For the steel ball bearing: Is the velocity at the bottom the same as the rubber bouncy ball? Is the stopping distance proportional to the ball mass? Support with data.
\vspace{2cm}
\item What factors might affect the accuracy of your friction coefficient calculation?
\vspace{2cm}
\end{enumerate}

\textbf{Submission:} Submit this completed template along with both graphs and all photos as a single document (FirstName\_LastName.pdf).

\end{document}
