\documentclass{beamer}
% Use DS9 global theme (includes pgfplots for visualization)
\usepackage{../../../shared/templates/ds9_theme}
\usepackage[overridenote]{pdfpc}

% Title page configuration
\title[Relativity Part 2]{PHYS12 CH:28.4-28.6}
\subtitle{Relativistic Mechanics}
\author[Mr. Gullo]{Mr. Gullo}
\date[Nov 21 2025]{November 21, 2025}

\begin{document}
\frame{\titlepage
\note{- Part 2: relativistic mechanics\\\\
- Build on Part 1 concepts (gamma, time dilation, length contraction)\\\\
- Today: velocities, momentum, energy\\\\
- Famous equation E=mc2 derived here}
}

\begin{frame}
\frametitle{Learning Objectives}
By the end of this lesson, you will be able to:
\pause
\begin{itemize}
    \item Apply the relativistic velocity addition formula.
    \pause
    \item Calculate relativistic momentum and compare it to classical momentum.
    \pause
    \item Define rest energy, total energy, and kinetic energy in relativistic terms.
    \pause
    \item Solve problems involving mass-energy equivalence ($E=mc^2$).
\end{itemize}
\note{- Four main objectives today\\\\
- Velocity addition: why you cant just add speeds\\\\
- Relativistic momentum: why p=mv fails at high speed\\\\
- Energy: rest energy, total energy, kinetic energy\\\\
- E=mc2: mass IS energy}
\end{frame}

\begin{frame}
\frametitle{Progression and Recap}
\begin{alertblock}{From Part 1}
\begin{itemize}
    \item $\gamma = \frac{1}{\sqrt{1 - v^2/c^2}}$
    \pause
    \item Time dilates ($\Delta t = \gamma \Delta t_0$)
    \pause
    \item Length contracts ($L = L_0 / \gamma$)
\end{itemize}
\end{alertblock}
\pause
\begin{block}{Today: Dynamics}
\begin{itemize}
    \item What happens when we push objects near the speed of light?
    \pause
    \item Does $F=ma$ still work? (Spoiler: Not simply)
    \pause
    \item How do we add velocities correctly?
\end{itemize}
\end{block}
\note{- Quick recap from Part 1\\\\
- gamma: Lorentz factor, always >= 1\\\\
- Time dilation: moving clocks run slow\\\\
- Length contraction: moving objects shorter\\\\
- Today: what about forces, momentum, energy?\\\\
- F=ma breaks down - need modifications}
\end{frame}

\section{28.4 Relativistic Addition of Velocities}

\begin{frame}
\frametitle{28.4 The Limit of Galilean Relativity}
Classically, if you run at $u'$ on a train moving at $v$, your ground speed is simply:
$$ u = v + u' $$
\pause
\begin{alertblock}{Problem}
If a spaceship moves at $0.8c$ and fires a probe forward at $0.8c$, classical physics says the probe moves at $1.6c$.
\pause
\textbf{This violates the second postulate ($c$ is the limit).}
\end{alertblock}
\note{- Galilean velocity addition: just add speeds\\\\
- Works fine for everyday speeds\\\\
- Problem: 0.8c + 0.8c = 1.6c - faster than light!\\\\
- Violates postulate 2: nothing exceeds c\\\\
- Need new formula that respects speed limit}
\end{frame}

\begin{frame}
\frametitle{Essential Equation: Velocity Addition}
\begin{columns}
\column{0.5\textwidth}
\begin{block}{Relativistic Velocity Addition}
$$ u = \frac{v + u'}{1 + \frac{vu'}{c^2}} $$
\end{block}

\column{0.5\textwidth}
\pause
\begin{itemize}
    \item $u$: Velocity relative to stationary frame.
    \pause
    \item $v$: Velocity of the moving frame (e.g., the ship).
    \pause
    \item $u'$: Velocity relative to the moving frame (e.g., the probe).
\end{itemize}
\end{columns}
\pause
\vspace{1em}
Notice: If $v \ll c$ and $u' \ll c$, the denominator $\approx 1$, giving back $u \approx v + u'$.
\note{- Key formula: denominator is the correction factor\\\\
- u = combined velocity from stationary frame\\\\
- v = frame velocity (ship)\\\\
- u' = velocity within moving frame (probe on ship)\\\\
- Denominator near 1 for slow speeds - reduces to Galilean\\\\
- But prevents result from ever exceeding c}
\end{frame}

\begin{frame}
\frametitle{Example: I Do - Velocity Addition}
\textbf{Problem}: A spaceship travels away from Earth at $v = 0.5c$. It fires a missile forward at speed $u' = 0.5c$ relative to the ship. What is the missile's speed relative to Earth?
\note{- Classic example: adding two 0.5c velocities\\\\
- Classically: 0.5c + 0.5c = 1.0c\\\\
- Relativistically: less than c\\\\
- Work through GUESS method}
\end{frame}

% GUESS Frame 1: G and U
\begin{frame}
\frametitle{I Do: Velocity Addition - G \& U}
\begin{columns}[T]
\column{0.48\textwidth}
\textbf{G - Givens}
\begin{itemize}
    \item $v = 0.5c$ (Ship relative to Earth)
    \pause
    \item $u' = 0.5c$ (Missile relative to Ship)
\end{itemize}

\column{0.48\textwidth}
\pause
\textbf{U - Unknown}
\begin{itemize}
    \item $u = ?$ (Missile relative to Earth)
\end{itemize}
\end{columns}
\note{- v = 0.5c: ship speed from Earth frame\\\\
- u' = 0.5c: missile speed from ship frame\\\\
- Find u: missile speed from Earth frame\\\\
- Key: identify which velocity is which}
\end{frame}

% GUESS Frame 2: E
\begin{frame}
\frametitle{I Do: Velocity Addition - Equation}
\textbf{E - Equation}
\pause
\begin{itemize}
    \item Use the relativistic addition formula:
    $$ u = \frac{v + u'}{1 + \frac{vu'}{c^2}} $$
\end{itemize}
\note{- Only one equation needed\\\\
- Numerator: sum of velocities (like Galilean)\\\\
- Denominator: correction factor that prevents exceeding c\\\\
- c2 in denominator makes it dimensionless}
\end{frame}

% GUESS Frame 3: S and S
\begin{frame}
\frametitle{I Do: Velocity Addition - Substitute \& Solve}
\textbf{S - Substitute}
\begin{itemize}
    \item $u = \frac{0.5c + 0.5c}{1 + \frac{(0.5c)(0.5c)}{c^2}}$
    \pause
    \item $u = \frac{1.0c}{1 + 0.25}$
\end{itemize}
\pause
\textbf{S - Solve}
\begin{itemize}
    \item $u = \frac{1.0c}{1.25} = 0.8c$
    \pause
    \item \boxed{u = 0.8c}
    \pause
    \item \textit{Result is less than $c$, as required.}
\end{itemize}
\note{- Numerator: 0.5c + 0.5c = 1.0c\\\\
- Denominator: 1 + (0.5)(0.5) = 1.25\\\\
- Result: 1.0c / 1.25 = 0.8c\\\\
- NOT 1.0c as classical physics predicts\\\\
- Formula automatically keeps result below c\\\\
- Even adding c + c gives c, not 2c!}
\end{frame}

\section{28.5 Relativistic Momentum}

\begin{frame}
\frametitle{28.5 Relativistic Momentum}
Newtonian momentum $p = mv$ is not conserved at high speeds. We must adjust the definition.
\pause
\begin{block}{Relativistic Momentum}
$$ p = \gamma mv = \frac{mv}{\sqrt{1 - \frac{v^2}{c^2}}} $$
\end{block}
\pause
\begin{itemize}
    \item As $v \to c$, $\gamma \to \infty$, so $p \to \infty$.
    \pause
    \item This explains why a massive object cannot reach $c$: it would require infinite momentum (and infinite energy).
\end{itemize}
\note{- Classical p=mv fails at high speed\\\\
- Fix: multiply by gamma\\\\
- p = gamma * m * v\\\\
- As v approaches c, gamma approaches infinity\\\\
- So momentum approaches infinity\\\\
- Infinite force needed to reach c - impossible for massive objects}
\end{frame}

\begin{frame}
\frametitle{Concept Visualization: Momentum Limit}
\begin{alertblock}{}
\begin{center}
		\includegraphics[width=0.5\linewidth]{pasted-images/ch28_slides_relativity_Part2-10-29-52.png}
	\end{center}
\pause
\begin{itemize}
    \item \textbf{Classical line}: Straight line ($p=mv$).
    \pause
    \item \textbf{Relativistic curve}: Follows classical line at low speeds, then curves upward sharply, approaching a vertical asymptote at $v=c$.
\end{itemize}
\end{alertblock}
\note{- Graph shows p vs v\\\\
- Classical: straight line through origin\\\\
- Relativistic: curves up dramatically near c\\\\
- Vertical asymptote at v=c\\\\
- At low speeds: curves overlap (Newton works)\\\\
- Near c: huge difference - momentum explodes}
\end{frame}

\section{28.6 Relativistic Energy}

\begin{frame}
\frametitle{28.6 Rest Energy and Total Energy}
Einstein's most famous equation relates mass and energy.
\pause
\begin{block}{Rest Energy ($E_0$)}
The energy an object has simply because it has mass.
$$ E_0 = mc^2 $$
\end{block}
\pause
\begin{block}{Total Energy ($E$)}
The sum of rest energy and kinetic energy.
$$ E = \gamma mc^2 = \frac{mc^2}{\sqrt{1 - \frac{v^2}{c^2}}} $$
\end{block}
\note{- Most famous equation in physics: E=mc2\\\\
- Rest energy E0: energy from mass alone (at rest)\\\\
- c2 is huge number - small mass = enormous energy\\\\
- Total energy E: rest energy plus kinetic\\\\
- E = gamma * mc2\\\\
- At rest: gamma=1, so E=E0}
\end{frame}

\begin{frame}
\frametitle{Relativistic Kinetic Energy}
Kinetic Energy is the "extra" energy due to motion.
$$ E = E_0 + KE $$
\pause
Rearranging for KE:
$$ KE = E - E_0 = \gamma mc^2 - mc^2 $$
\pause
\begin{block}{Relativistic Kinetic Energy}
$$ KE = (\gamma - 1)mc^2 $$
\end{block}
\pause
\textit{Note: At low speeds, this simplifies to $\frac{1}{2}mv^2$.}
\note{- Total E = rest energy + kinetic energy\\\\
- So KE = E - E0\\\\
- KE = gamma*mc2 - mc2 = (gamma-1)*mc2\\\\
- Factor out mc2\\\\
- At low v: gamma about 1 + v2/2c2\\\\
- So KE about (1/2)mv2 - recovers classical result!}
\end{frame}

\begin{frame}
\frametitle{We Do: Relativistic Energy}
\textbf{Problem}: An electron ($m = 9.11 \times 10^{-31}$ kg) is accelerated to $0.999c$. What is its total energy?
\pause
\vspace{1em}
\textbf{Steps}:
\begin{enumerate}
    \item Calculate $\gamma$ for $v = 0.999c$.
    \pause
    \item Use $E = \gamma mc^2$.
    \pause
    \item (Optional) Convert Joules to eV or MeV.
\end{enumerate}
\note{- v = 0.999c is very relativistic\\\\
- gamma = 1/sqrt(1-0.999^2) = about 22.4\\\\
- E0 = mc2 = 0.511 MeV for electron\\\\
- E = 22.4 * 0.511 MeV = about 11.4 MeV\\\\
- Huge increase from rest energy!\\\\
- Work through calculation together}
\end{frame}

\begin{frame}
\frametitle{You Do: Practice}
\textbf{Problem}: A proton has a rest energy of 938 MeV. It is moving such that its total energy is 2000 MeV.
\pause
\begin{enumerate}
    \item What is its kinetic energy? ($KE = E - E_0$)
    \pause
    \item What is the value of $\gamma$?
    \pause
    \item How fast is it moving? (Solve $\gamma$ for $v$).
\end{enumerate}
\note{- 5 min independent practice\\\\
- KE = 2000 - 938 = 1062 MeV\\\\
- gamma = E/E0 = 2000/938 = 2.13\\\\
- v: solve 2.13 = 1/sqrt(1-v2/c2)\\\\
- v = c*sqrt(1 - 1/2.13^2) = about 0.88c\\\\
- Walk around and check progress}
\end{frame}

\begin{frame}
\frametitle{Summary}
\pause
\begin{itemize}
    \item \textbf{Velocity Addition}: Velocities do not add linearly; $c$ is the max speed.
    \pause
    \item \textbf{Momentum}: $p = \gamma mv$. Infinite momentum is required to reach $c$.
    \pause
    \item \textbf{Mass-Energy}: Mass is a form of energy ($E_0 = mc^2$).
    \pause
    \item \textbf{Total Energy}: $E = \gamma mc^2$.
    \pause
    \item \textbf{Kinetic Energy}: $KE = (\gamma - 1)mc^2$.
\end{itemize}
\note{- Velocity addition: denominator prevents exceeding c\\\\
- Momentum: gamma factor makes p infinite at c\\\\
- E=mc2: mass is concentrated energy\\\\
- Total E = gamma * rest energy\\\\
- KE = extra energy from motion\\\\
- All formulas reduce to Newton at low speeds\\\\
- Questions? HW: practice problems 28.4-28.6}
\end{frame}

\end{document}
