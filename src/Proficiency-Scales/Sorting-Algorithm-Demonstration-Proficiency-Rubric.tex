\documentclass[10pt]{article}
\usepackage[margin=0.75in]{geometry}
\usepackage[T1]{fontenc}
\usepackage{tcolorbox}
\tcbuselibrary{skins}
\usepackage{enumitem}
\usepackage{xcolor}
\usepackage{titlesec}
\usepackage{multicol}

% Define proficiency colors
\definecolor{emerging}{HTML}{E57373}
\definecolor{developing}{HTML}{FFB74D}
\definecolor{proficient}{HTML}{81C784}
\definecolor{extending}{HTML}{64B5F6}

% Compact list settings
\setlist[itemize]{nosep, left=0pt, itemsep=1pt, topsep=2pt}

% Title formatting
\titleformat{\section}{\large\bfseries\sffamily}{}{0pt}{}
\titlespacing*{\section}{0pt}{8pt}{4pt}

% Proficiency box style
\tcbset{
  profbox/.style={
    boxrule=0pt,
    arc=3pt,
    left=6pt,
    right=6pt,
    top=4pt,
    bottom=4pt,
    fonttitle=\bfseries\sffamily,
    coltitle=black,
    attach boxed title to top left={yshift=-2mm, xshift=4mm},
    boxed title style={boxrule=0pt, arc=2pt}
  }
}

\pagestyle{empty}

\begin{document}

\begin{center}
{\LARGE\bfseries\sffamily Sorting Algorithm Demonstration Proficiency Rubric}\\[4pt]
{\small\itshape Live demonstration of sorting algorithms with playing cards}
\end{center}

\vspace{8pt}

\begin{multicols}{2}

%% EMERGING
\begin{tcolorbox}[profbox, colback=emerging!15, colframe=emerging!60,
  title={\colorbox{emerging!80}{\strut\hspace{2pt}EMERGING\hspace{2pt}}}]
\small\itshape Unable to complete the sorting process correctly for one or both algorithms
\tcblower
\small
\begin{itemize}
\item Gets stuck or uses wrong steps; cannot complete the sort without significant help
\item Cannot explain what they are doing; moves cards without describing the process
\item Confuses one algorithm with another or invents steps not part of the algorithm
\item Misses scheduled demonstration without communication or refuses to attempt
\end{itemize}
\end{tcolorbox}

\vspace{6pt}

%% DEVELOPING
\begin{tcolorbox}[profbox, colback=developing!15, colframe=developing!60,
  title={\colorbox{developing!80}{\strut\hspace{2pt}DEVELOPING\hspace{2pt}}}]
\small\itshape Completes sorting with some errors or needs occasional prompts
\tcblower
\small
\begin{itemize}
\item Sorts cards correctly but skips steps or makes minor errors; may need a hint to continue
\item Explains some steps but struggles to describe why comparisons or swaps happen
\item Shows basic understanding but cannot answer follow-up questions about the algorithm
\item Arrives late to demonstration or needs to reschedule at last minute
\end{itemize}
\end{tcolorbox}

\columnbreak

%% PROFICIENT
\begin{tcolorbox}[profbox, colback=proficient!15, colframe=proficient!60,
  title={\colorbox{proficient!80}{\strut\hspace{2pt}PROFICIENT\hspace{2pt}}}]
\small\itshape Completes both algorithms correctly with clear verbal explanation
\tcblower
\small
\begin{itemize}
\item Sorts cards using correct steps for both algorithms without errors or prompts
\item Clearly explains each step as it happens: what is being compared, why cards swap or move
\item Answers follow-up questions correctly showing genuine understanding, not memorization
\item Arrives on time and ready; communicates if rescheduling is needed
\end{itemize}
\end{tcolorbox}

\vspace{6pt}

%% EXTENDING
\begin{tcolorbox}[profbox, colback=extending!15, colframe=extending!60,
  title={\colorbox{extending!80}{\strut\hspace{2pt}EXTENDING\hspace{2pt}}}]
\small\itshape Demonstrates deep understanding and can explain algorithm behavior
\tcblower
\small
\begin{itemize}
\item Executes algorithms smoothly while explaining; identifies sorted/unsorted regions clearly
\item Can explain edge cases: what if cards were already sorted? What if all cards were the same?
\item Compares algorithms: explains why quick/merge sort is faster for large lists
\item Prepared and confident; could teach the algorithm to another student
\end{itemize}
\end{tcolorbox}

\end{multicols}

\vspace{4pt}
\noindent\small\textbf{Assessment Note:} You will demonstrate two randomly selected algorithms using 9 playing cards (2--10). One algorithm from \{bubble, selection, insertion\} and one from \{quick, merge\}. Explain each step as you sort. Be prepared for follow-up questions to show understanding.

\end{document}
