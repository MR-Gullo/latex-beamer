\documentclass[10pt]{article}
\usepackage[margin=0.75in]{geometry}
\usepackage[T1]{fontenc}
\usepackage{tcolorbox}
\usepackage{enumitem}
\usepackage{xcolor}
\usepackage{titlesec}
\usepackage{multicol}

% Define proficiency colors
\definecolor{emerging}{HTML}{E57373}
\definecolor{developing}{HTML}{FFB74D}
\definecolor{proficient}{HTML}{81C784}
\definecolor{extending}{HTML}{64B5F6}

% Compact list settings
\setlist[itemize]{nosep, left=0pt, itemsep=2pt, topsep=3pt}
\setlist[enumerate]{nosep, left=0pt, itemsep=2pt, topsep=3pt}

% Title formatting
\titleformat{\section}{\normalsize\bfseries\sffamily}{}{0pt}{}
\titlespacing*{\section}{0pt}{6pt}{3pt}

\pagestyle{empty}

\begin{document}

\begin{center}
{\LARGE\bfseries\sffamily Parent Guide: Understanding Proficiency Scales}
\end{center}

\vspace{6pt}

\begin{multicols}{2}

\section{Why no number grades (like 85\%)?}
We use \textbf{Proficiency Scales} instead of percentages---an incoming BC Ministry of Education requirement. The goal: shift from ``getting marks'' to ``improving learning.''

A number grade tells you \emph{where} your child stands but not \emph{how} to improve. A proficiency scale describes what your child can currently do and exactly what they need to do to reach the next level.

\section{What do the levels mean?}

\begin{tcolorbox}[colback=emerging!15, colframe=emerging!60, boxrule=0pt, arc=2pt, left=4pt, right=4pt, top=2pt, bottom=2pt]
\small\textbf{Emerging} --- Just starting. Has initial understanding but needs help.\\
\emph{``I am just learning this.''}
\end{tcolorbox}
\vspace{2pt}
\begin{tcolorbox}[colback=developing!15, colframe=developing!60, boxrule=0pt, arc=2pt, left=4pt, right=4pt, top=2pt, bottom=2pt]
\small\textbf{Developing} --- Understands some parts but is still inconsistent.\\
\emph{``I get it, but I still make mistakes.''}
\end{tcolorbox}
\vspace{2pt}
\begin{tcolorbox}[colback=proficient!15, colframe=proficient!60, boxrule=0pt, arc=2pt, left=4pt, right=4pt, top=2pt, bottom=2pt]
\small\textbf{Proficient} --- Fully meets grade-level expectations. \textbf{This is the goal.}\\
\emph{``I can do this on my own reliably.''}
\end{tcolorbox}
\vspace{2pt}
\begin{tcolorbox}[colback=extending!15, colframe=extending!60, boxrule=0pt, arc=2pt, left=4pt, right=4pt, top=2pt, bottom=2pt]
\small\textbf{Extending} --- Goes beyond requirements with deep thinking.\\
\emph{``I can use this in new and creative ways.''}
\end{tcolorbox}

\section{How does this compare to letter grades?}
\small
\begin{itemize}
\item \textbf{Extending:} A range
\item \textbf{Proficient/Proficient+:} B to A range \textbf{(goal for all students)}
\item \textbf{Developing/Developing+:} C to C+ range
\item \textbf{Emerging/Emerging+:} Passing (C-)
\item \textbf{Pre-Emerging:} Not yet passing (Incomplete)
\end{itemize}
\emph{Note: A ``+'' sign means your child is very close to the next level.}

\columnbreak

\section{Scales used in this class}
\small
\begin{itemize}
\item \textbf{Homework/Responsibility:} Organization, meeting deadlines, following submission rules
\item \textbf{Written Work:} Logical thinking, showing steps clearly, using units correctly
\item \textbf{Lab Documentation:} Recording experiments, data quality, photos, experimental notes
\item \textbf{Programming:} Coding skills, from basic syntax to solving complex problems independently
\item \textbf{Verbal Assessment:} Explaining concepts orally, using correct vocabulary
\item \textbf{Video Analysis:} Applying physics concepts (Newton's Laws) to real-world videos
\end{itemize}

\emph{Example:} To move from \textbf{Developing} to \textbf{Proficient} in \textbf{Verbal Assessment}, a student needs to go from ``answering with occasional teacher guidance'' to ``providing complete explanations without prompting.''

\section{How can I help my child improve?}

Ask your child to look at the scale with you:

\begin{enumerate}
\item Identify where they are right now (e.g., Developing)
\item Look at the description for the next level up (e.g., Proficient)
\item Ask: \emph{``What is the one specific thing you need to change in your next assignment to reach that level?''}
\end{enumerate}

\end{multicols}

\end{document}
