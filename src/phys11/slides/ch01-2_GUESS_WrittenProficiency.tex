\documentclass{beamer}
% Use DS9 global theme
\usepackage{../../../shared/templates/ds9_theme}

\title[Assessment Tips and Proficiency]{Assessment Strategy and Written Work Proficiency}
\subtitle{Show Clear Thinking on Every Question}
\author[Mr. Gullo]{Mr. Gullo}
\date[2024-25]{2024-25 School Year}

\begin{document}

\frame{\titlepage}

\section{Introduction: Two Types of Assessment}

\begin{frame}{Learning Objectives}
By the end of this presentation, you will be able to:
\begin{itemize}
\item Know the difference between math and thinking questions \pause
\item Use the Written Work Proficiency Scale on all questions \pause
\item Use the G.U.E.S.S. method for math questions \pause
\item Use sketches and diagrams when helpful \pause
\item Show clear, organized thinking \pause
\item Move from Emerging to Extending level
\end{itemize}
\end{frame}

\begin{frame}{Assessment for Learning and Written Work}
\begin{itemize}

\item -The Assessment for Learning category is worth 0 of your grade. It is for practice and feedback only. \pause
\item -On quizzes, you must show your work. If your steps are correct, you can get partial marks even if the final answer is wrong. \pause
\item On unit tests, midterm and final exam, written work will be worth 30 of the score.  \pause Up to three questions specified will be checked for written work. 
\end{itemize}
\end{frame}


\section{Method 1: G.U.E.S.S. for Problem-Solving}

\begin{frame}{G.U.E.S.S. Method Review}
For calculation-based questions, use the systematic G.U.E.S.S. approach:

\pause
\vspace{0.5cm}
\begin{flushleft}
\textbf{G} - Givens \& Diagram \pause \\
\textbf{U} - Unknowns \& Plan \pause \\
\textbf{E} - Equations \pause \\
\textbf{S} - Substitute \& Solve \pause \\
\textbf{S} - Solution \& Statement \\
\end{flushleft}

\pause
\vspace{0.5cm}
\textbf{Perfect for:} Questions with numerical values, formulas, and calculations
\end{frame}

\begin{frame}{The Challenge: Two Question Types}
\begin{columns}
\begin{column}{0.5\textwidth}
\begin{block}{Calculation-Based Questions}
\begin{itemize}
\item Require numerical problem-solving \pause
\item Use formulas and equations \pause
\item Show mathematical work \pause
\item \textcolor{ds9blue}{G.U.E.S.S. method applies perfectly}
\end{itemize}
\end{block}
\end{column} \pause
\begin{column}{0.5\textwidth}
\begin{block}{Concept-Focused Questions}
\begin{itemize}
\item Test understanding of principles \pause
\item Require explanation and reasoning \pause
\item No calculations involved \pause
\item \textcolor{ds9blue}{G.U.E.S.S. method can still help organize thinking}
\end{itemize}
\end{block}
\end{column}
\end{columns}

\pause
\vspace{0.5cm}
\textbf{Today's Goal:} Show clear thinking on ALL question types!
\end{frame}



\section{Understanding the Written Work Proficiency Scale}

\begin{frame}{The Four Proficiency Levels}
\begin{block}{Emerging}
\textcolor{ds9red}{Just starting to show your thinking}
\end{block}

\pause
\begin{block}{Developing}
\textcolor{orange}{Getting better at organized thinking}
\end{block}
\end{frame}

\begin{frame}{The Four Proficiency Levels (continued)}
\begin{block}{Proficient}
\textcolor{ds9blue}{Clear, organized thinking shown consistently}
\begin{itemize}
\item Lists all important information clearly \pause
\item Shows logical steps throughout all work \pause
\item Uses units consistently on all work \pause
\item Makes clear, labeled sketches when helpful
\end{itemize}
\end{block}

\pause
\begin{block}{Extending}
\textcolor{ds9blue}{Exceptional clarity and deep thinking}
\begin{itemize}
\item Includes hidden details and assumptions \pause
\item Plans approach before starting \pause
\item Uses units and checks them for accuracy \pause
\item Uses multiple sketches when helpful
\end{itemize}
\end{block}
\end{frame}



\section{What "Showing Your Work" Means}

\begin{frame}{Two Types of Questions}
\begin{columns}
\begin{column}{0.5\textwidth}
\begin{block}{Math Questions}
\begin{itemize}
\item List what you know \pause
\item Include sketch when helpful \pause
\item Show each calculation step \pause
\item Include units \pause
\item Write final answer clearly
\end{itemize}
\end{block}
\end{column} \pause
\begin{column}{0.5\textwidth}
\begin{block}{Thinking Questions}
\begin{itemize}
\item State what question asks \pause
\item Include sketch when helpful \pause
\item Explain your reasoning step-by-step \pause
\item Use science vocabulary \pause
\item Address all parts of question
\end{itemize}
\end{block}
\end{column}
\end{columns}

\pause
\vspace{0.5cm}
\textbf{Key Point:} Both types need organized, clear thinking!
\end{frame}






\section{Proficiency Examples with Real Questions}

\begin{frame}{Example 1: Classical Physics Application}
\textbf{Question:} Can classical physics be used to accurately describe a satellite moving at 7,500 m/s?

\textbf{Options:}
\begin{itemize}
\item[\textbf{A.}] Yes, because the satellite is moving much slower than light speed and is not in a strong gravitational field.
\item[\textbf{B.}] No, because the satellite is moving much slower than light speed and is not in a strong gravitational field.
\item[\textbf{C.}] No, because the satellite is moving much slower than light speed and is in a strong gravitational field.
\item[\textbf{D.}] Yes, because the satellite is moving much slower than light speed and is in a strong gravitational field.
\end{itemize}

\pause
\begin{block}{Emerging Level Response}
A. Yes because it's slow.
\end{block}

\pause
\begin{block}{Developing Level Response}  
A. Yes, because 7,500 m/s is much slower than the speed of light so classical physics works.
\end{block}
\end{frame}

\begin{frame}{Example 1: Higher Proficiency Levels}
\begin{block}{Proficient Level Response}
Answer A. Classical physics works for this satellite. The satellite speed is 7,500 m/s, which is much slower than light speed ($3 \times 10^8$ m/s). Also, Earth's gravity is weak, not strong. Both conditions are met for classical physics to work.
\end{block}

\pause
\begin{block}{Extending Level Response}
Answer A. Classical physics applies when two conditions are met: (1) speeds less than 1\% of light speed, and (2) weak gravitational fields. For condition 1: 7,500 m/s ÷ ($3 \times 10^8$ m/s) = 0.0025\% of light speed, well below the 1\% threshold. For condition 2: Earth's gravitational field is weak compared to extreme objects like neutron stars. Options B and C incorrectly conclude "No" despite meeting both conditions. Option D incorrectly describes Earth's gravity as "strong." Only A correctly identifies both conditions are satisfied.
\end{block}
\end{frame}

\begin{frame}{Example 2: Scientific Method}
\textbf{Question:} Which statement is a testable hypothesis about why ants gather in a specific area?

\textbf{Options:}
\begin{itemize}
\item[\textbf{A.}] There may be some food particles lying there.
\item[\textbf{B.}] The worker ants thought it was a nice location.
\item[\textbf{C.}] The worker ants may have to find a spot for the queen to lay eggs.
\item[\textbf{D.}] The worker ants are supposed to group together at a place.
\end{itemize}

\pause
\begin{block}{Emerging Level Response}
A. There may be food particles there.
\end{block}

\pause
\begin{block}{Developing Level Response}
A. There may be food particles there because you can test if there's food there.
\end{block}
\end{frame}

\begin{frame}{Example 2: Higher Proficiency Levels}
\begin{block}{Proficient Level Response}
Answer A: Food particles. A good hypothesis must be testable. We can test for food by looking for it and measuring it. This makes it a good scientific hypothesis.
\end{block}

\pause
\begin{block}{Extending Level Response}  
Answer A: Food particles. A testable hypothesis must be both observable and measurable with controlled experiments. Option A allows me to design a test: remove ants, clean the area, place different substances (food vs. non-food), then measure ant response. Options B, C, and D all contain subjective elements: B refers to ant "thoughts," C assumes ant "intentions" for egg-laying, and D implies ants are "supposed" to behave certain ways. These subjective concepts cannot be objectively measured, controlled, or falsified in scientific experiments. Only A meets the empirical criteria for scientific testability.
\end{block}
\end{frame}

\begin{frame}{Example 3: Math Question}
\textbf{Question:} A bathroom scale reads 65 kg with 3\% uncertainty. What is the uncertainty in kg?

\textbf{Options:}
\begin{itemize}
\item[\textbf{A.}] 2 kg
\item[\textbf{B.}] 98 kg  
\item[\textbf{C.}] 5 kg
\item[\textbf{D.}] 0 kg
\end{itemize}

\pause
\begin{block}{Proficient Level - G.U.E.S.S. Approach}
\begin{flushleft}
\textbf{G:} Mass = 65 kg, \% uncertainty = 3\% \\
\textbf{U:} Uncertainty in kg \\
\textbf{E:} Uncertainty = (3/100) × 65 kg \\
\textbf{S:} (0.03)(65) = 1.95 kg \\
\textbf{S:} Answer A: 2 kg (rounded appropriately)
\end{flushleft}
\end{block}
\end{frame}

\begin{frame}{Example 3: Extending Level}
\begin{block}{Extending Level - Enhanced G.U.E.S.S. Approach}
\begin{flushleft}
\textbf{G:} Mass = 65 kg, relative uncertainty = 3\% (assuming this is the standard uncertainty) \\
\textbf{U:} Absolute uncertainty in kg, with proper significant figures \\
\textbf{E:} Absolute uncertainty = relative uncertainty × measured value = (3.0/100) × 65 kg \\
\textbf{S:} (0.030)(65) = 1.95 kg \\
\textbf{S:} Answer A: 2.0 kg (rounded to 2 sig figs to match the precision of the 3\% uncertainty)
\end{flushleft}
\textbf{Check:} Answer A is correct. Option B (98 kg) appears to be 65 + 33, likely from adding instead of finding percentage. Option C (5 kg) might result from rounding errors or using 5% instead of 3\%. Option D (0 kg) is impossible since all measurements have uncertainty. Final answer: 65 ± 2 kg, indicating range is 63-67 kg.
\end{block}

\pause
\textbf{Key Point:} Use G.U.E.S.S. method for all math questions!
\end{frame}

\section{Strategic Application Guide}


\begin{frame}{Proficiency Scale Checklist}
\begin{block}{Moving from Developing to Proficient}
\begin{itemize}
\item $\checkmark$ Use G.U.E.S.S. method for math questions consistently \pause
\item $\checkmark$ Show all reasoning steps clearly \pause
\item $\checkmark$ Address all parts of the question \pause
\item $\checkmark$ Use proper physics terminology
\end{itemize}
\end{block}

\pause
\begin{block}{Moving from Proficient to Extending}
\begin{itemize}
\item $\checkmark$ Include implicit information and assumptions \pause
\item $\checkmark$ Explain the "why" behind your reasoning \pause
\item $\checkmark$ Connect to broader physics principles \pause
\item $\checkmark$ Consider alternative approaches or perspectives \pause
\item $\checkmark$ Check your reasoning for consistency
\end{itemize}
\end{block}
\end{frame}

\section{Summary and Final Strategy}

\begin{frame}{Your Assessment Strategy Toolkit}
\begin{block}{Before You Start}
\begin{itemize}
\item Read the question carefully - identify the type \pause
\item Use G.U.E.S.S. method for math questions \pause
\item Plan your response structure
\end{itemize}
\end{block}

\pause
\begin{block}{During Your Response}
\begin{itemize}
\item Follow your chosen method systematically \pause
\item Show all thinking steps clearly \pause
\item Use proper physics language and concepts \pause
\item Connect reasoning to the final answer
\end{itemize}
\end{block}

\pause
\begin{block}{Aiming for Extending}
\begin{itemize}
\item Include hidden assumptions and details \pause
\item Explain the deeper physics principles \pause
\item Check your reasoning multiple ways
\end{itemize}
\end{block}
\end{frame}

\begin{frame}{Final Reminders}
\begin{center}
\Large
\textcolor{ds9blue}{\textbf{Stay Systematic}} \\
\vspace{0.3cm}
\textcolor{ds9gold}{\textbf{Show Your Thinking}} \\
\vspace{0.3cm}
\textcolor{ds9blue}{\textbf{Choose the Right Method}} \\
\vspace{0.3cm}
\textcolor{ds9red}{\textbf{Aim for Understanding}} \\
\end{center}

\vspace{1cm}

\begin{center}
\textbf{Use G.U.E.S.S. for math, clear thinking for concepts!}
\end{center}
\end{frame}

\end{document}