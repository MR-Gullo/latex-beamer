\documentclass{beamer}
% Use DS9 global theme (includes pgfplots for visualization)
\usepackage{../../../../latex-beamer/shared/templates/ds9_theme}

% Title page configuration
\title[Waves and Properties]{PHYS12 CH:13.1-13.3}
\subtitle{Waves and Their Properties}
\author[Mr. Gullo]{Mr. Gullo}
\date[Dec 2025]{December 2025}

\begin{document}
\frame{\titlepage}

% Outline
\begin{frame}
\frametitle{Outline}
\tableofcontents
\end{frame}

\section{Introduction to Waves}

\begin{frame}
\frametitle{Learning Objectives}
By the end of this lesson, you will be able to:
\begin{itemize}
\item Distinguish transverse from longitudinal waves \pause
\item Identify mechanical vs electromagnetic waves \pause
\item Calculate wave speed using $v = f\lambda$ \pause
\item Relate period and frequency: $T = 1/f$ \pause
\item Explain the superposition principle \pause
\item Differentiate constructive and destructive interference \pause
\item Describe how standing waves form
\end{itemize}
\end{frame}

\begin{frame}
\frametitle{Physics 11 to Physics 12 Progression}
\textbf{Physics 11 Foundation:}
\begin{itemize}
\item Basic wave terminology (crest, trough, wavelength)
\item Simple harmonic motion concepts
\item Qualitative wave descriptions
\end{itemize}

\pause
\vspace{0.5cm}

\textbf{Physics 12 Extension:}
\begin{itemize}
\item Mathematical relationships between wave properties
\item Quantitative problem solving with $v = f\lambda$
\item Interference patterns and standing waves
\item Applications: earthquakes, acoustics, optics
\end{itemize}
\end{frame}

\section{Types of Waves}

\begin{frame}
\frametitle{What is a Wave?}
\textbf{Key Insight:} Waves transfer \textbf{energy} without transferring \textbf{matter}.

\pause
\vspace{0.5cm}

\textbf{Example:} Water waves
\begin{itemize}
\item A seagull bobs up and down as waves pass
\item The bird does NOT move horizontally with the wave
\item Only energy travels across the water surface
\end{itemize}

\pause
\vspace{0.5cm}

A \textbf{disturbance} is an oscillation produced by energy that creates a wave.
\end{frame}

\begin{frame}
\frametitle{Mechanical vs Electromagnetic Waves}
\begin{columns}[T]
\column{0.48\textwidth}
\textbf{Mechanical Waves}
\begin{itemize}
\item Require a medium (solid, liquid, gas)
\item Examples: sound, water waves, seismic waves
\item Cannot travel through vacuum
\end{itemize}

\pause
\column{0.48\textwidth}
\textbf{Electromagnetic Waves}
\begin{itemize}
\item Do NOT require a medium
\item Examples: light, radio, X-rays
\item Can travel through vacuum (space)
\end{itemize}
\end{columns}

\pause
\vspace{0.5cm}

\alert{Why can light travel through space but sound cannot?}\\
Sound is mechanical; light is electromagnetic.
\end{frame}

\begin{frame}
\frametitle{Transverse Waves}
\textbf{Definition:} Particle motion is \textbf{perpendicular} to wave direction.

\pause
\vspace{0.3cm}

\begin{alertblock}{[Diagram: Transverse Wave]}
Show a wave moving horizontally (right) with particles oscillating vertically (up/down).
\begin{itemize}
\item Arrow showing wave propagation direction $\rightarrow$
\item Arrows showing particle motion $\uparrow \downarrow$
\item Label: crest (highest point), trough (lowest point)
\end{itemize}
\end{alertblock}

\pause
\vspace{0.3cm}

\textbf{Examples:}
\begin{itemize}
\item Light waves (E and B fields oscillate perpendicular to travel)
\item S-waves (seismic shear waves)
\item Waves on a string
\end{itemize}
\end{frame}

\begin{frame}
\frametitle{Longitudinal Waves}
\textbf{Definition:} Particle motion is \textbf{parallel} to wave direction.

\pause
\vspace{0.3cm}

\begin{alertblock}{[Diagram: Longitudinal Wave]}
Show a spring or air column with:
\begin{itemize}
\item Compressions: regions where particles are close together
\item Rarefactions: regions where particles are spread apart
\item Wave direction arrow parallel to particle oscillation
\end{itemize}
\end{alertblock}

\pause
\vspace{0.3cm}

\textbf{Examples:}
\begin{itemize}
\item Sound waves in air (density variations)
\item P-waves (seismic pressure waves)
\item Compression waves in a slinky
\end{itemize}
\end{frame}

\begin{frame}
\frametitle{Seismic Waves: A Special Case}
Earthquakes produce both types of waves:

\pause
\begin{columns}[T]
\column{0.48\textwidth}
\textbf{P-waves (Primary)}
\begin{itemize}
\item Longitudinal (compression)
\item Travel through solids AND liquids
\item Faster: arrive first
\item Typically 6 km/s
\end{itemize}

\pause
\column{0.48\textwidth}
\textbf{S-waves (Secondary)}
\begin{itemize}
\item Transverse (shear)
\item Travel through solids ONLY
\item Slower: arrive second
\item Typically 2 km/s
\end{itemize}
\end{columns}

\pause
\vspace{0.5cm}

\alert{Fun Fact:} S-waves cannot travel through Earth's liquid outer core. This is how scientists discovered the core is liquid!
\end{frame}

\section{Wave Properties}

\begin{frame}
\frametitle{Wave Anatomy}
\begin{alertblock}{[Diagram: Labeled Wave]}
A sinusoidal wave with labels:
\begin{itemize}
\item Crest: highest point
\item Trough: lowest point
\item Amplitude (A): equilibrium to crest distance
\item Wavelength ($\lambda$): crest-to-crest distance
\item Equilibrium position: horizontal center line
\end{itemize}
\end{alertblock}

\pause
\vspace{0.3cm}

\textbf{Important:} Wave height = $2A$ (trough to crest)

If amplitude = 1 m, wave height = 2 m.
\end{frame}

\begin{frame}
\frametitle{Key Wave Properties}
\begin{columns}[T]
\column{0.48\textwidth}
\textbf{Wavelength} ($\lambda$)
\begin{itemize}
\item Distance for one complete cycle
\item Measured: crest to crest
\item Units: meters (m)
\end{itemize}

\pause
\textbf{Amplitude} (A)
\begin{itemize}
\item Maximum displacement from equilibrium
\item Related to energy
\item Units: meters (m)
\end{itemize}

\pause
\column{0.48\textwidth}
\textbf{Frequency} ($f$)
\begin{itemize}
\item Cycles per second
\item Units: Hertz (Hz = 1/s)
\end{itemize}

\pause
\textbf{Period} ($T$)
\begin{itemize}
\item Time for one complete cycle
\item Units: seconds (s)
\end{itemize}
\end{columns}

\pause
\vspace{0.5cm}

\boxed{T = \frac{1}{f} \quad \text{or} \quad f = \frac{1}{T}}
\end{frame}

\begin{frame}
\frametitle{The Wave Equation}
The speed of a wave relates wavelength and frequency:

\pause
\vspace{0.3cm}

\begin{center}
\Huge
$\boxed{v = f\lambda}$
\end{center}

\pause
\vspace{0.3cm}

Also written as: $v = \frac{\lambda}{T}$

\pause
\vspace{0.5cm}

\textbf{Key Points:}
\begin{itemize}
\item Wave speed depends on the \textbf{medium properties} (stiffness, density)
\item Amplitude does NOT affect wave speed
\item For a fixed medium: if $f$ increases, $\lambda$ must decrease
\end{itemize}
\end{frame}

\begin{frame}
\frametitle{Speaker Design: Woofers and Tweeters}
Why do quality speakers have multiple drivers?

\pause
\vspace{0.3cm}

\begin{columns}[T]
\column{0.48\textwidth}
\textbf{Woofer} (large)
\begin{itemize}
\item Low frequency sounds
\item Long wavelengths
\item Bass notes
\end{itemize}

\pause
\column{0.48\textwidth}
\textbf{Tweeter} (small)
\begin{itemize}
\item High frequency sounds
\item Short wavelengths
\item Treble notes
\end{itemize}
\end{columns}

\pause
\vspace{0.5cm}

The driver size matches the wavelength it reproduces best.

Human hearing: 20 Hz to 20,000 Hz
\end{frame}

\section{Wave Interactions}

\begin{frame}
\frametitle{The Superposition Principle}
When two waves meet, their \textbf{displacements add algebraically}.

\pause
\vspace{0.3cm}

\begin{alertblock}{[Diagram: Superposition]}
Show two waves approaching, overlapping, and passing through each other.
\begin{itemize}
\item Before: two separate pulses
\item During: combined amplitude
\item After: original pulses continue unchanged
\end{itemize}
\end{alertblock}

\pause
\vspace{0.3cm}

The waves pass through each other unchanged after interaction.

Only the \textbf{amplitude} is affected during overlap.
\end{frame}

\begin{frame}
\frametitle{Constructive Interference}
Occurs when waves are \textbf{in phase} (crests align with crests).

\pause
\vspace{0.3cm}

\begin{alertblock}{[Diagram: Constructive Interference]}
Two identical waves in phase:
\begin{itemize}
\item Wave 1: amplitude = A
\item Wave 2: amplitude = A
\item Resultant: amplitude = 2A
\end{itemize}
\end{alertblock}

\pause
\vspace{0.3cm}

\textbf{Result:} Larger amplitude (louder sound, brighter light)

\textbf{Example:} Sound is louder in certain spots in a room.
\end{frame}

\begin{frame}
\frametitle{Destructive Interference}
Occurs when waves are \textbf{out of phase} (crests align with troughs).

\pause
\vspace{0.3cm}

\begin{alertblock}{[Diagram: Destructive Interference]}
Two identical waves, 180° out of phase:
\begin{itemize}
\item Wave 1: amplitude = +A
\item Wave 2: amplitude = -A
\item Resultant: amplitude = 0
\end{itemize}
\end{alertblock}

\pause
\vspace{0.3cm}

\textbf{Result:} Smaller or zero amplitude (quiet zones)

\textbf{Application:} Noise-canceling headphones produce waves 180° out of phase with ambient noise!
\end{frame}

\begin{frame}
\frametitle{Standing Waves}
Form when identical waves travel in \textbf{opposite directions}.

\pause
\vspace{0.3cm}

\begin{alertblock}{[Diagram: Standing Wave on String]}
Show a vibrating string with:
\begin{itemize}
\item Nodes: points of zero displacement (no motion)
\item Antinodes: points of maximum displacement
\item Fixed ends of string
\end{itemize}
\end{alertblock}

\pause
\vspace{0.3cm}

\textbf{Examples:}
\begin{itemize}
\item Guitar strings
\item Organ pipes
\item Resonance in bridges
\end{itemize}

Standing wave frequency depends on wave speed and string length.
\end{frame}

\begin{frame}
\frametitle{Refraction}
Waves change direction when entering a medium of different density.

\pause
\vspace{0.3cm}

\textbf{Why it happens:}
\begin{itemize}
\item Wave speed changes with medium density
\item Part of wave slows down before the rest
\item This causes the wave to bend
\end{itemize}

\pause
\vspace{0.3cm}

\textbf{Examples:}
\begin{itemize}
\item Light bending through glass/water (distorted objects)
\item Water waves at shallow/deep boundary
\item Fiber optic cables use internal refraction
\end{itemize}

Waves travel faster in less dense media (if stiffness is same).
\end{frame}

\section{Practice Problems}

\begin{frame}
\frametitle{I Do: Seagull Bobbing Problem}
\textbf{Problem (Ch. 13, Q25):}\\
A seagull sitting in water bobs up and down once every 2 s. The distance between two crests is 3 m. What is the wave velocity?

\pause
\vspace{0.5cm}

\begin{columns}[T]
\column{0.48\textwidth}
\textbf{G - Givens}
\begin{itemize}
\item $T = 2$ s (period: bobs once every 2 s)
\item $\lambda = 3$ m (crest-to-crest distance)
\end{itemize}

\pause
\column{0.48\textwidth}
\textbf{U - Unknown}
\begin{itemize}
\item $v = ?$ (wave velocity)
\end{itemize}
\end{columns}
\end{frame}

\begin{frame}
\frametitle{I Do: Equation Selection}
\textbf{E - Equation}

\pause
\vspace{0.3cm}

We need to relate velocity, period, and wavelength.

\pause
\vspace{0.3cm}

Start with: $v = f\lambda$

\pause
\vspace{0.3cm}

Since $f = \frac{1}{T}$, substitute:

\pause
\vspace{0.3cm}

$$v = \frac{\lambda}{T}$$

\pause
\vspace{0.3cm}

No rearrangement needed; $v$ is already isolated.
\end{frame}

\begin{frame}
\frametitle{I Do: Substitute and Solve}
\textbf{S - Substitute}
\begin{itemize}
\item $v = \frac{\lambda}{T} = \frac{3 \text{ m}}{2 \text{ s}}$
\end{itemize}

\pause
\vspace{0.5cm}

\textbf{S - Solve}
\begin{itemize}
\item $v = 1.5$ m/s
\end{itemize}

\pause
\vspace{0.5cm}

$$\boxed{v = 1.5 \text{ m/s}}$$

\pause
\vspace{0.3cm}

\textbf{Check:} Units work out (m/s). Reasonable speed for water waves.
\end{frame}

\begin{frame}
\frametitle{We Do: Boat in Waves}
\textbf{Problem (Ch. 13, Q26):}\\
A boat in the trough of a wave takes 3 s to reach the highest point. The wave velocity is 5 m/s. What is the wavelength?

\pause
\vspace{0.5cm}

\begin{columns}[T]
\column{0.48\textwidth}
\textbf{G - Givens}
\begin{itemize}
\item Time trough to crest = 3 s
\item $v = 5$ m/s
\end{itemize}

\pause
\column{0.48\textwidth}
\textbf{U - Unknown}
\begin{itemize}
\item $\lambda = ?$
\end{itemize}
\end{columns}

\pause
\vspace{0.5cm}

\alert{Hint:} Trough to crest is only HALF a cycle. What is the full period?
\end{frame}

\begin{frame}
\frametitle{We Do: Equation and Setup}
\textbf{E - Equation}

\pause
\vspace{0.3cm}

First, find period: $T = 2 \times 3 \text{ s} = 6 \text{ s}$

\pause
\vspace{0.3cm}

Then find frequency: $f = \frac{1}{T} = \frac{1}{6}$ Hz

\pause
\vspace{0.3cm}

Use wave equation: $v = f\lambda$

\pause
\vspace{0.3cm}

Rearrange for $\lambda$: 
$$\lambda = \frac{v}{f}$$
\end{frame}

\begin{frame}
\frametitle{We Do: Substitute and Solve}
\textbf{S - Substitute}
\begin{itemize}
\item $\lambda = \frac{v}{f} = \frac{5 \text{ m/s}}{(1/6) \text{ Hz}}$
\end{itemize}

\pause
\vspace{0.5cm}

\textbf{S - Solve}
\begin{itemize}
\item $\lambda = 5 \times 6 = 30$ m
\end{itemize}

\pause
\vspace{0.5cm}

$$\boxed{\lambda = 30 \text{ m}}$$
\end{frame}

\begin{frame}
\frametitle{You Do: Earthquake Distance}
\textbf{Problem (Ch. 13, Q68):}\\
The time difference between a 2 km/s S-wave and a 6 km/s P-wave recorded at a certain point is 10 s. How far is the epicenter from that point?

\vspace{0.5cm}

\textbf{Given:}
\begin{itemize}
\item S-wave speed: $v_S = 2$ km/s
\item P-wave speed: $v_P = 6$ km/s
\item Time difference: $\Delta t = 10$ s
\end{itemize}

\vspace{0.3cm}

\textbf{Find:} Distance $D$ to epicenter

\vspace{0.5cm}

\textbf{Hint:} Both waves travel the same distance. Set up equations for the time each takes, then use the time difference.
\end{frame}

\section{Summary}

\begin{frame}
\frametitle{Reading Assignment}
\textbf{For next class, read and review:}
\begin{itemize}
\item Detailed seismic wave analysis (Earth structure mapping)
\item Fiber optic cable applications (total internal reflection)
\item Performance Task: Wave tank experiments (Q27)
\item Extended response questions 63-72
\end{itemize}

\vspace{0.5cm}

\textbf{Practice Problems:}
\begin{itemize}
\item Complete "You Do" earthquake problem
\item Problems 25-26, 32-42 from Chapter Review
\end{itemize}
\end{frame}

\begin{frame}
\frametitle{Chapter 13 Summary}
\textbf{Key Takeaways:}

\pause
\begin{enumerate}
\item Waves transfer \textbf{energy}, not matter \pause
\item Mechanical waves need a medium; EM waves do not \pause
\item Transverse: motion $\perp$ to propagation\\
      Longitudinal: motion $\parallel$ to propagation \pause
\item Wave equation: $v = f\lambda = \frac{\lambda}{T}$ \pause
\item Wave speed depends on \textbf{medium}, not amplitude \pause
\item Superposition: displacements add algebraically \pause
\item Constructive = in phase; Destructive = out of phase \pause
\item Standing waves: opposite-traveling waves create nodes and antinodes
\end{enumerate}
\end{frame}

\begin{frame}
\frametitle{Inquiry Questions}
\textbf{Think about these for discussion:}

\vspace{0.5cm}

\begin{itemize}
\item What factors affect wave behaviors? \pause
\item How would you investigate relationships between wave properties and medium properties? \pause
\item How can you determine which harmonics are audible in different musical instruments?
\end{itemize}

\vspace{0.5cm}

\pause
\textbf{Real-world connections:}
\begin{itemize}
\item Why can damage be worse far from an earthquake epicenter?
\item How do noise-canceling headphones work?
\item Why do objects look bent in water?
\end{itemize}
\end{frame}

\end{document}