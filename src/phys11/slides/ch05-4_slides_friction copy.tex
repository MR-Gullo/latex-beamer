\documentclass{beamer}
% Use DS9 global theme
\usepackage{../../../../shared/templates/ds9_theme}
% Title page configuration
\title[Friction and Inclined Planes]{PHYS11 CH:5.4}
\subtitle{Static and Kinetic Friction}
\author[Mr. Gullo]{Mr. Gullo}
\date[Nov 2024]{November 2024}
% Table of contents at the beginning of each section
\AtBeginSection[]
{
\begin{frame}
\frametitle{Table of Contents}
\tableofcontents[currentsection]
\end{frame}
}
% Add logo
\logo{\includegraphics[width=0.1\linewidth]{cinec_logo.png}}
\begin{document}
\frame{\titlepage}
\section{Introduction to Friction}
\begin{frame}
\frametitle{Learning Objectives}
By the end of this lesson, you will be able to:
\pause
\begin{itemize}
    \item Define friction and distinguish between static and kinetic friction
    \pause
    \item Apply friction formulas: fs≤μsNf_s \leq \mu_s N
fs​≤μs​N and fk=μkNf_k = \mu_k N
fk​=μk​N    \pause
    \item Resolve weight into components on inclined planes
    \pause
    \item Solve problems involving friction on horizontal and inclined surfaces
    \pause
    \item Use free body diagrams with rotated coordinate systems
\end{itemize}
\end{frame}

\begin{frame}
\frametitle{What is Friction?}
\begin{block}{Definition}
\textbf{Friction} is a force that opposes motion or attempted motion between surfaces in contact.
\end{block}
\pause
\vspace{1em}
\textbf{Key characteristics:}
\begin{itemize}
\item Acts parallel to contact surface
\pause
\item Direction: opposes motion (or attempted motion)
\pause
\item Magnitude: depends on surface properties and normal force
\pause
\item \textbf{Surprising fact}: Independent of contact area!
\end{itemize}
\end{frame}
\begin{frame}
\frametitle{Types of Friction}
\begin{itemize}
    \item \textbf{Static Friction} (fsf_s
fs​): Acts on objects at rest
    \begin{itemize}
        \item Variable force: adjusts to prevent motion
        \item Has maximum value: fs,max=μsNf_{s,max} = \mu_s N
fs,max​=μs​N    \end{itemize}
    \pause

