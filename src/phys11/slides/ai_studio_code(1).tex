\documentclass{beamer}
% Use DS9 global theme (includes pgfplots for visualization)
\usepackage{../../../shared/templates/ds9_theme}

% Title page configuration
\title[Problem Solving]{PHYS11 CH:Problem Solving Techniques}
\author[Mr. Gullo]{Mr. Gullo}
\date[Jul 22, 2025]{July 22, 2025}

\begin{document}

% Title Frame
\begin{frame}[allowframebreaks]
    \titlepage
\end{frame}

% Introduction to the GUESS Method
\begin{frame}[allowframebreaks]
    \frametitle{A Structured Approach to Problem Solving}
    \begin{block}{What is the GUESS Method?}
        Physics word problems can be complex. The \alert{GUESS method} is a systematic, step-by-step process designed to break down problems into manageable parts.
        \vspace{1em}

        It helps you organize information, select the right tools, and build a clear path to the solution.
        \vspace{1em}

        GUESS is an acronym that stands for:
        \begin{itemize}
            \item \textbf{G} - Givens
            \item \textbf{U} - Unknown
            \item \textbf{E} - Equation
            \item \textbf{S} - Substitute
            \item \textbf{S} - Solve
        \end{itemize}
    \end{block}
\end{frame}

% G - Givens
\begin{frame}[allowframebreaks]
    \frametitle{Step 1: G is for Givens}
    \begin{block}{Identify and List Knowns}
        The first step is to carefully read the problem and identify every piece of information provided.
        \begin{itemize}
            \item List all known quantities.
            \item Assign the correct variable symbol to each quantity (e.g., $v$ for velocity, $t$ for time).
            \item Always include the units (e.g., m/s, kg, s).
            \item Write down any relevant constants (e.g., acceleration due to gravity, $g \approx 9.8 \, \text{m/s}^2$).
        \end{itemize}
    \end{block}
    \begin{exampleblock}{Example}
        \textit{"A car travels at a constant speed of 15 m/s for 120 seconds."}
        \vspace{1em}
        
        \textbf{Givens:}
        \begin{itemize}
            \item Velocity ($v$) = 15 m/s
            \item Time ($t$) = 120 s
        \end{itemize}
    \end{exampleblock}
\end{frame}

% U - Unknown
\begin{frame}[allowframebreaks]
    \frametitle{Step 2: U is for Unknown}
    \begin{block}{Identify the Goal}
        Determine exactly what the problem is asking you to find. This is your target variable.
        \begin{itemize}
            \item Identify the quantity you need to solve for.
            \item Write it down using its variable symbol.
            \item Represent the unknown with a question mark (e.g., $d = ?$).
        \end{itemize}
        This step gives your work a clear focus.
    \end{block}
    \begin{exampleblock}{Example}
        \textit{"...How far does the car travel?"}
        \vspace{1em}
        
        \textbf{Unknown:}
        \begin{itemize}
            \item Distance ($d$) = ?
        \end{itemize}
    \end{exampleblock}
\end{frame}

% E - Equation
\begin{frame}[allowframebreaks]
    \frametitle{Step 3: E is for Equation}
    \begin{block}{Find the Right Tool}
        Based on your list of \alert{Givens} and your \alert{Unknown}, select a physics equation that connects them.
        \begin{itemize}
            \item Look at your formula sheet or recall relevant principles.
            \item The chosen equation should contain the variables you listed in the G and U steps.
            \item Sometimes you may need to rearrange the equation to solve for your unknown. It's often best to do this \textit{before} substituting numbers.
        \end{itemize}
    \end{block}
    \begin{exampleblock}{Example}
        \textbf{Givens:} $v$, $t$ \\
        \textbf{Unknown:} $d$
        \vspace{1em}

        \textbf{Equation:}
        The relationship between speed, distance, and time is:
        \[ v = \frac{d}{t} \]
        Or, solved for our unknown ($d$):
        \[ d = v \cdot t \]
    \end{exampleblock}
\end{frame}

% S - Substitute
\begin{frame}[allowframebreaks]
    \frametitle{Step 4: S is for Substitute}
    \begin{block}{Plug in the Numbers}
        Now that you have the correct equation, substitute the values from your \alert{Givens} list into the equation.
        \begin{itemize}
            \item Replace the variable symbols with their numerical values.
            \item \textbf{Crucially, include the units with the numbers.} This helps you check your work and ensure the final units are correct (dimensional analysis).
        \end{itemize}
    \end{block}
    \begin{exampleblock}{Example}
        \textbf{Equation:}
        \[ d = v \cdot t \]
        \vspace{1em}
        
        \textbf{Substitute:}
        \[ d = (15 \, \text{m/s}) \cdot (120 \, \text{s}) \]
    \end{exampleblock}
\end{frame}

% S - Solve
\begin{frame}[allowframebreaks]
    \frametitle{Step 5: S is for Solve}
    \begin{block}{Calculate the Answer}
        Perform the final calculation to find the value of your unknown.
        \begin{itemize}
            \item Do the math.
            \item Pay attention to how the units combine or cancel. In the example, seconds (s) in the denominator and numerator will cancel, leaving meters (m).
            \item Clearly state your final answer with the correct units and appropriate significant figures.
            \item Box your final answer to make it stand out.
        \end{itemize}
    \end{block}
    \begin{exampleblock}{Example}
        \textbf{Substitute:}
        \[ d = (15 \, \text{m/s}) \cdot (120 \, \text{s}) \]
        \vspace{1em}
        
        \textbf{Solve:}
        \[ d = 1800 \, \text{m} \]
        \begin{center}
        \fcolorbox{ds9blue}{ds9lightblue!20}{\large $d = 1.8$ km}
        \end{center}
    \end{exampleblock}
\end{frame}

% I Do - Fully Worked Example
\begin{frame}[allowframebreaks]
    \frametitle{Example Problem: "I Do"}
    \begin{block}{Problem}
        A ball is dropped from a height of 78 m. How long does it take to hit the ground? (Assume g = 9.8 m/s² and initial velocity is 0).
    \end{block}
    
    \begin{columns}[T]
        \begin{column}{0.5\textwidth}
            \begin{alertblock}{G - Givens}
                \begin{itemize}
                    \item Displacement ($\Delta y$) = 78 m
                    \item Initial velocity ($v_0$) = 0 m/s
                    \item Acceleration ($a$) = 9.8 m/s²
                \end{itemize}
            \end{alertblock}
            \begin{alertblock}{U - Unknown}
                \begin{itemize}
                    \item Time ($t$) = ?
                \end{itemize}
            \end{alertblock}
        \end{column}
        \begin{column}{0.5\textwidth}
            \begin{alertblock}{E - Equation}
                We need an equation with $\Delta y, v_0, a, t$.
                \[ \Delta y = v_0 t + \frac{1}{2} a t^2 \]
                Since $v_0 = 0$, this simplifies to:
                \[ \Delta y = \frac{1}{2} a t^2 \]
                Solving for t:  $t = \sqrt{\frac{2\Delta y}{a}}$
            \end{alertblock}
        \end{column}
    \end{columns}
    
    \begin{alertblock}{S - Substitute}
        \[ t = \sqrt{\frac{2 \cdot (78 \, \text{m})}{9.8 \, \text{m/s}^2}} \]
    \end{alertblock}
    
    \begin{alertblock}{S - Solve}
        \[ t = \sqrt{\frac{156 \, \text{m}}{9.8 \, \text{m/s}^2}} = \sqrt{15.918 \, \text{s}^2} \approx 3.99 \, \text{s} \]
        \begin{center}
        \fcolorbox{ds9blue}{ds9lightblue!20}{\large $t \approx 3.99$ s}
        \end{center}
    \end{alertblock}
\end{frame}

% Simple visualization for the example
\begin{frame}[allowframebreaks]
    \frametitle{Visualizing the Problem}
    A simple diagram helps confirm your understanding of the "Givens".
    \begin{figure}
    \begin{tikzpicture}[scale=1.5]
        % Draw building side
        \draw[thick] (0,4) -- (0,0);
        \fill[pattern=north east lines] (-0.2,4) rectangle (0,0);
        
        % Draw ground
        \draw[thick, fill=gray!30] (-2,-0.1) rectangle (2,0);
        
        % Ball at the top
        \node[circle, fill=ds9red, label=right:{$v_0 = 0$ m/s}] at (0.5, 3.8) {};
        
        % Displacement arrow and label
        \draw[<->, thick, ds9blue] (1,3.8) -- (1,0);
        \node[ds9blue, right] at (1, 2) {$\Delta y = 78$ m};
        
        % Acceleration vector
        \draw[->, thick, ds9orange] (0.5, 3) -- (0.5, 2.2);
        \node[ds9orange, right] at (0.5, 2.6) {$a = 9.8 \, \text{m/s}^2$};

        % Question mark for time
        \node[scale=2, ds9green, right] at (-1.5, 1) {$t = ?$};
        
    \end{tikzpicture}
    \caption{A falling ball under gravity.}
    \end{figure}
\end{frame}


% Summary Slide
\begin{frame}[allowframebreaks]
    \frametitle{Summary: Your Path to a Solution}
    \begin{block}{Remember to GUESS!}
        The GUESS method provides a reliable framework for tackling physics problems. It turns a potentially confusing word problem into a clear checklist.
    \end{block}
    \begin{itemize}
        \item[\textbf{G}] \textbf{Givens:} What do I know? List variables and units.
        \item[\textbf{U}] \textbf{Unknown:} What do I need to find? Identify the target.
        \item[\textbf{E}] \textbf{Equation:} What's the connection? Find the right formula.
        \item[\textbf{S}] \textbf{Substitute:} Plug in the known values \textit{with units}.
        \item[\textbf{S}] \textbf{Solve:} Calculate the answer and check your final units.
    \end{itemize}
    \begin{alertblock}{Master this process!}
        Using this method consistently will build your confidence and problem-solving skills.
    \end{alertblock}
\end{frame}


\end{document}