\documentclass{beamer}
% Use DS9 global theme (includes pgfplots for visualization)
\usepackage{../../../shared/templates/ds9_theme}

% Title page configuration
\title[Kinematics Graphs and Equations]{PHYS11 CH:2.3-2.4, 3.1-3.2}
\subtitle{Graphical Analysis of Motion \& Kinematic Equations}
\author[Mr. Gullo]{Mr. Gullo}
\date[Jul 23, 2025]{July 23, 2025}

\begin{document}
\frame{\titlepage}

\begin{frame}[allowframebreaks]
\frametitle{Learning Objectives}
\begin{block}{Lesson Goals}
After this lesson, you will be able to:
\begin{itemize}
    \item Analyze motion using position vs. time and velocity vs. time graphs.
    \item Interpret the slope of a position-time graph as velocity.
    \item Interpret the slope of a velocity-time graph as acceleration.
    \item Calculate displacement from the area under a velocity-time graph.
    \item Define and calculate acceleration as the rate of change of velocity.
    \item Solve one-dimensional motion problems involving constant acceleration using the kinematic equations.
\end{itemize}
\end{block}
\end{frame}

\section{Graphical Analysis of Motion}

\begin{frame}[allowframebreaks]
\frametitle{Position vs. Time (P-T) Graphs}
\begin{block}{What P-T Graphs Show}
A position vs. time graph plots an object's position on the vertical axis against time on the horizontal axis. It provides a visual representation of an object's motion.
\end{block}

\begin{block}{The Meaning of the Slope}
The most important feature of a position-time graph is its \alert{slope}.
\begin{itemize}
    \item \textbf{Slope = Velocity} ($ \text{slope} = \frac{\Delta \text{position}}{\Delta \text{time}} = \text{velocity} $)
    \item A \alert{steeper slope} indicates a \alert{higher speed}.
    \item A \alert{positive slope} indicates motion in the \alert{positive direction}.
    \item A \alert{negative slope} indicates motion in the \alert{negative direction}.
    \item A \alert{zero slope} (horizontal line) means the object is \alert{at rest}.
    \item A \alert{curved line} indicates that the velocity is changing, meaning the object is \alert{accelerating}.
\end{itemize}
\end{block}
\end{frame}

\begin{frame}[allowframebreaks]
\frametitle{Concept Visualization: P-T Graphs (Context)}
\begin{block}{Visualizing Motion}
The following graph illustrates how different types of motion appear on a position vs. time graph.

Pay close attention to the slope of each line and what it tells us about the object's velocity.
\end{block}
\end{frame}

\begin{frame}[allowframebreaks]
\frametitle{Concept Visualization: P-T Graphs}
\begin{figure}
\centering
\begin{tikzpicture}
\begin{axis}[
    width=\textwidth,
    height=0.7\textheight,
    xlabel={Time (s)},
    ylabel={Position (m)},
    xmin=0, xmax=6,
    ymin=-5, ymax=20,
    axis lines=center,
    label style={at={(ticklabel cs:1.05)},anchor=west},
    legend pos=outer north east,
    ]
    % Constant Positive Velocity
    \addplot[ds9blue, very thick, domain=0:5] {2*x + 5} node[pos=0.8, anchor=south west, sloped] {Constant +V};
    % Zero Velocity
    \addplot[green, very thick, domain=0:5] {18} node[pos=0.5, above] {Zero V};
    % Constant Negative Velocity
    \addplot[ds9red, very thick, domain=0:5] {-3*x + 15} node[pos=0.7, anchor=north west, sloped] {Constant -V};
    % Accelerating
    \addplot[orange, very thick, domain=0:4.5] {x^2 - 4} node[pos=0.8, anchor=west] {Accelerating};
\end{axis}
\end{tikzpicture}
\caption{Different motions on a P-T graph.}
\end{figure}
\end{frame}

\begin{frame}[allowframebreaks]
\frametitle{Velocity vs. Time (V-T) Graphs}
\begin{block}{What V-T Graphs Show}
A velocity vs. time graph plots an object's instantaneous velocity on the vertical axis against time on the horizontal axis.
\end{block}

\begin{block}{Key Features of V-T Graphs}
V-T graphs give us two crucial pieces of information about an object's motion:
\begin{enumerate}
    \item \textbf{Slope = Acceleration}
    \begin{itemize}
        \item The slope of the line reveals the object's acceleration ($ a = \frac{\Delta v}{\Delta t} $).
        \item A positive slope means positive acceleration; a negative slope means negative acceleration.
        \item A zero slope (horizontal line) means zero acceleration (constant velocity).
    \end{itemize}
    \item \textbf{Area = Displacement}
    \begin{itemize}
        \item The area between the graph line and the time axis is equal to the object's displacement ($ \Delta d = v \cdot t $).
    \end{itemize}
\end{enumerate}
\end{block}
\end{frame}

\begin{frame}[allowframebreaks]
\frametitle{Concept Visualization: V-T Graphs (Context)}
\begin{block}{Visualizing Acceleration and Displacement}
The next graph shows the motion of an object with constant positive acceleration.

We will use this single graph to visualize both its acceleration (from the slope) and its displacement (from the area under the curve).
\end{block}
\end{frame}

\begin{frame}[allowframebreaks]
\frametitle{Concept Visualization: V-T Graphs}
\begin{figure}
\centering
\begin{tikzpicture}
\begin{axis}[
    width=\textwidth,
    height=0.7\textheight,
    xlabel={Time (s)},
    ylabel={Velocity (m/s)},
    xmin=0, xmax=6,
    ymin=0, ymax=12,
    axis lines=left,
    label style={at={(ticklabel cs:1.05)},anchor=south},
    ]
    % Data plot
    \addplot[ds9blue, very thick] coordinates {(0, 2) (5, 10)}
    node[pos=0.6, sloped, above=2pt] {\textbf{Slope = Acceleration}};

    % Shaded Area for Displacement
    \addplot[fill=ds9blue!20, opacity=0.4, forget plot] coordinates {(0, 0) (4, 8.4) (4, 0)} -- cycle;
    
    \node at (axis cs:2, 3) {\textbf{Area = Displacement}};
\end{axis}
\end{tikzpicture}
\caption{Slope and Area on a V-T Graph.}
\end{figure}
\end{frame}

\section{Equations of Motion}

\begin{frame}[allowframebreaks]
\frametitle{Acceleration}
\begin{block}{Definition}
Acceleration is the rate at which an object's velocity changes over time. It is a vector quantity, meaning it has both magnitude and direction.
\end{block}

\begin{block}{Calculating Average Acceleration}
Average acceleration ($a$) is calculated by dividing the change in velocity ($\Delta v$) by the time interval ($\Delta t$).
\begin{equation*}
a = \frac{\Delta v}{\Delta t} = \frac{v_f - v_o}{t_f - t_o}
\end{equation*}
\begin{itemize}
    \item The standard unit for acceleration is \alert{meters per second squared (m/s²)}.
    \item \alert{Positive acceleration} means the velocity is becoming more positive.
    \item \alert{Negative acceleration} (or deceleration) means the velocity is becoming more negative.
\end{itemize}
\end{block}
\end{frame}

\begin{frame}[allowframebreaks]
\frametitle{The Kinematic Equations}
\begin{block}{For Motion with Constant Acceleration}
These equations relate displacement, velocity, acceleration, and time. They are the essential tools for solving problems where acceleration is constant.

\begin{itemize}
    \item[] \textbf{Velocity from time:} \hspace{1.1cm} $ v_f = v_o + at $
    \item[] \textbf{Displacement from time:} \quad $ d = d_o + v_o t + \frac{1}{2}at^2 $
    \item[] \textbf{Velocity from displacement:} $ v_f^2 = v_o^2 + 2a(d - d_o) $
    \item[] \textbf{Average velocity:} \hspace{1.4cm} $ \bar{v} = \frac{v_o + v_f}{2} $
\end{itemize}
\end{block}

\begin{block}{Variable Definitions}
\begin{itemize}
    \item $d$: final position (m)
    \item $d_o$: initial position (m)
    \item $v_f$: final velocity (m/s)
    \item $v_o$: initial velocity (m/s)
    \item $a$: acceleration (m/s²)
    \item $t$: time interval (s)
\end{itemize}
\end{block}
\end{frame}

\section{Problem Solving Workshop}

\begin{frame}[allowframebreaks]
\frametitle{"I Do" Example: Calculating Final Velocity}
\begin{block}{Problem}
A drag racer starts from rest and accelerates uniformly at 25 m/s² for 4.0 s. What is its final velocity?
\end{block}

\begin{alertblock}{G - Givens}
\begin{itemize}
    \item Initial velocity ($v_o$) = 0 m/s ("starts from rest")
    \item Acceleration ($a$) = 25 m/s²
    \item Time ($t$) = 4.0 s
\end{itemize}
\end{alertblock}

\begin{alertblock}{U - Unknown}
\begin{itemize}
    \item Final velocity ($v_f$) = ?
\end{itemize}
\end{alertblock}

\begin{alertblock}{E - Equation}
We need an equation that relates $v_f$, $v_o$, $a$, and $t$.
\begin{equation*}
v_f = v_o + at
\end{equation*}
\end{alertblock}

\begin{alertblock}{S - Substitute}
\begin{equation*}
v_f = 0 \text{ m/s} + (25 \text{ m/s}^2)(4.0 \text{ s})
\end{equation*}
\end{alertblock}

\begin{alertblock}{S - Solve}
\begin{equation*}
v_f = 100 \text{ m/s}
\end{equation*}
The racer's final velocity is 100 m/s.
\end{alertblock}
\end{frame}

\begin{frame}[allowframebreaks]
\frametitle{"We Do" Example: Calculating Acceleration}
\begin{block}{Problem}
A car is traveling at 15 m/s. It accelerates uniformly to 35 m/s over 5.0 s. What is its acceleration?
\end{block}

\begin{alertblock}{G - Givens}
\begin{itemize}
    \item Initial velocity ($v_o$) = 15 m/s
    \item Final velocity ($v_f$) = 35 m/s
    \item Time ($t$) = 5.0 s
\end{itemize}
\end{alertblock}

\begin{alertblock}{U - Unknown}
\begin{itemize}
    \item Acceleration ($a$) = ?
\end{itemize}
\end{alertblock}

\begin{alertblock}{E - Equation}
We can rearrange $v_f = v_o + at$ to solve for $a$.
\begin{equation*}
a = \frac{v_f - v_o}{t}
\end{equation*}
\end{alertblock}

\begin{alertblock}{S - Substitute}
\begin{itemize}
    \item \textit{Let's substitute the values together...}
\end{itemize}
\end{alertblock}

\begin{alertblock}{S - Solve}
\begin{itemize}
    \item \textit{What is the final answer with units?}
\end{itemize}
\end{alertblock}
\end{frame}

\begin{frame}[allowframebreaks]
\frametitle{"You Do" Example: Calculating Displacement}
\begin{block}{Problem}
A sprinter accelerates from rest at a rate of 5.0 m/s². How far does she travel in the first 3.0 seconds?
\end{block}

\begin{block}{Your Turn}
Use the GUESS method to solve the problem on your own.
\begin{itemize}
    \item \textbf{G}ivens: What information are you given?
    \item \textbf{U}nknown: What do you need to find?
    \item \textbf{E}quation: Which kinematic equation should you use?
    \item \textbf{S}ubstitute: Plug the known values into your equation.
    \item \textbf{S}olve: Calculate the final answer.
\end{itemize}
\end{block}
\end{frame}

\section{Conclusion}

\begin{frame}[allowframebreaks]
\frametitle{Summary}
\begin{block}{Key Takeaways}
\begin{itemize}
    \item \textbf{Graphs tell a story:} Motion can be clearly described and analyzed using position-time and velocity-time graphs.
    \item \textbf{Position-Time Graphs:}
    \begin{itemize}
        \item Slope is velocity.
    \end{itemize}
    \item \textbf{Velocity-Time Graphs:}
    \begin{itemize}
        \item Slope is acceleration.
        \item Area under the curve is displacement.
    \end{itemize}
    \item \textbf{Kinematic Equations:} A powerful set of tools for solving problems involving constant acceleration. Always use a structured method like GUESS to solve physics problems.
\end{itemize}
\end{block}
\end{frame}

\end{document}