\documentclass{beamer}
% Use DS9 global theme (includes pgfplots for visualization)
\usepackage{../../../../latex-beamer/shared/templates/ds9_theme}


% Title page configuration
\title[Acceleration and Kinematics]{PHYS11 CH:3.1 and 3.2}
\subtitle{Understanding Acceleration and the Kinematic Equations}
\author[Mr. Gullo]{Mr. Gullo}
\date[Sept 2025]{September 12, 2025}

\begin{document}
\frame{\titlepage}

\begin{frame}
\frametitle{Learning Objectives}
By the end of this lesson, you will be able to:
\pause
\begin{itemize}
    \item Explain acceleration and determine its direction and magnitude in one dimension.
    \pause
    \item Analyze motion in one dimension using kinematic equations.
    \pause
    \item Explain the kinematic equations related to acceleration and illustrate them with graphs.
    \pause
    \item Apply kinematic equations and related graphs to solve problems involving acceleration.
\end{itemize}
\end{frame}

\section{Defining Acceleration}

\begin{frame}
\frametitle{Key Concepts: What is Acceleration?}
\begin{block}{Definition}
    Acceleration is the rate of change of \textbf{velocity}. It is a measure of how quickly velocity changes.
\end{block}
\pause
\begin{itemize}
    \item \textbf{Vector Quantity:} Acceleration has both magnitude (a size) and direction.
    \pause
    \item \textbf{Units:} The SI unit for acceleration is meters per second squared ($\text{m/s}^2$).
    \pause
    \item \textbf{Average Acceleration Formula:}
    \[ \bar{a} = \frac{\Delta v}{\Delta t} = \frac{v_f - v_0}{t_f - t_0} \]
    Where $v_f$ is the final velocity and $v_0$ is the initial velocity.
\end{itemize}
\end{frame}

\begin{frame}
\frametitle{Key Concepts: Direction of Acceleration}
The direction of acceleration determines if an object is speeding up or slowing down.
\pause
\begin{columns}[T]
    \column{0.48\textwidth}
    \begin{alertblock}{Speeding Up}
        The acceleration vector points in the \textbf{same direction} as the velocity vector.
    \end{alertblock}
    
    \column{0.48\textwidth}
    \begin{alertblock}{Slowing Down}
        The acceleration vector points in the \textbf{opposite direction} of the velocity vector. This is often called \textit{deceleration}.
    \end{alertblock}
\end{columns}
\pause
\begin{block}{Important Note}
    Negative acceleration does \textit{not} automatically mean slowing down.
    \begin{itemize}
        \item If velocity is negative and acceleration is negative, the object is \textbf{speeding up} in the negative direction!
    \end{itemize}
\end{block}
\end{frame}

\begin{frame}
\frametitle{Context: Visualizing Acceleration Vectors}
To better understand how acceleration affects motion, we can visualize the velocity ($\vec{v}$) and acceleration ($\vec{a}$) vectors for a moving object.
\pause
\begin{itemize}
    \item When a car speeds up, its velocity increases. The acceleration vector points in the same direction as the velocity.
    \pause
    \item When the car slows down, its velocity decreases. The acceleration vector points opposite to the velocity.
\end{itemize}
\pause
The next slide shows a diagram of this concept.
\end{frame}

\begin{frame}
\frametitle{Concept Visualization: Acceleration Vectors}
\begin{alertblock}{[Diagram based on Figure 3.2]}
A diagram showing a car in two scenarios:
\begin{itemize}
    \item \textbf{(a) Speeding Up:} A velocity vector $\vec{v}$ points to the right. A smaller acceleration vector $\vec{a}$ also points to the right, in the same direction.
    \pause
    \item \textbf{(b) Slowing Down:} A velocity vector $\vec{v}$ points to the right. A smaller acceleration vector $\vec{a}$ points to the left, in the opposite direction.
\end{itemize}
This visual shows that the relative direction of acceleration and velocity determines the change in speed.
\end{alertblock}
\end{frame}

\section{Kinematic Equations for Constant Acceleration}

\begin{frame}
\frametitle{Essential Equations: The Kinematics}
These five equations describe motion for an object moving with \textbf{constant acceleration}. They are the main tools for solving motion problems.
\pause
\begin{enumerate}
    \item $d = d_0 + \bar{v}t$ \quad (when acceleration is zero)
    \pause
    \item $\bar{v} = \frac{v_0 + v_f}{2}$
    \pause
    \item $v_f = v_0 + at$
    \pause
    \item $d = d_0 + v_0 t + \frac{1}{2}at^2$
    \pause
    \item $v_f^2 = v_0^2 + 2a(d - d_0)$
\end{enumerate}
\pause
\begin{block}{Variables}
\begin{tabular}{ll}
    $d$ & final position (m) \\
    $d_0$ & initial position (m) \\
    $v_f$ & final velocity (m/s) \\
\end{tabular}
\quad
\begin{tabular}{ll}
    $v_0$ & initial velocity (m/s) \\
    $a$ & constant acceleration (m/s$^2$) \\
    $t$ & time (s) \\
\end{tabular}
\end{block}
\end{frame}

\section{Graphical Analysis of Motion}

\begin{frame}
\frametitle{Context: Graphs of Motion Under Gravity}
Let's consider an object thrown straight up, which then falls back down. Its motion is governed by the constant acceleration of gravity, $g = -9.80 \, \text{m/s}^2$.
\pause
\begin{itemize}
    \item How will its \textbf{position} change over time?
    \pause
    \item How will its \textbf{velocity} change over time?
    \pause
    \item What will its \textbf{acceleration} look like over time?
\end{itemize}
\pause
The next slide shows the three graphs that describe this motion.
\end{frame}

\begin{frame}
\frametitle{Concept Visualization: Motion Under Gravity}
\begin{alertblock}{[Graphs based on Figure 3.13]}
Three graphs showing the motion of a rock thrown upward:
\begin{itemize}
    \item \textbf{Position vs. Time:} An inverted parabola. The rock goes up, reaches a peak, and comes back down.
    \pause
    \item \textbf{Velocity vs. Time:} A straight line with a negative slope. The velocity starts positive, decreases to zero at the peak, and becomes negative as it falls. The slope of this line is the acceleration due to gravity.
    \pause
    \item \textbf{Acceleration vs. Time:} A horizontal line at $-9.80 \, \text{m/s}^2$. This shows that the acceleration is constant throughout the motion.
\end{itemize}
\end{alertblock}
\end{frame}

\section{Problem Solving with GUESS}

\begin{frame}
\frametitle{I Do: Accelerating Subway Train - Problem Setup}
\framesubtitle{Problem based on Ch. 3, page 6}
\begin{block}{Problem}
A subway train accelerates from rest to 30.0 km/h in 20.0 s. What is its average acceleration during that time interval?
\end{block}
\pause
\begin{columns}[T]
\column{0.48\textwidth}
\begin{block}{G - Givens}
\begin{itemize}
    \item $v_0 = 0 \, \text{m/s}$ (from rest)
    \item $v_f = 30.0 \, \text{km/h}$
    \item $\Delta t = 20.0 \, \text{s}$
\end{itemize}
\end{block}
\pause
\column{0.48\textwidth}
\begin{block}{U - Unknown}
\begin{itemize}
    \item $\bar{a} = ?$ (in m/s$^2$)
\end{itemize}
\end{block}
\end{columns}
\end{frame}

\begin{frame}
\frametitle{I Do: Accelerating Subway Train - Equation}
\begin{columns}[T]
\column{0.48\textwidth}
\begin{block}{G - Givens}
\begin{itemize}
    \item $v_0 = 0 \, \text{m/s}$
    \item $v_f = 30.0 \, \text{km/h}$
    \item $\Delta t = 20.0 \, \text{s}$
\end{itemize}
\end{block}
\pause
\column{0.48\textwidth}
\begin{block}{U - Unknown}
\begin{itemize}
    \item $\bar{a} = ?$ (in m/s$^2$)
\end{itemize}
\end{block}
\end{columns}
\pause
\begin{columns}[T]
\column{0.48\textwidth}
\begin{block}{E - Equation}
\begin{itemize}
    \item Select: $\bar{a} = \frac{\Delta v}{\Delta t} = \frac{v_f - v_0}{\Delta t}$
    \item No rearrangement needed.
    \item \alert{First, must convert units!}
\end{itemize}
\end{block}
\end{columns}
\end{frame}

\begin{frame}
\frametitle{I Do: Accelerating Subway Train - Solution}
\begin{block}{S - Substitute (with unit conversion)}
\begin{itemize}
    \item Convert km/h to m/s:
    \[ v_f = 30.0 \, \frac{\text{km}}{\text{h}} \times \frac{1000 \, \text{m}}{1 \, \text{km}} \times \frac{1 \, \text{h}}{3600 \, \text{s}} = 8.333 \, \text{m/s} \]
    \pause
    \item Plug values into the equation:
    \[ \bar{a} = \frac{8.333 \, \text{m/s} - 0 \, \text{m/s}}{20.0 \, \text{s}} \]
\end{itemize}
\end{block}
\pause
\begin{block}{S - Solve}
\begin{itemize}
    \item Calculate the final answer:
    \[ \bar{a} = \frac{8.333 \, \text{m/s}}{20.0 \, \text{s}} = 0.41665 \, \text{m/s}^2 \]
    \pause
    \item Apply sig figs (3 sig figs from givens):
    \item \boxed{\bar{a} = 0.417 \, \text{m/s}^2}
\end{itemize}
\end{block}
\end{frame}

\begin{frame}
\frametitle{We Do: Acceleration of a Dragster}
\framesubtitle{Problem based on Ch. 3, page 22}
\begin{block}{Problem}
A dragster accelerates from rest at a constant $26.0 \, \text{m/s}^2$ for a quarter mile (402 m). What is the final velocity of the dragster?
\end{block}
\pause
\begin{columns}[T]
\column{0.48\textwidth}
\begin{block}{G - Givens}
\begin{itemize}
    \item $v_0 = 0 \, \text{m/s}$ (from rest)
    \item $a = 26.0 \, \text{m/s}^2$
    \item $\Delta d = 402 \, \text{m}$
\end{itemize}
\end{block}
\pause
\column{0.48\textwidth}
\begin{block}{U - Unknown}
\begin{itemize}
    \item $v_f = ?$
\end{itemize}
\end{block}
\end{columns}
\pause
\begin{alertblock}{Question: Which kinematic equation should we use? (Hint: We don't know time)}
\end{alertblock}
\end{frame}

\begin{frame}
\frametitle{We Do: Acceleration of a Dragster - Solution}
\begin{block}{E - Equation}
The best equation is the one without time ($t$):
\[ v_f^2 = v_0^2 + 2a\Delta d \]
\pause
This equation is almost ready. How do we isolate $v_f$?
\begin{itemize}
    \item \alert{Take the square root of both sides!}
    \[ v_f = \sqrt{v_0^2 + 2a\Delta d} \]
\end{itemize}
\end{block}
\pause
\begin{block}{S - Substitute}
Let's plug in the values:
\[ v_f = \sqrt{(0 \, \text{m/s})^2 + 2(26.0 \, \text{m/s}^2)(402 \, \text{m})} \]
\end{block}
\pause
\begin{block}{S - Solve}
What is the final velocity?
\[ v_f = \sqrt{20904 \, \text{m}^2/\text{s}^2} = \ ? \]
\pause
\boxed{v_f \approx 145 \, \text{m/s}}
\end{block}
\end{frame}

\begin{frame}
\frametitle{You Do: Olympic Sprinter}
\framesubtitle{Problem based on Ch. 3, Problem 7, page 24}
\begin{block}{Problem}
An Olympic-class sprinter starts a race with an acceleration of $4.50 \, \text{m/s}^2$. Assuming she can maintain that acceleration, what is her speed 2.40 s later?
\end{block}
\vfill
\begin{alertblock}{Your Turn}
Use the GUESS method to solve this problem on your own.
\begin{itemize}
    \item What are the Givens? (Note: "starts a race" implies $v_0 = 0$)
    \item What is the Unknown?
    \item Which Equation will you use?
    \item Substitute and Solve for the final answer.
\end{itemize}
\end{alertblock}
\end{frame}

\begin{frame}
\frametitle{Reading Homework}
To ensure you have a strong foundation for the concepts we discussed today, please review the following sections from Chapter 2:
\begin{itemize}
    \item \textbf{Section 2.1: Relative Motion, Distance, and Displacement}
    \begin{itemize}
        \item Focus on the difference between scalar (distance) and vector (displacement) quantities.
    \end{itemize}
    \pause
    \item \textbf{Section 2.2: Speed and Velocity}
    \begin{itemize}
        \item Understand how speed relates to distance and velocity relates to displacement.
    \end{itemize}
    \pause
    \item \textbf{Section 2.3: Position vs. Time Graphs}
    \begin{itemize}
        \item Review how the slope of a position-time graph gives velocity.
    \end{itemize}
    \pause
    \item \textbf{Section 2.4: Velocity vs. Time Graphs}
    \begin{itemize}
        \item This will prepare you for our next lesson on graphical analysis.
    \end{itemize}
\end{itemize}
\end{frame}

\begin{frame}
\frametitle{Summary}
\begin{itemize}
    \item \textbf{Acceleration} is the rate of change of velocity ($\vec{a} = \Delta\vec{v} / \Delta t$) and is a vector.
    \pause
    \item The direction of acceleration relative to velocity determines if an object \textbf{speeds up} (same direction) or \textbf{slows down} (opposite directions).
    \pause
    \item For motion with \textbf{constant acceleration}, we use the kinematic equations to solve for unknown quantities like displacement, velocity, or time.
    \pause
    \item \textbf{Motion graphs} provide a powerful visual tool:
    \begin{itemize}
        \item The slope of a velocity-time graph is \alert{acceleration}.
        \item The area under a velocity-time graph is \alert{displacement}.
    \end{itemize}
\end{itemize}
\end{frame}

\end{document}