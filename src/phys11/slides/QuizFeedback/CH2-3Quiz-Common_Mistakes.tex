\documentclass[12pt]{article}
\usepackage{amsmath}
\usepackage{amssymb}
\usepackage{geometry}
\usepackage{fancyhdr}

\geometry{margin=1in}
\pagestyle{fancy}
\fancyhf{}
\rhead{Quiz 1 Solution Guide}
\lhead{Physics 12}
\cfoot{\thepage}

\title{CH 1-2 Quiz: Solution Guide for Common Mistakes}
\author{Physics 12}
\date{}

\begin{document}
\maketitle

\section{Target Questions for Review}
These are the 3 questions that showed the lowest performance on Quiz 1. Review these solutions carefully to understand the correct approach.

\subsection{Essential Equations}
\begin{align}
\text{Net displacement} &= \text{Area under velocity-time graph} \\
v &= \frac{\Delta x}{\Delta t} \quad \text{(constant velocity)} \\
a &= \frac{\Delta v}{\Delta t} \quad \text{(constant acceleration)} \\
\text{Slope of } v\text{-}t \text{ graph} &= \text{acceleration}
\end{align}

\section{Problem 1: Net Displacement from Velocity-Time Graph}
\textbf{Common Error:} Students calculated individual areas incorrectly or forgot to account for the triangular sections.

\textbf{G - Given:} Velocity-time graph with rectangular and triangular sections
\textbf{U - Unknown:} Net displacement
\textbf{E - Equation:} Net displacement = Area under $v$-$t$ graph

\textbf{S - Substitute and Solve:}
The area consists of a rectangle and triangle:
\begin{align}
\text{Rectangle area} &= \text{base} \times \text{height} = 30 \text{ s} \times 18 \text{ m/s} = 540 \text{ m} \\
\text{Triangle area} &= \frac{1}{2} \times \text{base} \times \text{height} = \frac{1}{2} \times 150 \text{ s} \times 30 \text{ m/s} = 2250 \text{ m} \\
\text{Total displacement} &= 540 + 2250 = 2790 \text{ m}
\end{align}
\boxed{2790 \text{ m}}

\textbf{Key Insight:} The area under a velocity-time graph always gives displacement, regardless of the shape. Break complex shapes into simple geometric forms.

\section{Problem 2: Reference Frames and Relative Motion}
\textbf{Common Error:} Students thought motion complexity determines reference frame validity, rather than understanding the fundamental relativity of motion.

\textbf{G - Given:} A train moving along a track
\textbf{U - Unknown:} Whether there is a single correct reference frame
\textbf{E - Equation:} Motion is relative to the chosen reference frame

\textbf{S - Substitute and Solve:}
Motion can be described from any reference frame:
\begin{itemize}
\item From the ground: train has velocity $v$
\item From the train: ground has velocity $-v$
\item From another moving object: train has velocity $v - v_{\text{object}}$
\end{itemize}
All descriptions are equally valid. There is no single "correct" frame.
\boxed{\text{Answer: B - Motion is relative}}

\textbf{Key Insight:} Motion is always described relative to a chosen reference frame. No single reference frame is more "correct" than another.

\section{Problem 3: Velocity-Time Graph Shape and Acceleration}
\textbf{Common Error:} Students confused the relationship between velocity and acceleration, thinking velocity determines graph shape rather than acceleration.

\textbf{G - Given:} Velocity versus time graph analysis
\textbf{U - Unknown:} What determines straight vs curved lines
\textbf{E - Equation:} Slope of $v$-$t$ graph = acceleration

\textbf{S - Substitute and Solve:}
For velocity-time graphs:
\begin{align}
\text{Straight line} &\Rightarrow \text{constant slope} \Rightarrow \text{constant acceleration} \\
\text{Curved line} &\Rightarrow \text{changing slope} \Rightarrow \text{changing acceleration}
\end{align}
The velocity value itself does not determine the graph shape.
\boxed{\text{Answer: C - Acceleration determines graph shape}}

\textbf{Key Insight:} The slope of a velocity-time graph represents acceleration. Constant acceleration produces straight lines, changing acceleration produces curves.

\section{Problem 4: Skydiver Acceleration Graph}
\textbf{Common Error:} Students forgot that acceleration decreases as air resistance increases, eventually reaching zero at terminal velocity.

\textbf{G - Given:} Skydiver affected by gravity and air resistance reaching terminal speed
\textbf{U - Unknown:} Shape of acceleration vs time graph
\textbf{E - Equation:} $a = g - \frac{F_{\text{air}}}{m}$ where $F_{\text{air}}$ increases with speed

\textbf{S - Substitute and Solve:}
Initially: $a = g = 9.8 \text{ m/s}^2$ (maximum downward acceleration)
As speed increases: Air resistance increases, so net acceleration decreases
At terminal velocity: $F_{\text{air}} = mg$, so $a = 0$

The graph starts at $g$ and decreases exponentially toward zero.
\boxed{\text{Answer: B - Nonzero y-intercept, downward slope leveling at zero}}

\textbf{Key Insight:} Net acceleration equals gravitational acceleration minus air resistance effects. As speed increases, air resistance grows, reducing net acceleration toward zero.

\section{Quick Review Tips}
\begin{itemize}
\item \textbf{Displacement from graphs:} Always calculate area under velocity-time curves, breaking into simple geometric shapes
\item \textbf{Reference frames:} Motion descriptions depend on chosen reference frame; no single frame is "correct"
\item \textbf{Graph interpretation:} Slope of position-time gives velocity; slope of velocity-time gives acceleration
\item \textbf{Terminal velocity:} Occurs when air resistance equals gravitational force, making net acceleration zero
\end{itemize}

\end{document}