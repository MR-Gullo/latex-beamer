\documentclass[12pt]{article}
\usepackage{amsmath}
\usepackage{amssymb} 
\usepackage{graphicx}
\usepackage{fancyhdr}
\usepackage{color}
\usepackage{hyperref}
\usepackage{float}
\usepackage{enumitem}
\usepackage[margin=1in]{geometry}
\usepackage{titlesec}

\titleformat{\section}
{\normalfont\Large\bfseries\color{blue}}{\thesection}{1em}{}

\pagestyle{fancy}
\fancyhead[L]{Physics Chapter 4}
\fancyhead[R]{Forces and Newton's Laws of Motion}
\fancyfoot[C]{\thepage}

\begin{document}

\begin{center}
{\LARGE\bfseries Physics Video Analysis Assignment\\[0.5cm]}
{\large\bfseries Analysis of Forces and Newton's Laws in Real-World Applications}\\[1cm]
\end{center}

\section*{Purpose}
To analyze a real-world video using physics concepts from Chapter 4, demonstrating understanding through precise application of formulas and principles with explicit textbook references.

\section{Group Structure and Roles}
This is a group project requiring 4-6 team members. Each member must contribute to all aspects, but will have primary responsibility for their assigned role:

\subsection{Project Leader (1 person)}
\begin{itemize}
\item Coordinates team meetings and timeline
\item Ensures all references are properly cited
\item Reviews final submission for completeness
\item Submits final work
\item Maintains communication with instructor
\end{itemize}

\subsection{Physics Analyst (1 person)}
\begin{itemize}
\item Leads mathematical analysis
\item Verifies all force calculations
\item Ensures proper use of Newton's Laws
\item Checks units and vector directions
\end{itemize}

\subsection{Technical Illustrator (1 person)}
\begin{itemize}
\item Creates all required free-body diagrams (Fig 4.2, 4.3)
\item Ensures proper labeling of forces and vectors
\item Maintains consistent vector notation
\item Produces clear, professional diagrams
\end{itemize}

\subsection{Documentation Specialist (1 person)}
\begin{itemize}
\item Manages page and equation references
\item Writes explanations and interpretations
\item Ensures clear documentation of process
\item Maintains organized project files
\end{itemize}

\section{Required Materials}
\begin{itemize}
\item Chapter 4 textbook (pages 116-142)
\item Selected internet video
\item Screenshot capability
\item Drawing tools for force diagrams
\item Shared workspace for collaboration
\end{itemize}

\section{Core Formula Reference}
All formulas must be cited with page and equation numbers from the textbook.

\subsection{Newton's First Law}
\begin{equation}
F_{net} = 0 \text{ or } \Sigma F = 0 \quad \text{(p.118, Eq. 4.1)}
\end{equation}

\subsection{Newton's Second Law}
\begin{equation}
F_{net} = ma \text{ or } \Sigma F = ma \quad \text{(p.122, Eq. 4.2)}
\end{equation}

\subsection{Force of Friction}
\begin{equation}
f = \mu N \quad \text{(p.119, Eq. 4.3)}
\end{equation}

\subsection{Normal Force}
\begin{equation}
N = mg \quad \text{(p.129, Eq. 4.17)}
\end{equation}

\section{Assignment Requirements}

\subsection{Video Selection \& Documentation}
\begin{itemize}
\item Include video URL/source
\item Screenshot of analyzed frame
\item Timestamp of analyzed moment
\item Brief description of forces involved
\end{itemize}

\subsection{Required Analysis Components}
Each section must include explicit textbook references:

\subsubsection{Force Analysis}
\begin{itemize}
\item Complete free-body diagram (p.116, Fig 4.2)
\item Vector notation for all forces
\item Net force calculations
\item Classification of forces (contact vs field forces)
\end{itemize}

\subsubsection{Newton's Laws Analysis}
\begin{itemize}
\item Application of First Law (equilibrium conditions)
\item Second Law calculations
\item Third Law force pairs identification
\item System definition and external forces
\end{itemize}

\subsubsection{Calculations \& Results}
\begin{itemize}
\item Mass and weight determinations
\item Acceleration calculations
\item Force component analysis
\item Complete step-by-step solutions
\end{itemize}

\section{Documentation Requirements}
Each analysis section must include:
\begin{enumerate}
\item Concept explanation (with page reference)
\item Relevant formula (with equation number)
\item Variable identification
\item Step-by-step calculations
\item Units analysis
\item Physical interpretation
\end{enumerate}

\section{Citation Format}
Example: "Using the work-energy theorem (p.274, Eq. 7.10), we calculate..."

\section{Group Presentation Requirements}
Each group will prepare and deliver a 5-10 minute presentation analyzing their video. The presentation must include:
\subsubsection{Required Slides (Minimum 5)}
\begin{enumerate}[label=\textbf{Slide \arabic*.}]
\item \textbf{Introduction}
\begin{itemize}
\item Title and group members
\item Video source and timestamp
\item Preview of key physics concepts to be analyzed
\item Physical scenario overview and relevance
\end{itemize}
\item \textbf{Physical Analysis}
\begin{itemize}
\item Professional technical diagrams
\item Clear labeling of all relevant quantities
\item System/boundary definitions
\item Key variable identification and relationships
\end{itemize}
\item \textbf{Theory Application}
\begin{itemize}
\item Application of relevant physical laws
\item Key equation implementations
\item Theoretical predictions
\item Textbook references and citations
\end{itemize}
\item \textbf{Calculations and Results}
\begin{itemize}
\item Step-by-step mathematical analysis
\item Quantitative determinations
\item Units and significant figures
\item Comparison of theory vs. observation
\end{itemize}
\item \textbf{Conclusions}
\begin{itemize}
\item Summary of key findings
\item Real-world applications
\item Sources of uncertainty
\item Connection to textbook principles
\end{itemize}
\end{enumerate}
\subsubsection{Presentation Requirements}
\begin{itemize}
\item Professional slide formatting
\item Clear, readable diagrams and equations
\item Equal participation from all members
\item Proper citation of textbook concepts
\item Prepared for peer questions
\end{itemize}

\section{Grading Rubric}
\begin{table}[h]
\centering
\begin{tabular}{|p{3cm}|p{8cm}|p{2cm}|p{2cm}|}
\hline
\textbf{Category} & \textbf{Criteria} & \textbf{Points} & \textbf{Score} \\
\hline
\textbf{Physics Analysis} & 
\begin{itemize}[leftmargin=*]
\item Correct application of Chapter 7 concepts
\item All calculations complete and accurate
\item Proper equation selection with references
\item Clear step-by-step problem solving
\end{itemize} & 30 & \\
\hline
\textbf{Documentation} & 
\begin{itemize}[leftmargin=*]
\item All textbook page numbers cited
\item Equation numbers referenced
\item Clear variable definitions
\item Professional presentation
\end{itemize} & 20 & \\
\hline
\textbf{Diagrams} & 
\begin{itemize}[leftmargin=*]
\item Complete free body diagrams (p.271, Fig 7.2)
\item Force vs. displacement graphs (p.274, Fig 7.3)
\item System diagrams with all forces labeled
\item Vector notations properly shown
\end{itemize} & 20 & \\
\hline
\textbf{Technical Execution} & 
\begin{itemize}[leftmargin=*]
\item Correct units throughout
\item Proper significant figures
\item Logical solution flow
\item Clear conclusions
\end{itemize} & 15 & \\
\hline
\textbf{Group Participation} & 
\begin{itemize}[leftmargin=*]
\item Active contribution to team meetings
\item Completion of assigned role tasks
\item Support of other team members
\item Meeting of deadlines
\end{itemize} & 15 & \\
\hline
\textbf{Total} & & 100 & \\
\hline
\end{tabular}
\caption{Video Analysis Group Project Grading Rubric}
\end{table}

\subsection{Score Interpretation}
\begin{itemize}
\item 90-100: Excellent - Demonstrates complete mastery of concepts and applications
\item 80-89: Good - Shows solid understanding with minor errors or omissions
\item 70-79: Satisfactory - Basic understanding present but needs improvement
\item 60-69: Needs Improvement - Significant gaps in understanding or application
\item Below 60: Unsatisfactory - Major deficiencies in understanding and execution
\end{itemize}

\section*{Important Notes}
\begin{itemize}
\item All equations must include textbook equation numbers
\item All concepts must include page number references
\item Direct quotes must include quotation marks and page numbers
\item Calculations must show complete work
\item Units must be carried through all calculations
\end{itemize}


\end{document}