\documentclass{beamer}
% Use DS9 global theme
\usepackage{../../../shared/templates/ds9_theme}

% Title page information
\title[Understanding Momentum]{PHYS11 CH8: Understanding Momentum}
\subtitle{From Everyday Motion to Conservation Laws}
\author[Mr. Gullo]{Mr. Gullo}
\date[]{}
\institute{}

\begin{document}

% Title frame
\begin{frame}
\titlepage
\end{frame}

% Opening Question Frame
\begin{frame}{Think About This...}
\begin{block}{Opening Scenario}
Why is it harder to stop...
\begin{itemize}
\item A heavy truck moving slowly, or
\item A light car moving quickly?
\end{itemize}
\end{block}
\begin{itemize}
\item This question introduces us to the concept of \textbf{momentum}
\item By the end of this lesson, you'll understand exactly why both situations are challenging!
\end{itemize}
\end{frame}

% Learning Objectives
\begin{frame}{Learning Objectives}
\begin{block}{By the end of this lesson, you will be able to:}
\begin{itemize}
\item \textbf{Explain} momentum using everyday examples
\item \textbf{Calculate} the momentum of moving objects
\item \textbf{Describe} how force and time relate to changing momentum
\item \textbf{Apply} conservation of momentum to real situations
\item \textbf{Analyze} different types of collisions
\end{itemize}
\end{block}
\end{frame}

% What is Momentum? Conceptual Introduction
\begin{frame}{Understanding Momentum: The Basics}
\begin{block}{Momentum: A Measure of Motion}
Think of momentum as an object's "motion strength"
\end{block}

\begin{itemize}
\item Like a moving bowling ball vs. a moving ping pong ball
\item Two factors determine momentum:
    \begin{itemize}
    \item How much stuff is moving (mass)
    \item How fast it's moving (velocity)
    \end{itemize}
\item More mass OR more velocity = more momentum
\end{itemize}

\begin{alertblock}{Key Point}
Momentum combines MASS and VELOCITY into a single measure of motion
\end{alertblock}
\end{frame}

% Mathematical Definition
\begin{frame}{The Mathematics of Momentum}
\begin{block}{Definition}
Momentum ($\vec{p}$) = mass × velocity
$$\vec{p} = m\vec{v}$$
\end{block}

\begin{columns}
\column{0.5\textwidth}
\textbf{Units:}
\begin{itemize}
\item Mass (kg)
\item Velocity (m/s)
\item Momentum (kg⋅m/s)
\end{itemize}

\column{0.5\textwidth}
\textbf{Remember:}
\begin{itemize}
\item Momentum is a vector
\item Direction matters!
\item Same direction as velocity
\end{itemize}
\end{columns}
\end{frame}

% Real-World Examples Frame
\begin{frame}{Momentum in Real Life}
\begin{block}{Sports Examples}
\begin{itemize}
\item Football player running (large mass, moderate velocity)
\item Baseball pitch (small mass, high velocity)
\item Ice skater gliding (medium mass, low velocity)
\end{itemize}
\end{block}

\begin{block}{Transportation Examples}
\begin{itemize}
\item Heavy truck at highway speed
\item Bicycle commuter
\item High-speed train
\end{itemize}
\end{block}
\end{frame}

% I Do: Worked Example
\begin{frame}{Example: Understanding Momentum (I Do)}
\begin{block}{Problem}
A 75 kg football player runs at 8 m/s. Calculate their momentum.
\end{block}

\pause

\begin{block}{Step-by-Step Solution}
1. Identify what we know:
   \begin{itemize}
   \item Mass (m) = 75 kg
   \item Velocity (v) = 8 m/s
   \end{itemize}

   \pause
   
2. Apply the momentum formula:
   $$\vec{p} = m\vec{v} = (75\text{ kg})(8\text{ m/s}) = 600\text{ kg⋅m/s}$$
\end{block}
\end{frame}

% We Do: Interactive Example
\begin{frame}{Let's Try Together (We Do)}
\begin{block}{Problem}
A 0.145 kg baseball is thrown at 40 m/s. Calculate:
\begin{itemize}
\item The ball's momentum
\item Compare it to the football player's momentum
\end{itemize}
\end{block}

\pause

\begin{block}{Solution Steps}
1. Calculate baseball momentum:
   $$\vec{p} = (0.145\text{ kg})(40\text{ m/s}) = 5.8\text{ kg⋅m/s}$$
   
2. Compare:
   \begin{itemize}
   \item Baseball: 5.8 kg⋅m/s
   \item Football player: 600 kg⋅m/s
   \end{itemize}
\end{block}
\end{frame}

% Impulse Introduction
\begin{frame}{Changing Momentum: Understanding Impulse}
\begin{block}{Key Concept: Impulse}
Impulse = Force × Time = Change in Momentum
$$F\Delta t = \Delta p$$
\end{block}

\begin{itemize}
\item Same effect can be achieved by:
    \begin{itemize}
    \item Large force for short time
    \item Small force for long time
    \end{itemize}
\item Examples:
    \begin{itemize}
    \item Catching a baseball (extend arms to increase time)
    \item Car airbags (increase collision time)
    \item Karate board break (large force, very short time)
    \end{itemize}
\end{itemize}
\end{frame}

% Conservation of Momentum Introduction
\begin{frame}{Conservation of Momentum}
\begin{block}{The Big Idea}
In an isolated system (no external forces), total momentum stays constant
\end{block}

\pause

\begin{columns}
\column{0.5\textwidth}
\textbf{Before Collision}
\begin{itemize}
\item Object 1 momentum
\item Object 2 momentum
\item Total = $p_1 + p_2$
\end{itemize}

\pause

\column{0.5\textwidth}
\textbf{After Collision}
\begin{itemize}
\item Object 1 new momentum
\item Object 2 new momentum
\item Total = $p'_1 + p'_2$
\end{itemize}
\end{columns}

\pause

\begin{alertblock}{Key Equation}
$$p_1 + p_2 = p'_1 + p'_2$$
\end{alertblock}
\end{frame}

% Types of Collisions
\begin{frame}{Understanding Collisions}
\begin{columns}
\column{0.5\textwidth}
\textbf{Elastic Collisions}
\begin{itemize}
\item Objects bounce apart
\item Kinetic energy preserved
\item Example: Pool balls
\item Perfect elasticity rare
\end{itemize}

\column{0.5\textwidth}
\textbf{Inelastic Collisions}
\begin{itemize}
\item Objects stick together
\item Energy converted to heat/sound
\item Example: Car crashes
\item More common in real life
\end{itemize}
\end{columns}

\pause

\begin{alertblock}{Remember}
Momentum is conserved in BOTH types of collisions!
\end{alertblock}
\end{frame}

% You Do: Practice Problem
\begin{frame}{Your Turn! (You Do)}
\begin{block}{Challenge Problem}
A 1200 kg car moving at 15 m/s collides with a stationary 800 kg car. They stick together.
What is their final velocity?
\end{block}

\begin{block}{Hints}
\begin{itemize}
\item This is an inelastic collision (they stick together)
\item Use conservation of momentum
\item Remember: mass$_1$v$_1$ + mass$_2$v$_2$ = (mass$_1$ + mass$_2$)v$_\text{final}$
\end{itemize}
\end{block}

\pause

\begin{block}{Solution Framework}
$(1200)(15) + (800)(0) = (1200 + 800)v_\text{final}$
\end{block}
\end{frame}

% Real-World Applications
\begin{frame}{Momentum in the Real World}
\begin{block}{Safety Applications}
\begin{itemize}
\item Vehicle crumple zones
\item Sports padding and helmets
\item Playground surface materials
\end{itemize}
\end{block}

\begin{block}{Engineering Applications}
\begin{itemize}
\item Rocket propulsion
\item Impact testing
\item Vehicle design
\end{itemize}
\end{block}
\end{frame}

% Summary Frame
\begin{frame}{Key Takeaways}
\begin{block}{Main Concepts}
\begin{itemize}
\item Momentum = mass × velocity
\item Impulse changes momentum
\item Momentum is conserved in isolated systems
\item Collisions can be elastic or inelastic
\end{itemize}
\end{block}

\begin{alertblock}{Why This Matters}
Understanding momentum helps us:
\begin{itemize}
\item Design safer vehicles
\item Improve sports equipment
\item Predict motion in collisions
\item Solve real-world problems
\end{itemize}
\end{alertblock}
\end{frame}

% Questions Frame
\begin{frame}{Questions to Consider}
\begin{block}{Think About}
\begin{itemize}
\item Why do heavy vehicles need longer to stop?
\item How do martial artists break boards?
\item Why do catchers "give" with the ball?
\item How do airbags protect us?
\end{itemize}
\end{block}

\begin{block}{Next Steps}
\begin{itemize}
\item Practice with example problems
\item Connect concepts to daily life
\item Observe momentum in action
\end{itemize}
\end{block}
\end{frame}

\end{document}