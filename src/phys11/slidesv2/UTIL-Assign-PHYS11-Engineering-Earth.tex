\documentclass[12pt]{article}
\usepackage[margin=1in]{geometry}
\usepackage{amsmath}
\usepackage{graphicx}
\graphicspath{{../images/}{../../shared/images/}}
\usepackage{enumitem}
\usepackage{fancyhdr}
\usepackage{tcolorbox}
\usepackage{hyperref}

\pagestyle{fancy}
\fancyhead{}
\fancyhead[L]{Physics 11}
\fancyhead[R]{Reality Check Assignment}
\fancyfoot{}
\fancyfoot[C]{\thepage}

\begin{document}

\begin{center}
\Large\textbf{Reality Check - Engineering Earth Analysis}\\[0.5em]
\normalsize Physics 11 - Mr. Gullo
\end{center}

\section*{Assignment}

Watch the Melodysheep video ``Engineering Earth'' and write a \textbf{500-750 word reflection} analyzing one concept from the video.

\begin{tcolorbox}[colback=blue!5!white,colframe=blue!75!black,title=Video Link]
\url{https://youtu.be/rN5f72lhJz8?si=8cZIaBs3SZaWmmYb}
\end{tcolorbox}

\section*{Your Reflection Must Include:}

\subsection*{1. Concept Description (100 words)}

What concept did you choose? What problem does it solve and how?

\subsection*{2. Physics Analysis (300 words)}

\begin{itemize}
\item Identify \textbf{2+ physics formulas} from your formula sheet that apply
\item State each formula clearly (e.g., $F = ma$, $E = mc^2$)
\item Explain how each formula connects to your concept
\item Discuss energy requirements, forces, materials, or scale
\end{itemize}

\subsection*{3. Reality Check (200 words)}

\begin{itemize}
\item Is this concept realistic today? Why or why not?
\item What physics challenges must be overcome?
\item Reference specific formulas/principles from your sheet
\end{itemize}

\subsection*{4. Impact (100 words)}

One positive and one negative consequence if this technology existed.

\newpage

\section*{Proficiency Rubric}

\subsection*{Emerging}

\textbf{Description:} Students demonstrate beginning awareness of physics concepts and their applications to engineering solutions. Work shows limited understanding of fundamental principles and requires significant guidance to make connections between physics formulas and real-world scenarios.

\textbf{Skills:}
\begin{itemize}
\item Identifies basic physics concepts but struggles to explain their underlying principles
\item Attempts to use physics formulas but shows minimal understanding of their meaning or relevance
\item Provides superficial analysis of technological feasibility without incorporating specific physics principles
\end{itemize}

\subsection*{Developing}

\textbf{Description:} Students show growing understanding of physics principles and can make simple connections to engineering applications with some support.

\textbf{Skills:}
\begin{itemize}
\item Describes chosen concept with basic accuracy but explanations may be incomplete
\item Uses 1-2 physics formulas appropriately but connections may be unclear
\item Evaluates technological realism using some physics principles but analysis lacks depth
\end{itemize}

\subsection*{Proficient}

\textbf{Description:} Students demonstrate solid understanding of physics concepts and can independently apply them to analyze engineering solutions.

\textbf{Skills:}
\begin{itemize}
\item Clearly explains chosen concept and accurately describes how physics principles enable or constrain the proposed solution
\item Correctly identifies and applies 2+ relevant physics formulas, explaining their significance
\item Provides realistic assessment of technological challenges using specific physics principles
\end{itemize}

\subsection*{Extending}

\textbf{Description:} Students demonstrate strong understanding of physics principles and can apply them thoughtfully to analyze engineering solutions.

\textbf{Skills:}
\begin{itemize}
\item Shows thorough understanding by making meaningful connections between the chosen concept and relevant physics principles
\item Correctly applies 2+ physics formulas with clear explanations and includes additional physics considerations
\item Provides well-reasoned assessment that incorporates multiple physics considerations
\end{itemize}

\end{document}
