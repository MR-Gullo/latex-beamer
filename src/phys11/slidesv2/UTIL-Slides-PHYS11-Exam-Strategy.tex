\documentclass{beamer}
% Use DS9 global theme
\usepackage{../../../shared/templates/ds9_theme}
\definecolor{ds9blue}{HTML}{0066CC}
\definecolor{ds9red}{HTML}{CC0000}

\title[Final Exam Strategy]{PHYS11/12: Final Exam Strategy}
\subtitle{The G.U.E.S.S. Method and Time Management}
\author[Mr. Gullo]{Mr. Gullo}
\date[June 2025]{June 2025}

\begin{document}

\frame{\titlepage}

\section{Learning Objectives}

\begin{frame}{Learning Objectives}
By the end of this presentation, you will be able to:
\begin{itemize}
\item Understand the proficiency scale used for grading open-ended questions \pause
\item Apply strategic time management techniques during the exam \pause
\item Use the G.U.E.S.S. method systematically for problem-solving \pause
\item Distinguish between Proficient and Extending level responses \pause
\item Demonstrate clear, logical thinking in your solutions
\end{itemize}
\end{frame}

\section{The Proficiency Scale}

\begin{frame}{Understanding the Proficiency Scale}
\begin{block}{Four Levels of Understanding}
\begin{itemize}
\item \textcolor{ds9red}{\textbf{Emerging:}} Limited understanding, incomplete solutions \pause
\item \textcolor{orange}{\textbf{Developing:}} Partial understanding, some correct elements \pause
\item \textcolor{green!60!black}{\textbf{Proficient:}} Clear understanding, systematic approach \pause
\item \textcolor{ds9blue}{\textbf{Extending:}} Deep understanding, sophisticated reasoning
\end{itemize}
\end{block}

\pause
\vspace{0.5cm}
\textbf{Key Point:} \emph{How} you solve a problem is as important as your final answer!
\end{frame}

\section{Strategic Setup}

\begin{frame}{The First Five Minutes: Strategic Setup}
\begin{enumerate}
\item \textbf{Brain Dump (30 seconds):} Write essential formulas, constants, and rules on scrap paper \pause
\item \textbf{Survey the Battlefield (2 minutes):} Scan the entire exam, identify open-response questions \pause
\item \textbf{Internalize the Goal:} Remember to be clear, systematic, and logical
\end{enumerate}

\pause
\vspace{0.5cm}
\begin{block}{Pro Tip}
Get worried formulas out of your head and onto paper immediately!
\end{block}
\end{frame}

\section{Three-Pass Strategy}

\begin{frame}{The Three-Pass Strategy}
\textbf{Don't do the exam in order!} Maximize your points.

\pause
\begin{block}{Pass 1: Quick Wins (10--15 min)}
Answer all easy multiple-choice questions you know instantly. Circle harder questions and move on.
\end{block}

\pause
\begin{block}{Pass 2: The Deep Dive (45--50 min)}
Focus on open-response questions using the G.U.E.S.S. method. Spend quality time here.
\end{block}

\pause
\begin{block}{Pass 3: The Finish Line (10--15 min)}
Attempt remaining questions, review work, transfer answers carefully.
\end{block}
\end{frame}

\section{The G.U.E.S.S. Method}

\begin{frame}{Introducing the G.U.E.S.S. Method}
Your roadmap to \textbf{Proficient} level responses:

\pause
\vspace{0.5cm}
\begin{flushleft}
\textbf{G} - Givens \& Diagram \pause \\
\textbf{U} - Unknowns \& Plan \pause \\
\textbf{E} - Equations \pause \\
\textbf{S} - Substitute \& Solve \pause \\
\textbf{S} - Solution \& Statement
\end{flushleft}

\pause
\vspace{0.5cm}
\textbf{Remember:} This systematic approach demonstrates your thinking process!
\end{frame}

\begin{frame}{G - Givens \& Diagram}
\begin{block}{What to Include}
\begin{itemize}
\item List all knowns and unknowns clearly \pause
\item Example: $V = 16$ V, $I_{\text{off}} = 0.40$ A, $I_{\text{on}} = 0.90$ A, $P_{\text{screen}} = ?$ \pause
\item Make a clear, labeled diagram
\end{itemize}
\end{block}

\pause
\begin{block}{Diagram Types}
\begin{itemize}
\item \textbf{Circuits:} Redraw the circuit clearly \pause
\item \textbf{Forces:} Draw free-body diagrams \pause
\item \textbf{Kinematics:} Sketch the motion path
\end{itemize}
\end{block}

\pause
\textbf{Key:} This directly addresses the ``Proficient'' criterion for clear organization.
\end{frame}

\begin{frame}{U - Unknowns \& Plan}
\begin{block}{State Your Strategy}
\begin{itemize}
\item Clearly state what you need to find \pause
\item Explain how you will solve it
\end{itemize}
\end{block}

\pause
\begin{block}{Example Planning Statement}
``To find the power of the screen, I will first find the current used by the screen alone. Then I will use the power formula $P = IV$.''
\end{block}

\pause
\vspace{0.5cm}
\textbf{Impact:} This single sentence elevates your response from ``Developing'' to ``Proficient'' by showing logical planning!
\end{frame}

\begin{frame}{E - Equations}
\begin{block}{Best Practices}
\begin{itemize}
\item Write base formulas \emph{before} plugging in numbers \pause
\item Show the physics principles you're using \pause
\item Examples: $P = IV$, $V = IR$, $F_{\text{net}} = ma$
\end{itemize}
\end{block}

\pause
\vspace{0.5cm}
\begin{block}{Why This Matters}
Demonstrates you understand the relevant physics concepts, not just arithmetic.
\end{block}
\end{frame}

\begin{frame}{S - Substitute \& Solve}
\begin{block}{Show Logical Steps}
\begin{itemize}
\item Work line-by-line, vertically down the page \pause
\item Use units consistently in every step \pause
\item Don't show scattered calculations
\end{itemize}
\end{block}

\pause
\begin{columns}
\begin{column}{0.5\textwidth}
\textcolor{green!60!black}{\textbf{Good Example:}}
\begin{align*}
I_{\text{screen}} &= I_{\text{on}} - I_{\text{off}} \\
&= 0.90~\text{A} - 0.40~\text{A} \\
&= 0.50~\text{A}
\end{align*}
\end{column}
\begin{column}{0.5\textwidth}
\textcolor{ds9red}{\textbf{Bad Example:}}
\[ 0.9 - 0.4 = 0.5 \]
\end{column}
\end{columns}

\pause
\vspace{0.3cm}
\textbf{Units are a major differentiator between ``Developing'' and ``Proficient''!}
\end{frame}

\begin{frame}{S - Solution \& Statement}
\begin{block}{Final Presentation}
\begin{itemize}
\item \textbf{Box your final answer} \pause
\item Write a concluding statement with units \pause
\item Example: ``Therefore, the power used by the screen alone is 8.0 W.''
\end{itemize}
\end{block}

\pause
\vspace{0.5cm}
\begin{block}{Complete Solution Format}
\[ \boxed{P_{\text{screen}} = 8.0~\text{W}} \]

``Therefore, the power used by the screen alone is 8.0 W.''
\end{block}
\end{frame}

\section{Reaching Extending Level}

\begin{frame}{Reaching the `Extending' Level}
\textbf{Extending} = Sophisticated understanding and complete command of physics

\pause
\begin{block}{Three Key Strategies}
\begin{enumerate}
\item \textbf{Find Hidden Details:} State implied information explicitly \pause
\item \textbf{Explain the Physics:} Don't just show math, explain \emph{why} \pause
\item \textbf{Check Your Answer:} Sense-check, units, alternative methods
\end{enumerate}
\end{block}

\pause
\begin{block}{Hidden Details Examples}
\begin{itemize}
\item ``starts from rest'' $\rightarrow$ $v_i = 0$
\item ``on the moon'' $\rightarrow$ use $g_{\text{moon}}$
\item ``no atmosphere'' $\rightarrow$ no air resistance
\end{itemize}
\end{block}
\end{frame}

\begin{frame}{Proficient vs. Extending: Capacitor Circuit}
\textbf{Question:} What is the voltage across and current through the capacitor a long time after the switch is closed?

\pause
\begin{columns}
\begin{column}{0.48\textwidth}
\begin{block}{Proficient Answer}
After a long time, the capacitor is fully charged.

$I_C = 0$ A \quad $V_C = 12$ V
\end{block}
\end{column}

\pause
\begin{column}{0.52\textwidth}
\begin{block}{Extending Answer}
After a long time, the capacitor is fully charged and acts like an \textbf{open circuit}.

\textbf{Current ($I_C$):} Because the circuit is open, no current can flow. $\therefore I_C = 0$ A.

\textbf{Voltage ($V_C$):} With zero current, there is no voltage drop across the resistor ($V_R = IR = 0$). All voltage must be across the capacitor. $\therefore V_C = 12$ V.
\end{block}
\end{column}
\end{columns}
\end{frame}

\section{Complete Example}

\begin{frame}{Complete G.U.E.S.S. Example}
\textbf{Problem:} A laptop uses 0.40 A when off and 0.90 A when on. If the voltage is 16 V, what power does the screen use?

\pause
\vspace{0.3cm}
\textbf{G - Givens:} $V = 16$ V, $I_{\text{off}} = 0.40$ A, $I_{\text{on}} = 0.90$ A

\pause
\textbf{U - Plan:} Find current used by screen alone, then use $P = IV$

\pause
\textbf{E - Equations:} $I_{\text{screen}} = I_{\text{on}} - I_{\text{off}}$, \quad $P = IV$

\pause
\textbf{S - Solve:}
\begin{align*}
I_{\text{screen}} &= 0.90~\text{A} - 0.40~\text{A} = 0.50~\text{A} \\
P &= (16~\text{V})(0.50~\text{A}) = 8.0~\text{W}
\end{align*}

\pause
\textbf{S - Statement:} $\boxed{P_{\text{screen}} = 8.0~\text{W}}$
\end{frame}

\section{Summary}

\begin{frame}{Summary: Your Path to Success}
\begin{block}{Remember the Strategy}
\begin{itemize}
\item \textbf{First 5 minutes:} Brain dump, survey, internalize goals \pause
\item \textbf{Three-pass approach:} Quick wins $\rightarrow$ Deep dive $\rightarrow$ Finish line \pause
\item \textbf{G.U.E.S.S. method:} Your systematic problem-solving framework
\end{itemize}
\end{block}

\pause
\begin{block}{Proficient $\rightarrow$ Extending}
\begin{itemize}
\item Find hidden details and state assumptions \pause
\item Explain the physics, not just the math \pause
\item Always check your answers multiple ways
\end{itemize}
\end{block}

\pause
\vspace{0.3cm}
\textbf{Final Reminder:} Show your thinking process clearly -- that's what the rubric measures!
\end{frame}

\begin{frame}{Final Words of Encouragement}
\begin{center}
\textcolor{ds9blue}{\Huge Stay Calm}

\pause
\vspace{0.5cm}
\textcolor{accent}{\Huge Be Systematic}

\pause
\vspace{0.5cm}
\textcolor{green!60!black}{\Huge Show What You Know}
\end{center}
\end{frame}

\end{document}
