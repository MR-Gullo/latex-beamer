\documentclass[12pt]{article}
\usepackage[margin=1in]{geometry}
\usepackage{amsmath}
\usepackage{graphicx}
\graphicspath{{../images/}{../../../shared/images/}}
\usepackage{float}
\usepackage{enumitem}
\usepackage{fancyhdr}
\usepackage{tcolorbox}
\usepackage{tikz}

\pagestyle{fancy}
\fancyhead{}
\fancyhead[L]{Physics 11 - Unit 3}
\fancyhead[R]{Measuring Angular Speed}
\fancyfoot{}
\fancyfoot[C]{\thepage}

\begin{document}

\begin{center}
{\Large \textbf{Laboratory Investigation:}}\\[0.5cm]
{\LARGE \textbf{Measuring Angular Speed in Circular Motion}}\\[1cm]
\rule{\textwidth}{0.4pt}
\end{center}

\section*{Lab Overview}
In this experiment, you will swing a stopper on a string in a horizontal circle and measure its angular speed ($\omega$) and tangential speed ($v$) for different radii. You will investigate the relationship between these quantities and the radius of circular motion.

\section*{Key Equations}
\begin{align*}
\omega &= \frac{2\pi}{T} \quad \text{(angular speed)} \\
v &= r\omega \quad \text{(tangential speed)} \\
v &= \frac{2\pi r}{T} \quad \text{(alternative form)}
\end{align*}

where:
\begin{itemize}
\item $\omega$ = angular speed (rad/s)
\item $v$ = tangential speed (m/s)
\item $r$ = radius of circular path (m)
\item $T$ = period (time for one complete revolution) (s)
\end{itemize}

\section*{Materials}
\begin{itemize}
\item Rubber stopper with hole
\item String (approximately 150 cm)
\item Stopwatch or phone timer
\item Meter stick or measuring tape
\item Safety goggles
\item Open space (gymnasium or outdoor area recommended)
\end{itemize}

\section*{Safety Considerations}
\begin{tcolorbox}[colback=yellow!20,colframe=red!75!black,title=\textbf{SAFETY WARNING}]
\begin{itemize}
\item Wear safety goggles at all times
\item Ensure you have at least 3 meters of clear space in all directions
\item Keep all observers at a safe distance
\item Check string for wear before each trial
\item Hold string firmly - do not let go during spinning
\item Stop immediately if string shows signs of breaking
\item Swing at waist level, not above head
\end{itemize}
\end{tcolorbox}

\section*{Procedure}

\subsection*{Setup}
\begin{enumerate}
\item Thread the string through the stopper and tie securely
\item Mark the string at 100 cm from the stopper (this will be your first radius)
\item Put on safety goggles
\item Clear the area of all obstacles and people
\end{enumerate}

\subsection*{Data Collection}
For each radius (100 cm, 90 cm, 80 cm, 70 cm, 60 cm, 50 cm):

\begin{enumerate}
\item Hold the string at the marked radius length
\item Begin swinging the stopper in a horizontal circle above your head
\item Once you achieve steady circular motion, start timing
\item Count 10 complete revolutions
\item Record the total time for 10 revolutions
\item Repeat 3 times for each radius and average the results
\item Calculate the period: $T = \frac{\text{total time}}{10}$
\end{enumerate}

\section*{Data Tables}

\subsection*{Radius: 100 cm = 1.00 m}
\begin{table}[H]
\centering
\begin{tabular}{|c|c|c|}
\hline
Trial & Time for 10 revs (s) & Period $T$ (s) \\
\hline
1 & & \\[0.4cm]
\hline
2 & & \\[0.4cm]
\hline
3 & & \\[0.4cm]
\hline
\textbf{Average} & & \\[0.4cm]
\hline
\end{tabular}
\end{table}

\subsection*{Radius: 90 cm = 0.90 m}
\begin{table}[H]
\centering
\begin{tabular}{|c|c|c|}
\hline
Trial & Time for 10 revs (s) & Period $T$ (s) \\
\hline
1 & & \\[0.4cm]
\hline
2 & & \\[0.4cm]
\hline
3 & & \\[0.4cm]
\hline
\textbf{Average} & & \\[0.4cm]
\hline
\end{tabular}
\end{table}

\subsection*{Radius: 80 cm = 0.80 m}
\begin{table}[H]
\centering
\begin{tabular}{|c|c|c|}
\hline
Trial & Time for 10 revs (s) & Period $T$ (s) \\
\hline
1 & & \\[0.4cm]
\hline
2 & & \\[0.4cm]
\hline
3 & & \\[0.4cm]
\hline
\textbf{Average} & & \\[0.4cm]
\hline
\end{tabular}
\end{table}

\subsection*{Radius: 70 cm = 0.70 m}
\begin{table}[H]
\centering
\begin{tabular}{|c|c|c|}
\hline
Trial & Time for 10 revs (s) & Period $T$ (s) \\
\hline
1 & & \\[0.4cm]
\hline
2 & & \\[0.4cm]
\hline
3 & & \\[0.4cm]
\hline
\textbf{Average} & & \\[0.4cm]
\hline
\end{tabular}
\end{table}

\subsection*{Radius: 60 cm = 0.60 m}
\begin{table}[H]
\centering
\begin{tabular}{|c|c|c|}
\hline
Trial & Time for 10 revs (s) & Period $T$ (s) \\
\hline
1 & & \\[0.4cm]
\hline
2 & & \\[0.4cm]
\hline
3 & & \\[0.4cm]
\hline
\textbf{Average} & & \\[0.4cm]
\hline
\end{tabular}
\end{table}

\subsection*{Radius: 50 cm = 0.50 m}
\begin{table}[H]
\centering
\begin{tabular}{|c|c|c|}
\hline
Trial & Time for 10 revs (s) & Period $T$ (s) \\
\hline
1 & & \\[0.4cm]
\hline
2 & & \\[0.4cm]
\hline
3 & & \\[0.4cm]
\hline
\textbf{Average} & & \\[0.4cm]
\hline
\end{tabular}
\end{table}

\section*{Calculations Summary Table}
\begin{table}[H]
\centering
\begin{tabular}{|c|c|c|c|}
\hline
Radius $r$ (m) & Period $T$ (s) & $\omega$ (rad/s) & $v$ (m/s) \\
\hline
1.00 & & & \\[0.4cm]
\hline
0.90 & & & \\[0.4cm]
\hline
0.80 & & & \\[0.4cm]
\hline
0.70 & & & \\[0.4cm]
\hline
0.60 & & & \\[0.4cm]
\hline
0.50 & & & \\[0.4cm]
\hline
\end{tabular}
\end{table}

\section*{Sample Calculation}
Show your work for one complete set of calculations (choose any radius):

\textbf{Given:}
\begin{itemize}
\item $r$ =
\item $T$ =
\end{itemize}

\vspace{1cm}

\textbf{Required:} $\omega$ and $v$

\vspace{1cm}

\textbf{Analysis \& Solution:}

Calculate $\omega$:
$$\omega = \frac{2\pi}{T} = $$

\vspace{2cm}

Calculate $v$:
$$v = r\omega = $$

\vspace{2cm}

\textbf{Paraphrase:}

\vspace{2cm}

\section*{Graphing}

\subsection*{Graph 1: Angular Speed vs. Radius}
Create a graph with:
\begin{itemize}
\item x-axis: Radius $r$ (m)
\item y-axis: Angular speed $\omega$ (rad/s)
\item Plot all 6 data points
\item Draw a best-fit line or curve
\item Title: "Angular Speed vs. Radius"
\end{itemize}

\textbf{Attach your graph here or on a separate sheet.}

\subsection*{Graph 2: Tangential Speed vs. Radius}
Create a graph with:
\begin{itemize}
\item x-axis: Radius $r$ (m)
\item y-axis: Tangential speed $v$ (m/s)
\item Plot all 6 data points
\item Draw a best-fit line or curve
\item Title: "Tangential Speed vs. Radius"
\end{itemize}

\textbf{Attach your graph here or on a separate sheet.}

\section*{Analysis Questions}
\begin{enumerate}
\item Describe the relationship between radius and angular speed based on your graph. Is it linear, inverse, or something else?

\vspace{3cm}

\item Describe the relationship between radius and tangential speed. How does it differ from the $\omega$ vs. $r$ relationship?

\vspace{3cm}

\item If you kept the angular speed constant, what would happen to the tangential speed as you increased the radius? Use the equation $v = r\omega$ to explain.

\vspace{3cm}

\item What would happen to the tension in the string if you doubled the radius while maintaining the same angular speed?

\vspace{3cm}

\item What were the major sources of experimental error in this lab? How could they be reduced?

\vspace{3cm}
\end{enumerate}

\section*{GRASP Check}

\begin{tcolorbox}[colback=blue!5,colframe=blue!75!black,title=\textbf{GRASP Reflection}]
Before submitting, verify:
\begin{itemize}
\item[\textbf{G}] \textbf{Given:} All measurements recorded with proper units
\item[\textbf{R}] \textbf{Required:} Both $\omega$ and $v$ calculated for all radii
\item[\textbf{A}] \textbf{Analysis:} Graphs completed and relationships identified
\item[\textbf{S}] \textbf{Solution:} Sample calculation shown with all steps
\item[\textbf{P}] \textbf{Paraphrase:} Results explained in physical context
\end{itemize}
\end{tcolorbox}

\section*{Photo Documentation}
\begin{tcolorbox}[colback=green!5,colframe=green!75!black,title=\textbf{Required Photos}]
Include the following photos with your submission:
\begin{enumerate}
\item Setup showing the stopper and string with marked radius
\item Action shot of the stopper in circular motion (ask partner to take photo)
\item Close-up of your data tables
\end{enumerate}
\end{tcolorbox}

\section*{Conclusion}
Write a conclusion that addresses:
\begin{itemize}
\item The relationship between radius, angular speed, and tangential speed
\item How your experimental results support (or don't support) the theoretical equations
\item The significance of the difference between angular and tangential speed
\item Sources of uncertainty and how they affected your results
\end{itemize}

\vspace{5cm}

\end{document}
