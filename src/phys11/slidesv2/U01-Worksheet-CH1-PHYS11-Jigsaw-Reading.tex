% Options for packages loaded elsewhere
\PassOptionsToPackage{unicode}{hyperref}
\PassOptionsToPackage{hyphens}{url}
\documentclass[
]{article}
\usepackage{xcolor}
\usepackage{amsmath,amssymb}
\setcounter{secnumdepth}{-\maxdimen} % remove section numbering
\usepackage{iftex}
\ifPDFTeX
  \usepackage[T1]{fontenc}
  \usepackage[utf8]{inputenc}
  \usepackage{textcomp} % provide euro and other symbols
\else % if luatex or xetex
  \usepackage{unicode-math} % this also loads fontspec
  \defaultfontfeatures{Scale=MatchLowercase}
  \defaultfontfeatures[\rmfamily]{Ligatures=TeX,Scale=1}
\fi
\usepackage{lmodern}
\ifPDFTeX\else
  % xetex/luatex font selection
\fi
% Use upquote if available, for straight quotes in verbatim environments
\IfFileExists{upquote.sty}{\usepackage{upquote}}{}
\IfFileExists{microtype.sty}{% use microtype if available
  \usepackage[]{microtype}
  \UseMicrotypeSet[protrusion]{basicmath} % disable protrusion for tt fonts
}{}
\makeatletter
\@ifundefined{KOMAClassName}{% if non-KOMA class
  \IfFileExists{parskip.sty}{%
    \usepackage{parskip}
  }{% else
    \setlength{\parindent}{0pt}
    \setlength{\parskip}{6pt plus 2pt minus 1pt}}
}{% if KOMA class
  \KOMAoptions{parskip=half}}
\makeatother
\usepackage{longtable,booktabs,array}
\newcounter{none} % for unnumbered tables
\usepackage{calc} % for calculating minipage widths
% Correct order of tables after \paragraph or \subparagraph
\usepackage{etoolbox}
\makeatletter
\patchcmd\longtable{\par}{\if@noskipsec\mbox{}\fi\par}{}{}
\makeatother
% Allow footnotes in longtable head/foot
\IfFileExists{footnotehyper.sty}{\usepackage{footnotehyper}}{\usepackage{footnote}}
\makesavenoteenv{longtable}
\setlength{\emergencystretch}{3em} % prevent overfull lines
\providecommand{\tightlist}{%
  \setlength{\itemsep}{0pt}\setlength{\parskip}{0pt}}
\usepackage{bookmark}
\IfFileExists{xurl.sty}{\usepackage{xurl}}{} % add URL line breaks if available
\urlstyle{same}
\hypersetup{
  hidelinks,
  pdfcreator={LaTeX via pandoc}}

\author{}
\date{}

\begin{document}

Jigsaw Reading and Problem Solving Activity

\textbf{Overview}

This Jigsaw Reading and Problem-Solving activity is designed to
encourage students to work collaboratively in small groups. Each group
is assigned a section of the chapter to read, understand, and then teach
to their peers. This approach promotes both individual responsibility
and teamwork as students rely on one another to learn the entire content
of the chapter.

\textbf{Learning Objectives}

- Encourage collaborative learning by dividing tasks and
responsibilities.

- Foster critical reading and comprehension of the assigned material.

- Enhance communication skills through peer teaching.

- Develop problem-solving skills through group discussions and analysis.

- Promote a deeper understanding of the subject matter through active
engagement.

\textbf{Instructions for Students}

1. Group Formation

- The class will be divided into small groups of 3--5 students.

- Each group will be assigned a section from the chapter to focus on.

2. Individual Study

- Each student in the group will read their assigned section silently.

- While reading, students should take notes on the key concepts,
definitions, and examples provided in their section.

3. Group Discussion

- Once everyone has finished reading, the group will discuss their
section to ensure that all members fully understand the material.

- Collaborate to clarify difficult concepts, using textbooks, additional
resources, or asking the teacher for help if necessary.

4. Teaching Preparation

- Each group will prepare a short presentation (5-10 minutes) to teach
the main points of their section to the rest of the class.

- Decide how to divide the presentation among group members. Make sure
everyone participates.

- Use visual aids, such as diagrams or written key points, to make the
explanation clear.

5. Class Presentation

- Each group will present their section to the class.

- As the group teaches, the rest of the class will take notes. Ask
questions if anything is unclear.

6. Problem Solving

- After the presentation, the group can pose a problem or question
related to their section for the class to solve. This could be a
calculation, an application of a concept, or a discussion question from
the textbook.

- Engage the class in solving the problem together.

\textbf{Assessment}

- Participation in both the group discussion and presentation will be
considered.

- Clarity and completeness of the group\textquotesingle s explanation
will be assessed.

- Peer feedback will be encouraged to help refine understanding and
communication skills.

\textbf{Tips for Success}

- Stay focused during group discussions---everyone\textquotesingle s
understanding is important.

- Communicate clearly when teaching your section. Imagine explaining the
topic to someone who has never heard of it.

- Ask questions if you don't understand something during another group's
presentation. It's a chance for everyone to learn.

{\def\LTcaptype{none} % do not increment counter
\begin{longtable}[]{@{}
  >{\raggedright\arraybackslash}p{(\linewidth - 2\tabcolsep) * \real{0.1877}}
  >{\raggedright\arraybackslash}p{(\linewidth - 2\tabcolsep) * \real{0.8123}}@{}}
\toprule\noalign{}
\endhead
\bottomrule\noalign{}
\endlastfoot
Level & Description \\
Excellent & Exceptional collaboration and presentation skills.
Demonstrates comprehensive understanding of all sections. Effectively
uses visual aids. Poses challenging problems and shows excellent
critical thinking. \\
Good & Good collaboration and clear presentation. Shows solid
understanding of most sections. Uses visual aids well. Poses relevant
problems and demonstrates good critical thinking. \\
Satisfactory & Adequate participation and presentation. Basic
understanding of assigned section. Limited use of visual aids. Poses
somewhat relevant problems with some critical thinking. \\
Needs Improvement & Minimal participation and unclear presentation.
Limited understanding of material. Rarely uses visual aids. Struggles to
pose relevant problems or demonstrate critical thinking. \\
\end{longtable}
}

\end{document}
