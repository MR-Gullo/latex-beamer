% CH14 Sound - PHYS11 Unit 06
\documentclass[aspectratio=169]{beamer}
\usepackage{../../../shared/templates/ds9_theme}
\usepackage[overridenote]{pdfpc}
\graphicspath{{../images/}{../../shared/images/}}

\title[Sound Waves]{PHYS11 CH14: Sound}
\subtitle{Speed, Intensity, Doppler Effect, and Resonance}
\author[Mr. Gullo]{Mr. Gullo}
\date{}

\begin{document}

\frame{\titlepage
\note{- CH14 Sound: 4 sections today\\\\
- Builds on CH13 Waves (transverse vs longitudinal)\\\\
- Key concepts: v=f*lambda, decibels, Doppler, resonance\\\\
- By end: calculate wave properties, intensity, frequency shifts}
}

\begin{frame}
\frametitle{Outline}
\tableofcontents
\end{frame}

%=============================================================================
\section{Introduction}
%=============================================================================

\begin{frame}
\frametitle{Learning Objectives}
By the end of this chapter, you will be able to:
\begin{itemize}
\item Relate wave characteristics to sound waves \pause
\item Describe speed of sound in various media \pause
\item Calculate using $v = f\lambda$ \pause
\item Relate amplitude to loudness and energy \pause
\item Use the decibel scale \pause
\item Explain the Doppler effect and sonic booms \pause
\item Describe resonance, beats, and harmonics
\end{itemize}
\note{- 7 objectives for this chapter\\\\
- First 3: wave basics (14.1)\\\\
- Next 2: intensity/decibels (14.2)\\\\
- Last 2: Doppler and resonance (14.3-14.4)\\\\
- Assessment: quiz next class on v=f*lambda problems}
\end{frame}

\begin{frame}
\frametitle{The Nature of Sound}
\begin{figure}
\centering
\includegraphics[width=0.6\textwidth]{phys11-sound-fig01.jpg}
\end{figure}

\pause
\textbf{Key Question:} If a tree falls in a forest and no one is there to hear it, does it make a sound?

\pause
\vspace{0.3cm}
\textbf{Physics answer:} Sound is a disturbance of matter transmitted from its source outward as \alert{longitudinal waves}.
\note{- Classic philosophy question - get students thinking\\\\
- Physics answer: YES, sound waves are created\\\\
- But perception requires an observer\\\\
- Key point: sound needs a MEDIUM to travel\\\\
- Name wheel: why no sound in space?\\\\
  ANSWER: no medium (air/water) to vibrate}
\end{frame}

%=============================================================================
\section{14.1 Speed of Sound, Frequency, and Wavelength}
%=============================================================================

\begin{frame}
\frametitle{Sound as a Longitudinal Wave}
\begin{columns}[T]
\column{0.55\textwidth}
\begin{figure}
\centering
\includegraphics[width=\textwidth]{phys11-sound-fig04.jpg}
\end{figure}

\pause
\column{0.45\textwidth}
\textbf{Sound waves consist of:}
\begin{itemize}
\item \textbf{Compressions} - high pressure regions \pause
\item \textbf{Rarefactions} - low pressure regions
\end{itemize}

\pause
\vspace{0.3cm}
Analogous to crests and troughs in transverse waves.
\end{columns}
\note{- Compare to transverse: particles move parallel to wave, not perpendicular\\\\
- Compression = high pressure = like a crest\\\\
- Rarefaction = low pressure = like a trough\\\\
- Name wheel: what type of wave is light?\\\\
  ANSWER: transverse (EM waves oscillate perpendicular)}
\end{frame}

\begin{frame}
\frametitle{How Sound Travels}
\begin{figure}
\centering
\includegraphics[width=0.7\textwidth]{phys11-sound-fig05.jpg}
\end{figure}

\pause
\textbf{Pressure variation:} Sound creates alternating high and low pressure regions as it travels through a medium.
\note{- Air molecules push together (compression) then spread apart (rarefaction)\\\\
- Wave moves forward but molecules just oscillate in place\\\\
- Name wheel: do the air molecules travel with the wave?\\\\
  ANSWER: no, they vibrate back and forth in place}
\end{frame}

\begin{frame}
\frametitle{Speed of Sound in Different Media}
The speed of sound depends on:
\begin{itemize}
\item \textbf{Rigidity} (or compressibility) of the medium \pause
\item \textbf{Density} of the medium
\end{itemize}

\pause
\vspace{0.3cm}
\begin{center}
\begin{tabular}{|l|c|}
\hline
\textbf{Medium} & \textbf{Speed (m/s)} \\
\hline
Air (0°C) & 331 \\
Air (20°C) & 343 \\
Water (25°C) & 1493 \\
Steel & 5960 \\
\hline
\end{tabular}
\end{center}

\pause
\vspace{0.3cm}
\alert{More rigid = faster; more dense = slower}
\note{- Rigid materials transmit vibrations faster\\\\
- Dense materials have more inertia, slower response\\\\
- Steel: very rigid, so very fast despite density\\\\
- Name wheel: why faster in water than air?\\\\
  ANSWER: water more rigid (harder to compress) outweighs higher density}
\end{frame}

\begin{frame}
\frametitle{Speed, Frequency, and Wavelength}
\begin{figure}
\centering
\includegraphics[width=0.5\textwidth]{phys11-sound-fig09.jpg}
\end{figure}

\pause
\begin{center}
\Huge
$\boxed{v = f\lambda}$
\end{center}

\pause
\textbf{Where:}
\begin{itemize}
\item $v$ = speed of sound (m/s)
\item $f$ = frequency (Hz)
\item $\lambda$ = wavelength (m)
\end{itemize}
\note{- Same equation as CH13 waves: v = f * lambda\\\\
- v = velocity (how fast wave moves through medium)\\\\
- f = frequency (cycles per second, Hz)\\\\
- lambda = wavelength (distance per cycle)\\\\
- Rearrangements: f=v/lambda, lambda=v/f\\\\
- Name wheel: which form to find wavelength?\\\\
  ANSWER: lambda=v/f (divide both sides by f)}
\end{frame}

\begin{frame}
\frametitle{Key Insight: Inverse Relationship}
In a given medium at constant temperature:
\begin{itemize}
\item Speed $v$ is constant for all frequencies \pause
\item Therefore: $f$ and $\lambda$ are \alert{inversely related}
\end{itemize}

\pause
\vspace{0.5cm}
\textbf{Higher frequency $\rightarrow$ shorter wavelength}

\textbf{Lower frequency $\rightarrow$ longer wavelength}

\pause
\vspace{0.5cm}
\textit{This is why music from all instruments arrives in cadence regardless of distance!}
\note{- Key insight: speed is fixed by medium, not frequency\\\\
- If v constant: double f means half lambda\\\\
- Orchestra example: high and low notes travel at same speed\\\\
- Name wheel: if you double the frequency, what happens to wavelength?\\\\
  ANSWER: wavelength halves (inverse relationship)}
\end{frame}

%-----------------------------------------------------------------------------
% I Do: Wavelengths of Audible Sounds
%-----------------------------------------------------------------------------

\begin{frame}
\frametitle{I Do: Wavelengths of Audible Sounds}
\textbf{Problem:}\\
Calculate the wavelengths of sounds at the extremes of the audible range, 20 Hz and 20,000 Hz, when sound travels at 348.7 m/s.

\pause
\vspace{0.5cm}

\begin{columns}[T]
\column{0.48\textwidth}
\textbf{G - Givens}
\begin{itemize}
\item $v = 348.7$ m/s
\item $f_{min} = 20$ Hz
\item $f_{max} = 20,000$ Hz
\end{itemize}

\pause
\column{0.48\textwidth}
\textbf{U - Unknown}
\begin{itemize}
\item $\lambda_{max} = ?$
\item $\lambda_{min} = ?$
\end{itemize}
\end{columns}
\note{- Read problem aloud\\\\
- Name wheel: what are the givens?\\\\
  ANSWER: v=348.7 m/s, f=20 Hz and 20000 Hz\\\\
- Name wheel: what are we solving for?\\\\
  ANSWER: wavelength lambda (two values)\\\\
- Note: audible range is 20-20000 Hz}
\end{frame}

\begin{frame}
\frametitle{I Do: Equation Selection}
\textbf{E - Equation}

\pause
\vspace{0.3cm}

Start with: $v = f\lambda$

\pause
\vspace{0.3cm}

Rearrange for unknown:
$$\lambda = \frac{v}{f}$$
\note{- Start with v = f * lambda\\\\
- Name wheel: how do we isolate lambda?\\\\
  ANSWER: divide both sides by f\\\\
- Algebra: v/f = (f*lambda)/f = lambda\\\\
- Final form: lambda = v/f}
\end{frame}

\begin{frame}
\frametitle{I Do: Substitute and Solve}
\textbf{S - Substitute}
\begin{itemize}
\item $\lambda_{max} = \frac{348.7 \text{ m/s}}{20 \text{ Hz}}$ \pause
\item $\lambda_{min} = \frac{348.7 \text{ m/s}}{20,000 \text{ Hz}}$
\end{itemize}

\pause
\vspace{0.5cm}

\textbf{S - Solve}
\begin{itemize}
\item $\lambda_{max} = 17$ m $\approx$ \boxed{20 \text{ m}}
\item $\lambda_{min} = 0.017$ m $\approx$ \boxed{2 \text{ cm}}
\end{itemize}

\pause
\vspace{0.5cm}

\textbf{Check:} The smaller frequency gives larger wavelength. Units correct.
\note{- Numbers only at the end\\\\
- lambda_max = 348.7/20 = 17 m (about 20 m)\\\\
- lambda_min = 348.7/20000 = 0.017 m (about 2 cm)\\\\
- Reasonableness: bass sounds have long waves (room-sized)\\\\
- High pitch sounds have short waves (finger-sized)}
\end{frame}

%-----------------------------------------------------------------------------
% We Do: Speed of Sound
%-----------------------------------------------------------------------------

\begin{frame}
\frametitle{We Do: Calculating Speed of Sound}
\textbf{Problem:}\\
What is the speed of a sound wave with frequency 2000 Hz and wavelength 0.4 m?

\pause
\vspace{0.5cm}

\begin{columns}[T]
\column{0.48\textwidth}
\textbf{G - Givens}
\begin{itemize}
\item $f = 2000$ Hz
\item $\lambda = 0.4$ m
\end{itemize}

\pause
\column{0.48\textwidth}
\textbf{U - Unknown}
\begin{itemize}
\item $v = ?$
\end{itemize}
\end{columns}

\pause
\vspace{0.5cm}
\alert{Hint:} Which equation relates $v$, $f$, and $\lambda$?
\note{- Name wheel: what are the givens?\\\\
  ANSWER: f=2000 Hz, lambda=0.4 m\\\\
- Name wheel: what equation relates v, f, lambda?\\\\
  ANSWER: v = f * lambda\\\\
- Name wheel: do we need to rearrange?\\\\
  ANSWER: no, already solved for v}
\end{frame}

\begin{frame}
\frametitle{We Do: Solution}
\textbf{E - Equation:} $v = f\lambda$

\pause
\vspace{0.3cm}

\textbf{S - Substitute:} $v = (2000 \text{ Hz})(0.4 \text{ m})$

\pause
\vspace{0.3cm}

\textbf{S - Solve:}
$$\boxed{v = 800 \text{ m/s}}$$

\pause
\vspace{0.3cm}
\textbf{Check:} Units: Hz $\times$ m = (1/s) $\times$ m = m/s. Correct!
\note{- Name wheel: substitute the values\\\\
  ANSWER: v = 2000 * 0.4 = 800 m/s\\\\
- Unit check: Hz = 1/s, so (1/s)*m = m/s\\\\
- Reasonableness: 800 m/s is faster than air (343)\\\\
- This would be in a solid like aluminum}
\end{frame}

%-----------------------------------------------------------------------------
% You Do: Dog Hearing
%-----------------------------------------------------------------------------

\begin{frame}
\frametitle{You Do: Dog Hearing}
\textbf{Problem:}\\
Dogs can hear frequencies up to 45 kHz. What is the wavelength of a sound wave with this frequency traveling in air at 0°C?

\vspace{0.5cm}

\textbf{Given:}
\begin{itemize}
\item $f = 45$ kHz $= 45,000$ Hz
\item Speed of sound at 0°C: $v = 331$ m/s
\end{itemize}

\textbf{Find:} Wavelength $\lambda$

\vspace{0.5cm}

\textbf{Hint:} Rearrange $v = f\lambda$ to solve for $\lambda$.
\note{- 5 min independent practice\\\\
- Walk around and check progress\\\\
- FULL SOLUTION for reference:\\\\
  Equation: v = f * lambda\\\\
  Rearrange: lambda = v / f\\\\
  Substitute: lambda = 331 / 45000\\\\
  ANSWER: lambda = 0.00736 m = 7.4 mm\\\\
- Reasonableness: very short, like a fingernail\\\\
- Fast finishers: what about elephants (14 Hz)?}
\end{frame}

%=============================================================================
\section{14.2 Sound Intensity and Sound Level}
%=============================================================================

\begin{frame}
\frametitle{Sound Intensity}
\begin{figure}
\centering
\includegraphics[width=0.5\textwidth]{phys11-sound-fig17.jpg}
\end{figure}

\pause
\textbf{Intensity} is the power per unit area carried by a wave:

\pause
\begin{center}
\Huge
$\boxed{I = \frac{P}{A}}$
\end{center}

\pause
\textbf{Units:} W/m$^2$
\note{- Intensity = power spread over area\\\\
- Same power over larger area = lower intensity\\\\
- Why speakers sound quieter far away: same power, bigger area\\\\
- Name wheel: units of intensity?\\\\
  ANSWER: W/m2 (watts per square meter)}
\end{frame}

\begin{frame}
\frametitle{Intensity and Pressure Amplitude}
Sound intensity depends on pressure amplitude:

\pause
\begin{center}
\Large
$I = \frac{(\Delta p)^2}{2\rho v_w}$
\end{center}

\pause
\vspace{0.3cm}
\textbf{Where:}
\begin{itemize}
\item $\Delta p$ = pressure amplitude (Pa)
\item $\rho$ = density of medium (kg/m$^3$)
\item $v_w$ = speed of sound (m/s)
\end{itemize}

\pause
\vspace{0.3cm}
\alert{Key:} Intensity is proportional to amplitude squared: $I \propto (\Delta p)^2$
\note{- delta-p = pressure variation from normal atmospheric\\\\
- rho = density of medium (kg/m3)\\\\
- v_w = wave speed in medium\\\\
- CRITICAL: I proportional to (delta-p)^2\\\\
- Name wheel: if pressure amplitude doubles, what happens to intensity?\\\\
  ANSWER: intensity quadruples (2^2 = 4)}
\end{frame}

\begin{frame}
\frametitle{Pressure vs. Intensity}
\begin{figure}
\centering
\includegraphics[width=0.7\textwidth]{phys11-sound-fig19.jpg}
\end{figure}

\pause
Greater amplitude $\rightarrow$ greater pressure variation $\rightarrow$ greater intensity (louder sound)
\note{- Point to high vs low amplitude waves in diagram\\\\
- Larger oscillations = more energy = louder sound\\\\
- Name wheel: which wave carries more energy, high or low amplitude?\\\\
  ANSWER: high amplitude (more energy per wave)}
\end{frame}

\begin{frame}
\frametitle{The Decibel Scale}
Human perception of loudness is \alert{logarithmic}, not linear.

\pause
\vspace{0.3cm}

\textbf{Sound Intensity Level:}
\begin{center}
\Huge
$\boxed{\beta(\text{dB}) = 10 \log_{10}\left(\frac{I}{I_0}\right)}$
\end{center}

\pause
\vspace{0.3cm}

\textbf{Where:}
\begin{itemize}
\item $\beta$ = sound level in decibels (dB)
\item $I$ = sound intensity (W/m$^2$)
\item $I_0 = 10^{-12}$ W/m$^2$ (threshold of hearing)
\end{itemize}
\note{- Logarithmic = each step is multiplication not addition\\\\
- I0 = threshold of hearing (barely audible)\\\\
- 10^-12 W/m2 is incredibly tiny\\\\
- log10 compresses huge range to manageable numbers\\\\
- Name wheel: why use log scale?\\\\
  ANSWER: human ears respond to ratios, not differences}
\end{frame}

\begin{frame}
\frametitle{Understanding Decibels}
\textbf{Key relationships:}
\begin{itemize}
\item Threshold of hearing: $I_0 = 10^{-12}$ W/m$^2$ $\rightarrow$ 0 dB \pause
\item Each factor of 10 in intensity $= +10$ dB \pause
\item Doubling intensity $\approx +3$ dB
\end{itemize}

\pause
\vspace{0.5cm}

\begin{center}
\begin{tabular}{|l|c|}
\hline
\textbf{Sound} & \textbf{Level (dB)} \\
\hline
Threshold of hearing & 0 \\
Whisper & 20 \\
Normal conversation & 60 \\
Busy traffic & 80 \\
Rock concert & 110 \\
Pain threshold & 130 \\
\hline
\end{tabular}
\end{center}
\note{- 0 dB = threshold of hearing (not silence!)\\\\
- Every 10 dB = 10x intensity\\\\
- Every 3 dB = 2x intensity (useful rule)\\\\
- 130 dB = pain threshold, hearing damage\\\\
- Name wheel: how many times more intense is 80 dB vs 60 dB?\\\\
  ANSWER: 100 times (20 dB = 10 x 10)}
\end{frame}

%-----------------------------------------------------------------------------
% I Do: Sound Intensity Level
%-----------------------------------------------------------------------------

\begin{frame}
\frametitle{I Do: Calculating Sound Intensity Level}
\textbf{Problem:}\\
Calculate the sound intensity level in decibels for a sound wave in air at 0°C having a pressure amplitude of 0.656 Pa.

\pause
\vspace{0.3cm}

\begin{columns}[T]
\column{0.48\textwidth}
\textbf{G - Givens}
\begin{itemize}
\item $\Delta p = 0.656$ Pa
\item $v = 331$ m/s (at 0°C)
\item $\rho = 1.29$ kg/m$^3$
\item $I_0 = 10^{-12}$ W/m$^2$
\end{itemize}

\pause
\column{0.48\textwidth}
\textbf{U - Unknown}
\begin{itemize}
\item $I = ?$
\item $\beta = ?$
\end{itemize}
\end{columns}
\note{- Two-step problem: first I, then beta\\\\
- Name wheel: what are the givens?\\\\
  ANSWER: delta-p=0.656 Pa, v=331 m/s, rho=1.29 kg/m3\\\\
- Name wheel: what are we finding?\\\\
  ANSWER: intensity I and decibel level beta}
\end{frame}

\begin{frame}
\frametitle{I Do: Equations}
\textbf{E - Equations}

\pause
\vspace{0.3cm}

Step 1 - Find intensity:
$$I = \frac{(\Delta p)^2}{2\rho v_w}$$

\pause
\vspace{0.3cm}

Step 2 - Find decibel level:
$$\beta(\text{dB}) = 10 \log_{10}\left(\frac{I}{I_0}\right)$$
\note{- Two-step solution: first intensity, then decibels\\\\
- Step 1: I = (delta-p)^2 / (2 * rho * v)\\\\
- Step 2: beta = 10 * log10(I / I0)\\\\
- Name wheel: why do we need two equations?\\\\
  ANSWER: we have pressure, need intensity, then convert to dB}
\end{frame}

\begin{frame}
\frametitle{I Do: Substitute and Solve}
\textbf{S - Substitute for intensity:}
$$I = \frac{(0.656 \text{ Pa})^2}{2(1.29 \text{ kg/m}^3)(331 \text{ m/s})}$$

\pause
\vspace{0.3cm}

\textbf{S - Solve:}
$$I = 5.04 \times 10^{-4} \text{ W/m}^2$$

\pause
\vspace{0.3cm}

\textbf{Now find decibels:}
$$\beta = 10 \log_{10}\left(\frac{5.04 \times 10^{-4}}{10^{-12}}\right) = 10 \log_{10}(5.04 \times 10^8)$$

\pause
$$\boxed{\beta = 87.0 \text{ dB}}$$
\note{- Substitute numbers only after algebra done\\\\
- I = (0.656)^2 / (2 * 1.29 * 331) = 5.04e-4 W/m2\\\\
- beta = 10 * log10(5.04e-4 / 1e-12) = 10 * log10(5.04e8)\\\\
- ANSWER: 87 dB (loud traffic or lawn mower level)\\\\
- Reasonableness: 0.656 Pa is moderate pressure variation}
\end{frame}

%-----------------------------------------------------------------------------
% We Do: Doubling Intensity
%-----------------------------------------------------------------------------

\begin{frame}
\frametitle{We Do: Doubling Intensity}
\textbf{Problem:}\\
Show that if one sound is twice as intense as another, it has a sound level about 3 dB higher.

\pause
\vspace{0.5cm}

\textbf{Given:} $\frac{I_2}{I_1} = 2$

\pause
\vspace{0.3cm}

\alert{Hint:} Use $\beta_2 - \beta_1 = 10 \log_{10}\left(\frac{I_2}{I_1}\right)$

\pause
\vspace{0.5cm}

\textbf{Solution:}
$$\beta_2 - \beta_1 = 10 \log_{10}(2.00) = 10(0.301) = \boxed{3.01 \text{ dB}}$$
\note{- This proves the 3 dB rule for doubling intensity\\\\
- Name wheel: what is log10(2)?\\\\
  ANSWER: 0.301 (memorize this or use calculator)\\\\
- 10 * 0.301 = 3.01 dB\\\\
- Practical: adding a second identical speaker = +3 dB\\\\
- Name wheel: what about 10x intensity?\\\\
  ANSWER: +10 dB (since log10(10) = 1)}
\end{frame}

%-----------------------------------------------------------------------------
% You Do: Calculate Intensity
%-----------------------------------------------------------------------------

\begin{frame}
\frametitle{You Do: Calculate Intensity}
\textbf{Problem:}\\
Calculate the intensity of a wave if the power transferred is 10 W and the area through which the wave is transferred is 5 square meters.

\vspace{0.5cm}

\textbf{Given:}
\begin{itemize}
\item $P = 10$ W
\item $A = 5$ m$^2$
\end{itemize}

\textbf{Find:} Intensity $I$

\vspace{0.5cm}

\textbf{Hint:} Use $I = \frac{P}{A}$
\note{- 3 min independent practice\\\\
- Walk around and check progress\\\\
- FULL SOLUTION:\\\\
  Equation: I = P / A\\\\
  Substitute: I = 10 W / 5 m2\\\\
  ANSWER: I = 2 W/m2\\\\
- Reasonableness: 2 W/m2 is extremely intense (painful!)\\\\
- Fast finishers: what decibel level is this?}
\end{frame}

\begin{frame}
\frametitle{Human Hearing}
\begin{figure}
\centering
\includegraphics[width=0.6\textwidth]{phys11-sound-fig24.jpg}
\end{figure}

\pause
\textbf{Hearing range:} 20 Hz to 20,000 Hz
\begin{itemize}
\item Below 20 Hz: \textbf{Infrasound}
\item Above 20,000 Hz: \textbf{Ultrasound}
\end{itemize}
\note{- Human hearing: 20 Hz to 20 kHz\\\\
- Infrasound: elephants communicate, earthquakes\\\\
- Ultrasound: medical imaging, dog whistles, bats\\\\
- Hearing degrades with age (lose high frequencies first)\\\\
- Name wheel: why can dogs hear dog whistles but we cannot?\\\\
  ANSWER: dog whistles are ultrasound (above 20 kHz)}
\end{frame}

%=============================================================================
\section{14.3 Doppler Effect and Sonic Booms}
%=============================================================================

\begin{frame}
\frametitle{The Doppler Effect}
\textbf{Definition:} A change in the observed frequency of a sound due to relative motion between source and observer.

\pause
\vspace{0.5cm}

\textbf{Examples:}
\begin{itemize}
\item Ambulance siren pitch changes as it passes \pause
\item Race car engine sound \pause
\item Train whistle
\end{itemize}

\pause
\vspace{0.3cm}
\alert{Moving toward} $\rightarrow$ higher frequency (higher pitch)

\alert{Moving away} $\rightarrow$ lower frequency (lower pitch)
\note{- Classic ambulance siren example\\\\
- Approaching: waves bunch up in front\\\\
- Receding: waves stretch out behind\\\\
- Name wheel: what happens to siren pitch as ambulance passes?\\\\
  ANSWER: pitch drops suddenly (high to low)\\\\
- Mnemonic: toward = tall pitch, away = awful pitch}
\end{frame}

\begin{frame}
\frametitle{Stationary Source}
\begin{figure}
\centering
\includegraphics[width=0.55\textwidth]{phys11-sound-fig31-1.jpg}
\end{figure}

\pause
When source and observer are stationary, wavelength and frequency are the same in all directions.
\note{- Point to diagram: circles are evenly spaced\\\\
- Same wavelength in all directions\\\\
- Observer hears same frequency regardless of position\\\\
- Name wheel: are the wave circles evenly spaced?\\\\
  ANSWER: yes, source is not moving}
\end{frame}

\begin{frame}
\frametitle{Moving Source}
\begin{figure}
\centering
\includegraphics[width=0.55\textwidth]{phys11-sound-fig31.jpg}
\end{figure}

\pause
\begin{itemize}
\item Waves compress in front (shorter $\lambda$, higher $f$) \pause
\item Waves stretch behind (longer $\lambda$, lower $f$)
\end{itemize}
\note{- Point to diagram: circles bunched in front\\\\
- Source catches up to its own waves\\\\
- Shorter lambda = higher frequency (v = f*lambda)\\\\
- Name wheel: which observer hears higher pitch, front or back?\\\\
  ANSWER: front observer (shorter wavelength = higher f)}
\end{frame}

\begin{frame}
\frametitle{Doppler Effect Equation - Moving Source}
For a \textbf{stationary observer} and \textbf{moving source}:

\pause
\begin{center}
\Huge
$\boxed{f_{obs} = f_s\left(\frac{v_w}{v_w \pm v_s}\right)}$
\end{center}

\pause
\vspace{0.3cm}
\textbf{Where:}
\begin{itemize}
\item $f_{obs}$ = observed frequency
\item $f_s$ = source frequency
\item $v_w$ = speed of sound
\item $v_s$ = speed of source
\end{itemize}

\pause
\vspace{0.3cm}
\textbf{Sign convention:}
\begin{itemize}
\item \alert{Minus (-)}: source moving \textbf{toward} observer
\item \alert{Plus (+)}: source moving \textbf{away} from observer
\end{itemize}
\note{- f_obs = observed frequency, f_s = source frequency\\\\
- v_w = wave speed in medium, v_s = source speed\\\\
- SIGN TRICK: toward = smaller denominator = higher f\\\\
- Mnemonic: minus for moving toward (both start with m)\\\\
- Name wheel: if source moves toward, denominator gets...?\\\\
  ANSWER: smaller (vw - vs), so f_obs gets bigger}
\end{frame}

\begin{frame}
\frametitle{Doppler Effect - Moving Observer}
For a \textbf{moving observer} and \textbf{stationary source}:

\pause
\begin{center}
\Large
$f_{obs} = f_s\left(\frac{v_w \pm v_{obs}}{v_w}\right)$
\end{center}

\pause
\vspace{0.3cm}
\textbf{Sign convention:}
\begin{itemize}
\item \alert{Plus (+)}: observer moving \textbf{toward} source
\item \alert{Minus (-)}: observer moving \textbf{away} from source
\end{itemize}
\note{- v_obs in numerator this time\\\\
- OPPOSITE signs from moving source!\\\\
- Plus for toward (bigger numerator = higher f)\\\\
- Name wheel: why opposite signs from moving source?\\\\
  ANSWER: speed is in numerator not denominator}
\end{frame}

%-----------------------------------------------------------------------------
% I Do: Train Horn
%-----------------------------------------------------------------------------

\begin{frame}
\frametitle{I Do: Train Horn Doppler Shift}
\textbf{Problem:}\\
A train with a 150 Hz horn moves at 35 m/s when the speed of sound is 340 m/s. What frequencies does a stationary observer hear as the train approaches and recedes?

\pause
\vspace{0.3cm}

\begin{columns}[T]
\column{0.48\textwidth}
\textbf{G - Givens}
\begin{itemize}
\item $f_s = 150$ Hz
\item $v_s = 35$ m/s
\item $v_w = 340$ m/s
\end{itemize}

\pause
\column{0.48\textwidth}
\textbf{U - Unknown}
\begin{itemize}
\item $f_{approaching} = ?$
\item $f_{receding} = ?$
\end{itemize}
\end{columns}
\note{- Two-part problem: approaching and receding\\\\
- Name wheel: what are the givens?\\\\
  ANSWER: fs=150 Hz, vs=35 m/s, vw=340 m/s\\\\
- Name wheel: which equation? moving source or observer?\\\\
  ANSWER: moving source (train moves, observer stationary)}
\end{frame}

\begin{frame}
\frametitle{I Do: Substitute and Solve}
\textbf{Approaching (use minus):}
$$f_{obs} = 150\left(\frac{340}{340 - 35}\right) = 150\left(\frac{340}{305}\right)$$

\pause
$$\boxed{f_{approaching} = 167 \text{ Hz} \approx 170 \text{ Hz}}$$

\pause
\vspace{0.3cm}

\textbf{Receding (use plus):}
$$f_{obs} = 150\left(\frac{340}{340 + 35}\right) = 150\left(\frac{340}{375}\right)$$

\pause
$$\boxed{f_{receding} = 136 \text{ Hz} \approx 140 \text{ Hz}}$$
\note{- Approaching: minus in denominator (340-35=305)\\\\
- 150 * (340/305) = 167 Hz (higher pitch)\\\\
- Receding: plus in denominator (340+35=375)\\\\
- 150 * (340/375) = 136 Hz (lower pitch)\\\\
- ANSWER: 170 Hz approaching, 140 Hz receding\\\\
- Shift is about 30 Hz total (noticeable!)}
\end{frame}

%-----------------------------------------------------------------------------
% We Do: Doppler
%-----------------------------------------------------------------------------

\begin{frame}
\frametitle{We Do: Doppler Calculation}
\textbf{Problem:}\\
What is the observed frequency when a source with frequency 3.0 kHz moves toward the observer at 100 m/s? (Speed of sound = 331 m/s)

\pause
\vspace{0.5cm}

\begin{columns}[T]
\column{0.48\textwidth}
\textbf{Givens}
\begin{itemize}
\item $f_s = 3000$ Hz
\item $v_s = 100$ m/s
\item $v_w = 331$ m/s
\end{itemize}

\column{0.48\textwidth}
\textbf{Unknown}
\begin{itemize}
\item $f_{obs} = ?$
\end{itemize}
\end{columns}

\pause
\vspace{0.5cm}
\alert{Hint:} Moving toward = use minus sign in denominator
\note{- Name wheel: what are the givens?\\\\
  ANSWER: fs=3000 Hz, vs=100 m/s, vw=331 m/s\\\\
- Name wheel: moving toward, plus or minus?\\\\
  ANSWER: minus in denominator\\\\
- Name wheel: write the equation with numbers\\\\
  ANSWER: f = 3000 * (331 / (331-100))}
\end{frame}

\begin{frame}
\frametitle{We Do: Solution}
$$f_{obs} = f_s\left(\frac{v_w}{v_w - v_s}\right)$$

\pause
\vspace{0.3cm}

$$f_{obs} = 3000\left(\frac{331}{331 - 100}\right) = 3000\left(\frac{331}{231}\right)$$

\pause
\vspace{0.3cm}

$$\boxed{f_{obs} = 4300 \text{ Hz} = 4.3 \text{ kHz}}$$
\note{- Name wheel: what is 331-100?\\\\
  ANSWER: 231\\\\
- 3000 * (331/231) = 4298 Hz\\\\
- ANSWER: 4300 Hz or 4.3 kHz\\\\
- Reasonableness: higher than source (3 kHz) because moving toward}
\end{frame}

%-----------------------------------------------------------------------------
% Sonic Boom
%-----------------------------------------------------------------------------

\begin{frame}
\frametitle{Sonic Booms}
\begin{figure}
\centering
\includegraphics[width=0.5\textwidth]{phys11-sound-fig34.jpg}
\end{figure}

\pause
\textbf{Sonic boom:} Constructive interference of sound created by an object moving \alert{faster than sound}.

\pause
\vspace{0.3cm}
\begin{itemize}
\item All sound waves pile up at once
\item Creates intense pressure wave
\item Aircraft creates two booms (nose and tail)
\end{itemize}
\note{- Faster than sound = supersonic\\\\
- Mach 1 = speed of sound (about 340 m/s or 1235 km/h)\\\\
- All waves created pile up in a cone (Mach cone)\\\\
- Thunder is a natural sonic boom (lightning is supersonic)\\\\
- Name wheel: why two booms from an aircraft?\\\\
  ANSWER: one from nose, one from tail}
\end{frame}

%-----------------------------------------------------------------------------
% You Do: Doppler
%-----------------------------------------------------------------------------

\begin{frame}
\frametitle{You Do: Train Whistle}
\textbf{Problem:}\\
A train is moving away from you at 50.0 m/s. If you hear the whistle at 305 Hz, what is the actual frequency of the whistle? (Speed of sound = 331 m/s)

\vspace{0.5cm}

\textbf{Given:}
\begin{itemize}
\item $f_{obs} = 305$ Hz
\item $v_s = 50.0$ m/s
\item $v_w = 331$ m/s
\end{itemize}

\textbf{Find:} $f_s$ (actual frequency)

\vspace{0.5cm}

\textbf{Hint:} Rearrange $f_{obs} = f_s\left(\frac{v_w}{v_w + v_s}\right)$ to solve for $f_s$.
\note{- 5 min independent practice\\\\
- Walk around and check progress\\\\
- FULL SOLUTION:\\\\
  Equation: f_obs = fs * (vw / (vw + vs))\\\\
  Rearrange: fs = f_obs * (vw + vs) / vw\\\\
  Substitute: fs = 305 * (331 + 50) / 331\\\\
  fs = 305 * (381/331) = 305 * 1.151\\\\
  ANSWER: fs = 351 Hz\\\\
- Reasonableness: higher than observed (moving away lowers pitch)\\\\
- Fast finishers: what if train was approaching?}
\end{frame}

%=============================================================================
\section{14.4 Sound Interference and Resonance}
%=============================================================================

\begin{frame}
\frametitle{Resonance}
\begin{figure}
\centering
\includegraphics[width=0.4\textwidth]{phys11-sound-fig39.jpg}
\end{figure}

\pause
\textbf{Resonance:} Driving a system at its \alert{natural frequency} causes maximum amplitude oscillations.

\pause
\vspace{0.3cm}
\textbf{Examples:}
\begin{itemize}
\item Pushing a child on a swing at the right rhythm
\item Piano strings vibrating in response to your voice
\item Tuning a radio to a specific frequency
\end{itemize}
\note{- Resonance = natural frequency matching driving frequency\\\\
- Small pushes at right time = big oscillations\\\\
- Tacoma Narrows Bridge collapse (wind resonance)\\\\
- Name wheel: what happens if you push a swing at the wrong time?\\\\
  ANSWER: you slow it down or cancel the motion}
\end{frame}

\begin{frame}
\frametitle{Beats}
\begin{figure}
\centering
\includegraphics[width=0.7\textwidth]{phys11-sound-fig41.jpg}
\end{figure}

\pause
\textbf{Beats:} Produced by superposition of two waves with slightly different frequencies.
\note{- Point to diagram: two waves slightly different frequencies\\\\
- Sometimes in phase (loud), sometimes out of phase (quiet)\\\\
- Creates wah-wah-wah sound\\\\
- Name wheel: what happens when waves are in phase?\\\\
  ANSWER: constructive interference, louder}
\end{frame}

\begin{frame}
\frametitle{Beat Frequency}
\begin{center}
\Huge
$\boxed{f_B = |f_1 - f_2|}$
\end{center}

\pause
\vspace{0.5cm}

\textbf{Applications:}
\begin{itemize}
\item Piano tuning - adjust until beats disappear
\item Guitar tuning
\item Musical instrument manufacturing
\end{itemize}

\pause
\vspace{0.3cm}
\textit{When two frequencies are similar, we hear the average frequency getting louder and softer at the beat frequency.}
\note{- fB = absolute value of difference\\\\
- Piano tuning: adjust until beats disappear (f1 = f2)\\\\
- Beats per second = difference in Hz\\\\
- Name wheel: if f1=440 Hz and f2=442 Hz, beat frequency?\\\\
  ANSWER: 2 Hz (2 beats per second)}
\end{frame}

\begin{frame}
\frametitle{Standing Waves in Tubes}
\begin{figure}
\centering
\includegraphics[width=0.6\textwidth]{phys11-sound-fig42.jpg}
\end{figure}

\pause
Sound waves reflect off closed ends and interfere to create \textbf{standing waves}.
\note{- Sound reflects off closed end, interferes with incoming\\\\
- Creates nodes (no motion) and antinodes (max motion)\\\\
- Musical instruments use this to produce specific notes\\\\
- Name wheel: what is a standing wave?\\\\
  ANSWER: wave that appears stationary due to interference}
\end{frame}

\begin{frame}
\frametitle{Closed-Pipe Resonator}
\begin{figure}
\centering
\includegraphics[width=0.55\textwidth]{phys11-sound-fig44.jpg}
\end{figure}

\pause
\textbf{Resonant frequencies:}
\begin{center}
\Large
$\boxed{f_n = n\frac{v}{4L}, \quad n = 1, 3, 5, ...}$
\end{center}

\pause
\textbf{Only odd harmonics!} (Node at closed end, antinode at open end)
\note{- Closed end = node (air can't move)\\\\
- Open end = antinode (air moves freely)\\\\
- fn = n * v / 4L where n = 1, 3, 5...\\\\
- Only ODD harmonics for closed pipe\\\\
- Name wheel: why only odd harmonics?\\\\
  ANSWER: need 1/4 wavelength to fit node-to-antinode}
\end{frame}

\begin{frame}
\frametitle{Open-Pipe Resonator}
\begin{figure}
\centering
\includegraphics[width=0.55\textwidth]{phys11-sound-fig45.jpg}
\end{figure}

\pause
\textbf{Resonant frequencies:}
\begin{center}
\Large
$\boxed{f_n = n\frac{v}{2L}, \quad n = 1, 2, 3, ...}$
\end{center}

\pause
\textbf{All harmonics!} (Antinodes at both ends)
\note{- Both ends open = antinodes at both ends\\\\
- fn = n * v / 2L where n = 1, 2, 3...\\\\
- ALL harmonics for open pipe\\\\
- Flute is an open pipe, clarinet is closed\\\\
- Name wheel: what is the fundamental wavelength for open pipe?\\\\
  ANSWER: lambda = 2L (half wavelength fits in pipe)}
\end{frame}

\begin{frame}
\frametitle{Comparing Resonators}
\begin{columns}[T]
\column{0.48\textwidth}
\textbf{Closed Pipe}
\begin{itemize}
\item One end closed
\item Only odd harmonics
\item $f_n = n\frac{v}{4L}$
\item $n = 1, 3, 5...$
\end{itemize}

\pause
\column{0.48\textwidth}
\textbf{Open Pipe}
\begin{itemize}
\item Both ends open
\item All harmonics
\item $f_n = n\frac{v}{2L}$
\item $n = 1, 2, 3...$
\end{itemize}
\end{columns}

\pause
\vspace{0.5cm}
\textbf{Note:} An open pipe has twice the fundamental frequency of a closed pipe of the same length!
\note{- KEY COMPARISON: closed = 4L, open = 2L\\\\
- Same length pipe: open is 2x frequency (octave higher)\\\\
- Closed has only odd harmonics (different timbre)\\\\
- Name wheel: which pipe has richer sound (more harmonics)?\\\\
  ANSWER: open pipe (has all harmonics, not just odd)}
\end{frame}

%-----------------------------------------------------------------------------
% I Do: Tube Length
%-----------------------------------------------------------------------------

\begin{frame}
\frametitle{I Do: Finding Tube Length}
\textbf{Problem:}\\
If sound travels at 344 m/s, what should be the length of a tube closed at one end to have a fundamental frequency of 128 Hz?

\pause
\vspace{0.3cm}

\begin{columns}[T]
\column{0.48\textwidth}
\textbf{G - Givens}
\begin{itemize}
\item $f_1 = 128$ Hz
\item $v = 344$ m/s
\item $n = 1$ (fundamental)
\end{itemize}

\pause
\column{0.48\textwidth}
\textbf{U - Unknown}
\begin{itemize}
\item $L = ?$
\end{itemize}
\end{columns}
\note{- Closed pipe = use fn = nv/4L\\\\
- Fundamental means n = 1\\\\
- Name wheel: what are the givens?\\\\
  ANSWER: f1=128 Hz, v=344 m/s, n=1\\\\
- Name wheel: which equation? closed or open?\\\\
  ANSWER: closed pipe, fn = nv/4L}
\end{frame}

\begin{frame}
\frametitle{I Do: Substitute and Solve}
\textbf{E - Equation:} $f_1 = \frac{v}{4L}$

\pause
\vspace{0.3cm}

\textbf{Rearrange:} $L = \frac{v}{4f_1}$

\pause
\vspace{0.3cm}

\textbf{S - Substitute:} $L = \frac{344 \text{ m/s}}{4(128 \text{ Hz})}$

\pause
\vspace{0.3cm}

\textbf{S - Solve:}
$$\boxed{L = 0.672 \text{ m} \approx 67 \text{ cm}}$$
\note{- Algebra: L = v / (4 * f1)\\\\
- Substitute: L = 344 / (4 * 128) = 344 / 512\\\\
- ANSWER: L = 0.672 m = 67 cm\\\\
- Reasonableness: about 2 feet, reasonable for a tube\\\\
- 128 Hz is middle C on piano}
\end{frame}

%-----------------------------------------------------------------------------
% We Do: Third Overtone
%-----------------------------------------------------------------------------

\begin{frame}
\frametitle{We Do: Finding Overtone}
\textbf{Problem:}\\
If a tube open at both ends has a fundamental frequency of 120 Hz, what is the frequency of its third overtone?

\pause
\vspace{0.5cm}

\textbf{Key insight:} For an open pipe:
\begin{itemize}
\item Fundamental = 1st harmonic ($n=1$)
\item 1st overtone = 2nd harmonic ($n=2$)
\item 2nd overtone = 3rd harmonic ($n=3$)
\item \alert{3rd overtone = 4th harmonic ($n=4$)}
\end{itemize}

\pause
\vspace{0.3cm}

Since $f_n = nf_1$:
$$f_4 = 4f_1 = 4(120 \text{ Hz}) = \boxed{480 \text{ Hz}}$$
\note{- TERMINOLOGY: overtone vs harmonic\\\\
- Fundamental = 1st harmonic = no overtones\\\\
- 1st overtone = 2nd harmonic\\\\
- 3rd overtone = 4th harmonic\\\\
- Name wheel: what is the 3rd overtone number n?\\\\
  ANSWER: n = 4 (add 1 to overtone number)\\\\
- ANSWER: f4 = 4 * 120 = 480 Hz}
\end{frame}

%-----------------------------------------------------------------------------
% I Do: Beat Frequency
%-----------------------------------------------------------------------------

\begin{frame}
\frametitle{I Do: Piano Tuning with Beats}
\textbf{Problem:}\\
A piano tuner hears 2 beats per second when comparing a piano note to a 256 Hz tuning fork. What are the possible frequencies of the piano?

\pause
\vspace{0.5cm}

\begin{columns}[T]
\column{0.48\textwidth}
\textbf{G - Givens}
\begin{itemize}
\item $f_B = 2$ Hz
\item $f_2 = 256$ Hz (fork)
\end{itemize}

\pause
\column{0.48\textwidth}
\textbf{U - Unknown}
\begin{itemize}
\item $f_1 = ?$ (piano)
\end{itemize}
\end{columns}

\pause
\vspace{0.3cm}

\textbf{Solution:} Since $f_B = |f_1 - f_2|$:
$$f_1 = f_2 \pm f_B = 256 \pm 2$$
$$\boxed{f_1 = 258 \text{ Hz or } 254 \text{ Hz}}$$
\note{- TWO possible answers (plus or minus)\\\\
- Piano could be sharp (258) or flat (254)\\\\
- Tuner adjusts tension to make beats disappear\\\\
- When fB = 0, piano is in tune\\\\
- ANSWER: 258 Hz or 254 Hz\\\\
- Name wheel: how would tuner know which one?\\\\
  ANSWER: tighten string, if beats slow down, it was flat}
\end{frame}

%-----------------------------------------------------------------------------
% You Do: Beat Frequency
%-----------------------------------------------------------------------------

\begin{frame}
\frametitle{You Do: Beat Frequency}
\textbf{Problem:}\\
Two sound waves have frequencies 250 Hz and 280 Hz. What is the beat frequency produced by their superposition?

\vspace{0.5cm}

\textbf{Given:}
\begin{itemize}
\item $f_1 = 250$ Hz
\item $f_2 = 280$ Hz
\end{itemize}

\textbf{Find:} Beat frequency $f_B$

\vspace{0.5cm}

\textbf{Hint:} Use $f_B = |f_1 - f_2|$
\note{- 2 min independent practice\\\\
- Walk around and check progress\\\\
- FULL SOLUTION:\\\\
  Equation: fB = |f1 - f2|\\\\
  Substitute: fB = |250 - 280|\\\\
  ANSWER: fB = 30 Hz (30 beats per second)\\\\
- Reasonableness: 30 Hz is fast but audible as a buzz\\\\
- Fast finishers: would you hear this as beats or a chord?}
\end{frame}

%=============================================================================
\section{Summary}
%=============================================================================

\begin{frame}
\frametitle{Chapter Summary}
\textbf{Key Takeaways:}

\pause
\begin{enumerate}
\item Sound is a longitudinal wave with compressions and rarefactions \pause
\item Speed of sound depends on medium properties (rigidity and density) \pause
\item Sound intensity is proportional to amplitude squared \pause
\item Decibels measure sound intensity level logarithmically \pause
\item Doppler effect: frequency changes with relative motion \pause
\item Resonance occurs when driving frequency equals natural frequency \pause
\item Standing waves in pipes produce harmonics
\end{enumerate}
\note{- 7 key takeaways from CH14\\\\
- Most important: v=f*lambda and decibel equation\\\\
- Doppler for moving sources (ambulance, train)\\\\
- Resonance for music and instrument design\\\\
- Name wheel: which concept was most surprising?\\\\
- HW: textbook problems 14.1-14.4}
\end{frame}

\begin{frame}
\frametitle{Key Equations}
\begin{align}
v &= f\lambda \quad \text{(wave equation)} \\[0.3cm]
I &= \frac{P}{A} \quad \text{(intensity)} \\[0.3cm]
\beta &= 10\log_{10}\left(\frac{I}{I_0}\right) \quad \text{(decibels)} \\[0.3cm]
f_{obs} &= f_s\left(\frac{v_w}{v_w \pm v_s}\right) \quad \text{(Doppler - moving source)} \\[0.3cm]
f_B &= |f_1 - f_2| \quad \text{(beat frequency)} \\[0.3cm]
f_n &= n\frac{v}{4L} \quad \text{(closed pipe, } n=1,3,5...) \\[0.3cm]
f_n &= n\frac{v}{2L} \quad \text{(open pipe, } n=1,2,3...)
\end{align}
\note{- These go on formula sheet\\\\
- v=f*lambda: most versatile, use for any wave problem\\\\
- I=P/A: intensity from power\\\\
- Decibel: logarithmic scale conversion\\\\
- Doppler: sign convention is key\\\\
- Pipes: closed = 4L (odd n), open = 2L (all n)\\\\
- Name wheel: when would you use the Doppler equation?\\\\
  ANSWER: when source or observer is moving}
\end{frame}

\begin{frame}
\frametitle{Questions?}
\begin{center}
\Large
Review the I Do / We Do / You Do problems\\[0.5cm]
Practice with the textbook problems\\[0.5cm]
\pause
\textbf{Next: Chapter 15 - Light}
\end{center}
\note{- Questions before we end?\\\\
- Key skill: GUESS method with algebra focus\\\\
- Practice problems in textbook\\\\
- Next class: Chapter 15 Light (electromagnetic waves)\\\\
- Preview: light as wave, reflection, refraction}
\end{frame}

\end{document}
