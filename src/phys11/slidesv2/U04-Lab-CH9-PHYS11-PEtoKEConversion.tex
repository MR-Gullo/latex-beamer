% Options for packages loaded elsewhere
\PassOptionsToPackage{unicode}{hyperref}
\PassOptionsToPackage{hyphens}{url}
\documentclass[
]{article}
\usepackage{xcolor}
\usepackage{amsmath,amssymb}
\setcounter{secnumdepth}{-\maxdimen} % remove section numbering
\usepackage{iftex}
\ifPDFTeX
  \usepackage[T1]{fontenc}
  \usepackage[utf8]{inputenc}
  \usepackage{textcomp} % provide euro and other symbols
\else % if luatex or xetex
  \usepackage{unicode-math} % this also loads fontspec
  \defaultfontfeatures{Scale=MatchLowercase}
  \defaultfontfeatures[\rmfamily]{Ligatures=TeX,Scale=1}
\fi
\usepackage{lmodern}
\ifPDFTeX\else
  % xetex/luatex font selection
\fi
% Use upquote if available, for straight quotes in verbatim environments
\IfFileExists{upquote.sty}{\usepackage{upquote}}{}
\IfFileExists{microtype.sty}{% use microtype if available
  \usepackage[]{microtype}
  \UseMicrotypeSet[protrusion]{basicmath} % disable protrusion for tt fonts
}{}
\makeatletter
\@ifundefined{KOMAClassName}{% if non-KOMA class
  \IfFileExists{parskip.sty}{%
    \usepackage{parskip}
  }{% else
    \setlength{\parindent}{0pt}
    \setlength{\parskip}{6pt plus 2pt minus 1pt}}
}{% if KOMA class
  \KOMAoptions{parskip=half}}
\makeatother
\usepackage{longtable,booktabs,array}
\newcounter{none} % for unnumbered tables
\usepackage{calc} % for calculating minipage widths
% Correct order of tables after \paragraph or \subparagraph
\usepackage{etoolbox}
\makeatletter
\patchcmd\longtable{\par}{\if@noskipsec\mbox{}\fi\par}{}{}
\makeatother
% Allow footnotes in longtable head/foot
\IfFileExists{footnotehyper.sty}{\usepackage{footnotehyper}}{\usepackage{footnote}}
\makesavenoteenv{longtable}
\setlength{\emergencystretch}{3em} % prevent overfull lines
\providecommand{\tightlist}{%
  \setlength{\itemsep}{0pt}\setlength{\parskip}{0pt}}
\usepackage{bookmark}
\IfFileExists{xurl.sty}{\usepackage{xurl}}{} % add URL line breaks if available
\urlstyle{same}
\hypersetup{
  hidelinks,
  pdfcreator={LaTeX via pandoc}}

\author{}
\date{}

\begin{document}

\textbf{Converting Potential Energy to Kinetic Energy}

In this activity, you will calculate the potential energy of an object
and predict the object\textquotesingle s speed when all that potential
energy has been converted to kinetic energy. You will then check your
prediction.

\textbf{SAFETY WARNING:} You will be dropping objects from a height. Be
sure to stay a safe distance from the edge. Don\textquotesingle t lean
over the railing too far. Make sure that you do not drop objects into an
area where people or vehicles pass by. Make sure that dropping objects
will not cause damage.

Materials Needed

For each pair of students:

- Four bouncy balls (or similar small, dense objects)

- Stopwatch

For the class:

- Metric measuring tape long enough to measure the chosen height

- A scale

Procedure

1. Work with a partner. Find and record the mass of four small, dense
objects per group.

2. Choose a location where the objects can be safely dropped from a
height of at least 15 meters. A bridge over water with a safe pedestrian
walkway will work well.

3. Measure the distance the object will fall.

4. Calculate the potential energy of the object before you drop it using
PE = mgh = (9.80)mh.

5. Predict the kinetic energy and velocity of the object when it lands
using PE = KE, so KE = ½mv² and v = √(2gh).

6. One partner drops the object while the other measures the time it
takes to fall.

7. Take turns being the dropper and the timer until you have made four
measurements.

8. Average your drop times and calculate the velocity of the object when
it landed using v = at = gt = (9.80)t.

9. Compare your results to your prediction.

GRASP CHECK

Question: Galileo\textquotesingle s experiments proved that, contrary to
popular belief, heavy objects do not fall faster than light objects. How
do the equations you used support this fact?

a) Heavy objects do not fall faster than light objects because while
conserving the mechanical energy of the system, the mass term gets
cancelled and the velocity is independent of the mass. In real life, the
variation in velocity of different objects is observed because of
non-zero air resistance.

b) Heavy objects do not fall faster than light objects because while
conserving the mechanical energy of the system, the mass term does not
get cancelled and the velocity is dependent on the mass. In real life,
the variation in velocity of different objects is observed because of
non-zero air resistance.

c) Heavy objects do not fall faster than light objects because while
conserving the mechanical energy of the system, the mass term gets
cancelled and the velocity is independent of the mass. In real life, the
variation in velocity of different objects is observed because of zero
air resistance.

d) Heavy objects do not fall faster than light objects because while
conserving the mechanical energy of the system, the mass term does not
get cancelled and the velocity is dependent on the mass. In real life,
the variation in velocity of different objects is observed because of
zero air resistance.

Student Submission Template: Converting Potential Energy to Kinetic
Energy

Name: \_\_\_\_\_\_\_\_\_\_\_\_\_\_\_\_\_\_\_\_\_\_\_\_\_\_\_\_\_\_\_\_

Partner(s):
\_\_\_\_\_\_\_\_\_\_\_\_\_\_\_\_\_\_\_\_\_\_\_\_\_\_\_\_\_\_\_\_

Date: \_\_\_\_\_\_\_\_\_\_\_\_\_\_\_\_\_\_\_\_\_\_\_\_\_\_\_\_\_\_\_\_

Class/Period:
\_\_\_\_\_\_\_\_\_\_\_\_\_\_\_\_\_\_\_\_\_\_\_\_\_\_\_\_\_\_\_\_

Pre-Lab Safety Checklist

- {[} {]} Safety location identified (≥15 m height, no
pedestrian/vehicle traffic below)

- {[} {]} Adult supervision confirmed (if required)

- {[} {]} Four bouncy balls obtained

- {[} {]} Stopwatch available and tested

- {[} {]} Measuring tape (≥15 m) available

- {[} {]} Scale available for mass measurements

- {[} {]} Clear drop zone established and marked

- {[} {]} Camera/phone available for photos

Drop Location Information

Location description:
\_\_\_\_\_\_\_\_\_\_\_\_\_\_\_\_\_\_\_\_\_\_\_\_\_\_\_\_\_\_\_\_\_\_\_\_\_\_\_\_\_\_\_\_\_\_\_\_\_

Measured drop height (h): \_\_\_\_\_\_\_ meters

Weather/Conditions:
\_\_\_\_\_\_\_\_\_\_\_\_\_\_\_\_\_\_\_\_\_\_\_\_\_\_\_\_\_\_\_\_\_\_\_\_\_\_\_\_\_\_\_\_\_\_\_\_\_

Mass Measurements

Mass of Bouncy Ball 1: \_\_\_\_\_\_\_ kg

Mass of Bouncy Ball 2: \_\_\_\_\_\_\_ kg

Mass of Bouncy Ball 3: \_\_\_\_\_\_\_ kg

Mass of Bouncy Ball 4: \_\_\_\_\_\_\_ kg

Procedure Notes

Document your drop technique, timing method, any challenges with wind or
air resistance, or observations about ball behavior:

\_\_\_\_\_\_\_\_\_\_\_\_\_\_\_\_\_\_\_\_\_\_\_\_\_\_\_\_\_\_\_\_\_\_\_\_\_\_\_\_\_\_\_\_\_\_\_\_\_\_\_\_\_\_\_\_\_\_\_\_\_\_\_

\_\_\_\_\_\_\_\_\_\_\_\_\_\_\_\_\_\_\_\_\_\_\_\_\_\_\_\_\_\_\_\_\_\_\_\_\_\_\_\_\_\_\_\_\_\_\_\_\_\_\_\_\_\_\_\_\_\_\_\_\_\_\_

\_\_\_\_\_\_\_\_\_\_\_\_\_\_\_\_\_\_\_\_\_\_\_\_\_\_\_\_\_\_\_\_\_\_\_\_\_\_\_\_\_\_\_\_\_\_\_\_\_\_\_\_\_\_\_\_\_\_\_\_\_\_\_

\textbf{Data Collection}

\paragraph{\texorpdfstring{ T\textbf{able: Drop Data for All
Objects}}{ Table: Drop Data for All Objects}}\label{table-drop-data-for-all-objects}

{\def\LTcaptype{none} % do not increment counter
\begin{longtable}[]{@{}
  >{\raggedright\arraybackslash}p{(\linewidth - 12\tabcolsep) * \real{0.0917}}
  >{\raggedright\arraybackslash}p{(\linewidth - 12\tabcolsep) * \real{0.1115}}
  >{\raggedright\arraybackslash}p{(\linewidth - 12\tabcolsep) * \real{0.1598}}
  >{\raggedright\arraybackslash}p{(\linewidth - 12\tabcolsep) * \real{0.2144}}
  >{\raggedright\arraybackslash}p{(\linewidth - 12\tabcolsep) * \real{0.0654}}
  >{\raggedright\arraybackslash}p{(\linewidth - 12\tabcolsep) * \real{0.1258}}
  >{\raggedright\arraybackslash}p{(\linewidth - 12\tabcolsep) * \real{0.2314}}@{}}
\toprule\noalign{}
\begin{minipage}[b]{\linewidth}\centering
\textbf{Object}
\end{minipage} & \begin{minipage}[b]{\linewidth}\centering
\textbf{Mass (kg)}
\end{minipage} & \begin{minipage}[b]{\linewidth}\centering
\textbf{Predicted PE (J)}
\end{minipage} & \begin{minipage}[b]{\linewidth}\centering
\textbf{Predicted Velocity (m/s)}
\end{minipage} & \begin{minipage}[b]{\linewidth}\centering
\textbf{Trial}
\end{minipage} & \begin{minipage}[b]{\linewidth}\centering
\textbf{Drop Time (s)}
\end{minipage} & \begin{minipage}[b]{\linewidth}\centering
\textbf{Calculated Velocity (m/s)}
\end{minipage} \\
\midrule\noalign{}
\endhead
\bottomrule\noalign{}
\endlastfoot
Ball \#1 & & & & 1 & & \\
& & & & 2 & & \\
& & & & 3 & & \\
& & & & 4 & & \\
& \textbf{Average} & \textbf{---} & \textbf{---} & & & \\
Ball \#2 & & & & 1 & & \\
& & & & 2 & & \\
& & & & 3 & & \\
& & & & 4 & & \\
& \textbf{Average} & \textbf{---} & \textbf{---} & & & \\
Ball \#3 & & & & 1 & & \\
& & & & 2 & & \\
& & & & 3 & & \\
& & & & 4 & & \\
& \textbf{Average} & \textbf{---} & \textbf{---} & & & \\
Ball \#4 & & & & 1 & & \\
& & & & 2 & & \\
& & & & 3 & & \\
& & & & 4 & & \\
& \textbf{Average} & \textbf{---} & \textbf{---} & & & \\
\end{longtable}
}

Sample Calculations

Show your complete work for one object (Ball 1):

Prediction Calculations:

Mass (m) = \_\_\_\_\_\_\_ kg

Height (h) = \_\_\_\_\_\_\_ m

Potential Energy: PE = mgh = \_\_\_\_\_\_\_ × 9.80 × \_\_\_\_\_\_\_ =
\_\_\_\_\_\_\_ J

Predicted Velocity: v = √(2gh) = √(2 × 9.80 × \_\_\_\_\_\_\_) =
\_\_\_\_\_\_\_ m/s

Experimental Calculations:

Average Drop Time = \_\_\_\_\_\_\_ s

Calculated Velocity: v = gt = 9.80 × \_\_\_\_\_\_\_ = \_\_\_\_\_\_\_ m/s

Percent Error:

\% Error = \textbar(Predicted - Experimental)\textbar{} / Predicted ×
100 = \_\_\_\_\_\_\_ \%

Comparison and Analysis

1. How did your predicted velocities compare to your experimentally
calculated velocities? Calculate percent error for each ball.

\_\_\_\_\_\_\_\_\_\_\_\_\_\_\_\_\_\_\_\_\_\_\_\_\_\_\_\_\_\_\_\_\_\_\_\_\_\_\_\_\_\_\_\_\_\_\_\_\_\_\_\_\_\_\_\_\_\_\_\_\_\_\_

\_\_\_\_\_\_\_\_\_\_\_\_\_\_\_\_\_\_\_\_\_\_\_\_\_\_\_\_\_\_\_\_\_\_\_\_\_\_\_\_\_\_\_\_\_\_\_\_\_\_\_\_\_\_\_\_\_\_\_\_\_\_\_

\_\_\_\_\_\_\_\_\_\_\_\_\_\_\_\_\_\_\_\_\_\_\_\_\_\_\_\_\_\_\_\_\_\_\_\_\_\_\_\_\_\_\_\_\_\_\_\_\_\_\_\_\_\_\_\_\_\_\_\_\_\_\_

2. What sources of error could account for differences between predicted
and experimental values?

\_\_\_\_\_\_\_\_\_\_\_\_\_\_\_\_\_\_\_\_\_\_\_\_\_\_\_\_\_\_\_\_\_\_\_\_\_\_\_\_\_\_\_\_\_\_\_\_\_\_\_\_\_\_\_\_\_\_\_\_\_\_\_

\_\_\_\_\_\_\_\_\_\_\_\_\_\_\_\_\_\_\_\_\_\_\_\_\_\_\_\_\_\_\_\_\_\_\_\_\_\_\_\_\_\_\_\_\_\_\_\_\_\_\_\_\_\_\_\_\_\_\_\_\_\_\_

\_\_\_\_\_\_\_\_\_\_\_\_\_\_\_\_\_\_\_\_\_\_\_\_\_\_\_\_\_\_\_\_\_\_\_\_\_\_\_\_\_\_\_\_\_\_\_\_\_\_\_\_\_\_\_\_\_\_\_\_\_\_\_

3. Did you observe any differences in behavior between different bouncy
balls? If so, what might explain these differences?

\_\_\_\_\_\_\_\_\_\_\_\_\_\_\_\_\_\_\_\_\_\_\_\_\_\_\_\_\_\_\_\_\_\_\_\_\_\_\_\_\_\_\_\_\_\_\_\_\_\_\_\_\_\_\_\_\_\_\_\_\_\_\_

\_\_\_\_\_\_\_\_\_\_\_\_\_\_\_\_\_\_\_\_\_\_\_\_\_\_\_\_\_\_\_\_\_\_\_\_\_\_\_\_\_\_\_\_\_\_\_\_\_\_\_\_\_\_\_\_\_\_\_\_\_\_\_

\_\_\_\_\_\_\_\_\_\_\_\_\_\_\_\_\_\_\_\_\_\_\_\_\_\_\_\_\_\_\_\_\_\_\_\_\_\_\_\_\_\_\_\_\_\_\_\_\_\_\_\_\_\_\_\_\_\_\_\_\_\_\_

GRASP CHECK

Question: Galileo\textquotesingle s experiments proved that, contrary to
popular belief, heavy objects do not fall faster than light objects. How
do the equations you used support this fact?

Answer: ☐ a ☐ b ☐ c ☐ d

Explanation of your choice:
\_\_\_\_\_\_\_\_\_\_\_\_\_\_\_\_\_\_\_\_\_\_\_\_\_\_\_\_\_\_\_\_\_\_\_\_\_\_\_\_\_\_\_\_\_\_\_\_\_

\_\_\_\_\_\_\_\_\_\_\_\_\_\_\_\_\_\_\_\_\_\_\_\_\_\_\_\_\_\_\_\_\_\_\_\_\_\_\_\_\_\_\_\_\_\_\_\_\_\_\_\_\_\_\_\_\_\_\_\_\_\_\_

\_\_\_\_\_\_\_\_\_\_\_\_\_\_\_\_\_\_\_\_\_\_\_\_\_\_\_\_\_\_\_\_\_\_\_\_\_\_\_\_\_\_\_\_\_\_\_\_\_\_\_\_\_\_\_\_\_\_\_\_\_\_\_

Photo Documentation

Instructions: Attach clear photos showing:

1. Your drop location/setup

2. The measuring tape showing the height

3. The bouncy balls you used

Label each photo with a brief description.

Photo 1 (Drop location):

{[}Insert photo here{]}

Description:
\_\_\_\_\_\_\_\_\_\_\_\_\_\_\_\_\_\_\_\_\_\_\_\_\_\_\_\_\_\_\_\_\_\_\_\_\_\_\_\_\_\_\_\_\_\_\_\_\_

Photo 2 (Height measurement):

{[}Insert photo here{]}

Description:
\_\_\_\_\_\_\_\_\_\_\_\_\_\_\_\_\_\_\_\_\_\_\_\_\_\_\_\_\_\_\_\_\_\_\_\_\_\_\_\_\_\_\_\_\_\_\_\_\_

Photo 3 (Bouncy balls):

{[}Insert photo here{]}

Description:
\_\_\_\_\_\_\_\_\_\_\_\_\_\_\_\_\_\_\_\_\_\_\_\_\_\_\_\_\_\_\_\_\_\_\_\_\_\_\_\_\_\_\_\_\_\_\_\_\_

Submsission Notes:

Submit this completed template along with photos as a single document
(FirstName\_LastName.pdf). Safety is paramount - any submissions missing
safety acknowledgment will not be accepted. Ensure calculations are
legible and all measurements include proper units.

\end{document}
