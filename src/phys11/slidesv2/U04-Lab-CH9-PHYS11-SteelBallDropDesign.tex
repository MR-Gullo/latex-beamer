% Options for packages loaded elsewhere
\PassOptionsToPackage{unicode}{hyperref}
\PassOptionsToPackage{hyphens}{url}
\documentclass[
]{article}
\usepackage{xcolor}
\usepackage{amsmath,amssymb}
\setcounter{secnumdepth}{-\maxdimen} % remove section numbering
\usepackage{iftex}
\ifPDFTeX
  \usepackage[T1]{fontenc}
  \usepackage[utf8]{inputenc}
  \usepackage{textcomp} % provide euro and other symbols
\else % if luatex or xetex
  \usepackage{unicode-math} % this also loads fontspec
  \defaultfontfeatures{Scale=MatchLowercase}
  \defaultfontfeatures[\rmfamily]{Ligatures=TeX,Scale=1}
\fi
\usepackage{lmodern}
\ifPDFTeX\else
  % xetex/luatex font selection
\fi
% Use upquote if available, for straight quotes in verbatim environments
\IfFileExists{upquote.sty}{\usepackage{upquote}}{}
\IfFileExists{microtype.sty}{% use microtype if available
  \usepackage[]{microtype}
  \UseMicrotypeSet[protrusion]{basicmath} % disable protrusion for tt fonts
}{}
\makeatletter
\@ifundefined{KOMAClassName}{% if non-KOMA class
  \IfFileExists{parskip.sty}{%
    \usepackage{parskip}
  }{% else
    \setlength{\parindent}{0pt}
    \setlength{\parskip}{6pt plus 2pt minus 1pt}}
}{% if KOMA class
  \KOMAoptions{parskip=half}}
\makeatother
\setlength{\emergencystretch}{3em} % prevent overfull lines
\providecommand{\tightlist}{%
  \setlength{\itemsep}{0pt}\setlength{\parskip}{0pt}}
\usepackage{bookmark}
\IfFileExists{xurl.sty}{\usepackage{xurl}}{} % add URL line breaks if available
\urlstyle{same}
\hypersetup{
  hidelinks,
  pdfcreator={LaTeX via pandoc}}

\author{}
\date{}

\begin{document}

\section{Lab Design Assignment: Work-Energy Investigation with Steel
Ball
Drop}\label{lab-design-assignment-work-energy-investigation-with-steel-ball-drop}

\subsection{Overview}\label{overview}

This experimental investigation has been structured to fit an 80-minute
lab period, with an additional 40-minute design period the week before.
Students will use a timer-controlled dropping mechanism and photo-gate
to study concepts from Chapter 9 on work, energy, and conservation
principles.

\subsection{Schedule}\label{schedule}

Week 1: Design Period (40 minutes)

Week 2: Lab Implementation (80 minutes)

\subsection{Required Equipment}\label{required-equipment}

\begin{itemize}
\item
  Timer-controlled dropping mechanism
\item
  Steel ball
\item
  Photo-gate timing system
\item
  Electronic balance
\item
  Calculator
\item
  Laptop with spreadsheet software (optional)
\item
  Lab notebook
\end{itemize}

\subsection{Pre-Lab Design Requirements (Week
1)}\label{pre-lab-design-requirements-week-1}

During the 40-minute design period, students should:

\begin{enumerate}
\def\labelenumi{\arabic{enumi}.}
\item
  Review textbook sections 9.1 and 9.2 to identify relevant equations
\item
  Design a systematic approach for data collection
\item
  Create a data table template
\item
  Develop a hypothesis about the relationship between drop height and
  ball velocity
\item
  Prepare prediction calculations for at least three different drop
  heights
\item
  Identify potential sources of experimental error
\end{enumerate}

\subsection{Experimental Procedure (Week
2)}\label{experimental-procedure-week-2}

\subsubsection{Setup Phase (15 minutes)}\label{setup-phase-15-minutes}

\begin{enumerate}
\def\labelenumi{\arabic{enumi}.}
\item
  Measure and record the mass of the steel ball
\item
  Mount the dropping mechanism securely above the photo-gate
\item
  Test the timing system and dropping mechanism
\item
  Establish a consistent method for measuring drop heights
\end{enumerate}

\subsubsection{Data Collection Phase (30
minutes)}\label{data-collection-phase-30-minutes}

\begin{enumerate}
\def\labelenumi{\arabic{enumi}.}
\item
  Select 5 different drop heights between 0.2m and 1.0m
\item
  For each height:

  \begin{itemize}
  \item
    Record the height measurement precisely
  \item
    Calculate the theoretical gravitational potential energy
  \item
    Predict the velocity at the photo-gate using conservation of energy
  \item
    Perform 3 trial drops, recording the photo-gate time measurements
  \item
    Calculate the actual velocity from the photo-gate data
  \end{itemize}
\end{enumerate}

\subsubsection{Analysis Phase (20
minutes)}\label{analysis-phase-20-minutes}

\begin{enumerate}
\def\labelenumi{\arabic{enumi}.}
\item
  Calculate for each trial:

  \begin{itemize}
  \item
    Initial gravitational potential energy (mgh)
  \item
    Theoretical final kinetic energy (½mv²) based on energy conservation
  \item
    Actual final kinetic energy based on measured velocity
  \item
    Percentage of energy conserved
  \item
    Work done by gravity
  \item
    Average power during the fall (work/time)
  \end{itemize}
\item
  Create a graph of potential energy vs. measured kinetic energy
\item
  Determine the slope of the best-fit line and its physical meaning
\end{enumerate}

\subsubsection{Discussion Phase (10
minutes)}\label{discussion-phase-10-minutes}

\begin{enumerate}
\def\labelenumi{\arabic{enumi}.}
\item
  Compare theoretical predictions to experimental results
\item
  Analyze discrepancies and potential energy losses
\item
  Discuss experimental uncertainties and their impact
\item
  Relate findings to the work-energy theorem and conservation principles
\end{enumerate}

\subsection{Submit:}\label{submit}

\textbf{By end of design period (Week 1):}

\begin{itemize}
\item
  Completed experimental design
\item
  Prediction calculations for at least three heights
\end{itemize}

\textbf{By end of lab period (Week 2):}

\begin{itemize}
\item
  Completed data tables
\item
  Preliminary graphs
\item
  Initial analysis of results
\end{itemize}

\textbf{Final report (due one week after lab):}

\begin{itemize}
\item
  Introduction with hypothesis
\item
  Methods section with detailed procedure
\item
  Results with data tables and graphs
\item
  Analysis connecting to work-energy theorem and conservation principles
\item
  Discussion of error sources
\item
  Conclusion evaluating the hypothesis
\end{itemize}

\subsection{Safety Considerations}\label{safety-considerations}

\begin{itemize}
\item
  Secure all equipment properly to prevent falling objects
\item
  Ensure clear space around the drop zone
\item
  Handle equipment with care to prevent damage
\end{itemize}

\section{Physics Laboratory Skills Proficiency
Rubric}\label{physics-laboratory-skills-proficiency-rubric}

\subsection{Emerging}\label{emerging}

Description: Students at this level require significant guidance to
complete laboratory tasks. They demonstrate basic understanding of
physics concepts but struggle to connect theory with experimental
practice. Work is primarily procedural and follows step-by-step
instructions with minimal independent thinking.

Skills and Abilities:

\begin{itemize}
\item
  Sets up basic laboratory equipment with teacher assistance and follows
  preset procedures
\item
  Records data with limited attention to precision or measurement
  techniques
\item
  Performs simple calculations using provided formulas without deep
  understanding of their significance
\item
  Identifies obvious sources of error but struggles to analyze their
  impact on results
\end{itemize}

\subsection{Developing}\label{developing}

Description: Students at this level can perform laboratory tasks with
some independence. They show a growing understanding of physics concepts
and can apply basic theory to experiments. They require occasional
guidance but can follow procedures and record data systematically.

Skills and Abilities:

\begin{itemize}
\item
  Sets up laboratory equipment correctly with minimal assistance and can
  calibrate basic instruments
\item
  Records data systematically with appropriate units and reasonable
  precision
\item
  Performs calculations correctly and begins to recognize the
  relationship between variables
\item
  Identifies several sources of experimental error and attempts basic
  quantitative error analysis
\end{itemize}

\subsection{Proficient}\label{proficient}

Description: Students at this level work independently in the laboratory
setting. They demonstrate solid understanding of physics concepts and
can apply theoretical principles to design and implement experiments.
They work methodically, maintain accuracy in their measurements, and can
analyze results in relation to physical laws.

Skills and Abilities:

\begin{itemize}
\item
  Sets up complex experimental apparatus independently and troubleshoots
  equipment issues
\item
  Collects data with appropriate precision, recognizing the limitations
  of measuring instruments
\item
  Performs comprehensive data analysis, including appropriate graphs and
  trend identification
\item
  Conducts thorough error analysis and can distinguish between
  systematic and random errors
\end{itemize}

\subsection{Extending}\label{extending}

Description: Students at this level demonstrate exceptional laboratory
skills and conceptual understanding. They work with complete
independence, often extending investigations beyond requirements. They
show sophisticated understanding of physical principles and can design
novel experimental approaches to test hypotheses or explore
relationships between variables.

Skills and Abilities:

\begin{itemize}
\item
  Designs and implements original experimental procedures to investigate
  physics concepts
\item
  Develops innovative data collection strategies to maximize precision
  and minimize uncertainty
\item
  Performs advanced analysis connecting multiple physics concepts to
  explain experimental results
\item
  Applies critical thinking to evaluate experimental design and proposes
  refinements based on theoretical principles
\end{itemize}

\end{document}
