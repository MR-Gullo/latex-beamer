% Options for packages loaded elsewhere
\PassOptionsToPackage{unicode}{hyperref}
\PassOptionsToPackage{hyphens}{url}
\documentclass[
]{article}
\usepackage{xcolor}
\usepackage{amsmath,amssymb}
\setcounter{secnumdepth}{-\maxdimen} % remove section numbering
\usepackage{iftex}
\ifPDFTeX
  \usepackage[T1]{fontenc}
  \usepackage[utf8]{inputenc}
  \usepackage{textcomp} % provide euro and other symbols
\else % if luatex or xetex
  \usepackage{unicode-math} % this also loads fontspec
  \defaultfontfeatures{Scale=MatchLowercase}
  \defaultfontfeatures[\rmfamily]{Ligatures=TeX,Scale=1}
\fi
\usepackage{lmodern}
\ifPDFTeX\else
  % xetex/luatex font selection
\fi
% Use upquote if available, for straight quotes in verbatim environments
\IfFileExists{upquote.sty}{\usepackage{upquote}}{}
\IfFileExists{microtype.sty}{% use microtype if available
  \usepackage[]{microtype}
  \UseMicrotypeSet[protrusion]{basicmath} % disable protrusion for tt fonts
}{}
\makeatletter
\@ifundefined{KOMAClassName}{% if non-KOMA class
  \IfFileExists{parskip.sty}{%
    \usepackage{parskip}
  }{% else
    \setlength{\parindent}{0pt}
    \setlength{\parskip}{6pt plus 2pt minus 1pt}}
}{% if KOMA class
  \KOMAoptions{parskip=half}}
\makeatother
\setlength{\emergencystretch}{3em} % prevent overfull lines
\providecommand{\tightlist}{%
  \setlength{\itemsep}{0pt}\setlength{\parskip}{0pt}}
\usepackage{bookmark}
\IfFileExists{xurl.sty}{\usepackage{xurl}}{} % add URL line breaks if available
\urlstyle{same}
\hypersetup{
  hidelinks,
  pdfcreator={LaTeX via pandoc}}

\author{}
\date{}

\begin{document}

PHYS 11 Unit 1 Review

CH1

30. A generation is about one-third of a lifetime. Approximately how
many generations have passed since the year \$O\$ AD?

Solution

\$\$

\textbackslash text \{ history \} \textbackslash times
\textbackslash frac\{10\^{}\{11\}
\textbackslash mathrm\{\textasciitilde s\}\}\{\textbackslash text \{
history \}\} \textbackslash times \textbackslash frac\{1
\textbackslash text \{ generation \}\}\{1 / 3 \textbackslash text \{
lifetime \}\} \textbackslash times \textbackslash frac\{0.5
\textbackslash text \{ lifetime \}\}\{10\^{}\{9\}
\textbackslash mathrm\{\textasciitilde s\}\}=\textbackslash underline\{150
\textbackslash text \{ generations \}\}

\$\$

CH2

28 Suppose a train is moving along a track.

Yes or no-ls there a single, correct reference frame from which to
describe the train\textquotesingle s motion?

A Yes, there is a single, correct frame of reference because motion is a
relative term.

B Yes, there is a single, correct frame of reference which is in terms
of Earth\textquotesingle s position.

C No, there is not a single, correct frame of reference because motion
is a relative term.

D No, there is not a single, correct frame of reference because motion
is independent of frame of reference.

Solution The correct answer is (C). No, there is not a single correct
frame of reference because motion is a relative term. If you define the
frame of reference you are interested in, then you can state the motion
you are looking for.

51 Calculate that object\textquotesingle s net displacement over the
time shown.

!{[}{]}(https://cdn.mathpix.com/cropped/2024\_09\_18\_b4a969e4ccaae538488fg-25.jpg?height=804\&width=847\&top\_left\_y=308\&top\_left\_x=669)

A 540 m

B \$2,520 \textbackslash mathrm\{\textasciitilde m\}\$

C \$2,790 \textbackslash mathrm\{\textasciitilde m\}\$

D \$5,040 \textbackslash mathrm\{\textasciitilde m\}\$

Solution The correct answer is (C). The net displacement is the area
under the line, equal to the sum of the areas of the rectangle and
triangle that make up the area under the line: \$18 \textbackslash times
30+150 \textbackslash times 30 / 2=2,790
\textbackslash mathrm\{\textasciitilde m\}\$.

CH3

32 You throw a ball straight up with an initial velocity of \$15.0
\textbackslash mathrm\{\textasciitilde m\} /
\textbackslash mathrm\{s\}\$. It passes a tree branch on the way up at a
height of 7.00 m . How much additional time will pass before the ball
passes the tree branch on the way back down?

A 0.574 s

B 0.956 s

C 1.53 s

D 1.91 s

Solution The correct answer is (D). The first time found is the time
from passing the branch to the top of the arc. This time must be doubled
to find the total time.

\$\$

\textbackslash begin\{aligned\}

\& v\^{}\{2\}=v\_\{0\}\^{}\{2\}+2
a\textbackslash left(x-x\_\{0\}\textbackslash right)
\textbackslash\textbackslash{}

\& v=\textbackslash sqrt\{v\_\{0\}\^{}\{2\}+2
a\textbackslash left(x-x\_\{0\}\textbackslash right)\}
\textbackslash\textbackslash{}

\& v=\textbackslash sqrt\{(15.0
\textbackslash mathrm\{\textasciitilde m\} /
\textbackslash mathrm\{s\})\^{}\{2\}+2\textbackslash left(-9.8
\textbackslash mathrm\{\textasciitilde m\} /
\textbackslash mathrm\{s\}\^{}\{2\}\textbackslash right)(7
\textbackslash mathrm\{\textasciitilde m\}-0
\textbackslash mathrm\{\textasciitilde m\})\}=9.37
\textbackslash mathrm\{\textasciitilde m\} / \textbackslash mathrm\{s\}
\textbackslash\textbackslash{}

\& v=v\_\{0\}+a t \textbackslash\textbackslash{}

\&
t\_\{1\}=\textbackslash frac\{v-v\_\{0\}\}\{a\}=\textbackslash frac\{0
\textbackslash mathrm\{\textasciitilde m\} /
\textbackslash mathrm\{s\}-9.37
\textbackslash mathrm\{\textasciitilde m\} /
\textbackslash mathrm\{s\}\}\{-9.8
\textbackslash mathrm\{\textasciitilde m\} /
\textbackslash mathrm\{s\}\^{}\{2\}\}=0.956
\textbackslash mathrm\{\textasciitilde s\}
\textbackslash\textbackslash{}

\& t=2 t\_\{1\}=2(0.956 \textbackslash mathrm\{\textasciitilde s\})=1.91
\textbackslash mathrm\{\textasciitilde s\}

\textbackslash end\{aligned\}

\$\$

\# Comprehensive Physics Problems and Solutions

\#\# Chapter 1

\#\#\# 30

A generation is about one-third of a lifetime. Approximately how many
generations have passed since the year 0 AD?

Solution:

\$\$

\textbackslash text\{history\} \textbackslash cdot
\textbackslash frac\{10\^{}\{11\} \textbackslash text\{
s\}\}\{\textbackslash text\{history\}\} \textbackslash cdot
\textbackslash frac\{1 \textbackslash text\{ generation\}\}\{1/3
\textbackslash text\{ lifetime\}\} \textbackslash cdot
\textbackslash frac\{0.5 \textbackslash text\{ lifetime\}\}\{10\^{}\{9\}
\textbackslash text\{ s\}\} = \textbackslash underline\{150
\textbackslash text\{ generations\}\}

\$\$

Explanation:

1. Start with the total time since 0 AD (history).

2. Convert this time to seconds: \$\textbackslash frac\{10\^{}\{11\}
\textbackslash text\{ s\}\}\{\textbackslash text\{history\}\}\$

3. Convert seconds to lifetimes: \$\textbackslash frac\{0.5
\textbackslash text\{ lifetime\}\}\{10\^{}\{9\} \textbackslash text\{
s\}\}\$

4. Convert lifetimes to generations: \$\textbackslash frac\{1
\textbackslash text\{ generation\}\}\{1/3 \textbackslash text\{
lifetime\}\}\$

5. Multiply all these factors together:

\$\textbackslash text\{history\} \textbackslash cdot
\textbackslash frac\{10\^{}\{11\} \textbackslash text\{
s\}\}\{\textbackslash text\{history\}\} \textbackslash cdot
\textbackslash frac\{1 \textbackslash text\{ generation\}\}\{1/3
\textbackslash text\{ lifetime\}\} \textbackslash cdot
\textbackslash frac\{0.5 \textbackslash text\{ lifetime\}\}\{10\^{}\{9\}
\textbackslash text\{ s\}\}\$

6. This simplifies to 150 generations.

The algebra involves unit cancellation and fraction multiplication. The
trigonometry concept is not directly used in this problem.

\#\# Chapter 2

\#\#\# 28

Suppose a train is moving along a track.

Yes or no-Is there a single, correct reference frame from which to
describe the train\textquotesingle s motion?

A. Yes, there is a single, correct frame of reference because motion is
a relative term.

B. Yes, there is a single, correct frame of reference which is in terms
of Earth\textquotesingle s position.

C. No, there is not a single, correct frame of reference because motion
is a relative term.

D. No, there is not a single, correct frame of reference because motion
is independent of frame of reference.

Solution: The correct answer is **C**. No, there is not a single correct
frame of reference because motion is a relative term. If you define the
frame of reference you are interested in, then you can state the motion
you are looking for.

Explanation:

This question doesn\textquotesingle t involve direct mathematical
calculations but rather conceptual understanding of reference frames in
physics.

1. Motion is relative: The movement of an object can be described
differently depending on the observer\textquotesingle s frame of
reference.

2. Multiple valid frames: For a moving train, we could describe its
motion relative to the ground, a passenger inside the train, or even
another moving object.

3. Each frame is correct: The motion described from each of these
perspectives would be correct for that particular frame of reference.

4. No absolute frame: There isn\textquotesingle t a single, universally
"correct" frame of reference that takes precedence over others.

This concept is fundamental in physics, especially when studying
kinematics and later in relativity.

\#\#\# 51

Calculate that object\textquotesingle s net displacement over the time
shown.

!{[}{]}(https://cdn.mathpix.com/cropped/2024\_09\_18\_b4a969e4ccaae538488fg-25.jpg?height=804\&width=847\&top\_left\_y=308\&top\_left\_x=669)

A. 540 m

B. 2,520 m

C. 2,790 m

D. 5,040 m

Solution: The correct answer is **C**. The net displacement is the area
under the line, equal to the sum of the areas of the rectangle and
triangle that make up the area under the line: \$18 \textbackslash cdot
30 + 150 \textbackslash cdot 30 / 2 = 2,790 \textbackslash text\{ m\}\$.

Explanation:

1. Identify the shape: The area under the line consists of a rectangle
and a triangle.

2. Calculate the area of the rectangle:

- Width = 30 s

- Height = 18 m/s

- Area of rectangle = 18 m/s · 30 s = 540 m

3. Calculate the area of the triangle:

- Base = 30 s

- Height = 150 m/s - 18 m/s = 132 m/s

- Area of triangle = (1/2) · base · height = (1/2) · 30 s · 132 m/s =
1,980 m

4. Sum the areas:

Total area = Rectangle area + Triangle area

Total area = 540 m + 1,980 m = 2,520 m

This problem involves basic geometry (areas of rectangles and triangles)
and the concept that in a velocity-time graph, the area under the curve
represents displacement.

\#\# Chapter 3

\#\#\# 32

You throw a ball straight up with an initial velocity of 15.0 m/s. It
passes a tree branch on the way up at a height of 7.00 m. How much
additional time will pass before the ball passes the tree branch on the
way back down?

A. 0.574 s

B. 0.956 s

C. 1.53 s

D. 1.91 s

Solution: The correct answer is **D**. The first time found is the time
from passing the branch to the top of the arc. This time must be doubled
to find the total time.

\$\$

\textbackslash begin\{aligned\}

\& v\^{}2 = v\_0\^{}2 + 2a(x-x\_0) \textbackslash\textbackslash{}

\& v = \textbackslash sqrt\{v\_0\^{}2 + 2a(x-x\_0)\}
\textbackslash\textbackslash{}

\& v = \textbackslash sqrt\{(15.0 \textbackslash text\{ m/s\})\^{}2 +
2(-9.8 \textbackslash text\{ m/s\}\^{}2)(7 \textbackslash text\{ m\} - 0
\textbackslash text\{ m\})\} = 9.37 \textbackslash text\{ m/s\}
\textbackslash\textbackslash{}

\& v = v\_0 + at \textbackslash\textbackslash{}

\& t\_1 = \textbackslash frac\{v - v\_0\}\{a\} = \textbackslash frac\{0
\textbackslash text\{ m/s\} - 9.37 \textbackslash text\{ m/s\}\}\{-9.8
\textbackslash text\{ m/s\}\^{}2\} = 0.956 \textbackslash text\{ s\}
\textbackslash\textbackslash{}

\& t = 2t\_1 = 2(0.956 \textbackslash text\{ s\}) = 1.91
\textbackslash text\{ s\}

\textbackslash end\{aligned\}

\$\$

Explanation:

1. Use the equation \$v\^{}2 = v\_0\^{}2 + 2a(x-x\_0)\$ to find the
velocity at the branch:

- \$v\_0 = 15.0 \textbackslash text\{ m/s\}\$ (initial velocity)

- \$a = -9.8 \textbackslash text\{ m/s\}\^{}2\$ (acceleration due to
gravity, negative because it\textquotesingle s upward motion)

- \$x - x\_0 = 7 \textbackslash text\{ m\}\$ (height of the branch)

- Solve for \$v\$: \$v = 9.37 \textbackslash text\{ m/s\}\$

2. Use the equation \$v = v\_0 + at\$ to find the time from the branch
to the top:

- \$v = 0 \textbackslash text\{ m/s\}\$ (velocity at the top)

- \$v\_0 = 9.37 \textbackslash text\{ m/s\}\$ (velocity at the branch)

- \$a = -9.8 \textbackslash text\{ m/s\}\^{}2\$

- Solve for \$t\_1\$: \$t\_1 = 0.956 \textbackslash text\{ s\}\$

3. Double this time for the total time up and down:

\$t = 2t\_1 = 2(0.956 \textbackslash text\{ s\}) = 1.91
\textbackslash text\{ s\}\$

This problem involves kinematic equations, specifically those dealing
with constant acceleration (gravity in this case). The symmetry of the
motion (same time up as down) is a key concept here.

\end{document}
