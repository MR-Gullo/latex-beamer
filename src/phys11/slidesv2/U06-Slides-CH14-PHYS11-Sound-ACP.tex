% CH14 Sound - PHYS11 Unit 06
\documentclass[aspectratio=169]{beamer}
\usepackage{../../../shared/templates/ds9_theme}
\usepackage[overridenote]{pdfpc}
\graphicspath{{../images/}{../../shared/images/}}

\title[Sound Waves]{PHYS11 CH14: Sound}
\subtitle{Speed, Intensity, Doppler Effect, and Resonance}
\author[Mr. Gullo]{Mr. Gullo}
\date{}

\begin{document}

\frame{\titlepage
\note{[SAY] "Today we're covering Chapter 14: Sound"\\\\
[OVERVIEW] 4 sections: speed of sound, intensity, Doppler, resonance\\\\
[CONNECT] Builds on CH13 Waves - now we apply wave concepts to sound\\\\
[EMPHASIZE] Key equations: v equals f lambda, decibel formula, Doppler\\\\
[TRANSITION] Let's start with learning objectives...}
}

\begin{frame}
\frametitle{Outline}
\tableofcontents
\end{frame}

%=============================================================================
\section{Introduction}
%=============================================================================

\begin{frame}
\frametitle{Learning Objectives}
By the end of this chapter, you will be able to:
\begin{itemize}
\item Relate wave characteristics to sound waves \pause
\item Describe speed of sound in various media \pause
\item Calculate using $v = f\lambda$ \pause
\item Relate amplitude to loudness and energy \pause
\item Use the decibel scale \pause
\item Explain the Doppler effect and sonic booms \pause
\item Describe resonance, beats, and harmonics
\end{itemize}
\note{[QUICK] 7 objectives - don't dwell on each one\\\\
[SAY] "By end of chapter, you'll master v equals f lambda"\\\\
[EMPHASIZE] Most important: wave equation and decibel formula\\\\
[PREVIEW] Quiz next class on wave equation problems\\\\
[TRANSITION] Let's see what sound actually IS...}
\end{frame}

\begin{frame}
\frametitle{The Nature of Sound}
\begin{figure}
\centering
\includegraphics[width=0.6\textwidth,height=0.5\textheight,keepaspectratio]{phys11-sound-fig01.jpg}
\end{figure}

\pause
\textbf{Key Question:} If a tree falls in a forest and no one is there to hear it, does it make a sound?

\pause
\vspace{0.3cm}
\textbf{Physics answer:} Sound is a disturbance of matter transmitted from its source outward as \alert{longitudinal waves}.
\note{[SAY] "Classic philosophy question - what do you think?"\\\\
[PAUSE] Let students ponder for a moment\\\\
[EMPHASIZE] Physics answer: YES, sound waves exist whether we hear them or not\\\\
[KEY POINT] Sound needs a MEDIUM to travel - no sound in vacuum\\\\
[CHECK] Thumbs up/down: can sound travel through space?\\\\
ANSWER: No - no medium (air, water, solid) to vibrate\\\\
[TRANSITION] So how does sound travel through air?}
\end{frame}

%=============================================================================
\section{14.1 Speed of Sound, Frequency, and Wavelength}
%=============================================================================

\begin{frame}
\frametitle{Sound as a Longitudinal Wave}
\begin{columns}[T]
\column{0.55\textwidth}
\begin{figure}
\centering
\includegraphics[width=\textwidth,height=0.55\textheight,keepaspectratio]{phys11-sound-fig04.jpg}
\end{figure}

\pause
\column{0.45\textwidth}
\textbf{Sound waves consist of:}
\begin{itemize}
\item \textbf{Compressions} - high pressure regions \pause
\item \textbf{Rarefactions} - low pressure regions
\end{itemize}

\pause
\vspace{0.3cm}
Analogous to crests and troughs in transverse waves.
\end{columns}
\note{[DEFINITION] Longitudinal: particles move PARALLEL to wave direction\\\\
[COMPARE] Compression equals high pressure (like a crest)\\\\
[COMPARE] Rarefaction equals low pressure (like a trough)\\\\
[CHECK] Quick poll: what type of wave is light?\\\\
ANSWER: Transverse - EM waves oscillate perpendicular to travel\\\\
[TRANSITION] Let's see this in more detail...}
\end{frame}

\begin{frame}
\frametitle{How Sound Travels}
\begin{figure}
\centering
\includegraphics[width=0.7\textwidth,height=0.5\textheight,keepaspectratio]{phys11-sound-fig05.jpg}
\end{figure}

\pause
\textbf{Pressure variation:} Sound creates alternating high and low pressure regions as it travels through a medium.
\note{[SAY] "Watch the air molecules - they don't travel with the wave"\\\\
[EMPHASIZE] Wave moves forward, molecules just oscillate in place\\\\
[CHECK] Name wheel: do air molecules travel with the sound wave?\\\\
ANSWER: No - they vibrate back and forth in place, wave energy moves\\\\
[TRANSITION] Now let's look at speed in different materials...}
\end{frame}

\begin{frame}
\frametitle{Speed of Sound in Different Media}
The speed of sound depends on:
\begin{itemize}
\item \textbf{Rigidity} (or compressibility) of the medium \pause
\item \textbf{Density} of the medium
\end{itemize}

\pause
\vspace{0.3cm}
\begin{center}
\begin{tabular}{|l|c|}
\hline
\textbf{Medium} & \textbf{Speed (m/s)} \\
\hline
Air (0°C) & 331 \\
Air (20°C) & 343 \\
Water (25°C) & 1493 \\
Steel & 5960 \\
\hline
\end{tabular}
\end{center}

\pause
\vspace{0.3cm}
\alert{More rigid = faster; more dense = slower}
\note{[KEY RULE] More rigid equals faster; more dense equals slower\\\\
[EXAMPLE] Steel: very rigid, so very fast despite high density\\\\
[CHECK] Name wheel: why is sound faster in water than air?\\\\
ANSWER: Water is more rigid (harder to compress) - outweighs the higher density\\\\
[TRANSITION] Now for the key equation...}
\end{frame}

\begin{frame}
\frametitle{Speed, Frequency, and Wavelength}
\begin{figure}
\centering
\includegraphics[width=0.5\textwidth,height=0.45\textheight,keepaspectratio]{phys11-sound-fig09.jpg}
\end{figure}

\pause
\begin{center}
\Huge
$\boxed{v = f\lambda}$
\end{center}

\pause
\textbf{Where:}
\begin{itemize}
\item $v$ = speed of sound (m/s)
\item $f$ = frequency (Hz)
\item $\lambda$ = wavelength (m)
\end{itemize}
\note{[EMPHASIZE] This is THE key equation for this chapter!\\\\
[SAY] "v equals f times lambda - velocity equals frequency times wavelength"\\\\
[VARIABLES] v is speed in meters per second, f is frequency in Hertz, lambda is wavelength in meters\\\\
[REARRANGEMENTS] f equals v over lambda, lambda equals v over f\\\\
[CHECK] Name wheel: which form to find wavelength?\\\\
ANSWER: Lambda equals v over f - divide both sides by f\\\\
[TRANSITION] Key insight about this relationship...}
\end{frame}

\begin{frame}
\frametitle{Key Insight: Inverse Relationship}
In a given medium at constant temperature:
\begin{itemize}
\item Speed $v$ is constant for all frequencies \pause
\item Therefore: $f$ and $\lambda$ are \alert{inversely related}
\end{itemize}

\pause
\vspace{0.5cm}
\textbf{Higher frequency $\rightarrow$ shorter wavelength}

\textbf{Lower frequency $\rightarrow$ longer wavelength}

\pause
\vspace{0.5cm}
\textit{This is why music from all instruments arrives in cadence regardless of distance!}
\note{[EMPHASIZE] Speed is fixed by medium - frequency doesn't change it!\\\\
[KEY INSIGHT] If v is constant, double f means half lambda (inverse)\\\\
[REAL EXAMPLE] Orchestra: high and low notes arrive together - same speed!\\\\
[CHECK] Name wheel: if you double frequency, wavelength...?\\\\
ANSWER: Halves - inverse relationship, like a seesaw\\\\
[TRANSITION] Now let's put this equation to work...}
\end{frame}

%-----------------------------------------------------------------------------
% Attempt → Compare → Practice: Wave Equation
%-----------------------------------------------------------------------------

\begin{frame}
\frametitle{Attempt: Wave Equation Problem}
\textbf{Try this on your own (3 min, silent):}

\vspace{0.5cm}

Calculate the wavelengths of sounds at the extremes of the audible range, 20 Hz and 20,000 Hz, when sound travels at 348.7 m/s.

\vspace{1cm}

\begin{center}
\Large
\textit{Work individually. It's okay to get stuck!}
\end{center}
\note{[SAY] "Try this on your own. It's okay to get stuck."\\\\
[TIMING] 3-4 min SILENT individual work\\\\
[CIRCULATE] Note who finishes, who struggles, common errors\\\\
[WATCH FOR] Students forgetting to rearrange, or dividing wrong way\\\\
[DON'T HELP] Let them struggle - learning happens in Compare}
\end{frame}

\begin{frame}
\frametitle{Compare: Discuss with Partner}
\textbf{Turn to your neighbor (3 min):}

\vspace{0.5cm}

\begin{itemize}
\item What equation did you start with? \pause
\item Where did you get stuck? \pause
\item Did you start the same way?
\end{itemize}

\pause
\vspace{0.5cm}

\textbf{Name wheel:} Share your \alert{approach} (not your answer)
\note{[SAY] "Compare with your neighbor. Where did you get stuck?"\\\\
[TIMING] 3-4 min pair discussion\\\\
[CHECK] Name wheel: call 1-2 pairs to share APPROACH, not answer\\\\
[EXPECTED APPROACH] Start with v equals f lambda, rearrange for lambda\\\\
[COMMON ERROR] Forgetting to rearrange, or unit conversion (kHz to Hz)}
\end{frame}

\begin{frame}
\frametitle{Reveal: Solution Step-by-Step}
\textbf{Mark corrections in a different color!}

\pause
\vspace{0.3cm}

\textbf{Equation:} $v = f\lambda$ \quad $\rightarrow$ \quad $\lambda = \frac{v}{f}$

\pause
\vspace{0.3cm}

\textbf{For low frequency (20 Hz):}
$$\lambda_{max} = \frac{348.7}{20} = \boxed{17 \text{ m}}$$

\pause
\vspace{0.3cm}

\textbf{For high frequency (20,000 Hz):}
$$\lambda_{min} = \frac{348.7}{20000} = \boxed{0.017 \text{ m} = 1.7 \text{ cm}}$$

\pause
\vspace{0.3cm}

\textbf{Check:} Lower frequency $\rightarrow$ longer wavelength. Bass sounds are room-sized!
\note{[ALGEBRA]\\\\
- "Start with v equals f lambda"\\\\
- "Divide both sides by f to isolate lambda"\\\\
- "Lambda equals v over f"\\\\
[ANSWER] Low freq: 17 m (room-sized), High freq: 1.7 cm (finger-sized)\\\\
[REASONABLENESS] Bass sounds are BIG, treble sounds are TINY\\\\
[IF CONFUSED] Draw the formula triangle: v on top, f and lambda on bottom\\\\
[TRANSITION] Let's move on to sound intensity...}
\end{frame}

%=============================================================================
\section{14.2 Sound Intensity and Sound Level}
%=============================================================================

\begin{frame}
\frametitle{Sound Intensity}
\begin{figure}
\centering
\includegraphics[width=0.5\textwidth,height=0.4\textheight,keepaspectratio]{phys11-sound-fig17.jpg}
\end{figure}

\pause
\textbf{Intensity} is the power per unit area carried by a wave:

\pause
\begin{center}
\Huge
$\boxed{I = \frac{P}{A}}$
\end{center}

\pause
\textbf{Units:} W/m$^2$
\note{[DEFINITION] Intensity equals power divided by area\\\\
[SAY] "I equals P over A - power spread over area"\\\\
[REAL EXAMPLE] Why speakers sound quieter far away: same power, bigger area\\\\
[CHECK] Quick - units of intensity?\\\\
ANSWER: Watts per square meter (W/m²)\\\\
[TRANSITION] Now the relationship to pressure...}
\end{frame}

\begin{frame}
\frametitle{Intensity and Pressure Amplitude}
Sound intensity depends on pressure amplitude:

\pause
\begin{center}
\Large
$I = \frac{(\Delta p)^2}{2\rho v_w}$
\end{center}

\pause
\vspace{0.3cm}
\textbf{Where:}
\begin{itemize}
\item $\Delta p$ = pressure amplitude (Pa)
\item $\rho$ = density of medium (kg/m$^3$)
\item $v_w$ = speed of sound (m/s)
\end{itemize}

\pause
\vspace{0.3cm}
\alert{Key:} Intensity is proportional to amplitude squared: $I \propto (\Delta p)^2$
\note{[DEFINITION] Delta-p is pressure variation from normal atmospheric\\\\
[VARIABLES] rho is density (kg per cubic meter), v-w is wave speed\\\\
[EMPHASIZE] Intensity proportional to pressure SQUARED!\\\\
[CHECK] Name wheel: if pressure doubles, intensity becomes...?\\\\
ANSWER: 4 times larger - two squared equals four\\\\
[TRANSITION] Let's see this relationship visually...}
\end{frame}

\begin{frame}
\frametitle{Pressure vs. Intensity}
\begin{figure}
\centering
\includegraphics[width=0.7\textwidth,height=0.5\textheight,keepaspectratio]{phys11-sound-fig19.jpg}
\end{figure}

\pause
Greater amplitude $\rightarrow$ greater pressure variation $\rightarrow$ greater intensity (louder sound)
\note{[SAY] "Point to the tall wave versus the short wave"\\\\
[EMPHASIZE] Larger oscillations equals more energy equals louder sound\\\\
[CHECK] Quick poll: which wave carries more energy?\\\\
ANSWER: High amplitude - more energy per wave cycle\\\\
[TRANSITION] Now we need a way to measure loudness...}
\end{frame}

\begin{frame}
\frametitle{The Decibel Scale}
Human perception of loudness is \alert{logarithmic}, not linear.

\pause
\vspace{0.3cm}

\textbf{Sound Intensity Level:}
\begin{center}
\Huge
$\boxed{\beta(\text{dB}) = 10 \log_{10}\left(\frac{I}{I_0}\right)}$
\end{center}

\pause
\vspace{0.3cm}

\textbf{Where:}
\begin{itemize}
\item $\beta$ = sound level in decibels (dB)
\item $I$ = sound intensity (W/m$^2$)
\item $I_0 = 10^{-12}$ W/m$^2$ (threshold of hearing)
\end{itemize}
\note{[EMPHASIZE] This is the second key equation!\\\\
[SAY] "Beta equals 10 times log base 10 of I over I-naught"\\\\
[DEFINITION] I-naught equals ten to the negative twelve - threshold of hearing\\\\
[KEY INSIGHT] Log scale because ears respond to RATIOS, not differences\\\\
[CHECK] Name wheel: why use log scale?\\\\
ANSWER: Human ears respond to ratios - 10x louder feels like one step\\\\
[TRANSITION] Let's see how this scale works...}
\end{frame}

\begin{frame}
\frametitle{Understanding Decibels}
\textbf{Key relationships:}
\begin{itemize}
\item Threshold of hearing: $I_0 = 10^{-12}$ W/m$^2$ $\rightarrow$ 0 dB \pause
\item Each factor of 10 in intensity $= +10$ dB \pause
\item Doubling intensity $\approx +3$ dB
\end{itemize}

\pause
\vspace{0.5cm}

\begin{center}
\begin{tabular}{|l|c|}
\hline
\textbf{Sound} & \textbf{Level (dB)} \\
\hline
Threshold of hearing & 0 \\
Whisper & 20 \\
Normal conversation & 60 \\
Busy traffic & 80 \\
Rock concert & 110 \\
Pain threshold & 130 \\
\hline
\end{tabular}
\end{center}
\note{[KEY RULES]\\\\
- 0 dB equals threshold of hearing (not silence!)\\\\
- Every 10 dB equals 10x intensity\\\\
- Every 3 dB equals 2x intensity (useful shortcut)\\\\
[EMPHASIZE] 130 dB causes pain and hearing damage!\\\\
[CHECK] Name wheel: 80 dB vs 60 dB - how many times more intense?\\\\
ANSWER: 100 times - 20 dB difference equals 10 times 10\\\\
[TRANSITION] Now let's try a calculation...}
\end{frame}

%-----------------------------------------------------------------------------
% Attempt → Compare → Practice: Sound Intensity
%-----------------------------------------------------------------------------

\begin{frame}
\frametitle{Attempt: Intensity Problem}
\textbf{Try this on your own (3 min, silent):}

\vspace{0.5cm}

Calculate the intensity of a wave if the power transferred is 10 W and the area through which the wave is transferred is 5 square meters.

\vspace{0.5cm}

Then: What decibel level is this? ($I_0 = 10^{-12}$ W/m$^2$)

\vspace{0.8cm}

\begin{center}
\Large
\textit{Work individually. It's okay to get stuck!}
\end{center}
\note{[SAY] "Try this on your own. It's okay to get stuck."\\\\
[TIMING] 3-4 min SILENT individual work\\\\
[CIRCULATE] Two-step problem: intensity THEN decibels\\\\
[WATCH FOR] Students forgetting the log, or calculator errors\\\\
[DON'T HELP] Let them struggle - learning happens in Compare}
\end{frame}

\begin{frame}
\frametitle{Compare: Discuss with Partner}
\textbf{Turn to your neighbor (3 min):}

\vspace{0.5cm}

\begin{itemize}
\item How did you find intensity? \pause
\item Did you remember to use the decibel equation? \pause
\item What value did you get for log?
\end{itemize}

\pause
\vspace{0.5cm}

\textbf{Name wheel:} Share your \alert{approach} (not your answer)
\note{[SAY] "Compare with your neighbor. Where did you get stuck?"\\\\
[TIMING] 2-3 min pair discussion\\\\
[CHECK] Name wheel: call 1-2 pairs to share APPROACH\\\\
[EXPECTED APPROACH] First I equals P over A, then beta equals 10 log I over I-naught\\\\
[COMMON ERROR] Forgetting I-naught, or log calculator confusion}
\end{frame}

\begin{frame}
\frametitle{Reveal: Solution Step-by-Step}
\textbf{Mark corrections in a different color!}

\pause
\vspace{0.3cm}

\textbf{Step 1 - Intensity:}
$$I = \frac{P}{A} = \frac{10 \text{ W}}{5 \text{ m}^2} = \boxed{2 \text{ W/m}^2}$$

\pause
\vspace{0.3cm}

\textbf{Step 2 - Decibels:}
$$\beta = 10 \log_{10}\left(\frac{2}{10^{-12}}\right) = 10 \log_{10}(2 \times 10^{12})$$

\pause
$$\beta = 10 \times 12.3 = \boxed{123 \text{ dB}}$$

\pause
\vspace{0.3cm}

\textbf{Check:} 123 dB is near pain threshold (130 dB). Extremely loud!
\note{[ALGEBRA]\\\\
- "First, I equals P over A equals 10 divided by 5"\\\\
- "I equals 2 watts per square meter"\\\\
- "Now beta equals 10 log of 2 over ten to the negative twelve"\\\\
- "That's 10 log of 2 times ten to the twelfth"\\\\
[ANSWER] I = 2 W/m², beta = 123 dB\\\\
[REASONABLENESS] 123 dB is jet engine loud - causes hearing damage!\\\\
[IF CONFUSED] Break into two steps: intensity first, THEN decibels}
\end{frame}

\begin{frame}
\frametitle{Human Hearing}
\begin{figure}
\centering
\includegraphics[width=0.6\textwidth,height=0.5\textheight,keepaspectratio]{phys11-sound-fig24.jpg}
\end{figure}

\pause
\textbf{Hearing range:} 20 Hz to 20,000 Hz
\begin{itemize}
\item Below 20 Hz: \textbf{Infrasound}
\item Above 20,000 Hz: \textbf{Ultrasound}
\end{itemize}
\note{[SAY] "Human hearing: 20 Hz to 20,000 Hz"\\\\
[REAL EXAMPLES] Infrasound: elephants, earthquakes\\\\
[REAL EXAMPLES] Ultrasound: medical imaging, dog whistles, bats\\\\
[KEY FACT] Hearing degrades with age - lose high frequencies first\\\\
[CHECK] Name wheel: why can dogs hear dog whistles but we can't?\\\\
ANSWER: Dog whistles are ultrasound - above 20 kHz\\\\
[TRANSITION] Now let's explore the Doppler effect...}
\end{frame}

%=============================================================================
\section{14.3 Doppler Effect and Sonic Booms}
%=============================================================================

\begin{frame}
\frametitle{The Doppler Effect}
\textbf{Definition:} A change in the observed frequency of a sound due to relative motion between source and observer.

\pause
\vspace{0.5cm}

\textbf{Examples:}
\begin{itemize}
\item Ambulance siren pitch changes as it passes \pause
\item Race car engine sound \pause
\item Train whistle
\end{itemize}

\pause
\vspace{0.3cm}
\alert{Moving toward} $\rightarrow$ higher frequency (higher pitch)

\alert{Moving away} $\rightarrow$ lower frequency (lower pitch)
\note{[SAY] "Think of an ambulance siren as it passes you"\\\\
[KEY INSIGHT] Approaching: waves bunch up in front (higher pitch)\\\\
[KEY INSIGHT] Receding: waves stretch out behind (lower pitch)\\\\
[CHECK] Volunteers: what happens to pitch as ambulance passes?\\\\
ANSWER: Pitch drops suddenly from high to low\\\\
[MNEMONIC] Toward equals Tall pitch, Away equals Awful pitch\\\\
[TRANSITION] Let's see why this happens...}
\end{frame}

\begin{frame}
\frametitle{Stationary Source}
\begin{figure}
\centering
\includegraphics[width=0.55\textwidth,height=0.5\textheight,keepaspectratio]{phys11-sound-fig31-1.jpg}
\end{figure}

\pause
When source and observer are stationary, wavelength and frequency are the same in all directions.
\note{[SAY] "Look at the diagram - all circles are evenly spaced"\\\\
[EMPHASIZE] Same wavelength in all directions when source is stationary\\\\
[CHECK] Thumbs up/down: are the wave circles evenly spaced?\\\\
ANSWER: Yes - source is not moving\\\\
[TRANSITION] Now watch what happens when the source moves...}
\end{frame}

\begin{frame}
\frametitle{Moving Source}
\begin{figure}
\centering
\includegraphics[width=0.55\textwidth,height=0.5\textheight,keepaspectratio]{phys11-sound-fig31.jpg}
\end{figure}

\pause
\begin{itemize}
\item Waves compress in front (shorter $\lambda$, higher $f$) \pause
\item Waves stretch behind (longer $\lambda$, lower $f$)
\end{itemize}
\note{[SAY] "Look - circles bunched up in front, spread out behind"\\\\
[KEY INSIGHT] Source catches up to its own waves in front\\\\
[EMPHASIZE] Shorter lambda means higher frequency (v equals f lambda)\\\\
[CHECK] Name wheel: who hears higher pitch, front or back observer?\\\\
ANSWER: Front observer - shorter wavelength equals higher frequency\\\\
[TRANSITION] Let's see the equation for this...}
\end{frame}

\begin{frame}
\frametitle{Doppler Effect Equation - Moving Source}
For a \textbf{stationary observer} and \textbf{moving source}:

\pause
\begin{center}
\Huge
$\boxed{f_{obs} = f_s\left(\frac{v_w}{v_w \pm v_s}\right)}$
\end{center}

\pause
\vspace{0.2cm}
\textbf{Variables:} $f_{obs}$ = observed, $f_s$ = source, $v_w$ = sound speed, $v_s$ = source speed

\pause
\vspace{0.2cm}
\textbf{Sign:} \alert{Minus} = toward (higher pitch) \quad \alert{Plus} = away (lower pitch)
\note{[EMPHASIZE] This is the Doppler equation for moving source!\\\\
[SAY] "f-observed equals f-source times v-w over v-w plus or minus v-s"\\\\
[SIGN TRICK] Toward uses MINUS (smaller denominator = higher f)\\\\
[MNEMONIC] Minus for Moving toward - both start with M\\\\
[CHECK] Name wheel: if source moves toward, denominator gets...?\\\\
ANSWER: Smaller (v-w minus v-s), so f-observed gets BIGGER\\\\
[TRANSITION] What about a moving observer?}
\end{frame}

\begin{frame}
\frametitle{Doppler Effect - Moving Observer}
For a \textbf{moving observer} and \textbf{stationary source}:

\pause
\begin{center}
\Large
$f_{obs} = f_s\left(\frac{v_w \pm v_{obs}}{v_w}\right)$
\end{center}

\pause
\vspace{0.3cm}
\textbf{Sign convention:}
\begin{itemize}
\item \alert{Plus (+)}: observer moving \textbf{toward} source
\item \alert{Minus (-)}: observer moving \textbf{away} from source
\end{itemize}
\note{[EMPHASIZE] v-obs is in the NUMERATOR this time!\\\\
[KEY DIFFERENCE] Opposite signs from moving source equation\\\\
[SAY] "Plus for toward - bigger numerator means higher frequency"\\\\
[CHECK] Name wheel: why are signs opposite from moving source?\\\\
ANSWER: Speed is in numerator, not denominator - math works opposite\\\\
[TRANSITION] Let's practice with a Doppler problem...}
\end{frame}

%-----------------------------------------------------------------------------
% Attempt → Compare → Practice: Doppler Effect
%-----------------------------------------------------------------------------

\begin{frame}
\frametitle{Attempt: Doppler Problem}
\textbf{Try this on your own (3 min, silent):}

\vspace{0.5cm}

A train with a 150 Hz horn moves at 35 m/s toward you. The speed of sound is 340 m/s.

\vspace{0.3cm}

What frequency do you hear?

\vspace{0.8cm}

\begin{center}
\Large
\textit{Work individually. It's okay to get stuck!}
\end{center}
\note{[SAY] "Try this on your own. It's okay to get stuck."\\\\
[TIMING] 3-4 min SILENT individual work\\\\
[CIRCULATE] Key decision: plus or minus in denominator?\\\\
[WATCH FOR] Students using wrong sign convention\\\\
[DON'T HELP] Let them struggle - learning happens in Compare}
\end{frame}

\begin{frame}
\frametitle{Compare: Discuss with Partner}
\textbf{Turn to your neighbor (3 min):}

\vspace{0.5cm}

\begin{itemize}
\item Did you use plus or minus in the denominator? \pause
\item Should the observed frequency be higher or lower? \pause
\item Does your answer match that prediction?
\end{itemize}

\pause
\vspace{0.5cm}

\textbf{Name wheel:} Share your \alert{reasoning for the sign}
\note{[SAY] "Compare with your neighbor. Did you use plus or minus?"\\\\
[TIMING] 2-3 min pair discussion\\\\
[KEY INSIGHT] Toward means MINUS - smaller denominator means higher frequency\\\\
[CHECK] Name wheel: call pairs to explain their sign reasoning\\\\
[COMMON ERROR] Using plus instead of minus for approaching source}
\end{frame}

\begin{frame}
\frametitle{Reveal: Solution Step-by-Step}
\textbf{Mark corrections in a different color!}

\pause
\vspace{0.3cm}

\textbf{Equation (moving source toward):}
$$f_{obs} = f_s\left(\frac{v_w}{v_w - v_s}\right)$$

\pause
\vspace{0.3cm}

\textbf{Substitute:}
$$f_{obs} = 150\left(\frac{340}{340 - 35}\right) = 150\left(\frac{340}{305}\right)$$

\pause
\vspace{0.3cm}

$$\boxed{f_{obs} = 167 \text{ Hz}}$$

\pause

\textbf{Check:} Moving toward $\rightarrow$ higher pitch. 167 > 150. Correct!
\note{[ALGEBRA]\\\\
- "f-observed equals f-source times v-w over v-w MINUS v-s"\\\\
- "MINUS because source is moving TOWARD us"\\\\
- "That's 150 times 340 over 340 minus 35"\\\\
- "150 times 340 over 305"\\\\
[ANSWER] f-observed equals 167 Hz\\\\
[REASONABLENESS] Moving toward means higher pitch - 167 is greater than 150, correct!\\\\
[IF CONFUSED] Remember: Minus for Moving toward - both start with M\\\\
[TRANSITION] What happens if source goes faster than sound?}
\end{frame}

%-----------------------------------------------------------------------------
% Sonic Boom
%-----------------------------------------------------------------------------

\begin{frame}
\frametitle{Sonic Booms}
\begin{figure}
\centering
\includegraphics[width=0.5\textwidth,height=0.45\textheight,keepaspectratio]{phys11-sound-fig34.jpg}
\end{figure}

\pause
\textbf{Sonic boom:} Constructive interference of sound created by an object moving \alert{faster than sound}.

\pause
\vspace{0.3cm}
\begin{itemize}
\item All sound waves pile up at once
\item Creates intense pressure wave
\item Aircraft creates two booms (nose and tail)
\end{itemize}
\note{[DEFINITION] Supersonic means faster than sound\\\\
[KEY FACT] Mach 1 equals speed of sound - about 340 m/s or 1235 km/h\\\\
[EMPHASIZE] All waves pile up in a cone - the Mach cone\\\\
[REAL EXAMPLE] Thunder is a natural sonic boom - lightning is supersonic!\\\\
[CHECK] Name wheel: why do jets create TWO booms?\\\\
ANSWER: One from the nose, one from the tail\\\\
[TRANSITION] Now let's explore resonance and standing waves...}
\end{frame}

%=============================================================================
\section{14.4 Sound Interference and Resonance}
%=============================================================================

\begin{frame}
\frametitle{Resonance}
\begin{figure}
\centering
\includegraphics[width=0.4\textwidth,height=0.4\textheight,keepaspectratio]{phys11-sound-fig39.jpg}
\end{figure}

\pause
\textbf{Resonance:} Driving a system at its \alert{natural frequency} causes maximum amplitude oscillations.

\pause
\vspace{0.3cm}
\textbf{Examples:}
\begin{itemize}
\item Pushing a child on a swing at the right rhythm
\item Piano strings vibrating in response to your voice
\item Tuning a radio to a specific frequency
\end{itemize}
\note{[DEFINITION] Resonance: driving frequency matches natural frequency\\\\
[SAY] "Small pushes at the right time build up to big oscillations"\\\\
[REAL EXAMPLE] Tacoma Narrows Bridge collapse - wind matched natural frequency\\\\
[CHECK] Name wheel: what happens if you push a swing at the WRONG time?\\\\
ANSWER: You slow it down or cancel the motion - destructive interference\\\\
[TRANSITION] Now let's see what happens with two similar frequencies...}
\end{frame}

\begin{frame}
\frametitle{Beats}
\begin{figure}
\centering
\includegraphics[width=0.7\textwidth,height=0.5\textheight,keepaspectratio]{phys11-sound-fig41.jpg}
\end{figure}

\pause
\textbf{Beats:} Produced by superposition of two waves with slightly different frequencies.
\note{[SAY] "Point to diagram: two waves with SLIGHTLY different frequencies"\\\\
[EMPHASIZE] Sometimes in phase (loud), sometimes out of phase (quiet)\\\\
[REAL EXAMPLE] Creates wah-wah-wah sound you hear when tuning a guitar\\\\
[CHECK] Name wheel: what happens when waves are in phase?\\\\
ANSWER: Constructive interference - louder\\\\
[TRANSITION] Here's the equation for beat frequency...}
\end{frame}

\begin{frame}
\frametitle{Beat Frequency}
\begin{center}
\Huge
$\boxed{f_B = |f_1 - f_2|}$
\end{center}

\pause
\vspace{0.5cm}

\textbf{Applications:}
\begin{itemize}
\item Piano tuning - adjust until beats disappear
\item Guitar tuning
\item Musical instrument manufacturing
\end{itemize}

\pause
\vspace{0.3cm}
\textit{When two frequencies are similar, we hear the average frequency getting louder and softer at the beat frequency.}
\note{[SAY] "f-B equals absolute value of f-1 minus f-2"\\\\
[KEY APPLICATION] Piano tuning: adjust until beats disappear (means f1 equals f2)\\\\
[EMPHASIZE] Beats per second equals the difference in Hz\\\\
[CHECK] Name wheel: if f1 equals 440 Hz and f2 equals 442 Hz, beat frequency?\\\\
ANSWER: 2 Hz - you hear 2 wah-wah's per second\\\\
[TRANSITION] Now let's look at standing waves in tubes...}
\end{frame}

\begin{frame}
\frametitle{Standing Waves in Tubes}
\begin{figure}
\centering
\includegraphics[width=0.6\textwidth,height=0.5\textheight,keepaspectratio]{phys11-sound-fig42.jpg}
\end{figure}

\pause
Sound waves reflect off closed ends and interfere to create \textbf{standing waves}.
\note{[SAY] "Sound reflects off closed end, interferes with incoming wave"\\\\
[DEFINITION] Nodes: no motion; Antinodes: maximum motion\\\\
[REAL APPLICATION] Musical instruments use this to produce specific notes\\\\
[CHECK] Name wheel: what is a standing wave?\\\\
ANSWER: Wave that appears stationary due to interference of two waves\\\\
[TRANSITION] Let's compare closed and open pipes...}
\end{frame}

\begin{frame}
\frametitle{Closed-Pipe Resonator}
\begin{figure}
\centering
\includegraphics[width=0.55\textwidth,height=0.5\textheight,keepaspectratio]{phys11-sound-fig44.jpg}
\end{figure}

\pause
\textbf{Resonant frequencies:}
\begin{center}
\Large
$\boxed{f_n = n\frac{v}{4L}, \quad n = 1, 3, 5, ...}$
\end{center}

\pause
\textbf{Only odd harmonics!} (Node at closed end, antinode at open end)
\note{[KEY RULE] Closed end equals node (air can't move)\\\\
[KEY RULE] Open end equals antinode (air moves freely)\\\\
[SAY] "f-n equals n times v over 4L, where n is 1, 3, 5..."\\\\
[EMPHASIZE] Only ODD harmonics for closed pipe!\\\\
[CHECK] Name wheel: why only odd harmonics?\\\\
ANSWER: Need one-quarter wavelength to fit node-to-antinode\\\\
[TRANSITION] Now compare to open pipe...}
\end{frame}

\begin{frame}
\frametitle{Open-Pipe Resonator}
\begin{figure}
\centering
\includegraphics[width=0.55\textwidth,height=0.5\textheight,keepaspectratio]{phys11-sound-fig45.jpg}
\end{figure}

\pause
\textbf{Resonant frequencies:}
\begin{center}
\Large
$\boxed{f_n = n\frac{v}{2L}, \quad n = 1, 2, 3, ...}$
\end{center}

\pause
\textbf{All harmonics!} (Antinodes at both ends)
\note{[KEY RULE] Both ends open means antinodes at both ends\\\\
[SAY] "f-n equals n times v over 2L, where n is 1, 2, 3..."\\\\
[EMPHASIZE] ALL harmonics for open pipe - richer sound!\\\\
[REAL EXAMPLES] Flute is open pipe, clarinet is closed pipe\\\\
[CHECK] Name wheel: fundamental wavelength for open pipe?\\\\
ANSWER: Lambda equals 2L - half a wavelength fits in the pipe\\\\
[TRANSITION] Let's compare the two types...}
\end{frame}

\begin{frame}
\frametitle{Comparing Resonators}
\begin{columns}[T]
\column{0.48\textwidth}
\textbf{Closed Pipe}
\begin{itemize}
\item One end closed
\item Only odd harmonics
\item $f_n = n\frac{v}{4L}$
\item $n = 1, 3, 5...$
\end{itemize}

\pause
\column{0.48\textwidth}
\textbf{Open Pipe}
\begin{itemize}
\item Both ends open
\item All harmonics
\item $f_n = n\frac{v}{2L}$
\item $n = 1, 2, 3...$
\end{itemize}
\end{columns}

\pause
\vspace{0.5cm}
\textbf{Note:} An open pipe has twice the fundamental frequency of a closed pipe of the same length!
\note{[KEY COMPARISON] Closed uses 4L, open uses 2L in denominator\\\\
[EMPHASIZE] Same length pipe: open pipe is 2x frequency - one octave higher!\\\\
[KEY INSIGHT] Closed has only odd harmonics - different timbre\\\\
[CHECK] Name wheel: which pipe has richer sound?\\\\
ANSWER: Open pipe - has ALL harmonics, not just odd ones\\\\
[TRANSITION] Now let's practice with a closed pipe problem...}
\end{frame}

%-----------------------------------------------------------------------------
% ACP: Closed Pipe Resonance
%-----------------------------------------------------------------------------

\begin{frame}
\frametitle{Attempt: Closed Pipe Length}
\textbf{Try this on your own (3 min, silent):}

\vspace{0.3cm}

If sound travels at 344 m/s, what should be the length of a tube closed at one end to have a fundamental frequency of 128 Hz?

\vspace{0.5cm}

\textbf{Given:}
\begin{itemize}
\item $f_1 = 128$ Hz (fundamental frequency)
\item $v = 344$ m/s (speed of sound)
\item Tube closed at one end
\end{itemize}

\textbf{Find:} Length $L$

\vspace{0.3cm}

\textit{Work individually. It's okay to get stuck.}
\note{[SAY] "Try this on your own. It's okay to get stuck."\\\\
[TIMING] 3-4 min SILENT individual work\\\\
[CIRCULATE] Key decision: which pipe equation to use?\\\\
[WATCH FOR] Students not recognizing "closed at one end" means closed pipe\\\\
[DON'T HELP] Let them struggle - learning happens in Compare}
\end{frame}

\begin{frame}
\frametitle{Compare: Closed Pipe Length}
\textbf{Turn and talk (2 min):}

\vspace{0.3cm}

\begin{enumerate}
\item Share your approach with your partner
\item Which equation did you choose? Why?
\item How did you rearrange for $L$?
\end{enumerate}

\vspace{0.5cm}

\pause
\alert{Name wheel:} One pair share your approach (not your answer).
\note{[SAY] "Turn to your partner. Which equation did you use?"\\\\
[TIMING] 2-3 min pair discussion\\\\
[CHECK] Name wheel: call a pair to share their approach\\\\
[EXPECTED APPROACH] Identify closed pipe, use f-1 equals v over 4L, solve for L\\\\
[COMMON ERROR] Using open pipe formula (2L) instead of closed pipe (4L)}
\end{frame}

\begin{frame}
\frametitle{Reveal: Closed Pipe Solution}
\textbf{Self-correct in a different color:}

\vspace{0.3cm}

\textbf{E - Equation:} Closed pipe: $f_1 = \frac{v}{4L}$

\pause
\vspace{0.2cm}

\textbf{Rearrange for L:}
$$f_1 = \frac{v}{4L} \quad \rightarrow \quad 4Lf_1 = v \quad \rightarrow \quad L = \frac{v}{4f_1}$$

\pause
\vspace{0.2cm}

\textbf{S - Substitute:} $L = \frac{344 \text{ m/s}}{4(128 \text{ Hz})} = \frac{344}{512}$

\pause
\vspace{0.2cm}

$$\boxed{L = 0.672 \text{ m} \approx 67 \text{ cm}}$$

\textbf{Check:} About 2 feet - reasonable for a musical tube.
\note{[ALGEBRA]\\\\
- "Start with f-1 equals v over 4L"\\\\
- "Multiply both sides by 4L to get 4L times f-1 equals v"\\\\
- "Divide both sides by 4 times f-1"\\\\
- "L equals v over 4 times f-1"\\\\
[ANSWER] L equals 0.672 m, about 67 cm (two feet)\\\\
[REASONABLENESS] 128 Hz is middle C - this is about the length of a flute\\\\
[IF CONFUSED] Remember: closed pipe uses 4L, open pipe uses 2L\\\\
[TRANSITION] Now let's try a beat frequency problem...}
\end{frame}

%-----------------------------------------------------------------------------
% ACP: Beat Frequency
%-----------------------------------------------------------------------------

\begin{frame}
\frametitle{Attempt: Piano Tuning Problem}
\textbf{Try this on your own (3 min, silent):}

\vspace{0.3cm}

A piano tuner hears 2 beats per second when comparing a piano note to a 256 Hz tuning fork. What are the possible frequencies of the piano?

\vspace{0.5cm}

\textbf{Given:}
\begin{itemize}
\item Beat frequency $f_B = 2$ Hz
\item Tuning fork frequency $f_2 = 256$ Hz
\end{itemize}

\textbf{Find:} Piano frequency $f_1$

\vspace{0.3cm}

\textit{There may be more than one answer!}
\note{[SAY] "Try this on your own. Hint: there may be more than one answer!"\\\\
[TIMING] 3-4 min SILENT individual work\\\\
[CIRCULATE] Key insight: absolute value means TWO possibilities\\\\
[WATCH FOR] Students getting only one answer\\\\
[DON'T HELP] Let them struggle - the two-answer insight is the learning}
\end{frame}

\begin{frame}
\frametitle{Compare: Piano Tuning}
\textbf{Turn and talk (2 min):}

\vspace{0.3cm}

\begin{enumerate}
\item Share your approach with your partner
\item Did you get one answer or two?
\item How did the absolute value affect your solution?
\end{enumerate}

\vspace{0.5cm}

\pause
\alert{Name wheel:} One pair explain why there might be two answers.
\note{[SAY] "Did you get one answer or two? Compare with your partner."\\\\
[TIMING] 2-3 min pair discussion\\\\
[CHECK] Name wheel: call a pair to explain why TWO answers\\\\
[EXPECTED INSIGHT] Absolute value means f1 could be above OR below f2\\\\
[KEY VOCABULARY] Piano could be "sharp" (above) or "flat" (below)}
\end{frame}

\begin{frame}
\frametitle{Reveal: Piano Tuning Solution}
\textbf{Self-correct in a different color:}

\vspace{0.3cm}

\textbf{E - Equation:} $f_B = |f_1 - f_2|$

\pause
\vspace{0.2cm}

\textbf{Rearrange:} Since $|f_1 - f_2| = 2$, we have:
$$f_1 - f_2 = +2 \quad \text{OR} \quad f_1 - f_2 = -2$$

\pause
\vspace{0.2cm}

\textbf{S - Substitute:}
$$f_1 = 256 + 2 = 258 \text{ Hz} \quad \text{(piano sharp)}$$
$$f_1 = 256 - 2 = 254 \text{ Hz} \quad \text{(piano flat)}$$

\pause
\vspace{0.2cm}

$$\boxed{f_1 = 258 \text{ Hz or } 254 \text{ Hz}}$$
\note{[ALGEBRA]\\\\
- "f-B equals absolute value of f-1 minus f-2"\\\\
- "2 equals absolute value of f-1 minus 256"\\\\
- "So f-1 minus 256 equals positive 2 OR negative 2"\\\\
[ANSWER] f-1 equals 258 Hz (sharp) OR 254 Hz (flat)\\\\
[KEY INSIGHT] How does tuner know which? Tighten the string:\\\\
- Beats slow down means piano was flat (now getting closer)\\\\
- Beats speed up means piano was sharp (now getting further)\\\\
[EMPHASIZE] When beat frequency equals zero, piano is in tune!}
\end{frame}

%=============================================================================
\section{Summary}
%=============================================================================

\begin{frame}
\frametitle{Chapter Summary}
\textbf{Key Takeaways:}

\pause
\begin{enumerate}
\item Sound is a longitudinal wave with compressions and rarefactions \pause
\item Speed of sound depends on medium properties (rigidity and density) \pause
\item Sound intensity is proportional to amplitude squared \pause
\item Decibels measure sound intensity level logarithmically \pause
\item Doppler effect: frequency changes with relative motion \pause
\item Resonance occurs when driving frequency equals natural frequency \pause
\item Standing waves in pipes produce harmonics
\end{enumerate}
\note{[QUICK] 7 key takeaways - review briefly\\\\
[EMPHASIZE] Most important: v equals f lambda, and the decibel formula\\\\
[REAL WORLD] Doppler for ambulances, resonance for musical instruments\\\\
[CHECK] Turn and tell: which concept was most surprising?\\\\
[TRANSITION] Let's see the key equations together...}
\end{frame}

\begin{frame}
\frametitle{Key Equations}
\begin{align}
v &= f\lambda \quad \text{(wave equation)} \\
I &= \frac{P}{A} \quad \text{(intensity)} \\
\beta &= 10\log_{10}\left(\frac{I}{I_0}\right) \quad \text{(decibels)} \\
f_{obs} &= f_s\left(\frac{v_w}{v_w \pm v_s}\right) \quad \text{(Doppler)} \\
f_B &= |f_1 - f_2| \quad \text{(beats)} \\
f_n &= n\frac{v}{4L} \quad \text{(closed pipe)} \\
f_n &= n\frac{v}{2L} \quad \text{(open pipe)}
\end{align}
\note{[SAY] "These all go on your formula sheet"\\\\
[QUICK] Run through each equation's use:\\\\
- v equals f lambda: any wave problem\\\\
- I equals P over A: intensity from power\\\\
- Decibel formula: log scale conversion\\\\
- Doppler: sign convention is key\\\\
- Pipes: closed uses 4L (odd n), open uses 2L (all n)\\\\
[CHECK] Name wheel: when would you use the Doppler equation?\\\\
ANSWER: When source or observer is moving relative to the other\\\\
[TRANSITION] Any final questions?}
\end{frame}

\begin{frame}
\frametitle{Homework}
\begin{center}
\Large
Complete the assigned problems\\[0.3cm]
posted on the LMS\\[0.8cm]
\pause
\textbf{Questions?}
\end{center}
\note{[SAY] "Homework is posted on the LMS"\\\\
[TIMING] Due date: check LMS\\\\
[CHECK] Any questions before we end?\\\\
[TRANSITION] Next class: Chapter 15 Light - electromagnetic waves}
\end{frame}

\end{document}
