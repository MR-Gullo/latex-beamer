\documentclass[12pt]{article}
\usepackage[margin=1in]{geometry}
\usepackage{amsmath}
\usepackage{graphicx}
\graphicspath{{../images/}{../../shared/images/}}
\usepackage{enumitem}
\usepackage{fancyhdr}
\usepackage{tcolorbox}
\usepackage{hyperref}

\pagestyle{fancy}
\fancyhead{}
\fancyhead[L]{Physics 11 - Unit 1}
\fancyhead[R]{Virtual Physics Lab}
\fancyfoot{}
\fancyfoot[C]{\thepage}

\begin{document}

\begin{center}
\Large\textbf{Virtual Physics: Graphing Lines}\\[0.5em]
\normalsize PhET Simulation Activity
\end{center}

\vspace{1em}

\section*{Overview}

In this simulation you will examine how changing the slope and $y$-intercept of an equation changes the appearance of a plotted line. Select \textbf{slope-intercept form} and drag the blue circles along the line to change the line's characteristics. Then, play the line game and see if you can determine the slope or $y$-intercept of a given line.

\begin{tcolorbox}[colback=blue!5!white,colframe=blue!75!black,title=Simulation Link]
\textbf{Graphing Lines} (University of Colorado - PhET)\\[0.5em]
\url{https://phet.colorado.edu/sims/html/graphing-lines/latest/graphing-lines_en.html}
\end{tcolorbox}

\vspace{1em}

\section*{Grasp Check}

How would the following changes affect a line that is neither horizontal nor vertical and has a positive slope?

\begin{enumerate}[label=\arabic*.]
\item Increase the slope but keep the $y$-intercept constant
\item Increase the $y$-intercept but keep the slope constant
\end{enumerate}

\vspace{1em}

\textbf{Select the correct answer:}

\begin{enumerate}[label=\Alph*.]
\item Increasing the slope will cause the line to rotate \textbf{clockwise} around the $y$-intercept. Increasing the $y$-intercept will cause the line to move \textbf{vertically up} on the graph without changing the line's slope.

\item Increasing the slope will cause the line to rotate \textbf{counter-clockwise} around the $y$-intercept. Increasing the $y$-intercept will cause the line to move \textbf{vertically up} on the graph without changing the line's slope.

\item Increasing the slope will cause the line to rotate \textbf{clockwise} around the $y$-intercept. Increasing the $y$-intercept will cause the line to move \textbf{horizontally right} on the graph without changing the line's slope.

\item Increasing the slope will cause the line to rotate \textbf{counter-clockwise} around the $y$-intercept. Increasing the $y$-intercept will cause the line to move \textbf{horizontally right} on the graph without changing the line's slope.
\end{enumerate}

\end{document}
