\documentclass[12pt]{article}
\usepackage[margin=1in]{geometry}
\usepackage{amsmath}
\usepackage{amssymb}
\usepackage{graphicx}
\graphicspath{{../images/}{../../shared/images/}}
\usepackage{enumitem}
\usepackage{fancyhdr}
\usepackage{tcolorbox}
\usepackage{hyperref}
\usepackage{longtable}
\usepackage{array}
\usepackage{booktabs}

\pagestyle{fancy}
\fancyhead{}
\fancyhead[L]{Physics 11 - Unit 4}
\fancyhead[R]{Elastic Collisions Lab}
\fancyfoot{}
\fancyfoot[C]{\thepage}

\begin{document}

\begin{center}
\Large\textbf{Ice Cubes and Elastic Collisions}\\[0.5em]
\normalsize Physics 11 - Mr. Gullo
\end{center}

\vspace{0.5em}
Name: \underline{\hspace{6cm}} \hfill Date: \underline{\hspace{3cm}}

\section*{Introduction}

This activity investigates elastic collisions by sliding ice cubes on a smooth surface. By minimizing friction and heat loss, the collision closely approximates perfect elastic behavior where both momentum and kinetic energy are conserved.

\section*{Materials Needed}

\begin{itemize}
\item Several ice cubes (uniform cube shape required)
\item Smooth surface (kitchen tabletop or glass table)
\end{itemize}

\section*{Procedure}

\begin{enumerate}
\item Find several ice cubes that are approximately the same size and a smooth kitchen tabletop or table with a glass top.
\item Place the ice cubes on the surface several centimeters away from each other.
\item Flick one ice cube toward a stationary ice cube and observe the path and velocities of the ice cubes after the collision. Try to avoid edge-on collisions and collisions with rotating ice cubes.
\item Explain the speeds and directions of the ice cubes using momentum principles.
\end{enumerate}

\begin{tcolorbox}[colback=blue!5!white,colframe=blue!75!black,title=Tips for Success]
Here's a trick for remembering which collisions are elastic and which are inelastic: \textbf{Elastic} is a bouncy material, so when objects bounce off one another in the collision and separate, it is an elastic collision. When they don't, the collision is inelastic.
\end{tcolorbox}

\section*{GRASP CHECK}

Was the collision elastic or inelastic?

\begin{enumerate}[label=\alph*.]
\item perfectly elastic
\item perfectly inelastic
\item nearly perfect elastic
\item nearly perfect inelastic
\end{enumerate}

\newpage

\section*{Lab Report}

Name: \underline{\hspace{5cm}} \hfill Partner: \underline{\hspace{5cm}}

Date: \underline{\hspace{3cm}} \hfill Class: \underline{\hspace{3cm}}

\subsection*{Pre-Lab Checklist}

$\square$ Ice cubes obtained (uniform size) \hfill
$\square$ Smooth surface identified \hfill
$\square$ Clear space for collision observations

\subsection*{Procedure Notes}

Record observations during the collision experiments:

\vspace{2cm}

\subsection*{Data Table: Collision Observations}

\begin{center}
\begin{tabular}{|c|p{2.5cm}|p{1.8cm}|p{1.8cm}|p{1.8cm}|p{1.8cm}|p{2cm}|}
\hline
\textbf{Trial} & \textbf{Collision Setup} & \textbf{A Initial} & \textbf{B Initial} & \textbf{A Final} & \textbf{B Final} & \textbf{Type} \\
\hline
1 & A moving, B stationary & & 0 (rest) & & & \\
\hline
2 & A fast, B stationary & & 0 (rest) & & & \\
\hline
3 & A slow, B stationary & & 0 (rest) & & & \\
\hline
4 & A fast, B slow (same dir) & & & & & \\
\hline
5 & Head-on collision & & & & & \\
\hline
6 & Glancing/off-center & & & & & \\
\hline
7 & Melted ice (wet) & & 0 (rest) & & & \\
\hline
8 & Your choice: & & & & & \\
\hline
\end{tabular}
\end{center}

\subsection*{Additional Observations}

Note any sticking, rotation, unusual motion, or surface conditions:

\vspace{2cm}

\newpage

\section*{Analysis Questions}

\begin{enumerate}
\item Describe the motion of both ice cubes after the collision. Did the moving ice cube stop? Did the stationary ice cube move?

\vspace{2cm}

\item Using the principle of conservation of momentum, explain why the ice cubes moved the way they did after the collision.

\vspace{2cm}

\item Was kinetic energy conserved in this collision? Explain your reasoning based on your observations.

\vspace{2cm}

\item What factors might cause this collision to be slightly inelastic rather than perfectly elastic?

\vspace{2cm}

\item As ice cubes melt, a thin layer of water forms between the ice and the surface. Explain how this water layer could cause surface tension effects that might make the ice cubes stick together momentarily during collision, resulting in a seemingly inelastic collision.

\vspace{2cm}

\item If this same experiment were performed outdoors in Canada during winter (at $-20$°C), how would the results differ? Would the collision be more or less elastic? Explain your reasoning.

\vspace{2cm}

\item As the ice cubes sit on the warm surface, they lose mass due to melting. If one ice cube has been on the surface longer than the other and has melted more, how would this mass difference affect the collision outcomes? Use momentum equations to support your answer.

\vspace{2cm}

\item Compare the results when a fast-moving ice cube hits a stationary one versus when both ice cubes are moving toward each other at similar speeds. Which scenario shows more dramatic post-collision velocities? Explain using conservation of momentum.

\vspace{2cm}

\end{enumerate}

\subsection*{GRASP CHECK Response}

Selected answer: \underline{\hspace{2cm}}

Explanation:

\vspace{2cm}

\subsection*{Photo Documentation}

Include photos showing:
\begin{itemize}
\item Experimental setup with ice cubes on smooth surface
\item Collision sequence (if possible)
\item Final positions after collision
\end{itemize}

\textbf{Submission:} Submit this completed template along with all graphs and photos as a single document (LastName\_FirstName.PDF required).

\end{document}
