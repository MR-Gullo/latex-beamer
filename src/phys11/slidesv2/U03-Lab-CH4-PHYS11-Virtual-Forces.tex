\documentclass[12pt]{article}
\usepackage[margin=1in]{geometry}
\usepackage{amsmath}
\usepackage{graphicx}
\graphicspath{{../images/}{../../../shared/images/}}
\usepackage{float}
\usepackage{enumitem}
\usepackage{fancyhdr}
\usepackage{tcolorbox}

\pagestyle{fancy}
\fancyhead{}
\fancyhead[L]{Physics 11 - Unit 3}
\fancyhead[R]{Virtual Forces Lab}
\fancyfoot{}
\fancyfoot[C]{\thepage}

\begin{document}

\begin{center}
{\Large \textbf{Virtual Lab Investigation:}}\\[0.5cm]
{\LARGE \textbf{Exploring Forces with PhET Simulations}}\\[1cm]
\rule{\textwidth}{0.4pt}
\end{center}

\section*{Lab Overview}
In this virtual lab, you will explore the fundamental concepts of forces using PhET interactive simulations. You will investigate forces in three distinct contexts: tug of war scenarios, friction analysis, and acceleration relationships.

\section*{Materials}
\begin{itemize}
\item Computer with internet access
\item PhET Forces and Motion: Basics simulation
\item PhET Friction simulation
\item Calculator
\item This lab handout
\end{itemize}

\section*{Part 1: Tug of War}

\subsection*{Procedure}
\begin{enumerate}
\item Navigate to the PhET "Forces and Motion: Basics" simulation
\item Select the "Tug of War" tab
\item Experiment with different combinations of people on each side
\item Complete 5 different scenarios in the data table below
\item For each scenario, record the number of people on each side and the result (left wins, right wins, or balanced)
\end{enumerate}

\subsection*{Data Table}
\begin{table}[H]
\centering
\begin{tabular}{|c|c|c|c|}
\hline
Trial & Left Side (people) & Right Side (people) & Result \\
\hline
1 & & & \\[0.3cm]
\hline
2 & & & \\[0.3cm]
\hline
3 & & & \\[0.3cm]
\hline
4 & & & \\[0.3cm]
\hline
5 & & & \\[0.3cm]
\hline
\end{tabular}
\end{table}

\subsection*{Screenshot}
Paste a screenshot of one balanced tug of war scenario here:

\vspace{4cm}

\section*{Part 2: Friction Analysis}

\subsection*{Procedure}
\begin{enumerate}
\item Open the "Friction" tab in the simulation
\item Place different objects on different surfaces
\item Apply forces and observe when objects start moving
\item Complete the friction comparison table
\item Observe how friction force changes with applied force
\end{enumerate}

\subsection*{Data Table}
\begin{table}[H]
\centering
\begin{tabular}{|c|c|c|}
\hline
Surface Type & Object & Force to Start Motion (N) \\
\hline
Wood & Book & \\[0.3cm]
\hline
Wood & Person & \\[0.3cm]
\hline
Ice & Book & \\[0.3cm]
\hline
Ice & Person & \\[0.3cm]
\hline
\end{tabular}
\end{table}

\subsection*{Analysis Questions}
\begin{enumerate}
\item Which surface has higher friction: wood or ice? How do you know?

\vspace{2cm}

\item Does a heavier object require more or less force to start moving? Explain.

\vspace{2cm}

\item What is the relationship between applied force and friction force before the object moves?

\vspace{2cm}
\end{enumerate}

\section*{Part 3: Acceleration Lab}

\subsection*{Procedure}
\begin{enumerate}
\item Open the "Acceleration" tab
\item Apply different amounts of force to objects of different masses
\item Record the resulting accelerations
\item Complete the data table for at least 6 trials
\end{enumerate}

\subsection*{Data Table}
\begin{table}[H]
\centering
\begin{tabular}{|c|c|c|}
\hline
Applied Force (N) & Mass (kg) & Acceleration (m/s²) \\
\hline
& & \\[0.3cm]
\hline
& & \\[0.3cm]
\hline
& & \\[0.3cm]
\hline
& & \\[0.3cm]
\hline
& & \\[0.3cm]
\hline
& & \\[0.3cm]
\hline
\end{tabular}
\end{table}

\subsection*{Analysis Questions}
\begin{enumerate}
\item What happens to acceleration when you double the force (keeping mass constant)?

\vspace{2cm}

\item What happens to acceleration when you double the mass (keeping force constant)?

\vspace{2cm}

\item Write Newton's Second Law equation and explain how your observations support it.

\vspace{2cm}
\end{enumerate}

\section*{GRASP Check}

\begin{tcolorbox}[colback=blue!5,colframe=blue!75!black,title=\textbf{GRASP Reflection}]
Before submitting, ensure your work demonstrates:
\begin{itemize}
\item[\textbf{G}] \textbf{Given:} Clear identification of known values in each scenario
\item[\textbf{R}] \textbf{Required:} Understanding of what you needed to find
\item[\textbf{A}] \textbf{Analysis:} Thoughtful responses to all questions
\item[\textbf{S}] \textbf{Solution:} Complete data tables with proper units
\item[\textbf{P}] \textbf{Paraphrase:} Clear explanations in your own words
\end{itemize}
\end{tcolorbox}

\section*{Submission}
Attach this completed handout along with your screenshots. Ensure all data tables are filled and all analysis questions are answered completely.

\end{document}
