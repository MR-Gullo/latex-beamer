\documentclass[12pt]{article}
\usepackage[margin=1in]{geometry}
\usepackage{amsmath}
\usepackage{graphicx}
\graphicspath{{../images/}{../../../shared/images/}}
\usepackage{float}
\usepackage{enumitem}
\usepackage{fancyhdr}
\usepackage{tcolorbox}

\pagestyle{fancy}
\fancyhead{}
\fancyhead[L]{Physics 11 - Unit 3}
\fancyhead[R]{Snap Friction Investigation}
\fancyfoot{}
\fancyfoot[C]{\thepage}

\begin{document}

\begin{center}
{\Large \textbf{Quick Investigation:}}\\[0.5cm]
{\LARGE \textbf{Exploring Friction Forces}}\\[1cm]
\rule{\textwidth}{0.4pt}
\end{center}

\section*{Investigation Overview}
In this quick investigation, you will explore how friction behaves on different surfaces and with different materials. You will make qualitative observations about friction and relate them to the key equation.

\section*{Key Equation}
The friction force is given by:
$$\boxed{f = \mu N}$$

where:
\begin{itemize}
\item $f$ = friction force
\item $\mu$ = coefficient of friction (depends on surface materials)
\item $N$ = normal force (typically equal to weight on horizontal surface)
\end{itemize}

\section*{Materials}
\begin{itemize}
\item Book or textbook
\item Coin
\item Eraser
\item Small wooden block (or similar object)
\item Various surfaces: desk, carpet, paper, fabric
\item Your hand for qualitative testing
\end{itemize}

\section*{Procedure}
\begin{enumerate}
\item Select an object from your materials
\item Place it on one of the test surfaces
\item Push the object gently with your hand and note how easily it moves
\item Rate the friction on a scale of 1-5:
    \begin{itemize}
    \item 1 = Very low friction (slides very easily)
    \item 5 = Very high friction (very difficult to slide)
    \end{itemize}
\item Repeat for different object-surface combinations
\item Complete the observation table below
\end{enumerate}

\section*{Observation Table}
\begin{table}[H]
\centering
\begin{tabular}{|p{3cm}|p{3cm}|c|p{4cm}|}
\hline
Object & Surface & Friction Rating (1-5) & Observations \\
\hline
& & & \\[1cm]
\hline
& & & \\[1cm]
\hline
& & & \\[1cm]
\hline
& & & \\[1cm]
\hline
& & & \\[1cm]
\hline
& & & \\[1cm]
\hline
& & & \\[1cm]
\hline
& & & \\[1cm]
\hline
\end{tabular}
\end{table}

\section*{Additional Observations}
Test the same object on the same surface with added weight (place another object on top). Describe what happens to the friction:

\vspace{3cm}

\section*{Analysis Questions}
\begin{enumerate}
\item Which combination of object and surface had the highest friction? Why do you think this is?

\vspace{3cm}

\item Which combination had the lowest friction?

\vspace{2cm}

\item How does adding weight to an object affect the friction force? Relate this to the equation $f = \mu N$.

\vspace{3cm}

\item In the equation $f = \mu N$, which factor depends on the materials (object and surface), and which depends on the weight?

\vspace{3cm}

\item Give a real-world example where:
\begin{enumerate}
\item High friction is desirable:

\vspace{2cm}

\item Low friction is desirable:

\vspace{2cm}
\end{enumerate}

\item Based on your observations, predict what would happen if you tried to slide a heavy book across:
\begin{enumerate}
\item A smooth wooden desk:

\vspace{2cm}

\item A carpeted floor:

\vspace{2cm}
\end{enumerate}
\end{enumerate}

\section*{Conclusion}
Write a brief summary of what you learned about friction from this investigation:

\vspace{4cm}

\section*{Extension Challenge}
If you have time, design a simple experiment to estimate the coefficient of friction between two specific materials. Describe your method:

\vspace{4cm}

\end{document}
