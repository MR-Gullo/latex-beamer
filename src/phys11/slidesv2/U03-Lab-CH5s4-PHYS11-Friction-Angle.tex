\documentclass[12pt]{article}
\usepackage[margin=1in]{geometry}
\usepackage{amsmath}
\usepackage{graphicx}
\graphicspath{{../images/}{../../../shared/images/}}
\usepackage{float}
\usepackage{enumitem}
\usepackage{fancyhdr}
\usepackage{tcolorbox}

\pagestyle{fancy}
\fancyhead{}
\fancyhead[L]{Physics 11 - Unit 3}
\fancyhead[R]{Friction at an Angle}
\fancyfoot{}
\fancyfoot[C]{\thepage}

\begin{document}

\begin{center}
{\Large \textbf{Laboratory Investigation:}}\\[0.5cm]
{\LARGE \textbf{Measuring Coefficient of Kinetic Friction}}\\[0.3cm]
{\large Using the Angle Method}\\[1cm]
\rule{\textwidth}{0.4pt}
\end{center}

\section*{Pre-Lab Checklist}
\begin{tcolorbox}[colback=yellow!10,colframe=orange!75!black,title=\textbf{Before Starting}]
Complete this checklist before beginning the experiment:
\begin{itemize}
\item[$\square$] Read through entire procedure
\item[$\square$] Understand the theory: $\mu_k = \tan\theta$
\item[$\square$] Gather all materials
\item[$\square$] Understand how to measure angles accurately
\item[$\square$] Review GRASP problem-solving method
\end{itemize}
\end{tcolorbox}

\section*{Theory}
When an object slides down an inclined plane at constant velocity, the forces are balanced. The component of gravity pulling the object down the slope equals the friction force opposing motion.

From force analysis:
\begin{align*}
F_{\parallel} &= F_{\text{friction}} \\
mg\sin\theta &= \mu_k N \\
mg\sin\theta &= \mu_k mg\cos\theta \\
\mu_k &= \frac{\sin\theta}{\cos\theta} = \tan\theta
\end{align*}

\textbf{Key Equation:} $\boxed{\mu_k = \tan\theta}$

where $\theta$ is the critical angle at which the object begins to slide with constant velocity.

\section*{Materials}
\begin{itemize}
\item Hardcover book (serves as inclined plane)
\item Coin (1g mass)
\item 100g mass (or similar small object)
\item Protractor or angle measuring app
\item Calculator
\end{itemize}

\section*{Procedure}

\subsection*{Part 1: Light Object (1g Coin)}
\begin{enumerate}
\item Place the coin on the horizontal book
\item Slowly lift one end of the book to increase the angle
\item Observe carefully and note the angle when the coin \textit{just begins} to slide
\item Give the coin a gentle tap to start it moving
\item Adjust the angle until the coin slides at approximately constant velocity
\item Measure and record this critical angle
\item Repeat 5 times for consistency
\item Calculate $\mu_k = \tan\theta$ for each trial
\end{enumerate}

\subsection*{Part 2: Heavier Object (100g Mass)}
\begin{enumerate}
\item Repeat the same procedure with the 100g mass
\item Record 5 trials
\item Calculate $\mu_k$ for each trial
\end{enumerate}

\section*{Data Tables}

\subsection*{Part 1: 1g Coin}
\begin{table}[H]
\centering
\begin{tabular}{|c|c|c|}
\hline
Trial & Angle $\theta$ (degrees) & $\mu_k = \tan\theta$ \\
\hline
1 & & \\[0.4cm]
\hline
2 & & \\[0.4cm]
\hline
3 & & \\[0.4cm]
\hline
4 & & \\[0.4cm]
\hline
5 & & \\[0.4cm]
\hline
\textbf{Average} & & \\[0.4cm]
\hline
\end{tabular}
\end{table}

\subsection*{Part 2: 100g Mass}
\begin{table}[H]
\centering
\begin{tabular}{|c|c|c|}
\hline
Trial & Angle $\theta$ (degrees) & $\mu_k = \tan\theta$ \\
\hline
1 & & \\[0.4cm]
\hline
2 & & \\[0.4cm]
\hline
3 & & \\[0.4cm]
\hline
4 & & \\[0.4cm]
\hline
5 & & \\[0.4cm]
\hline
\textbf{Average} & & \\[0.4cm]
\hline
\end{tabular}
\end{table}

\section*{Sample Calculation}
Show your work for calculating $\mu_k$ from one trial:

\textbf{Given:}

\vspace{1cm}

\textbf{Required:}

\vspace{1cm}

\textbf{Analysis \& Solution:}

\vspace{3cm}

\textbf{Paraphrase:}

\vspace{1cm}

\section*{Analysis Questions}
\begin{enumerate}
\item Does the mass of the object affect the coefficient of kinetic friction? Use your data to support your answer.

\vspace{3cm}

\item Why is it important to ensure the object slides at constant velocity?

\vspace{3cm}

\item What are the major sources of error in this experiment? How would you reduce them?

\vspace{3cm}

\item Compare your measured coefficient of friction to published values for similar material pairs. How close are your results?

\vspace{3cm}

\item Explain why $\mu_k = \tan\theta$ is independent of mass using the force equations.

\vspace{3cm}
\end{enumerate}

\section*{GRASP Check}

\begin{tcolorbox}[colback=blue!5,colframe=blue!75!black,title=\textbf{GRASP Reflection}]
Before submitting, verify:
\begin{itemize}
\item[\textbf{G}] \textbf{Given:} Angles clearly recorded with proper units
\item[\textbf{R}] \textbf{Required:} Coefficient of friction calculated for all trials
\item[\textbf{A}] \textbf{Analysis:} Force diagram understanding demonstrated
\item[\textbf{S}] \textbf{Solution:} All calculations shown with proper steps
\item[\textbf{P}] \textbf{Paraphrase:} Results explained in context
\end{itemize}
\end{tcolorbox}

\section*{Conclusion}
Write a brief conclusion summarizing:
\begin{itemize}
\item Your measured coefficient of kinetic friction
\item Whether mass affects friction coefficient
\item The validity of the theoretical equation $\mu_k = \tan\theta$
\item Sources of experimental uncertainty
\end{itemize}

\vspace{4cm}

\end{document}
