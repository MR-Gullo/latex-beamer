\documentclass[12pt]{article}
\usepackage[margin=1in]{geometry}
\usepackage{amsmath}
\usepackage{graphicx}
\graphicspath{{../images/}{../../shared/images/}}
\usepackage{enumitem}
\usepackage{fancyhdr}
\usepackage{tcolorbox}
\usepackage{hyperref}

\pagestyle{fancy}
\fancyhead{}
\fancyhead[L]{Physics 11 - Unit 2}
\fancyhead[R]{Field Lab: Kinematics}
\fancyfoot{}
\fancyfoot[C]{\thepage}

\begin{document}

\begin{center}
\Large\textbf{Physics 11 Field Lab: Kinematics Analysis of Shanghai Metro Line 3}\\[0.5em]
\normalsize Mr. Gullo
\end{center}

\section{Objective}

Measure and analyze acceleration and velocity of Shanghai Metro Line 3 train during departure from station using only basic tools (paper, pen, stopwatch, measuring tape), and compare experimental results with official train performance specifications.

\section{Background}

Urban rail systems like Shanghai Metro provide excellent real-world contexts for studying one-dimensional kinematics. Trains accelerate from rest, reach cruising speed, then decelerate: clear phases of motion governed by equations of constant acceleration. Shanghai Metro Line 3, opened 2001, operates modern AC03-series trains with well-documented performance characteristics.

\subsection{Official Train Specifications (for comparison)}

\begin{itemize}
\item Maximum operating speed: 80 km/h ($\approx 22.2$ m/s)
\item Average speed (including stops): 34.4 km/h ($\approx 9.6$ m/s)
\item Service acceleration: 1.0 km/(h$\cdot$s), equivalent to 0.278 m/s$^2$
\item Maximum (design) acceleration: 0.9-1.0 km/(h$\cdot$s) under normal conditions
\item Emergency deceleration: up to 1.3 km/(h$\cdot$s) ($\approx 0.361$ m/s$^2$)
\item Train model: AC03 (also designated 03A01), 6-car aluminum-bodied train manufactured by Alstom and CSR Nanjing Puzhen
\item Power supply: 1,500 V DC via overhead lines
\end{itemize}

These values provide benchmark for evaluating student-collected data.

\section{Materials (per group)}

\begin{itemize}
\item Paper
\item Pen or pencil
\item Stopwatch (smartphone acceptable)
\item Measuring tape
\end{itemize}

\section{Location}

Any above-ground station on Shanghai Metro Line 3, such as Shilong Road, Longcao Road, or Caoxi Road. These stations offer clear visibility of train motion along platform and are easily accessible. Line 3 runs from North Jiangyang Road to Shanghai South Railway Station, covering 40.3 km with frequent service.

\begin{tcolorbox}[colback=red!5!white,colframe=red!75!black,title=Safety First]
\begin{itemize}
\item Stay behind yellow safety line at all times
\item Do not distract other passengers
\item Follow all station staff instructions
\item Teacher must supervise all measurements
\end{itemize}
\end{tcolorbox}

\section{Procedure}

\subsection{Part 1: Establish a Known Distance}

\begin{itemize}
\item Use measuring tape to mark straight segment parallel to tracks along platform edge
\item Recommended distance: 10.0 meters (e.g., between two pillars or platform markers)
\item Record this as $\Delta x = $ \underline{\hspace{2cm}} m
\end{itemize}

\subsection{Part 2: Time the Train's Motion}

\begin{itemize}
\item Wait for train to arrive and come to complete stop
\item When train departs:
\begin{itemize}
\item Start stopwatch instant front of train passes start of measured segment
\item Stop stopwatch when front of train passes end of segment
\end{itemize}
\item Record time as $\Delta t$
\item Repeat for 1-2 train departures (time permitting) to obtain average time and reduce human reaction error
\end{itemize}

\textbf{Optional Extension:} Also time train over first 5 meters after it begins moving to isolate acceleration phase.

\section{Calculations (to be completed in class)}

\subsection{Case 1: Assuming Constant Acceleration from Rest}

If train is still accelerating over your measured segment:
\[
\Delta x = \frac{1}{2}a(\Delta t)^2 \quad \Rightarrow \quad a = \frac{2\Delta x}{(\Delta t)^2}
\]

\subsection{Case 2: Assuming Constant Velocity}

If train has already reached cruising speed (more likely beyond first 10 m):
\[
v = \frac{\Delta x}{\Delta t}
\]

\section{Analysis Questions}

\begin{enumerate}
\item What is your experimental acceleration (in m/s$^2$)? How does it compare to official service acceleration of 0.278 m/s$^2$?

\vspace{2cm}

\item What is your estimated speed (in m/s and km/h)? Is it reasonable given train's maximum speed of 80 km/h?

\vspace{2cm}

\item Why is average speed of Line 3 only 34.4 km/h despite top speed of 80 km/h?

\vspace{2cm}

\item Identify two sources of experimental error (e.g., reaction time, distance estimation). How might they affect your results?

\vspace{2cm}

\end{enumerate}

\section{Extensions (Optional)}

\begin{itemize}
\item Measure deceleration as train approaches station
\item Compare acceleration during peak vs. off-peak hours (though train performance should be consistent)
\item Estimate distance traveled during acceleration using your calculated $a$ and known time to reach cruising speed
\end{itemize}

\section{References}

\begin{itemize}
\item Shanghai Metro Line 3. Wikipedia. \url{https://en.wikipedia.org/wiki/Line_3_(Shanghai_Metro)}
\item Shanghai Metro AC03. Wikipedia. \url{https://en.wikipedia.org/wiki/Shanghai_Metro_AC03}
\item Shanghai Metro Line 3. UrbanRail.Net. \url{https://www.urbanrail.net/as/shan/shanghai.htm}
\end{itemize}

\end{document}
