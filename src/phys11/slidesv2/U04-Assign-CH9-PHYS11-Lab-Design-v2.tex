\documentclass[12pt]{article}
\usepackage[margin=1in]{geometry}
\usepackage{amsmath}
\usepackage{graphicx}
\graphicspath{{../images/}{../../shared/images/}}
\usepackage{enumitem}
\usepackage{fancyhdr}
\usepackage{tcolorbox}
\usepackage{hyperref}

\pagestyle{fancy}
\fancyhead{}
\fancyhead[L]{Physics 11 - Unit 4}
\fancyhead[R]{Lab Design Assignment}
\fancyfoot{}
\fancyfoot[C]{\thepage}

\begin{document}

\begin{center}
\Large\textbf{Lab Design Assignment: Work-Energy Investigation}\\[0.3em]
\large Steel Ball Drop\\[0.5em]
\normalsize Physics 11 - Mr. Gullo
\end{center}

\section*{Overview}

This experimental investigation has been structured to fit an 80-minute lab period, with an additional 40-minute design period the week before. Students will use a timer-controlled dropping mechanism and photo-gate to study concepts from Chapter 9 on work, energy, and conservation principles.

\section*{Schedule}

\textbf{Week 1:} Design Period (40 minutes)

\textbf{Week 2:} Lab Implementation (80 minutes)

\section*{Required Equipment}

\begin{itemize}
\item Timer-controlled dropping mechanism
\item Steel ball
\item Photo-gate timing system
\item Electronic balance
\item Calculator
\item Laptop with spreadsheet software (optional)
\item Lab notebook
\end{itemize}

\section*{Pre-Lab Design Requirements (Week 1)}

During the 40-minute design period, students should:

\begin{enumerate}
\item Review textbook sections 9.1 and 9.2 to identify relevant equations
\item Design a systematic approach for data collection
\item Create a data table template
\item Develop a hypothesis about the relationship between drop height and ball velocity
\item Prepare prediction calculations for at least three different drop heights
\item Identify potential sources of experimental error
\end{enumerate}

\section*{Experimental Procedure (Week 2)}

\subsection*{Setup Phase (15 minutes)}

\begin{enumerate}
\item Measure and record the mass of the steel ball
\item Mount the dropping mechanism securely above the photo-gate
\item Test the timing system and dropping mechanism
\item Establish a consistent method for measuring drop heights
\end{enumerate}

\subsection*{Data Collection Phase (30 minutes)}

\begin{enumerate}
\item Select 5 different drop heights between 0.2m and 1.0m
\item For each height:
\begin{itemize}
\item Record the height measurement precisely
\item Calculate the theoretical gravitational potential energy
\item Predict the velocity at the photo-gate using conservation of energy
\item Perform 3 trial drops, recording the photo-gate time measurements
\item Calculate the actual velocity from the photo-gate data
\end{itemize}
\end{enumerate}

\subsection*{Analysis Phase (20 minutes)}

\begin{enumerate}
\item Calculate for each trial:
\begin{itemize}
\item Initial gravitational potential energy ($mgh$)
\item Theoretical final kinetic energy ($\frac{1}{2}mv^2$) based on energy conservation
\item Actual final kinetic energy based on measured velocity
\item Percentage of energy conserved
\item Work done by gravity
\item Average power during the fall (work/time)
\end{itemize}
\item Create a graph of potential energy vs. measured kinetic energy
\item Determine the slope of the best-fit line and its physical meaning
\end{enumerate}

\subsection*{Discussion Phase (10 minutes)}

\begin{enumerate}
\item Compare theoretical predictions to experimental results
\item Analyze discrepancies and potential energy losses
\item Discuss experimental uncertainties and their impact
\item Relate findings to the work-energy theorem and conservation principles
\end{enumerate}

\section*{Submission Requirements}

\textbf{By end of design period (Week 1):}
\begin{itemize}
\item Completed experimental design
\item Prediction calculations for at least three heights
\end{itemize}

\textbf{By end of lab period (Week 2):}
\begin{itemize}
\item Completed data tables
\item Preliminary graphs
\item Initial analysis of results
\end{itemize}

\textbf{Final report (due one week after lab):}
\begin{itemize}
\item Introduction with hypothesis
\item Methods section with detailed procedure
\item Results with data tables and graphs
\item Analysis connecting to work-energy theorem and conservation principles
\item Discussion of error sources
\item Conclusion evaluating the hypothesis
\end{itemize}

\section*{Safety Considerations}

\begin{itemize}
\item Secure all equipment properly to prevent falling objects
\item Ensure clear space around the drop zone
\item Handle equipment with care to prevent damage
\end{itemize}

\end{document}
