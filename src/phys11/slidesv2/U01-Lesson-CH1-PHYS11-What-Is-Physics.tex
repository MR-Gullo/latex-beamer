% Options for packages loaded elsewhere
\PassOptionsToPackage{unicode}{hyperref}
\PassOptionsToPackage{hyphens}{url}
\documentclass[
]{article}
\usepackage{xcolor}
\usepackage{amsmath,amssymb}
\setcounter{secnumdepth}{-\maxdimen} % remove section numbering
\usepackage{iftex}
\ifPDFTeX
  \usepackage[T1]{fontenc}
  \usepackage[utf8]{inputenc}
  \usepackage{textcomp} % provide euro and other symbols
\else % if luatex or xetex
  \usepackage{unicode-math} % this also loads fontspec
  \defaultfontfeatures{Scale=MatchLowercase}
  \defaultfontfeatures[\rmfamily]{Ligatures=TeX,Scale=1}
\fi
\usepackage{lmodern}
\ifPDFTeX\else
  % xetex/luatex font selection
\fi
% Use upquote if available, for straight quotes in verbatim environments
\IfFileExists{upquote.sty}{\usepackage{upquote}}{}
\IfFileExists{microtype.sty}{% use microtype if available
  \usepackage[]{microtype}
  \UseMicrotypeSet[protrusion]{basicmath} % disable protrusion for tt fonts
}{}
\makeatletter
\@ifundefined{KOMAClassName}{% if non-KOMA class
  \IfFileExists{parskip.sty}{%
    \usepackage{parskip}
  }{% else
    \setlength{\parindent}{0pt}
    \setlength{\parskip}{6pt plus 2pt minus 1pt}}
}{% if KOMA class
  \KOMAoptions{parskip=half}}
\makeatother
\usepackage{graphicx}
\graphicspath{{../images/}{../../shared/images/}}
\makeatletter
\newsavebox\pandoc@box
\newcommand*\pandocbounded[1]{% scales image to fit in text height/width
  \sbox\pandoc@box{#1}%
  \Gscale@div\@tempa{\textheight}{\dimexpr\ht\pandoc@box+\dp\pandoc@box\relax}%
  \Gscale@div\@tempb{\linewidth}{\wd\pandoc@box}%
  \ifdim\@tempb\p@<\@tempa\p@\let\@tempa\@tempb\fi% select the smaller of both
  \ifdim\@tempa\p@<\p@\scalebox{\@tempa}{\usebox\pandoc@box}%
  \else\usebox{\pandoc@box}%
  \fi%
}
% Set default figure placement to htbp
\def\fps@figure{htbp}
\makeatother
\setlength{\emergencystretch}{3em} % prevent overfull lines
\providecommand{\tightlist}{%
  \setlength{\itemsep}{0pt}\setlength{\parskip}{0pt}}
\usepackage{bookmark}
\IfFileExists{xurl.sty}{\usepackage{xurl}}{} % add URL line breaks if available
\urlstyle{same}
\hypersetup{
  hidelinks,
  pdfcreator={LaTeX via pandoc}}

\author{}
\date{}

\begin{document}

\section{1 \textbar{} What Is Physics?}\label{what-is-physics}

The Instructor Answer Guide for Physics contains the worked-out answers
to questions that appear at the~\textbf{end of each~chapter}.~ Grasp
Check Questions, which occur in the body text of sections, and Practice
Problems and Check Your Understanding exercises, which occur at the end
of sections, are not included in the Instructor Answer Guide for this
book.

\subsection{Chapter Review}\label{chapter-review}

\subsubsection{Concept Items}\label{concept-items}

\paragraph{1.1 Physics: Definitions and
Applications}\label{physics-definitions-and-applications}

\begin{enumerate}
\def\labelenumi{\arabic{enumi}.}
\item
  Which statement best compares and contrasts the aims and topics of
  natural philosophy versus physics?
\end{enumerate}

\begin{enumerate}
\def\labelenumi{\Alph{enumi}.}
\item
  Natural philosophy included all aspects of nature including physics.
\item
  Natural philosophy included all aspects of nature excluding physics.
\item
  Natural philosophy and physics are different.
\item
  Natural philosophy and physics are essentially the same thing.
\end{enumerate}

\textbf{Solution} The correct answer is (A). Natural philosophy included
all aspects of nature by lumping physics in with other fields of
science, such as chemistry and biology. Now, physics mainly describes
the most fundamental aspects of our universe, such as motion, energy,
matter, and space.

\begin{enumerate}
\def\labelenumi{\arabic{enumi}.}
\setcounter{enumi}{1}
\item
  Which statement is NOT an underlying assumption essential to
  scientific understandings?
\end{enumerate}

\begin{enumerate}
\def\labelenumi{\Alph{enumi}.}
\item
  Characteristics of the physical universe can be perceived and
  objectively measured by human beings.
\item
  Explanations of natural phenomena can be established with absolute
  certainty.
\item
  Fundamental physical processes dictate how characteristics of the
  physical universe evolve.
\item
  The fundamental processes of nature operate the same way everywhere
  and at all times.
\end{enumerate}

\textbf{Solution} The correct answer is (B). Scientific explanations may
be supported by all available evidence at a particular point in time,
but science never precludes the possibility that some future observation
could contradict the explanation, requiring a modification to the
existing explanation or an entirely new explanation.

\begin{enumerate}
\def\labelenumi{\arabic{enumi}.}
\setcounter{enumi}{2}
\item
  Which question regarding a strain of genetically modified rice is NOT
  one that science can answer?
\end{enumerate}

\begin{enumerate}
\def\labelenumi{\Alph{enumi}.}
\item
  How does the yield of the genetically modified rice compare with that
  of existing rice?
\item
  Is the genetically modified rice more resistant to infestation than
  existing rice?
\item
  How does the nutritional value of the genetically modified rice
  compare to that of existing rice?
\item
  Should the genetically modified rice be grown commercially and sold in
  the marketplace?
\end{enumerate}

\textbf{Solution} The correct answer is (D). Administrators need to make
a judgment call based on scientific evidence regarding the safety and
commercial viability of the genetically modified rice.

\begin{enumerate}
\def\labelenumi{\arabic{enumi}.}
\setcounter{enumi}{3}
\item
  What conditions imply that we can use classical physics without
  considering special relativity or quantum mechanics?
\end{enumerate}

\begin{enumerate}
\def\labelenumi{\Alph{enumi}.}
\item
  1. Matter is moving at speeds of less than roughly 1 percent the speed
  of light.
\end{enumerate}

\begin{quote}
2. Objects are large enough to be seen with the naked eye.

3. There is the involvement of a strong gravitational field.
\end{quote}

\begin{enumerate}
\def\labelenumi{\Alph{enumi}.}
\setcounter{enumi}{1}
\item
  1. Matter is moving at speeds greater than roughly 1 percent the speed
  of light.
\end{enumerate}

\begin{quote}
2. Objects are large enough to be seen with the naked eye.

3. There is the involvement of a strong gravitational field.
\end{quote}

\begin{enumerate}
\def\labelenumi{\Alph{enumi}.}
\setcounter{enumi}{2}
\item
  1. Matter is moving at speeds of less than roughly 1 percent the speed
  of light.
\end{enumerate}

\begin{quote}
2. Objects are too small to be seen with the naked eye.

3. There is the involvement of only a weak gravitational field.
\end{quote}

\begin{enumerate}
\def\labelenumi{\Alph{enumi}.}
\setcounter{enumi}{3}
\item
  1. Matter is moving at speeds of less than roughly 1 percent the speed
  of light.
\end{enumerate}

\begin{quote}
2. Objects are large enough to be seen with the naked eye.

3. There is the involvement of a weak gravitational field.
\end{quote}

\textbf{Solution} The correct answer is (D). The conditions that must be
met include: (1) matter must be moving at speeds less than about 1
percent of the speed of light; (2) microscopic (not subatomic) particles
as well as those visible with the naked eye; (3) only weak gravitational
fields, such as the field generated by Earth, can be involved.

\begin{enumerate}
\def\labelenumi{\arabic{enumi}.}
\setcounter{enumi}{4}
\item
  How could physics be useful in weather prediction?
\end{enumerate}

\begin{enumerate}
\def\labelenumi{\Alph{enumi}.}
\item
  Physics helps predict how burning fossil fuel releases pollutants.
\item
  Physics helps predict dynamics and movement of weather phenomena.
\item
  Physics helps predict the motion of tectonic plates.
\item
  Physics helps predict how the flowing water affects Earth's surface.
\end{enumerate}

\textbf{Solution} The correct answer is (B). Physics is useful in
weather prediction because the physical laws of motion and energy govern
the movement of air masses and energy in our atmosphere, which is what
generates weather.

\begin{enumerate}
\def\labelenumi{\arabic{enumi}.}
\setcounter{enumi}{5}
\item
  How do physical therapists use physics while on the job? Explain.
\end{enumerate}

\begin{enumerate}
\def\labelenumi{\Alph{enumi}.}
\item
  Physical therapists do not require knowledge of physics because their
  job is mainly therapy and not physics.
\item
  Physical therapists do not require knowledge of physics because their
  job is more social in nature and unscientific.
\item
  Physical therapists require knowledge of physics to know about muscle
  contraction and release of energy.
\item
  Physical therapists require knowledge of physics to know about
  chemical reactions inside the body and make decisions accordingly.
\end{enumerate}

\textbf{Solution} The correct answer is (C). Physical therapists must
understand how muscles generate forces that allow the body to move and
bend. Physics allows us to understand this.

\begin{enumerate}
\def\labelenumi{\arabic{enumi}.}
\setcounter{enumi}{6}
\item
  What is meant when a physical law is said to be universal?
\end{enumerate}

\begin{enumerate}
\def\labelenumi{\Alph{enumi}.}
\item
  The law can explain everything in the universe.
\item
  The law is applicable to all physical phenomena.
\item
  The law applies everywhere in the universe.
\item
  The law is the most basic one and all laws are derived from it.
\end{enumerate}

\textbf{Solution} The correct answer is (C). A universal law is one that
explains a specific physics phenomenon, such as how the force of gravity
is based on the mass of an object and that object's distance from
another object. To be universal, this law should apply everywhere in the
universe.

\begin{enumerate}
\def\labelenumi{\arabic{enumi}.}
\setcounter{enumi}{7}
\item
  What subfield of physics could describe small objects traveling at
  high speeds or experiencing a strong gravitational field?
\end{enumerate}

\begin{enumerate}
\def\labelenumi{\Alph{enumi}.}
\item
  General theory of relativity
\item
  Classical physics
\item
  Quantum relativity
\item
  Special theory of relativity
\end{enumerate}

\textbf{Solution} The correct answer is (C). Quantum relativity
describes small objects traveling at high speeds or experiencing a
strong gravitational field.

\begin{enumerate}
\def\labelenumi{\arabic{enumi}.}
\setcounter{enumi}{8}
\item
  Why is Einstein's theory of relativity considered part of modern
  physics, as opposed to classical physics?
\end{enumerate}

\begin{enumerate}
\def\labelenumi{\Alph{enumi}.}
\item
  Because it is considered less outstanding than the classics of
  physics, such as classical mechanics
\item
  Because popular physics is enjoyed by average people today, instead of
  physics studied by the elite
\item
  Because the theory deals with very slow-moving objects and weak
  gravitational fields
\item
  Because it was among the new 19th-century discoveries that changed
  physics
\end{enumerate}

\textbf{Solution} The correct answer is (D). Einstein's theory of
relativity is considered part of modern physics because it was developed
alongside other far-reaching scientific advances in the 19th century.
These advances radically changed ideas in physics from those accepted in
the previous (classical) period dominated by Newton and Galileo.

\paragraph{1.2 The Scientific Methods}\label{the-scientific-methods}

\begin{enumerate}
\def\labelenumi{\arabic{enumi}.}
\setcounter{enumi}{9}
\item
  What is the difference between an observation and a hypothesis?
\end{enumerate}

\begin{enumerate}
\def\labelenumi{\Alph{enumi}.}
\item
  An observation is seeing what happens; a hypothesis is a testable,
  educated guess.
\item
  An observation is a hypothesis that has been confirmed.
\item
  Hypotheses and observations are independent of each other.
\item
  Hypotheses are conclusions based on some observations.
\end{enumerate}

\textbf{Solution} The correct answer is (A). An observation is a pattern
or trend that a person sees in nature while a hypothesis is an educated
guess that attempts to explain something about the observation.

\begin{enumerate}
\def\labelenumi{\arabic{enumi}.}
\setcounter{enumi}{10}
\item
  How is modeling useful in studying the structure of the atom?
\end{enumerate}

\begin{enumerate}
\def\labelenumi{\Alph{enumi}.}
\item
  Modeling replaces the real system with something similar but easier to
  examine.
\item
  Modeling replaces the real system with something more interesting to
  examine.
\item
  Modeling replaces the real system with something with more realistic
  properties.
\item
  Modeling includes more details than are present in the real system.
\end{enumerate}

\textbf{Solution} The correct answer is (A). Modeling is useful in
studying the structure of the atom because models replace the real atom
with something simpler that has similar features and similar expected
behavior.

\begin{enumerate}
\def\labelenumi{\arabic{enumi}.}
\setcounter{enumi}{11}
\item
  How strongly is a hypothesis supported by evidence compared to a
  theory?
\end{enumerate}

\begin{enumerate}
\def\labelenumi{\Alph{enumi}.}
\item
  A theory is supported by little evidence, if any, at first. A
  hypothesis is supported by a large amount of available evidence.
\item
  A hypothesis is supported by little evidence, if any, at first. A
  theory is supported by a large amount of available evidence.
\item
  A hypothesis is supported by little evidence, if any, at first. A
  theory does not need any experiments for support.
\item
  A theory is supported by little evidence, if any, at first. A
  hypothesis does not need any experiments for support.
\end{enumerate}

\textbf{Solution} The correct answer is (B). A hypothesis is an
experimental test and doesn't necessarily need much evidence to prove
the case. A theory, however, has been proven and needs to be supported
so that any retrial shows that same theory holds true.

\paragraph{1.3 The Language of Physics: Physical Quantities and
Units}\label{the-language-of-physics-physical-quantities-and-units}

\begin{enumerate}
\def\labelenumi{\arabic{enumi}.}
\setcounter{enumi}{12}
\item
  Which option does NOT contribute to uncertainty?
\end{enumerate}

\begin{enumerate}
\def\labelenumi{\Alph{enumi}.}
\item
  The limitations of the measuring device
\item
  The skill of the person making the measurement
\item
  The regularities in the object being measured
\item
  Other factors that affect the outcome (depending on the situation)
\end{enumerate}

\textbf{Solution} The correct answer is (C). The object itself is what
it is made of and does not take a skill in measurements like the rest of
the choices.

\begin{enumerate}
\def\labelenumi{\arabic{enumi}.}
\setcounter{enumi}{13}
\item
  How does the independent variable in a graph differ from the dependent
  variable?
\end{enumerate}

\begin{enumerate}
\def\labelenumi{\Alph{enumi}.}
\item
  The dependent variable varies linearly with the independent variable.
\item
  The dependent variable depends on the scale of the axis chosen, while
  the independent variable does not.
\item
  The independent variable is directly manipulated or controlled by the
  person doing the experiment, while dependent variable is the one that
  changes as a result.
\item
  The dependent and independent variables are fixed by convention;
  hence, they are the same.
\end{enumerate}

\textbf{Solution} The correct answer is (C). The independent variable
does not change as a result of changes in the dependent variable. The
dependent variable, on the other hand, does change as a result of
changes in the independent variable.

\begin{enumerate}
\def\labelenumi{\arabic{enumi}.}
\setcounter{enumi}{14}
\item
  What could you conclude about these two lines?
\end{enumerate}

\begin{quote}
Line A has a slope of
\includegraphics[width=0.34236in,height=0.15764in]{media/image1.png}.

Line B has a slope of 12.0.
\end{quote}

\begin{enumerate}
\def\labelenumi{\Alph{enumi}.}
\item
  Line A is decreasing while line B is increasing, with line A being
  much steeper than line~B.
\item
  Line A is decreasing while line B is increasing, with line B being
  much steeper than line~A.
\item
  Line B is decreasing while line A is increasing, with line A being
  much steeper than line~B.
\item
  Line B is decreasing while line A is increasing, with line B being
  much steeper than line~A.
\end{enumerate}

\textbf{Solution} The correct answer is (B). You could conclude that
line A is decreasing while line~B is increasing, with line B being much
steeper than line A.

\begin{enumerate}
\def\labelenumi{\arabic{enumi}.}
\setcounter{enumi}{15}
\item
  Velocity, or speed, is measured using the formula
\end{enumerate}

\begin{quote}
\includegraphics[width=0.40764in,height=0.44444in]{media/image2.png},

where \emph{v} is velocity, \emph{d} is the distance traveled, and
\emph{t} is the time the object took to travel the distance.

If the velocity-time data are plotted on a graph, which variable will be
on which\\
axis? Why?
\end{quote}

\begin{enumerate}
\def\labelenumi{\Alph{enumi}.}
\item
  Time would be on the \emph{x}-axis and velocity on the \emph{y}-axis,
  because time is an independent variable and velocity is a dependent
  variable.
\item
  Velocity would be on the \emph{x}-axis and time on the \emph{y}-axis,
  because time is the independent variable and velocity is the dependent
  variable.
\item
  Time would be on the \emph{x}-axis and velocity on the \emph{y}-axis,
  because time is a dependent variable and velocity is an independent
  variable.
\item
  Velocity would be on the \emph{x}-axis and time on the \emph{y}-axis,
  because time is a dependent variable and velocity is an independent
  variable.
\end{enumerate}

\textbf{Solution} The correct answer is (A). Time would be on the
\emph{x}-axis and distance would be on the \emph{y}-axis, because time
passes independently of how far the car travels. Therefore, time is the
independent variable while distance is the dependent variable.

\begin{enumerate}
\def\labelenumi{\arabic{enumi}.}
\setcounter{enumi}{16}
\item
  A good-quality measuring tape can be off by 0.50 cm over a distance of
  20 m. What is its percent uncertainty in scientific notation?
\end{enumerate}

\begin{enumerate}
\def\labelenumi{\Alph{enumi}.}
\item
  \includegraphics[width=0.77778in,height=0.22222in]{media/image3.png}
\item
  \includegraphics[width=0.71319in,height=0.22222in]{media/image4.png}
\item
  \includegraphics[width=0.71319in,height=0.22222in]{media/image5.png}
\item
  \includegraphics[width=0.71319in,height=0.22222in]{media/image6.png}
\end{enumerate}

\textbf{Solution} The correct answer is (A).

\includegraphics[width=2.85208in,height=0.44444in]{media/image7.png}

\begin{enumerate}
\def\labelenumi{\arabic{enumi}.}
\setcounter{enumi}{17}
\item
  What is the definition of uncertainty?
\end{enumerate}

\begin{enumerate}
\def\labelenumi{\Alph{enumi}.}
\item
  Uncertainty is the number of assumptions made prior to the measurement
  of a physical quantity.
\item
  Uncertainty is a measure of error in a measurement due to the use of a
  noncalibrated instrument.
\item
  Uncertainty is a measure of deviation of the measured value from the
  standard value.
\item
  Uncertainty is a measure of error in measurement due to external
  factors like air friction and temperature.
\end{enumerate}

\textbf{Solution} The correct answer is (C). Uncertainty is a
quantitative measure of how much your measured values deviate from a
standard or expected value.

\subsubsection{Critical Thinking Items}\label{critical-thinking-items}

\paragraph{1.1 Physics: Definitions and
Applications}\label{physics-definitions-and-applications-1}

\begin{enumerate}
\def\labelenumi{\arabic{enumi}.}
\setcounter{enumi}{18}
\item
  In what sense does Einstein's theory of relativity illustrate that
  physics describes fundamental aspects of our universe?
\end{enumerate}

\begin{enumerate}
\def\labelenumi{\Alph{enumi}.}
\item
  It describes how speed affects different observers' measurements of
  time and space.
\item
  It describes how different parts of the universe are far apart and do
  not affect each other.
\item
  It describes how people think of other people's views from their own
  frame of reference.
\item
  It describes how a frame of reference is necessary to describe
  position or motion.
\end{enumerate}

\textbf{Solution} The correct answer is (A). According to
Einstein\textquotesingle s theory of relativity, gravity or the speed of
the observer can affect what they observe as measured time intervals and
measured distances, as occurs with length contraction or time dilation.

\begin{enumerate}
\def\labelenumi{\arabic{enumi}.}
\setcounter{enumi}{19}
\item
  Yes or no---Can classical physics be used to accurately describe a
  satellite moving at a speed of
  \includegraphics[width=0.73125in,height=0.21319in]{media/image8.png}?
  Explain why or why not.
\end{enumerate}

\begin{enumerate}
\def\labelenumi{\Alph{enumi}.}
\item
  No, because the satellite is moving at a speed much slower than the
  speed of the light and is not in a strong gravitational field
\item
  No, because the satellite is moving at a speed much slower than the
  speed of the light and is in a strong gravitational field
\item
  Yes, because the satellite is moving at a speed much slower than the
  speed of the light and it is not in a strong gravitational field
\item
  Yes, because the satellite is moving at a speed much slower than the
  speed of the light and is in a strong gravitational field
\end{enumerate}

\textbf{Solution} The correct answer is (C). Yes, classical physics
could describe the motion of a satellite, because it is moving at a
speed much slower than the speed of light, because it is not in a strong
gravitational field, and because it involves an object that is visible
to the naked eye.

\begin{enumerate}
\def\labelenumi{\arabic{enumi}.}
\setcounter{enumi}{20}
\item
  What would be some ways in which physics was involved in building the
  features of the room you are in right now?
\end{enumerate}

\begin{enumerate}
\def\labelenumi{\Alph{enumi}.}
\item
  Physics is involved in the structural strength and dimensions of the
  room.
\item
  Physics is involved in the air composition inside the room.
\item
  Physics is involved in the desk arrangement inside the room.
\item
  Physics is involved in the behavior of living beings inside the room.
\end{enumerate}

\textbf{Solution} The correct answer is (A). The physics of gravity,
weight, and forces was used to design the walls, ceilings, and floors in
the room so they could hold up people and furniture. Knowledge of
physics involving electricity went into developing lighting, fans,
electrical outlets, and other electrical devices in the room.

\begin{enumerate}
\def\labelenumi{\arabic{enumi}.}
\setcounter{enumi}{21}
\item
  What theory of modern physics describes the interrelationships between
  space, time, speed, and gravity?
\end{enumerate}

\begin{enumerate}
\def\labelenumi{\Alph{enumi}.}
\item
  Atomic theory
\item
  Nuclear physics
\item
  Quantum mechanics
\item
  General relativity
\end{enumerate}

\textbf{Solution} The correct answer is (D). Einstein came up with
general relativity to explain how gravity is a result of bending of
space and time.

\begin{enumerate}
\def\labelenumi{\arabic{enumi}.}
\setcounter{enumi}{22}
\item
  According to Einstein's theory of relativity, how could you
  effectively travel many years into Earth's future, but not age very
  much yourself?
\end{enumerate}

\begin{enumerate}
\def\labelenumi{\Alph{enumi}.}
\item
  By traveling at a speed equal to the speed of light
\item
  By traveling at a speed faster than the speed of light
\item
  By traveling at a speed much slower than the speed of light
\item
  By traveling at a speed slightly slower than the speed of light
\end{enumerate}

\textbf{Solution} The correct answer is (D). You could do this by
traveling away from Earth at near the speed of light, because measured
time intervals are shorter in the frame of reference of the spaceship,
in which you are at rest, than measured from Earth, where you are seen
as moving.

\paragraph{1.2 The Scientific Methods}\label{the-scientific-methods-1}

\begin{enumerate}
\def\labelenumi{\arabic{enumi}.}
\setcounter{enumi}{23}
\item
  You notice that the water level flowing in a stream near your house
  increases when it rains and the water turns brown. Which hypothesis
  best explains why the water turns brown? Assume you have all of the
  means to test the contents of the stream water.
\end{enumerate}

\begin{enumerate}
\def\labelenumi{\Alph{enumi}.}
\item
  The water in the stream turns brown because molecular forces between
  water molecules are stronger than mud molecules.
\item
  The water in the stream turns brown because of the breakage of a weak
  chemical bond with the hydrogen atom in the water molecule.
\item
  The water in the stream turns brown because it picks up dirt from the
  bank as the water level increases when it rains.
\item
  The water in the stream turns brown because the density of the water
  increases with increase in water level.
\end{enumerate}

\textbf{Solution} The correct answer is (C). The brown color of water
comes from the dirt that is picked up from the edges of the banks of the
water.

\begin{enumerate}
\def\labelenumi{\arabic{enumi}.}
\setcounter{enumi}{24}
\item
  Light travels as waves at an approximate speed of 300,000,000 m/s
  (186,000 mi/s). Designers of devices that use mirrors and lenses model
  the traveling light by straight lines, or light rays. Why would it be
  useful to model the light as rays of light instead of describing them
  accurately as electromagnetic waves?
\end{enumerate}

\begin{enumerate}
\def\labelenumi{\Alph{enumi}.}
\item
  A model can be constructed in such a way that the speed of light
  decreases.
\item
  Studying a model makes it easier to analyze the path that the light
  follows.
\item
  Studying a model will help us visualize why light travels at such
  great speed.
\item
  Modeling cannot be used to study traveling light because our eyes
  cannot track the motion of light.
\end{enumerate}

\textbf{Solution} The correct answer is (B). It simplifies describing
how light travels through space, and how lenses affect its path.

\begin{enumerate}
\def\labelenumi{\arabic{enumi}.}
\setcounter{enumi}{25}
\item
  A friend says that he doesn't trust scientific explanations because
  they are just theories, which are basically educated guesses. What
  could you say to convince him that scientific theories are different
  from the everyday use of the word theory?
\end{enumerate}

\begin{enumerate}
\def\labelenumi{\Alph{enumi}.}
\item
  A theory is a scientific explanation that has been repeatedly tested
  and supported by many experiments.
\item
  A theory is a hypothesis that has been tested and supported by some
  experiments.
\item
  A theory is a set of educated guesses, but at least one of the guesses
  remains true in each experiment.
\item
  A theory is a set of scientific explanations that has at least one
  experiment in support of it.
\end{enumerate}

\textbf{Solution} The correct answer is (A). You could tell your friend
that the word theory has a different meaning when used in science versus
everyday speech. In science, a theory is a scientific explanation that
has been repeatedly tested and supported by many experiments.

\begin{enumerate}
\def\labelenumi{\arabic{enumi}.}
\setcounter{enumi}{26}
\item
  Which hypothesis cannot be tested experimentally?
\end{enumerate}

\begin{enumerate}
\def\labelenumi{\Alph{enumi}.}
\item
  The structure of any part of the broccoli is similar to the whole
  structure of the broccoli.
\item
  Ghosts are the souls of people who have died.
\item
  The average speed of air molecules increases with temperature.
\item
  A vegetarian is less likely to be affected by night blindness.
\end{enumerate}

\textbf{Solution} The correct answer is (B). Ghosts represent the souls
of people who have died. This is a hypothesis that involves testing the
metaphysical.

\begin{enumerate}
\def\labelenumi{\arabic{enumi}.}
\setcounter{enumi}{27}
\item
  Would it be possible to scientifically prove that a supreme being
  exists or not? Briefly explain your answer.
\end{enumerate}

\begin{enumerate}
\def\labelenumi{\Alph{enumi}.}
\item
  It can be proved scientifically because it is a testable hypothesis.
\item
  It cannot be proved scientifically because it is not a testable
  hypothesis.
\item
  It can be proved scientifically because it is not a testable
  hypothesis.
\item
  It cannot be proved scientifically because it is a testable
  hypothesis.
\end{enumerate}

\textbf{Solution} The correct answer is (B). It would not be possible to
test for or against the existence of a supreme being because there is no
way to generate a testable hypothesis along these lines. This is because
data could never be collected, as no one knows what characteristics we
could look for or measure that would indicate the presence of such a
being.

\paragraph{\texorpdfstring{1.3 The Language of Physics: Physical
Quantities and Units
}{1.3 The Language of Physics: Physical Quantities and Units }}\label{the-language-of-physics-physical-quantities-and-units-1}

\begin{enumerate}
\def\labelenumi{\arabic{enumi}.}
\setcounter{enumi}{28}
\item
  A marathon runner completes a 42.188-km course in 2 h, 30 min, and 12
  s. There is an uncertainty of 25 m in the distance traveled and an
  uncertainty of 1 s in the elapsed time.
\end{enumerate}

\begin{enumerate}
\def\labelenumi{\arabic{enumi}.}
\item
  Calculate the percent uncertainty in the distance.
\item
  Calculate the uncertainty in the elapsed time.
\item
  What is the average speed in meters per second?
\item
  What is the uncertainty in the average speed?
\end{enumerate}

\begin{enumerate}
\def\labelenumi{\Alph{enumi}.}
\item
  0.059\%, 0.01\%, 0.468 m/s, 0.0003 m/s
\item
  0.059\%, 0.01\%, 0.468 m/s, 0.07 m/s
\item
  0.59\%, 8.33\%, 4.681 m/s, 0.003 m/s
\item
  0.059\%, 0.01\%, 4.681 m/s, 0.003 m/s
\end{enumerate}

\textbf{Solution} The correct answer is (D).

\includegraphics[width=3.68542in,height=2.37014in]{media/image9.png}

\begin{enumerate}
\def\labelenumi{\arabic{enumi}.}
\setcounter{enumi}{29}
\item
  A car engine moves a piston with a circular cross section of
  \includegraphics[width=1.19444in,height=0.19444in]{media/image10.png}
  diameter a distance of
  \includegraphics[width=1.19444in,height=0.19444in]{media/image11.png}
  to compress the gas in the cylinder. By what amount did the gas
  decrease in volume in cubic centimeters? Find the uncertainty in this
  volume.
\end{enumerate}

\begin{enumerate}
\def\labelenumi{\Alph{enumi}.}
\item
  \includegraphics[width=1.25in,height=0.22222in]{media/image12.png}
\item
  \includegraphics[width=1.25in,height=0.22222in]{media/image13.png}
\item
  \includegraphics[width=1.25in,height=0.22222in]{media/image14.png}
\item
  \includegraphics[width=1.07431in,height=0.22222in]{media/image15.png}
\end{enumerate}

\textbf{Solution} The correct answer is (D).

\includegraphics[width=5.66667in,height=0.73125in]{media/image16.png}

\begin{enumerate}
\def\labelenumi{\arabic{enumi}.}
\setcounter{enumi}{30}
\item
  What would be the slope for a line passing through these two points?
\end{enumerate}

\begin{quote}
Point 1: (1, 0.1)

Point 2: (7, 26.8)
\end{quote}

\begin{enumerate}
\def\labelenumi{\Alph{enumi}.}
\item
  2.4
\item
  4.5
\item
  6.2
\item
  6.8
\end{enumerate}

\textbf{Solution} The correct answer is (B).

\includegraphics[width=1.94444in,height=0.43542in]{media/image17.png}

\begin{enumerate}
\def\labelenumi{\arabic{enumi}.}
\setcounter{enumi}{31}
\item
  The sides of a small rectangular box are measured 1.80~cm and 2.05~cm
  long and 3.1~cm high. What is its volume and uncertainty in cubic
  centimeters? Assume the measuring device is accurate to
  \includegraphics[width=0.75in,height=0.19444in]{media/image18.png}
\end{enumerate}

\begin{enumerate}
\def\labelenumi{\Alph{enumi}.}
\item
  \includegraphics[width=0.99097in,height=0.22222in]{media/image19.png}
\item
  \includegraphics[width=0.99097in,height=0.22222in]{media/image20.png}
\item
  \includegraphics[width=0.99097in,height=0.22222in]{media/image21.png}
\item
  \includegraphics[width=1.07431in,height=0.22222in]{media/image22.png}
\end{enumerate}

\textbf{Solution} The correct answer is (C). The volume and uncertainty
in volume is calculated as follows:

\includegraphics[width=4.05556in,height=1.29653in]{media/image23.png}

\begin{enumerate}
\def\labelenumi{\arabic{enumi}.}
\setcounter{enumi}{32}
\item
  What is the approximate number of atoms in a bacterium? Assume that
  the average mass of an atom in the bacterium is 10 times the mass of a
  hydrogen atom. (Hint---The mass of a hydrogen atom is approximately
  \includegraphics[width=0.58333in,height=0.26875in]{media/image24.png}
  and the mass of a bacterium is approximately
  \includegraphics[width=0.66667in,height=0.26875in]{media/image25.png}
\end{enumerate}

\begin{enumerate}
\def\labelenumi{\Alph{enumi}.}
\item
  \includegraphics[width=0.30556in,height=0.22222in]{media/image26.png}
  atoms
\item
  \includegraphics[width=0.30556in,height=0.22222in]{media/image27.png}
  atoms
\item
  \includegraphics[width=0.30556in,height=0.22222in]{media/image28.png}
  atoms
\item
  \includegraphics[width=0.30556in,height=0.22222in]{media/image29.png}
  atoms
\end{enumerate}

\textbf{Solution} The correct answer is (B). Assume the ratio of atoms
to mass is the same in both hydrogen and bacterium. Therefore,

\includegraphics[width=2.52778in,height=1.49097in]{media/image30.png}

\subsubsection{\texorpdfstring{Problems }{Problems }}\label{problems}

\paragraph{1.3 The Language of Physics: Physical Quantities and
Units}\label{the-language-of-physics-physical-quantities-and-units-2}

\begin{enumerate}
\def\labelenumi{\arabic{enumi}.}
\setcounter{enumi}{33}
\item
  A commemorative coin that sells for \$40 is advertised to be plated
  with 15 mg gold. Suppose gold is worth about \$1,300 per ounce. Which
  option best represents the value of the gold in the coin?
\end{enumerate}

\begin{enumerate}
\def\labelenumi{\Alph{enumi}.}
\item
  \$0.33
\item
  \$0.69
\item
  \$3.30
\item
  \$6.90
\end{enumerate}

\textbf{Solution} The correct answer is (B).

\includegraphics[width=2.27778in,height=0.46319in]{media/image31.png}

\begin{enumerate}
\def\labelenumi{\arabic{enumi}.}
\setcounter{enumi}{34}
\item
  If a marathon runner ran 9.5 mi in one direction, 8.89 mi in another
  direction, and 2.333 mi in a third direction, how much distance did
  the runner run? Be sure to report your answer using the proper number
  of significant figures.
\end{enumerate}

\begin{enumerate}
\def\labelenumi{\Alph{enumi}.}
\item
  20 mi
\item
  20.7 mi
\item
  20.72 mi
\item
  20.732 mi
\end{enumerate}

\textbf{Solution} The correct answer is (B).

\includegraphics[width=1.86111in,height=0.19444in]{media/image32.png}

We can only report two significant figures because, when adding, the
answer should have the same number of places as the least-precise
starting value, which in this case is the tenths place in 9.5.
Therefore, the final answer is 20.7 mi.

\begin{enumerate}
\def\labelenumi{\arabic{enumi}.}
\setcounter{enumi}{35}
\item
  The speed limit on some interstate highways is roughly 80 km/h. What
  is this in meters per second? How many miles per hour is this?
\end{enumerate}

\begin{enumerate}
\def\labelenumi{\Alph{enumi}.}
\item
  62 m/s, 27.8 mph
\item
  22.2 m/s, 49.7 mph
\item
  62 m/s, 2.78 mph
\item
  2.78 m/s, 62 mph
\end{enumerate}

\textbf{Solution} The correct answer is (B).

\includegraphics[width=2.63889in,height=0.86111in]{media/image33.png}

\begin{enumerate}
\def\labelenumi{\arabic{enumi}.}
\setcounter{enumi}{36}
\item
  The length and width of a rectangular room are measured to be
  \includegraphics[width=1.19444in,height=0.19444in]{media/image34.png}
  by \includegraphics[width=1.25in,height=0.19444in]{media/image35.png}
  What is the area of the room and its uncertainty in square meters?
\end{enumerate}

\begin{enumerate}
\def\labelenumi{\Alph{enumi}.}
\item
  \includegraphics[width=1.08333in,height=0.22222in]{media/image36.png}
\item
  \includegraphics[width=1.08333in,height=0.22222in]{media/image37.png}
\item
  \includegraphics[width=1.08333in,height=0.22222in]{media/image38.png}
\item
  \includegraphics[width=1.08333in,height=0.22222in]{media/image39.png}
\end{enumerate}

\textbf{Solution} The correct answer is (D).

\includegraphics[width=2.77778in,height=0.75in]{media/image40.png}

\subsubsection{\texorpdfstring{Performance Task
}{Performance Task }}\label{performance-task}

\paragraph{1.3 The Language of Physics: Physical Quantities and
Units}\label{the-language-of-physics-physical-quantities-and-units-3}

\begin{enumerate}
\def\labelenumi{\arabic{enumi}.}
\setcounter{enumi}{37}
\item
  Part A. Create a new system of units to describe something that
  interests you. Your unit should be described using at least two
  subunits. For example, you can decide to measure the quality of songs
  using a new unit called \emph{song awesomeness}. Song awesomeness is
  measured by the number of songs downloaded and the number of times the
  song was used in movies.
\end{enumerate}

\begin{quote}
Part B. Create an equation that shows how to calculate your unit. Then,
using your equation, create a sample data set that you could graph. Are
your two subunits related linearly, quadratically, or inversely?
\end{quote}

\textbf{Solution} Answers will vary. Sample answer:

Part A. Someone could create a system where 1 week is 10 days, like the
French decimal time that was proposed.

Part B. If we assume one day is the same in both the French system and
our current system of time, we can create a ratio of 1 standard week/1
French week = 7~standard days/10 French days or
\includegraphics[width=0.62986in,height=0.21319in]{media/image41.png}
where \emph{x} is the number of French weeks and \emph{y} is the number
of standard weeks. These two different measurements of weeks are
linearly related.

\subsection{Test Prep}\label{test-prep}

\subsubsection{Multiple Choice}\label{multiple-choice}

\paragraph{1.1 Physics: Definitions and
Applications}\label{physics-definitions-and-applications-2}

\begin{enumerate}
\def\labelenumi{\arabic{enumi}.}
\setcounter{enumi}{38}
\item
  Modern physics could best be described as the combination of which
  theories?
\end{enumerate}

\begin{enumerate}
\def\labelenumi{\Alph{enumi}.}
\item
  Quantum mechanics and Einstein's theory of relativity
\item
  Quantum mechanics and classical physics
\item
  Newton's laws of motion and classical physics
\item
  Newton's laws of motion and Einstein's theory of relativity
\end{enumerate}

\textbf{Solution} The correct answer is (A). Classical physics and
Newton's laws of motion were studied much earlier and are not considered
in the modern physics field.

\begin{enumerate}
\def\labelenumi{\arabic{enumi}.}
\setcounter{enumi}{39}
\item
  Which topic could be studied accurately using classical physics?
\end{enumerate}

\begin{enumerate}
\def\labelenumi{\Alph{enumi}.}
\item
  The strength of gravity within a black hole
\item
  The motion of a plane through the sky
\item
  The collisions of subatomic particles
\item
  The effect of gravity on the passage of time
\end{enumerate}

\textbf{Solution} The correct answer is (B). Studying gravity requires
general relativity and subatomic particles requires quantum mechanics.

\begin{enumerate}
\def\labelenumi{\arabic{enumi}.}
\setcounter{enumi}{40}
\item
  Which statement best describes why knowledge of physics is necessary
  to understand all other sciences?
\end{enumerate}

\begin{enumerate}
\def\labelenumi{\Alph{enumi}.}
\item
  Physics explains how energy passes from one object to another.
\item
  Physics explains how gravity works.
\item
  Physics explains the motion of objects that can be seen with the naked
  eye.
\item
  Physics explains the fundamental aspects of the universe.
\end{enumerate}

\textbf{Solution} The correct answer is (D). Physics explains the
fundamental aspects of the universe that are the basis for other
sciences, such as chemistry and biology.

\begin{enumerate}
\def\labelenumi{\arabic{enumi}.}
\setcounter{enumi}{41}
\item
  What does radiation therapy, used to treat cancer patients, have to do
  with physics?
\end{enumerate}

\begin{enumerate}
\def\labelenumi{\Alph{enumi}.}
\item
  Understanding how cells reproduce is mainly about physics.
\item
  Predictions of the side effects from the radiation therapy are based
  on physics.
\item
  The devices used for generating some kinds of radiation are based on
  principles of physics.
\item
  Predictions of the life expectancy of patients receiving radiation
  therapy are based on physics.
\end{enumerate}

\textbf{Solution} The correct answer is (C). Physics provides an
understanding of radiation works, and how to construct a device that
emits it.

\paragraph{1.2 The Scientific Methods}\label{the-scientific-methods-2}

\begin{enumerate}
\def\labelenumi{\arabic{enumi}.}
\setcounter{enumi}{42}
\item
  The free-electron model of metals explains some of the important
  behaviors of metals by assuming the metal's electrons move freely
  through the metal without repelling one another. In what sense is the
  free-electron theory based on a model?
\end{enumerate}

\begin{enumerate}
\def\labelenumi{\Alph{enumi}.}
\item
  Its use requires constructing replicas of the metal wire in the lab.
\item
  It involves analyzing an imaginary system simpler than the real wire
  it resembles.
\item
  It examines a model, or ideal, behavior that other metals should
  imitate.
\item
  It attempts to examine the metal in a very realistic, or model, way.
\end{enumerate}

\textbf{Solution} The correct answer is (B). It involves analyzing an
imaginary system simpler than the real wire it resembles. Creating a
model like this helps us get predicable results without having to do
complicated calculations.

\begin{enumerate}
\def\labelenumi{\arabic{enumi}.}
\setcounter{enumi}{43}
\item
  A scientist wishes to study the motion of about 1,000 molecules of gas
  in a container by modeling them as tiny billiard balls bouncing
  randomly off one another. What is needed to calculate and store data
  on the balls' detailed motion?
\end{enumerate}

\begin{enumerate}
\def\labelenumi{\Alph{enumi}.}
\item
  A group of hypotheses that cannot be practically tested in real life
\item
  A computer that can store and perform calculations on large data sets
\item
  A large amount of experimental results on the molecules and their
  motion
\item
  A collection of hypotheses that have not yet been tested regarding the
  molecules
\end{enumerate}

\textbf{Solution} The correct answer is (B). Without a computer, it
would be impossible to track each ball's motion and interactions
associated with it.

\begin{enumerate}
\def\labelenumi{\arabic{enumi}.}
\setcounter{enumi}{44}
\item
  When a large body of experimental evidence supports a hypothesis, what
  may the hypothesis eventually be considered?
\end{enumerate}

\begin{enumerate}
\def\labelenumi{\Alph{enumi}.}
\item
  Observation
\item
  Insight
\item
  Conclusion
\item
  Law
\end{enumerate}

\textbf{Solution} The correct answer is (D). A scientific law is
supported by a large body of experimental evidence.

\begin{enumerate}
\def\labelenumi{\arabic{enumi}.}
\setcounter{enumi}{45}
\item
  While watching some ants outside of your home, you notice that the
  worker ants gather in a specific area on your lawn. Which statement is
  a testable hypothesis that attempts to explain why the ants gather in
  that specific area on the lawn?
\end{enumerate}

\begin{enumerate}
\def\labelenumi{\Alph{enumi}.}
\item
  The worker ants thought it was a nice location.
\item
  The worker ants may have to find a spot for the queen to lay eggs.
\item
  There may be some food particles lying there.
\item
  The worker ants are supposed to group together at a place.
\end{enumerate}

\textbf{Solution} The correct answer is (C). The ants gather in that
spot of the yard because there is a large supply of food particles on
the ground. This hypothesis is testable.

\paragraph{1.3 The Language of Physics: Physical Quantities and
Units}\label{the-language-of-physics-physical-quantities-and-units-4}

\begin{enumerate}
\def\labelenumi{\arabic{enumi}.}
\setcounter{enumi}{46}
\item
  Which option describes a length that is
  \includegraphics[width=0.65764in,height=0.22222in]{media/image42.png}
  of a meter?
\end{enumerate}

\begin{enumerate}
\def\labelenumi{\Alph{enumi}.}
\item
  2.0 kilometers
\item
  2.0 megameters
\item
  2.0 millimeters
\item
  2.0 micrometers
\end{enumerate}

\textbf{Solution} The correct answer is (C). The prefix notation for
\includegraphics[width=0.33333in,height=0.25in]{media/image43.png} is
\emph{milli--}.

\begin{enumerate}
\def\labelenumi{\arabic{enumi}.}
\setcounter{enumi}{47}
\item
  Suppose that a bathroom scale reads a person's mass as 65 kg with a
  3~percent uncertainty. What is the uncertainty in that person's mass
  in kilograms?
\end{enumerate}

\begin{enumerate}
\def\labelenumi{\Alph{enumi}.}
\item
  2 kg
\item
  98 kg
\item
  5 kg
\item
  0 kg
\end{enumerate}

\textbf{Solution} The correct answer is (A). The scale has an
uncertainty of 2 kg. If you multiply 65~kg by 3 percent, the result is 2
kg.

\begin{enumerate}
\def\labelenumi{\arabic{enumi}.}
\setcounter{enumi}{48}
\item
  Which option best describes a variable?
\end{enumerate}

\begin{enumerate}
\def\labelenumi{\Alph{enumi}.}
\item
  A trend that shows an exponential relationship
\item
  Something whose value can change over multiple measurements
\item
  A measure of how much a plot line changes along the \emph{y}-axis
\item
  Something that remains constant over multiple measurements
\end{enumerate}

\textbf{Solution} The correct answer is (B). Variables depend on the
interactions and trends that result from them.

\begin{enumerate}
\def\labelenumi{\arabic{enumi}.}
\setcounter{enumi}{49}
\item
  A high school track coach has just purchased a new stopwatch that has
  an uncertainty of
  \includegraphics[width=0.59236in,height=0.19444in]{media/image44.png}.
  Runners on the team regularly complete 100-m sprints in 12.49~s to
  15.01~s. At the school's last track meet, the first-place sprinter
  finished in 12.04~s and the second-place sprinter finished in 12.07~s.
\end{enumerate}

\begin{quote}
Yes or no---Will the coach's new stopwatch be helpful in timing the
sprint team? Why or why not?
\end{quote}

\begin{enumerate}
\def\labelenumi{\Alph{enumi}.}
\item
  No, the uncertainty in the stopwatch is too large to effectively
  differentiate between the sprint times.
\item
  No, the uncertainty in the stopwatch is too small to effectively
  differentiate between the sprint times.
\item
  Yes, the uncertainty in the stopwatch is too large to effectively
  differentiate between the sprint times.
\item
  Yes, the uncertainty in the stopwatch is too small to effectively
  differentiate between the sprint times.
\end{enumerate}

\textbf{Solution} The correct answer is (A). No, the uncertainty in the
stopwatch is too great to effectively differentiate between the sprint
times. The difference between the second set of times is 0.03 s, which
is greater than the uncertainty given, 0.05 s.

\subsubsection{\texorpdfstring{Short Answer
}{Short Answer }}\label{short-answer}

\paragraph{1.1 Physics: Definitions and
Applications}\label{physics-definitions-and-applications-3}

\begin{enumerate}
\def\labelenumi{\arabic{enumi}.}
\setcounter{enumi}{50}
\item
  What are the aims of physics?
\end{enumerate}

\begin{enumerate}
\def\labelenumi{\Alph{enumi}.}
\item
  Physics aims to explain the fundamental aspects of our universe and
  how these aspects interact with one another.
\item
  Physics aims to explain the biological aspects of our universe and how
  these aspects interact with one another.
\item
  Physics aims to explain the composition, structure, and changes in
  matter occurring in the universe.
\item
  Physics aims to explain the social behavior of living beings in the
  universe.
\end{enumerate}

\textbf{Solution} The correct answer is (A). Physics aims to describe
the fundamental aspects of our universe, namely energy, matter, space,
motion and time. It also describes how these aspects of our universe
interact with one another.

\begin{enumerate}
\def\labelenumi{\arabic{enumi}.}
\setcounter{enumi}{51}
\item
  What are the fields of magnetism and electricity? State how are they
  are related.
\end{enumerate}

\begin{enumerate}
\def\labelenumi{\Alph{enumi}.}
\item
  Magnetism describes the attractive force between a magnetized object
  and a metal like iron. Electricity involves the study of electric
  charges and their movements. Magnetism is not related to electricity.
\item
  Magnetism describes the attractive force between a magnetized object
  and a metal like iron. Electricity involves the study of electric
  charges and their movements. Magnetism is produced by a flow
  electrical charges.
\item
  Magnetism involves the study of electric charges and their movements.
  Electricity describes the attractive force between a magnetized object
  and a metal. Magnetism is not related to electricity.
\item
  Magnetism involves the study of electric charges and their movements.
  Electricity describes the attractive force between a magnetized object
  and a metal. Magnetism is produced by the flow electrical charges.
\end{enumerate}

\textbf{Solution} The correct answer is (B). Magnetism describes the
attractive force between a magnetized object and a metal like iron.
Electricity involves the study of electric charges and their movements.
One way in which magnetism is produced is by a flow of electrical
charges. Although these two areas, magnetism and electricity, seem
different, in fact they are related to each other as we will find out
later on.

\begin{enumerate}
\def\labelenumi{\arabic{enumi}.}
\setcounter{enumi}{52}
\item
  Which two topics are physicists trying to unify with relativistic
  quantum mechanics? How will this unification create a greater
  understanding of our universe?
\end{enumerate}

\begin{enumerate}
\def\labelenumi{\Alph{enumi}.}
\item
  Relativistic quantum mechanics unifies quantum mechanics with
  Einstein's theory of relativity. The unified theory creates a greater
  understanding of our universe because it can explain objects of all
  sizes and masses.
\item
  Relativistic quantum mechanics unifies classical mechanics with
  Einstein's theory of relativity. The unified theory creates a greater
  understanding of our universe because it can explain objects of all
  sizes and masses.
\item
  Relativistic quantum mechanics unifies quantum mechanics with
  Einstein's theory of relativity. The unified theory creates a greater
  understanding of our universe because it is unable to explain objects
  of all sizes and masses.
\item
  Relativistic quantum mechanics unifies classical mechanics with the
  Einstein's theory of relativity. The unified theory creates a greater
  understanding of our universe because it is unable to explain objects
  of all sizes and masses.
\end{enumerate}

\textbf{Solution} The correct answer is (A). Relativistic quantum
mechanics unifies quantum mechanics with Einstein's theory of
relativity. Unifying these theories would create a greater understanding
of our universe by explaining how large-scale processes, such as time,
space, and gravity, relate to very small-scale objects, such as the
subatomic particles studied in quantum mechanics.

\begin{enumerate}
\def\labelenumi{\arabic{enumi}.}
\setcounter{enumi}{53}
\item
  The findings of studies in quantum mechanics have been described as
  strange or weird compared to those of classical physics. Why is this
  so?
\end{enumerate}

\begin{enumerate}
\def\labelenumi{\Alph{enumi}.}
\item
  Because the phenomena it explains are outside the normal range of
  human experience which deals with much larger objects
\item
  Because the phenomena it explains can be perceived easily, namely,
  ordinary-sized objects
\item
  Because the phenomena it explains are outside the normal range of
  human experience, namely, very large and very fast objects
\item
  Because the phenomena it explains can be perceived easily, namely,
  very large and very fast objects
\end{enumerate}

\textbf{Solution} The correct answer is (A). Findings in quantum
mechanics may be described as weird or strange because they involve
phenomena that are outside the normal range of human experience, which
deals with much larger objects.

\begin{enumerate}
\def\labelenumi{\arabic{enumi}.}
\setcounter{enumi}{54}
\item
  How could knowledge of physics help you find a faster way to drive
  from your house to your school?
\end{enumerate}

\begin{enumerate}
\def\labelenumi{\Alph{enumi}.}
\item
  Physics can explain the traffic on a particular street and help us
  know about the traffic in advance.
\item
  Physics can explain the ongoing construction of roads on a particular
  street and help us know about delays in the traffic in advance.
\item
  Physics can explain distances and speed limits on a particular street,
  and help us calculate faster routes.
\item
  Physics can explain the closing of a particular street and help us
  calculate faster routes.
\end{enumerate}

\textbf{Solution} The correct answer is (C). By studying the distances
of certain streets, as well as their speed limits or the car's average
velocity, it would be possible to calculate how long certain routes take
and which are the shortest.

\begin{enumerate}
\def\labelenumi{\arabic{enumi}.}
\setcounter{enumi}{55}
\item
  How could knowledge of physics help you build a sound and
  energy-efficient house?
\end{enumerate}

\begin{enumerate}
\def\labelenumi{\Alph{enumi}.}
\item
  An understanding of force, pressure, heat, and electricity, which all
  involve physics, will help me design a sound and energy-efficient
  house.
\item
  An understanding of the air composition and chemical composition of
  matter, which involves physics, will help me design a sound and
  energy-efficient house.
\item
  An understanding of material cost and economic factors involving
  physics will help me design a sound and energy-efficient house.
\item
  An understanding of geographical location and social environment,
  which involves physics, will help me design a sound and
  energy-efficient house.
\end{enumerate}

\textbf{Solution} The correct answer is (A). An understanding of force,
weight, and pressure will help you design a structurally sound house,
while a knowledge of heat and electricity would help you build an
energy-efficient house. Although external considerations of location and
materials may be important to an individual, the concept of forces and
pressure make sure your house is stable under any stressful condition
while heat and electricity are important for energy efficiency.

\begin{enumerate}
\def\labelenumi{\arabic{enumi}.}
\setcounter{enumi}{56}
\item
  What aspects of physics would a chemist likely study in trying to
  discover a new chemical reaction?
\end{enumerate}

\begin{enumerate}
\def\labelenumi{\Alph{enumi}.}
\item
  Physics is involved in understanding whether the reactants and
  products dissolve in water.
\item
  Physics is involved in understanding the amount of energy released or
  required in a chemical reaction.
\item
  Physics is involved in what the products of the reaction will be.
\item
  Physics is involved in understanding the types of ions produced in a
  chemical reaction.
\end{enumerate}

\textbf{Solution} The correct answer is (B). A chemist would likely need
knowledge of the atoms involved in the chemical reaction, as well as the
amount of energy needed or given off by the reaction.

\paragraph{1.2 The Scientific Methods}\label{the-scientific-methods-3}

\begin{enumerate}
\def\labelenumi{\arabic{enumi}.}
\setcounter{enumi}{57}
\item
  You notice that it takes more force to get a large box to start
  sliding across the floor than it takes to get the box sliding faster
  once it is already moving. Which option is a testable hypothesis that
  attempts to explain this observation?
\end{enumerate}

\begin{enumerate}
\def\labelenumi{\Alph{enumi}.}
\item
  The floor has greater distortions of space-time for moving the sliding
  box faster than for the box at rest.
\item
  The floor has greater distortions of space-time for the box at rest
  than for the sliding box.
\item
  The resistance between the floor and the box is less when the box is
  sliding than when the box is at rest.
\item
  The floor dislikes having objects move across it and therefore holds
  the box rigidly in place until it cannot resist the force.
\end{enumerate}

\textbf{Solution} The correct answer is (C). The box is harder to move
from rest because there is greater resistance.

\begin{enumerate}
\def\labelenumi{\arabic{enumi}.}
\setcounter{enumi}{58}
\item
  Which experiment tests the hypothesis?
\end{enumerate}

\begin{quote}
Driving on a gravel road causes greater damage to a car than driving on
a dirt road.
\end{quote}

\begin{enumerate}
\def\labelenumi{\Alph{enumi}.}
\item
  To test the hypothesis, compare the damage to the car by driving it on
  a smooth road and a gravel road.
\item
  To test the hypothesis, compare the damage to the car by driving it on
  a smooth road and a dirt road.
\item
  To test the hypothesis, compare the damage to the car by driving it on
  a gravel road and the dirt road.
\item
  This hypothesis is not testable.
\end{enumerate}

\textbf{Solution} The correct answer is (C). To test the hypothesis,
compare the damage to the car by driving it on a gravel road and the
dirt road.

\begin{enumerate}
\def\labelenumi{\arabic{enumi}.}
\setcounter{enumi}{59}
\item
  How is a physical model, such as a spherical mass held in place by
  springs, used to represent an atom vibrating in a solid, similar to a
  computer-based model, such as one predicting how gravity affects the
  orbits of the planets?
\end{enumerate}

\begin{enumerate}
\def\labelenumi{\Alph{enumi}.}
\item
  Both a physical model and a computer-based model should be built
  around a hypothesis and could be able to test the hypothesis.
\item
  Both a physical model and a computer-based model should be built
  around a hypothesis, but they cannot be used to test the hypothesis.
\item
  Both a physical model and a computer-based model should be built
  around the\\
  results of scientific studies and could be used to make predictions
  about the system under study.
\item
  Both a physical model and a computer-based model should be built
  around the\\
  results of scientific studies but cannot be used to make predictions
  about the system under study.
\end{enumerate}

\textbf{Solution} The correct answer is (C). Both a physical model and a
computer-based model should be built around the results of scientific
studies, and could be used to make predictions about the system under
study. After forming this model, it should be tested under any possible
condition to make sure predictable results occur.

\begin{enumerate}
\def\labelenumi{\arabic{enumi}.}
\setcounter{enumi}{60}
\item
  What are the advantages and disadvantages of using a model to predict
  a life-or-death situation, such as whether an asteroid will strike
  Earth?
\end{enumerate}

\begin{enumerate}
\def\labelenumi{\Alph{enumi}.}
\item
  The advantage of using a model is that it provides predictions
  quickly, but the disadvantage of using a model is that it could make
  erroneous predictions.
\item
  The advantage of using a model is that it provides accurate
  predictions, but the disadvantage of using a model is that it takes a
  long time to make predictions.
\item
  The advantage of using a model is that it provides predictions quickly
  without any error. There are no disadvantages of using a scientific
  model.
\item
  The disadvantage of using models is that it takes longer to make
  predictions and the predictions are inaccurate. There are no
  advantages to using a scientific model.
\end{enumerate}

\textbf{Solution} The correct answer is (A). A model involves
simplification and approximation of the real system and can give
inaccurate results. The advantages of using a model for a life-or-death
situation is that it could provide predictions quickly, once it is
constructed, and would not involve sending people into dangerous
situations. The disadvantage, however, would be the fact that most
models cannot incorporate every variable in a system. Therefore, the
model may make erroneous predictions.

\begin{enumerate}
\def\labelenumi{\arabic{enumi}.}
\setcounter{enumi}{61}
\item
  A friend tells you that a scientific law cannot be changed. Is your
  friend correct? Briefly explain your answer.
\end{enumerate}

\begin{enumerate}
\def\labelenumi{\Alph{enumi}.}
\item
  Correct, because laws are theories that have been proven true
\item
  Correct, because theories are laws that have been proven true
\item
  Incorrect, because a law is changed if new evidence contradicts it
\item
  Incorrect, because a law is changed when a theory contradicts it
\end{enumerate}

\textbf{Solution} The correct answer is (C). The friend is incorrect
because scientific laws and theories are continually tested, especially
in light of new scientific discoveries. Therefore, even a long-supported
scientific law can be proven incorrect by new discoveries.

\begin{enumerate}
\def\labelenumi{\arabic{enumi}.}
\setcounter{enumi}{62}
\item
  How does a scientific law compare to a local law, such as that
  governing parking at your school, in terms of whether laws can be
  changed, and how universal a law is?
\end{enumerate}

\begin{enumerate}
\def\labelenumi{\Alph{enumi}.}
\item
  A local law applies only in a specific area, but a scientific law is
  applicable throughout the universe. Both the local law and the
  scientific law can change.
\item
  A local law applies only in a specific area, but a scientific law is
  applicable throughout the universe. A local law can change, but a
  scientific law cannot be changed.
\item
  A local law applies throughout the universe but a scientific law is
  applicable only in a specific area. Both the local and the scientific
  law can change.
\item
  A local law applies throughout the universe, but a scientific law is
  applicable only in a specific area. A local law can change, but a
  scientific law cannot be changed.
\end{enumerate}

\textbf{Solution} The correct answer is (A). Both scientific and local
laws can be changed, but they differ in their universality. A local law
applies only in a specific area, such as in a city or a single school.
Scientific laws apply throughout the universe.

\begin{enumerate}
\def\labelenumi{\arabic{enumi}.}
\setcounter{enumi}{63}
\item
  Can the validity of a model be limited or must it be universally
  valid? How does this compare to the required validity of a theory or a
  law?
\end{enumerate}

\begin{enumerate}
\def\labelenumi{\Alph{enumi}.}
\item
  Models, theories, and laws must be universally valid.
\item
  Models, theories, and laws have only limited validity.
\item
  Models have limited validity, while theories and laws are universally
  valid.
\item
  Models and theories have limited validity, while laws are universally
  valid.
\end{enumerate}

\textbf{Solution} The correct answer is (D). Models often have limited
validity because it is very hard to fully model a complicated system.
However, models can still be very useful despite not being 100 percent
valid. Theories also do not have to be universally valid and can explain
either very broad or very narrow observations. Laws, however, should be
universally valid.

\paragraph{1.3 The Language of Physics: Physical Quantities and
Units}\label{the-language-of-physics-physical-quantities-and-units-5}

\begin{enumerate}
\def\labelenumi{\arabic{enumi}.}
\setcounter{enumi}{64}
\item
  The speed of sound is measured at 342 m/s on a certain day. What is
  this in kilometers per hour? Report your answer in scientific
  notation.
\end{enumerate}

\begin{enumerate}
\def\labelenumi{\Alph{enumi}.}
\item
  \includegraphics[width=1.11111in,height=0.25in]{media/image45.png}
\item
  \includegraphics[width=1.08333in,height=0.25in]{media/image46.png}
\item
  \includegraphics[width=1in,height=0.25in]{media/image47.png}
\item
  \includegraphics[width=1.14792in,height=0.25in]{media/image48.png}
\end{enumerate}

\textbf{Solution} The correct answer is (B).

\includegraphics[width=3.13889in,height=0.44444in]{media/image49.png}

\begin{enumerate}
\def\labelenumi{\arabic{enumi}.}
\setcounter{enumi}{65}
\item
  What is the main difference between the metric system and the U.S.
  customary system?
\end{enumerate}

\begin{enumerate}
\def\labelenumi{\Alph{enumi}.}
\item
  In the metric system, unit changes are based on powers of 10, while in
  the U.S. customary system, each unit conversion has unrelated
  conversion factors.
\item
  In the metric system, each unit conversion has unrelated conversion
  factors, while in the U.S. customary system, unit changes are based on
  powers of 10.
\item
  In the metric system, unit changes are based on powers of 2, while in
  the U.S. customary system, each unit conversion has unrelated
  conversion factors.
\item
  In the metric system, each unit conversion has unrelated conversion
  factors, while in the U.S. customary system, unit changes are based on
  powers of 2.
\end{enumerate}

\textbf{Solution} The correct answer is (A). The metric system and the
U.S. customary system differ in how the units appropriate for different
size ranges convert from one to another. In the metric system, the units
in different size ranges are related by powers of 10 (for example
\includegraphics[width=1.07431in,height=0.21319in]{media/image50.png}).
In the customary system, the units change by unique amounts for each
range and each kind of measurement, such as how there are 8 ounces in a
cup but then 2 cups in a pint.

\begin{enumerate}
\def\labelenumi{\arabic{enumi}.}
\setcounter{enumi}{66}
\item
  An infant's pulse rate is measured to be
  \includegraphics[width=1.28681in,height=0.22222in]{media/image51.png}
  What is the percent uncertainty in this measurement?
\end{enumerate}

\begin{enumerate}
\def\labelenumi{\Alph{enumi}.}
\item
  2\%
\item
  3\%
\item
  4\%
\item
  5\%
\end{enumerate}

\textbf{Solution} The correct answer is (C).

\includegraphics[width=3.53681in,height=0.44444in]{media/image52.png}

\begin{enumerate}
\def\labelenumi{\arabic{enumi}.}
\setcounter{enumi}{67}
\item
  How does the uncertainty of a measurement relate to the accuracy and
  precision of the measuring device? Include the definitions of accuracy
  and precision in your answer.
\end{enumerate}

\begin{enumerate}
\def\labelenumi{\Alph{enumi}.}
\item
  A decrease in the precision of a measurement increases the uncertainty
  of the measurement, while a decrease in accuracy does not.
\item
  A decrease in either the precision or accuracy of a measurement
  increases the uncertainty of the measurement.
\item
  An increase in either the precision or accuracy of a measurement will
  increase the uncertainty of that measurement.
\item
  An increase in the accuracy of a measurement will increase the
  uncertainty of that measurement, while an increase in precision will
  not.
\end{enumerate}

\textbf{Solution} The correct answer is (B). Decreases in either the
precision or accuracy of a measurement will increase the uncertainty of
that measurement. In the case of precision, if the measuring device
cannot reliably give the same measurement, it will increase the
uncertainty of the measurement. Likewise, if the measuring device is not
accurate, it will not produce measurements that represent reality, also
increasing the uncertainty of the measurement.

\begin{enumerate}
\def\labelenumi{\arabic{enumi}.}
\setcounter{enumi}{68}
\item
  What are all of the characteristics that can be determined about a
  straight line with a slope of
  \includegraphics[width=0.22222in,height=0.19444in]{media/image53.png}
  and a \emph{y}-intercept of 50 on a graph?
\end{enumerate}

\begin{enumerate}
\def\labelenumi{\Alph{enumi}.}
\item
  Based on the information, the line has a negative slope. Because its
  \emph{y}-intercept is 50 and its slope is negative, this line
  gradually rises on the graph as the \emph{x}-value increases.
\item
  Based on the information, the line has a negative slope. Because its
  \emph{y}-intercept is 50 and its slope is negative, this line
  gradually moves downward on the graph as the \emph{x}-value increases.
\item
  Based on the information, the line has a positive slope. Because its
  \emph{y}-intercept is 50 and its slope is positive, this line
  gradually rises on the graph as the \emph{x}-value increases.
\item
  Based on the information, the line has a positive slope. Because its
  \emph{y}-intercept is 50 and its slope is positive, this line
  gradually moves downward on the graph as the \emph{x}-value increases.
\end{enumerate}

\textbf{Solution} The correct answer is (B). Based on this information,
one could tell that the line drops, as it has a negative slope. In
addition, because the \emph{y}-intercept is 50 and the slope is
\includegraphics[width=0.26875in,height=0.20347in]{media/image54.png}
this line makes a very gradual drop as the \emph{x}-values increase.

\begin{enumerate}
\def\labelenumi{\arabic{enumi}.}
\setcounter{enumi}{69}
\item
  The graph shows the temperature change over time of a heated cup of
  water.
\end{enumerate}

\begin{quote}
\includegraphics[width=3.14075in,height=3.03111in,alt={A graph titled Temperature Change of Water Over Time. The x-axis plots time in minutes, from 0 to 10, in increments of 1. The y-axis plots temperature in degrees Celsius, from 0 to 100, in increments of 10. A line begins at 0 on the x-axis and 20 on the y-axis, slopes steeply upward to six on the x-axis and 100 on the y-axis, then continues horizontally to 10 on the x-axis.}]{media/image55.png}

What is the slope of the graph between the time period 2 min and 5 min?
\end{quote}

\begin{enumerate}
\def\labelenumi{\Alph{enumi}.}
\item
  \includegraphics[width=0.82431in,height=0.21319in]{media/image56.png}
\item
  \includegraphics[width=0.94444in,height=0.21319in]{media/image57.png}
\item
  \includegraphics[width=0.86111in,height=0.21319in]{media/image58.png}
\item
  \includegraphics[width=0.72222in,height=0.21319in]{media/image59.png}
\end{enumerate}

\textbf{Solution} The correct answer is (D). The slope of the line
between 2 min and 5 min is
\includegraphics[width=0.73125in,height=0.22222in]{media/image60.png}

\includegraphics[width=2.77778in,height=0.43542in]{media/image61.png}

\subsubsection{\texorpdfstring{Extended Response
}{Extended Response }}\label{extended-response}

\paragraph{1.2 The Scientific Methods}\label{the-scientific-methods-4}

\begin{enumerate}
\def\labelenumi{\arabic{enumi}.}
\setcounter{enumi}{70}
\item
  You wish to perform an experiment on the stopping distance of your new
  car. Which specific experiment measures the distance? Be sure to
  specifically state how you will set up and take data during your
  experiment.
\end{enumerate}

\begin{enumerate}
\def\labelenumi{\Alph{enumi}.}
\item
  Drive the car at exactly 50 mph and then press harder on the
  accelerator pedal until the velocity reaches 60 mph and record the
  distance this takes.
\item
  Drive the car at exactly 50 mph and then apply the brakes until it
  stops and record the distance this takes.
\item
  Drive the car at exactly 50 mph and then apply the brakes until it
  stops and record the time it takes.
\item
  Drive the car at exactly 50 mph and then apply the accelerator until
  it reaches 60~mph and record the time it takes.
\end{enumerate}

\textbf{Solution} The correct answer is (B). Drive the car at exactly 50
mph and then apply the brakes until it stops and record the distance
this takes. For example:

Hypothesis: I hypothesize that the car will come to a complete stop in 3
seconds when going 50 mph.

Experiment: Drive the car exactly at 30 mph and then time from when the
brake is depressed to when you feel the lurching back of the car that
indicates it has stopped.

\begin{enumerate}
\def\labelenumi{\arabic{enumi}.}
\setcounter{enumi}{71}
\item
  You wish to make a model showing how traffic flows around your city or
  local area. What are some hypotheses that your model could test, and
  what are the model's limitations in terms of what could NOT be tested?
\end{enumerate}

\begin{enumerate}
\def\labelenumi{\Alph{enumi}.}
\item
  1. Testable hypotheses: The gravitational pull on each vehicle while
  in motion and the average speed of vehicles is 40 mph\\
  2. Nontestable hypotheses: The average number of vehicles passing is
  935 per day and carbon emission from each moving vehicle
\item
  1. Testable hypotheses: The average number of vehicles passing is 935
  per day and the average speed of vehicles is 40 mph\\
  2. Nontestable hypotheses: The gravitational pull on each vehicle
  while in motion and the carbon emission from each of the moving
  vehicle
\item
  1. Testable hypotheses: The average number of vehicles passing is 935
  per day and the carbon emission from each of the moving vehicle\\
  2. Nontestable hypotheses: The gravitational pull on each vehicle
  while in motion and the average speed of the vehicles is 40 mph
\item
  1. Testable hypothesis: The average number of vehicles passing is 935
  per day and the gravitational pull on each vehicle while in motion\\
  2. Nontestable hypothesis: The average speed of vehicles is 40 mph and
  the carbon emission from each moving vehicle
\end{enumerate}

\textbf{Solution} The correct answer is (B). The model could be
constructed by first taking data on the major streets by counting how
many cars pass within a certain timeframe. If only one timeframe is used
in constructing the data (i.e., 5--7 p.m.), and then you could use the
model to predict which route would be the quickest to travel to reach
home. However, the model could not predict which route is the quickest
outside of the sampled timeframe.

\begin{enumerate}
\def\labelenumi{\arabic{enumi}.}
\setcounter{enumi}{72}
\item
  What would play the most important role in an experiment in the
  scientific world becoming a scientific law?
\end{enumerate}

\begin{enumerate}
\def\labelenumi{\Alph{enumi}.}
\item
  Further testing would need to show it is a universally followed rule.
\item
  The observation would have to be described in a published scientific
  article.
\item
  The experiment would have to be repeated once or twice.
\item
  The observer would need to be a well-known scientist whose authority
  was accepted.
\end{enumerate}

\textbf{Solution} The correct answer is (A). It would be necessary to
find sound scientific evidence, in the form of experiment and possibly
other observation---that it is a general rule followed universally in
nature.

\paragraph{1.3 The Language of Physics: Physical Quantities and
Units}\label{the-language-of-physics-physical-quantities-and-units-6}

\begin{enumerate}
\def\labelenumi{\arabic{enumi}.}
\setcounter{enumi}{73}
\item
  Tectonic plates are large segments of Earth's crust that move slowly.
  Suppose that one such plate has an average speed of 4.0 cm/year. What
  distance does it move in 1.0~s at this speed? What is its speed in
  kilometers per million years? Report your answers using scientific
  notation.
\end{enumerate}

\begin{enumerate}
\def\labelenumi{\Alph{enumi}.}
\item
  \includegraphics[width=2.67569in,height=0.26875in]{media/image62.png}
\item
  \includegraphics[width=2.67569in,height=0.26875in]{media/image63.png}
\item
  \includegraphics[width=2.77778in,height=0.26875in]{media/image64.png}
\item
  \includegraphics[width=2.77778in,height=0.26875in]{media/image65.png}
\end{enumerate}

\textbf{Solution} The correct answer is (A).

\includegraphics[width=5.12986in,height=0.92569in]{media/image66.png}

\begin{enumerate}
\def\labelenumi{\arabic{enumi}.}
\setcounter{enumi}{74}
\item
  At
  \includegraphics[width=0.37986in,height=0.19444in]{media/image67.png},
  a function
  \includegraphics[width=0.37986in,height=0.26875in]{media/image68.png}
  has a positive value, with a positive slope that is decreasing in
  magnitude with increasing \emph{x}. Which option could correspond to
  \includegraphics[width=0.37986in,height=0.26875in]{media/image68.png}?
\end{enumerate}

\begin{enumerate}
\def\labelenumi{\Alph{enumi}.}
\item
  \pandocbounded{\includegraphics[keepaspectratio]{media/image69.wmf}}
\item
  \pandocbounded{\includegraphics[keepaspectratio]{media/image70.wmf}}
\item
  \pandocbounded{\includegraphics[keepaspectratio]{media/image71.wmf}}
\item
  \pandocbounded{\includegraphics[keepaspectratio]{media/image72.wmf}}
\end{enumerate}

\textbf{Solution} The correct answer is (C). All four options have
positive values at
\includegraphics[width=0.37986in,height=0.19444in]{media/image67.png},
but only the first three options have positive slopes at that
\emph{x}-value. In option (A) the slope is constant, while the slope
increases with \emph{x} in option (B). Only the function in Option (C)
features a slope that decreases in magnitude with \emph{x}, while
satisfying the other conditions.

\end{document}
