\documentclass{beamer}
\usepackage{../../../shared/templates/ds9_theme}
\usepackage[overridenote]{pdfpc}
\graphicspath{{../images/}{../../shared/images/}}

\title[Heat and Work]{PHYS11 CH:12 The Universe's Accounting System}
\subtitle{How Energy Becomes Unavailable}
\author[Mr. Gullo]{Mr. Gullo}
\date[December 2025]{December 2025}

\begin{document}

\frame{\titlepage
\note{[THE HOOK] Today we discover why no engine can be perfect.\\\\
- Same laws limit steam engines, car engines, and your body\\\\
- Three revelations: thermal equilibrium, energy conservation, entropy's arrow\\\\
[THE WONDER] By end, you'll understand why the universe runs in only one direction.\\\\
- This is the physics of why time moves forward}
}

\begin{frame}
\frametitle{Outline}
\tableofcontents
\end{frame}

\section{Introduction}

\begin{frame}
\frametitle{The Mystery of the Perfect Engine}
\begin{center}
\Large What if you could build an engine\\
\textit{that converts all heat into work?}
\end{center}

\pause
\vspace{0.5cm}
Engineers have been trying for 300 years...

\pause
\vspace{0.3cm}
\alert{Nature says: Impossible.}
\note{[P0] "What if you could build a perfect engine?"\\\\
[P1] "Engineers have tried for 300 years - steam engines, car engines, jet engines"\\\\
[P2] [THE CONFLICT] "Nature says impossible. The laws of thermodynamics set hard limits"\\\\
[THE WONDER] Today we discover why the universe has unbreakable accounting rules}
\end{frame}

\begin{frame}
\frametitle{Energy Transforms}
\begin{figure}
\centering
\includegraphics[width=0.7\textwidth,height=0.5\textheight,keepaspectratio]{phys11-thermodynamics-fig12-1.jpg}
\caption{Steam engine: burning fuel transfers heat to do work}
\end{figure}

\pause
\textbf{The challenge:} Most thermal energy escapes as waste heat.
\note{[P0][Fig 12.1: A steam engine uses]  "Steam engines revolutionized civilization"\\\\
[P1] "But most energy is wasted - only 25 to 30 percent becomes useful work"\\\\
[THE CONNECTION - Kinetic Archetype] "Like running a race where only 30 percent of effort moves you forward"\\\\
[THE WONDER] Laws of thermodynamics explain why this waste is unavoidable}
\end{frame}

\section{12.1 Zeroth Law: Thermal Equilibrium}

\begin{frame}
\frametitle{Learning Objectives}
\begin{block}{By the end of this section, you will be able to:}
\begin{itemize}
\item \textbf{12.1:} Explain the zeroth law of thermodynamics
\end{itemize}
\end{block}
\note{- One objective: understand thermal equilibrium\\\\
- Foundation for temperature measurement\\\\
- Why it's called zeroth law\\\\
- Assessment: conceptual questions}
\end{frame}

\begin{frame}
\frametitle{12.1 When Things Stop Changing}
\begin{exampleblock}{The Mental Model}
Place ice in warm water. What happens?
\end{exampleblock}

\pause
\vspace{0.3cm}

\begin{itemize}
\item Ice melts (gains heat) \pause
\item Water cools (loses heat) \pause
\item Eventually: both reach same temperature \pause
\item Heat transfer stops
\end{itemize}

\pause
\vspace{0.3cm}

\begin{block}{Nature's Rule}
They are now in \textbf{thermal equilibrium}.
\end{block}
\note{[P0] "Place ice in warm water - what happens?"\\\\
[P1] "Ice melts - gains heat"\\\\
[P2] "Water cools - loses heat"\\\\
[P3] "Eventually both reach same temperature"\\\\
[P4] "Heat transfer stops"\\\\
[P5] [THE REVELATION] "Thermal equilibrium - same temp, no more energy flow"\\\\
[THE WONDER] This is how thermometers work}
\end{frame}

\begin{frame}
\frametitle{12.1 The Zeroth Law}
\begin{block}{Universal Law: The Transitive Property}
If system A is in thermal equilibrium with system B,\\
and B is in thermal equilibrium with system C,\\
then A is also in thermal equilibrium with C.
\end{block}

\pause
\vspace{0.3cm}

\begin{exampleblock}{In Math Terms}
If $a = b$ and $b = c$, then $a = c$.
\end{exampleblock}

\pause
\vspace{0.3cm}

\alert{Why "zeroth"?} Discovered after first and second laws, but more fundamental.
\note{[P0] "The zeroth law of thermodynamics"\\\\
[P1] "Like transitive property in math"\\\\
[P2] "Why zeroth? Discovered AFTER first and second but realized it's foundational"\\\\
[THE HUMILITY] Science doesn't always discover in logical order\\\\
[THE CONNECTION - Digital Archetype] "Like version 1.0 then realizing you need 0.0"}
\end{frame}

\begin{frame}
\frametitle{12.1 Real-World: Neonatal Incubators}
\begin{figure}
\centering
\includegraphics[width=0.6\textwidth,height=0.45\textheight,keepaspectratio]{phys11-thermodynamics-fig12-2.jpg}
\caption{Premature baby in incubator}
\end{figure}

\pause
\textbf{Zeroth law in action:}\\
Air, incubator walls, and baby all reach thermal equilibrium at safe temperature.
\note{- Premature babies have little body covering\\\\
- Incubator maintains constant temperature\\\\
- Air, walls, baby all reach thermal equilibrium\\\\
- Baby safe and comfortable\\\\
- Zeroth law guarantees temperature consistency throughout}
\end{frame}

\begin{frame}
\frametitle{12.1 Why Not Earth and Sun?}
\begin{alertblock}{Question for Discussion}
Sun transfers heat to Earth. Why don't they reach thermal equilibrium?
\end{alertblock}

\pause
\vspace{0.3cm}

Answer: Empty space separates them.

\pause
\vspace{0.3cm}

Thermal equilibrium requires \textbf{thermal contact} - ability to freely exchange energy.

\pause
\vspace{0.3cm}

\alert{Fortunately!} Otherwise Earth would be as hot as sun's surface ($\sim$5800 K).
\note{[P0][Fig 12.2: An engineer makes a]  [THE CONFLICT] "Why doesn't Earth heat to match sun?"\\\\
[P1] "Empty space separates them"\\\\
[P2] "Thermal equilibrium needs thermal contact"\\\\
[P3] [THE WONDER] "If thermal contact with sun, life impossible"\\\\
- Space is insulator - we get radiation but not direct heat transfer}
\end{frame}

\section{12.2 First Law: Energy Conservation}

\begin{frame}
\frametitle{Learning Objectives}
\begin{block}{By the end of this section, you will be able to:}
\begin{itemize}
\item \textbf{12.2:} Describe how pressure, volume, temperature relate using ideal gas law \pause
\item \textbf{12.2:} Describe pressure-volume work \pause
\item \textbf{12.2:} State first law verbally and mathematically \pause
\item \textbf{12.2:} Solve first law problems
\end{itemize}
\end{block}
\note{[P0] "Four objectives for first law"\\\\
[P1] "Understand ideal gas law"\\\\
[P2] "Learn pressure-volume work"\\\\
[P3] "State first law"\\\\
[P4] "Solve practical problems - assessment"}
\end{frame}

\begin{frame}
\frametitle{12.2 Pressure: Force over Area}
\begin{block}{Definition: Pressure}
\begin{center}
\Large $\boxed{P = \frac{F}{A}}$
\end{center}
Pressure is force per unit area perpendicular to surface.
\end{block}

\pause
\vspace{0.3cm}

\textbf{SI unit:} Pascal (Pa), where $1 \text{ Pa} = 1 \text{ N/m}^2$
\note{[P0] [ALGEBRA] "P equals F over A"\\\\
[P1] "Pascal: one newton per square meter"\\\\
[THE CONNECTION - Kinetic Archetype] "Why sharp objects pierce skin - same force, tiny area, huge pressure"\\\\
- Large area: low pressure\\\\
- Tiny area: high pressure}
\end{frame}

\begin{frame}
\frametitle{12.2 Same Force, Different Pressure}
\begin{figure}
\centering
\includegraphics[width=0.7\textwidth,height=0.5\textheight,keepaspectratio]{phys11-thermodynamics-fig12-3.jpg}
\end{figure}

\pause
\begin{alertblock}{Civilian View vs. Reality}
\textbf{Civilian:} "The needle pushes harder."\\
\textbf{Physicist:} "Same force, smaller area = higher pressure."
\end{alertblock}
\note{[P0] [Fig 12.3: (a) Although the person] "Same force applied two ways"\\\\
[P1] [THE CONFLICT] "Civilians think needle pushes harder. Same force!"\\\\
- Finger: large area, low pressure\\\\
- Needle: tiny area, high pressure\\\\
[THE REVELATION] Pressure equals force divided by area}
\end{frame}

\begin{frame}
\frametitle{12.2 The Ideal Gas Law}
\begin{block}{Universal Law: Gas Behavior}
\begin{center}
\Large $\boxed{PV = NkT}$
\end{center}
Pressure times volume equals particles times Boltzmann constant times absolute temperature.
\end{block}

\pause
\vspace{0.3cm}

Where:
\begin{itemize}
\item $P$ = pressure (Pa)
\item $V$ = volume (m$^3$)
\item $N$ = number of particles
\item $k = 1.38 \times 10^{-23}$ J/K (Boltzmann constant)
\item $T$ = absolute temperature (K)
\end{itemize}
\note{[P0] [ALGEBRA] "P-V equals N-k-T"\\\\
[P1] "One of physics' most important equations"\\\\
- Works for ideal gases: low pressure, high temp\\\\
- Boltzmann constant links microscopic to macroscopic\\\\
[THE WONDER] One equation describes all gases everywhere}
\end{frame}

\begin{frame}
\frametitle{12.2 Gas Law Relationships}
For fixed amount of gas:

\vspace{0.3cm}

\begin{itemize}
\item \textbf{Constant volume:} $P \propto T$ (pressure rises with temperature) \pause
\item \textbf{Constant temperature:} $P \propto \frac{1}{V}$ (pressure inverse to volume) \pause
\item \textbf{Constant pressure:} $V \propto T$ (volume rises with temperature)
\end{itemize}

\pause
\vspace{0.5cm}

\begin{exampleblock}{The Mental Model}
Pumping tire: volume increases, then pressure builds, tire warms up.
\end{exampleblock}
\note{[P0] "Three relationships from ideal gas law"\\\\
[P1] "Constant V: P up means T up"\\\\
[P2] "Constant T: P up means V down"\\\\
[P3] "Constant P: V up means T up"\\\\
[P4] [THE CONNECTION - Kinetic Archetype] "Pump tire - feel all three"\\\\
- First volume grows, then pressure builds, tire gets warm}
\end{frame}

\begin{frame}
\frametitle{12.2 Pumping a Tire}
\begin{figure}
\centering
\includegraphics[width=0.7\textwidth,height=0.5\textheight,keepaspectratio]{phys11-thermodynamics-fig12-4.jpg}
\end{figure}

\pause
(a) Volume increases. (b) Pressure increases. (c) Temperature increases.
\note{[P0] [Fig 12.4: (a) When air is] "Three stages of pumping air into deflated tire"\\\\
- All three variables P, V, T connected by ideal gas law\\\\
[P1] "Stage a: volume expands. Stage b: walls resist, pressure rises. Stage c: temp rises"\\\\
[THE CONNECTION - Digital Archetype] "Three linked variables in physics simulation"\\\\
[THE WONDER] Ideal gas law predicts all three stages}
\end{frame}

\begin{frame}
\frametitle{12.2 Pressure-Volume Work}
\begin{figure}
\centering
\includegraphics[width=0.7\textwidth,height=0.45\textheight,keepaspectratio]{phys11-thermodynamics-fig12-2.jpg}
\caption{Gas expansion does work pushing piston}
\end{figure}

\pause
\begin{block}{Nature's Rule for Gases}
\begin{center}
\Large $\boxed{W = P\Delta V}$
\end{center}
Work equals pressure times change in volume.
\end{block}
\note{[P0][Fig 12.2: An engineer makes a]  "When gas expands, it does work"\\\\
[P1] [ALGEBRA] "W equals P delta V"\\\\
- Gas pushes piston through distance\\\\
- Force times distance equals work\\\\
- P times A times d equals P times delta V\\\\
[THE REVELATION] For fluids: pressure like force, volume change like distance}
\end{frame}

\begin{frame}
\frametitle{12.2 The First Law of Thermodynamics}
\begin{block}{Universal Law: Energy Conservation}
\begin{center}
\Large $\boxed{\Delta U = Q - W}$
\end{center}
Change in internal energy equals heat added minus work done by system.
\end{block}

\pause
\vspace{0.3cm}

Where:
\begin{itemize}
\item $\Delta U$ = change in internal energy
\item $Q$ = net heat into system (positive if in, negative if out)
\item $W$ = net work by system (positive if out, negative if in)
\end{itemize}

\pause
\vspace{0.3cm}

\alert{This is conservation of energy for thermal systems.}
\note{[P0][Fig 12.2: An engineer makes a]  [ALGEBRA] "Delta U equals Q minus W"\\\\
[P1] "Q and W are energy in transit - delta U is stored"\\\\
[P2] "This is conservation of energy"\\\\
[THE REVELATION] Energy cannot be created or destroyed\\\\
[THE HUMILITY] Sign conventions confuse everyone at first}
\end{frame}

\begin{frame}
\frametitle{12.2 Understanding the Signs}
\begin{columns}[T]
\column{0.48\textwidth}
\textbf{Heat Q:}
\begin{itemize}
\item Positive: flows IN\\(adds energy)
\item Negative: flows OUT\\(removes energy)
\end{itemize}

\pause
\column{0.48\textwidth}
\textbf{Work W:}
\begin{itemize}
\item Positive: done BY system\\(removes energy)
\item Negative: done ON system\\(adds energy)
\end{itemize}
\end{columns}

\pause
\vspace{0.5cm}

\begin{alertblock}{Key Insight}
Positive $Q$ adds energy. Positive $W$ removes energy.
\end{alertblock}
\note{[P0] "Sign conventions critical"\\\\
[P1] "Q positive: heat in, energy up. Q negative: heat out, energy down"\\\\
[P2] "W positive: system does work, energy down. W negative: work on system, energy up"\\\\
[THE HUMILITY] This trips everyone up - track energy flow direction carefully}
\end{frame}

\begin{frame}
\frametitle{12.2 Energy Flow Diagram}
\begin{figure}
\centering
\includegraphics[width=0.7\textwidth,height=0.5\textheight,keepaspectratio]{phys11-thermodynamics-fig12-8.jpg}
\end{figure}

\vspace{0.3cm}

$Q$ in adds energy. $W$ out removes energy. $\Delta U$ is net change.
\note{[Fig 12.8: Two different processes] - System like bank account\\\\
- Q is deposits and withdrawals\\\\
- W is spending\\\\
- Delta U is change in balance\\\\
[THE CONNECTION - Digital Archetype] "Like tracking energy credits in game"}
\end{frame}

\begin{frame}
\frametitle{Attempt: Energy Accounting}
\begin{exampleblock}{The Challenge (3 min, silent)}
System absorbs 40.0 J of heat, does 10.0 J of work.\\
Later, 25.0 J heat leaves, 4.0 J work done ON system.

\vspace{0.3cm}

\textbf{Find:} Net change in internal energy $\Delta U$

\vspace{0.3cm}

\textit{Can you track energy? Work silently.}
\end{exampleblock}
\note{[THE CHALLENGE] Can they track energy like accountant?\\\\
[SAY] "Try on your own. Watch signs carefully."\\\\
[TIMING] 3-4 min SILENT work\\\\
[CIRCULATE] Note who reverses signs\\\\
[WATCH FOR] Confusion: heat in vs out, work by vs on\\\\
[DON'T HELP] Let them struggle with signs}
\end{frame}

\begin{frame}
\frametitle{Compare: Energy Tracking}
\textbf{Turn and talk (2 min):}

\vspace{0.3cm}

\begin{enumerate}
\item What was net heat $Q$? How calculate?
\item What was net work $W$? Signs correct?
\item Did you use $\Delta U = Q - W$?
\end{enumerate}

\vspace{0.5cm}

\pause
\alert{Name wheel:} One pair share approach (not answer).
\note{[TIMING] 2-3 min pair discussion\\\\
[CIRCULATE] Listen for sign confusion\\\\
[CHECK] Name wheel: call pair\\\\
[EXPECTED APPROACH] Q = 40 - 25 = 15 J. W = 10 - 4 = 6 J. Delta U = Q - W\\\\
[COMMON ERROR] Forget work ON system is negative W}
\end{frame}

\begin{frame}
\frametitle{Reveal: Energy Conservation}
\textbf{Self-correct in different color:}

\vspace{0.3cm}

\textbf{Step 1 - Net heat:}
$$Q = 40.0 \text{ J} - 25.0 \text{ J} = 15.0 \text{ J}$$

\pause
\vspace{0.2cm}

\textbf{Step 2 - Net work:}
$$W = 10.0 \text{ J} - 4.0 \text{ J} = 6.0 \text{ J}$$

\pause
\vspace{0.2cm}

\textbf{Step 3 - First law:}
$$\Delta U = Q - W = 15.0 \text{ J} - 6.0 \text{ J}$$

\pause
$$\boxed{\Delta U = 9.0 \text{ J}}$$

\pause
\textbf{Check:} More heat in than work out $\rightarrow$ internal energy increases.
\note{[P0] "Self-correct in different color"\\\\
[P1] [ALGEBRA] "Net heat: 40 in minus 25 out = 15 J"\\\\
[P2] "Net work: 10 by minus 4 on = 6 J"\\\\
[P3] "First law: delta U = Q - W"\\\\
[P4] [ANSWER] "9.0 J - internal energy increased"\\\\
[THE WONDER] You just did energy accounting like thermodynamics engineer}
\end{frame}

\begin{frame}
\frametitle{12.2 Biology: Your Body as Heat Engine}
\begin{figure}
\centering
\includegraphics[width=0.6\textwidth,height=0.4\textheight,keepaspectratio]{phys11-thermodynamics-fig12-7.jpg}
\end{figure}

\pause
$$\Delta U = Q - W + \text{food energy}$$

\pause
\begin{itemize}
\item Food adds chemical potential energy
\item Work (exercise) removes energy
\item Heat (body temp) removes energy
\item Leftover stored as fat
\end{itemize}
\note{[P0] [Fig 12.7: (a) The first law] "Your body obeys first law"\\\\
[P1] [ALGEBRA] "Delta U = Q - W + food"\\\\
[P2] "Food is input. Exercise is work. Body heat is thermal output"\\\\
[THE CONNECTION - Kinetic Archetype] "Athletes burn more calories - more work output"\\\\
[THE WONDER] Same physics: steam engines and human metabolism}
\end{frame}

\section{12.3 Second Law: Entropy}

\begin{frame}
\frametitle{Learning Objectives}
\begin{block}{By the end of this section, you will be able to:}
\begin{itemize}
\item \textbf{12.3:} Describe entropy \pause
\item \textbf{12.3:} Describe second law of thermodynamics \pause
\item \textbf{12.3:} Solve entropy problems
\end{itemize}
\end{block}
\note{[P0] "Three objectives for second law"\\\\
[P1] "Understand entropy"\\\\
[P2] "Learn second law"\\\\
[P3] "Calculate entropy - assessment"\\\\
- This explains arrow of time}
\end{frame}

\begin{frame}
\frametitle{12.3 The Arrow of Time}
\begin{center}
\Large Why does time only move forward?
\end{center}

\pause
\vspace{0.5cm}

\begin{itemize}
\item Ice melts in warm water \pause
\item But water never spontaneously freezes around ice \pause
\item Gas expands to fill room \pause
\item But gas never spontaneously compresses to corner
\end{itemize}

\pause
\vspace{0.3cm}

\alert{Nature has preferred direction.}
\note{[P0] "Why does time only move forward?"\\\\
[P1] "Ice melts in warm water"\\\\
[P2] "Reverse never happens spontaneously"\\\\
[P3] "Gas expands to fill room"\\\\
[P4] "Never spontaneously gathers in corner"\\\\
[P5] [THE REVELATION] "Nature prefers disorder"\\\\
[THE WONDER] This why we remember past, not future}
\end{frame}

\begin{frame}
\frametitle{12.3 Entropy: Measure of Disorder}
\begin{block}{Definition: Entropy}
\begin{center}
\Large $\boxed{\Delta S = \frac{Q}{T}}$
\end{center}
Change in entropy equals heat transfer divided by absolute temperature.
\end{block}

\pause
\vspace{0.3cm}

\textbf{What entropy measures:}
\begin{itemize}
\item Disorder in system
\item Energy unavailable to do work
\item Direction of spontaneous processes
\end{itemize}

\pause
\vspace{0.3cm}

\textbf{SI unit:} J/K (joules per kelvin)
\note{[P0] [ALGEBRA] "Delta S = Q over T"\\\\
[P1] "Entropy is disorder, unavailable energy, process direction"\\\\
[THE HUMILITY] Entropy abstract - hardest concept in thermodynamics\\\\
[THE REVELATION] Higher entropy = more disorder, less usable energy}
\end{frame}

\begin{frame}
\frametitle{12.3 Ice Melting: Entropy Increases}
\begin{figure}
\centering
\includegraphics[width=0.7\textwidth,height=0.5\textheight,keepaspectratio]{phys11-thermodynamics-fig12-3.jpg}
\caption{Ice melts: ordered crystal becomes disordered liquid}
\end{figure}

\pause
\textbf{Entropy increases because:}
\begin{itemize}
\item Structured ice $\rightarrow$ random liquid
\item System becomes more disordered
\end{itemize}
\note{[P0][Fig 12.3: (a) Although the person]  "Ice melting: classic entropy increase"\\\\
[P1] "Ice: molecules locked in crystal. Water: molecules move randomly"\\\\
- Ordered to disordered\\\\
- Structure to chaos\\\\
[THE WONDER] Can see entropy increase when ice melts}
\end{frame}

\begin{frame}
\frametitle{12.3 The Second Law}
\begin{block}{Universal Law: Entropy Always Increases}
For any spontaneous process, total entropy of universe either increases or remains constant. Never decreases.
\end{block}

\pause
\vspace{0.3cm}

$$\Delta S_{\text{total}} = \Delta S_{\text{system}} + \Delta S_{\text{env}} \geq 0$$

\pause
\vspace{0.3cm}

\begin{alertblock}{Key Consequences}
\begin{itemize}
\item Heat flows spontaneously hot to cold, never cold to hot
\item Energy becomes less available over time
\item Disorder increases
\end{itemize}
\end{alertblock}
\note{[P0] [Fig 12.3: (a) Although the person] "Second law - entropy never decreases"\\\\
[P1] [ALGEBRA] "Delta S-total = system + environment, always ≥ 0"\\\\
[P2] "Heat flows hot to cold. Energy becomes less useful. Disorder grows"\\\\
[THE REVELATION] This is arrow of time\\\\
[THE WONDER] Explains aging, inefficiency, universe trending to chaos}
\end{frame}

\begin{frame}
\frametitle{12.3 Heat Flow and Entropy}
\begin{figure}
\centering
\includegraphics[width=0.7\textwidth,height=0.5\textheight,keepaspectratio]{phys11-thermodynamics-fig12-4.jpg}
\end{figure}

\pause
\textbf{Why heat flows hot to cold:}\\
Larger entropy increase at low T than decrease at high T.

\pause
$$\Delta S = \frac{Q}{T} \rightarrow \text{smaller } T \text{ means larger } \Delta S$$
\note{[P0][Fig 12.4: (a) When air is]  "Heat always flows hot to cold - why?"\\\\
[P1] [ALGEBRA] "Delta S = Q/T. Smaller T = larger delta S"\\\\
[P2] "Cold gains more entropy than hot loses"\\\\
[THE REVELATION] Total entropy increases when heat flows hot to cold\\\\
- Reverse would decrease total entropy - forbidden}
\end{frame}

\begin{frame}
\frametitle{12.3 Can Entropy Decrease?}
\textbf{Yes, locally!} But total entropy of universe must increase.

\pause
\vspace{0.3cm}

\begin{exampleblock}{Local Entropy Decrease Examples}
\begin{itemize}
\item Clean room (you do work)
\item Build bridge from ore (energy input)
\item Plant grows (uses solar energy)
\item Freezer makes ice (work input)
\end{itemize}
\end{exampleblock}

\pause
\vspace{0.3cm}

\alert{In all cases, environment entropy increases MORE than system entropy decreases.}
\note{[P0] [Fig 12.4: (a) When air is] "Can entropy decrease? Yes, locally, with energy input"\\\\
[P1] "Clean room, build bridge, plant grow - all decrease local entropy"\\\\
[P2] "But environment entropy up MORE - total still up"\\\\
[THE HUMILITY] Second law applies to universe, not just one system\\\\
[THE WONDER] Life is local entropy decrease powered by sun}
\end{frame}

\begin{frame}
\frametitle{Attempt: Ice Melting Entropy}
\begin{exampleblock}{The Challenge (3 min, silent)}
Find entropy increase when 1.00 kg ice at $0^\circ$C melts to water at $0^\circ$C.

\vspace{0.3cm}

\textbf{Given:}
\begin{itemize}
\item Mass: $m = 1.00$ kg
\item Temperature: $T = 0^\circ$C $= 273$ K
\item Latent heat fusion: $L_f = 334$ kJ/kg
\end{itemize}

\textbf{Find:} $\Delta S$

\vspace{0.3cm}

\textit{Can you quantify disorder? Work silently.}
\end{exampleblock}
\note{[THE CHALLENGE] Can they calculate measure of disorder?\\\\
[SAY] "Try on your own. Remember Kelvin."\\\\
[TIMING] 3-4 min SILENT work\\\\
[CIRCULATE] Note who forgets Kelvin, who doesn't calc Q first\\\\
[WATCH FOR] Dividing by Celsius not Kelvin\\\\
[DON'T HELP] Let them work through}
\end{frame}

\begin{frame}
\frametitle{Compare: Entropy Calculation}
\textbf{Turn and talk (2 min):}

\vspace{0.3cm}

\begin{enumerate}
\item Formula for heat $Q$?
\item Convert to Kelvin?
\item Equation connecting $Q$ and $\Delta S$?
\end{enumerate}

\vspace{0.5cm}

\pause
\alert{Name wheel:} One pair share approach (not answer).
\note{[TIMING] 2-3 min pair discussion\\\\
[CIRCULATE] Listen for temperature conversion issues\\\\
[CHECK] Name wheel: call pair\\\\
[EXPECTED APPROACH] Q = m×L_f, then ΔS = Q/T in Kelvin\\\\
[COMMON ERROR] Using Celsius not Kelvin}
\end{frame}

\begin{frame}
\frametitle{Reveal: Entropy of Melting}
\textbf{Self-correct in different color:}

\vspace{0.3cm}

\textbf{Step 1 - Heat to melt:}
$$Q = mL_f = (1.00)(334 \text{ kJ/kg}) = 3.34 \times 10^5 \text{ J}$$

\pause
\vspace{0.2cm}

\textbf{Step 2 - Convert temp:}
$$T = 0^\circ\text{C} = 273 \text{ K}$$

\pause
\vspace{0.2cm}

\textbf{Step 3 - Entropy:}
$$\Delta S = \frac{Q}{T} = \frac{3.34 \times 10^5}{273}$$

\pause
$$\boxed{\Delta S = 1.22 \times 10^3 \text{ J/K}}$$

\pause
\textbf{Check:} Positive - disorder increased as ice melted.
\note{[P0] "Self-correct different color"\\\\
[P1] [ALGEBRA] "Q = m×L_f = 334,000 J"\\\\
[P2] "0 Celsius = 273 Kelvin - always absolute temp"\\\\
[P3] [ALGEBRA] "Delta S = Q/T"\\\\
[P4] [ANSWER] "1220 J/K - large entropy increase"\\\\
[THE WONDER] You quantified increase in disorder during phase change}
\end{frame}

\section{12.4 Heat Engines}

\begin{frame}
\frametitle{Learning Objectives}
\begin{block}{By the end of this section, you will be able to:}
\begin{itemize}
\item \textbf{12.4:} Explain how heat engines work \pause
\item \textbf{12.4:} Describe thermal efficiency \pause
\item \textbf{12.4:} Solve efficiency problems
\end{itemize}
\end{block}
\note{[P0] "Three objectives for heat engines"\\\\
[P1] "Understand engine operation"\\\\
[P2] "Learn efficiency"\\\\
[P3] "Calculate efficiency - assessment"\\\\
- Explains why no engine perfect}
\end{frame}

\begin{frame}
\frametitle{12.4 What Is Heat Engine?}
\begin{block}{Definition: Heat Engine}
Machine that converts thermal energy into mechanical work using heat transfer.
\end{block}

\pause
\vspace{0.3cm}

\textbf{Examples:}
\begin{itemize}
\item Car engines (gasoline, diesel)
\item Jet engines
\item Steam turbines
\item Your body
\end{itemize}

\pause
\vspace{0.3cm}

\alert{All use cyclical processes.}
\note{[P0] "Heat engine converts heat to work"\\\\
[P1] "Car, jet, steam, even body"\\\\
[P2] "Cyclical: returns to start after each cycle"\\\\
[THE CONNECTION - Kinetic Archetype] "Every muscle contraction is heat engine cycle"\\\\
[THE WONDER] Civilization runs on heat engines}
\end{frame}

\begin{frame}
\frametitle{12.4 How Heat Engines Work}
\begin{figure}
\centering
\includegraphics[width=0.7\textwidth,height=0.5\textheight,keepaspectratio]{phys11-thermodynamics-fig12-7.jpg}
\end{figure}

\pause
\begin{enumerate}
\item Absorb heat $Q_h$ from hot reservoir
\item Do work $W$
\item Reject waste heat $Q_c$ to cold reservoir
\end{enumerate}

\pause
$$W = Q_h - Q_c$$
\note{[P0][Fig 12.7: (a) The first law]  "Heat engine operation"\\\\
[P1] "Take heat from hot, do work, dump waste to cold"\\\\
[P2] [ALGEBRA] "W = Q_h - Q_c"\\\\
[THE REVELATION] Cannot convert all heat to work - some MUST be wasted\\\\
[THE CONFLICT] Second law forbids perfect conversion}
\end{frame}

\begin{frame}
\frametitle{12.4 Thermal Efficiency}
\begin{block}{Definition: Efficiency}
\begin{center}
\Large $\boxed{\text{Eff} = \frac{W}{Q_h}}$
\end{center}
Efficiency equals useful work divided by energy input.
\end{block}

\pause
\vspace{0.3cm}

\textbf{Typical efficiencies:}
\begin{itemize}
\item Gasoline car: 25-30\%
\item Diesel engine: 35-40\%
\item Coal plant: 40-45\%
\item Human body: ~25\%
\end{itemize}

\pause
\vspace{0.3cm}

\alert{100\% impossible!} (Second law forbids)
\note{[P0][Fig 12.7: (a) The first law]  [ALGEBRA] "Eff = W/Q_h"\\\\
[P1] "Car engines: only 25-30\% efficient"\\\\
[P2] "100\% impossible - second law sets hard limit"\\\\
[THE HUMILITY] Engineers tried centuries to beat this\\\\
[THE WONDER] Nature's accounting rules unbreakable}
\end{frame}

\begin{frame}
\frametitle{12.4 Why Engines Cannot Be Perfect}
\textbf{Perfect efficiency requires:} $Q_c = 0$ (no waste heat)

\pause
\vspace{0.3cm}

\textbf{But second law says:}
\begin{itemize}
\item Entropy must increase
\item Heat MUST flow to cold reservoir
\item Some energy MUST become unavailable
\end{itemize}

\pause
\vspace{0.3cm}

\begin{alertblock}{Fundamental Limit}
Second law sets absolute ceiling on efficiency.
\end{alertblock}
\note{[P0] "Perfect efficiency needs zero waste"\\\\
[P1] "But second law requires entropy increase = some heat wasted"\\\\
[P2] "Fundamental limit - not just engineering"\\\\
[THE CONFLICT] We want perfect, nature forbids\\\\
[THE WONDER] Universe has unbreakable rules}
\end{frame}

\begin{frame}
\frametitle{Attempt: Power Plant Efficiency}
\begin{exampleblock}{The Challenge (3 min, silent)}
Coal plant absorbs $2.50 \times 10^{14}$ J, releases $1.48 \times 10^{14}$ J as waste in one day.

\vspace{0.3cm}

\textbf{Find:}
\begin{enumerate}
\item Work output $W$
\item Efficiency
\end{enumerate}

\vspace{0.3cm}

\textit{Can you measure wastefulness? Work silently.}
\end{exampleblock}
\note{[THE CHALLENGE] Can they calculate real efficiency?\\\\
[SAY] "Try on your own. Where does energy go?"\\\\
[TIMING] 3-4 min SILENT\\\\
[CIRCULATE] Note who forgets W = Q_h - Q_c\\\\
[WATCH FOR] Wrong efficiency formula\\\\
[DON'T HELP] Let them apply what learned}
\end{frame}

\begin{frame}
\frametitle{Compare: Efficiency Analysis}
\textbf{Turn and talk (2 min):}

\vspace{0.3cm}

\begin{enumerate}
\item How find work $W$?
\item Which formula for efficiency?
\item Is 40\% good or bad?
\end{enumerate}

\vspace{0.5cm}

\pause
\alert{Name wheel:} One pair share approach (not answer).
\note{[TIMING] 2-3 min pair discussion\\\\
[CIRCULATE] Listen for understanding\\\\
[CHECK] Name wheel: call pair\\\\
[EXPECTED APPROACH] W = Q_h - Q_c, Eff = W/Q_h\\\\
[COMMON ERROR] Dividing by wrong heat}
\end{frame}

\begin{frame}
\frametitle{Reveal: Power Plant Analysis}
\textbf{Self-correct in different color:}

\vspace{0.3cm}

\textbf{(a) Work output:}
$$W = Q_h - Q_c = 2.50 \times 10^{14} - 1.48 \times 10^{14}$$

\pause
$$\boxed{W = 1.02 \times 10^{14} \text{ J}}$$

\pause
\vspace{0.3cm}

\textbf{(b) Efficiency:}
$$\text{Eff} = \frac{W}{Q_h} = \frac{1.02 \times 10^{14}}{2.50 \times 10^{14}}$$

\pause
$$\boxed{\text{Eff} = 0.408 = 40.8\%}$$

\pause
\textbf{Check:} Typical for coal. 59.2\% wasted!
\note{[P0] "Self-correct different color"\\\\
[P1] [ALGEBRA] "W = heat in - heat out"\\\\
[P2] [ANSWER] "1.02×10^14 J"\\\\
[P3] [ALGEBRA] "Eff = W/Q_h"\\\\
[P4] [ANSWER] "40.8\% - typical for coal"\\\\
[THE WONDER] Almost 60\% wasted - climate change challenge}
\end{frame}

\begin{frame}
\frametitle{12.4 Heat Pumps and Refrigerators}
\begin{figure}
\centering
\includegraphics[width=0.7\textwidth,height=0.5\textheight,keepaspectratio]{phys11-thermodynamics-fig12-11.jpg}
\end{figure}

\pause
\textbf{Heat engines in reverse:}
\begin{itemize}
\item Use work to move heat cold to hot
\item Refrigerators cool interior, warm exterior
\item Heat pumps warm house using outside air
\end{itemize}
\note{[P0] [Fig 12.11: These ice floes melt] "Heat pumps/fridges: engines backward"\\\\
[P1] "Use work to move heat cold to hot - opposite spontaneous"\\\\
[THE REVELATION] Don't create cold, move heat\\\\
[THE CONNECTION - Digital Archetype] "Like running game backward"\\\\
- Requires energy to fight natural tendency}
\end{frame}

\section{Summary}

\begin{frame}
\frametitle{What You Now Know}
\begin{block}{Four Laws of Thermodynamics}
\begin{enumerate}
\item \textbf{Zeroth:} Temperature equilibrium transitive \pause
\item \textbf{First:} Energy conserved: $\Delta U = Q - W$ \pause
\item \textbf{Second:} Entropy always increases \pause
\item \textbf{Third:} (Not covered) Absolute zero unreachable
\end{enumerate}
\end{block}

\pause
\vspace{0.3cm}

\begin{alertblock}{Universe's Accounting System}
Energy conserved, but becomes less useful over time.
\end{alertblock}
\note{[P0] "Four laws"\\\\
[P1] "Zeroth: equilibrium transitive"\\\\
[P2] "First: energy conservation"\\\\
[P3] "Second: entropy up"\\\\
[P4] "Third: we didn't cover - absolute zero unreachable"\\\\
[THE WONDER] Four laws govern every energy transformation}
\end{frame}

\begin{frame}[shrink]
\frametitle{Key Equations}
\begin{align}
P &= \frac{F}{A} \quad \text{(Pressure)} \\
PV &= NkT \quad \text{(Ideal gas)} \\
W &= P\Delta V \quad \text{(P-V work)} \\
\Delta U &= Q - W \quad \text{(First law)} \\
\Delta S &= \frac{Q}{T} \quad \text{(Entropy)} \\
\text{Eff} &= \frac{W}{Q_h} \quad \text{(Efficiency)}
\end{align}
\note{- Six key equations\\\\
- Pressure, ideal gas, work, first law, entropy, efficiency\\\\
- Know when to use each\\\\
- Questions before end?}
\end{frame}

\begin{frame}
\frametitle{Arrow of Time}
\begin{center}
\Large You now understand why time moves forward.
\end{center}

\pause
\vspace{0.5cm}

\textbf{Entropy increases:}
\begin{itemize}
\item Ice melts, doesn't spontaneously freeze
\item Gas expands, doesn't spontaneously compress
\item Engines waste heat - cannot recover
\item We age - time cannot run backward
\end{itemize}

\pause
\vspace{0.3cm}

\alert{Second law gives time its direction.}
\note{[P0] "You now understand arrow of time"\\\\
[P1] "Entropy increases - why time moves forward"\\\\
[P2] "Second law explains aging, inefficiency, one-way time"\\\\
[THE WONDER] You learned deepest truth about universe\\\\
[THE HUMILITY] Took humanity centuries to figure this}
\end{frame}

\begin{frame}
\frametitle{Homework}
\begin{center}
\Large
Complete assigned problems\\[0.3cm]
posted on LMS
\end{center}
\note{[SAY] "Homework on LMS"\\\\
[TIMING] Due date: check LMS\\\\
[CHECK] Questions before end?\\\\
[TRANSITION] Next: Review for Unit 5 test}
\end{frame}

\end{document}
