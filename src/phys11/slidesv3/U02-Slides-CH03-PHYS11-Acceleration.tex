\documentclass{beamer}
\usepackage{../../../shared/templates/ds9_theme}
\usepackage[overridenote]{pdfpc}
\graphicspath{{../images/}{../../shared/images/}}

\title[The Rate of Change]{PHYS11 CH:3 The Rate of Change}
\subtitle{Understanding Acceleration}
\author[Mr. Gullo]{Mr. Gullo}
\date[December 2025]{December 2025}

\begin{document}

\frame{\titlepage
\note{[THE HOOK] Today we discover how to measure change itself.\\\\
- Motion is one thing - but the CHANGE in motion? That's acceleration\\\\
- Two revelations: what acceleration IS, how we MEASURE it with equations and graphs\\\\
[THE WONDER] By end of class, you'll understand the force that pins you to your seat in a car}
}

\begin{frame}
\frametitle{Outline}
\tableofcontents
\end{frame}

\section{Introduction}

\begin{frame}
\frametitle{The Mystery of Motion}
\begin{center}
\Large What if you could feel\\
\textit{the rate at which change happens?}
\end{center}

\pause
\vspace{0.5cm}
From the airplane landing to the dragster launching...

\pause
\vspace{0.3cm}
\alert{You experience acceleration every day.}
\note{[P0] "What if you could feel the rate at which change happens?"\\\\
[P1] "From the airplane landing to the dragster launching..."\\\\
[P2] [THE WONDER] "You experience acceleration every day - in cars, elevators, even walking"\\\\
[THE CONNECTION - Kinetic Archetype] Athletes: every time you speed up or slow down}
\end{frame}

\begin{frame}
\frametitle{Landing in St. Maarten}
\begin{figure}
\centering
\includegraphics[width=0.8\textwidth,height=0.6\textheight,keepaspectratio]{phys11-acceleration-fig3-1.jpg}
\caption{A plane slows down as it comes in for landing}
\end{figure}

\pause
\begin{alertblock}{The Paradox}
\textbf{Civilian:} "Acceleration means speeding up."\\
\textbf{Physicist:} "Acceleration is ANY change in velocity - speeding up, slowing down, or turning."
\end{alertblock}
\note{[P0][Fig 3.1: A plane slows down]  "Plane landing at St. Maarten - velocity decreasing"\\\\
[P1] [THE CONFLICT] "Civilians think acceleration means going fast"\\\\
- "Physicist knows: acceleration is CHANGE in velocity"\\\\
- "This plane IS accelerating - just in opposite direction to motion"\\\\
[THE HUMILITY] Your brain evolved to feel changes, not measure them precisely}
\end{frame}

\section{3.1 Acceleration}

\begin{frame}
\frametitle{Learning Objectives}
\begin{block}{By the end of this section, you will be able to:}
\begin{itemize}
\item \textbf{3.1:} Explain acceleration and determine direction and magnitude in one dimension \pause
\item \textbf{3.1:} Analyze motion using kinematic equations and graphic representations
\end{itemize}
\end{block}
\note{[P0] "Two objectives for section 3.1"\\\\
[P1] "First: understand what acceleration IS and how to find it"\\\\
- "Second: use equations and graphs to analyze accelerated motion"\\\\
- Assessment: quiz next week on kinematics}
\end{frame}

\begin{frame}
\frametitle{3.1 The Source Code of Change}
\begin{block}{Nature's Rule for Acceleration}
\begin{center}
\Large $\boxed{\bar{a} = \frac{\Delta v}{\Delta t} = \frac{v_f - v_0}{t_f - t_0}}$
\end{center}
Acceleration equals change in velocity divided by change in time.
\end{block}

\pause
\vspace{0.3cm}

\textbf{SI Units:} meters per second per second ($\text{m/s}^2$)

\pause
\begin{exampleblock}{The Mental Model}
If velocity is how fast you're going, acceleration is how fast your "how fast" is changing.
\end{exampleblock}
\note{[P0] [THE REVELATION] "a-bar equals v-f minus v-zero over t-f minus t-zero"\\\\
[P1] "Units: meters per second per second - velocity change per second"\\\\
[P2] [THE CONNECTION - Digital Archetype] "Like your speedometer app showing not just speed but how quickly speed changes"\\\\
[THE WONDER] Same equation works for rocket ships and bicycles}
\end{frame}

\begin{frame}
\frametitle{3.1 Understanding the Sign}
\begin{columns}[T]
\column{0.48\textwidth}
\textbf{Positive Acceleration}
\begin{itemize}
\item Velocity and acceleration in same direction
\item Speeding up to the right
\item Slowing down to the left
\end{itemize}

\pause
\column{0.48\textwidth}
\textbf{Negative Acceleration}
\begin{itemize}
\item Velocity and acceleration in opposite directions
\item Slowing down to the right
\item Speeding up to the left
\end{itemize}
\end{columns}

\pause
\vspace{0.3cm}
\begin{alertblock}{Key Insight}
The sign tells you the DIRECTION, not whether you're speeding up or slowing down!
\end{alertblock}
\note{[P0] "Two types based on direction"\\\\
[P1] "Negative acceleration doesn't mean slowing down"\\\\
[P2] [THE CONFLICT] "Your brain thinks negative means slowing. Physics disagrees"\\\\
- "Negative just means direction - could be speeding up leftward"\\\\
[THE HUMILITY] Even physics students mix this up at first}
\end{frame}

\begin{frame}
\frametitle{3.1 Speeding Up and Slowing Down}
\begin{figure}
\centering
\includegraphics[width=0.7\textwidth,height=0.5\textheight,keepaspectratio]{phys11-acceleration-fig3-2.jpg}
\caption{(a) Car speeding up, (b) Car slowing down}
\end{figure}

\pause
\textbf{The Rule:}
\begin{itemize}
\item Same direction = speeding up
\item Opposite direction = slowing down
\end{itemize}
\note{[P0] [Fig 3.2: The car is speeding] "Two scenarios with vector arrows"\\\\
[P1] "When arrows point same way: speeding up. Opposite: slowing down"\\\\
[THE CONNECTION - Kinetic Archetype] "In a car: gas pedal adds acceleration forward. Brake adds acceleration backward"\\\\
- "Both are acceleration - just different directions"}
\end{frame}

\begin{frame}
\frametitle{3.1 Acceleration is a Vector}
\textbf{Vector quantities have both magnitude AND direction:}
\begin{itemize}
\item Displacement \pause
\item Velocity \pause
\item Acceleration
\end{itemize}

\pause
\vspace{0.3cm}

\textbf{Critical insight:} An object traveling at constant speed can still accelerate if it changes direction!

\pause
\begin{exampleblock}{Real-World: Turning}
When you turn the steering wheel in a moving car, the car accelerates even if the speedometer doesn't change.
\end{exampleblock}
\note{[P0] "Three vector quantities we study"\\\\
[P1] "Displacement"\\\\
[P2] "Velocity"\\\\
[P3] "Acceleration - change in velocity includes direction changes"\\\\
[P4] [THE REVELATION] "Constant speed in a circle? Still accelerating"\\\\
[P5] [THE CONNECTION - Kinetic Archetype] "Runners: turning a corner at constant speed requires acceleration toward the center"\\\\
[THE WONDER] Direction is as fundamental as magnitude}
\end{frame}

\begin{frame}
\frametitle{Attempt: Subway Train Accelerating}
\begin{exampleblock}{The Challenge (3 min, silent)}
A subway train accelerates from rest to 30.0 km/h in 20.0 s.

\vspace{0.3cm}

\textbf{Given:}
\begin{itemize}
\item Initial velocity: $v_0 = 0$ (starts from rest)
\item Final velocity: $v_f = 30.0$ km/h
\item Time interval: $\Delta t = 20.0$ s
\end{itemize}

\textbf{Find:} Average acceleration in m/s$^2$

\vspace{0.3cm}

\textit{Can you decode this motion? Work silently. Remember to convert units!}
\end{exampleblock}
\note{[THE CHALLENGE] Can you find how quickly the train speeds up?\\\\
[SAY] "Try this on your own. It's okay to get stuck on the units."\\\\
[TIMING] 3-4 min SILENT individual work\\\\
[CIRCULATE] Note who forgets to convert km/h to m/s\\\\
[WATCH FOR] Common error: keeping mixed units\\\\
[DON'T HELP] Let them struggle - learning happens in Compare}
\end{frame}

\begin{frame}
\frametitle{Compare: Unit Conversion Strategy}
\textbf{Turn and talk (2 min):}

\vspace{0.3cm}

\begin{enumerate}
\item What equation did you use for acceleration?
\item How did you convert km/h to m/s?
\item What multiplication factors did you use?
\end{enumerate}

\vspace{0.5cm}

\pause
\alert{Name wheel:} One pair share your approach (not your answer).
\note{[TIMING] 2-3 min pair discussion\\\\
[CIRCULATE] Listen for conversion approaches\\\\
[CHECK] Name wheel: call a pair to share\\\\
[EXPECTED APPROACH] a-bar equals delta-v over delta-t, convert using 1000 m per km and 3600 s per h\\\\
[COMMON ERROR] Multiplying by 3600 instead of dividing}
\end{frame}

\begin{frame}
\frametitle{Reveal: The Acceleration Calculation}
\textbf{Self-correct in a different color:}

\vspace{0.3cm}

\textbf{Step 1:} Convert 30.0 km/h to m/s

\pause
$$30.0 \frac{\text{km}}{\text{h}} \times \frac{1000\text{ m}}{1\text{ km}} \times \frac{1\text{ h}}{3600\text{ s}} = 8.333 \text{ m/s}$$

\pause
\textbf{Step 2:} Calculate $\Delta v = v_f - v_0 = 8.333 - 0 = 8.333$ m/s

\pause
\textbf{Step 3:} Apply the equation
$$\bar{a} = \frac{\Delta v}{\Delta t} = \frac{8.333\text{ m/s}}{20.0\text{ s}} = \boxed{+0.417 \text{ m/s}^2}$$

\pause
\textbf{Check:} Positive sign means acceleration to the right. Reasonable for train speeding up!
\note{[P0] "Self-correct in different color"\\\\
[P1] [ALGEBRA] "30 km per h times 1000 m per km times 1 h per 3600 s equals 8.333 m per s"\\\\
[P2] "Change in velocity: 8.333 minus zero equals 8.333"\\\\
[P3] "a-bar equals 8.333 divided by 20 equals 0.417 meters per second squared"\\\\
[P4] [ANSWER] "Plus 0.417 - positive because train speeds up to the right"\\\\
[THE WONDER] You just calculated what subway engineers design for}
\end{frame}

\begin{frame}
\frametitle{Attempt: Subway Train Slowing Down}
\begin{exampleblock}{The Challenge (3 min, silent)}
Now the train slows to a stop from 30.0 km/h in 8.00 s.

\vspace{0.3cm}

\textbf{Given:}
\begin{itemize}
\item Initial velocity: $v_0 = 30.0$ km/h = 8.333 m/s
\item Final velocity: $v_f = 0$ (comes to rest)
\item Time interval: $\Delta t = 8.00$ s
\end{itemize}

\textbf{Find:} Average acceleration in m/s$^2$

\vspace{0.3cm}

\textit{Will the sign be positive or negative? Why?}
\end{exampleblock}
\note{[THE CHALLENGE] Now the train is slowing - what changes?\\\\
[SAY] "Same process, but train is stopping. Think about the sign."\\\\
[TIMING] 3 min SILENT work\\\\
[CIRCULATE] Note who gets the negative sign\\\\
[THE CONNECTION - Kinetic Archetype] "Like hitting the brakes"}
\end{frame}

\begin{frame}
\frametitle{Compare: Sign of Acceleration}
\textbf{Turn and talk (2 min):}

\vspace{0.3cm}

\begin{enumerate}
\item What did you get for $\Delta v$?
\item Is it positive or negative?
\item What does the sign of acceleration tell you?
\end{enumerate}

\vspace{0.5cm}

\pause
\alert{Name wheel:} Share your reasoning about the sign.
\note{[TIMING] 2 min pair discussion\\\\
[CIRCULATE] Listen for understanding of negative sign\\\\
[CHECK] Name wheel call pair\\\\
[EXPECTED APPROACH] Delta-v equals 0 minus 8.333 equals negative 8.333\\\\
[KEY INSIGHT] Negative because velocity is decreasing}
\end{frame}

\begin{frame}
\frametitle{Reveal: Deceleration Calculation}
\textbf{Self-correct in a different color:}

\vspace{0.3cm}

\textbf{Step 1:} Calculate $\Delta v = v_f - v_0 = 0 - 8.333 = -8.333$ m/s

\pause
\textbf{Step 2:} Apply the equation
$$\bar{a} = \frac{\Delta v}{\Delta t} = \frac{-8.333\text{ m/s}}{8.00\text{ s}} = \boxed{-1.04 \text{ m/s}^2}$$

\pause
\vspace{0.3cm}

\textbf{Check:} Negative sign means acceleration to the left (opposite to velocity). Train is slowing down!

\pause
\begin{alertblock}{Physics vs Civilian Language}
\textbf{Civilian:} "The train is decelerating."\\
\textbf{Physicist:} "The train has negative acceleration."
\end{alertblock}
\note{[P0] "Self-correct in different color"\\\\
[P1] [ALGEBRA] "Delta-v equals zero minus 8.333 equals negative 8.333"\\\\
[P2] [ANSWER] "Negative 1.04 meters per second squared - acceleration opposes motion"\\\\
[P3] [THE CONFLICT] "Civilians say deceleration. Physicists just say negative acceleration"\\\\
[THE WONDER] Same math describes braking and launching - just different signs}
\end{frame}

\section{3.2 Representing Acceleration with Equations and Graphs}

\begin{frame}
\frametitle{Learning Objectives}
\begin{block}{By the end of this section, you will be able to:}
\begin{itemize}
\item \textbf{3.2:} Explain kinematic equations related to acceleration and illustrate with graphs \pause
\item \textbf{3.2:} Apply kinematic equations and graphs to problems involving acceleration
\end{itemize}
\end{block}
\note{[P0] "Two objectives for section 3.2"\\\\
[P1] "First: understand the five kinematic equations and their graphs"\\\\
- "Second: solve real problems with these tools"\\\\
- "These are the prediction engines of motion"}
\end{frame}

\begin{frame}
\frametitle{3.2 The Five Kinematic Equations}
\textbf{For constant acceleration only:}

\begin{align}
d &= d_0 + \bar{v}t \\
\bar{v} &= \frac{v_0 + v_f}{2} \\
v &= v_0 + at \\
d &= d_0 + v_0 t + \frac{1}{2}at^2 \\
v^2 &= v_0^2 + 2a(d - d_0)
\end{align}

\pause
\begin{exampleblock}{The Mental Model}
These five equations are the grammar of motion. Learn which one to use when.
\end{exampleblock}
\note{[P0] "Five equations - memorize these patterns"\\\\
[P1] [THE REVELATION] "Each relates different variables"\\\\
- "Equation 1: displacement, average velocity, time"\\\\
- "Equation 3: no displacement - useful when you don't know or need it"\\\\
- "Equation 5: no time - useful when time isn't given"\\\\
[THE CONNECTION - Digital Archetype] "Like function calls - pick the right one for your inputs"}
\end{frame}

\begin{frame}
\frametitle{3.2 Displacement vs Time}
\begin{figure}
\centering
\includegraphics[width=0.7\textwidth,height=0.5\textheight,keepaspectratio]{phys11-acceleration-fig3-6.jpg}
\caption{Slope of displacement vs time gives velocity}
\end{figure}

\pause
\textbf{Key insight:}
$$\bar{v} = \frac{d}{t} \quad \text{(when starting from origin)}$$

\pause
The slope IS the velocity!
\note{[P0] "Graph of displacement vs time"\\\\
[P1] "Slope equals rise over run equals displacement over time equals velocity"\\\\
[P2] [THE WONDER] "Every line on a d-vs-t graph tells you a velocity"\\\\
- "Steeper slope means faster motion"\\\\
- "This is why we graph - pictures reveal patterns"}
\end{frame}

\begin{frame}
\frametitle{3.2 Velocity vs Time}
\begin{figure}
\centering
\includegraphics[width=0.7\textwidth,height=0.5\textheight,keepaspectratio]{phys11-acceleration-fig3-4.jpg}
\caption{Slope of velocity vs time gives acceleration}
\end{figure}

\pause
\textbf{Key insight:}
$$a = \frac{v}{t} \quad \text{(when starting from rest)}$$

\pause
The slope IS the acceleration!
\note{[P0] [Fig 3.4: ] "Graph of velocity vs time"\\\\
[P1] "Slope equals velocity change over time equals acceleration"\\\\
[P2] [THE REVELATION] "Every v-vs-t graph encodes acceleration"\\\\
- "Straight line means constant acceleration"\\\\
- "Curved line means changing acceleration"\\\\
[THE CONNECTION - Digital Archetype] "Derivative in calculus - slope of tangent line"}
\end{frame}

\begin{frame}
\frametitle{3.2 Choosing the Right Equation}
\textbf{Strategy:}
\begin{enumerate}
\item List the knowns \pause
\item Identify the unknown \pause
\item Pick equation with unknown and all knowns
\end{enumerate}

\pause
\vspace{0.3cm}

\begin{block}{Example Decision Tree}
\begin{itemize}
\item Time not given? Use equation 5: $v^2 = v_0^2 + 2a(d - d_0)$
\item Displacement not needed? Use equation 3: $v = v_0 + at$
\item From rest ($v_0 = 0$)? Equations simplify!
\end{itemize}
\end{block}
\note{[P0] "Problem-solving strategy"\\\\
[P1] "List knowns from problem"\\\\
[P2] "What are you solving for?"\\\\
[P3] "Match equation to your variables"\\\\
[THE HUMILITY] Even experts check their equation choice twice\\\\
[THE CONNECTION - Digital Archetype] "Like choosing the right algorithm for your data"}
\end{frame}

\begin{frame}
\frametitle{Attempt: Dragster Problem}
\begin{exampleblock}{The Challenge (3 min, silent)}
A dragster accelerates from rest at $26.0$ m/s$^2$ for a quarter mile (402 m).

\vspace{0.3cm}

\textbf{Given:}
\begin{itemize}
\item $v_0 = 0$ (starts from rest)
\item $a = 26.0$ m/s$^2$
\item $d - d_0 = 402$ m
\end{itemize}

\textbf{Find:} Final velocity $v_f$

\vspace{0.3cm}

\textit{Which kinematic equation should you use? Why?}
\end{exampleblock}
\note{[THE CHALLENGE] Can you predict the speed of a dragster?\\\\
[SAY] "Real race car problem. Choose your equation wisely."\\\\
[TIMING] 3-4 min SILENT work\\\\
[CIRCULATE] Note who picks equation 5 (no time given)\\\\
[WATCH FOR] Square root at the end\\\\
[THE CONNECTION - Kinetic Archetype] "This is actual drag racing acceleration"}
\end{frame}

\begin{frame}
\frametitle{Compare: Equation Selection}
\textbf{Turn and talk (2 min):}

\vspace{0.3cm}

\begin{enumerate}
\item Which equation did you choose?
\item Why is that equation appropriate?
\item What variables does it NOT include?
\end{enumerate}

\vspace{0.5cm}

\pause
\alert{Name wheel:} One pair explain their equation choice.
\note{[TIMING] 2-3 min discussion\\\\
[CIRCULATE] Listen for equation 5 reasoning\\\\
[CHECK] Name wheel\\\\
[EXPECTED APPROACH] "Equation 5: v-squared equals v-zero-squared plus 2ad because time is not given"\\\\
[KEY INSIGHT] Choose equation based on what you DON'T know}
\end{frame}

\begin{frame}
\frametitle{Reveal: Dragster Speed}
\textbf{Self-correct in a different color:}

\vspace{0.3cm}

\textbf{Step 1:} Choose equation 5 (no time): $v^2 = v_0^2 + 2a(d - d_0)$

\pause
\textbf{Step 2:} Since $v_0 = 0$: $v^2 = 2a(d - d_0)$

\pause
\textbf{Step 3:} Substitute values
$$v^2 = 2(26.0)(402) = 2.09 \times 10^4 \text{ m}^2/\text{s}^2$$

\pause
\textbf{Step 4:} Take square root
$$v = \sqrt{2.09 \times 10^4} = \boxed{145 \text{ m/s}}$$

\pause
\textbf{Check:} About 324 mph - reasonable for dragster!
\note{[P0] "Self-correct"\\\\
[P1] [ALGEBRA] "Equation 5: v-squared equals v-zero-squared plus 2ad"\\\\
[P2] "v-zero is zero, so v-squared equals 2 times 26 times 402"\\\\
[P3] "Equals 2.09 times 10 to the 4"\\\\
[P4] [ANSWER] "145 meters per second - about 324 miles per hour"\\\\
[THE WONDER] You just calculated drag race speed from pure math}
\end{frame}

\begin{frame}
\frametitle{3.2 Acceleration Due to Gravity}
\begin{block}{Nature's Constant}
\begin{center}
\Large $\boxed{g = 9.80 \text{ m/s}^2}$
\end{center}
Near Earth's surface, all objects fall with this acceleration (ignoring air resistance).
\end{block}

\pause
\vspace{0.3cm}

\textbf{Convention:} When using $g$ in equations, give it a negative sign because gravity points downward.

\pause
\begin{exampleblock}{The Mental Model}
Every second of free fall, velocity increases by 9.80 m/s downward.
\end{exampleblock}
\note{[P0] [THE REVELATION] "g equals 9.80 meters per second squared"\\\\
[P1] "Negative in equations because downward"\\\\
[P2] [THE CONNECTION - Kinetic Archetype] "Every jump, every throw - this is the number"\\\\
[THE WONDER] Same value whether you drop a feather or a hammer (on the Moon)\\\\
[THE HUMILITY] Galileo discovered this fighting 2000 years of wrong ideas}
\end{frame}

\begin{frame}
\frametitle{Attempt: Rock Thrown Upward}
\begin{exampleblock}{The Challenge (3 min, silent)}
A rock is thrown straight up with initial velocity $v_0 = 13.0$ m/s.

\vspace{0.3cm}

\textbf{Given:}
\begin{itemize}
\item $v_0 = 13.0$ m/s (upward)
\item $a = -9.80$ m/s$^2$ (gravity)
\item $t = 1.00$ s
\end{itemize}

\textbf{Find:}
\begin{itemize}
\item Position $y$ at 1.00 s
\item Velocity $v$ at 1.00 s
\end{itemize}

\textit{Choose your equations wisely!}
\end{exampleblock}
\note{[THE CHALLENGE] Predict where a thrown rock will be\\\\
[SAY] "You need TWO equations - one for position, one for velocity"\\\\
[TIMING] 3-4 min SILENT work\\\\
[CIRCULATE] Note who uses equation 4 for position, equation 3 for velocity\\\\
[WATCH FOR] Negative sign on acceleration}
\end{frame}

\begin{frame}
\frametitle{Compare: Gravity Problems}
\textbf{Turn and talk (2 min):}

\vspace{0.3cm}

\begin{enumerate}
\item Which equations did you use?
\item How did you handle the negative sign for gravity?
\item Is the rock still going up or coming down?
\end{enumerate}

\vspace{0.5cm}

\pause
\alert{Name wheel:} Share your approach.
\note{[TIMING] 2-3 min discussion\\\\
[CIRCULATE] Listen for equation choices\\\\
[CHECK] Name wheel\\\\
[EXPECTED] Equation 4 for y, equation 3 for v\\\\
[KEY QUESTION] How do you know if rock is going up or down? Check velocity sign!}
\end{frame}

\begin{frame}
\frametitle{Reveal: Rock Position and Velocity}
\textbf{Self-correct in a different color:}

\textbf{Position:} $y = y_0 + v_0 t + \frac{1}{2}at^2$

\pause
$$y = 0 + (13.0)(1.00) + \frac{1}{2}(-9.80)(1.00)^2 = \boxed{8.10 \text{ m}}$$

\pause
\textbf{Velocity:} $v = v_0 + at$

\pause
$$v = 13.0 + (-9.80)(1.00) = \boxed{3.20 \text{ m/s}}$$

\pause
\vspace{0.3cm}

\textbf{Check:} Positive position (above starting point) and positive velocity (still going up). Makes sense!
\note{[P0] "Self-correct"\\\\
[P1] [ALGEBRA] "y equals 13 times 1 plus one-half times negative 9.8 times 1 squared equals 8.10 m"\\\\
[P2] "Rock is 8.10 meters above starting point"\\\\
[P3] "v equals 13 plus negative 9.8 times 1"\\\\
[P4] [ANSWER] "3.20 m/s - positive means still going up, but slowing"\\\\
[THE WONDER] You just predicted the future using 400-year-old equations}
\end{frame}

\section{Summary}

\begin{frame}
\frametitle{What You Now Know}
\begin{block}{The Revelations}
\begin{enumerate}
\item Acceleration = rate of change of velocity (direction matters!) \pause
\item $\bar{a} = \Delta v / \Delta t$ - the definition of acceleration \pause
\item Five kinematic equations predict motion \pause
\item Graphs reveal acceleration as slopes \pause
\item $g = 9.80$ m/s$^2$ - Earth's gravitational acceleration \pause
\item Choose equations based on knowns and unknowns
\end{enumerate}
\end{block}
\note{[P0] "Six revelations today"\\\\
[P1] "Acceleration is rate of velocity change"\\\\
[P2] "Master equation: a-bar equals delta-v over delta-t"\\\\
[P3] "Five kinematic equations for constant acceleration"\\\\
[P4] "Graphs show acceleration as slopes"\\\\
[P5] "g equals 9.80 - the constant of falling"\\\\
[P6] "Problem-solving: match equation to variables"\\\\
[THE WONDER] You can now predict motion like a physicist}
\end{frame}

\begin{frame}[shrink]
\frametitle{Key Equations}
\begin{align}
\bar{a} &= \frac{\Delta v}{\Delta t} = \frac{v_f - v_0}{t_f - t_0} \\
d &= d_0 + \bar{v}t \\
\bar{v} &= \frac{v_0 + v_f}{2} \\
v &= v_0 + at \\
d &= d_0 + v_0 t + \frac{1}{2}at^2 \\
v^2 &= v_0^2 + 2a(d - d_0) \\
g &= 9.80 \text{ m/s}^2
\end{align}
\note{- Seven equations - these are your tools\\\\
- Equation 1: definition of acceleration\\\\
- Equations 2-6: kinematic equations for constant acceleration\\\\
- Equation 7: gravitational acceleration\\\\
- Know when to use each\\\\
- Questions?}
\end{frame}

\begin{frame}
\frametitle{Homework}
\begin{center}
\Large
Complete the assigned problems\\[0.3cm]
posted on the LMS
\end{center}
\note{[SAY] "Homework posted on LMS"\\\\
[TIMING] Due date: check LMS\\\\
[CHECK] Questions before we end?\\\\
[TRANSITION] Next class: Chapter 4 Motion in Two Dimensions}
\end{frame}

\end{document}
