\documentclass{beamer}
\usepackage{../../../shared/templates/ds9_theme}
\usepackage[overridenote]{pdfpc}
\graphicspath{{../images/}{../../shared/images/}}

\title[Work and Energy]{PHYS11 CH:9 The Currency of the Universe}
\subtitle{How Energy Powers Everything}
\author[Mr. Gullo]{Mr. Gullo}
\date[December 2025]{December 2025}

\begin{document}

\frame{\titlepage
\note{[THE HOOK] Today we learn the universe's currency - energy.\\\\
- Same rules on roller coasters as on falling apples\\\\
- Two revelations: work changes energy, energy is conserved\\\\
[THE WONDER] By end of class, you'll understand what makes roller coasters work\\\\
- This is the foundation of thermodynamics, electricity, and every machine ever built}
}

\begin{frame}
\frametitle{Outline}
\tableofcontents
\end{frame}

\section{Introduction}

\begin{frame}
\frametitle{The Mystery}
\begin{center}
\Large What if nothing ever stopped?\\[0.3cm]
\textit{What if motion could last forever?}
\end{center}

\pause
\vspace{0.5cm}
Balls bounce lower. Roller coasters slow down. Pendulums stop swinging.

\pause
\vspace{0.3cm}
\alert{Where does the motion go?}
\note{[P0] "What if nothing ever stopped?"\\\\
[P1] "Balls bounce lower. Roller coasters slow down. Pendulums stop swinging."\\\\
[P2] [THE CONFLICT] "Where does the motion go? Motion doesn't vanish. It transforms."\\\\
[THE REVELATION] Today you discover the universe's accounting system}
\end{frame}

\begin{frame}
\frametitle{The Roller Coaster Experience}
\begin{figure}
\centering
\includegraphics[width=0.8\textwidth,height=0.6\textheight,keepaspectratio]{phys11-work-energy-fig01.jpg}
\caption{Roller coaster: energy transformation in action}
\end{figure}

\pause
Lifted to the top. Released. Speed builds. Climbs again. Slows down.

\pause
\textbf{Same energy. Different forms.}
\note{[P0] "Roller coaster - pure physics playground"\\\\
[P1] "Lifted to the top. Released. Speed builds. Climbs again. Slows down."\\\\
[P2] [THE REVELATION] "Same energy. Different forms. This is what we decode today."\\\\
[THE CONNECTION - Kinetic Archetype] "Every twist your body feels is energy transforming"\\\\
[THE WONDER] Engineers use these equations to keep riders safe at 100 km/h}
\end{frame}

\section{Work, Power, and the Work-Energy Theorem}

\begin{frame}
\frametitle{Learning Objectives}
\begin{block}{By the end of this section, you will be able to:}
\begin{itemize}
\item \textbf{9.1:} Describe and apply the work-energy theorem \pause
\item \textbf{9.1:} Describe and calculate work and power
\end{itemize}
\end{block}
\note{[P0] "Two objectives for work and energy"\\\\
[P1] "First: work-energy theorem connects force to motion. Second: power is rate of doing work"\\\\
- Assessment: problem set and quiz}
\end{frame}

\begin{frame}
\frametitle{9.1 What Is Work?}
\begin{alertblock}{Civilian View vs. Reality}
\textbf{Civilian:} "Homework is work. Holding a heavy box is hard work."\\
\textbf{Physicist:} "Neither is work. No motion, no work."
\end{alertblock}

\pause
\vspace{0.3cm}

\begin{block}{The Universal Law of Work}
\begin{center}
\Large $\boxed{W = Fd}$
\end{center}
Work equals force times distance moved in the direction of the force.
\end{block}
\note{[P0] [THE CONFLICT] "Civilians use work to mean effort. Physicists mean force moving an object."\\\\
[P1] [THE REVELATION] "W equals F d - force times distance"\\\\
- Lifting a rock upward: work\\\\
- Carrying a rock horizontally at constant speed: not work\\\\
- Holding a rock still: not work\\\\
[THE HUMILITY] This feels wrong. Our language lies. The math tells truth.}
\end{frame}

\begin{frame}
\frametitle{9.1 Three Examples}
\textbf{Consider these scenarios:}

\pause
\begin{enumerate}
\item Homework - objects not moving over distance \pause
\item Lifting a rock upward - force moves object upward \pause
\item Carrying a rock horizontally at constant speed
\end{enumerate}

\pause
\vspace{0.3cm}

\begin{exampleblock}{The Mental Model}
Work requires TWO things: force AND motion in the direction of force.
\end{exampleblock}
\note{[P0] "Three scenarios - which is work?"\\\\
[P1] "Homework - no objects moving"\\\\
[P2] "Lifting a rock - force moves object upward - this IS work"\\\\
[P3] "Carrying rock horizontally at constant speed"\\\\
[P4] "No net force needed for constant velocity. Gravity pulls down, you push up, forces cancel. No work."\\\\
[THE CONNECTION - Kinetic Archetype] "Skaters know this - once you're gliding, no force needed"}
\end{frame}

\begin{frame}
\frametitle{9.1 Energy: The Ability to Do Work}
\textbf{Two forms of mechanical energy:}

\pause
\vspace{0.3cm}

\begin{block}{Kinetic Energy (KE)}
Energy of motion. A moving object has kinetic energy.
\end{block}

\pause

\begin{block}{Potential Energy (PE)}
Stored energy. Gravitational PE is energy an object has due to its position above Earth's surface.
\end{block}

\pause
\vspace{0.3cm}
Both measured in joules (J), same unit as work.
\note{[P0] "Two forms of mechanical energy"\\\\
[P1] "Kinetic energy - energy of motion"\\\\
[P2] "Potential energy - stored energy from position"\\\\
[P3] "Both measured in joules - same unit as work"\\\\
[THE REVELATION] Work and energy use the same unit because they convert into each other\\\\
[THE WONDER] One joule - force of 1 newton over 1 meter}
\end{frame}

\begin{frame}
\frametitle{9.1 Gravitational Potential Energy}
Lift a rock off the ground. You exert force upward over distance.

\pause
\vspace{0.3cm}

The force equals the rock's weight: $F = w = mg$

\pause

Work done equals force times distance:

\begin{block}{Gravitational Potential Energy}
\begin{center}
\Large $\boxed{PE = mgh}$
\end{center}
Potential energy equals mass times gravity times height.
\end{block}
\note{[P0] "Lift a rock - you do work against gravity"\\\\
[P1] "Force equals weight: F equals m g"\\\\
[P2] "Work done equals force times distance: W equals m g h"\\\\
[P3] [THE REVELATION] "This work becomes stored as potential energy: PE equals m g h"\\\\
- m: mass in kg\\\\
- g: 9.8 m/s squared\\\\
- h: height in meters\\\\
[THE WONDER] Energy stored at top becomes motion at bottom}
\end{frame}

\begin{frame}
\frametitle{9.1 Kinetic Energy}
Drop the rock. Gravity does work on it. Rock speeds up.

\pause
\vspace{0.3cm}

\begin{block}{Kinetic Energy}
\begin{center}
\Large $\boxed{KE = \frac{1}{2}mv^2}$
\end{center}
Kinetic energy equals one-half mass times velocity squared.
\end{block}

\pause
\vspace{0.3cm}

\begin{exampleblock}{The Mental Model}
Heavier objects and faster objects have more KE. Velocity matters more because it's squared.
\end{exampleblock}
\note{[P0] "Drop the rock - gravity does work, rock speeds up"\\\\
[P1] [THE REVELATION] "KE equals one-half m v squared"\\\\
[P2] "Heavier and faster means more energy. Velocity matters MORE because it's squared."\\\\
[THE CONNECTION - Kinetic Archetype] "Doubling speed quadruples kinetic energy. That's why crashes are deadly."\\\\
[THE WONDER] Same equation for baseball, car, asteroid}
\end{frame}

\begin{frame}
\frametitle{9.1 The Work-Energy Theorem}
\begin{block}{Nature's Source Code}
\begin{center}
\Large $\boxed{W = \Delta KE = \frac{1}{2}mv_2^2 - \frac{1}{2}mv_1^2}$
\end{center}
Net work equals change in kinetic energy.
\end{block}

\pause
\vspace{0.3cm}

When you do work on an object, you change its kinetic energy.

\pause

\textbf{Subscripts:} $_1$ is initial, $_2$ is final.
\note{[P0] [THE REVELATION] "Work equals change in kinetic energy"\\\\
[P1] "When you do work on an object, you change its KE"\\\\
[P2] "Subscript 1 is initial, subscript 2 is final"\\\\
- James Joule discovered this\\\\
- Joule: unit named after him\\\\
[THE WONDER] Push a cart, increase its KE. Same law on Earth and Mars.}
\end{frame}

\begin{frame}
\frametitle{9.1 James Joule}
\begin{figure}
\centering
\includegraphics[width=0.5\textwidth,height=0.4\textheight,keepaspectratio]{phys11-work-energy-fig02.jpg}
\caption{James Joule (1818-1889)}
\end{figure}

\pause
The joule (J) is named after this physicist who proved the work-energy connection.

\pause
\vspace{0.3cm}
$1.0 \text{ J} = 1.0 \text{ N} \cdot \text{m} = 1.0 \text{ kg} \cdot \text{m}^2/\text{s}^2$
\note{[P0] "James Joule - English physicist"\\\\
[P1] "The joule is named after him"\\\\
[P2] "1 joule equals 1 newton-meter, which equals 1 kg m squared per s squared"\\\\
- Units reveal relationships\\\\
- Joule proved heat is a form of energy\\\\
[THE HUMILITY] Took decades to convince other scientists}
\end{frame}

\begin{frame}
\frametitle{9.1 Power: Rate of Doing Work}
Work tells you how much. Power tells you how fast.

\pause

\begin{block}{Power}
\begin{center}
\Large $\boxed{P = \frac{W}{t}}$
\end{center}
Power equals work divided by time.
\end{block}

\pause
\vspace{0.3cm}

\textbf{Unit:} Watt (W)

\pause
One watt equals one joule per second.
\note{[P0] "Power - rate of doing work"\\\\
[P1] [THE REVELATION] "P equals W over t"\\\\
[P2] "Unit is the watt"\\\\
[P3] "One watt equals one joule per second"\\\\
- Light bulb: 60 watts means 60 joules per second\\\\
- Electricity sold in kilowatt-hours: power times time equals energy\\\\
[THE CONNECTION - Digital Archetype] "Your gaming PC draws hundreds of watts"}
\end{frame}

\begin{frame}
\frametitle{9.1 Work vs. Power}
\begin{figure}
\centering
\includegraphics[width=0.7\textwidth,height=0.5\textheight,keepaspectratio]{phys11-work-energy-fig03.jpg}
\caption{Two ways to move a TV to the fourth floor}
\end{figure}

\pause
\textbf{Same work. Different power.}

Pulley (2 min) generates more power than stairs (5 min).
\note{[P0] "Two ways to move TV to fourth floor"\\\\
[P1] "Same work - same mass, same height, same force times distance"\\\\
[P2] "Pulley takes 2 minutes. Stairs take 5 minutes. Pulley generates more power."\\\\
- Power equals work over time\\\\
- Smaller time in denominator means more power\\\\
[THE CONNECTION - Kinetic Archetype] "Sprinting up stairs: more power than walking"}
\end{frame}

\begin{frame}
\frametitle{9.1 James Watt and the Steam Engine}
\begin{figure}
\centering
\includegraphics[width=0.6\textwidth,height=0.4\textheight,keepaspectratio]{phys11-work-energy-fig04.jpg}
\caption{James Watt (1736-1819)}
\end{figure}

\pause
Watt improved the steam engine, converting reciprocal motion to circular motion.

\pause
This innovation powered the industrial revolution.
\note{[P0] "James Watt - Scottish engineer"\\\\
[P1] "Improved steam engine - converted reciprocal to circular motion"\\\\
[P2] "This powered the industrial revolution"\\\\
- Watt unit named after him\\\\
- Suddenly trains, factories, ships could move\\\\
[THE WONDER] Power available to humanity increased tenfold in decades}
\end{frame}

\begin{frame}
\frametitle{Attempt: The Skater's Push}
\begin{exampleblock}{The Challenge (3 min, silent)}
An ice skater with mass 50 kg glides at 8 m/s. Her friend pushes, increasing speed to 12 m/s.

\vspace{0.3cm}

\textbf{Given:}
\begin{itemize}
\item $m = 50$ kg
\item $v_1 = 8$ m/s
\item $v_2 = 12$ m/s
\end{itemize}

\textbf{Find:} How much work did the friend do on the skater?

\vspace{0.3cm}

\textit{Can you calculate the work? Try it silently.}
\end{exampleblock}
\note{[THE CHALLENGE] Can they use work-energy theorem?\\\\
[SAY] "Try this on your own. It's okay to get stuck."\\\\
[TIMING] 3-4 min SILENT individual work\\\\
[CIRCULATE] Note who uses W = delta KE, who tries other approaches\\\\
[WATCH FOR] Students forgetting to subtract initial KE\\\\
[DON'T HELP] Let them struggle - learning happens in Compare}
\end{frame}

\begin{frame}
\frametitle{Compare: The Skater's Push}
\textbf{Turn and talk (2 min):}

\vspace{0.3cm}

\begin{enumerate}
\item What equation did you use?
\item Did you calculate KE initial and KE final?
\item What operation connects them to work?
\end{enumerate}

\vspace{0.5cm}

\pause
\alert{Name wheel:} One pair share your approach (not your answer).
\note{[TIMING] 2-3 min pair discussion\\\\
[CIRCULATE] Listen for common approaches\\\\
[CHECK] Name wheel: call a pair to share\\\\
[EXPECTED APPROACH] W equals delta KE equals one-half m v-2 squared minus one-half m v-1 squared\\\\
[COMMON ERROR] Forgetting the subtraction, or using wrong velocities}
\end{frame}

\begin{frame}
\frametitle{Reveal: The Skater's Push}
\textbf{Self-correct in a different color:}

\vspace{0.3cm}

\textbf{Equation:} $W = \Delta KE = \frac{1}{2}mv_2^2 - \frac{1}{2}mv_1^2$

\pause

\textbf{Factor out:} $W = \frac{1}{2}m(v_2^2 - v_1^2)$

\pause

\textbf{Substitute:} $W = \frac{1}{2}(50)(12^2 - 8^2)$

\pause

$W = 25(144 - 64) = 25(80)$

\pause

$$\boxed{W = 2000 \text{ J}}$$

\pause
\textbf{Check:} Energy increased because friend did work. Reasonable!
\note{[P0] "Self-correct in a different color"\\\\
[P1] [ALGEBRA] "W equals one-half m times quantity v-2 squared minus v-1 squared"\\\\
[P2] "Factor out one-half m"\\\\
[P3] "Substitute: one-half times 50 times quantity 144 minus 64"\\\\
[P4] "25 times 80"\\\\
[P5] [ANSWER] "2000 joules"\\\\
[P6] "Energy increased because friend did work. KE went from 1600 J to 3600 J, gain of 2000 J."\\\\
[THE WONDER] You just calculated what happens in every hockey check, every car acceleration}
\end{frame}

\section{Mechanical Energy and Conservation of Energy}

\begin{frame}
\frametitle{Learning Objectives}
\begin{block}{By the end of this section, you will be able to:}
\begin{itemize}
\item \textbf{9.2:} Explain the law of conservation of energy in terms of kinetic and potential energy \pause
\item \textbf{9.2:} Perform calculations related to kinetic and potential energy and apply conservation of energy
\end{itemize}
\end{block}
\note{[P0] "Two objectives for energy conservation"\\\\
[P1] "First: understand conservation law. Second: solve problems using it"\\\\
- This is one of the most powerful laws in all of physics}
\end{frame}

\begin{frame}
\frametitle{9.2 The Universe's Accounting System}
\begin{center}
\Large Energy is never created or destroyed.\\[0.3cm]
\textit{It only transforms.}
\end{center}

\pause
\vspace{0.5cm}

\begin{block}{The Law of Conservation of Energy}
In a closed system, total energy remains constant.
\end{block}

\pause
\vspace{0.3cm}

\begin{alertblock}{The Illusion}
\textbf{Civilian:} "The ball lost energy when it stopped bouncing."\\
\textbf{Physicist:} "Energy transformed to heat from friction and sound."
\end{alertblock}
\note{[P0] "Energy is never created or destroyed. It only transforms."\\\\
[P1] [THE REVELATION] "Law of conservation of energy - total energy is constant in closed system"\\\\
[P2] [THE CONFLICT] "Civilians think energy disappears. Physicists know it transforms."\\\\
- Bouncing ball: mechanical energy to heat and sound\\\\
- No energy lost, just transformed to less useful forms\\\\
[THE WONDER] This law has never been violated. Never. Not once in recorded history.}
\end{frame}

\begin{frame}
\frametitle{9.2 Energy Transformations}
Lift a rock to the top. You do work. Rock gains PE.

\pause

Release the rock. PE converts to KE as it falls.

\pause

Hit the ground. KE converts to heat and sound.

\pause
\vspace{0.3cm}

\begin{exampleblock}{The Mental Model}
Energy changes form constantly. The total amount stays the same.
\end{exampleblock}
\note{[P0] "Lift a rock - you do work, rock gains PE"\\\\
[P1] "Release - PE converts to KE as it falls"\\\\
[P2] "Hit ground - KE converts to heat and sound"\\\\
[P3] "Energy changes form constantly. Total stays same."\\\\
[THE CONNECTION - Harmonic Archetype] "Like money changing currency - dollars to euros, still same value"\\\\
[THE WONDER] Universe keeps perfect books. Energy always balances.}
\end{frame}

\begin{frame}
\frametitle{9.2 The Roller Coaster}
\begin{figure}
\centering
\includegraphics[width=0.8\textwidth,height=0.6\textheight,keepaspectratio]{phys11-work-energy-fig05.jpg}
\caption{Energy transformations on a roller coaster}
\end{figure}

\pause
\textbf{Top:} High PE, low KE (slow)

\pause
\textbf{Bottom:} Low PE, high KE (fast)

\pause
\textbf{Next hill:} KE converts back to PE
\note{[P0] "Roller coaster - energy transformation in action"\\\\
[P1] "Top: high PE, low KE - moving slowly"\\\\
[P2] "Bottom: low PE, high KE - moving fast"\\\\
[P3] "Next hill: KE converts back to PE, car slows down"\\\\
- Total energy constant (ignoring friction)\\\\
- PE plus KE at top equals PE plus KE at bottom\\\\
[THE WONDER] Engineers design the track so energy carries you through}
\end{frame}

\begin{frame}
\frametitle{9.2 Conservation Equation}
Assume no energy lost to friction.

\pause

\begin{block}{Conservation of Mechanical Energy}
\begin{center}
\Large $\boxed{KE_1 + PE_1 = KE_2 + PE_2}$
\end{center}
Initial kinetic plus initial potential equals final kinetic plus final potential.
\end{block}

\pause
\vspace{0.3cm}

Either side equals total mechanical energy.

\pause

\textbf{Closed system:} No energy lost to surroundings.
\note{[P0] "Assume no energy lost to friction"\\\\
[P1] [THE REVELATION] "KE-1 plus PE-1 equals KE-2 plus PE-2"\\\\
[P2] "Either side equals total mechanical energy"\\\\
[P3] "Closed system means no energy escapes"\\\\
- Subscript 1: initial state\\\\
- Subscript 2: final state\\\\
[THE HUMILITY] Real systems lose energy to friction, but this is excellent approximation}
\end{frame}

\begin{frame}
\frametitle{9.2 Equations Summary}
\begin{align*}
KE &= \frac{1}{2}mv^2 \\
PE &= mgh \\
KE_1 + PE_1 &= KE_2 + PE_2
\end{align*}

\pause
\vspace{0.3cm}

Combining them:
$$\frac{1}{2}mv_1^2 + mgh_1 = \frac{1}{2}mv_2^2 + mgh_2$$

\pause
\vspace{0.3cm}

\textbf{Pro tip:} Mass often cancels out!
\note{[P0] "Three key equations"\\\\
[P1] "Combine them into one master equation"\\\\
[P2] "Pro tip: m appears in every term, so it often cancels"\\\\
- Can solve problems without knowing mass\\\\
- This simplifies many calculations\\\\
[THE WONDER] Same equations for baseball, meteor, moon}
\end{frame}

\begin{frame}
\frametitle{Attempt: The Falling Rock}
\begin{exampleblock}{The Challenge (3 min, silent)}
A 10 kg rock falls from a 20 m cliff. When it has fallen 10 m, what are its KE and PE?

\vspace{0.3cm}

\textbf{Given:}
\begin{itemize}
\item $m = 10$ kg
\item $h_1 = 20$ m (initial height)
\item $h_2 = 10$ m (after falling 10 m)
\item $v_1 = 0$ (dropped from rest)
\item $g = 9.8$ m/s$^2$
\end{itemize}

\textbf{Find:} $KE_2$ and $PE_2$

\vspace{0.3cm}

\textit{Can you use conservation of energy? Work silently.}
\end{exampleblock}
\note{[THE CHALLENGE] Can they apply conservation of energy?\\\\
[SAY] "Try this on your own. It's okay to get stuck."\\\\
[TIMING] 3-4 min SILENT individual work\\\\
[CIRCULATE] Note who uses conservation equation, who calculates PE first\\\\
[WATCH FOR] Students forgetting initial KE is zero\\\\
[DON'T HELP] Let them struggle productively}
\end{frame}

\begin{frame}
\frametitle{Compare: The Falling Rock}
\textbf{Turn and talk (2 min):}

\vspace{0.3cm}

\begin{enumerate}
\item What is the initial KE? Why?
\item How did you calculate PE at 10 m height?
\item How did you find KE at 10 m height?
\end{enumerate}

\vspace{0.5cm}

\pause
\alert{Name wheel:} One pair share your approach (not your answer).
\note{[TIMING] 2-3 min pair discussion\\\\
[CIRCULATE] Listen for approaches\\\\
[CHECK] Name wheel: call a pair\\\\
[EXPECTED APPROACH] Initial KE is zero, calculate PE-2, use conservation to find KE-2\\\\
[COMMON ERROR] Not recognizing initial KE is zero, mixing up h-1 and h-2}
\end{frame}

\begin{frame}
\frametitle{Reveal: The Falling Rock}
\textbf{Self-correct in a different color:}

\vspace{0.3cm}

\textbf{Initial energy:} $KE_1 = 0$ (at rest), $PE_1 = mgh_1 = 10 \times 9.8 \times 20 = 1960$ J

\pause

\textbf{At 10 m height:} $PE_2 = mgh_2 = 10 \times 9.8 \times 10 = 980$ J

\pause

\textbf{Conservation:} $KE_1 + PE_1 = KE_2 + PE_2$

\pause

$0 + 1960 = KE_2 + 980$

\pause

$$\boxed{KE_2 = 980 \text{ J}}$$

\pause

\textbf{Check:} Lost 980 J of PE, gained 980 J of KE. Energy conserved!
\note{[P0] "Self-correct in different color"\\\\
[P1] [ALGEBRA] "Initial: KE-1 is zero, PE-1 equals m g h-1 equals 1960 joules"\\\\
[P2] "At 10 m: PE-2 equals m g h-2 equals 980 joules"\\\\
[P3] "Conservation: KE-1 plus PE-1 equals KE-2 plus PE-2"\\\\
[P4] "Zero plus 1960 equals KE-2 plus 980"\\\\
[P5] [ANSWER] "KE-2 equals 980 joules"\\\\
[P6] "Lost 980 J of PE, gained 980 J of KE. Perfect energy balance."\\\\
[THE WONDER] Universe's accounting system never makes mistakes}
\end{frame}

\begin{frame}
\frametitle{9.2 Why Conservation Works}
Total mechanical energy at any point equals energy at start.

\pause
\vspace{0.3cm}

At the top of roller coaster: mostly PE, little KE

\pause

At the bottom: mostly KE, little PE

\pause

At any point in between: PE + KE = constant

\pause
\vspace{0.3cm}

\begin{alertblock}{The Paradox}
\textbf{Your brain says:} "The roller coaster is slowing down, losing energy."\\
\textbf{Reality:} "KE converting to PE. Total energy unchanged."
\end{alertblock}
\note{[P0] "Total energy at any point equals energy at start"\\\\
[P1] "Top: mostly PE, little KE"\\\\
[P2] "Bottom: mostly KE, little PE"\\\\
[P3] "In between: PE plus KE equals constant"\\\\
[P4] [THE CONFLICT] "Brain says losing energy. Reality: energy transforming."\\\\
[THE HUMILITY] Friction does steal some energy as heat, but engineers minimize this\\\\
[THE WONDER] This law governs atoms, planets, galaxies}
\end{frame}

\begin{frame}
\frametitle{9.2 Real Systems and Friction}
\textbf{Ideal system:} No friction, energy perfectly conserved

\pause

\textbf{Real system:} Friction converts mechanical energy to heat

\pause
\vspace{0.3cm}

\begin{exampleblock}{The Mental Model}
Friction is not a violation of conservation. Heat is still energy, just less useful.
\end{exampleblock}

\pause
\vspace{0.3cm}

Engineers design to minimize friction losses.
\note{[P0] "Ideal system: no friction, energy perfectly conserved"\\\\
[P1] "Real system: friction converts mechanical energy to heat"\\\\
[P2] "Friction doesn't violate conservation. Heat is still energy, just less useful."\\\\
[P3] "Engineers design to minimize friction"\\\\
- Lubricants reduce friction\\\\
- Smooth surfaces reduce friction\\\\
- Air resistance is friction with air\\\\
[THE CONNECTION - Digital Archetype] "Hard drives spin on air bearings to reduce friction"}
\end{frame}

\begin{frame}
\frametitle{9.2 The Closed System}
\textbf{Closed system:} No energy enters or leaves

\pause

In reality, perfectly closed systems don't exist.

\pause

But many systems are approximately closed for short times.

\pause
\vspace{0.3cm}

\begin{block}{The Approximation}
Falling objects, roller coasters, pendulums: closed system is good approximation if friction is small.
\end{block}
\note{[P0] "Closed system: no energy in or out"\\\\
[P1] "Perfectly closed systems don't exist in reality"\\\\
[P2] "But many systems approximately closed for short times"\\\\
[P3] "Falling objects, roller coasters, pendulums: good approximation"\\\\
- Air resistance small for dense objects\\\\
- Friction small for smooth surfaces\\\\
[THE HUMILITY] Physicists use approximations, not perfection}
\end{frame}

\section{Summary}

\begin{frame}
\frametitle{What You Now Know}
\begin{block}{The Revelations}
\begin{enumerate}
\item Work = force times distance (only if motion happens) \pause
\item Work changes energy (work-energy theorem) \pause
\item Energy has two forms: kinetic (motion) and potential (stored) \pause
\item Power = rate of doing work \pause
\item Energy is conserved: KE + PE = constant \pause
\item Energy transforms but never vanishes
\end{enumerate}
\end{block}
\note{[P0] "Six revelations today"\\\\
[P1] "Work equals force times distance"\\\\
[P2] "Work changes energy - work-energy theorem"\\\\
[P3] "Two forms: kinetic and potential"\\\\
[P4] "Power is rate of doing work"\\\\
[P5] "Energy conserved: KE plus PE equals constant"\\\\
[P6] "Energy transforms but never vanishes"\\\\
[THE WONDER] You now understand the currency of the universe\\\\
- Name wheel: which revelation surprised you most?}
\end{frame}

\begin{frame}[shrink]
\frametitle{Key Equations}
\begin{align*}
W &= Fd \\
KE &= \frac{1}{2}mv^2 \\
PE &= mgh \\
W &= \Delta KE = \frac{1}{2}mv_2^2 - \frac{1}{2}mv_1^2 \\
P &= \frac{W}{t} \\
KE_1 + PE_1 &= KE_2 + PE_2
\end{align*}
\note{- Six foundational equations\\\\
- Work, kinetic energy, potential energy\\\\
- Work-energy theorem\\\\
- Power\\\\
- Conservation of energy\\\\
- Master these and you master energy\\\\
- Questions before we end?}
\end{frame}

\begin{frame}
\frametitle{Homework}
\begin{center}
\Large
Complete the assigned problems\\[0.3cm]
posted on the LMS
\end{center}
\note{[SAY] "Homework posted on LMS"\\\\
[TIMING] Due date: check LMS\\\\
[CHECK] Questions before we end?\\\\
[TRANSITION] Next class: Simple Machines - how to multiply force}
\end{frame}

\end{document}
