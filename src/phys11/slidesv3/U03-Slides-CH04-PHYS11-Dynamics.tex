\documentclass{beamer}
\usepackage{../../../shared/templates/ds9_theme}
\usepackage[overridenote]{pdfpc}
\graphicspath{{../images/}{../../shared/images/}}

\title[Rules of Reality]{PHYS11 CH:4 The Three Laws That Run the Universe}
\subtitle{From Newton to You}
\author[Mr. Gullo]{Mr. Gullo}
\date[December 2025]{December 2025}

\begin{document}

\frame{\titlepage
\note{[THE HOOK] Today we discover the three laws that govern ALL motion.\\\\
- Same laws explain your skateboard AND distant galaxies\\\\
- Four revelations: force, inertia, acceleration, action-reaction\\\\
[THE WONDER] By end of class, you'll understand why nothing moves without a reason.\\\\
- Newton figured this out 350 years ago - still perfect}
}

\begin{frame}
\frametitle{Outline}
\tableofcontents
\end{frame}

\section{Introduction}

\begin{frame}
\frametitle{The Mystery}
\begin{center}
\Large Why does a dolphin jump the way it does?\\[0.3cm]
\textit{What invisible rules guide its motion?}
\end{center}

\pause
\vspace{0.5cm}
Three simple laws explain ALL motion in the universe.

\pause
\vspace{0.3cm}
\alert{From dolphins to rockets to you.}
\note{[P0] "Why does a dolphin jump the way it does?"\\\\
[P1] "Three simple laws explain ALL motion in the universe"\\\\
[P2] [THE WONDER] "From dolphins to rockets to you. Newton discovered these 350 years ago."\\\\
[THE CONNECTION - Kinetic Archetype] Every athlete uses these laws without knowing it}
\end{frame}

\begin{frame}
\frametitle{Dolphin in Motion}
\begin{figure}
\centering
\includegraphics[width=0.8\textwidth,height=0.6\textheight,keepaspectratio]{phys11-dynamics-fig4-1.jpg}
\end{figure}

\pause
\begin{exampleblock}{The Mental Model}
The dolphin's path is not random. Physics predicts every curve, every arc.
\end{exampleblock}
\note{[P0][Fig 4.1: Newton's laws of motion]  "The dolphin jumping out of water - elegant, beautiful"\\\\
[P1] [THE REVELATION] "The path is not random. Physics predicts every curve, every arc"\\\\
[THE CONNECTION - Digital Archetype] "Like game physics engines - same equations"\\\\
[THE WONDER] Nature follows rules we can write down}
\end{frame}

\section{Force}

\begin{frame}
\frametitle{Learning Objectives}
\begin{block}{By the end of this lesson, you will be able to:}
\begin{itemize}
\item \textbf{4.1:} Differentiate between force, net force, and dynamics \pause
\item \textbf{4.1:} Draw a free-body diagram
\end{itemize}
\end{block}
\note{[P0] "Two objectives for force"\\\\
[P1] "Understand what force IS and how multiple forces combine"\\\\
- Free-body diagrams are your X-ray vision for forces}
\end{frame}

\begin{frame}
\frametitle{4.1 The Source Code of Motion}
\begin{block}{Nature's Operating System}
\textbf{Dynamics} is the study of how forces affect motion.
\end{block}

\pause
\vspace{0.3cm}

\textbf{Force:} A push or pull on an object.

\pause
\vspace{0.3cm}

Forces have:
\begin{itemize}
\item \textbf{Magnitude} - how strong \pause
\item \textbf{Direction} - which way
\end{itemize}

\pause
\begin{exampleblock}{The Mental Model}
Force is like an invisible hand pushing or pulling objects.
\end{exampleblock}
\note{[P0] [THE REVELATION] "Dynamics: study of how forces affect motion - from Greek word meaning power"\\\\
[P1] "Force: a push or pull on an object"\\\\
[P2] "Magnitude: how strong"\\\\
[P3] "Direction: which way"\\\\
[P4] [THE CONNECTION - Digital Archetype] "Think of force vectors in video games"\\\\
- Forces are vectors - arrows with length and direction}
\end{frame}

\begin{frame}
\frametitle{4.1 Combining Forces}
When multiple forces act on an object, they combine.

\pause
\vspace{0.3cm}

\begin{block}{The Universal Law}
\textbf{Net Force} = sum of all forces acting on an object
\end{block}

\pause
\vspace{0.3cm}

Forces in opposite directions have opposite signs:
\begin{itemize}
\item Right or up: positive (+) \pause
\item Left or down: negative (-)
\end{itemize}

\pause
\begin{alertblock}{Key Point}
Opposite forces can cancel each other out!
\end{alertblock}
\note{[P0] "When multiple forces act, they combine"\\\\
[P1] "Net force equals sum of all forces"\\\\
[P2] "Right or up: positive"\\\\
[P3] "Left or down: negative"\\\\
[P4] [THE CONFLICT] "Opposite forces can cancel - push right with 50 N, friction left with 50 N, net force is ZERO"\\\\
- Like tug of war when both sides pull equally hard}
\end{frame}

\begin{frame}
\frametitle{4.1 The Free-Body Diagram}
A \textbf{free-body diagram} shows:
\begin{itemize}
\item The object as a single point \pause
\item All external forces as arrows \pause
\item Arrow length = force magnitude
\end{itemize}

\pause
\vspace{0.3cm}

\begin{figure}
\centering
\includegraphics[width=0.5\textwidth,height=0.35\textheight,keepaspectratio]{phys11-dynamics-fig4-10.jpg}
\end{figure}

\pause
\begin{exampleblock}{The Tool}
Free-body diagrams are the first step to solving ANY force problem.
\end{exampleblock}
\note{[P0] "Free-body diagram: our most important tool"\\\\
[P1] "Object as single point"\\\\
[P2] "All external forces as arrows"\\\\
[P3] "Arrow length shows force magnitude"\\\\
[P4] [THE REVELATION] "This diagram shows tension up, weight down - equal magnitude, so net force is zero"\\\\
- You'll draw these for every problem this year}
\end{frame}

\begin{frame}
\frametitle{4.1 Balanced Forces}
\begin{center}
\includegraphics[width=0.5\textwidth,height=0.4\textheight,keepaspectratio]{phys11-dynamics-fig4-10.jpg}
\end{center}

\pause
Tension force (up) = Weight force (down)

\pause
\vspace{0.3cm}

Equal magnitude, opposite directions

\pause
\vspace{0.3cm}

\alert{Net force = ZERO}

\pause
\vspace{0.3cm}

Object hangs motionless.
\note{[P0] "Object hanging from rope - two forces"\\\\
[P1] "Tension force up equals weight force down"\\\\
[P2] "Equal magnitude, opposite directions"\\\\
[P3] "Net force equals ZERO"\\\\
[P4] "Object hangs motionless - no acceleration"\\\\
- This is balance - forces cancel perfectly}
\end{frame}

\section{Newton's First Law}

\begin{frame}
\frametitle{Learning Objectives}
\begin{block}{By the end of this section, you will be able to:}
\begin{itemize}
\item \textbf{4.2:} Describe Newton's first law and friction \pause
\item \textbf{4.2:} Discuss the relationship between mass and inertia
\end{itemize}
\end{block}
\note{[P0] "Two objectives for Newton's first law"\\\\
[P1] "Understand the law of inertia and what opposes motion"\\\\
- Second: understand why massive objects resist changes in motion}
\end{frame}

\begin{frame}
\frametitle{The Law of Laziness}
\begin{center}
\Large Objects don't like to change what they're doing.
\end{center}

\pause
\vspace{0.5cm}

\begin{block}{Universal Law I: Newton's First Law}
\begin{enumerate}
\item A body at rest stays at rest
\item A body in motion stays in motion at constant velocity
\end{enumerate}
\textbf{...unless} acted on by a net external force.
\end{block}

\note{[P0] [THE HOOK] "Objects are lazy - they don't like to change"\\\\
[P1] [THE REVELATION] "Newton's First Law: at rest stays at rest, in motion stays in motion"\\\\
- Unless net external force acts\\\\
[THE CONFLICT] But wait - sliding objects slow down on their own, right? Wrong!}
\end{frame}

\begin{frame}
\frametitle{The Intuition Trap}
\begin{alertblock}{What Your Brain Gets Wrong}
\textbf{Your brain says:} Moving objects naturally slow down and stop.\\[0.3cm]
\textbf{Reality:} A hidden force is slowing them down.
\end{alertblock}

\pause
\vspace{0.3cm}

That hidden force is \textbf{friction}.

\pause
\vspace{0.3cm}

Without friction, objects would glide forever at constant velocity.
\note{[P0] [THE CONFLICT] "Your brain says moving objects naturally slow down"\\\\
[P1] "Reality: a hidden force is slowing them - friction"\\\\
[P2] "Without friction, objects glide forever at constant velocity"\\\\
[THE HUMILITY] Our intuition evolved where friction exists everywhere\\\\
[THE WONDER] In space, no friction - objects coast forever}
\end{frame}

\begin{frame}
\frametitle{4.2 Friction: The Hidden Resistance}
\begin{figure}
\centering
\includegraphics[width=0.7\textwidth,height=0.45\textheight,keepaspectratio]{phys11-dynamics-fig4-3.jpg}
\end{figure}

\pause
\textbf{Friction} acts opposite to the direction of motion.

\pause
\vspace{0.3cm}

It's why things slow down on their own (seemingly).
\note{[P0] [Fig 4.3: For a box sliding] "Box sliding across floor"\\\\
[P1] "Friction acts opposite to direction of motion - always opposes"\\\\
[P2] "This is why things slow down - not natural, but friction"\\\\
[THE CONNECTION - Kinetic Archetype] "Athletes: why you slide farther on ice than on concrete"\\\\
- Air hockey table: friction almost zero, puck glides forever}
\end{frame}

\begin{frame}
\frametitle{4.2 Constant Velocity Means Zero Net Force}
A man pushes a box with +50 N.

\pause
\vspace{0.3cm}

The box moves at \textbf{constant velocity}.

\pause
\vspace{0.3cm}

What is the force of friction?

\pause
\vspace{0.3cm}

\begin{block}{The Answer}
Friction = -50 N
\end{block}

\pause
\vspace{0.3cm}

\textbf{Why?} Newton's first law says constant velocity means net force = 0.

\pause

+50 N + (-50 N) = 0
\note{[P0] "Man pushes box with 50 N"\\\\
[P1] "Box moves at constant velocity"\\\\
[P2] "What is friction force?"\\\\
[P3] [ANSWER] "Friction equals negative 50 N"\\\\
[P4] [ALGEBRA] "Why? Constant velocity means net force equals zero"\\\\
[P5] "Plus 50 plus negative 50 equals zero - forces cancel perfectly"\\\\
- This is the key: constant velocity is your clue}
\end{frame}

\begin{frame}
\frametitle{4.2 Inertia: The Resistance to Change}
\begin{block}{The Universal Law}
\textbf{Inertia} is the tendency to maintain your state of motion.
\end{block}

\pause
\vspace{0.3cm}

Greater mass = greater inertia

\pause
\vspace{0.3cm}

\begin{exampleblock}{In the Real World}
Changing the motion of a truck is harder than changing the motion of a skateboard.
\end{exampleblock}

\pause
\vspace{0.3cm}

\textbf{Mass} is the measure of inertia.
\note{[P0] [THE REVELATION] "Inertia: tendency to maintain your state of motion"\\\\
[P1] "Greater mass equals greater inertia"\\\\
[P2] [THE CONNECTION - Kinetic Archetype] "Truck vs skateboard - which is harder to stop?"\\\\
[P3] "Mass is the measure of inertia - measured in kilograms"\\\\
[THE HUMILITY] Mass and weight are different - mass doesn't change on moon}
\end{frame}

\begin{frame}
\frametitle{4.2 Mass vs Weight}
\begin{alertblock}{Civilian View vs. Reality}
\textbf{Civilian:} Mass and weight are the same thing.\\[0.3cm]
\textbf{Physicist:} Mass is matter. Weight is gravitational force.
\end{alertblock}

\pause
\vspace{0.3cm}

\textbf{Mass:} Amount of matter (same on Earth and Moon)

\pause
\vspace{0.3cm}

\textbf{Weight:} Gravitational force (changes on Moon)
\note{[P0] [THE CONFLICT] "Civilians use mass and weight interchangeably - wrong"\\\\
[P1] "Mass: amount of matter - same everywhere"\\\\
[P2] "Weight: gravitational force - changes with gravity"\\\\
[THE REVELATION] You have same mass on moon, but one-sixth the weight\\\\
- Your body didn't change, but gravity pulling on you did}
\end{frame}

\section{Newton's Second Law}

\begin{frame}
\frametitle{Learning Objectives}
\begin{block}{By the end of this section, you will be able to:}
\begin{itemize}
\item \textbf{4.3:} Describe Newton's second law, both verbally and mathematically \pause
\item \textbf{4.3:} Use Newton's second law to solve problems
\end{itemize}
\end{block}
\note{[P0] "Two objectives for Newton's second law"\\\\
[P1] "Understand the equation that connects force, mass, and acceleration"\\\\
- Second: use it to calculate real-world motion}
\end{frame}

\begin{frame}
\frametitle{The Universal Pushback}
\begin{center}
\Large What happens when net force is NOT zero?
\end{center}

\pause
\vspace{0.5cm}

The object \textbf{accelerates}.

\pause
\vspace{0.3cm}

\begin{block}{Universal Law II: Newton's Second Law}
\begin{center}
\Large $\boxed{\vec{F}_{\text{net}} = m\vec{a}}$
\end{center}
Net force equals mass times acceleration.
\end{block}
\note{[P0] "What happens when net force is NOT zero?"\\\\
[P1] "The object accelerates - changes velocity"\\\\
[P2] [THE REVELATION] "F-net equals m a - the most important equation in physics"\\\\
[THE WONDER] This equation predicts every motion in the universe\\\\
[THE CONNECTION - Digital Archetype] "Game physics engines use this every frame"}
\end{frame}

\begin{frame}
\frametitle{4.3 Reading the Equation}
$$\vec{F}_{\text{net}} = m\vec{a}$$

\pause
\vspace{0.3cm}

\textbf{Acceleration is directly proportional to force:}

Double the force $\rightarrow$ double the acceleration

\pause
\vspace{0.3cm}

\textbf{Acceleration is inversely proportional to mass:}

Double the mass $\rightarrow$ half the acceleration

\pause
\vspace{0.3cm}

\begin{alertblock}{Key Insight}
Same force on different masses produces different accelerations!
\end{alertblock}
\note{[P0] [ALGEBRA] "F-net equals m a"\\\\
[P1] "Double the force, double the acceleration - directly proportional"\\\\
[P2] "Double the mass, half the acceleration - inversely proportional"\\\\
[P3] [THE CONFLICT] "Push basketball: big acceleration. Push car: tiny acceleration. Same force, different mass"\\\\
- This explains why massive objects resist changes in motion}
\end{frame}

\begin{frame}
\frametitle{4.3 Same Force, Different Results}
\begin{figure}
\centering
\includegraphics[width=0.8\textwidth,height=0.55\textheight,keepaspectratio]{phys11-dynamics-fig4-4.jpg}
\end{figure}

\pause
Same force, different masses, different accelerations.
\note{[P0] [Fig 4.4: (a) The wagon and] "Boy pushes basketball with same force as car"\\\\
[P1] "Same force, different masses, different accelerations"\\\\
[THE CONNECTION - Kinetic Archetype] "Why you can throw a ball farther than a boulder"\\\\
- Free-body diagrams are identical, but results are vastly different}
\end{frame}

\begin{frame}
\frametitle{4.3 The Newton}
The SI unit of force is the \textbf{newton} (N).

\pause
\vspace{0.3cm}

\begin{block}{Definition}
$$1 \text{ N} = 1 \text{ kg} \cdot \frac{\text{m}}{\text{s}^2}$$
\end{block}

\pause
\vspace{0.3cm}

A 1-kg mass accelerated at 1 m/s² requires 1 N of force.

\pause
\vspace{0.3cm}

In the US: $1 \text{ N} = 0.225 \text{ lb}$
\note{[P0] "Unit of force: newton, abbreviated N"\\\\
[P1] "1 N equals 1 kg times 1 m per s squared"\\\\
[P2] "A 1-kg mass accelerated at 1 m per s squared requires 1 N"\\\\
[P3] "In US: 1 N equals 0.225 pounds"\\\\
- Named after Isaac Newton, who discovered these laws}
\end{frame}

\begin{frame}
\frametitle{4.3 Weight: A Special Force}
\textbf{Weight} is the gravitational force on an object.

\pause
\vspace{0.3cm}

From Newton's second law:
$$\vec{F}_{\text{net}} = m\vec{a}$$

\pause
For a falling object: $\vec{F}_{\text{net}} = \vec{W}$ and $\vec{a} = \vec{g}$

\pause
\vspace{0.3cm}

\begin{block}{The Universal Law}
\begin{center}
\Large $\boxed{W = mg}$
\end{center}
Weight equals mass times gravitational acceleration.
\end{block}
\note{[P0] "Weight is gravitational force on an object"\\\\
[P1] [ALGEBRA] "From Newton's second law: F-net equals m a"\\\\
[P2] "For falling object: F-net equals W and a equals g"\\\\
[P3] [THE REVELATION] "W equals m g - weight equals mass times g"\\\\
- On Earth, g equals 9.8 m per s squared\\\\
- 1-kg object weighs 9.8 N on Earth}
\end{frame}

\begin{frame}
\frametitle{4.3 Weight Changes, Mass Doesn't}
On Earth: $g = 9.8 \text{ m/s}^2$

\pause
On the Moon: $g = 1.67 \text{ m/s}^2$

\pause
\vspace{0.3cm}

\begin{exampleblock}{Example: 1.0-kg mass}
\begin{itemize}
\item Earth: $W = (1.0)(9.8) = 9.8$ N \pause
\item Moon: $W = (1.0)(1.67) = 1.7$ N
\end{itemize}
\end{exampleblock}

\pause
\vspace{0.3cm}

\textbf{Same mass, different weight!}
\note{[P0] "On Earth, g equals 9.8 m per s squared"\\\\
[P1] "On Moon, g equals 1.67 m per s squared"\\\\
[P2] "1-kg mass on Earth: W equals 9.8 N"\\\\
[P3] "Same mass on Moon: W equals 1.7 N"\\\\
[P4] [THE WONDER] "Same object, different weight - mass is constant, weight depends on gravity"\\\\
- Astronauts weigh less on moon but have same mass}
\end{frame}

\begin{frame}
\frametitle{Attempt: Decoding Lawn Mower Motion}
\begin{exampleblock}{The Challenge (3 min, silent)}
Net external force on a lawn mower is 51 N parallel to the ground.\\
Mass of mower is 24 kg.

\vspace{0.3cm}

\textbf{Given:}
\begin{itemize}
\item $F_{\text{net}} = 51$ N
\item $m = 24$ kg
\end{itemize}

\textbf{Find:} Acceleration $a$

\vspace{0.3cm}

\textit{Can you predict its acceleration? Work silently.}
\end{exampleblock}
\note{[THE CHALLENGE] Can they decode motion like Newton?\\\\
[SAY] "Try this on your own. It's okay to get stuck."\\\\
[TIMING] 3-4 min SILENT individual work\\\\
[CIRCULATE] Note who uses F equals m a vs a equals F over m\\\\
[WATCH FOR] Students forgetting to rearrange equation\\\\
[DON'T HELP] Let them struggle - learning happens in Compare}
\end{frame}

\begin{frame}
\frametitle{Compare: Lawn Mower Strategy}
\textbf{Turn and talk (2 min):}

\vspace{0.3cm}

\begin{enumerate}
\item What equation did you start with?
\item How did you rearrange it to solve for acceleration?
\item What units should your answer have?
\end{enumerate}

\vspace{0.5cm}

\pause
\alert{Name wheel:} One pair share your approach (not your answer).
\note{[TIMING] 2-3 min pair discussion\\\\
[CIRCULATE] Listen for common approaches\\\\
[CHECK] Name wheel: call a pair to share approach\\\\
[EXPECTED APPROACH] Start with F equals m a, rearrange to a equals F over m\\\\
[COMMON ERROR] Multiplying instead of dividing}
\end{frame}

\begin{frame}
\frametitle{Reveal: The Math of Acceleration}
\textbf{Self-correct in a different color:}

\vspace{0.3cm}

\textbf{E - Equation:} $F_{\text{net}} = ma$

\pause
\vspace{0.2cm}

\textbf{Rearrange:} $a = \frac{F_{\text{net}}}{m}$

\pause
\vspace{0.2cm}

\textbf{S - Substitute:} $a = \frac{51 \text{ N}}{24 \text{ kg}}$

\pause
\vspace{0.2cm}

$$\boxed{a = 2.1 \text{ m/s}^2}$$

\pause
\textbf{Check:} Speed increases by 2.1 m/s every second. Reasonable for a person pushing!
\note{[P0] "Self-correct in a different color"\\\\
[P1] [ALGEBRA] "Equation: F-net equals m a"\\\\
[P2] "Rearrange: a equals F-net over m"\\\\
[P3] "Substitute: a equals 51 N over 24 kg"\\\\
[P4] [ANSWER] "a equals 2.1 m per s squared"\\\\
[P5] "Check: speed increases by 2.1 m/s every second - reasonable"\\\\
[THE WONDER] You just predicted motion using Newton's law}
\end{frame}

\section{Newton's Third Law}

\begin{frame}
\frametitle{Learning Objectives}
\begin{block}{By the end of this section, you will be able to:}
\begin{itemize}
\item \textbf{4.4:} Describe Newton's third law, both verbally and mathematically \pause
\item \textbf{4.4:} Use Newton's third law to solve problems
\end{itemize}
\end{block}
\note{[P0] "Two objectives for Newton's third law"\\\\
[P1] "Understand action-reaction pairs and how they work"\\\\
- Second: use this to analyze real-world systems}
\end{frame}

\begin{frame}
\frametitle{The Great Exchange}
\begin{center}
\Large Why does punching a wall hurt \textit{your} hand?
\end{center}

\pause
\vspace{0.5cm}

\alert{The wall punched you back.}

\pause
\vspace{0.3cm}

\begin{exampleblock}{The Mental Model}
You cannot touch something without being touched back.
\end{exampleblock}
\note{[P0] [THE HOOK] "Why does punching a wall hurt your hand?"\\\\
[P1] "The wall punched you back - let that sink in"\\\\
[P2] [THE REVELATION] "You cannot touch without being touched back"\\\\
[THE CONNECTION - Kinetic Archetype] "Skaters: how do you push off? You push ice backward, ice pushes you forward"\\\\
- Every interaction is an exchange}
\end{frame}

\begin{frame}
\frametitle{Universal Law III: Newton's Third Law}
\begin{block}{The Law of Action and Reaction}
\begin{center}
\Large $\boxed{\vec{F}_{A \to B} = -\vec{F}_{B \to A}}$
\end{center}
When object A exerts a force on object B,\\
object B exerts an equal and opposite force on object A.
\end{block}

\pause
\vspace{0.3cm}

Forces always come in \textbf{action-reaction pairs}.

\pause
\vspace{0.3cm}

\alert{Equal magnitude, opposite direction.}
\note{[P0] [THE REVELATION] "Newton's Third Law: action-reaction pairs"\\\\
[P1] "Forces always come in pairs - you can't have one without the other"\\\\
[P2] "Equal magnitude, opposite direction"\\\\
[THE WONDER] Without this law, we couldn't move - interaction requires exchange\\\\
- ALWAYS true, no exceptions ever recorded}
\end{frame}

\begin{frame}
\frametitle{4.4 Swimmer Pushing Off Wall}
\begin{figure}
\centering
\includegraphics[width=0.7\textwidth,height=0.5\textheight,keepaspectratio]{phys11-dynamics-fig4-5.jpg}
\end{figure}

\pause
Swimmer pushes wall backward $\rightarrow$ Wall pushes swimmer forward

\pause
\vspace{0.3cm}

Same force, opposite directions.
\note{[P0] [Fig 4.5: The same force exerted] "Swimmer pushes off pool wall"\\\\
[P1] "Swimmer pushes wall backward, wall pushes swimmer forward"\\\\
[P2] "Same force magnitude, opposite directions"\\\\
[THE CONNECTION - Kinetic Archetype] "Every time you push off to accelerate, you're using third law"\\\\
- Push in opposite direction of where you want to go}
\end{frame}

\begin{frame}
\frametitle{The Paradox}
\begin{alertblock}{Civilian View vs. Reality}
\textbf{Civilian:} "The truck hits the bug harder than the bug hits the truck."\\[0.3cm]
\textbf{Physicist:} "Same force. Different acceleration."
\end{alertblock}

\pause
\vspace{0.3cm}

$F = ma$

\pause
\vspace{0.3cm}

Bug has tiny mass $\rightarrow$ huge acceleration $\rightarrow$ splat

\pause
\vspace{0.3cm}

Truck has huge mass $\rightarrow$ tiny acceleration $\rightarrow$ barely notices
\note{[P0] [THE CONFLICT] "Civilians think truck hits bug harder - feels wrong but it's true"\\\\
[P1] "F equals m a - rearrange to a equals F over m"\\\\
[P2] "Bug: tiny mass, so huge acceleration - splat"\\\\
[P3] "Truck: huge mass, so tiny acceleration - barely notices"\\\\
[THE HUMILITY] This feels wrong. Our intuition lies. Math tells truth.\\\\
- We see damage and assume unequal force}
\end{frame}

\begin{frame}
\frametitle{4.4 The Normal Force}
When an object rests on a surface:

\pause
\vspace{0.3cm}

\textbf{Weight} pulls down

\pause
\vspace{0.3cm}

\textbf{Normal force} pushes up (perpendicular to surface)

\pause
\vspace{0.3cm}

\begin{block}{For horizontal surface}
$$N = mg$$
\end{block}

\pause
\vspace{0.3cm}

Equal magnitude, opposite direction $\rightarrow$ net force = 0
\note{[P0] "Object rests on surface - what forces act?"\\\\
[P1] "Weight pulls down - gravitational force"\\\\
[P2] "Normal force pushes up - surface supporting object"\\\\
[P3] "For horizontal surface: N equals m g"\\\\
[P4] "Equal magnitude, opposite direction, so net force equals zero"\\\\
- This is why objects don't fall through floors}
\end{frame}

\begin{frame}
\frametitle{4.4 Tension in a Rope}
\begin{figure}
\centering
\includegraphics[width=0.5\textwidth,height=0.35\textheight,keepaspectratio]{phys11-dynamics-fig4-3.jpg}
\end{figure}

\pause
\textbf{Tension} is the pulling force along a connector (rope, string, cable).

\pause
\vspace{0.3cm}

For a stationary mass: $T = W = mg$

\pause
\vspace{0.3cm}

Rope pulls up on mass, mass pulls down on rope.
\note{[P0][Fig 4.3: For a box sliding]  "Person holding mass on rope"\\\\
[P1] "Tension: pulling force along connector - acts parallel to length"\\\\
[P2] "For stationary mass: T equals W equals m g"\\\\
[P3] "Rope pulls up on mass, mass pulls down on rope - action-reaction pair"\\\\
- Tension must equal weight for object to remain stationary}
\end{frame}

\begin{frame}
\frametitle{4.4 Thrust: The Rocket Force}
Rockets move by expelling gas backward at high velocity.

\pause
\vspace{0.3cm}

Rocket pushes gas backward $\rightarrow$ Gas pushes rocket forward

\pause
\vspace{0.3cm}

This forward force is called \textbf{thrust}.

\pause
\vspace{0.3cm}

\begin{alertblock}{Misconception}
Rockets don't push on the ground or air.\\
They push on the gas they expel!
\end{alertblock}
\note{[P0] [Fig 4.3: For a box sliding] "Rockets move by expelling gas backward"\\\\
[P1] "Rocket pushes gas backward, gas pushes rocket forward"\\\\
[P2] "This forward force is thrust"\\\\
[P3] [THE CONFLICT] "Misconception: rockets push on ground or air - WRONG"\\\\
[THE REVELATION] Rockets push on gas they expel - work better in vacuum\\\\
[THE WONDER] Same principle as swimmer pushing off wall}
\end{frame}

\begin{frame}
\frametitle{Attempt: Equipment Cart}
\begin{exampleblock}{The Challenge (3 min, silent)}
A teacher pushes a cart. Her foot applies 150 N backward on the floor.\\
Friction opposing motion is 24.0 N.

\vspace{0.3cm}

\textbf{Given:}
\begin{itemize}
\item $F_{\text{floor}} = 150$ N (Newton's 3rd law)
\item $f = 24.0$ N (friction)
\item Total mass: $m = 65.0 + 12.0 + 7.0 = 84.0$ kg
\end{itemize}

\textbf{Find:} Acceleration $a$

\vspace{0.3cm}

\textit{Can you decode this system? Work silently.}
\end{exampleblock}
\note{[THE CHALLENGE] Can they identify the system and find net force?\\\\
[SAY] "Try this on your own. It's okay to get stuck."\\\\
[TIMING] 3-4 min SILENT individual work\\\\
[CIRCULATE] Note who identifies floor force correctly\\\\
[WATCH FOR] Students including internal forces\\\\
[DON'T HELP] Let them struggle - Compare reveals strategy}
\end{frame}

\begin{frame}
\frametitle{Compare: Cart Strategy}
\textbf{Turn and talk (2 min):}

\vspace{0.3cm}

\begin{enumerate}
\item What forces act on the system?
\item How did you find net force?
\item What equation did you use to find acceleration?
\end{enumerate}

\vspace{0.5cm}

\pause
\alert{Name wheel:} One pair share your approach (not your answer).
\note{[TIMING] 2-3 min pair discussion\\\\
[CIRCULATE] Listen for common approaches\\\\
[CHECK] Name wheel: call a pair to share approach\\\\
[EXPECTED APPROACH] Floor pushes forward 150 N, friction opposes 24 N, net force equals 126 N, then a equals F-net over m\\\\
[COMMON ERROR] Forgetting friction is negative, or including teacher's force on cart (internal)}
\end{frame}

\begin{frame}
\frametitle{Reveal: The Math of Systems}
\textbf{Self-correct in a different color:}

\vspace{0.3cm}

\textbf{Step 1:} Find net force
$$F_{\text{net}} = F_{\text{floor}} - f = 150 - 24.0 = 126 \text{ N}$$

\pause
\vspace{0.2cm}

\textbf{Step 2:} Find total mass
$$m = 65.0 + 12.0 + 7.0 = 84.0 \text{ kg}$$

\pause
\vspace{0.2cm}

\textbf{Step 3:} Calculate acceleration
$$a = \frac{F_{\text{net}}}{m} = \frac{126}{84.0} = \boxed{1.5 \text{ m/s}^2}$$

\pause
\textbf{Check:} Speed increases by 1.5 m/s every second. Reasonable!
\note{[P0] "Self-correct in a different color"\\\\
[P1] "Net force: 150 N minus 24 N equals 126 N forward"\\\\
[P2] "Total mass: 65 plus 12 plus 7 equals 84 kg"\\\\
[P3] [ALGEBRA] "a equals F-net over m equals 126 over 84 equals 1.5 m per s squared"\\\\
[P4] [ANSWER] "1.5 m per s squared - reasonable for pushing cart"\\\\
[THE WONDER] You identified the system, found external forces, calculated motion - that's physics}
\end{frame}

\section{Summary}

\begin{frame}
\frametitle{What You Now Know}
\begin{block}{The Three Laws That Run Reality}
\begin{enumerate}
\item \textbf{Law of Inertia:} Objects maintain their motion unless forced to change \pause
\item \textbf{Law of Acceleration:} $\vec{F}_{\text{net}} = m\vec{a}$ connects force, mass, and motion \pause
\item \textbf{Law of Action-Reaction:} Forces always come in equal and opposite pairs
\end{enumerate}
\end{block}

\pause
\vspace{0.3cm}

These three laws explain ALL motion in the universe.
\note{[P0] "Three laws that run reality"\\\\
[P1] "First: objects maintain motion unless forced to change"\\\\
[P2] "Second: F equals m a - the engine of prediction"\\\\
[P3] "Third: forces always come in pairs"\\\\
[THE WONDER] These three laws explain dolphins, rockets, galaxies - everything\\\\
- Newton figured this out 350 years ago - still perfect today}
\end{frame}

\begin{frame}
\frametitle{Key Equations}
\begin{align}
\text{Newton's Second Law:} \quad \vec{F}_{\text{net}} &= m\vec{a} \\
\text{Weight:} \quad W &= mg \\
\text{Friction:} \quad f &= \mu N \\
\text{Normal Force (horizontal):} \quad N &= mg
\end{align}
\note{- Four key equations you'll use constantly\\\\
- Newton's second law: the core engine\\\\
- Weight: gravitational force on object\\\\
- Friction: opposes motion, depends on surfaces\\\\
- Normal force: surface supporting object\\\\
- Questions before we end?}
\end{frame}

\begin{frame}
\frametitle{Homework}
\begin{center}
\Large
Complete the assigned problems\\[0.3cm]
posted on the LMS
\end{center}
\note{[SAY] "Homework is posted on the LMS"\\\\
[TIMING] Due date: check LMS\\\\
[CHECK] Questions before we end?\\\\
[TRANSITION] Next class: applying these laws to real-world problems}
\end{frame}

\end{document}
