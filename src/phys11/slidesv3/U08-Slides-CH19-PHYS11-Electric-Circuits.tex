\documentclass{beamer}
\usepackage{../../../shared/templates/ds9_theme}
\usepackage[overridenote]{pdfpc}
\graphicspath{{../images/}{../../shared/images/}}

\title[Electric Circuits]{PHYS11 CH:19 The Invisible River of Energy}
\subtitle{How Charge Flows Through the World}
\author[Mr. Gullo]{Mr. Gullo}
\date[December 2025]{December 2025}

\begin{document}

\frame{\titlepage
\note{[THE HOOK] Today we harness the invisible river that powers civilization\\\\
- Same physics in your phone charger AND the power grid\\\\
- Four revelations: Ohm's law, series circuits, parallel circuits, electric power\\\\
[THE WONDER] By end of class, you'll decode every circuit around you\\\\
- This is how humanity tamed electricity}
}

\begin{frame}
\frametitle{Outline}
\tableofcontents
\end{frame}

\section{Introduction}

\begin{frame}
\frametitle{The Mystery}
\begin{center}
\Large What if you could control\\
\textit{an invisible river of charge?}
\end{center}

\pause
\vspace{0.5cm}
From nerve impulses in your brain to hydroelectric dams sending power across continents...

\pause
\vspace{0.3cm}
\alert{Same laws guide the flow.}
\note{[P0] "What if you could control an invisible river of charge?"\\\\
[P1] "From nerve impulses in your brain to hydroelectric dams sending power across continents"\\\\
[P2] [THE WONDER] "Same laws guide the flow - we call it electric current"\\\\
[THE CONNECTION - Digital Archetype] Every device you own runs on these principles}
\end{frame}

\begin{frame}
\frametitle{The Invisible River}
\begin{figure}
\centering
\includegraphics[width=0.8\textwidth,height=0.6\textheight,keepaspectratio]{phys11-circuits-fig19-1.jpg}
\end{figure}

\pause
\begin{exampleblock}{The Mental Model}
Electric current is like water flowing through pipes - voltage is the pressure, resistance is the friction.
\end{exampleblock}
\note{[P0][Fig 19.1: Hydroelectric facility with flowing water] "This analogy is pedagogically powerful because students already understand water flow intuitively - leverage this prior knowledge to scaffold abstract electric current concept. Water pressure maps to voltage, flow rate to current, pipe narrowness to resistance. Ask: 'What pushes water through pipes?' then connect to voltage pushing charge."\\\\
[P1] [THE REVELATION] "Electric current is like water flowing through pipes"\\\\
- Voltage is pressure pushing charges\\\\
- Resistance slows the flow\\\\
[THE WONDER] We've harnessed this to improve quality of life}
\end{frame}

\section{19.1 Ohm's Law}

\begin{frame}
\frametitle{Learning Objectives}
\begin{block}{By the end of this section, you will be able to:}
\begin{itemize}
\item \textbf{19.1:} Describe how current is related to charge and time \pause
\item \textbf{19.1:} Distinguish between direct and alternating current \pause
\item \textbf{19.1:} Define resistance and verbally describe Ohm's law \pause
\item \textbf{19.1:} Calculate current and solve problems involving Ohm's law
\end{itemize}
\end{block}
\note{[P0] "Four objectives for Ohm's law"\\\\
[P1] "First: understand current as moving charge"\\\\
[P2] "Second: DC vs AC current"\\\\
[P3] "Third: define resistance and Ohm's law"\\\\
[P4] "Fourth: solve circuit problems"\\\\
- This is foundational for all circuits}
\end{frame}

\begin{frame}
\frametitle{19.1 The Flow of Charge}
\begin{block}{The Definition}
\begin{center}
\Large $\boxed{I = \frac{\Delta Q}{\Delta t}}$
\end{center}
Current equals charge per unit time. Measured in amperes (A).
\end{block}

\pause
\vspace{0.3cm}

One ampere = one coulomb per second: $1 \text{ A} = 1 \text{ C/s}$

\pause
\begin{exampleblock}{The Mental Model}
If 5 coulombs flow past a point in 1 second, the current is 5 A.
\end{exampleblock}
\note{[P0] [THE REVELATION] "Current I equals charge Q over time t - rate of charge flow"\\\\
[P1] "One ampere equals one coulomb per second"\\\\
[P2] "If 5 coulombs flow past a point in 1 second, current is 5 A"\\\\
[THE CONNECTION - Harmonic Archetype] Like measuring water flow in liters per second}
\end{frame}

\begin{frame}
\frametitle{19.1 The Direction Paradox}
\begin{figure}
\centering
\includegraphics[width=0.7\textwidth,height=0.45\textheight,keepaspectratio]{phys11-circuits-fig19-2.jpg}
\end{figure}

\pause
\begin{alertblock}{The Paradox}
\textbf{Reality:} Electrons flow from negative to positive.\\
\textbf{Convention:} Current flows from positive to negative.
\end{alertblock}
\note{[P0][Fig 19.2: Battery showing electron flow from negative to positive terminal] "Students find this confusing - address misconception explicitly. The diagram shows physical reality (electron movement) vs convention (current direction). Emphasize: both perspectives work mathematically, but we must be consistent. This builds critical thinking about scientific conventions vs physical reality."\\\\
[P1] [THE CONFLICT] "Electrons carry negative charge, so they flow from negative terminal to positive"\\\\
- But we define conventional current as positive charge flow\\\\
- Blame Benjamin Franklin - he guessed wrong in 1750s\\\\
[THE HUMILITY] We're stuck with this backwards convention - science admits mistakes but keeps useful conventions}
\end{frame}

\begin{frame}
\frametitle{19.1 Direct Current vs Alternating Current}
\begin{columns}[T]
\column{0.48\textwidth}
\textbf{Direct Current (DC)}
\begin{itemize}
\item Flows one direction
\item Constant over time
\item Batteries provide DC
\end{itemize}

\pause
\column{0.48\textwidth}
\textbf{Alternating Current (AC)}
\begin{itemize}
\item Direction alternates
\item Smoothly reverses
\item Wall sockets provide AC
\end{itemize}
\end{columns}

\pause
\vspace{0.3cm}

\begin{figure}
\centering
\includegraphics[width=0.6\textwidth,height=0.3\textheight,keepaspectratio]{phys11-circuits-fig19-6.jpg}
\end{figure}
\note{[P0][Fig 19.6: Graph showing sinusoidal AC current reversing direction periodically] "Graph literacy critical here - students must read temporal patterns. Trace the curve with your finger showing how current magnitude AND direction change. Ask: 'Why would we use AC instead of DC?' Lead to transmission efficiency. This graph appears in every electrical engineering context."\\\\
[P1] "DC flows one direction - batteries, flashlights, cars"\\\\
[P2] "AC alternates back and forth - wall sockets, power grid"\\\\
- AC graph shows current reversing smoothly at 60 Hz in North America\\\\
- Most devices use transformers to convert AC to DC\\\\
[THE CONNECTION - Digital Archetype] Your laptop charger is really an AC-to-DC converter - look for the heavy black box}
\end{frame}

\begin{frame}
\frametitle{19.1 The Universal Law of Resistance}
\begin{block}{Ohm's Law}
\begin{center}
\Large $\boxed{V = IR}$
\end{center}
Voltage equals current times resistance. The pushback equation.
\end{block}

\pause
\vspace{0.3cm}

\textbf{Units:} Resistance measured in ohms ($\Omega$), where $1 \Omega = 1 \text{ V/A}$

\pause
\begin{exampleblock}{The Mental Model}
Resistance is friction for electrons - it converts electrical energy to heat.
\end{exampleblock}
\note{[P0] [THE REVELATION] "V equals I R - discovered by Georg Ohm in 1827"\\\\
[P1] "Resistance measured in ohms - Greek letter omega"\\\\
[P2] "Resistance is friction for electrons - creates heat"\\\\
[THE WONDER] This equation works for most materials at normal temperatures\\\\
- Called ohmic materials}
\end{frame}

\begin{frame}
\frametitle{Attempt: Lightning Strike Current}
\begin{exampleblock}{The Challenge (3 min, silent)}
A lightning strike transfers $10^{20}$ electrons from cloud to ground in 2 ms.

\vspace{0.3cm}

\textbf{Given:}
\begin{itemize}
\item $n = 10^{20}$ electrons
\item $e = -1.60 \times 10^{-19}$ C
\item $\Delta t = 2 \times 10^{-3}$ s
\end{itemize}

\textbf{Find:} Average current $I$

\vspace{0.3cm}

\textit{Can you calculate the current in a lightning bolt? Work silently.}
\end{exampleblock}
\note{[THE CHALLENGE] Can they calculate current in nature's most dramatic circuit?\\\\
[SAY] "Try this on your own - it's okay to get stuck"\\\\
[TIMING] 3-4 min SILENT individual work\\\\
[CIRCULATE] Note who calculates total charge first, who's stuck\\\\
[WATCH FOR] Students forgetting negative sign on electron charge\\\\
[DON'T HELP] Let them struggle - learning happens in Compare}
\end{frame}

\begin{frame}
\frametitle{Compare: Lightning Strike}
\textbf{Turn and talk (2 min):}

\vspace{0.3cm}

\begin{enumerate}
\item How did you find the total charge?
\item What formula did you use for current?
\item What sign did you get for your answer?
\end{enumerate}

\vspace{0.5cm}

\pause
\alert{Name wheel:} One pair share your approach (not your answer).
\note{[TIMING] 2-3 min pair discussion\\\\
[CIRCULATE] Listen for common approaches\\\\
[CHECK] Name wheel: call a pair to share approach\\\\
[EXPECTED APPROACH] Total charge Q equals n times e, then I equals Q over t\\\\
[COMMON ERROR] Forgetting that current is negative because electrons carry negative charge}
\end{frame}

\begin{frame}
\frametitle{Reveal: The Power of Lightning}
\textbf{Self-correct in a different color:}

\vspace{0.3cm}

\textbf{Step 1:} Total charge: $\Delta Q = ne = (10^{20})(-1.60 \times 10^{-19} \text{ C}) = -16.0 \text{ C}$

\pause
\vspace{0.2cm}

\textbf{Step 2:} Current: $I = \frac{\Delta Q}{\Delta t}$

\pause
\vspace{0.2cm}

$$I = \frac{-16.0 \text{ C}}{2 \times 10^{-3} \text{ s}} = \boxed{-8 \text{ kA}}$$

\pause
\textbf{Check:} Negative sign means electrons flow down. Conventional current flows up!
\note{[P0] "Self-correct in a different color"\\\\
[P1] [ALGEBRA] "Total charge equals n times e - 10 to the 20 times negative 1.6 times 10 to the negative 19"\\\\
[P2] "Current equals charge over time"\\\\
[P3] [ANSWER] "Negative 8 kiloamps - 8000 amperes!"\\\\
[P4] "Negative sign: electrons flow down, but positive current flows up"\\\\
[THE WONDER] You just calculated the current in a lightning bolt - same equation NASA uses}
\end{frame}

\begin{frame}
\frametitle{Attempt: Headlight Resistance}
\begin{exampleblock}{The Challenge (3 min, silent)}
An automobile headlight has 2.50 A flowing through it when 12.0 V is applied.

\vspace{0.3cm}

\textbf{Given:}
\begin{itemize}
\item $I = 2.50$ A
\item $V = 12.0$ V
\end{itemize}

\textbf{Find:} Resistance $R$ of the headlight

\vspace{0.3cm}

\textit{Can you find the resistance using Ohm's law?}
\end{exampleblock}
\note{[THE CHALLENGE] Can they rearrange Ohm's law?\\\\
[SAY] "Use V equals I R - solve for R"\\\\
[TIMING] 3-4 min SILENT work\\\\
[CIRCULATE] Note who rearranges correctly\\\\
[WATCH FOR] Students dividing instead of multiplying\\\\
[DON'T HELP] Productive struggle}
\end{frame}

\begin{frame}
\frametitle{Compare: Resistance}
\textbf{Turn and talk (2 min):}

\vspace{0.3cm}

\begin{enumerate}
\item What equation did you start with?
\item How did you solve for R?
\item What units did you get?
\end{enumerate}

\vspace{0.5cm}

\pause
\alert{Name wheel:} One pair share your approach.
\note{[TIMING] 2-3 min pair discussion\\\\
[CIRCULATE] Listen for algebra steps\\\\
[CHECK] Name wheel: call a pair\\\\
[EXPECTED APPROACH] V equals I R, so R equals V over I\\\\
[COMMON ERROR] Multiplying instead of dividing}
\end{frame}

\begin{frame}
\frametitle{Reveal: Headlight Resistance}
\textbf{Self-correct in a different color:}

\vspace{0.3cm}

\textbf{Start with Ohm's law:} $V = IR$

\pause
\vspace{0.2cm}

\textbf{Solve for R:} $R = \frac{V}{I}$

\pause
\vspace{0.2cm}

\textbf{Substitute:} $R = \frac{12.0 \text{ V}}{2.50 \text{ A}}$

\pause
\vspace{0.2cm}

$$\boxed{R = 4.8 \Omega}$$

\pause
\textbf{Check:} 4.8 ohms - relatively small resistance for a headlight.
\note{[P0] "Self-correct in a different color"\\\\
[P1] [ALGEBRA] "Start with V equals I R"\\\\
[P2] "Solve for R - divide both sides by I"\\\\
[P3] "R equals 12 volts over 2.5 amperes"\\\\
[P4] [ANSWER] "4.8 ohms - small resistance"\\\\
[THE WONDER] Same law in your headlight as in power plants across the world}
\end{frame}

\section{19.2 Series Circuits}

\begin{frame}
\frametitle{Learning Objectives}
\begin{block}{By the end of this section, you will be able to:}
\begin{itemize}
\item \textbf{19.2:} Interpret circuit diagrams and diagram basic circuit elements \pause
\item \textbf{19.2:} Calculate equivalent resistance of resistors in series
\end{itemize}
\end{block}
\note{[P0] "Two objectives for series circuits"\\\\
[P1] "First: read and draw circuit diagrams"\\\\
[P2] "Second: calculate equivalent resistance in series"\\\\
- Foundation for analyzing real circuits}
\end{frame}

\begin{frame}
\frametitle{19.2 The Language of Circuits}
\begin{figure}
\centering
\includegraphics[width=0.7\textwidth,height=0.5\textheight,keepaspectratio]{phys11-circuits-fig19-9.jpg}
\end{figure}

\pause
\textbf{Circuit symbols:}
\begin{itemize}
\item Battery: long line = positive, short line = negative
\item Resistor: zigzag element
\item Wire: perfect conductor (no resistance)
\item Ground: reference point (voltage = 0)
\end{itemize}
\note{[P0][Fig 19.9: Circuit diagram paired with analogous water circuit] "Symbolic literacy is non-negotiable for circuit analysis - this is their 'Rosetta Stone' moment. Dual representation (schematic + physical analogy) helps students who struggle with pure abstraction. Point out: battery = pump (energy source), resistor = sand filter (energy dissipation), wire = frictionless pipe (ideal conductor). Build fluency by having students verbally describe each symbol."\\\\
[P1] "Learn to read the schematic language"\\\\
- Battery: longer line is positive terminal\\\\
- Resistor: zigzag shows opposition to flow\\\\
- Wires assumed perfect - no voltage drop\\\\
[THE CONNECTION - Digital Archetype] Like learning to read code - symbols represent complex physical processes}
\end{frame}

\begin{frame}
\frametitle{19.2 Resistors in Series}
\begin{figure}
\centering
\includegraphics[width=0.7\textwidth,height=0.45\textheight,keepaspectratio]{phys11-circuits-fig19-14.jpg}
\end{figure}

\pause
\begin{block}{The Rule for Series}
\begin{center}
\Large $\boxed{R_{\text{equiv}} = R_1 + R_2 + R_3}$
\end{center}
Series resistances add. One path, obstacles accumulate.
\end{block}
\note{[P0][Fig 19.14: Left shows three resistors in series, right shows equivalent single resistor] "Students struggle visualizing how multiple components reduce to one - this side-by-side comparison builds mental model of equivalence. Point out same current flows through all three, forcing voltage drops to add."\\\\
[P1] [THE REVELATION] "Series resistances add - R-equiv equals R-1 plus R-2 plus R-3"\\\\
- Same current flows through all\\\\
- Voltage drops across each resistor\\\\
[THE CONNECTION - Kinetic Archetype] Like running through multiple obstacles in a row - each slows you down, total resistance accumulates}
\end{frame}

\begin{frame}
\frametitle{19.2 The Voltage Loop}
\textbf{Key insight:} Going around a complete loop, voltage changes sum to zero.

\pause
\vspace{0.3cm}

\begin{exampleblock}{The Mental Model}
Like hiking in hills - total elevation gained equals total elevation lost when you return to start.
\end{exampleblock}

\pause
\vspace{0.3cm}

For series circuit: $V_{\text{battery}} = V_1 + V_2 + V_3$

\pause
\vspace{0.3cm}

Using Ohm's law: $V = I(R_1 + R_2 + R_3)$
\note{[P0] "Voltage loop rule - conservation of energy"\\\\
[P1] "Like hiking - what goes up must come down"\\\\
[P2] "Battery voltage equals sum of voltage drops"\\\\
[P3] "Leads to series resistance formula"\\\\
[THE WONDER] Same reasoning works for any circuit}
\end{frame}

\begin{frame}
\frametitle{Attempt: Series Circuit}
\begin{exampleblock}{The Challenge (3 min, silent)}
Three resistors in series: $R_1 = 1.0 \Omega$, $R_2 = 6.0 \Omega$, $R_3 = 13 \Omega$. Battery voltage is 12 V.

\vspace{0.3cm}

\textbf{Given:}
\begin{itemize}
\item $R_1 = 1.0 \Omega$, $R_2 = 6.0 \Omega$, $R_3 = 13 \Omega$
\item $V = 12$ V
\end{itemize}

\textbf{Find:} (a) Equivalent resistance, (b) Current through circuit

\vspace{0.3cm}

\textit{Can you reduce the circuit to a single resistance?}
\end{exampleblock}
\note{[THE CHALLENGE] Two-part problem - resistance then current\\\\
[SAY] "Find equivalent resistance first, then use Ohm's law"\\\\
[TIMING] 3-4 min SILENT work\\\\
[CIRCULATE] Note who completes part a but gets stuck on part b\\\\
[DON'T HELP] Let them work through both parts}
\end{frame}

\begin{frame}
\frametitle{Compare: Series Resistance}
\textbf{Turn and talk (2 min):}

\vspace{0.3cm}

\begin{enumerate}
\item How did you find equivalent resistance?
\item What equation did you use for current?
\item Did you get the same values?
\end{enumerate}

\vspace{0.5cm}

\pause
\alert{Name wheel:} One pair share your approach.
\note{[TIMING] 2-3 min pair discussion\\\\
[CIRCULATE] Listen for strategy\\\\
[CHECK] Name wheel: call a pair\\\\
[EXPECTED APPROACH] Add resistances, then I equals V over R-equiv\\\\
[COMMON ERROR] Trying to find current before finding equivalent resistance}
\end{frame}

\begin{frame}
\frametitle{Reveal: Series Solution}
\textbf{Self-correct in a different color:}

\vspace{0.3cm}

\textbf{Part (a):} $R_{\text{equiv}} = R_1 + R_2 + R_3$

\pause
$$R_{\text{equiv}} = 1.0 \Omega + 6.0 \Omega + 13 \Omega = \boxed{20 \Omega}$$

\pause
\vspace{0.3cm}

\textbf{Part (b):} $I = \frac{V}{R_{\text{equiv}}}$

\pause
$$I = \frac{12 \text{ V}}{20 \Omega} = \boxed{0.60 \text{ A}}$$

\pause
\textbf{Check:} Voltage drops: $V_1 = 0.6$ V, $V_2 = 3.6$ V, $V_3 = 7.8$ V. Sum = 12 V!
\note{[P0] "Self-correct in a different color"\\\\
[P1] [ALGEBRA] "R-equiv equals 1 plus 6 plus 13"\\\\
[P2] [ANSWER] "20 ohms"\\\\
[P3] "Current equals 12 volts over 20 ohms"\\\\
[P4] [ANSWER] "0.60 amperes"\\\\
[P5] "Check: individual drops add to 12 V - energy conserved!"\\\\
[THE WONDER] You just analyzed a circuit like an engineer}
\end{frame}

\section{19.3 Parallel Circuits}

\begin{frame}
\frametitle{Learning Objectives}
\begin{block}{By the end of this section, you will be able to:}
\begin{itemize}
\item \textbf{19.3:} Interpret circuit diagrams with parallel resistors \pause
\item \textbf{19.3:} Calculate equivalent resistance of resistor combinations
\end{itemize}
\end{block}
\note{[P0] "Two objectives for parallel circuits"\\\\
[P1] "First: understand parallel configuration"\\\\
[P2] "Second: calculate equivalent resistance"\\\\
- More complex than series}
\end{frame}

\begin{frame}
\frametitle{19.3 Resistors in Parallel}
\begin{figure}
\centering
\includegraphics[width=0.7\textwidth,height=0.45\textheight,keepaspectratio]{phys11-circuits-fig19-16.jpg}
\end{figure}

\pause
\begin{block}{The Rule for Parallel}
\begin{center}
\Large $\boxed{R_{\text{equiv}} = \frac{1}{\frac{1}{R_1} + \frac{1}{R_2} + \frac{1}{R_3}}}$
\end{center}
Parallel resistance is reciprocal of sum of reciprocals.
\end{block}
\note{[P0][Fig 19.16: Left shows three resistors in parallel, right shows equivalent single resistor] "Most counterintuitive concept in circuits - explicitly confront misconception. Diagram shows current splitting into three paths - use traffic analogy: more lanes means less congestion. Critical pedagogical moment: students who memorize formula without understanding the 'why' will fail complex problems. Emphasize: parallel provides MORE paths for charge, reducing effective opposition."\\\\
[P1] [THE REVELATION] "Parallel resistance uses reciprocals - more complex formula"\\\\
- Same voltage across all resistors (connected to same two points)\\\\
- Current splits between paths (sum of branch currents equals total)\\\\
- More paths means LESS total resistance\\\\
[THE CONFLICT] Counterintuitive - adding resistors decreases total resistance! Like opening more checkout lanes at grocery store}
\end{frame}

\begin{frame}
\frametitle{19.3 The Parallel Paradox}
\begin{alertblock}{What Your Brain Gets Wrong}
\textbf{Intuition:} More resistors means more resistance.\\
\textbf{Reality:} Parallel resistors provide MORE paths, so LESS resistance!
\end{alertblock}

\pause
\vspace{0.3cm}

\textbf{Key insight:} $R_{\text{equiv}}$ is always LESS than smallest resistor in parallel.

\pause
\begin{exampleblock}{The Mental Model}
Three identical resistors $R$ in parallel: $R_{\text{equiv}} = R/3$
\end{exampleblock}
\note{[P0] [THE CONFLICT] "More resistors, less resistance - seems backwards!"\\\\
[P1] "Equivalent resistance always less than smallest resistor"\\\\
[P2] "Three identical R in parallel gives R over 3"\\\\
[THE HUMILITY] This confuses everyone at first\\\\
[THE WONDER] Current takes all available paths - nature's efficiency}
\end{frame}

\begin{frame}
\frametitle{19.3 Current Conservation}
\textbf{Key principle:} Current entering a junction equals current leaving.

\pause
\vspace{0.3cm}

For parallel resistors: $I = I_1 + I_2 + I_3$

\pause
\vspace{0.3cm}

\begin{exampleblock}{The Mental Model}
Like a river splitting into three channels - total water flow is conserved.
\end{exampleblock}

\pause
\vspace{0.3cm}

Using Ohm's law on each: $I = V(\frac{1}{R_1} + \frac{1}{R_2} + \frac{1}{R_3})$
\note{[P0] "Charge cannot be created or destroyed"\\\\
[P1] "Total current equals sum of branch currents"\\\\
[P2] "Like river splitting into channels"\\\\
[P3] "Leads to parallel resistance formula"\\\\
[THE WONDER] Conservation laws guide everything}
\end{frame}

\begin{frame}
\frametitle{Attempt: Parallel Circuit}
\begin{exampleblock}{The Challenge (3 min, silent)}
Three resistors in parallel: $R_1 = 10 \Omega$, $R_2 = 25 \Omega$, $R_3 = 15 \Omega$. Battery voltage is 3 V.

\vspace{0.3cm}

\textbf{Given:}
\begin{itemize}
\item $R_1 = 10 \Omega$, $R_2 = 25 \Omega$, $R_3 = 15 \Omega$
\item $V = 3$ V
\end{itemize}

\textbf{Find:} (a) Equivalent resistance, (b) Total current

\vspace{0.3cm}

\textit{Can you use the reciprocal formula?}
\end{exampleblock}
\note{[THE CHALLENGE] Parallel is trickier than series\\\\
[SAY] "Use reciprocal formula - watch your algebra"\\\\
[TIMING] 3-4 min SILENT work\\\\
[CIRCULATE] Note who struggles with reciprocals\\\\
[WATCH FOR] Calculator errors\\\\
[DON'T HELP] Productive struggle}
\end{frame}

\begin{frame}
\frametitle{Compare: Parallel Resistance}
\textbf{Turn and talk (2 min):}

\vspace{0.3cm}

\begin{enumerate}
\item How did you handle the reciprocals?
\item What calculator steps did you use?
\item Is your answer less than 10 ohms?
\end{enumerate}

\vspace{0.5cm}

\pause
\alert{Name wheel:} One pair share your approach.
\note{[TIMING] 2-3 min pair discussion\\\\
[CIRCULATE] Listen for calculation strategy\\\\
[CHECK] Name wheel: call a pair\\\\
[EXPECTED APPROACH] Calculate 1/R-1 plus 1/R-2 plus 1/R-3, then take reciprocal\\\\
[COMMON ERROR] Forgetting final reciprocal step}
\end{frame}

\begin{frame}
\frametitle{Reveal: Parallel Solution}
\textbf{Self-correct in a different color:}

\vspace{0.3cm}

\textbf{Part (a):} $R_{\text{equiv}} = \frac{1}{\frac{1}{R_1} + \frac{1}{R_2} + \frac{1}{R_3}}$

\pause
$$R_{\text{equiv}} = \frac{1}{\frac{1}{10} + \frac{1}{25} + \frac{1}{15}} = \frac{1}{0.1 + 0.04 + 0.0667} = \boxed{4.84 \Omega}$$

\pause
\vspace{0.3cm}

\textbf{Part (b):} $I = \frac{V}{R_{\text{equiv}}} = \frac{3 \text{ V}}{4.84 \Omega} = \boxed{0.62 \text{ A}}$

\pause
\textbf{Check:} 4.84 < 10 (smallest resistor). Current splits: $I_1=0.30$ A, $I_2=0.12$ A, $I_3=0.20$ A!
\note{[P0] "Self-correct in a different color"\\\\
[P1] [ALGEBRA] "One over open paren 1 over 10 plus 1 over 25 plus 1 over 15 close paren"\\\\
[P2] [ANSWER] "4.84 ohms - less than 10!"\\\\
[P3] "Current equals 3 volts over 4.84 ohms equals 0.62 amperes"\\\\
[P4] "Check: branch currents add to 0.62 A - charge conserved!"\\\\
[THE WONDER] You just solved a parallel circuit - same math in power grids}
\end{frame}

\section{19.4 Electric Power}

\begin{frame}
\frametitle{Learning Objectives}
\begin{block}{By the end of this section, you will be able to:}
\begin{itemize}
\item \textbf{19.4:} Define electric power and describe the power equation \pause
\item \textbf{19.4:} Calculate power in circuits
\end{itemize}
\end{block}
\note{[P0] "Two objectives for electric power"\\\\
[P1] "First: understand power as energy transfer rate"\\\\
[P2] "Second: calculate power in circuits"\\\\
- This determines your electric bill}
\end{frame}

\begin{frame}
\frametitle{19.4 The Energy Transfer Rate}
\begin{block}{The Power Law}
\begin{center}
\Large $\boxed{P = IV}$
\end{center}
Power equals current times voltage. Energy per unit time.
\end{block}

\pause
\vspace{0.3cm}

\textbf{Units:} Watts (W), where $1 \text{ W} = 1 \text{ J/s}$

\pause
\begin{exampleblock}{The Mental Model}
A 60-W bulb transfers 60 joules per second from electrical energy to light and heat.
\end{exampleblock}
\note{[P0] [THE REVELATION] "P equals I V - power equals current times voltage"\\\\
[P1] "Watts measure energy transfer rate - joules per second"\\\\
[P2] "60-W bulb converts 60 joules per second"\\\\
[THE CONNECTION - Digital Archetype] Your electric bill charges for power times time\\\\
[THE WONDER] Same equation in LED bulbs and power plants}
\end{frame}

\begin{frame}
\frametitle{19.4 Three Forms of Power}
Using Ohm's law $V = IR$, we can derive alternate forms:

\pause
\vspace{0.3cm}

\begin{align}
P &= IV \\
P &= I^2R \quad \text{(using } V = IR \text{)} \\
P &= \frac{V^2}{R} \quad \text{(using } I = \frac{V}{R} \text{)}
\end{align}

\pause
\vspace{0.3cm}

\textbf{Choose the form that matches your known quantities.}
\note{[P0] "Three equivalent power formulas"\\\\
[P1] "P equals I V - when you know current and voltage"\\\\
[P2] "P equals I squared R - when you know current and resistance"\\\\
"P equals V squared over R - when you know voltage and resistance"\\\\
[THE HUMILITY] All three are Ohm's law in disguise}
\end{frame}

\begin{frame}
\frametitle{19.4 The 25-W vs 60-W Mystery}
\begin{figure}
\centering
\includegraphics[width=0.6\textwidth,height=0.4\textheight,keepaspectratio]{phys11-circuits-fig19-20.jpg}
\end{figure}

\pause
\textbf{Question:} Both run on 120 V. Why different brightness?

\pause
\begin{alertblock}{The Revelation}
They have DIFFERENT resistances! Using $P = \frac{V^2}{R}$, higher power means LOWER resistance.
\end{alertblock}
\note{[P0][Fig 19.20: Two bulbs labeled 25W and 60W showing brightness difference] "Visual comparison makes abstract power concept concrete. Students incorrectly assume 'bigger number = more resistance' - this image forces cognitive conflict. Guide reasoning: both see 120V, but 60W dissipates more energy per second, so current must be larger (P=IV), which requires SMALLER resistance (V=IR). This synthesis of multiple equations builds problem-solving sophistication."\\\\
[P1] "Both operate on 120 V - so why different brightness?"\\\\
[P2] [THE REVELATION] "Different resistances! P equals V squared over R"\\\\
- Higher power requires lower resistance at same voltage\\\\
- 60-W bulb has R=240Ω, 25-W bulb has R=576Ω (calculate if time permits)\\\\
[THE WONDER] Brightness is about resistance, not just voltage - manufacturer designs different filament thicknesses}
\end{frame}

\begin{frame}
\frametitle{Attempt: Lightbulb Current}
\begin{exampleblock}{The Challenge (3 min, silent)}
A 60-W incandescent bulb operates on 120 V.

\vspace{0.3cm}

\textbf{Given:}
\begin{itemize}
\item $P = 60$ W
\item $V = 120$ V
\end{itemize}

\textbf{Find:} Current through the bulb

\vspace{0.3cm}

\textit{Which power formula should you use?}
\end{exampleblock}
\note{[THE CHALLENGE] Choose correct formula\\\\
[SAY] "You know P and V - which formula?"\\\\
[TIMING] 3-4 min SILENT work\\\\
[CIRCULATE] Note who chooses P equals I V\\\\
[WATCH FOR] Students using wrong formula\\\\
[DON'T HELP] Let them choose}
\end{frame}

\begin{frame}
\frametitle{Compare: Power Formula}
\textbf{Turn and talk (2 min):}

\vspace{0.3cm}

\begin{enumerate}
\item Which power formula did you use?
\item How did you solve for current?
\item What units did you get?
\end{enumerate}

\vspace{0.5cm}

\pause
\alert{Name wheel:} One pair share your approach.
\note{[TIMING] 2-3 min pair discussion\\\\
[CIRCULATE] Listen for formula choice\\\\
[CHECK] Name wheel: call a pair\\\\
[EXPECTED APPROACH] P equals I V, so I equals P over V\\\\
[COMMON ERROR] Using P equals I squared R without knowing R}
\end{frame}

\begin{frame}
\frametitle{Reveal: Lightbulb Current}
\textbf{Self-correct in a different color:}

\vspace{0.3cm}

\textbf{Start with:} $P = IV$

\pause
\vspace{0.2cm}

\textbf{Solve for I:} $I = \frac{P}{V}$

\pause
\vspace{0.2cm}

\textbf{Substitute:} $I = \frac{60 \text{ W}}{120 \text{ V}}$

\pause
\vspace{0.2cm}

$$\boxed{I = 0.50 \text{ A}}$$

\pause
\textbf{Check:} Half an ampere - significant current for a light bulb!
\note{[P0] "Self-correct in a different color"\\\\
[P1] [ALGEBRA] "P equals I V"\\\\
[P2] "Solve for I - divide both sides by V"\\\\
[P3] "I equals 60 watts over 120 volts"\\\\
[P4] [ANSWER] "0.50 amperes - half an amp!"\\\\
[THE WONDER] Every time you flip a light switch, you control a half-ampere current}
\end{frame}

\section{Summary}

\begin{frame}
\frametitle{What You Now Know}
\begin{block}{The Revelations}
\begin{enumerate}
\item Current = charge per time: $I = \frac{\Delta Q}{\Delta t}$ \pause
\item Ohm's law = the pushback equation: $V = IR$ \pause
\item Series resistances add: $R_{\text{equiv}} = R_1 + R_2 + \cdots$ \pause
\item Parallel uses reciprocals: $R_{\text{equiv}} = \frac{1}{\frac{1}{R_1} + \frac{1}{R_2} + \cdots}$ \pause
\item Power = energy transfer rate: $P = IV = I^2R = \frac{V^2}{R}$
\end{enumerate}
\end{block}
\note{[P0] "Five revelations today"\\\\
[P1] "Current is charge flow rate"\\\\
[P2] "Ohm's law relates voltage, current, resistance"\\\\
[P3] "Series resistances add"\\\\
[P4] "Parallel resistances use reciprocals"\\\\
[P5] "Power has three forms"\\\\
[THE WONDER] You now understand every circuit around you\\\\
- Name wheel: which was most surprising?}
\end{frame}

\begin{frame}[shrink]
\frametitle{Key Equations}
\begin{align}
I &= \frac{\Delta Q}{\Delta t} \quad \text{(current)} \\
V &= IR \quad \text{(Ohm's law)} \\
R_{\text{series}} &= R_1 + R_2 + \cdots + R_N \\
R_{\text{parallel}} &= \frac{1}{\frac{1}{R_1} + \frac{1}{R_2} + \cdots + \frac{1}{R_N}} \\
P &= IV \\
P &= I^2R \\
P &= \frac{V^2}{R}
\end{align}
\note{[SAY] "Seven essential equations"\\\\
- Current definition\\\\
- Ohm's law\\\\
- Series and parallel rules\\\\
- Three power formulas\\\\
[THE WONDER] These seven equations describe electricity in the entire world}
\end{frame}

\begin{frame}
\frametitle{Homework}
\begin{center}
\Large
Complete the assigned problems\\[0.3cm]
posted on the LMS
\end{center}
\note{[SAY] "Homework posted on LMS"\\\\
[TIMING] Due date: check LMS\\\\
[CHECK] Questions before we end?\\\\
[TRANSITION] Next: magnetism - electricity's partner in crime}
\end{frame}

\end{document}
