\documentclass{beamer}
\usepackage{../../../shared/templates/ds9_theme}
\usepackage{../../../shared/templates/semantic-physics-colors}
\usepackage[overridenote]{pdfpc}
\graphicspath{{../images/}{../../shared/images/}}

\title[Forces in Your Hair]{PHYS11 CH:18 The Force That Moves Everything}
\subtitle{Electric Charge and Conservation}
\author[Mr. Gullo]{Mr. Gullo}
\date[December 2025]{December 2025}

\begin{document}

\frame{\titlepage
\note{[THE HOOK] Today we discover the invisible force inside everything.\\\\
- Same force explains lightning AND the screen you stare at\\\\
- Four revelations: what charge IS, how it's conserved, conductors vs insulators, how it transfers\\\\
[THE WONDER] By end of class, you'll see electricity isn't magic - it's nature's accounting system.\\\\
- Section 18.1 only - foundation for circuits and electric fields}
}

\begin{frame}
\frametitle{Outline}
\tableofcontents
\end{frame}

\section{Introduction}

\begin{frame}
\frametitle{Why Does Hair Stand on End?}
\begin{figure}
\centering
\includegraphics[width=0.7\textwidth,height=0.5\textheight,keepaspectratio]{phys11-electric-charge-fig18-1.jpg}
\end{figure}

\pause
\textit{What invisible force makes strands of hair repel each other?}

\note{[P0] [Fig 18.1: Static hair] "Every strand is pushing away from every other strand - same charge repels. This child IS the experiment."\\\\
[P1] [THE HOOK] "Invisible force making hair strands repel each other"\\\\
[THE CONNECTION - Kinetic Archetype] "Ever slide down a plastic slide and zap someone? Same thing."\\\\
[THE WONDER] This is the electromagnetic force - one of four fundamental forces in nature}
\end{frame}

\begin{frame}
\frametitle{The Invisible World}
\begin{center}
\Large What if everything you touch\\
\textit{is held together by invisible forces?}
\end{center}

\pause
\vspace{0.5cm}
The atoms in your fingertips never actually touch the atoms in this desk...

\pause
\vspace{0.3cm}
\alert{Electric forces keep them apart.}

\note{[P0] "What if everything you touch is held together by invisible forces?"\\\\
[P1] "Your fingers never actually touch the desk"\\\\
[P2] [THE REVELATION] "Electric forces keep atoms apart - you're floating on electric fields"\\\\
[THE HUMILITY] Ancient Greeks knew amber attracted straw but didn't understand why\\\\
[THE WONDER] Took 2000 years to understand this force}
\end{frame}

\section{Electrical Charges, Conservation of Charge, and Transfer of Charge}

\begin{frame}
\frametitle{Learning Objectives}
\begin{block}{By the end of this lesson, you will be able to:}
\begin{itemize}
\item \textbf{18.1:} Describe positive and negative electric charges \pause
\item \textbf{18.1:} Use conservation of charge to calculate charge transfers \pause
\item \textbf{18.1:} Characterize conductors vs insulators \pause
\item \textbf{18.1:} Describe electric polarization and charging by induction
\end{itemize}
\end{block}
\note{[P0] "Four objectives today"\\\\
[P1] "First: what are positive and negative charges"\\\\
[P2] "Second: conservation of charge - nature's accounting system"\\\\
[P3] "Third: conductors vs insulators"\\\\
[P4] "Fourth: polarization and induction - charge without touching"\\\\
- Assessment: quiz next week on charge calculations}
\end{frame}

\begin{frame}
\frametitle{18.1 Two Types of Charge}
\begin{block}{Nature's Binary Code}
Electric charge is a property of matter that causes objects to attract or repel each other.
\end{block}

\pause
\vspace{0.3cm}

\textbf{The discovery:}
\begin{itemize}
\item Glass rubbed with silk: glass becomes positive, silk becomes negative \pause
\item Like charges \alert{repel} each other \pause
\item Unlike charges \alert{attract} each other
\end{itemize}

\pause
\begin{exampleblock}{The Mental Model}
Charge is nature's binary system: positive or negative. No neutral charge exists.
\end{exampleblock}

\note{[P0] [THE REVELATION] "Electric charge - property of matter causing attraction or repulsion"\\\\
[P1] "Glass rubbed with silk - charge separation discovered experimentally"\\\\
[P2] "Like charges repel - two positive or two negative push apart"\\\\
[P3] "Unlike charges attract - positive and negative pull together"\\\\
[P4] [THE CONNECTION - Digital Archetype] "Like binary code: only two states, positive or negative"\\\\
[THE HUMILITY] Greeks called it elektron after amber - didn't understand it for 2000 years}
\end{frame}

\begin{frame}
\frametitle{18.1 Experimental Evidence}
\begin{figure}
\centering
\includegraphics[width=0.7\textwidth,height=0.55\textheight,keepaspectratio]{phys11-electric-charge-fig18-2.jpg}
\end{figure}

\pause
\textbf{Pattern:} Glass rods repel each other. Silk cloths repel each other. Glass attracts silk.

\note{[P0] [Fig 18.2: Glass rod and silk] "Three experiments prove charges exist: charged glass repels charged glass, charged silk repels charged silk, but glass attracts silk. The pattern IS the proof."\\\\
[P1] [THE PATTERN] "Same materials repel, different materials attract"\\\\
[THE CONNECTION - Kinetic Archetype] "Like magnets you've played with - push and pull"\\\\
[THE WONDER] This simple pattern explains all electric phenomena}
\end{frame}

\begin{frame}
\frametitle{18.1 The Discovery of the Electron}
\textbf{The mystery (1897):} What carries negative charge?

\pause
\vspace{0.3cm}

\textbf{The experiment:} J.J. Thomson studied cathode rays

\pause
\textbf{The revelation:} Cathode rays are particles carrying negative charge

\pause
\vspace{0.3cm}

\begin{block}{Universal Law: The Electron}
\begin{center}
The electron carries the fundamental unit of \textit{negative} electric charge.
\end{center}
\end{block}

\note{[P0] "Late 1800s: mystery of cathode rays"\\\\
[P1] "J.J. Thomson experimenting with cathode ray tubes"\\\\
[P2] "Jean Perrin showed cathode rays carry negative charge"\\\\
[P3] [THE REVELATION] "Thomson discovered the electron - fundamental particle"\\\\
[THE HUMILITY] Took centuries to discover what charge actually is\\\\
[THE WONDER] Electrons are in every atom in your body}
\end{frame}

\begin{frame}
\frametitle{18.1 Inside the Atom}
\begin{figure}
\centering
\includegraphics[width=0.7\textwidth,height=0.5\textheight,keepaspectratio]{phys11-electric-charge-fig18-3.jpg}
\end{figure}

\pause
Rutherford's model: electrons orbit a tiny, dense nucleus of protons.

\note{[P0] [Fig 18.3: Plum pudding vs nuclear] "Left shows Thomson's wrong model - positive pudding with electron plums. Right shows Rutherford's correct model - electrons orbit tiny nucleus. Even geniuses are wrong until experiments prove otherwise."\\\\
[P1] [THE REVELATION] "Protons in tiny dense nucleus, electrons orbit around"\\\\
[THE HUMILITY] Even great scientists like Thomson had wrong models\\\\
[THE WONDER] Nucleus is 100,000 times smaller than atom but contains almost all the mass}
\end{frame}

\begin{frame}
\frametitle{18.1 The Fundamental Unit of Charge}
\begin{block}{Nature's Quantum}
\begin{center}
\Large $\charge{e} = 1.602 \times 10^{-19}$ C
\end{center}
The fundamental unit of electric \charge{charge} (magnitude).
\end{block}

\pause
\vspace{0.3cm}

\textbf{The carriers:}
\begin{itemize}
\item Proton \charge{charge}: $+\charge{e} = +1.602 \times 10^{-19}$ C \pause
\item Electron \charge{charge}: $-\charge{e} = -1.602 \times 10^{-19}$ C
\end{itemize}

\pause
\vspace{0.3cm}

\begin{alertblock}{The Paradox}
\textbf{Civilian:} "Why is charge so tiny?"\\
\textbf{Physicist:} "It takes $6.25 \times 10^{18}$ protons to make just 1 coulomb!"
\end{alertblock}

\note{[P0] [THE REVELATION] "e equals 1.602 times 10 to negative 19 coulombs - fundamental constant"\\\\
[P1] "Proton: plus e. Electron: minus e"\\\\
[P2] "Exactly equal magnitude, opposite sign"\\\\
[P3] [THE CONFLICT] "Charge seems small because atoms are tiny"\\\\
[THE WONDER] This number is the same everywhere in the universe - never changes}
\end{frame}

\begin{frame}
\frametitle{18.1 Measuring the Electron Charge}
\begin{figure}
\centering
\includegraphics[width=0.6\textwidth,height=0.45\textheight,keepaspectratio]{phys11-electric-charge-fig18-4.jpg}
\end{figure}

\pause
\textbf{Millikan Oil-Drop Experiment (1909):}
\begin{itemize}
\item Spray oil droplets between charged plates
\item Balance electric force against gravity
\item Measure charge on individual drops
\end{itemize}

\note{[P0] [Fig 18.4: Millikan oil-drop] "Spray oil droplets between charged plates, electric force pulls up, gravity pulls down. Balance forces and measure charge. Like weighing individual atoms - except you're counting electrons."\\\\
[P1] "Spray oil between charged plates, balance forces"\\\\
[THE CONNECTION - Digital Archetype] "Like balancing a character in a game using forces"\\\\
[THE HUMILITY] Millikan's values were slightly off but method was brilliant\\\\
[THE REVELATION] Discovered charge comes in discrete units - quantized}
\end{frame}

\begin{frame}
\frametitle{18.1 Charge Quantization}
\textbf{Discovery:} \charge{Charge} always comes in multiples of $\charge{e}$

\pause
\vspace{0.3cm}

\begin{block}{Universal Law: Charge Quantization}
\begin{center}
\Large $\boxed{\charge{Q} = \particles{n}\charge{e}}$
\end{center}
where $\particles{n}$ is an integer ($\pm 1, \pm 2, \pm 3, \ldots$)
\end{block}

\pause
\vspace{0.3cm}

\textbf{Meaning:} You can have 5 electrons or 5 million, but never 5.5 electrons.

\note{[P0] "Charge is quantized - comes in discrete packets"\\\\
[P1] [THE REVELATION] "Q equals n e, where n is an integer"\\\\
[P2] [THE CONFLICT] "You can't have half an electron - nature has minimum units"\\\\
[THE CONNECTION - Digital Archetype] "Like pixels on a screen - minimum unit size"\\\\
[THE WONDER] Quantization is everywhere in physics - energy, angular momentum, charge}
\end{frame}

\begin{frame}
\frametitle{18.1 Conservation of Charge}
\begin{block}{Nature's Accounting System}
\begin{center}
\Large $\boxed{\charge{q}_{\text{initial}} = \charge{q}_{\text{final}}}$
\end{center}
Electric \charge{charge} cannot be created or destroyed.
\end{block}

\pause
\vspace{0.3cm}

\textbf{What this means:}
\begin{itemize}
\item Total \charge{charge} before interaction = total \charge{charge} after \pause
\item \charge{Charge} can move, but net \charge{charge} stays constant \pause
\item Most fundamental conservation law in physics
\end{itemize}

\note{[P0] [THE REVELATION] "Conservation of charge - most fundamental law"\\\\
[P1] "Total charge before equals total charge after"\\\\
[P2] "Charge can move around but net charge stays constant"\\\\
[P3] "Never been violated - ever"\\\\
[THE CONNECTION - Digital Archetype] "Like inventory system - items move but total count stays same"\\\\
[THE WONDER] Works at every scale - atoms to galaxies}
\end{frame}

\begin{frame}
\frametitle{18.1 Conductors vs Insulators}
\begin{columns}[T]
\column{0.48\textwidth}
\begin{block}{Conductors}
Materials that allow \charge{charge} to move freely
\end{block}

\textbf{Examples:}
\begin{itemize}
\item Metals (copper, silver, aluminum)
\item Electrons loosely bound
\end{itemize}

\pause
\column{0.48\textwidth}
\begin{block}{Insulators}
Materials that prevent \charge{charge} from moving
\end{block}

\textbf{Examples:}
\begin{itemize}
\item Rubber, plastic, glass, wood
\item Electrons tightly bound
\end{itemize}
\end{columns}

\vspace{0.3cm}
\pause

\begin{exampleblock}{The Mental Model}
Conductor = highway for electrons. Insulator = roadblock.
\end{exampleblock}

\note{[P0] "Two categories of materials by electrical behavior"\\\\
[P1] "Insulators: electrons tightly bound to atoms"\\\\
[P2] [THE REVELATION] "Conductors let charge flow, insulators don't"\\\\
[THE CONNECTION - Kinetic Archetype] "Like difference between ice and water - same stuff, different flow"\\\\
[THE WONDER] Your body is a conductor - that's why you get shocked}
\end{frame}

\begin{frame}
\frametitle{18.1 The Conductivity Spectrum}
\begin{figure}
\centering
\includegraphics[width=0.8\textwidth,height=0.55\textheight,keepaspectratio]{phys11-electric-charge-fig18-8.jpg}
\end{figure}

\pause
\textbf{Semiconductors:} Between conductors and insulators (silicon, germanium)

\note{[Fig 18.8: conductivity spectrum from conductors to insulators]\\\\
[P0] "Spectrum of conductivity from best conductors to best insulators"\\\\
[P1] [THE REVELATION] "Semiconductors can be controlled - sometimes conduct, sometimes insulate"\\\\
[THE CONNECTION - Digital Archetype] "Your phone's processor uses billions of semiconductors as switches"\\\\
[THE WONDER] Gap between conductors and insulators is enormous - 20 plus orders of magnitude}
\end{frame}

\begin{frame}
\frametitle{18.1 Charge Distribution}
\begin{figure}
\centering
\includegraphics[width=0.7\textwidth,height=0.5\textheight,keepaspectratio]{phys11-electric-charge-fig18-10.jpg}
\end{figure}

\pause
\textbf{Conductor:} \charge{Charge} spreads to outer surface (repulsion wins).\\
\textbf{Insulator:} \charge{Charge} stays in place (can't move).

\note{[P0] [Fig 18.10: Charge redistribution] "Before contact: 100 electrons on left, 50 on right. At contact: electrons flow because they repel each other. After contact: 75 on each - charges equalized. Nature always equalizes."\\\\
[P1] [THE CONFLICT] "Conductor: like charges repel, spread to outer surface. Insulator: charge can't move, stays put"\\\\
[THE WONDER] All excess charge on conductor lives on surface - zero charge inside}
\end{frame}

\begin{frame}
\frametitle{18.1 Transferring Charge: Contact}
\textbf{Charging by contact:} Surfaces touch and share electrons

\pause
\vspace{0.3cm}

\textbf{Example:} Walking across carpet
\begin{itemize}
\item Shoes rub carpet \pause
\item Electrons transfer from carpet to shoes \pause
\item You accumulate excess negative charge \pause
\item Touch doorknob $\rightarrow$ \alert{ZAP!}
\end{itemize}

\pause
\vspace{0.3cm}

\begin{exampleblock}{Real-World: Static Shock}
Rubbing increases contact between materials, transferring more electrons.
\end{exampleblock}

\note{[P0] "First method of charge transfer - contact"\\\\
[P1] "Walking across carpet - shoes rub against fibers"\\\\
[P2] "Electrons transfer from carpet to shoes"\\\\
[P3] "You accumulate excess negative charge"\\\\
[P4] "Touch doorknob - electrons jump, creating spark"\\\\
[P5] [THE CONNECTION - Kinetic Archetype] "Ever shocked someone after sliding down a slide? Same thing"\\\\
[THE WONDER] That spark is mini lightning}
\end{frame}

\begin{frame}
\frametitle{18.1 Transferring Charge: Conduction}
\begin{figure}
\centering
\includegraphics[width=0.7\textwidth,height=0.5\textheight,keepaspectratio]{phys11-electric-charge-fig18-10.jpg}
\end{figure}

\pause
\textbf{Charging by conduction:} Touching charged object to neutral object

\pause
Charges redistribute to equalize - like water finding same level.

\note{[P0] [Fig 18.10: Conduction mechanism] "This IS conduction: two conductors touch, electrons jump because they repel each other. 25 electrons move in the instant of contact. After separation, both have equal charge - nature's equilibrium."\\\\
[P1] "Two spheres touch - charges redistribute"\\\\
[P2] [THE REVELATION] "Charge flows until balanced between objects"\\\\
[THE CONNECTION - Harmonic Archetype] "Like equalizing pressure between two connected tanks"\\\\
[THE WONDER] Electrons repel each other so strongly they'll move to balance out}
\end{frame}

\begin{frame}
\frametitle{18.1 Polarization}
\textbf{What is polarization?} Separation of positive and negative charges within an object

\pause
\vspace{0.3cm}

\textbf{Example:} Child on slide (Figure 18.1)
\begin{itemize}
\item Excess positive charge on body \pause
\item Positive charges repel each other \pause
\item Spread to extremities (hair) \pause
\item Hair strands repel $\rightarrow$ stand on end!
\end{itemize}

\pause
\vspace{0.3cm}

\begin{exampleblock}{The Mental Model}
Like dominos - push on one end, disturbance travels to other end.
\end{exampleblock}

\note{[P0] "Polarization - charge separation without contact"\\\\
[P1] "Child on slide has excess positive charge from friction"\\\\
[P2] "Positive charges repel each other"\\\\
[P3] "Spread to extremities: hands, feet, hair"\\\\
[P4] "Each hair strand has excess positive - strands repel"\\\\
[P5] [THE CONNECTION - Kinetic Archetype] "Like dominos falling - one pushes next"\\\\
[THE REVELATION] No electrons traveled - they pushed each other}
\end{frame}

\begin{frame}
\frametitle{18.1 Charging by Induction}
\begin{figure}
\centering
\includegraphics[width=0.7\textwidth,height=0.6\textheight,keepaspectratio]{phys11-electric-charge-fig18-11.jpg}
\end{figure}

\pause
\textbf{Induction:} Creating charge separation by approaching a charged object (no touching!)

\note{[P0] [Fig 18.11: Induction steps] "Four frames show the magic: (a) neutral spheres touching, (b) charged rod approaches - electrons flee to far sphere, (c) separate while rod still near, (d) rod removed - left sphere is positive, right is negative. Charged without touching!"\\\\
[P1] [THE REVELATION] "Charged rod approaches, induces separation, spheres separate with opposite charges"\\\\
[THE WONDER] Charged object without ever touching - action at a distance\\\\
[THE HUMILITY] This seemed like magic to ancient scientists}
\end{frame}

\begin{frame}
\frametitle{18.1 Van de Graaff Generator}
\begin{figure}
\centering
\includegraphics[width=0.6\textwidth,height=0.45\textheight,keepaspectratio]{phys11-electric-charge-fig18-12.jpg}
\end{figure}

\pause
\textbf{How it works:}
\begin{itemize}
\item Belt transfers electrons to metal globe
\item Electrons spread over outer surface (repulsion)
\item Can accumulate millions of volts!
\end{itemize}

\note{[P0] [Fig 18.12: Van de Graaff mechanism] "Belt acts like a conveyor belt for electrons. Lower comb puts charge on belt, upper comb removes it to sphere. Electrons repel to outer surface. Keep adding charge until... ZAP! Spark to ground."\\\\
[P1] "Belt continuously transfers electrons to globe, accumulates huge charge"\\\\
[THE CONNECTION - Kinetic Archetype] "Touch it and your hair stands up - same as child on slide"\\\\
[THE WONDER] Can create sparks over a meter long - mini lightning in classroom}
\end{frame}

\begin{frame}
\frametitle{18.1 Hair-Raising Physics}
\begin{figure}
\centering
\includegraphics[width=0.6\textwidth,height=0.5\textheight,keepaspectratio]{phys11-electric-charge-fig18-13.jpg}
\end{figure}

\pause
\textbf{Why does hair stand up?}
\begin{itemize}
\item Each hair strand gets excess charge (same sign)
\item Like charges repel
\item Strands push away from each other as far as possible
\end{itemize}

\note{[P0] [Fig 18.13: Hair repulsion demo] "Man is electrically connected to sphere - electrons flow to him. Every hair gets excess electrons. Each hair strand repels every other strand. Hair stands up seeking maximum separation. Same physics as Fig 18.1 - we've come full circle."\\\\
[P1] [THE REVELATION] "Each hair has same charge, repels neighboring hairs"\\\\
[THE CONNECTION - Harmonic Archetype] "Like trying to push same poles of magnets together"\\\\
[THE WONDER] Same effect as opening photograph - electric repulsion in action}
\end{frame}

\section{Worked Example}

\begin{frame}
\frametitle{Attempt: Conservation Challenge}
\begin{exampleblock}{The Challenge (3 min, silent)}
Two metal spheres initially have charges of +4 C and +8 C. After touching each other, one sphere has +10 C.

\vspace{0.3cm}

\textbf{Given:}
\begin{itemize}
\item Blue sphere initial: $\charge{q_1} = +4$ C
\item Red sphere initial: $\charge{q_2} = +8$ C
\item Blue sphere final: $\charge{q_1'} = +10$ C
\end{itemize}

\textbf{Find:} Final \charge{charge} on red sphere $\charge{q_2'}$

\vspace{0.3cm}

\textit{Can you use conservation of charge? Work silently.}
\end{exampleblock}

\note{[THE CHALLENGE] Can they apply conservation law?\\\\
[SAY] "Try this on your own. Use conservation of charge."\\\\
[TIMING] 3-4 min SILENT individual work\\\\
[CIRCULATE] Note who remembers initial equals final\\\\
[WATCH FOR] Students forgetting to add initial charges\\\\
[DON'T HELP] Let them struggle - learning happens in Compare}
\end{frame}

\begin{frame}
\frametitle{Compare: Conservation Strategy}
\textbf{Turn and talk (2 min):}

\vspace{0.3cm}

\begin{enumerate}
\item What law did you use?
\item What is the total initial charge?
\item How did you find the final charge on the red sphere?
\end{enumerate}

\vspace{0.5cm}

\pause
\alert{Name wheel:} One pair share your approach (not your answer).

\note{[TIMING] 2-3 min pair discussion\\\\
[CIRCULATE] Listen for conservation law application\\\\
[CHECK] Name wheel: call a pair to share approach\\\\
[EXPECTED APPROACH] Initial charge equals final charge, solve for unknown\\\\
[COMMON ERROR] Forgetting total initial charge is 4 plus 8}
\end{frame}

\begin{frame}
\frametitle{Reveal: Conservation in Action}
\textbf{Self-correct in a different color:}

\vspace{0.3cm}

\textbf{Law:} $\charge{q}_{\text{initial}} = \charge{q}_{\text{final}}$

\pause
\vspace{0.2cm}

\textbf{Step 1:} Find total initial \charge{charge}
$$\charge{q}_{\text{initial}} = +4\text{ C} + 8\text{ C} = +12\text{ C}$$

\pause
\vspace{0.2cm}

\textbf{Step 2:} Apply conservation law
$$\charge{q}_{\text{final}} = +10\text{ C} + \charge{q_2'}$$

\pause
\vspace{0.2cm}

\textbf{Step 3:} Solve for $\charge{q_2'}$
$$+12\text{ C} = +10\text{ C} + \charge{q_2'}$$
$$\boxed{\charge{q_2'} = +2\text{ C}}$$

\pause
\textbf{Check:} $12 = 10 + 2$ \checkmark \charge{Charge} is conserved!

\note{[P0] "Self-correct in a different color"\\\\
[P1] [ALGEBRA] "Total initial charge equals 4 plus 8 equals 12 coulombs"\\\\
[P2] "Apply conservation: initial equals final"\\\\
[P3] "Final equals 10 plus q-2-prime"\\\\
[P4] [ANSWER] "Red sphere has plus 2 coulombs"\\\\
[P5] "Check: 12 equals 10 plus 2 - conserved!"\\\\
[THE WONDER] Nature's accounting system never fails - charge always balances}
\end{frame}

\section{Practice Application}

\begin{frame}
\frametitle{Attempt: Counting Electrons}
\begin{exampleblock}{The Challenge (3 min, silent)}
An ink droplet in a printer has net \charge{charge} $\charge{q} = -1.0 \times 10^{-10}$ C after passing through an electron beam.

\vspace{0.3cm}

\textbf{Given:}
\begin{itemize}
\item Droplet \charge{charge}: $\charge{q} = -1.0 \times 10^{-10}$ C
\item Electron \charge{charge}: $\charge{e} = -1.602 \times 10^{-19}$ C
\end{itemize}

\textbf{Find:} Number of electrons captured by droplet

\vspace{0.3cm}

\textit{How many electrons does it take? Work silently.}
\end{exampleblock}

\note{[THE CHALLENGE] Can they connect charge to number of particles?\\\\
[SAY] "Try this on your own. Think about quantization."\\\\
[TIMING] 3-4 min SILENT individual work\\\\
[CIRCULATE] Note who uses division vs multiplication\\\\
[WATCH FOR] Sign errors with negative charges\\\\
[DON'T HELP] Let them reason through quantization}
\end{frame}

\begin{frame}
\frametitle{Compare: Quantization Strategy}
\textbf{Turn and talk (2 min):}

\vspace{0.3cm}

\begin{enumerate}
\item What equation relates total \charge{charge} to number of electrons?
\item Did you multiply or divide?
\item What happens with the negative signs?
\end{enumerate}

\vspace{0.5cm}

\pause
\alert{Name wheel:} One pair share your approach (not your answer).

\note{[TIMING] 2-3 min pair discussion\\\\
[CIRCULATE] Listen for Q equals n e reasoning\\\\
[CHECK] Name wheel: call a pair to share\\\\
[EXPECTED APPROACH] Divide total charge by charge per electron\\\\
[COMMON ERROR] Sign confusion with negative charges}
\end{frame}

\begin{frame}
\frametitle{Reveal: Quantization at Work}
\textbf{Self-correct in a different color:}

\vspace{0.3cm}

\textbf{Equation:} $\charge{Q} = \particles{n}\charge{e}$, so $\particles{n} = \frac{\charge{Q}}{\charge{e}}$

\pause
\vspace{0.2cm}

\textbf{Substitute:}
$$\particles{n} = \frac{\charge{q}}{\charge{q_{e^-}}} = \frac{-1.0 \times 10^{-10}\text{ C}}{-1.602 \times 10^{-19}\text{ C}}$$

\pause
\vspace{0.2cm}

\textbf{Simplify:}
$$\particles{n} = \frac{1.0 \times 10^{-10}}{1.602 \times 10^{-19}} = 6.24 \times 10^{8}$$

\pause
\vspace{0.2cm}

$$\boxed{\particles{n} = 6 \times 10^{8} \text{ electrons}}$$

\pause
\textbf{Check:} About 600 million electrons - seems large but atoms have $10^{16}$ atoms!

\note{[P0] "Self-correct in a different color"\\\\
[P1] [ALGEBRA] "n equals Q over e"\\\\
[P2] "Substitute values - negatives cancel"\\\\
[P3] "Divide: 1.0 times 10 to negative 10 over 1.602 times 10 to negative 19"\\\\
[P4] [ANSWER] "6 times 10 to the 8 electrons - about 600 million"\\\\
[P5] "Seems huge but droplet has 10 to the 16 atoms - only tiny fraction charged"\\\\
[THE WONDER] You just counted individual electrons - quantum physics in action}
\end{frame}

\section{Summary}

\begin{frame}
\frametitle{What You Now Know}
\begin{block}{The Revelations}
\begin{enumerate}
\item Two types of \charge{charge}: positive (protons) and negative (electrons) \pause
\item Fundamental \charge{charge}: $\charge{e} = 1.602 \times 10^{-19}$ C \pause
\item \charge{Charge} quantization: $\charge{Q} = \particles{n}\charge{e}$ (only integer multiples) \pause
\item Conservation: $\charge{q}_{\text{initial}} = \charge{q}_{\text{final}}$ (never violated) \pause
\item Conductors let \charge{charge} move; insulators don't \pause
\item Transfer methods: contact, conduction, induction
\end{enumerate}
\end{block}

\note{[P0] "Six revelations today"\\\\
[P1] "Two charge types - positive and negative, like and unlike"\\\\
[P2] "Fundamental charge e - proton plus, electron minus"\\\\
[P3] "Quantization - charge comes in discrete packets"\\\\
[P4] "Conservation - nature's accounting system"\\\\
[P5] "Conductors vs insulators - materials matter"\\\\
[P6] "Three ways to transfer charge"\\\\
[THE WONDER] You now understand the force behind lightning, computers, and your heartbeat\\\\
- Name wheel: which revelation surprised you most?}
\end{frame}

\begin{frame}
\frametitle{Key Equations}
\begin{align}
\charge{e} &= 1.602 \times 10^{-19}\text{ C} && \text{(fundamental charge)} \\
\charge{Q} &= \particles{n}\charge{e} && \text{(charge quantization)} \\
\charge{q}_{\text{initial}} &= \charge{q}_{\text{final}} && \text{(conservation of charge)}
\end{align}

\note{- Three foundational equations\\\\
- Fundamental charge: same everywhere in universe\\\\
- Quantization: charge comes in units of e\\\\
- Conservation: most important - never violated\\\\
- Know when to use each\\\\
- Questions before we end?}
\end{frame}

\begin{frame}
\frametitle{Homework}
\begin{center}
\Large
Complete the assigned problems\\[0.3cm]
posted on the LMS
\end{center}

\note{[SAY] "Homework posted on LMS - charge calculations and concept questions"\\\\
[TIMING] Due date: check LMS\\\\
[CHECK] Questions before we end?\\\\
[TRANSITION] Next class: Coulomb's Law - calculating electric forces}
\end{frame}

\end{document}
