\documentclass{beamer}
\usepackage{../../../shared/templates/ds9_theme}
\usepackage[overridenote]{pdfpc}
\graphicspath{{../images/}{../../shared/images/}}

\title[Sound]{PHYS11 CH:14 Invisible Vibrations}
\subtitle{From Silence to Symphony}
\author[Mr. Gullo]{Mr. Gullo}
\date[December 2025]{December 2025}

\begin{document}

\frame{\titlepage
\note{[THE HOOK] Today we explore invisible vibrations that connect everything.\\\\
- Sound waves carry energy, information, and emotion\\\\
- Same physics explains whispers and explosions\\\\
- Four revelations: speed and wavelength, intensity and decibels, Doppler effect, interference and resonance\\\\
[THE WONDER] By end of class, you'll understand how nature transmits information through matter}
}

\begin{frame}
\frametitle{Outline}
\tableofcontents
\end{frame}

\section{Introduction}

\begin{frame}
\frametitle{The Mystery}
\begin{center}
\Large If a tree falls in the forest\\
and no one is there to hear it,\\
\textit{does it make a sound?}
\end{center}

\pause
\vspace{0.5cm}
The answer depends on how you define sound...

\pause
\vspace{0.3cm}
\alert{Physics says: yes, but no one perceives it.}
\note{[P0] "The old philosophical question"\\\\
[P1] "The answer depends on how you define sound"\\\\
[P2] [THE REVELATION] "Physics: colliding objects disturb matter, creating waves. That's sound. Perception requires an observer."\\\\
[THE CONNECTION - Harmonic Archetype] "Musicians: sound IS vibration. Without matter to vibrate, silence."\\\\
[THE WONDER] In space, no one can hear you scream - no air to carry waves}
\end{frame}

\begin{frame}
\frametitle{Fallen Tree}
\begin{figure}
\centering
\includegraphics[width=0.8\textwidth,height=0.6\textheight,keepaspectratio]{phys11-sound-fig01.jpg}
\end{figure}

\pause
\begin{exampleblock}{The Mental Model}
Tree hits ground $\rightarrow$ disturbs air particles $\rightarrow$ creates pressure waves $\rightarrow$ sound wave travels outward.
\end{exampleblock}
\note{[P0] "Tree fell some time ago"\\\\
[P1] [THE REVELATION] "When it fell, energy disturbed air particles. This disturbance IS sound."\\\\
[THE HUMILITY] "Our ears evolved to detect these vibrations. Sound existed before ears."\\\\
[THE WONDER] Matter remembers the collision through wave motion}
\end{frame}

\section{Speed of Sound, Frequency, and Wavelength}

\begin{frame}
\frametitle{Learning Objectives}
\begin{block}{By the end of this section, you will be able to:}
\begin{itemize}
\item \textbf{14.1:} Relate characteristics of waves to properties of sound \pause
\item \textbf{14.1:} Describe speed of sound and how it changes in media \pause
\item \textbf{14.1:} Relate speed of sound to frequency and wavelength
\end{itemize}
\end{block}
\note{[P0] "Three objectives for speed of sound"\\\\
[P1] "First: connect wave properties to sound"\\\\
[P2] "Second: how speed changes in different materials"\\\\
[P3] "Third: relationship between speed, frequency, wavelength"\\\\
- Assessment: quiz next week}
\end{frame}

\begin{frame}
\frametitle{14.1 Sound as a Mechanical Wave}
\begin{block}{Nature's Rule}
Sound is a disturbance of matter transmitted from source outward as a longitudinal wave.
\end{block}

\pause
\vspace{0.3cm}

\textbf{What makes it sound:}
\begin{itemize}
\item Matter must vibrate (compression and rarefaction) \pause
\item Energy transfers through medium \pause
\item \alert{Requires matter} - no sound in vacuum
\end{itemize}
\note{[P0] [THE REVELATION] "Sound is matter disturbed from equilibrium"\\\\
[P1] "Matter must vibrate - alternating high and low pressure"\\\\
[P2] "Energy transfers through medium"\\\\
[P3] "Requires matter - Star Wars battles in space are fiction"\\\\
[THE CONFLICT] Movies lie - explosions in space would be silent\\\\
[THE WONDER] Sound is the universe speaking through matter}
\end{frame}

\begin{frame}
\frametitle{14.1 Vibrating String Creates Sound}
\begin{figure}
\centering
\includegraphics[width=0.7\textwidth,height=0.5\textheight,keepaspectratio]{phys11-sound-fig02.jpg}
\end{figure}

\pause
String oscillates $\rightarrow$ compresses air $\rightarrow$ creates pressure waves $\rightarrow$ longitudinal sound wave
\note{[P0] "Vibrating string moving to the right"\\\\
[P1] "Compresses air in front, expands air behind"\\\\
- Creates alternating high pressure (compression) and low pressure (rarefaction)\\\\
- Energy from string transferred to air as sound\\\\
[THE CONNECTION - Harmonic Archetype] "Guitarists: your strings create these exact pressure waves"}
\end{frame}

\begin{frame}
\frametitle{14.1 Compressions and Rarefactions}
\begin{figure}
\centering
\includegraphics[width=0.7\textwidth,height=0.45\textheight,keepaspectratio]{phys11-sound-fig04.jpg}
\end{figure}

\pause
\begin{exampleblock}{Analogy to Transverse Waves}
\begin{itemize}
\item Compression = crest (high pressure)
\item Rarefaction = trough (low pressure)
\item Wavelength = distance between compressions
\end{itemize}
\end{exampleblock}
\note{[P0] "Graph shows gauge pressure vs distance"\\\\
[P1] "Compression is like crest, rarefaction is like trough"\\\\
- Wavelength: one complete compression-rarefaction cycle\\\\
- Longitudinal wave oscillates parallel to direction of travel\\\\
[THE REVELATION] Same wave math, different geometry}
\end{frame}

\begin{frame}
\frametitle{14.1 Sound Wave Enters Ear}
\begin{figure}
\centering
\includegraphics[width=0.6\textwidth,height=0.45\textheight,keepaspectratio]{phys11-sound-fig05.jpg}
\end{figure}

\pause
Compressions and rarefactions force eardrum to vibrate $\rightarrow$ converted to nerve impulses $\rightarrow$ brain interprets as sound
\note{[P0] "Sound wave travels up ear canal"\\\\
[P1] "Pressure differences force eardrum to vibrate"\\\\
- Net force on eardrum from pressure variations\\\\
- Complicated mechanism converts vibrations to neural signals\\\\
[THE WONDER] Your ear is a pressure sensor evolved to decode vibrations}
\end{frame}

\begin{frame}
\frametitle{14.1 Speed of Sound}
\begin{alertblock}{The Intuition Trap}
\textbf{Your brain expects:} Denser material = slower sound\\
\textbf{Reality:} Speed depends on BOTH rigidity and density.
\end{alertblock}

\pause
\vspace{0.3cm}

\textbf{The rules:}
\begin{itemize}
\item More rigid (less compressible) = faster sound \pause
\item Greater density = slower sound \pause
\item Solids: very rigid, so sound travels FAST despite density
\end{itemize}
\note{[P0] [THE CONFLICT] "Your intuition lies about dense materials"\\\\
[P1] "More rigid means faster"\\\\
[P2] "Greater density means slower"\\\\
[P3] "Solids win: rigidity beats density"\\\\
[THE HUMILITY] "This confused me when I first learned it"\\\\
[THE REVELATION] Steel transmits sound 15 times faster than air}
\end{frame}

\begin{frame}
\frametitle{14.1 Fireworks and Light vs Sound}
\begin{figure}
\centering
\includegraphics[width=0.7\textwidth,height=0.5\textheight,keepaspectratio]{phys11-sound-fig06.jpg}
\end{figure}

\pause
You see the flash BEFORE you hear the boom. Why?

\pause
Light: $3 \times 10^8$ m/s \quad Sound: $\sim 340$ m/s in air
\note{[P0] "Fireworks display"\\\\
[P1] "You see flash before hearing boom - why?"\\\\
[P2] "Light travels nearly a million times faster than sound"\\\\
- Light: 300,000 km/s. Sound: 0.34 km/s\\\\
- Thunder and lightning: same phenomenon\\\\
[THE CONNECTION - Digital Archetype] "Gamers: network lag is like this - signal delay"}
\end{frame}

\begin{frame}
\frametitle{14.1 The Universal Wave Equation}
\begin{block}{The Law of All Waves}
\begin{center}
\Large $\boxed{v = f\lambda}$
\end{center}
Speed equals frequency times wavelength.
\end{block}

\pause
\vspace{0.3cm}

\textbf{For sound:}
\begin{itemize}
\item $v$ = speed of sound (m/s) - depends on medium
\item $f$ = frequency (Hz) - set by source
\item $\lambda$ = wavelength (m) - adjusts automatically
\end{itemize}
\note{[P0] [THE REVELATION] "Same equation for ALL waves"\\\\
[P1] "For sound: v depends on medium, f depends on source"\\\\
- In a given medium, v is constant\\\\
- So if f increases, lambda must decrease\\\\
- Inverse relationship between frequency and wavelength\\\\
[THE WONDER] One equation connects pitch to distance}
\end{frame}

\begin{frame}
\frametitle{14.1 Sound Wave Anatomy}
\begin{figure}
\centering
\includegraphics[width=0.7\textwidth,height=0.45\textheight,keepaspectratio]{phys11-sound-fig07.jpg}
\end{figure}

\pause
Source vibrates at frequency $f$ $\rightarrow$ propagates at $v$ $\rightarrow$ wavelength $\lambda$

\pause
Distance between adjacent compressions = one wavelength
\note{[P0] "Sound wave emanating from source"\\\\
[P1] "Source vibrates at f, wave propagates at v, wavelength is lambda"\\\\
[P2] "Distance between compressions equals one wavelength"\\\\
- Frequency of wave equals frequency of source\\\\
- Tuning fork at 256 Hz produces 256 Hz sound\\\\
[THE REVELATION] Source controls tempo, medium controls speed}
\end{frame}

\begin{frame}
\frametitle{14.1 Speed Independent of Frequency}
\textbf{Critical fact:} Speed of sound is nearly independent of frequency.

\pause
\vspace{0.3cm}

\textbf{Why this matters:}

If high frequencies traveled faster than low frequencies, you'd hear the flute BEFORE the tuba at a concert!

\pause
\vspace{0.3cm}

But all instruments arrive in sync, regardless of distance.

\pause
\begin{exampleblock}{The Consequence}
Since $v = f\lambda$ and $v$ is constant, higher frequency means shorter wavelength.
\end{exampleblock}
\note{[P0] "Speed is nearly independent of frequency"\\\\
[P1] "If not true, high pitch would arrive before low pitch"\\\\
[P2] "But music arrives in cadence regardless of distance"\\\\
[P3] "So frequency and wavelength are inversely related"\\\\
[THE CONNECTION - Harmonic Archetype] "Musicians: this is why orchestras sound coherent"\\\\
[THE WONDER] Nature preserves musical harmony through constant speed}
\end{frame}

\begin{frame}
\frametitle{Attempt: Decoding Audible Sound}
\begin{exampleblock}{The Challenge (3 min, silent)}
Calculate the wavelengths of sounds at the extremes of human hearing, 20 Hz and 20,000 Hz, when sound travels at 348.7 m/s.

\vspace{0.3cm}

\textbf{Given:}
\begin{itemize}
\item $v = 348.7$ m/s
\item $f_{\text{min}} = 20$ Hz, $f_{\text{max}} = 20,000$ Hz
\end{itemize}

\textbf{Find:} $\lambda_{\text{max}}$ and $\lambda_{\text{min}}$

\vspace{0.3cm}

\textit{Can you decode the range? Work silently.}
\end{exampleblock}
\note{[THE CHALLENGE] Can they map frequency to wavelength?\\\\
[SAY] "Try this on your own. It's okay to get stuck."\\\\
[TIMING] 3-4 min SILENT individual work\\\\
[CIRCULATE] Note who uses correct equation, who inverts fraction\\\\
[WATCH FOR] Confusion about which frequency gives max wavelength\\\\
[DON'T HELP] Let them struggle - learning happens in Compare}
\end{frame}

\begin{frame}
\frametitle{Compare: Wavelength Calculation}
\textbf{Turn and talk (2 min):}

\vspace{0.3cm}

\begin{enumerate}
\item What equation connects speed, frequency, and wavelength?
\item How did you solve for wavelength?
\item Which frequency gives the LONGEST wavelength? Why?
\end{enumerate}

\vspace{0.5cm}

\pause
\alert{Name wheel:} One pair share your approach (not your answer).
\note{[TIMING] 2-3 min pair discussion\\\\
[CIRCULATE] Listen for correct equation and rearrangement\\\\
[CHECK] Name wheel: call a pair to share approach\\\\
[EXPECTED APPROACH] Use v equals f lambda, solve for lambda, lowest frequency gives longest wavelength\\\\
[COMMON ERROR] Confusing which frequency gives max vs min wavelength}
\end{frame}

\begin{frame}
\frametitle{Reveal: The Range of Human Hearing}
\textbf{Self-correct in a different color:}

\vspace{0.3cm}

\textbf{Equation:} $v = f\lambda$ $\rightarrow$ $\lambda = \frac{v}{f}$

\pause
\vspace{0.2cm}

\textbf{Maximum wavelength (lowest frequency):}
$$\lambda_{\text{max}} = \frac{348.7 \text{ m/s}}{20 \text{ Hz}} = 17.4 \text{ m} \approx 20 \text{ m}$$

\pause
\vspace{0.2cm}

\textbf{Minimum wavelength (highest frequency):}
$$\lambda_{\text{min}} = \frac{348.7 \text{ m/s}}{20,000 \text{ Hz}} = 0.017 \text{ m} \approx 2 \text{ cm}$$

\pause
\textbf{Check:} Deep bass (20 Hz) has wavelength of a bus. High treble (20 kHz) is the size of your thumb.
\note{[P0] "Self-correct in different color"\\\\
[P1] [ALGEBRA] "Lambda equals v over f"\\\\
[P2] "Lowest frequency: 348.7 divided by 20 equals 17.4 m"\\\\
[P3] "Highest frequency: 348.7 divided by 20,000 equals 0.017 m"\\\\
[P4] [ANSWER] "Range: 2 cm to 20 m - three orders of magnitude"\\\\
[THE WONDER] Your ears decode vibrations from thumb-size to bus-size}
\end{frame}

\section{Sound Intensity and Decibels}

\begin{frame}
\frametitle{Learning Objectives}
\begin{block}{By the end of this section, you will be able to:}
\begin{itemize}
\item \textbf{14.2:} Relate amplitude to loudness and energy \pause
\item \textbf{14.2:} Describe the decibel scale for measuring intensity \pause
\item \textbf{14.2:} Solve problems involving sound intensity \pause
\item \textbf{14.2:} Describe how humans produce and hear sounds
\end{itemize}
\end{block}
\note{[P0] "Four objectives for sound intensity"\\\\
[P1] "First: amplitude drives loudness and energy"\\\\
[P2] "Second: decibel scale - logarithmic not linear"\\\\
[P3] "Third: calculate intensity problems"\\\\
[P4] "Fourth: biology of sound production and hearing"}
\end{frame}

\begin{frame}
\frametitle{14.2 Loudness and Amplitude}
\begin{figure}
\centering
\includegraphics[width=0.6\textwidth,height=0.4\textheight,keepaspectratio]{phys11-sound-fig09.jpg}
\end{figure}

\pause
Loudness relates to how energetically the source vibrates.

\pause
\begin{exampleblock}{The Connection}
Louder sound = greater amplitude = more energy transferred
\end{exampleblock}
\note{[P0] "Crowded roadway in Delhi - hard to hear without shouting"\\\\
[P1] "Loudness relates to energy of vibration"\\\\
[P2] "Greater amplitude means more energy"\\\\
[THE CONNECTION - Harmonic Archetype] "Musicians: forte means hit harder, more amplitude"\\\\
[THE WONDER] Screaming person vibrates vocal cords more energetically}
\end{frame}

\begin{frame}
\frametitle{14.2 Sound Intensity}
\begin{block}{Universal Law: Intensity}
\begin{center}
$I = \frac{P}{A}$
\end{center}
Intensity equals power per unit area.
\end{block}

\pause
\vspace{0.3cm}

\begin{itemize}
\item $I$ = intensity (W/m$^2$) - energy flow per area
\item $P$ = power (W) - rate of energy transfer
\item $A$ = area (m$^2$) - surface through which wave passes
\end{itemize}

\pause
\vspace{0.3cm}

\textbf{Key insight:} $I \propto (\Delta p)^2$ where $\Delta p$ is pressure amplitude.

Intensity is proportional to amplitude squared!
\note{[P0] [THE REVELATION] "Intensity is power per area"\\\\
[P1] "Power is rate of energy transfer, area is surface"\\\\
[P2] "Intensity proportional to amplitude squared"\\\\
- Double amplitude means four times energy\\\\
- Triple amplitude means nine times energy\\\\
[THE WONDER] Tiny amplitude changes make huge energy differences}
\end{frame}

\begin{frame}
\frametitle{14.2 Pressure Amplitude Graphs}
\begin{figure}
\centering
\includegraphics[width=0.7\textwidth,height=0.5\textheight,keepaspectratio]{phys11-sound-fig10.jpg}
\end{figure}

\pause
More intense sound has larger pressure maxima and minima, greater forces on objects.
\note{[P0] "Two sound waves with different intensities"\\\\
[P1] "More intense sound has larger amplitude oscillations"\\\\
- Greater pressure variations mean larger forces\\\\
- Louder sound literally pushes harder\\\\
[THE REVELATION] Loudness is force}
\end{frame}

\begin{frame}
\frametitle{14.2 Why Decibels?}
\begin{alertblock}{Civilian View vs. Reality}
\textbf{Civilian:} "Intensity in W/m$^2$ makes sense."\\
\textbf{Physicist:} "Human ears perceive logarithmically, not linearly."
\end{alertblock}

\pause
\vspace{0.3cm}

Ears respond to RATIO of intensities, not difference.

\pause
\begin{block}{The Decibel Scale}
\begin{center}
$\beta \text{ (dB)} = 10 \log_{10}\left(\frac{I}{I_0}\right)$
\end{center}
where $I_0 = 10^{-12}$ W/m$^2$ (threshold of human hearing)
\end{block}
\note{[P0] [THE CONFLICT] "W/m squared is SI unit, but not how ears work"\\\\
[P1] "Ears respond to RATIO - logarithmic perception"\\\\
[P2] "Decibel scale: 10 times log of intensity ratio"\\\\
- Reference: I-naught is quietest sound humans can hear\\\\
- Zero dB is threshold of hearing, not silence\\\\
[THE WONDER] Your brain compresses enormous range into manageable scale}
\end{frame}

\begin{frame}
\frametitle{14.2 Understanding the Decibel Scale}
\textbf{Key patterns:}

\begin{itemize}
\item Each factor of 10 in intensity = 10 dB \pause
\item 90 dB is $10^3$ times more intense than 60 dB \pause
\item Doubling intensity adds about 3 dB \pause
\item 0 dB = threshold of hearing ($10^{-12}$ W/m$^2$)
\end{itemize}

\pause
\vspace{0.3cm}

\begin{exampleblock}{Examples}
\begin{itemize}
\item Whisper: 20 dB
\item Conversation: 60 dB
\item Rock concert: 120 dB (pain threshold)
\end{itemize}
\end{exampleblock}
\note{[P0] "Key patterns in decibel scale"\\\\
[P1] "Factor of 10 in intensity equals 10 dB"\\\\
[P2] "90 dB is thousand times more intense than 60 dB"\\\\
[P3] "Doubling intensity adds 3 dB"\\\\
[P4] "Examples from whisper to rock concert"\\\\
[THE HUMILITY] "120 dB causes pain - protect your hearing"\\\\
[THE WONDER] Scale compresses trillion-to-one range into 0-120}
\end{frame}

\begin{frame}
\frametitle{Attempt: Calculating Decibels}
\begin{exampleblock}{The Challenge (3 min, silent)}
A sound wave in air at 0°C has pressure amplitude 0.656 Pa. Calculate the sound intensity level in decibels.

\vspace{0.3cm}

\textbf{Given:}
\begin{itemize}
\item $\Delta p = 0.656$ Pa
\item $v = 331$ m/s (air at 0°C)
\item $\rho = 1.29$ kg/m$^3$ (air density)
\item $I_0 = 10^{-12}$ W/m$^2$
\end{itemize}

\textbf{Find:} $\beta$ in dB

\vspace{0.3cm}

\textit{Two steps: find intensity, then decibels. Work silently.}
\end{exampleblock}
\note{[THE CHALLENGE] Can they chain two formulas?\\\\
[SAY] "Two-step problem. Find intensity first, then decibels."\\\\
[TIMING] 3-4 min SILENT work\\\\
[CIRCULATE] Note who remembers intensity formula, who forgets log\\\\
[WATCH FOR] Calculator errors with scientific notation\\\\
[DON'T HELP] Let them struggle with formula lookup}
\end{frame}

\begin{frame}
\frametitle{Compare: Decibel Calculation}
\textbf{Turn and talk (2 min):}

\vspace{0.3cm}

\begin{enumerate}
\item What formula did you use to find intensity from pressure?
\item What values did you substitute?
\item What formula converts intensity to decibels?
\end{enumerate}

\vspace{0.5cm}

\pause
\alert{Name wheel:} One pair share your approach.
\note{[TIMING] 2-3 min pair discussion\\\\
[CIRCULATE] Listen for two-step approach\\\\
[CHECK] Name wheel: call pair to share\\\\
[EXPECTED APPROACH] Use I equals delta-p squared over 2 rho v, then beta equals 10 log of I over I-naught\\\\
[COMMON ERROR] Forgetting to square pressure amplitude}
\end{frame}

\begin{frame}[shrink]
\frametitle{Reveal: From Pressure to Decibels}
\textbf{Self-correct in a different color:}

\vspace{0.2cm}

\textbf{Step 1 - Find intensity:} $I = \frac{(\Delta p)^2}{2\rho v}$

\pause
$$I = \frac{(0.656 \text{ Pa})^2}{2(1.29 \text{ kg/m}^3)(331 \text{ m/s})} = 5.04 \times 10^{-4} \text{ W/m}^2$$

\pause
\vspace{0.2cm}

\textbf{Step 2 - Convert to decibels:} $\beta = 10\log_{10}\left(\frac{I}{I_0}\right)$

\pause
$$\beta = 10\log_{10}\left(\frac{5.04 \times 10^{-4}}{10^{-12}}\right) = 10\log_{10}(5.04 \times 10^8)$$

\pause
$$\boxed{\beta = 87.0 \text{ dB}}$$

\pause
\textbf{Check:} 87 dB is about as loud as heavy traffic - reasonable.
\note{[P0] "Self-correct in different color"\\\\
[P1] [ALGEBRA] "I equals delta-p squared over 2 rho v"\\\\
[P2] "Substitute: 0.656 squared over 2 times 1.29 times 331"\\\\
[P3] "Beta equals 10 log of I over I-naught"\\\\
[P4] "Log of 5.04 times 10 to the 8 equals 8.70"\\\\
[P5] [ANSWER] "87 dB - heavy traffic level"\\\\
[THE WONDER] Pressure amplitude under 1 Pa creates loud sound}
\end{frame}

\begin{frame}
\frametitle{14.2 How We Hear}
\begin{figure}
\centering
\includegraphics[width=0.6\textwidth,height=0.45\textheight,keepaspectratio]{phys11-sound-fig11.jpg}
\end{figure}

\pause
Sound waves $\rightarrow$ eardrum vibrates $\rightarrow$ bones amplify $\rightarrow$ cochlea converts to electrical signals $\rightarrow$ brain interprets
\note{[P0] "Anatomy of the ear"\\\\
[P1] "Sound pressure wave hits eardrum, vibrates bones, stimulates cochlea"\\\\
- Outer ear: collects sound\\\\
- Middle ear: amplifies with lever system\\\\
- Inner ear: converts mechanical to electrical\\\\
[THE WONDER] Your ear is transducer - pressure to electricity}
\end{frame}

\section{Doppler Effect and Sonic Booms}

\begin{frame}
\frametitle{Learning Objectives}
\begin{block}{By the end of this section, you will be able to:}
\begin{itemize}
\item \textbf{14.3:} Describe the Doppler effect of sound waves \pause
\item \textbf{14.3:} Explain a sonic boom \pause
\item \textbf{14.3:} Calculate frequency shift using Doppler formula
\end{itemize}
\end{block}
\note{[P0] "Three objectives for Doppler effect"\\\\
[P1] "First: what is Doppler effect and why it happens"\\\\
[P2] "Second: sonic booms from supersonic motion"\\\\
[P3] "Third: calculate observed frequency"\\\\
- Real-world applications: radar, astronomy, medical ultrasound}
\end{frame}

\begin{frame}
\frametitle{Why Ambulances Lie to You}
\begin{center}
\Large The siren isn't changing pitch.\\
\textit{Your ears are being fooled.}
\end{center}

\pause
\vspace{0.5cm}

\begin{alertblock}{The Illusion}
Ambulance plays one constant note. You hear two different pitches approaching vs. receding.

What's happening?
\end{alertblock}
\note{[P0] [THE HOOK] "The siren isn't changing pitch"\\\\
[P1] "Ambulance plays constant note, you hear pitch shift"\\\\
- Approaching: higher pitch\\\\
- Receding: lower pitch\\\\
[THE CONFLICT] Your brain interprets this as changing frequency\\\\
[THE WONDER] Same effect lets us measure galaxy speeds - redshift}
\end{frame}

\begin{frame}
\frametitle{14.3 Stationary Source and Observers}
\begin{figure}
\centering
\includegraphics[width=0.7\textwidth,height=0.5\textheight,keepaspectratio]{phys11-sound-fig14.jpg}
\end{figure}

\pause
When source and observers are stationary, wavelength and frequency are same in all directions.
\note{[P0] "Sound emitted by stationary source"\\\\
[P1] "Spherical waves spread out uniformly"\\\\
- Wavelength same in all directions\\\\
- Frequency same for all observers\\\\
- This is the baseline - no Doppler shift}
\end{frame}

\begin{frame}
\frametitle{14.3 Moving Source}
\begin{figure}
\centering
\includegraphics[width=0.7\textwidth,height=0.5\textheight,keepaspectratio]{phys11-sound-fig15.jpg}
\end{figure}

\pause
Source moving right $\rightarrow$ waves bunch up ahead, spread out behind

\pause
Observer on right: shorter $\lambda$, higher $f$

Observer on left: longer $\lambda$, lower $f$
\note{[P0] "Source moving to the right"\\\\
[P1] "Waves emitted while source was at different positions"\\\\
[P2] "Right side: compressed waves, higher frequency. Left side: stretched waves, lower frequency"\\\\
- Source chases its own waves on right\\\\
- Source runs away from waves on left\\\\
[THE REVELATION] Motion changes wavelength, not wave speed}
\end{frame}

\begin{frame}
\frametitle{14.3 The Doppler Effect Formula}
\begin{block}{For Moving Source, Stationary Observer}
\begin{center}
$f_{\text{obs}} = f_s\left(\frac{v_w}{v_w \pm v_s}\right)$
\end{center}
Use minus for motion toward observer, plus for motion away.
\end{block}

\pause
\vspace{0.3cm}

\textbf{Intuition check:}
\begin{itemize}
\item Source approaching: denominator smaller $\rightarrow$ frequency higher
\item Source receding: denominator larger $\rightarrow$ frequency lower
\end{itemize}
\note{[P0] [THE REVELATION] "Observed frequency depends on source motion"\\\\
[P1] "Approaching: minus sign makes denominator smaller, frequency higher"\\\\
- Receding: plus sign makes denominator larger, frequency lower\\\\
- Greater source speed means greater shift\\\\
[THE HUMILITY] "Easier to understand intuition than memorize signs"}
\end{frame}

\begin{frame}
\frametitle{14.3 Sonic Booms}
What happens when source speed approaches sound speed?

\pause
\vspace{0.3cm}

As $v_s \rightarrow v_w$, denominator in $f_{\text{obs}} = f_s\left(\frac{v_w}{v_w - v_s}\right)$ approaches zero...

\pause
Observed frequency approaches infinity!

\pause
\vspace{0.3cm}

\begin{block}{Sonic Boom}
Constructive interference of sound created by object moving faster than sound.

All waves superimpose at same instant $\rightarrow$ huge amplitude $\rightarrow$ BOOM!
\end{block}
\note{[P0] "What happens at speed of sound?"\\\\
[P1] "Denominator approaches zero"\\\\
[P2] "Frequency approaches infinity"\\\\
[P3] [THE REVELATION] "All waves stack up, constructive interference creates boom"\\\\
- Aircraft creates two booms: nose and tail\\\\
- Supersonic flights banned over cities - windows break\\\\
[THE WONDER] Breaking sound barrier is breaking wave spacing}
\end{frame}

\begin{frame}
\frametitle{14.3 Sonic Boom Geometry}
\begin{figure}
\centering
\includegraphics[width=0.7\textwidth,height=0.5\textheight,keepaspectratio]{phys11-sound-fig17.jpg}
\end{figure}

\pause
Two booms: one from nose, one from tail. Time separation equals time for aircraft to pass by a point.
\note{[P0] "Sonic boom from aircraft"\\\\
[P1] "Nose boom and tail boom separated by aircraft length"\\\\
- Observers on ground hear booms AFTER aircraft passes\\\\
- Shock wave travels at angle to flight path\\\\
[THE CONNECTION - Digital Archetype] "Gamers: like Mach cone in flight sims"}
\end{frame}

\begin{frame}
\frametitle{Attempt: Doppler Shift Calculation}
\begin{exampleblock}{The Challenge (3 min, silent)}
A train has 150 Hz horn and moves at 35 m/s. Speed of sound is 340 m/s. What frequencies are observed by stationary person as train approaches and recedes?

\vspace{0.3cm}

\textbf{Given:}
\begin{itemize}
\item $f_s = 150$ Hz
\item $v_s = 35$ m/s
\item $v_w = 340$ m/s
\end{itemize}

\textbf{Find:} $f_{\text{obs, approaching}}$ and $f_{\text{obs, receding}}$

\vspace{0.3cm}

\textit{Which sign for approaching? Receding? Work silently.}
\end{exampleblock}
\note{[THE CHALLENGE] Can they apply Doppler formula correctly?\\\\
[SAY] "Two calculations - approaching and receding. Watch your signs."\\\\
[TIMING] 3-4 min SILENT work\\\\
[CIRCULATE] Note who uses wrong sign, who mixes up numerator/denominator\\\\
[WATCH FOR] Forgetting which sign for which direction\\\\
[DON'T HELP] Let them work through sign logic}
\end{frame}

\begin{frame}
\frametitle{Compare: Doppler Strategy}
\textbf{Turn and talk (2 min):}

\vspace{0.3cm}

\begin{enumerate}
\item Which formula did you use?
\item Approaching: plus or minus sign? Why?
\item Receding: plus or minus sign? Why?
\item How did you check if answer was reasonable?
\end{enumerate}

\vspace{0.5cm}

\pause
\alert{Name wheel:} One pair share your sign logic.
\note{[TIMING] 2-3 min pair discussion\\\\
[CIRCULATE] Listen for correct sign reasoning\\\\
[CHECK] Name wheel: call pair to explain signs\\\\
[EXPECTED APPROACH] Minus for approaching (higher f), plus for receding (lower f)\\\\
[COMMON ERROR] Backwards signs - check by asking "should frequency increase or decrease?"}
\end{frame}

\begin{frame}
\frametitle{Reveal: Train Horn Doppler Shift}
\textbf{Self-correct in a different color:}

\vspace{0.3cm}

\textbf{Approaching (use minus):} $f_{\text{obs}} = f_s\left(\frac{v_w}{v_w - v_s}\right)$

\pause
$$f_{\text{obs}} = 150\left(\frac{340}{340-35}\right) = 150\left(\frac{340}{305}\right) = 167 \text{ Hz} \approx 170 \text{ Hz}$$

\pause
\vspace{0.2cm}

\textbf{Receding (use plus):} $f_{\text{obs}} = f_s\left(\frac{v_w}{v_w + v_s}\right)$

\pause
$$f_{\text{obs}} = 150\left(\frac{340}{340+35}\right) = 150\left(\frac{340}{375}\right) = 136 \text{ Hz} \approx 140 \text{ Hz}$$

\pause
\textbf{Check:} Shift up by 20 Hz, down by 10 Hz. Asymmetric - correct!
\note{[P0] "Self-correct in different color"\\\\
[P1] "Approaching: 150 times 340 over 305 equals 167 Hz"\\\\
[P2] "Receding: 150 times 340 over 375 equals 136 Hz"\\\\
[P3] [ANSWER] "170 Hz approaching, 140 Hz receding"\\\\
[P4] "Shifts not symmetric - physics predicts this"\\\\
[THE WONDER] You just calculated what your ears detect automatically}
\end{frame}

\section{Sound Interference and Resonance}

\begin{frame}
\frametitle{Learning Objectives}
\begin{block}{By the end of this section, you will be able to:}
\begin{itemize}
\item \textbf{14.4:} Describe resonance and beats \pause
\item \textbf{14.4:} Define fundamental frequency and harmonics \pause
\item \textbf{14.4:} Contrast open-pipe and closed-pipe resonators \pause
\item \textbf{14.4:} Solve problems involving harmonics and beat frequency
\end{itemize}
\end{block}
\note{[P0] "Four objectives for interference and resonance"\\\\
[P1] "First: resonance and beats from superposition"\\\\
[P2] "Second: fundamental and overtones"\\\\
[P3] "Third: open vs closed pipes"\\\\
[P4] "Fourth: calculate harmonic frequencies and beats"}
\end{frame}

\begin{frame}
\frametitle{14.4 Resonance}
\begin{block}{Universal Law: Resonance}
Systems oscillate best at their natural frequency. Driving a system at its natural frequency produces resonance.
\end{block}

\pause
\vspace{0.3cm}

\textbf{Examples:}
\begin{itemize}
\item Piano strings vibrate when you sing at their frequency \pause
\item Child on swing pushed at swing's natural frequency \pause
\item Tuning fork and air column resonate at matching frequency
\end{itemize}

\pause
\begin{exampleblock}{The Mental Model}
Resonance is when driving frequency matches natural frequency, creating maximum energy transfer.
\end{exampleblock}
\note{[P0] [THE REVELATION] "Resonance: driving at natural frequency"\\\\
[P1] "Piano strings sing back at you"\\\\
[P2] "Swing goes highest when pushed at right tempo"\\\\
[P3] "Tuning fork makes air column sing"\\\\
[P4] "Maximum energy transfer at resonance"\\\\
[THE WONDER] Universe has preferred frequencies - matter remembers}
\end{frame}

\begin{frame}
\frametitle{14.4 Paddle Ball Resonance}
\begin{figure}
\centering
\includegraphics[width=0.6\textwidth,height=0.45\textheight,keepaspectratio]{phys11-sound-fig18.jpg}
\end{figure}

\pause
Move finger at ball's natural frequency $\rightarrow$ amplitude grows dramatically

\pause
Move too slow or too fast $\rightarrow$ amplitude stays small
\note{[P0] "Paddle ball on elastic band"\\\\
[P1] "At natural frequency, amplitude increases with each cycle"\\\\
[P2] "Away from natural frequency, little response"\\\\
- Resonance curve: peak at natural frequency\\\\
- Energy transfer most efficient at resonance\\\\
[THE CONNECTION - Kinetic Archetype] "Athletes: timing is everything"}
\end{frame}

\begin{frame}
\frametitle{14.4 Beat Frequency}
\begin{block}{Beats from Superposition}
Two waves with slightly different frequencies superimpose $\rightarrow$ alternating constructive and destructive interference $\rightarrow$ amplitude varies in time.
\end{block}

\pause
\vspace{0.3cm}

\begin{center}
$f_B = |f_1 - f_2|$
\end{center}

\pause
You hear average frequency getting louder and softer at beat frequency.
\note{[P0] [THE REVELATION] "Beats: interference pattern in time"\\\\
[P1] "Beat frequency equals difference of frequencies"\\\\
[P2] "Hear average pitch wobbling at beat rate"\\\\
- Piano tuners listen for beats\\\\
- When beats disappear, strings in tune\\\\
[THE CONNECTION - Harmonic Archetype] "Musicians: beating tells you you're out of tune"}
\end{frame}

\begin{frame}
\frametitle{14.4 Beat Pattern}
\begin{figure}
\centering
\includegraphics[width=0.7\textwidth,height=0.5\textheight,keepaspectratio]{phys11-sound-fig20.jpg}
\end{figure}

\pause
Amplitude oscillates at beat frequency while wave oscillates at average frequency.
\note{[P0] "Superposition of two slightly different frequencies"\\\\
[P1] "Amplitude envelope oscillates at beat frequency"\\\\
- Constructive interference: amplitude maximum\\\\
- Destructive interference: amplitude minimum\\\\
- Wave frequency is average, beat frequency is difference\\\\
[THE WONDER] Time-domain interference pattern}
\end{frame}

\begin{frame}
\frametitle{14.4 Standing Waves in Closed Pipe}
\begin{figure}
\centering
\includegraphics[width=0.6\textwidth,height=0.4\textheight,keepaspectratio]{phys11-sound-fig23.jpg}
\end{figure}

\pause
Closed end: node (no displacement)

Open end: antinode (maximum displacement)

\pause
Fundamental: $\lambda = 4L$ so $f_1 = \frac{v}{4L}$
\note{[P0] "Tube closed at one end"\\\\
[P1] "Closed end has node, open end has antinode"\\\\
[P2] "Fundamental wavelength is 4 times length"\\\\
- Quarter wavelength fits in tube\\\\
- Disturbance reflects from closed end\\\\
- Standing wave forms from interference\\\\
[THE REVELATION] Boundary conditions determine frequencies}
\end{frame}

\begin{frame}
\frametitle{14.4 Harmonics in Closed Pipe}
\begin{figure}
\centering
\includegraphics[width=0.7\textwidth,height=0.5\textheight,keepaspectratio]{phys11-sound-fig26.jpg}
\end{figure}

\pause
\textbf{Closed-pipe resonator:} $f_n = n\frac{v}{4L}$ where $n = 1, 3, 5, \ldots$

Only odd harmonics!
\note{[P0] "Fundamental and first three overtones"\\\\
[P1] "Only odd multiples of fundamental"\\\\
- Must have node at closed end, antinode at open end\\\\
- Only odd harmonics satisfy boundary conditions\\\\
- Clarinet is closed-pipe resonator\\\\
[THE WONDER] Geometry restricts allowed frequencies}
\end{frame}

\begin{frame}
\frametitle{14.4 Open-Pipe Resonator}
\begin{figure}
\centering
\includegraphics[width=0.7\textwidth,height=0.5\textheight,keepaspectratio]{phys11-sound-fig27.jpg}
\end{figure}

\pause
Both ends: antinodes (maximum displacement)

\pause
\textbf{Open-pipe resonator:} $f_n = n\frac{v}{2L}$ where $n = 1, 2, 3, \ldots$

All harmonics!
\note{[P0] "Tube open at both ends"\\\\
[P1] "Antinodes at both ends"\\\\
[P2] "All integer multiples of fundamental"\\\\
- Fundamental frequency twice that of closed pipe same length\\\\
- Richer sound - more overtones\\\\
- Flute and organ pipes are open resonators\\\\
[THE REVELATION] Open pipe has even and odd harmonics}
\end{frame}

\begin{frame}
\frametitle{Attempt: Closed-Pipe Length}
\begin{exampleblock}{The Challenge (3 min, silent)}
Sound travels at 344 m/s. What length should a tube closed at one end have for fundamental frequency of 128 Hz?

\vspace{0.3cm}

\textbf{Given:}
\begin{itemize}
\item $f_1 = 128$ Hz (fundamental)
\item $v = 344$ m/s
\end{itemize}

\textbf{Find:} Length $L$

\vspace{0.3cm}

\textit{Which formula for closed pipe? Work silently.}
\end{exampleblock}
\note{[THE CHALLENGE] Can they rearrange harmonic formula?\\\\
[SAY] "Closed pipe - remember boundary conditions."\\\\
[TIMING] 3-4 min SILENT work\\\\
[CIRCULATE] Note who uses open-pipe formula by mistake\\\\
[WATCH FOR] Confusion about n equals 1 for fundamental\\\\
[DON'T HELP] Let them look up correct formula}
\end{frame}

\begin{frame}
\frametitle{Compare: Pipe Length Strategy}
\textbf{Turn and talk (2 min):}

\vspace{0.3cm}

\begin{enumerate}
\item Closed pipe or open pipe formula?
\item What is $n$ for the fundamental?
\item How did you solve for length $L$?
\end{enumerate}

\vspace{0.5cm}

\pause
\alert{Name wheel:} One pair share your approach.
\note{[TIMING] 2-3 min pair discussion\\\\
[CIRCULATE] Listen for closed-pipe formula\\\\
[CHECK] Name wheel: call pair to share\\\\
[EXPECTED APPROACH] Use f-1 equals v over 4L, solve for L, n equals 1\\\\
[COMMON ERROR] Using open-pipe formula or wrong value of n}
\end{frame}

\begin{frame}
\frametitle{Reveal: Tube Length for Resonance}
\textbf{Self-correct in a different color:}

\vspace{0.3cm}

\textbf{Closed-pipe formula:} $f_n = n\frac{v}{4L}$ with $n = 1$ for fundamental

\pause
$$f_1 = \frac{v}{4L}$$

\pause
\textbf{Solve for L:} $L = \frac{v}{4f_1}$

\pause
$$L = \frac{344 \text{ m/s}}{4(128 \text{ Hz})} = \frac{344}{512} = \boxed{0.672 \text{ m} = 67.2 \text{ cm}}$$

\pause
\textbf{Check:} About 2 feet - reasonable for a musical instrument.
\note{[P0] "Self-correct in different color"\\\\
[P1] "Fundamental: n equals 1, so f-1 equals v over 4L"\\\\
[P2] "Solve for L: equals v over 4 f-1"\\\\
[P3] [ALGEBRA] "344 divided by 4 times 128"\\\\
[P4] [ANSWER] "L equals 0.672 m or 67 cm"\\\\
[THE WONDER] Wind instruments use this exact physics - finger holes change effective length}
\end{frame}

\section{Summary}

\begin{frame}
\frametitle{What You Now Know}
\begin{block}{The Revelations}
\begin{enumerate}
\item Sound = matter disturbed, propagating as longitudinal wave \pause
\item $v = f\lambda$ connects speed, pitch, and wavelength \pause
\item Intensity $\propto$ amplitude$^2$, measured in decibels \pause
\item Doppler effect: motion changes observed frequency \pause
\item Resonance: driving at natural frequency maximizes energy \pause
\item Harmonics: geometry restricts allowed frequencies
\end{enumerate}
\end{block}
\note{[P0] "Six revelations about sound"\\\\
[P1] "Sound requires matter - vibrations of medium"\\\\
[P2] "Universal wave equation applies to sound"\\\\
[P3] "Intensity squared relationship, logarithmic perception"\\\\
[P4] "Doppler: relative motion shifts frequency"\\\\
[P5] "Resonance: matching frequencies transfer energy"\\\\
[P6] "Standing waves: boundary conditions select harmonics"\\\\
[THE WONDER] You now understand invisible vibrations that connect the universe}
\end{frame}

\begin{frame}
\frametitle{Key Equations}
\begin{align}
v &= f\lambda \\
I &= \frac{P}{A} = \frac{(\Delta p)^2}{2\rho v} \\
\beta \text{ (dB)} &= 10\log_{10}\left(\frac{I}{I_0}\right) \\
f_{\text{obs}} &= f_s\left(\frac{v_w}{v_w \pm v_s}\right) \\
f_B &= |f_1 - f_2| \\
f_n &= n\frac{v}{4L} \text{ (closed)}, \quad n = 1,3,5,\ldots \\
f_n &= n\frac{v}{2L} \text{ (open)}, \quad n = 1,2,3,\ldots
\end{align}
\note{- Seven key equations for sound\\\\
- Wave equation, intensity, decibels, Doppler, beats, harmonics\\\\
- Know when to use each\\\\
- Questions before we end?}
\end{frame}

\begin{frame}
\frametitle{Homework}
\begin{center}
\Large
Complete the assigned problems\\[0.3cm]
posted on the LMS
\end{center}
\note{[SAY] "Homework is posted on the LMS"\\\\
[TIMING] Due date: check LMS\\\\
[CHECK] Questions before we end?\\\\
[TRANSITION] Next class: Chapter 15 Light and Color}
\end{frame}

\end{document}
