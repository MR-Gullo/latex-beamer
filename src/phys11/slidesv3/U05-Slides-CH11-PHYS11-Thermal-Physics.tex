\documentclass{beamer}
\usepackage{../../../shared/templates/ds9_theme}
\usepackage[overridenote]{pdfpc}
\graphicspath{{../images/}{../../shared/images/}}

\title[Thermal Physics]{PHYS11 CH:11 The Hidden Energy}
\subtitle{Temperature, Heat, and Phase Change}
\author[Mr. Gullo]{Mr. Gullo}
\date[December 2025]{December 2025}

\begin{document}

\frame{\titlepage
\note{[THE HOOK] Today we decode invisible energy.\\\\
- Why metal feels colder than wood at same temp\\\\
- Why ice cream melts, why sweat cools you\\\\
- Three revelations: temperature vs thermal energy, heat transfer modes, phase change\\\\
[THE WONDER] By end of class, you'll understand the energy that shapes everything from weather to cooking}
}

\begin{frame}
\frametitle{Outline}
\tableofcontents
\end{frame}

\section{Introduction}

\begin{frame}
\frametitle{The Mystery of the Welder}
\begin{figure}
\centering
\includegraphics[width=0.8\textwidth,height=0.6\textheight,keepaspectratio]{phys11-thermal-fig11-2.jpg}
\end{figure}

\pause
How does thermal energy travel from the arc to your skin meters away?
\note{[P0] "Welder's arc - hot enough to melt metal"\\\\
[P1] [THE HOOK] "You can feel the heat meters away. How does that energy reach you?"\\\\
- No physical contact required\\\\
- Energy transferred invisibly through space\\\\
[THE WONDER] Same radiation warms Earth from Sun 150 million km away}
\end{frame}

\section{Temperature and Thermal Energy}

\begin{frame}
\frametitle{Learning Objectives}
\begin{block}{By the end of this section, you will be able to:}
\begin{itemize}
\item \textbf{11.1:} Explain that temperature is a measure of internal kinetic energy \pause
\item \textbf{11.1:} Interconvert temperatures between Celsius, Kelvin, and Fahrenheit scales
\end{itemize}
\end{block}
\note{[P0] "Two objectives for temperature"\\\\
[P1] "First: what temperature really measures at molecular level. Second: convert between scales"\\\\
- This is foundational for all thermal physics\\\\
- You'll use these conversions constantly}
\end{frame}

\begin{frame}
\frametitle{11.1 What Is Temperature?}
\begin{alertblock}{The Illusion}
\textbf{You think:} Temperature measures heat\\
\textbf{Reality:} Temperature measures average kinetic energy of molecules
\end{alertblock}

\pause
\vspace{0.3cm}

\begin{block}{The Source Code}
Temperature = what we measure on a thermometer\\
Heat = transfer of energy due to temperature difference
\end{block}

\pause
\vspace{0.3cm}

\alert{Heat and temperature are NOT the same thing.}
\note{[P0] [THE CONFLICT] "Common confusion - temperature is NOT heat"\\\\
[P1] "Temperature: average molecular kinetic energy"\\\\
- Heat: energy transfer from hot to cold\\\\
[P2] "You're sensitive to flow of energy, not temperature itself"\\\\
[THE HUMILITY] This confuses everyone at first}
\end{frame}

\begin{frame}
\frametitle{11.1 Thermal Energy: The Invisible Motion}
\begin{exampleblock}{The Mental Model}
Atoms and molecules are constantly bouncing around in random directions.\\
Faster motion = higher temperature.
\end{exampleblock}

\pause
\vspace{0.3cm}

\textbf{Thermal energy} = average kinetic energy of particles in a substance

\pause
\vspace{0.3cm}

\begin{center}
\Large $KE = \frac{1}{2}mv^2$
\end{center}

\pause
Higher speed $\rightarrow$ greater kinetic energy $\rightarrow$ higher temperature
\note{[P0] "Atoms constantly moving, bouncing off each other"\\\\
[P1] "Thermal energy: average kinetic energy of particles"\\\\
[P2] "Kinetic energy equals half m v squared"\\\\
[P3] [THE REVELATION] "Higher speed means higher temperature"\\\\
[THE CONNECTION - Kinetic Archetype] "Like a mosh pit - faster dancers, more energy in the room"}
\end{frame}

\begin{frame}
\frametitle{11.1 Temperature Scales}
\begin{figure}
\centering
\includegraphics[width=0.7\textwidth,height=0.5\textheight,keepaspectratio]{phys11-thermal-fig11-5.jpg}
\caption{Fahrenheit, Celsius, and Kelvin scales compared}
\end{figure}

\pause
\textbf{Three scales, same physics:}
\begin{itemize}
\item Celsius: water freezes at 0, boils at 100
\item Fahrenheit: water freezes at 32, boils at 212
\item Kelvin: absolute zero at 0, water freezes at 273.15
\end{itemize}
\note{[P0] "Three temperature scales - all measure same thing"\\\\
[P1] "Celsius and Fahrenheit: relative scales with arbitrary zero. Kelvin: absolute scale"\\\\
- Scientists prefer Kelvin - based on true zero\\\\
- 100 Celsius degrees = 180 Fahrenheit degrees\\\\
- One Celsius degree = 1.8 Fahrenheit degrees}
\end{frame}

\begin{frame}
\frametitle{11.1 Absolute Zero: The Coldest Possible}
\begin{block}{The Ultimate Limit}
\textbf{Absolute zero} = 0 K = -273.15 C\\
The temperature at which all molecular motion ceases.
\end{block}

\pause
\vspace{0.3cm}

\textbf{Can we reach it?}
\begin{itemize}
\item Theoretically possible \pause
\item Never achieved in practice \pause
\item Coldest lab temperature: $1.0 \times 10^{-10}$ K (Helsinki)
\end{itemize}

\pause
\vspace{0.3cm}

\alert{At absolute zero, particles have minimum possible energy (quantum zero-point energy).}
\note{[P0] "Absolute zero: theoretical lowest temperature"\\\\
[P1] "All molecular motion stops - no thermal energy"\\\\
[P2] "Theoretically possible"\\\\
[P3] "Never reached - closest is 0.0000000001 K"\\\\
[P4] [THE WONDER] "Even at absolute zero, quantum mechanics says particles still vibrate slightly"\\\\
[THE HUMILITY] Nature always has secrets}
\end{frame}

\begin{frame}
\frametitle{11.1 Converting Temperature Scales}
\begin{block}{Key Conversion Equations}
\begin{align*}
T_{\circ F} &= \frac{9}{5}T_{\circ C} + 32 \\
T_{\circ C} &= \frac{5}{9}(T_{\circ F} - 32) \\
T_{K} &= T_{\circ C} + 273.15 \\
T_{\circ C} &= T_{K} - 273.15
\end{align*}
\end{block}

\pause
\begin{exampleblock}{Quick Check}
Room temperature: 25 C = ? F = ? K
\end{exampleblock}
\note{[P0] "Four most common conversions"\\\\
- Celsius to Fahrenheit: multiply by 9/5, add 32\\\\
- Kelvin is just Celsius shifted by 273.15\\\\
[P1] "Room temp: 25 C = 77 F = 298 K"\\\\
[THE CONNECTION - Digital Archetype] "Like converting between units in a video game"}
\end{frame}

\begin{frame}
\frametitle{Attempt: Temperature Conversion}
\begin{exampleblock}{The Challenge (3 min, silent)}
Your body temperature is normally 98.6 F.

\vspace{0.3cm}

\textbf{Given:}
\begin{itemize}
\item $T_{\circ F} = 98.6$ F
\end{itemize}

\textbf{Find:} Body temperature in Celsius and Kelvin

\vspace{0.3cm}

\textit{Can you decode your body's thermal state? Work silently.}
\end{exampleblock}
\note{[THE CHALLENGE] Can they convert between scales?\\\\
[SAY] "Try this on your own. Use the conversion equations."\\\\
[TIMING] 3-4 min SILENT individual work\\\\
[CIRCULATE] Note who uses correct equation\\\\
[WATCH FOR] Students forgetting to subtract 32 first\\\\
[DON'T HELP] Let them struggle with order of operations}
\end{frame}

\begin{frame}
\frametitle{Compare: Temperature Conversion}
\textbf{Turn and talk (2 min):}

\vspace{0.3cm}

\begin{enumerate}
\item Which equation did you use to convert F to C?
\item Did you subtract 32 before or after multiplying by 5/9?
\item How did you convert C to K?
\end{enumerate}

\vspace{0.5cm}

\pause
\alert{Name wheel:} One pair share your approach (not your answer).
\note{[TIMING] 2-3 min pair discussion\\\\
[CIRCULATE] Listen for order of operations errors\\\\
[CHECK] Name wheel: call a pair to share\\\\
[EXPECTED APPROACH] Subtract 32, multiply by 5/9 for Celsius, then add 273.15 for Kelvin\\\\
[COMMON ERROR] Multiplying before subtracting 32}
\end{frame}

\begin{frame}
\frametitle{Reveal: Decoding Body Temperature}
\textbf{Self-correct in a different color:}

\vspace{0.3cm}

\textbf{Step 1:} Convert F to C using $T_{\circ C} = \frac{5}{9}(T_{\circ F} - 32)$

\pause
\vspace{0.2cm}

\textbf{Substitute:} $T_{\circ C} = \frac{5}{9}(98.6 - 32) = \frac{5}{9}(66.6)$

\pause
\vspace{0.2cm}

$$T_{\circ C} = \boxed{37.0\text{ C}}$$

\pause
\vspace{0.2cm}

\textbf{Step 2:} Convert C to K using $T_{K} = T_{\circ C} + 273.15$

\pause
\vspace{0.2cm}

$$T_{K} = 37.0 + 273.15 = \boxed{310.2\text{ K}}$$

\pause
\textbf{Check:} 37 C is normal body temp. Makes sense!
\note{[P0] "Self-correct in different color"\\\\
[P1] [ALGEBRA] "Five-ninths times 98.6 minus 32"\\\\
[P2] "37.0 degrees Celsius"\\\\
[P3] "Now add 273.15 for Kelvin"\\\\
[P4] [ANSWER] "310.2 Kelvin"\\\\
[P5] "Check: 37 C is exactly normal body temp"\\\\
[THE WONDER] Your body maintains 310 K within 1 degree to keep you alive}
\end{frame}

\section{Heat, Specific Heat, and Heat Transfer}

\begin{frame}
\frametitle{Learning Objectives}
\begin{block}{By the end of this section, you will be able to:}
\begin{itemize}
\item \textbf{11.2:} Explain heat, heat capacity, and specific heat \pause
\item \textbf{11.2:} Distinguish between conduction, convection, and radiation \pause
\item \textbf{11.2:} Solve problems involving specific heat and heat transfer
\end{itemize}
\end{block}
\note{[P0] "Three objectives for heat transfer"\\\\
[P1] "First: what is specific heat and why materials heat differently"\\\\
[P2] "Second: three modes of heat transfer"\\\\
[P3] "Third: calculations with heat equation Q=mcΔT"\\\\
- These explain everything from cooking to climate}
\end{frame}

\begin{frame}
\frametitle{11.2 Heat: Energy on the Move}
\begin{block}{The Universal Law}
\textbf{Heat} = transfer of thermal energy from hot to cold
\end{block}

\pause
\vspace{0.3cm}

\textbf{Key insights:}
\begin{itemize}
\item Heat is NOT a substance - it's energy in transit \pause
\item Heat always flows from high to low temperature \pause
\item Heat stops flowing when temperatures equalize \pause
\item Heat is measured in joules (J), like all energy
\end{itemize}
\note{[P0] [THE REVELATION] "Heat is energy transfer, not a thing"\\\\
[P1] "It's NOT a substance you pour into something"\\\\
[P2] "Always flows from hot to cold - never reversed naturally"\\\\
[P3] "Stops when temperatures equal - thermal equilibrium"\\\\
[P4] "Joules: same unit as kinetic energy, potential energy"\\\\
[THE WONDER] Heat is just energy changing location}
\end{frame}

\begin{frame}
\frametitle{11.2 The Heat Equation}
\begin{block}{Nature's Rule for Temperature Change}
\begin{center}
\Large $\boxed{Q = mc\Delta T}$
\end{center}
Heat transferred = mass × specific heat × temperature change
\end{block}

\pause
\vspace{0.3cm}

\textbf{What each variable means:}
\begin{itemize}
\item $Q$ = heat transferred (J)
\item $m$ = mass (kg)
\item $c$ = specific heat (J/kg·C)
\item $\Delta T$ = temperature change (C or K)
\end{itemize}
\note{[P0] [THE REVELATION] "Q equals m c delta T"\\\\
- Q: heat transferred in joules\\\\
- m: mass in kilograms\\\\
- c: specific heat - how stubborn the material is about changing temp\\\\
- Delta T: temperature change\\\\
[P1] "Three factors affect heat needed: how much stuff, what it's made of, how much temp change"\\\\
[THE CONNECTION - Digital Archetype] "Like damage calculation: amount times resistance times hit strength"}
\end{frame}

\begin{frame}
\frametitle{11.2 Specific Heat: Material Stubbornness}
\begin{exampleblock}{The Mental Model}
Specific heat = how much energy needed to raise 1 kg of material by 1 C
\end{exampleblock}

\pause
\vspace{0.3cm}

\textbf{Common values:}
\begin{itemize}
\item Water: 4186 J/kg·C (very high!) \pause
\item Aluminum: 900 J/kg·C \pause
\item Iron: 450 J/kg·C \pause
\item Copper: 387 J/kg·C
\end{itemize}

\pause
\vspace{0.3cm}

\alert{Water requires 5× more energy than iron to heat the same amount!}
\note{[P0] "Specific heat: resistance to temperature change"\\\\
[P1] "Water: 4186 - highest of common materials"\\\\
[P2] "Aluminum: 900"\\\\
[P3] "Iron: 450"\\\\
[P4] "Copper: 387"\\\\
[P5] [THE WONDER] "Water's high specific heat stabilizes Earth's climate"\\\\
- Oceans store huge amounts of heat\\\\
- Coastal cities have milder weather than inland}
\end{frame}

\begin{frame}
\frametitle{11.2 Why Metal Feels Colder}
\begin{alertblock}{The Paradox}
Wood and metal at room temperature.\\
\textbf{Why does metal feel colder when you touch it?}
\end{alertblock}

\pause
\vspace{0.3cm}

\textbf{The answer:}
\begin{itemize}
\item Both are same temperature \pause
\item Metal conducts heat faster than wood \pause
\item Metal pulls heat from your hand rapidly \pause
\item You sense rate of heat loss, not temperature!
\end{itemize}
\note{[P0] [THE CONFLICT] "Touch wood desk, then metal chair. Same temp, different feel. Why?"\\\\
[P1] "Both at room temperature - thermometer would show same"\\\\
[P2] "Metal is better conductor - moves heat faster"\\\\
[P3] "Pulls thermal energy from your skin rapidly"\\\\
[P4] [THE REVELATION] "You're sensitive to heat flow rate, not absolute temperature"\\\\
[THE HUMILITY] Your senses evolved to detect danger, not measure temperature accurately}
\end{frame}

\begin{frame}
\frametitle{11.2 Three Modes of Heat Transfer}
\begin{figure}
\centering
\includegraphics[width=0.7\textwidth,height=0.5\textheight,keepaspectratio]{phys11-thermal-fig11-3.jpg}
\caption{Fireplace: all three modes at once}
\end{figure}

\pause
\begin{itemize}
\item \textbf{Conduction:} through physical contact
\item \textbf{Convection:} by fluid movement
\item \textbf{Radiation:} by electromagnetic waves
\end{itemize}
\note{[P0] "Fireplace demonstrates all three modes simultaneously"\\\\
[P1] "Conduction: pan handle gets hot. Convection: hot air rises up chimney. Radiation: you feel warmth across room"\\\\
- Most heat to room comes from radiation\\\\
- Cold air enters around windows (convection)\\\\
- Name wheel: give another example of each mode}
\end{frame}

\begin{frame}
\frametitle{11.2 Conduction: Touch Transfer}
\begin{exampleblock}{The Mental Model}
Molecules vibrating faster bump into slower neighbors, sharing energy.\\
Like dominoes falling in sequence.
\end{exampleblock}

\pause
\vspace{0.3cm}

\begin{figure}
\centering
\includegraphics[width=0.6\textwidth,height=0.4\textheight,keepaspectratio]{phys11-thermal-fig11-4.jpg}
\caption{Energy transfers through collisions}
\end{figure}

\pause
\textbf{Good conductors:} metals (copper, aluminum, gold)\\
\textbf{Poor conductors:} wood, plastic, rubber (insulators)
\note{[P0] "Conduction: direct physical contact"\\\\
- Fast-moving particles collide with slow ones\\\\
- Energy transferred through collisions\\\\
[P1] "Hot side: high kinetic energy. Cold side: low kinetic energy"\\\\
[P2] "Metals: free electrons carry energy quickly. Wood: tightly bound molecules, slow transfer"\\\\
[THE CONNECTION - Kinetic Archetype] "Like a crowd at concert - energy ripples through by contact"}
\end{frame}

\begin{frame}
\frametitle{11.2 Convection: Fluid Flow}
\begin{columns}[T]
\column{0.48\textwidth}
\begin{figure}
\centering
\includegraphics[width=\linewidth,height=0.5\textheight,keepaspectratio]{phys11-thermal-fig11-5.jpg}
\caption{Heated air rises, cool air sinks}
\end{figure}

\pause
\column{0.48\textwidth}
\begin{figure}
\centering
\includegraphics[width=\linewidth,height=0.5\textheight,keepaspectratio]{phys11-thermal-fig11-6.jpg}
\caption{Water circulation in pot}
\end{figure}
\end{columns}

\pause
\vspace{0.3cm}

\textbf{Key idea:} Hot fluid expands, becomes less dense, rises
\note{[P0] "Convection: heat transfer by moving fluid"\\\\
[P1] "House: hot air rises from furnace, circulates room, cools at ceiling, sinks"\\\\
[P2] "Pot: hot water rises, cool water sinks, creates circulation loop"\\\\
[P3] "Buoyancy drives convection - hot less dense than cold"\\\\
- Weather: giant convection currents\\\\
- Ocean currents distribute heat globally}
\end{frame}

\begin{frame}
\frametitle{11.2 Radiation: No Medium Needed}
\begin{exampleblock}{The Mystery}
How does Sun's energy reach Earth through vacuum of space?
\end{exampleblock}

\pause
\vspace{0.3cm}

\begin{block}{The Source Code}
\textbf{Radiation} = energy transfer by electromagnetic waves\\
No physical medium required!
\end{block}

\pause
\vspace{0.3cm}

\textbf{Examples:}
\begin{itemize}
\item Feel warmth from fire without touching
\item Microwave oven heating food
\item Infrared heat lamps
\item Sun warming Earth
\end{itemize}
\note{[P0] [THE HOOK] "Space is vacuum - no air, no particles. How does Sun's energy cross 150 million km?"\\\\
[P1] [THE REVELATION] "Radiation: EM waves carry energy through empty space"\\\\
- Light is EM wave you can see\\\\
- Infrared is EM wave you feel as heat\\\\
[P2] "Fire radiates IR to your skin. You feel it before touching"\\\\
[THE WONDER] All objects above absolute zero radiate electromagnetic energy}
\end{frame}

\begin{frame}
\frametitle{11.2 Color and Radiation}
\textbf{Color affects absorption and emission:}
\begin{itemize}
\item \textbf{Black:} best absorber AND radiator \pause
\item \textbf{White:} worst absorber AND radiator \pause
\item \textbf{Shiny:} reflects radiation like mirror
\end{itemize}

\pause
\vspace{0.3cm}

\begin{alertblock}{Real-World Application}
Hot climates: wear white to reflect heat\\
Cold nights: black asphalt radiates heat faster than grass\\
Space blankets: shiny surface reflects body heat back
\end{alertblock}
\note{[P0] "Color determines how objects interact with radiation"\\\\
[P1] "Black absorbs all wavelengths - heats up fastest"\\\\
[P2] "White reflects all wavelengths - stays cooler"\\\\
[P3] [THE CONNECTION - Harmonic Archetype] "Black parking lot on summer day vs white roof"\\\\
- Same sun, different temps\\\\
- Black asphalt: 65 C. White concrete: 40 C\\\\
[THE REVELATION] Why people in deserts wear white robes}
\end{frame}

\begin{frame}
\frametitle{Attempt: Heating Water}
\begin{exampleblock}{The Challenge (4 min, silent)}
You heat 2.0 kg of water from 20 C to 80 C.

\vspace{0.3cm}

\textbf{Given:}
\begin{itemize}
\item $m = 2.0$ kg
\item $c_{water} = 4186$ J/kg·C
\item $T_i = 20$ C, $T_f = 80$ C
\end{itemize}

\textbf{Find:} Heat energy required (Q)

\vspace{0.3cm}

\textit{Can you calculate energy needed? Work silently.}
\end{exampleblock}
\note{[THE CHALLENGE] Can they use Q=mcΔT correctly?\\\\
[SAY] "Try this on your own. Calculate delta T first."\\\\
[TIMING] 4 min SILENT individual work\\\\
[CIRCULATE] Note who forgets to calculate ΔT first\\\\
[WATCH FOR] Students using 80 instead of 60 for ΔT\\\\
[DON'T HELP] Let them work through it}
\end{frame}

\begin{frame}
\frametitle{Compare: Heat Calculation}
\textbf{Turn and talk (2 min):}

\vspace{0.3cm}

\begin{enumerate}
\item What equation did you use?
\item How did you calculate $\Delta T$?
\item What units did you get for Q?
\end{enumerate}

\vspace{0.5cm}

\pause
\alert{Name wheel:} One pair share your approach (not your answer).
\note{[TIMING] 2-3 min pair discussion\\\\
[CIRCULATE] Listen for ΔT calculation errors\\\\
[CHECK] Name wheel: call a pair to share\\\\
[EXPECTED APPROACH] Q=mcΔT, where ΔT = 80-20 = 60 C\\\\
[COMMON ERROR] Using 80 C instead of 60 C for ΔT}
\end{frame}

\begin{frame}
\frametitle{Reveal: Energy to Heat Water}
\textbf{Self-correct in a different color:}

\vspace{0.3cm}

\textbf{Step 1:} Calculate temperature change
$$\Delta T = T_f - T_i = 80 - 20 = 60\text{ C}$$

\pause
\vspace{0.2cm}

\textbf{Step 2:} Use $Q = mc\Delta T$

\pause
\vspace{0.2cm}

\textbf{Substitute:} $Q = (2.0\text{ kg})(4186\text{ J/kg·C})(60\text{ C})$

\pause
\vspace{0.2cm}

$$Q = 502,320\text{ J} = \boxed{502\text{ kJ}}$$

\pause
\textbf{Check:} About 500 kJ to heat 2 L water by 60 degrees. Reasonable!
\note{[P0] "Self-correct in different color"\\\\
[P1] [ALGEBRA] "Delta T equals 80 minus 20 equals 60 degrees"\\\\
[P2] "Q equals m c delta T"\\\\
[P3] "2.0 times 4186 times 60"\\\\
[P4] [ANSWER] "502 kilojoules - about half a megajoule"\\\\
[P5] "Check: heating 2 L by 60 C takes serious energy"\\\\
[THE WONDER] Your electric kettle does this in 3 minutes - pulling 2700 watts}
\end{frame}

\section{Phase Change and Latent Heat}

\begin{frame}
\frametitle{Learning Objectives}
\begin{block}{By the end of this section, you will be able to:}
\begin{itemize}
\item \textbf{11.3:} Explain changes in heat during changes of state \pause
\item \textbf{11.3:} Describe latent heats of fusion and vaporization \pause
\item \textbf{11.3:} Solve problems involving phase changes
\end{itemize}
\end{block}
\note{[P0] "Three objectives for phase change"\\\\
[P1] "First: what happens when substances change state"\\\\
[P2] "Second: latent heat - hidden energy of phase change"\\\\
[P3] "Third: calculations with Q=mL equations"\\\\
- This explains ice melting, water boiling, dry ice}
\end{frame}

\begin{frame}
\frametitle{11.3 The Four Phases of Matter}
\begin{figure}
\centering
\includegraphics[width=0.8\textwidth,height=0.6\textheight,keepaspectratio]{phys11-thermal-fig11-8.jpg}
\caption{Solid, liquid, gas, and plasma}
\end{figure}

\pause
\textbf{Energy ranking:} Solid < Liquid < Gas < Plasma
\note{[P0] "Four states of matter in order of energy"\\\\
[P1] "Solid: particles locked in place. Liquid: particles slide past each other. Gas: particles free and far apart. Plasma: electrons ripped from atoms"\\\\
- Most of universe is plasma - stars, lightning\\\\
- We mostly deal with first three on Earth\\\\
[THE WONDER] Water exists in all three states naturally on Earth}
\end{frame}

\begin{frame}
\frametitle{11.3 Phase Changes: Energy In/Out}
\textbf{Adding energy (heating):}
\begin{itemize}
\item Melting: solid $\rightarrow$ liquid \pause
\item Vaporization: liquid $\rightarrow$ gas \pause
\item Sublimation: solid $\rightarrow$ gas (skips liquid!)
\end{itemize}

\pause
\vspace{0.3cm}

\textbf{Removing energy (cooling):}
\begin{itemize}
\item Freezing: liquid $\rightarrow$ solid \pause
\item Condensation: gas $\rightarrow$ liquid \pause
\item Deposition: gas $\rightarrow$ solid
\end{itemize}
\note{[P0] "Six phase changes between three states"\\\\
[P1] "Melting: ice to water"\\\\
[P2] "Vaporization: water to steam"\\\\
[P3] "Sublimation: dry ice directly to CO2 gas"\\\\
[P4] "Freezing: water to ice"\\\\
[P5] "Condensation: steam to water droplets"\\\\
[P6] "Deposition: frost forms directly from water vapor"\\\\
- Sublimation and deposition skip liquid phase entirely}
\end{frame}

\begin{frame}
\frametitle{11.3 Latent Heat: The Hidden Energy}
\begin{alertblock}{The Paradox}
You heat ice at 0 C. Temperature stays at 0 C even as you add energy.\\
\textbf{Where does the energy go?}
\end{alertblock}

\pause
\vspace{0.3cm}

\begin{block}{The Revelation}
Energy breaks bonds between molecules, not increase speed.\\
No temperature change during phase transition!
\end{block}

\pause
\vspace{0.3cm}

\textbf{Latent heat} = hidden energy used to change phase without changing temperature
\note{[P0] [THE CONFLICT] "Add heat to ice at zero C. Temp doesn't rise. Where's the energy?"\\\\
[P1] [THE REVELATION] "Energy breaks molecular bonds, doesn't increase kinetic energy"\\\\
- Temperature = average kinetic energy\\\\
- During phase change, energy goes to potential, not kinetic\\\\
[P2] "Latent: Latin for hidden. Temp stays constant while phase changes"\\\\
[THE HUMILITY] This confused scientists for centuries}
\end{frame}

\begin{frame}
\frametitle{11.3 Phase Diagram: Ice to Steam}
\begin{figure}
\centering
\includegraphics[width=0.8\textwidth,height=0.6\textheight,keepaspectratio]{phys11-thermal-fig11-10.jpg}
\caption{Temperature vs energy for water}
\end{figure}

\pause
\textbf{Flat regions} = phase changes (temp constant)\\
\textbf{Sloped regions} = temperature increasing
\note{[P0] "Graph shows heating ice at -20 C until it becomes steam"\\\\
[P1] "Flat at 0 C: melting (334 kJ/kg). Flat at 100 C: boiling (2256 kJ/kg)"\\\\
- Sloped regions: temperature rising\\\\
- Flat regions: phase changing\\\\
[THE REVELATION] Vaporization takes 7 times more energy than melting\\\\
- Why steam burns are so severe}
\end{frame}

\begin{frame}
\frametitle{11.3 Latent Heat Equations}
\begin{block}{Nature's Rules for Phase Change}
\begin{align*}
Q &= mL_f \quad \text{(melting/freezing)} \\
Q &= mL_v \quad \text{(vaporization/condensation)}
\end{align*}
\end{block}

\pause
\vspace{0.3cm}

\textbf{For water:}
\begin{itemize}
\item $L_f = 334$ kJ/kg (latent heat of fusion)
\item $L_v = 2256$ kJ/kg (latent heat of vaporization)
\end{itemize}

\pause
\vspace{0.3cm}

\alert{Note: No $\Delta T$ in these equations - temperature doesn't change!}
\note{[P0] "Two equations for phase change"\\\\
- Q = m L-f for melting or freezing\\\\
- Q = m L-v for boiling or condensing\\\\
[P1] "Water's values: 334 kJ/kg to melt, 2256 kJ/kg to boil"\\\\
[P2] [THE REVELATION] "No delta T - temperature is constant during phase change"\\\\
[THE CONNECTION - Digital Archetype] "Like evolving Pokemon - specific energy requirement, no level change during evolution"}
\end{frame}

\begin{frame}
\frametitle{Attempt: Melting Ice}
\begin{exampleblock}{The Challenge (3 min, silent)}
How much energy is needed to melt 0.50 kg of ice at 0 C?

\vspace{0.3cm}

\textbf{Given:}
\begin{itemize}
\item $m = 0.50$ kg
\item $L_f = 334$ kJ/kg
\item Ice already at 0 C (melting point)
\end{itemize}

\textbf{Find:} Heat energy Q required

\vspace{0.3cm}

\textit{Can you calculate the hidden energy? Work silently.}
\end{exampleblock}
\note{[THE CHALLENGE] Can they use Q=mL correctly?\\\\
[SAY] "Try this on your own. Ice is already at melting point."\\\\
[TIMING] 3 min SILENT individual work\\\\
[CIRCULATE] Note who uses wrong L value\\\\
[WATCH FOR] Students trying to use Q=mcΔT instead\\\\
[DON'T HELP] Let them choose equation}
\end{frame}

\begin{frame}
\frametitle{Compare: Latent Heat}
\textbf{Turn and talk (2 min):}

\vspace{0.3cm}

\begin{enumerate}
\item Which equation did you choose?
\item Why didn't you use $Q = mc\Delta T$?
\item What value did you use for latent heat?
\end{enumerate}

\vspace{0.5cm}

\pause
\alert{Name wheel:} One pair share your reasoning.
\note{[TIMING] 2-3 min pair discussion\\\\
[CIRCULATE] Listen for equation confusion\\\\
[CHECK] Name wheel: call a pair to explain choice\\\\
[EXPECTED APPROACH] Q=mL-f because phase change, not temperature change\\\\
[COMMON ERROR] Trying to use Q=mcΔT when ΔT=0}
\end{frame}

\begin{frame}
\frametitle{Reveal: Energy to Melt Ice}
\textbf{Self-correct in a different color:}

\vspace{0.3cm}

\textbf{Key insight:} Phase change, so use $Q = mL_f$

\pause
\vspace{0.2cm}

\textbf{Substitute:} $Q = (0.50\text{ kg})(334\text{ kJ/kg})$

\pause
\vspace{0.2cm}

$$Q = \boxed{167\text{ kJ}}$$

\pause
\vspace{0.3cm}

\textbf{Check:} That's enough energy to raise 1 kg of water by 40 C!\\
\alert{Phase changes require enormous energy.}
\note{[P0] "Self-correct in different color"\\\\
[P1] [ALGEBRA] "Q equals m L-f"\\\\
[P2] "0.50 times 334 equals 167 kilojoules"\\\\
[P3] [ANSWER] "167 kJ - that's huge!"\\\\
[P4] "Compare: same energy would heat 1 kg water by 40 degrees"\\\\
[THE WONDER] Breaking molecular bonds takes massive energy - that's why sweating cools you so effectively}
\end{frame}

\begin{frame}
\frametitle{11.3 Why Sweating Cools You}
\begin{exampleblock}{Real-World Application}
When sweat evaporates:
\begin{enumerate}
\item Water absorbs 2256 kJ/kg from your skin \pause
\item Undergoes phase change: liquid $\rightarrow$ gas \pause
\item Your skin temperature drops
\end{enumerate}
\end{exampleblock}

\pause
\vspace{0.3cm}

\alert{Evaporation is most effective cooling method for your body!}

\pause
\vspace{0.3cm}

\textbf{On humid days:} Less evaporation $\rightarrow$ less cooling $\rightarrow$ you feel hotter
\note{[P0] "Why you sweat when hot"\\\\
[P1] "Each gram of sweat absorbs 2256 J from skin when evaporating"\\\\
[P2] "Liquid to gas transition pulls heat from body"\\\\
[P3] "Skin temp drops significantly"\\\\
[P4] "Much more effective than conduction or radiation"\\\\
[P5] "Humidity prevents evaporation - sweat just sits on skin, doesn't cool"\\\\
[THE CONNECTION - Kinetic Archetype] "Athletes know: dry heat bearable, humid heat brutal"}
\end{frame}

\begin{frame}
\frametitle{11.3 Condensation Releases Heat}
\begin{block}{The Reverse Process}
Condensation releases same energy that vaporization absorbed.\\
Gas $\rightarrow$ liquid releases 2256 kJ/kg for water.
\end{block}

\pause
\vspace{0.3cm}

\textbf{Why hurricanes are so powerful:}
\begin{itemize}
\item Water vapor condenses in storm \pause
\item Releases enormous latent heat \pause
\item Heats surrounding air \pause
\item Creates powerful updrafts and winds
\end{itemize}
\note{[P0] [THE REVELATION] "Condensation is vaporization in reverse - releases energy instead of absorbing"\\\\
[P1] "Hurricanes powered by condensation"\\\\
[P2] "Billions of kg of water vapor condense"\\\\
[P3] "Each kg releases 2256 kJ into atmosphere"\\\\
[P4] "Total energy: equivalent to hundreds of nuclear bombs"\\\\
[THE WONDER] Phase changes don't just affect your drink - they power Earth's most violent storms}
\end{frame}

\section{Summary}

\begin{frame}
\frametitle{The Revelations of Thermal Physics}
\begin{block}{What You Now Know}
\begin{enumerate}
\item Temperature = average molecular kinetic energy \pause
\item Heat = energy transfer from hot to cold \pause
\item $Q = mc\Delta T$ for temperature changes \pause
\item Three transfer modes: conduction, convection, radiation \pause
\item $Q = mL$ for phase changes (no temp change!) \pause
\item Water's high specific heat stabilizes climate \pause
\item Phase changes require enormous energy
\end{enumerate}
\end{block}
\note{[P0] "Seven revelations today"\\\\
[P1] "Temperature measures molecular motion"\\\\
[P2] "Heat is energy in transit"\\\\
[P3] "Q=mcΔT when temperature changes"\\\\
[P4] "Contact, flow, and radiation"\\\\
[P5] "Q=mL when phase changes - temp constant"\\\\
[P6] "Water's high c moderates Earth's temperature"\\\\
[P7] "Breaking bonds takes massive energy"\\\\
[THE WONDER] You now understand the invisible energy that shapes weather, cooking, and life itself}
\end{frame}

\begin{frame}[shrink]
\frametitle{Key Equations}
\textbf{Temperature Conversions:}
\begin{align}
T_{\circ F} &= \frac{9}{5}T_{\circ C} + 32 \\
T_{\circ C} &= \frac{5}{9}(T_{\circ F} - 32) \\
T_{K} &= T_{\circ C} + 273.15
\end{align}

\textbf{Heat Transfer:}
\begin{align}
Q &= mc\Delta T \quad \text{(temperature change)} \\
Q &= mL_f \quad \text{(melting/freezing)} \\
Q &= mL_v \quad \text{(vaporization/condensation)}
\end{align}

\textbf{Water Constants:}
\begin{itemize}
\item $c_{water} = 4186$ J/kg·C
\item $L_f = 334$ kJ/kg
\item $L_v = 2256$ kJ/kg
\end{itemize}
\note{- Six key equations plus three constants\\\\
- Temp conversions: know order of operations\\\\
- Q=mcΔT: when temp changes\\\\
- Q=mL: when phase changes\\\\
- Memorize water's values - most common\\\\
- Questions before homework?}
\end{frame}

\begin{frame}
\frametitle{Homework}
\begin{center}
\Large
Complete the assigned problems\\[0.3cm]
posted on the LMS
\end{center}
\note{[SAY] "Homework posted on LMS"\\\\
[TIMING] Due date: check LMS\\\\
[CHECK] Questions before we end?\\\\
[TRANSITION] Next: applying thermal physics to real systems}
\end{frame}

\end{document}
