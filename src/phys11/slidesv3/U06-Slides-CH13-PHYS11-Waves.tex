\documentclass{beamer}
\usepackage{../../../shared/templates/ds9_theme}
\usepackage[overridenote]{pdfpc}
\graphicspath{{../images/}{../../shared/images/}}

\title[Disturbances That Travel]{PHYS11 CH13: Invisible Disturbances}
\subtitle{How Energy Moves Without Moving Matter}
\author[Mr. Gullo]{Mr. Gullo}
\date[December 2025]{December 2025}

\begin{document}

\frame{\titlepage
\note{[THE HOOK] Today we discover how energy travels across space without moving matter.\\\\
- Sound, light, earthquakes, ocean waves - all transport energy\\\\
- Three revelations: types of waves, wave properties (v equals f lambda), wave interactions\\\\
[THE WONDER] By end of class, you'll understand why you hear around corners and how earthquakes work.\\\\
- This is foundational for sound, light, and all vibrations}
}

\begin{frame}
\frametitle{Outline}
\tableofcontents
\end{frame}

\section{Introduction}

\begin{frame}
\frametitle{The Mystery}
\begin{center}
\Large How does energy travel from one place to another\\
\textit{without moving matter?}
\end{center}

\pause
\vspace{0.5cm}
Drop a rock in a pond. Ripples spread outward, but the water stays in place...

\pause
\vspace{0.3cm}
\alert{Only the disturbance moves.}
\note{[P0] "How does energy travel without moving matter?"\\\\
[P1] "Drop a rock in a pond - ripples spread outward, but water stays in place"\\\\
[P2] [THE WONDER] "Only the disturbance moves. This is a wave"\\\\
- Cork on water bobs up and down, doesn't travel horizontally\\\\
- Energy moves through medium without transporting the medium itself}
\end{frame}

\begin{frame}
\frametitle{Ocean Waves: Energy in Motion}
\begin{figure}
\centering
\includegraphics[width=0.8\textwidth,height=0.6\textheight,keepaspectratio]{phys11-waves-fig13-1.jpg}
\end{figure}

\pause
\begin{exampleblock}{The Mental Model}
Wave = disturbance that travels and carries energy, not mass.
\end{exampleblock}
\note{[P0][Fig 13.1: Waves in the ocean]  "Ocean waves - beautiful example of energy transport"\\\\
[P1] [THE REVELATION] "Wave equals disturbance that travels and carries energy, not mass"\\\\
[THE CONNECTION - Kinetic Archetype] "Surfers ride the energy, not the water"\\\\
[THE WONDER] Same principle governs sound in air, light in space, earthquakes underground}
\end{frame}

\section{Types of Waves}

\begin{frame}
\frametitle{Learning Objectives}
\begin{block}{By the end of this section, you will be able to:}
\begin{itemize}
\item \textbf{13.1:} Define mechanical waves and medium \pause
\item \textbf{13.1:} Distinguish pulse wave from periodic wave \pause
\item \textbf{13.1:} Distinguish longitudinal from transverse waves
\end{itemize}
\end{block}
\note{[P0] "Three objectives for types of waves"\\\\
[P1] "First: mechanical waves need a medium"\\\\
[P2] "Second: pulse versus periodic waves"\\\\
[P3] "Third: transverse versus longitudinal"\\\\
- Assessment: quiz on wave properties next week}
\end{frame}

\begin{frame}
\frametitle{13.1 Mechanical Waves}
\begin{block}{Nature's Rule}
Mechanical waves require a medium (solid, liquid, or gas) to travel through.
\end{block}

\pause
\vspace{0.3cm}

\textbf{Examples:}
\begin{itemize}
\item Sound waves in air \pause
\item Water waves in ocean \pause
\item Earthquake waves in Earth \pause
\item Waves on guitar strings
\end{itemize}

\pause
\begin{alertblock}{The Exception}
Light doesn't need a medium - it travels through vacuum of space!
\end{alertblock}
\note{[P0] [THE REVELATION] "Mechanical waves require a medium - matter to vibrate through"\\\\
[P1] "Sound waves in air"\\\\
[P2] "Water waves in ocean"\\\\
[P3] "Earthquake waves through Earth"\\\\
[P4] "Waves on guitar strings"\\\\
[P5] [THE CONFLICT] "Light is exception - electromagnetic wave, needs no medium"\\\\
[THE WONDER] Sound cannot travel in space - no air to vibrate}
\end{frame}

\begin{frame}
\frametitle{13.1 Pulse Wave vs. Periodic Wave}
\begin{columns}[T]
\column{0.48\textwidth}
\textbf{Pulse Wave}
\begin{itemize}
\item Sudden disturbance
\item One or few waves
\item Examples: thunder, explosion, pebble in water
\end{itemize}

\pause
\column{0.48\textwidth}
\textbf{Periodic Wave}
\begin{itemize}
\item Repeating oscillation
\item Many cycles
\item Examples: wave pool, guitar string, radio
\end{itemize}
\end{columns}

\pause
\vspace{0.3cm}
Periodic waves involve simple harmonic motion.
\note{[P0] "Two categories by duration"\\\\
[P1] "Periodic waves involve simple harmonic motion - back and forth through equilibrium"\\\\
[P2] "Pulse is brief disturbance, periodic is sustained oscillation"\\\\
- Name wheel: give another example of each type}
\end{frame}

\begin{frame}
\frametitle{13.1 Water Wave Anatomy}
\begin{figure}
\centering
\includegraphics[width=0.7\textwidth,height=0.5\textheight,keepaspectratio]{phys11-waves-fig13-2.jpg}
\end{figure}

\pause
\textbf{Key parts:}
\begin{itemize}
\item Crest = highest point
\item Trough = lowest point
\item Seagull bobs up and down in simple harmonic motion
\end{itemize}
\note{[P0][Fig 13.2: Seagull bobs on sine wave] "Seagull bobbing on periodic ocean wave - perfect for showing SHM. Teaching hint: Ask students to trace the seagull's vertical motion - it follows sinusoidal pattern but doesn't travel horizontally."\\\\
[P1] "Crest is highest point, trough is lowest"\\\\
- Seagull moves vertically, not horizontally\\\\
- Wave energy passes underneath bird\\\\
[THE CONNECTION - Kinetic Archetype] "Like bobbing in ocean - you go up and down as waves pass"}
\end{frame}

\begin{frame}
\frametitle{13.1 Transverse Waves}
\begin{figure}
\centering
\includegraphics[width=0.7\textwidth,height=0.5\textheight,keepaspectratio]{phys11-waves-fig13-3.jpg}
\end{figure}

\pause
\begin{block}{The Source Code}
Transverse wave: disturbance \textbf{perpendicular} to direction of propagation.
\end{block}

\pause
\textbf{Examples:} waves on strings, light, water waves (mostly)
\note{[P0][Fig 13.3: Slinky transverse wave] "Woman moves slinky up/down creating transverse waves. Teaching hint: Emphasize perpendicular motion - wave travels horizontally while disturbing medium vertically. Demo with actual slinky if available."\\\\
[P1] [THE REVELATION] "Transverse: disturbance perpendicular to propagation"\\\\
[P2] "Wave moves horizontally, spring moves vertically"\\\\
[THE CONNECTION - Harmonic Archetype] "Like guitar string - vibrates perpendicular to string direction"\\\\
- Light is transverse electromagnetic wave}
\end{frame}

\begin{frame}
\frametitle{13.1 Longitudinal Waves}
\begin{figure}
\centering
\includegraphics[width=0.7\textwidth,height=0.5\textheight,keepaspectratio]{phys11-waves-fig13-4.jpg}
\end{figure}

\pause
\begin{block}{The Source Code}
Longitudinal wave: disturbance \textbf{parallel} to direction of propagation.
\end{block}

\pause
\textbf{Examples:} sound waves, pressure waves, P-waves in earthquakes
\note{[P0][Fig 13.4: Slinky longitudinal wave] "Woman stretches/compresses slinky horizontally creating longitudinal waves. Teaching hint: Emphasize parallel motion - both wave and disturbance travel same direction. Contrast with previous transverse slide."\\\\
[P1] [THE REVELATION] "Longitudinal: disturbance parallel to propagation"\\\\
[P2] "Wave moves horizontally, spring oscillates horizontally"\\\\
[THE CONNECTION - Harmonic Archetype] "Sound is compression wave - air molecules push forward and backward"\\\\
- Also called compression waves}
\end{frame}

\begin{frame}
\frametitle{13.1 Sound: Longitudinal Wave}
\begin{figure}
\centering
\includegraphics[width=0.7\textwidth,height=0.5\textheight,keepaspectratio]{phys11-waves-fig13-5.jpg}
\end{figure}

\pause
\begin{exampleblock}{In the Real World}
\textbf{Guitar string:} transverse wave (vibrates side-to-side)\\
\textbf{Sound from speaker:} longitudinal wave (air vibrates forward-backward)
\end{exampleblock}
\note{[P0][Fig 13.5: Guitar to speaker system] "Guitar string disturbed vertically creates sound waves through amplifier. Teaching hint: Point out transformation - transverse wave on string becomes longitudinal compression wave in air. Ask why string alone is quiet but speaker is loud."\\\\
[P1] "Speaker cone creates compression wave in air - longitudinal"\\\\
- String vibrates perpendicular to its length\\\\
- But sound it creates moves parallel to air motion\\\\
- Two different wave types from same instrument}
\end{frame}

\begin{frame}
\frametitle{13.1 Earthquake Waves}
\textbf{Earthquakes create BOTH types:}

\pause
\vspace{0.3cm}

\begin{itemize}
\item \textbf{P-waves} (Primary/Pressure): longitudinal \pause
\begin{itemize}
\item Fastest, arrive first
\item Travel through solids and liquids
\end{itemize}
\pause
\item \textbf{S-waves} (Secondary/Shear): transverse \pause
\begin{itemize}
\item Slower, arrive second
\item Travel only through solids
\end{itemize}
\end{itemize}

\pause
\vspace{0.3cm}
S-waves cannot pass through Earth's liquid core!
\note{[P0] "Earthquakes create both longitudinal and transverse waves"\\\\
[P1] "P-waves are longitudinal - pressure waves"\\\\
[P2] "Fastest, arrive first, travel through solids and liquids"\\\\
[P3] "S-waves are transverse - shear waves"\\\\
[P4] "Slower, arrive second"\\\\
[P5] "S-waves cannot travel through liquids - this is how we know Earth's core is liquid"\\\\
[THE WONDER] Time difference between arrivals tells distance to epicenter}
\end{frame}

\begin{frame}
\frametitle{13.1 The Physics of Surfing}
\begin{figure}
\centering
\includegraphics[width=0.7\textwidth,height=0.5\textheight,keepaspectratio]{phys11-waves-fig13-6.jpg}
\end{figure}

\pause
Ocean waves are \textbf{orbital progressive waves}:
\begin{itemize}
\item Water particles move in circular paths
\item Combination of transverse and longitudinal motion
\end{itemize}
\note{[P0][Fig 13.6: Surfer riding giant wave] "Surfer gliding down giant wave while another watches from crest. Teaching hint: Connect to energy transport - surfer rides energy down wave face, not water mass itself. Ask why surfers position at wave crest before dropping in."\\\\
[P1] "Ocean waves are complex - circular motion of water particles"\\\\
- As wave approaches shore, energy compresses into smaller volume\\\\
- This creates higher waves - shoaling effect\\\\
[THE CONNECTION - Kinetic Archetype] "Surfers ride the cascading energy down the wave face"\\\\
- Without this circular motion, surfing would just be bobbing in place}
\end{frame}

\section{Wave Properties}

\begin{frame}
\frametitle{Learning Objectives}
\begin{block}{By the end of this section, you will be able to:}
\begin{itemize}
\item \textbf{13.2:} Define amplitude, frequency, period, wavelength, velocity \pause
\item \textbf{13.2:} Relate wave frequency, period, wavelength, and velocity \pause
\item \textbf{13.2:} Solve problems involving wave properties
\end{itemize}
\end{block}
\note{[P0][Fig reused from surfing] "Wave properties objectives - reuses surfing figure for section transition. Teaching hint: Acceptable visual anchor connecting wave types to wave quantification."\\\\
[P1] "First: define the five key wave variables"\\\\
[P2] "Second: understand relationships between them"\\\\
[P3] "Third: solve problems using v equals f lambda"\\\\
- These are practical skills you'll use for sound and light}
\end{frame}

\begin{frame}
\frametitle{13.2 Wave Variables}
\begin{block}{Universal Law: The Five Variables}
\textbf{Amplitude} $A$: Maximum displacement from equilibrium\\
\textbf{Wavelength} $\lambda$: Distance between adjacent crests\\
\textbf{Period} $T$: Time for one complete cycle\\
\textbf{Frequency} $f$: Number of cycles per second (Hz)\\
\textbf{Wave velocity} $v_w$: Speed of disturbance
\end{block}

\pause
\vspace{0.3cm}
These five variables describe ALL waves in the universe.
\note{[P0] [THE REVELATION] "Five variables describe all waves in the universe"\\\\
- Amplitude: how far it displaces from rest\\\\
- Wavelength: distance between repeating parts\\\\
- Period: time for one oscillation\\\\
- Frequency: how many per second\\\\
- Velocity: how fast disturbance propagates\\\\
[P1] [THE WONDER] Same variables for sound, light, water, earthquakes - universal}
\end{frame}

\begin{frame}
\frametitle{13.2 Wave Anatomy Diagram}
\begin{figure}
\centering
\includegraphics[width=0.7\textwidth,height=0.5\textheight,keepaspectratio]{phys11-waves-fig13-7.jpg}
\end{figure}

\pause
\begin{itemize}
\item $\lambda$ = wavelength (crest to crest)
\item $A$ = amplitude (rest to crest)
\item $v_w$ = wave velocity (disturbance speed)
\end{itemize}
\note{[P0][Fig 13.7: Labeled wave anatomy] "Seagull on sinusoidal wave with wavelength λ, amplitude X, wave velocity marked. Teaching hint: Emphasize seagull travels 2X per cycle vertically but disturbance travels λ horizontally - different motions. Use this to preview v=fλ relationship."\\\\
[P1] "Lambda is wavelength - crest to crest or trough to trough"\\\\
- Amplitude is distance from rest to crest\\\\
- Wave velocity is how fast the pattern moves right\\\\
- Seagull travels total distance 2A in one cycle\\\\
- But disturbance travels one wavelength in one period}
\end{frame}

\begin{frame}
\frametitle{13.2 The Universal Relationship}
\begin{block}{Nature's Source Code}
\begin{center}
\Large $\boxed{f = \frac{1}{T}}$
\end{center}
Frequency equals one over period.
\end{block}

\pause
\vspace{0.3cm}

\textbf{Meaning:}
\begin{itemize}
\item Higher frequency $\rightarrow$ shorter period \pause
\item Lower frequency $\rightarrow$ longer period
\end{itemize}

\pause
\vspace{0.3cm}
\textbf{Units:} Frequency in hertz (Hz) = cycles per second
\note{[P0] [THE REVELATION] "Frequency and period are inversely related"\\\\
[P1] "Higher frequency means shorter period"\\\\
[P2] "Lower frequency means longer period"\\\\
[P3] "Hertz is cycles per second - unit named after Heinrich Hertz"\\\\
[THE CONNECTION - Harmonic Archetype] "High-pitched sound has high frequency, short period"\\\\
- Low-pitched sound has low frequency, long period}
\end{frame}

\begin{frame}
\frametitle{13.2 The Master Equation}
\begin{block}{Universal Law: Wave Equation}
\begin{center}
\Large $\boxed{v_w = f \lambda}$
\end{center}
Wave velocity equals frequency times wavelength.
\end{block}

\pause
\vspace{0.3cm}

\textbf{Alternative form:}
$$v_w = \frac{\lambda}{T}$$

\pause
\textbf{Key insight:}
In a given medium where $v_w$ is constant, higher frequency means shorter wavelength.
\note{[P0] [THE REVELATION] "v-w equals f lambda - the master equation for all waves"\\\\
[P1] "Or v-w equals lambda over T - wavelength per period"\\\\
[P2] "In constant medium: high frequency creates short wavelength"\\\\
[THE WONDER] This equation works for sound, light, water, earthquakes - all waves\\\\
- Distance per cycle times cycles per second equals distance per second\\\\
- Beautiful dimensional analysis}
\end{frame}

\begin{frame}
\frametitle{13.2 Frequency and Wavelength}
\begin{figure}
\centering
\includegraphics[width=0.7\textwidth,height=0.5\textheight,keepaspectratio]{phys11-waves-fig13-8.jpg}
\end{figure}

\pause
\begin{exampleblock}{In the Real World: Sound Speakers}
\textbf{Woofer} (large): low frequency, long wavelength\\
\textbf{Tweeter} (small): high frequency, short wavelength
\end{exampleblock}
\note{[P0][Fig 13.8: Speaker comparison] "Smaller speaker emits high-freq short-λ wave, larger emits low-freq long-λ wave. Teaching hint: Ask why speaker size matches wavelength - physical resonance requires matching dimensions. Connect v=fλ: same v means high f needs small λ."\\\\
[P1] [THE CONNECTION - Harmonic Archetype] "Woofer reproduces bass - low frequency, long wavelength"\\\\
- Tweeter reproduces treble - high frequency, short wavelength\\\\
- Both travel at same speed in air (343 m/s)\\\\
- Different frequencies mean different wavelengths\\\\
- Hard tight cone for short wavelengths, soft large cone for long wavelengths}
\end{frame}

\begin{frame}
\frametitle{13.2 Earthquake Energy}
\begin{figure}
\centering
\includegraphics[width=0.6\textwidth,height=0.45\textheight,keepaspectratio]{phys11-waves-fig13-9.jpg}
\end{figure}

\pause
\textbf{Wave properties in earthquakes:}
\begin{itemize}
\item P-waves: 4-7 km/s in Earth's crust
\item S-waves: 2-5 km/s in Earth's crust
\item Both faster in more rigid materials
\item Energy related to amplitude - large amplitude = more damage
\end{itemize}
\note{[P0][Fig 13.9: Earthquake destruction] "Town with collapsed buildings after earthquake - visceral evidence of wave energy. Teaching hint: Connect amplitude to damage - larger A means more energy. Ask students why some buildings fall while others stand (resonance frequencies)."\\\\
[P1] "P-waves faster than S-waves - time difference tells distance to epicenter"\\\\
- Both slow down in less rigid materials like sediments\\\\
- Richter scale measures both amplitude and energy\\\\
- Large-amplitude earthquakes create large ground displacements\\\\
- Amplitude decreases as waves spread out}
\end{frame}

\begin{frame}
\frametitle{Attempt: Decoding Ocean Motion}
\begin{exampleblock}{The Challenge (3 min, silent)}
Ocean waves have wavelength 10.0 m. A seagull bobs up and down once every 5.00 s.

\vspace{0.3cm}

\textbf{Given:}
\begin{itemize}
\item $\lambda = 10.0$ m
\item $T = 5.00$ s
\end{itemize}

\textbf{Find:} Wave velocity $v_w$

\vspace{0.3cm}

\textit{Can you predict the wave speed? Work silently.}
\end{exampleblock}
\note{[THE CHALLENGE] Can they decode the ocean's motion?\\\\
[SAY] "Try this on your own. It's okay to get stuck."\\\\
[TIMING] 3-4 min SILENT individual work\\\\
[CIRCULATE] Note who uses v equals f lambda versus v equals lambda over T\\\\
[WATCH FOR] Students forgetting to calculate frequency first\\\\
[DON'T HELP] Let them struggle - learning happens in Compare}
\end{frame}

\begin{frame}
\frametitle{Compare: Wave Speed}
\textbf{Turn and talk (2 min):}

\vspace{0.3cm}

\begin{enumerate}
\item What formula did you choose?
\item Did you use $v_w = f\lambda$ or $v_w = \frac{\lambda}{T}$?
\item If you used frequency, how did you find it?
\end{enumerate}

\vspace{0.5cm}

\pause
\alert{Name wheel:} One pair share your approach (not your answer).
\note{[TIMING] 2-3 min pair discussion\\\\
[CIRCULATE] Listen for two approaches\\\\
[CHECK] Name wheel: call a pair to share approach\\\\
[EXPECTED APPROACH 1] Use v equals lambda over T directly\\\\
[EXPECTED APPROACH 2] Find f equals 1 over T, then v equals f lambda\\\\
[BOTH WORK] Either path gives same answer}
\end{frame}

\begin{frame}
\frametitle{Reveal: The Speed of Waves}
\textbf{Self-correct in a different color:}

\vspace{0.3cm}

\textbf{Method 1:} Direct calculation
\pause
$$v_w = \frac{\lambda}{T} = \frac{10.0 \text{ m}}{5.00 \text{ s}} = 2.00 \text{ m/s}$$

\pause
\vspace{0.3cm}

\textbf{Method 2:} Using frequency
\pause
$$f = \frac{1}{T} = \frac{1}{5.00 \text{ s}} = 0.200 \text{ Hz}$$

\pause
$$v_w = f\lambda = (0.200 \text{ Hz})(10.0 \text{ m}) = 2.00 \text{ m/s}$$

\pause
\textbf{Check:} 2 meters per second - reasonable for gentle ocean wave.
\note{[P0] "Self-correct in a different color"\\\\
[P1] [ALGEBRA] "v-w equals lambda over T equals 10 meters over 5 seconds equals 2 m/s"\\\\
[P2] "Method 2: using frequency"\\\\
[P3] "f equals 1 over T equals 0.2 hertz"\\\\
[P4] "v-w equals f lambda equals 0.2 times 10 equals 2 m/s"\\\\
[P5] [ANSWER] "2.00 m/s - slow speed, reasonable for ocean wave"\\\\
[THE WONDER] Both methods give same answer - mathematics is self-consistent}
\end{frame}

\begin{frame}
\frametitle{Attempt: Toy Spring Wave}
\begin{exampleblock}{The Challenge (3 min, silent)}
A woman creates 2 waves per second on a toy spring. Each wave travels 0.9 m in one complete cycle.

\vspace{0.3cm}

\textbf{Given:}
\begin{itemize}
\item $f = 2$ Hz (2 waves per second)
\item $\lambda = 0.9$ m (one cycle)
\end{itemize}

\textbf{Find:} (a) Period $T$ \quad (b) Wave velocity $v_w$

\vspace{0.3cm}

\textit{Two-part challenge. Work individually.}
\end{exampleblock}
\note{[THE CHALLENGE] Can they decode the spring's motion?\\\\
[SAY] "Two parts - find period, then velocity"\\\\
[TIMING] 3-4 min SILENT individual work\\\\
[CIRCULATE] Note who remembers T equals 1 over f\\\\
[WATCH FOR] Students using correct formula for velocity\\\\
[DON'T HELP] Productive struggle builds understanding}
\end{frame}

\begin{frame}
\frametitle{Compare: Spring Motion}
\textbf{Turn and talk (2 min):}

\vspace{0.3cm}

\begin{enumerate}
\item How did you find the period from frequency?
\item Which velocity formula did you use?
\item Did your units work out correctly?
\end{enumerate}

\vspace{0.5cm}

\pause
\alert{Name wheel:} One pair share your approach for both parts.
\note{[TIMING] 2-3 min pair discussion\\\\
[CIRCULATE] Listen for correct formulas\\\\
[CHECK] Name wheel: call a pair to share\\\\
[EXPECTED APPROACH] Part a: T equals 1 over f\\\\
[EXPECTED APPROACH] Part b: v equals f lambda\\\\
[COMMON ERROR] Mixing up which variable goes where}
\end{frame}

\begin{frame}
\frametitle{Reveal: Spring Wave Solution}
\textbf{Self-correct in a different color:}

\vspace{0.3cm}

\textbf{Part (a):} Find period
\pause
$$T = \frac{1}{f} = \frac{1}{2 \text{ s}^{-1}} = 0.5 \text{ s}$$

\pause
\vspace{0.3cm}

\textbf{Part (b):} Find wave velocity
\pause
$$v_w = f\lambda = (2 \text{ s}^{-1})(0.9 \text{ m}) = 1.8 \text{ m/s}$$

\pause
\vspace{0.3cm}

\textbf{Check:} Could also use $v_w = \frac{\lambda}{T} = \frac{0.9}{0.5} = 1.8$ m/s
\note{[P0] "Self-correct in a different color"\\\\
[P1] [ALGEBRA] "T equals 1 over f equals 1 over 2 equals 0.5 seconds"\\\\
[P2] "Period is half a second per wave"\\\\
[P3] [ALGEBRA] "v-w equals f lambda equals 2 times 0.9 equals 1.8 m/s"\\\\
[P4] [ANSWER] "1.8 m/s - could verify using v equals lambda over T"\\\\
[P5] "Both methods give same answer - self-checking"\\\\
[THE WONDER] You just calculated wave motion using universal equation}
\end{frame}

\section{Wave Interaction}

\begin{frame}
\frametitle{Learning Objectives}
\begin{block}{By the end of this section, you will be able to:}
\begin{itemize}
\item \textbf{13.3:} Describe superposition of waves \pause
\item \textbf{13.3:} Distinguish constructive from destructive interference \pause
\item \textbf{13.3:} Describe standing waves \pause
\item \textbf{13.3:} Distinguish reflection from refraction
\end{itemize}
\end{block}
\note{[P0] "Four objectives for wave interactions"\\\\
[P1] "First: superposition - waves combine"\\\\
[P2] "Second: interference - constructive versus destructive"\\\\
[P3] "Third: standing waves - waves that don't propagate"\\\\
[P4] "Fourth: reflection versus refraction"\\\\
- These explain music, noise cancellation, earthquakes}
\end{frame}

\begin{frame}
\frametitle{13.3 Complex Wave Patterns}
\begin{figure}
\centering
\includegraphics[width=0.7\textwidth,height=0.5\textheight,keepaspectratio]{phys11-waves-fig13-10.jpg}
\end{figure}

\pause
Real waves look complex because multiple waves combine - \textbf{superposition}.
\note{[P0][Fig 13.10: Lake ripples] "Waves ripple across lake by mountains - naturalistic complex interference pattern. Teaching hint: Ask students to identify areas of constructive vs destructive interference. Connect to upcoming superposition principle - nature does vector addition automatically."\\\\
[P1] "Multiple waves from different sources superimpose"\\\\
- Waves reflect off wall, combine with incoming waves\\\\
- Creates peaks and valleys in specific locations\\\\
[THE WONDER] Even chaotic-looking patterns follow mathematical rules}
\end{frame}

\begin{frame}
\frametitle{13.3 Superposition of Waves}
\begin{block}{Universal Law: The Principle of Superposition}
When two or more waves meet, they combine by adding their disturbances.
\end{block}

\pause
\vspace{0.3cm}

\textbf{Key insight:}
\begin{itemize}
\item Disturbances correspond to forces \pause
\item Forces add vectorially \pause
\item Resulting wave = sum of individual disturbances
\end{itemize}
\note{[P0] [THE REVELATION] "Superposition: waves combine by adding disturbances"\\\\
[P1] "Disturbances are forces"\\\\
[P2] "Forces add vectorially"\\\\
[P3] "Resulting wave is algebraic sum"\\\\
- If both push up, total is larger\\\\
- If one pushes up and one down, they can cancel\\\\
[THE WONDER] Simple addition creates complex patterns}
\end{frame}

\begin{frame}
\frametitle{13.3 Constructive Interference}
\begin{figure}
\centering
\includegraphics[width=0.7\textwidth,height=0.5\textheight,keepaspectratio]{phys11-waves-fig13-11.jpg}
\end{figure}

\pause
\begin{block}{Nature's Rule}
Constructive interference: waves exactly in phase combine to produce larger amplitude.
\end{block}

\pause
Amplitude doubles when two identical waves align crest-to-crest!
\note{[P0][Fig 13.11: Perfect constructive interference] "Wave 1 and Wave 2 perfectly in phase, resultant has twice amplitude. Teaching hint: Emphasize vector addition - both push up at same time, forces add. Ask why amplitude exactly doubles (identical waves, perfect alignment)."\\\\
[P1] [THE REVELATION] "Constructive interference: crests align, troughs align"\\\\
[P2] "Resulting amplitude is sum - twice the original"\\\\
- Same wavelength as original waves\\\\
- Pure constructive interference is rare - requires perfect alignment\\\\
[THE CONNECTION - Harmonic Archetype] "Like two speakers playing same note in sync - louder"}
\end{frame}

\begin{frame}
\frametitle{13.3 Destructive Interference}
\begin{figure}
\centering
\includegraphics[width=0.7\textwidth,height=0.5\textheight,keepaspectratio]{phys11-waves-fig13-12.jpg}
\end{figure}

\pause
\begin{block}{Nature's Rule}
Destructive interference: waves exactly out of phase combine to cancel each other.
\end{block}

\pause
\begin{alertblock}{The Paradox}
Two waves can add to create... nothing! Zero amplitude.
\end{alertblock}
\note{[P0][Fig 13.12: Perfect destructive interference] "Wave 1 and Wave 2 perfectly out of phase, resultant has zero amplitude. Teaching hint: Counterintuitive - two disturbances create silence/stillness. Connect to noise-cancelling headphones - mic samples ambient noise, speaker produces inverted wave."\\\\
[P1] [THE REVELATION] "Destructive interference: crest aligns with trough"\\\\
[P2] [THE CONFLICT] "Two disturbances add to create zero disturbance"\\\\
- One pushes up, other pushes down by same amount\\\\
- Complete cancellation\\\\
[THE CONNECTION - Digital Archetype] "Noise-cancelling headphones use this principle"}
\end{frame}

\begin{frame}
\frametitle{13.3 Mixed Interference}
\begin{figure}
\centering
\includegraphics[width=0.7\textwidth,height=0.5\textheight,keepaspectratio]{phys11-waves-fig13-13.jpg}
\end{figure}

\pause
Most real-world waves show \textbf{partial} constructive and destructive interference.

\pause
Creates complex patterns that vary in space and time.
\note{[P0][Fig 13.13: Dissimilar wave superposition] "Wave 1 large amplitude/low freq, Wave 2 small amplitude/high freq, resultant is squiggly non-sinusoidal. Teaching hint: Most real sounds are combinations - musical instruments create rich timbres from multiple frequencies interfering."\\\\
[P1] "Most real waves show partial interference"\\\\
[P2] "Creates complicated patterns"\\\\
- Stereo speakers: loud in some spots, soft in others\\\\
- Jet engines: volume fluctuates as interference varies\\\\
- No longer sinusoidal shape}
\end{frame}

\begin{frame}
\frametitle{13.3 Standing Waves}
\begin{figure}
\centering
\includegraphics[width=0.7\textwidth,height=0.5\textheight,keepaspectratio]{phys11-waves-fig13-14.jpg}
\end{figure}

\pause
\begin{block}{The Source Code}
Standing wave: formed by superposition of two identical waves moving in opposite directions.
\end{block}

\pause
Pattern oscillates in place - doesn't propagate!
\note{[P0][Fig 13.14: Standing wave formation] "Two identical waves moving opposite directions alternate between no disturbance (destructive) and doubled disturbance (constructive). Teaching hint: Show wave doesn't travel - pattern location fixed. Demo with rope if available."\\\\
[P1] [THE REVELATION] "Two identical waves moving opposite directions"\\\\
[P2] "Pattern alternates between constructive and destructive interference"\\\\
- Doesn't travel left or right\\\\
- Amplitude varies at fixed locations\\\\
[THE CONNECTION - Harmonic Archetype] "Guitar strings create standing waves"}
\end{frame}

\begin{frame}
\frametitle{13.3 Nodes and Antinodes}
\begin{figure}
\centering
\includegraphics[width=0.6\textwidth,height=0.4\textheight,keepaspectratio]{phys11-waves-fig13-15.jpg}
\end{figure}

\pause
\textbf{Node:} point of zero amplitude (no motion)

\pause
\textbf{Antinode:} point of maximum amplitude

\pause
\vspace{0.3cm}
Fixed ends must be nodes - string cannot move there.
\note{[P0][Fig 13.15: First harmonic nodes/antinodes] "One antinode and two nodes created by single standing wave. Teaching hint: Emphasize nodes are always destructive interference points, antinodes always constructive. Count together - fundamental has 1 antinode."\\\\
[P1] "Node is point of zero displacement - destructive interference"\\\\
[P2] "Antinode is point of maximum displacement - constructive interference"\\\\
[P3] "Fixed ends must be nodes"\\\\
- String cannot move at fixed points\\\\
- Pattern depends on wavelength and string length\\\\
- Different frequencies create different standing wave patterns}
\end{frame}

\begin{frame}
\frametitle{13.3 Reflection of Waves}
\begin{figure}
\centering
\includegraphics[width=0.7\textwidth,height=0.5\textheight,keepaspectratio]{phys11-waves-fig13-17.jpg}
\end{figure}

\pause
\begin{block}{Nature's Rule}
Reflection: wave bounces off barrier and changes direction.
\end{block}

\pause
\textbf{Inversion:} wave reflects from fixed end as inverted (crest becomes trough).
\note{[P0][Fig 13.17: Wave reflection at fixed end] "Wave travels right, hits fixed end, flips vertically, travels left. Teaching hint: Ask why inversion occurs - fixed end can't move so wave's upward force creates downward reaction. Connect to standing waves - reflection creates opposite-moving wave."\\\\
[P1] [THE REVELATION] "Reflection changes direction - wave bounces back"\\\\
[P2] "Inversion: wave flips vertically at fixed end"\\\\
- Crest reflects as trough\\\\
- Trough reflects as crest\\\\
- Free end reflects without inversion\\\\
- Standing waves result from reflection}
\end{frame}

\begin{frame}
\frametitle{13.3 Refraction of Waves}
\begin{figure}
\centering
\includegraphics[width=0.7\textwidth,height=0.5\textheight,keepaspectratio]{phys11-waves-fig13-18.jpg}
\end{figure}

\pause
\begin{block}{Nature's Rule}
Refraction: wave bends when passing from one medium to another.
\end{block}

\pause
\textbf{What changes:} speed, wavelength, direction\\
\textbf{What stays same:} frequency
\note{[P0][Fig 13.18: Wave refraction at boundary] "Wave bends slightly right as it crosses into another medium. Teaching hint: Emphasize frequency constant (source unchanged) but v and λ change (medium properties). Ask why wave bends - one side enters new medium first, changes speed first."\\\\
[P1] [THE REVELATION] "Refraction bends wave at boundary between media"\\\\
[P2] "Speed and wavelength change, frequency stays constant"\\\\
- Water waves: deep to shallow water slow down, bend\\\\
- Light: air to glass slows down, bends\\\\
- Depends on properties of media (density, depth)\\\\
[THE WONDER] This is why pencil looks bent in water}
\end{frame}

\begin{frame}
\frametitle{13.3 Earthquakes and Standing Waves}
\begin{exampleblock}{Real-World Application}
Earthquake waves reflect off denser rocks, creating standing waves.
\end{exampleblock}

\pause
\vspace{0.3cm}

\textbf{Result:}
\begin{itemize}
\item Constructive interference at some locations (more damage) \pause
\item Destructive interference at other locations (less damage) \pause
\item Areas \textit{farther} from epicenter can be \textit{more} damaged!
\end{itemize}

\pause
\begin{alertblock}{The Paradox}
Distance from epicenter doesn't always predict damage - interference patterns matter!
\end{alertblock}
\note{[P0] "Earthquake waves create standing wave patterns"\\\\
[P1] "Constructive interference amplifies shaking - more damage"\\\\
[P2] "Destructive interference reduces shaking - less damage"\\\\
[P3] "Counterintuitive: farther areas can be more damaged"\\\\
[P4] [THE CONFLICT] "Distance doesn't predict damage - wave physics does"\\\\
[THE WONDER] Complex physics explains seemingly random destruction patterns}
\end{frame}

\section{Summary}

\begin{frame}
\frametitle{What You Now Know}
\begin{block}{The Revelations}
\begin{enumerate}
\item Waves = disturbances that transport energy, not matter \pause
\item Transverse: perpendicular; Longitudinal: parallel \pause
\item Five variables: $A$, $\lambda$, $T$, $f$, $v_w$ \pause
\item Master equation: $v_w = f\lambda$ \pause
\item Superposition: waves add disturbances \pause
\item Interference: constructive amplifies, destructive cancels \pause
\item Standing waves, reflection, refraction
\end{enumerate}
\end{block}
\note{[P0] "Seven revelations today"\\\\
[P1] "Waves transport energy without moving matter"\\\\
[P2] "Transverse versus longitudinal motion"\\\\
[P3] "Five variables describe all waves"\\\\
[P4] "v equals f lambda - universal wave equation"\\\\
[P5] "Superposition adds disturbances"\\\\
[P6] "Interference creates louder or quieter patterns"\\\\
[P7] "Standing waves, reflection, refraction explain complex phenomena"\\\\
[THE WONDER] You now understand ocean, sound, earthquakes, light - same principles}
\end{frame}

\begin{frame}[shrink]
\frametitle{Key Equations}
\begin{align}
f &= \frac{1}{T} \quad \text{(frequency and period)}\\
T &= \frac{1}{f}\\
v_w &= f\lambda \quad \text{(wave equation)}\\
v_w &= \frac{\lambda}{T}
\end{align}

\vspace{0.3cm}

\textbf{Remember:}
\begin{itemize}
\item In constant medium, higher frequency $\rightarrow$ shorter wavelength
\item Amplitude is independent of velocity
\item All waves obey these relationships
\end{itemize}
\note{- Four fundamental equations\\\\
- Frequency times wavelength equals velocity\\\\
- Period is inverse of frequency\\\\
- These work for sound, light, water, all mechanical waves\\\\
- Know when to use each form\\\\
- Questions before we end?}
\end{frame}

\begin{frame}
\frametitle{Homework}
\begin{center}
\Large
Complete the assigned problems\\[0.3cm]
posted on the LMS
\end{center}
\note{[SAY] "Homework is posted on the LMS"\\\\
[TIMING] Due date: check LMS\\\\
[CHECK] Questions before we end?\\\\
[TRANSITION] Next class: Chapter 14 Sound - applying wave principles to vibrations in air}
\end{frame}

\end{document}
