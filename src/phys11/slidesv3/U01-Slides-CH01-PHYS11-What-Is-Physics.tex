\documentclass{beamer}
\usepackage{../../../shared/templates/ds9_theme}
\usepackage[overridenote]{pdfpc}
\graphicspath{{../images/}{../../shared/images/}}

\title[Rules of Reality]{PHYS11 CH:1 The Rules That Run the Universe}
\subtitle{From Atoms to Galaxies}
\author[Mr. Gullo]{Mr. Gullo}
\date[December 2025]{December 2025}

\begin{document}

\frame{\titlepage
\note{[THE HOOK] Today we discover the rules that govern everything.\\\\
- Same laws explain your phone AND distant galaxies\\\\
- Three revelations: what physics IS, how we DISCOVER truth, how we MEASURE reality\\\\
[THE WONDER] By end of class, you'll see the universe differently.\\\\
- This is foundational for everything we'll do this year}
}

\begin{frame}
\frametitle{Outline}
\tableofcontents
\end{frame}

\section{Introduction}

\begin{frame}
\frametitle{The Mystery}
\begin{center}
\Large What if one set of rules explained\\
\textit{everything in the universe?}
\end{center}

\pause
\vspace{0.5cm}
From the atoms in your fingertips to galaxies 2.5 million light years away...

\pause
\vspace{0.3cm}
\alert{The same laws apply.}
\note{[P0] "What if one set of rules explained everything?"\\\\
[P1] "From the atoms in your fingertips to galaxies 2.5 million light years away..."\\\\
[P2] [THE WONDER] "The same laws apply. Humans discovered these rules over 400 years. Today you join that journey."}
\end{frame}

\begin{frame}
\frametitle{2.5 Million Light Years Away}
\begin{figure}
\centering
\includegraphics[width=0.8\textwidth,height=0.6\textheight,keepaspectratio]{phys11-physics-fig01.jpg}
\end{figure}

\pause
\begin{exampleblock}{The Mental Model}
The force holding you in your seat is the same force arranging billions of stars in Andromeda.
\end{exampleblock}
\note{[P0] "The Andromeda galaxy - 2.5 million light years away, billions of stars"\\\\
[P1] [THE REVELATION] "The force holding you in your seat is the same force arranging those stars"\\\\
[THE CONNECTION - Digital Archetype] "Like one physics engine running the entire universe"\\\\
[THE WONDER] Universal laws work everywhere, always}
\end{frame}

\section{Physics: Definitions and Applications}

\begin{frame}
\frametitle{Learning Objectives}
\begin{block}{By the end of this lesson, you will be able to:}
\begin{itemize}
\item \textbf{1.1:} Describe the definition, aims, and branches of physics \pause
\item \textbf{1.1:} Distinguish classical physics from modern physics \pause
\item \textbf{1.1:} Describe how physics is used in other sciences and everyday technology
\end{itemize}
\end{block}
\note{[P0] "Three objectives today"\\\\
[P1] "First: what is physics and what are its branches"\\\\
[P2] "Second: classical vs modern physics. Third: real-world applications"\\\\
- Assessment: quiz next week}
\end{frame}

\begin{frame}
\frametitle{1.1 The Source Code of Reality}
\begin{block}{Nature's Operating System}
Physics is the science of the fundamental rules governing our universe.
\end{block}

\pause
\vspace{0.3cm}

\textbf{What makes it physics:}
\begin{itemize}
\item Studies what exists and why \pause
\item Discovers patterns in nature \pause
\item \alert{Must be testable} - if you can't test it, it's not physics
\end{itemize}

\pause
\begin{exampleblock}{The Mental Model}
Physics is like discovering the source code that runs reality.
\end{exampleblock}
\note{[P0] [THE REVELATION] "Physics is the universe's source code - from Greek 'phusis' meaning nature"\\\\
[P1] "Studies what exists and why"\\\\
[P2] "Discovers patterns in nature"\\\\
[P3] "If you can't test it, it's not physics"\\\\
[P4] [THE CONNECTION - Digital Archetype] "Think of it as reverse-engineering reality"\\\\
[THE HUMILITY] Scientists can only support, never prove}
\end{frame}

\begin{frame}
\frametitle{1.1 Physics in Your Phone}
\begin{figure}
\centering
\includegraphics[width=0.6\textwidth,height=0.45\textheight,keepaspectratio]{phys11-physics-fig07.jpg}
\end{figure}

\pause
\begin{exampleblock}{Real-World: Smartphone Physics}
\begin{itemize}
\item Electric circuits and current flow
\item GPS: relationship between speed, distance, time
\item Screen: optics and light
\end{itemize}
\end{exampleblock}
\note{- Name wheel: what physics is in your phone?\\\\
- POSSIBLE ANSWERS: electricity, magnetism, light, sound, motion\\\\
- Engineers use physics to design every component\\\\
- Without understanding physics, we couldn't build this technology}
\end{frame}

\begin{frame}
\frametitle{1.1 Ancient Physics: Stonehenge}
\begin{figure}
\centering
\includegraphics[width=0.7\textwidth,height=0.5\textheight,keepaspectratio]{phys11-physics-fig07.jpg}
\end{figure}

\pause
Built 3000-1000 BC as an astronomical observatory.
\note{- Ancient people studied kinematics: motion of objects\\\\
- Observed sun, moon, stars\\\\
- Built observatories to track celestial motion\\\\
- This led to understanding of simple machines\\\\
- Levers, pulleys, ramps, wheels}
\end{frame}

\begin{frame}
\frametitle{1.1 Branches of Physics}
\begin{columns}[T]
\column{0.48\textwidth}
\textbf{Classical Physics}
\begin{itemize}
\item Mechanics (motion)
\item Thermodynamics (heat)
\item Electricity and Magnetism
\item Optics (light)
\item Acoustics (sound)
\end{itemize}

\pause
\column{0.48\textwidth}
\textbf{Modern Physics}
\begin{itemize}
\item Relativity
\item Quantum Mechanics
\item Nuclear Physics
\item Particle Physics
\end{itemize}
\end{columns}
\note{[P0] "Two categories of physics"\\\\
[P1] "Modern physics: 20th century discoveries for extremes - very fast, tiny, or massive"\\\\
- Classical works for everyday speeds and sizes\\\\
- We'll focus mainly on classical this year}
\end{frame}

\begin{frame}
\frametitle{1.1 The Intuition Trap}
\textbf{Classical physics works when:}
\begin{enumerate}
\item Speeds less than 1\% of light speed \pause
\item Objects visible to naked eye \pause
\item Weak gravity (like Earth's)
\end{enumerate}

\pause

\begin{alertblock}{What Your Brain Gets Wrong}
Your intuition evolved for everyday speeds and sizes.\\
At extremes (tiny, fast, massive), \textbf{intuition fails completely}.
\end{alertblock}
\note{[P0] "Classical physics works when..."\\\\
[P1] "Speeds less than 1 percent of light speed"\\\\
[P2] "Objects visible to naked eye"\\\\
[P3] "Weak gravity like Earth's"\\\\
[P4] [THE CONFLICT] "Your intuition evolved for everyday speeds. At extremes, it fails completely"\\\\
[THE HUMILITY] Your ancestors didn't need to understand atoms or galaxies\\\\
[THE WONDER] Yet humans figured it out anyway}
\end{frame}

\begin{frame}
\frametitle{1.1 Relativity: Time and Space}
\begin{figure}
\centering
\includegraphics[width=0.7\textwidth,height=0.5\textheight,keepaspectratio]{phys11-physics-fig11.jpg}
\end{figure}

\pause
\textbf{Einstein's discoveries:}
\begin{itemize}
\item Time slows down at high speeds
\item Length contracts at high speeds
\item Gravity warps space-time
\end{itemize}
\note{- Einstein revolutionized physics in 1905\\\\
- Time is not absolute - it depends on motion\\\\
- Trip to nearest star: 4.5 Earth years, but astronaut ages only 0.5 years\\\\
- Gravity bends space like bowling ball on trampoline\\\\
- GPS satellites must correct for time dilation}
\end{frame}

\begin{frame}
\frametitle{1.1 Quantum Mechanics}
\begin{center}
\includegraphics[width=0.6\textwidth,height=0.45\textheight,keepaspectratio]{phys11-physics-fig12.jpg}

\small Individual atoms visible with scanning tunneling microscope
\end{center}

\pause
\textbf{Studies:}
\begin{itemize}
\item Atoms and subatomic particles
\item Behavior at tiny scales
\item Particles moving near light speed
\end{itemize}
\note{- Quantum mechanics: the physics of atoms\\\\
- Atoms are smallest units of elements\\\\
- Made of protons, neutrons, electrons\\\\
- Particles behave very differently from everyday objects\\\\
- Led to transistors, lasers, computer chips}
\end{frame}

\begin{frame}
\frametitle{1.1 Particle Colliders}
\begin{center}
\includegraphics[width=0.7\textwidth,height=0.5\textheight,keepaspectratio]{phys11-physics-fig13.jpg}

\small Fermilab particle accelerator
\end{center}

\pause
Accelerate particles to near light speed to study their properties.
\note{- Large Hadron Collider: 27 km long tunnel\\\\
- Particles travel at 99.999 percent speed of light\\\\
- Discovered Higgs-Boson particle in 2012\\\\
- Higgs gives matter the property of mass\\\\
- Studying antimatter and fundamental forces}
\end{frame}

\begin{frame}
\frametitle{1.1 Microwaves and Metal}
\begin{figure}
\centering
\includegraphics[width=0.6\textwidth,height=0.45\textheight,keepaspectratio]{phys11-physics-fig14.jpg}
\end{figure}

\pause
\begin{exampleblock}{Real-World: Why Metal Sparks}
Microwaves increase electron movement in metal $\rightarrow$ electrical current $\rightarrow$ sparks!
\end{exampleblock}

\pause
\begin{alertblock}{Warning}
Never put metal in a microwave - fire hazard!
\end{alertblock}
\note{- Turn and talk: who has seen sparks in microwave?\\\\
- Metal reflects microwaves, can damage oven\\\\
- Moving electrons create electric field\\\\
- This ionizes air around metal, causing sparks\\\\
- Real-world application of electromagnetism}
\end{frame}

\begin{frame}
\frametitle{1.1 Physics in Other Sciences}
\textbf{Physics is foundational for:}
\begin{itemize}
\item \textbf{Chemistry:} atomic and molecular physics \pause
\item \textbf{Biology:} cell membranes, energy transfer \pause
\item \textbf{Medicine:} X-rays, MRI, ultrasound \pause
\item \textbf{Geology:} radioactive dating, earthquakes \pause
\item \textbf{Engineering:} structural design, acoustics \pause
\item \textbf{Architecture:} stability, heating, lighting
\end{itemize}
\note{[P0] "Physics is foundational for all other sciences"\\\\
[P1] "Chemistry uses atomic and molecular physics"\\\\
[P2] "Biology uses cell membranes and energy transfer"\\\\
[P3] "Medicine uses X-rays, MRI, ultrasound"\\\\
[P4] "Geology uses radioactive dating"\\\\
[P5] "Engineering uses structural design and acoustics"\\\\
[P6] "Architecture uses stability, heating, lighting"\\\\
- Name wheel: give another example}
\end{frame}

\begin{frame}
\frametitle{1.1 Medical Applications}
\begin{columns}[T]
\column{0.48\textwidth}
\begin{center}
\includegraphics[width=\linewidth,height=0.5\textheight,keepaspectratio]{phys11-physics-fig16.jpg}

\small MRI scan
\end{center}

\pause
\column{0.48\textwidth}
\begin{center}
\includegraphics[width=\linewidth,height=0.5\textheight,keepaspectratio]{phys11-physics-fig17.jpg}

\small Cell walls
\end{center}
\end{columns}

\vspace{0.3cm}
MRI uses electromagnetic waves. Cell walls use physics of selective permeability.
\note{- MRI: magnetic resonance imaging\\\\
- Uses strong magnetic fields and radio waves\\\\
- No radiation like X-rays\\\\
- Cell walls: physics of diffusion and osmosis\\\\
- Physics explains how senses work: sight, hearing, touch}
\end{frame}

\begin{frame}
\frametitle{1.1 Rosetta Mission}
\begin{center}
\includegraphics[width=0.7\textwidth,height=0.5\textheight,keepaspectratio]{phys11-physics-fig18.jpg}

\small Rosetta spacecraft with Philae lander
\end{center}

\pause
\textbf{Achievement (2014):} First spacecraft to orbit and land on a comet.
\note{- Traveled 6.4 billion km from Earth\\\\
- Landed on comet only 4 km wide\\\\
- Used gravity assist from Mars\\\\
- Solar powered even 800 million km from sun\\\\
- Shows range of physics: orbital mechanics to solar cells}
\end{frame}

\begin{frame}
\frametitle{1.1 Voyager Missions}
\begin{center}
\includegraphics[width=0.7\textwidth,height=0.5\textheight,keepaspectratio]{phys11-physics-fig19.jpg}

\small Voyager trajectory using planetary gravity
\end{center}

\pause
\textbf{Voyager 1:} Launched 1977, now in interstellar space!
\note{- Voyager 2 used Saturn's gravity to slingshot to Uranus and Neptune\\\\
- Voyager 1 left solar system in 2012\\\\
- Still sending data after 45 plus years\\\\
- Uses radioisotope generators, not solar\\\\
- Physics of orbital mechanics made this possible}
\end{frame}

\section{The Scientific Methods}

\begin{frame}
\frametitle{Learning Objectives}
\begin{block}{By the end of this section, you will be able to:}
\begin{itemize}
\item \textbf{1.2:} Explain how the methods of science are used to make discoveries \pause
\item \textbf{1.2:} Define a scientific model and describe examples \pause
\item \textbf{1.2:} Compare and contrast hypothesis, theory, and law
\end{itemize}
\end{block}
\note{- Three objectives for scientific method\\\\
- First: understand the scientific process\\\\
- Second: what are models and why we use them\\\\
- Third: vocabulary - hypothesis vs theory vs law\\\\
- These are tools you'll use all year}
\end{frame}

\begin{frame}
\frametitle{1.2 The Scientific Method}
\begin{enumerate}
\item Make an \textbf{observation} \pause
\item Ask a \textbf{question} \pause
\item Form a \textbf{hypothesis} (testable educated guess) \pause
\item Design and perform an \textbf{experiment} \pause
\item Analyze \textbf{data} \pause
\item Draw \textbf{conclusions} (support or reject hypothesis) \pause
\item Communicate \textbf{results}
\end{enumerate}
\note{[P0] "The foundation of science - seven steps"\\\\
[P1] "Make an observation"\\\\
[P2] "Ask a question"\\\\
[P3] "Form a testable hypothesis"\\\\
[P4] "Design and perform an experiment"\\\\
[P5] "Analyze data"\\\\
[P6] "Draw conclusions - support or reject hypothesis"\\\\
[P7] "Communicate results"\\\\
- Not always linear - often cycle back}
\end{frame}

\begin{frame}
\frametitle{1.2 Example: Car Won't Start}
\textbf{Observation:} Car won't start

\pause
\vspace{0.3cm}

\textbf{Hypothesis 1:} No gasoline in tank

\pause
\textbf{Test:} Add gasoline and try to start

\pause
\vspace{0.3cm}

\textbf{Result:} Still won't start $\rightarrow$ Hypothesis rejected

\pause
\vspace{0.3cm}

\textbf{Hypothesis 2:} Fuel pump is broken

\pause
\textbf{Test:} Replace fuel pump...
\note{[P0] "Observation: car won't start"\\\\
[P1] "Hypothesis 1: no gasoline"\\\\
[P2] "Test: add gasoline and try to start"\\\\
[P3] "Result: still won't start - hypothesis rejected"\\\\
[P4] "Hypothesis 2: fuel pump is broken"\\\\
[P5] "Test: replace fuel pump..."\\\\
- Science is about eliminating possibilities\\\\
- Name wheel: what other hypotheses?}
\end{frame}

\begin{frame}
\frametitle{1.2 Scientific Models}
\begin{block}{Definition: Scientific Model}
A representation of something too difficult or impossible to study directly.
\end{block}

\pause
\vspace{0.3cm}

\textbf{Types of models:}
\begin{itemize}
\item Physical models (3D atom model) \pause
\item Mathematical equations \pause
\item Computer simulations \pause
\item Diagrams and visualizations
\end{itemize}
\note{[P0] "A model represents something too difficult to study directly"\\\\
[P1] "Physical models like 3D atoms"\\\\
[P2] "Mathematical equations"\\\\
[P3] "Computer simulations"\\\\
[P4] "Diagrams and visualizations"\\\\
- Models are always approximate - trade-off between accuracy and simplicity}
\end{frame}

\begin{frame}
\frametitle{1.2 Electron Cloud Model}
\begin{center}
\includegraphics[width=0.6\textwidth,height=0.45\textheight,keepaspectratio]{phys11-physics-fig25.jpg}

\small Electron probability clouds around atom nucleus
\end{center}

\pause
\textbf{Shows:} Where electrons are likely to be found

\textbf{Limitation:} Cannot show exact position at any moment
\note{- This is quantum mechanical model\\\\
- Clouds show probability, not exact paths\\\\
- Electrons don't orbit like planets\\\\
- Model helps us visualize the invisible\\\\
- Turn and talk: why are models useful in physics?\\\\
- ANSWER: study things we can't observe directly}
\end{frame}

\begin{frame}
\frametitle{1.2 The Vocabulary of Discovery}
\begin{block}{The Ladder of Certainty}
\begin{description}
\item[Hypothesis:] Educated guess - testable \pause
\item[Theory:] Explanation verified many times over \pause
\item[Law:] Pattern true in ALL studied cases
\end{description}
\end{block}

\pause

\begin{alertblock}{Civilian View vs. Reality}
\textbf{Civilian:} "It's just a theory" = probably wrong\\
\textbf{Physicist:} "Theory" = extensively tested and supported
\end{alertblock}
\note{[P0] "The ladder of certainty in science"\\\\
[P1] "Hypothesis: educated guess - testable"\\\\
[P2] "Theory: explanation verified many times over"\\\\
[P3] "Law: pattern true in ALL studied cases"\\\\
[P4] [THE CONFLICT] "Civilians say 'just a theory' means probably wrong. Physicists know 'theory' means extensively tested"\\\\
[THE REVELATION] Law describes WHAT, theory explains WHY\\\\
[THE HUMILITY] Both can change with new evidence}
\end{frame}

\begin{frame}
\frametitle{1.2 Nature's Source Code}
\textbf{Scientific laws are:}
\begin{itemize}
\item Universal (work EVERYWHERE in the cosmos) \pause
\item Concise (often one equation) \pause
\item Discovered, not invented
\end{itemize}

\pause
\vspace{0.3cm}

\begin{block}{Universal Law: The Pushback}
\begin{center}
\Large $\boxed{F = ma}$
\end{center}
Force equals mass times acceleration. Works on Earth, Mars, and distant galaxies.
\end{block}
\note{[P0] "Scientific laws are universal, concise, and discovered not invented"\\\\
[P1] "Work everywhere in the cosmos"\\\\
[P2] "Often just one equation"\\\\
[P3] "Discovered, not invented"\\\\
[P4] [THE REVELATION] "F equals m a - works on Earth, Mars, and distant galaxies"\\\\
[THE WONDER] Same equation in your classroom as on Mars\\\\
[THE CONNECTION - Kinetic Archetype] "Every time you push something, you're using this law"}
\end{frame}

\begin{frame}
\frametitle{1.2 Science Is Self-Correcting}
\textbf{Key point:} Even well-established laws and theories can change with new evidence.

\pause
\vspace{0.3cm}

\textbf{Examples:}
\begin{itemize}
\item Microscopes revealed cells \pause
\item Telescopes revealed galaxies \pause
\item Particle accelerators revealed subatomic particles
\end{itemize}

\pause
\vspace{0.3cm}

Scientists say theories are \textbf{supported}, not \textbf{proven}.
\note{[P0] "Even well-established laws and theories can change with new evidence"\\\\
[P1] "Microscopes revealed cells"\\\\
[P2] "Telescopes revealed galaxies"\\\\
[P3] "Particle accelerators revealed subatomic particles"\\\\
[P4] "Scientists say theories are supported, not proven"\\\\
- Cannot prove anything absolutely}
\end{frame}

\section{Physical Quantities and Units}

\begin{frame}
\frametitle{Learning Objectives}
\begin{block}{By the end of this section, you will be able to:}
\begin{itemize}
\item \textbf{1.3:} Use SI units and perform conversions \pause
\item \textbf{1.3:} Apply significant figures in calculations \pause
\item \textbf{1.3:} Create and interpret graphs of physical relationships
\end{itemize}
\end{block}
\note{- Three objectives for units and measurement\\\\
- First: metric system and conversions\\\\
- Second: precision in measurements\\\\
- Third: graphing skills\\\\
- These are practical skills you'll use constantly}
\end{frame}

\begin{frame}
\frametitle{1.3 Standard Units}
\begin{center}
\includegraphics[width=0.6\textwidth,height=0.4\textheight,keepaspectratio]{phys11-physics-fig32.jpg}

\small Distance without units is meaningless!
\end{center}

\pause
\textbf{Units are standardized values for measurement.}

Without them, we can't compare or communicate measurements.
\note{- Imagine measuring distance in "steps"\\\\
- My step is different from your step\\\\
- Science requires standard units\\\\
- SI system: International System of Units\\\\
- Used by scientists worldwide}
\end{frame}

\begin{frame}
\frametitle{1.3 SI Base Units}
\begin{center}
\begin{tabular}{ll}
\textbf{Quantity} & \textbf{SI Unit} \\ \hline
Length & meter (m) \\
Mass & kilogram (kg) \\
Time & second (s) \\
Electric current & ampere (A) \\
Temperature & kelvin (K) \\
Amount of substance & mole (mol) \\
Luminous intensity & candela (cd)
\end{tabular}
\end{center}

\pause
\vspace{0.3cm}
All other units are \textbf{derived} from these seven.
\note{- Seven fundamental units\\\\
- Everything else is combination of these\\\\
- Example: speed is meters per second (m/s)\\\\
- Force is kg times m/s squared (Newton)\\\\
- We'll mainly use first three: m, kg, s}
\end{frame}

\begin{frame}
\frametitle{1.3 The Meter}
\textbf{Current definition (1983):}

The distance light travels in a vacuum in $\frac{1}{299,792,458}$ of a second.

\pause
\vspace{0.3cm}

\textbf{Historical definitions:}
\begin{itemize}
\item 1791: One ten-millionth of distance from equator to North Pole \pause
\item 1889: Distance between marks on platinum-iridium bar \pause
\item 1960: Wavelengths of krypton light \pause
\item 1983: Based on speed of light (current)
\end{itemize}
\note{- Definition gets more precise over time\\\\
- Speed of light is constant, so it's reliable\\\\
- Light speed: 299,792,458 m/s exactly\\\\
- This links meter to second\\\\
- Shows how science improves measurements}
\end{frame}

\begin{frame}
\frametitle{1.3 The Kilogram}
\textbf{Old definition:} Mass of platinum-iridium cylinder in Paris

\pause
\textbf{Problem:} Surface contamination changed mass slightly over time

\pause
\vspace{0.3cm}

\textbf{New definition (2019):} Based on Planck's constant $h$

\pause
More stable and reproducible!
\note{- Old standard: physical object in vault\\\\
- Problem: even platinum changes over time\\\\
- New definition uses quantum mechanics\\\\
- Planck's constant relates photon energy to frequency\\\\
- Now kilogram is based on fundamental constant}
\end{frame}

\begin{frame}
\frametitle{1.3 The Second}
\begin{center}
\includegraphics[width=0.6\textwidth,height=0.4\textheight,keepaspectratio]{phys11-physics-fig35.jpg}

\small Atomic clock
\end{center}

\pause
\textbf{Definition:} Time for 9,192,631,770 cesium atom vibrations

\pause
Accurate to one microsecond per year!
\note{- Old definition: fraction of solar day\\\\
- Problem: Earth's rotation is slowing\\\\
- Cesium atoms vibrate at constant rate\\\\
- Atomic clocks are incredibly precise\\\\
- GPS satellites use atomic clocks}
\end{frame}

\begin{frame}
\frametitle{1.3 Metric Prefixes}
\begin{center}
\small
\begin{tabular}{llll}
\textbf{Prefix} & \textbf{Symbol} & \textbf{Power of 10} & \textbf{Example} \\ \hline
giga- & G & $10^9$ & gigameter \\
mega- & M & $10^6$ & megawatt \\
kilo- & k & $10^3$ & kilometer \\
(base) & - & $10^0$ & meter \\
centi- & c & $10^{-2}$ & centimeter \\
milli- & m & $10^{-3}$ & millimeter \\
micro- & $\mu$ & $10^{-6}$ & micrometer \\
nano- & n & $10^{-9}$ & nanometer
\end{tabular}
\end{center}

\pause
\vspace{0.3cm}
Conversions are easy - just move decimal point!
\note{- Each prefix is power of 10\\\\
- 1 km equals 1000 m\\\\
- 1 m equals 100 cm equals 1000 mm\\\\
- Much easier than feet, inches, miles\\\\
- Just move decimal and change prefix}
\end{frame}

\begin{frame}
\frametitle{1.3 Range of Measurements}
\begin{itemize}
\item Diameter of proton: $10^{-15}$ m \pause
\item Diameter of atom: $10^{-10}$ m \pause
\item Human height: $10^{0}$ m \pause
\item Diameter of Earth: $10^{7}$ m \pause
\item Diameter of Sun: $10^{9}$ m \pause
\item Distance to nearest star: $10^{16}$ m
\end{itemize}

\pause
\vspace{0.3cm}
Physics spans 31 orders of magnitude!
\note{- One system covers tiny to enormous\\\\
- Proton to stars: 31 powers of 10\\\\
- That's the beauty of metric system\\\\
- Same units, just different prefixes\\\\
- Shows vastness of universe we study}
\end{frame}

\begin{frame}
\frametitle{1.3 Scientific Notation}
\textbf{Format:} $x \times 10^y$

\pause
\vspace{0.3cm}

\textbf{Examples:}
\begin{itemize}
\item $840,000,000,000,000 = 8.4 \times 10^{14}$ \pause
\item $0.0000045 = 4.5 \times 10^{-6}$
\end{itemize}

\pause
\vspace{0.3cm}

\textbf{Positive exponent:} Move decimal right (large number)

\textbf{Negative exponent:} Move decimal left (small number)
\note{- Essential for very large or small numbers\\\\
- x is the significant digits\\\\
- y tells you how many places to move decimal\\\\
- Positive y: big number\\\\
- Negative y: fraction\\\\
- Calculator uses E notation: 8.4 E 14}
\end{frame}

\begin{frame}
\frametitle{1.3 Order of Magnitude}
\textbf{Definition:} The power of 10 in scientific notation

\pause
\vspace{0.3cm}

\textbf{Examples:}
\begin{itemize}
\item 800 = $8 \times 10^2$ \pause
\item 450 = $4.5 \times 10^2$ \pause
\item Both are order of magnitude $10^2$
\end{itemize}

\pause
\vspace{0.3cm}

\alert{Ballpark estimate} for scale of a value.
\note{- Order of magnitude: rough size\\\\
- 800 and 450 are same order of magnitude\\\\
- Diameter of atom: 10 to the negative 9 m\\\\
- Diameter of sun: 10 to the 9 m\\\\
- 18 orders of magnitude apart\\\\
- Useful for quick estimates}
\end{frame}

\begin{frame}
\frametitle{1.3 Unit Conversion}
\textbf{Conversion factor:} A ratio equal to 1

\pause
\vspace{0.3cm}

\textbf{Example:} Convert 1 hour to seconds
\begin{align*}
1 \text{ h} &\times \frac{60 \text{ min}}{1 \text{ h}} \times \frac{60 \text{ s}}{1 \text{ min}} \\
&= 3600 \text{ s} = 3.6 \times 10^3 \text{ s}
\end{align*}

\pause
\textbf{Key:} Units cancel like algebra!
\note{- Conversion factor equals 1, so doesn't change value\\\\
- Multiply to cancel unwanted units\\\\
- Leave desired units\\\\
- Hours cancel, minutes cancel, seconds remain\\\\
- Check: did you get the right units?}
\end{frame}

\begin{frame}
\frametitle{Attempt: Decoding Motion}
\begin{exampleblock}{The Challenge (3 min, silent)}
A car travels 10.0 km in 20.0 min.

\vspace{0.3cm}

\textbf{Given:}
\begin{itemize}
\item distance = 10.0 km
\item time = 20.0 min
\end{itemize}

\textbf{Find:} Average speed in km/h

\vspace{0.3cm}

\textit{Can you decode this motion? Work silently.}
\end{exampleblock}
\note{[THE CHALLENGE] Can they decode motion like a physicist?\\\\
[SAY] "Try this on your own. It's okay to get stuck."\\\\
[TIMING] 3-4 min SILENT individual work\\\\
[CIRCULATE] Note who finishes early, who's stuck on conversion\\\\
[WATCH FOR] Students dividing instead of multiplying\\\\
[DON'T HELP] Let them struggle - learning happens in Compare}
\end{frame}

\begin{frame}
\frametitle{Compare: Unit Conversion}
\textbf{Turn and talk (2 min):}

\vspace{0.3cm}

\begin{enumerate}
\item What formula did you use for average speed?
\item How did you convert minutes to hours?
\item Did you multiply or divide by 60?
\end{enumerate}

\vspace{0.5cm}

\pause
\alert{Name wheel:} One pair share your approach (not your answer).
\note{[TIMING] 2-3 min pair discussion\\\\
[CIRCULATE] Listen for common approaches\\\\
[CHECK] Name wheel: call a pair to share approach\\\\
[EXPECTED APPROACH] Speed equals distance over time, then convert min to h by multiplying by 60 min per 1 h\\\\
[COMMON ERROR] Using 1 h over 60 min (upside down)}
\end{frame}

\begin{frame}
\frametitle{Reveal: The Math of Motion}
\textbf{Self-correct in a different color:}

\vspace{0.3cm}

\textbf{Step 1:} Average speed = $\frac{\text{distance}}{\text{time}}$

\pause
\vspace{0.2cm}

\textbf{Step 2:} $\frac{10.0 \text{ km}}{20.0 \text{ min}} = 0.500 \frac{\text{km}}{\text{min}}$

\pause
\vspace{0.2cm}

\textbf{Step 3:} Convert using $\frac{60 \text{ min}}{1 \text{ h}}$

\pause
\vspace{0.2cm}

$$0.500 \frac{\text{km}}{\text{min}} \times \frac{60 \text{ min}}{1 \text{ h}} = \boxed{30.0 \text{ km/h}}$$

\pause
\textbf{Check:} 10 km in 1/3 hour = 30 km in 1 hour. Reasonable!
\note{[P0] "Self-correct in a different color"\\\\
[P1] [ALGEBRA] "Average speed equals distance over time"\\\\
[P2] "10 km divided by 20 min equals 0.5 km per min"\\\\
[P3] "Multiply by 60 min over 1 hour - minutes cancel"\\\\
[P4] [ANSWER] "30.0 km/h - 10 km in third of hour, so 30 km in full hour"\\\\
[THE WONDER] You just did what GPS satellites do constantly}
\end{frame}

\begin{frame}
\frametitle{1.3 Accuracy vs Precision}
\begin{columns}[T]
\column{0.48\textwidth}
\begin{block}{Accuracy}
How close measurement is to true value
\end{block}

\pause
\column{0.48\textwidth}
\begin{block}{Precision}
How close repeated measurements are to each other
\end{block}
\end{columns}

\pause
\vspace{0.3cm}

\begin{alertblock}{Key Difference}
Accuracy = correctness. Precision = consistency. You can have one without the other!
\end{alertblock}
\note{[P0] "Two different qualities of measurement"\\\\
[P1] "Accuracy: how close to the true value"\\\\
[P2] "Precision: how close repeated measurements are to each other"\\\\
[P3] "You can have one without the other - we want both!"}
\end{frame}

\begin{frame}
\frametitle{1.3 Target Analogy}
\begin{columns}[T]
\column{0.32\textwidth}
\begin{center}
\includegraphics[width=\linewidth,height=0.35\textheight,keepaspectratio]{phys11-physics-fig48.jpg}

\tiny Accurate, not precise
\end{center}

\pause
\column{0.32\textwidth}
\begin{center}
\includegraphics[width=\linewidth,height=0.35\textheight,keepaspectratio]{phys11-physics-fig49.jpg}

\tiny Precise, not accurate
\end{center}

\pause
\column{0.32\textwidth}
\begin{center}
\includegraphics[width=\linewidth,height=0.35\textheight,keepaspectratio]{phys11-physics-fig50.jpg}

\tiny Accurate and precise
\end{center}
\end{columns}
\note{[P0] "GPS trying to locate a restaurant - three patterns"\\\\
[P1] "Left: spread out but near center - accurate but not precise"\\\\
[P2] "Middle: clustered but off center - precise but not accurate"\\\\
[P3] "Right: clustered near center - both! That's what we want"}
\end{frame}

\begin{frame}
\frametitle{1.3 Significant Figures}
\begin{block}{Definition: Significant Figures}
All measured digits plus one estimated digit
\end{block}

\pause
\vspace{0.2cm}

\textbf{Rules:}
\begin{enumerate}
\item Non-zero digits are always significant \pause
\item Zeros between non-zero digits are significant \pause
\item Leading zeros are NOT significant \pause
\item Trailing zeros after decimal ARE significant
\end{enumerate}

\pause

\begin{alertblock}{Common Mistake}
Leading zeros (0.0045) are NOT significant - they're just placeholders!
\end{alertblock}
\note{[P0] "All measured digits plus one estimated digit"\\\\
[P1] "Non-zero digits are always significant"\\\\
[P2] "Zeros between non-zero digits are significant"\\\\
[P3] "Leading zeros are NOT significant"\\\\
[P4] "Trailing zeros after decimal ARE significant"\\\\
[P5] "Common mistake: 0.0045 - leading zeros are just placeholders, only 2 sig figs"}
\end{frame}

\begin{frame}
\frametitle{1.3 Sig Figs in Calculations}
\textbf{Multiplication and Division:}

Answer has same number of sig figs as least precise value.

\pause
\vspace{0.3cm}

\textbf{Addition and Subtraction:}

Answer has same decimal place as least precise value.

\pause
\vspace{0.3cm}

\textbf{Example:}
$$\pi r^2 = (3.1415927) \times (2.0 \text{ m})^2 = 4.5238934 \text{ m}^2$$

\pause
Round to 2 sig figs: $\boxed{4.5 \text{ m}^2}$
\note{- Least precise input determines output precision\\\\
- Radius has 2 sig figs, so area has 2 sig figs\\\\
- Don't round during calculation\\\\
- Only round final answer\\\\
- Keep extra digits in calculator while working}
\end{frame}

\begin{frame}
\frametitle{1.3 Uncertainty in Measurements}
\textbf{All measurements have uncertainty.}

\pause
\vspace{0.3cm}

\textbf{Sources:}
\begin{itemize}
\item Limitations of measuring device \pause
\item Skill of person measuring \pause
\item Irregularities in object \pause
\item Environmental factors
\end{itemize}

\pause
\vspace{0.3cm}

\textbf{Notation:} $11.0 \pm 0.2$ inches

Means: actual value between 10.8 and 11.2 inches
\note{- Nothing is perfectly known\\\\
- Always some uncertainty\\\\
- Good practice: identify sources of uncertainty\\\\
- Reduce uncertainty when possible\\\\
- Report measurements with uncertainty\\\\
- Plus or minus shows range of possible values}
\end{frame}

\begin{frame}
\frametitle{1.3 Percent Uncertainty}
\textbf{Formula:}
$$\text{Percent uncertainty} = \frac{\delta A}{A} \times 100\%$$

\pause
\vspace{0.3cm}

\textbf{Example:} 5-lb bag of apples has uncertainty of $\pm 0.4$ lb

\pause
$$\text{Percent uncertainty} = \frac{0.4 \text{ lb}}{5 \text{ lb}} \times 100\% = 8\%$$

\pause
The bag weighs $5 \text{ lb} \pm 8\%$
\note{- Percent uncertainty: relative measure\\\\
- Tells you how big uncertainty is compared to value\\\\
- 8 percent means fairly uncertain\\\\
- Smaller percent means more precise\\\\
- Must multiply by 100 to get percent!}
\end{frame}

\section{Summary}

\begin{frame}
\frametitle{What You Now Know}
\begin{block}{The Revelations}
\begin{enumerate}
\item Physics = the source code of reality (atoms to galaxies) \pause
\item Scientific method = humanity's truth-detection system \pause
\item Models = visualizing the invisible \pause
\item SI units = universal language of measurement \pause
\item Sig figs = honesty about precision \pause
\item Uncertainty = the humility of science
\end{enumerate}
\end{block}
\note{[P0] "Six revelations today"\\\\
[P1] "Physics is the source code of reality"\\\\
[P2] "Scientific method is humanity's truth-detection system"\\\\
[P3] "Models let us visualize the invisible"\\\\
[P4] "SI units are the universal language"\\\\
[P5] "Sig figs show honesty about precision"\\\\
[P6] "Uncertainty is the humility of science"\\\\
[THE WONDER] You now see the universe differently - 400-year tradition\\\\
- Name wheel: which was most surprising?}
\end{frame}

\begin{frame}
\frametitle{Key Equations}
\begin{align}
\text{Average speed} &= \frac{\text{distance}}{\text{time}} \\
\text{Percent uncertainty} &= \frac{\delta A}{A} \times 100\%
\end{align}
\note{- These are foundational formulas\\\\
- Average speed: we'll use this a lot\\\\
- Percent uncertainty: for lab reports\\\\
- Know when to use each\\\\
- Questions before we end?}
\end{frame}

\begin{frame}
\frametitle{Homework}
\begin{center}
\Large
Complete the assigned problems\\[0.3cm]
posted on the LMS
\end{center}
\note{[SAY] "Homework is posted on the LMS"\\\\
[TIMING] Due date: check LMS\\\\
[CHECK] Questions before we end?\\\\
[TRANSITION] Next class: Chapter 2 Motion Along a Straight Line}
\end{frame}

\end{document}
