\documentclass{beamer}
\usepackage{../../../shared/templates/ds9_theme}
\usepackage[overridenote]{pdfpc}
\graphicspath{{../images/}{../../shared/images/}}

\title[Measuring Motion]{PHYS11 CH:2 How to Measure Motion}
\subtitle{Distance, Displacement, Speed, and Velocity}
\author[Mr. Gullo]{Mr. Gullo}
\date[December 2025]{December 2025}

\begin{document}

\frame{\titlepage
\note{[THE HOOK] Today we learn the physicist's toolkit for measuring motion.\\\\
- Two revelations: how to describe WHERE (distance vs displacement), how to describe HOW FAST (speed vs velocity)\\\\
[THE WONDER] Same language astronomers use to track planets, engineers use for spacecraft\\\\
- This is foundational for everything we'll do in kinematics}
}

\begin{frame}
\frametitle{Outline}
\tableofcontents
\end{frame}

\section{Introduction}

\begin{frame}
\frametitle{The Mystery}
\begin{center}
\Large How do you know something is moving?
\end{center}

\pause
\vspace{0.5cm}
Everything is in motion. Your heart pumps blood. Atoms vibrate. Earth orbits the Sun.

\pause
\vspace{0.3cm}
\alert{But motion is always relative to something.}
\note{[P0] "How do you know something is moving?"\\\\
[P1] "Everything is in motion - heart pumping, atoms vibrating, Earth orbiting"\\\\
[P2] [THE REVELATION] "Motion is always relative to something. You cannot describe motion without a reference frame"\\\\
[THE WONDER] This idea revolutionized physics 400 years ago}
\end{frame}

\begin{frame}
\frametitle{What Is Kinematics?}
\begin{block}{The Study of Motion}
Kinematics is the study of motion without considering its causes.
\end{block}

\pause
\vspace{0.3cm}

From Greek \textit{kinema} = motion

\pause
\vspace{0.3cm}

\begin{exampleblock}{Real-World Examples}
\begin{itemize}
\item Tennis ball flying across court
\item Spacecraft orbiting Mars
\item Your walk to class
\end{itemize}
\end{exampleblock}
\note{[P0] "Kinematics is the study of motion without considering its causes"\\\\
[P1] "From Greek word kinema meaning motion - same root as cinema, movies"\\\\
[P2] [THE CONNECTION - Kinetic Archetype] "Every sport is applied kinematics"\\\\
- Tennis serves, basketball shots, track and field\\\\
[THE HUMILITY] We start simple: straight-line motion only}
\end{frame}

\section{Relative Motion, Distance, and Displacement}

\begin{frame}
\frametitle{Learning Objectives}
\begin{block}{By the end of this section, you will be able to:}
\begin{itemize}
\item \textbf{2.1:} Describe motion in different reference frames \pause
\item \textbf{2.1:} Define distance and displacement, and distinguish between the two \pause
\item \textbf{2.1:} Solve problems involving distance and displacement
\end{itemize}
\end{block}
\note{[P0] "Three objectives for section 2.1"\\\\
[P1] "First: understand reference frames - viewpoint matters"\\\\
[P2] "Second: two ways to measure how far - distance and displacement"\\\\
[P3] "Third: calculations with both quantities"\\\\
- Assessment: practice problems and quiz}
\end{frame}

\begin{frame}
\frametitle{2.1 The Reference Frame Problem}
\begin{figure}
\centering
\includegraphics[width=0.7\textwidth,height=0.5\textheight,keepaspectratio]{phys11-kinematics-fig2-1.jpg}
\caption{Are clouds a useful reference frame for airplane passengers?}
\end{figure}

\pause
\textbf{Reference frame:} The coordinate system from which positions are described.
\note{[P0][Fig 2.1: Shanghai Maglev. At this]  "Are clouds a useful reference frame?"\\\\
[P1] "No! Clouds move with the plane"\\\\
[THE CONFLICT] Your intuition says motion is obvious. Physics says: compared to what?\\\\
- Need a reference frame that is either stationary or has known motion\\\\
- Earth is common reference frame, but not the only one}
\end{frame}

\begin{frame}
\frametitle{2.1 Galileo's Revelation}
\begin{center}
\includegraphics[width=0.5\textwidth,height=0.4\textheight,keepaspectratio]{phys11-kinematics-fig2-4.jpg}

\small Galileo Galilei (1564-1642)
\end{center}

\pause
\textbf{Thought experiment:} Person in windowless ship on calm sea.

\pause
Can they tell if the ship is moving?
\note{[P0][Fig 2.4: Your total change in]  "Galileo - one of the first to explore reference frames"\\\\
[P1] "Thought experiment: windowless ship on calm sea"\\\\
[P2] "Can person inside tell if ship is moving?"\\\\
[ANSWER] No! Only by looking at shore can they compare\\\\
[THE REVELATION] Motion is only meaningful when you specify a reference frame\\\\
[THE WONDER] This insight led to Einstein's relativity 300 years later}
\end{frame}

\begin{frame}
\frametitle{2.1 Galileo's Conclusion}
\begin{block}{Universal Truth}
Only by comparing to each other can observers describe relative motion.
\end{block}

\pause
\vspace{0.3cm}

Person on ship: "Shore is moving past me."

Person on shore: "Ship is moving past me."

\pause
\vspace{0.3cm}

\begin{alertblock}{The Paradox}
Both are correct! Description depends on the reference frame.
\end{alertblock}
\note{[P0] "Galileo's conclusion: comparison is everything"\\\\
[P1] "Person on ship sees shore moving"\\\\
[P2] [THE CONFLICT] "Both observers are correct - no preferred reference frame"\\\\
[THE HUMILITY] This confused philosophers for centuries\\\\
- Name wheel: who is really moving?\\\\
[ANSWER] Both, relative to each other}
\end{frame}

\begin{frame}
\frametitle{2.1 Two Ways to Measure Motion}
\begin{block}{Distance}
Total length of the path traveled. (Scalar)
\end{block}

\pause

\begin{block}{Displacement}
Straight-line change in position from start to finish. (Vector)
\end{block}

\pause

\begin{alertblock}{Key Difference}
\textbf{Civilian:} "I drove 100 km today."\\
\textbf{Physicist:} "But your displacement was 0 - you ended where you started."
\end{alertblock}
\note{[P0] "Two ways to measure how far you've gone"\\\\
[P1] "Distance: total length of path - like odometer reading"\\\\
[P2] "Displacement: straight line start to finish - as crow flies"\\\\
[P3] [THE CONFLICT] "Drive to store and back - distance 100 km, displacement zero"\\\\
[THE CONNECTION - Kinetic Archetype] "Runners care about distance. GPS cares about displacement"\\\\
[THE WONDER] Same journey, different numbers - physics needs both}
\end{frame}

\begin{frame}
\frametitle{2.1 Visualizing Distance vs. Displacement}
\begin{figure}
\centering
\includegraphics[width=0.7\textwidth,height=0.5\textheight,keepaspectratio]{phys11-kinematics-fig2-10.jpg}
\caption{Short line is displacement. Curved path is distance.}
\end{figure}

\pause
Distance $\geq$ Displacement (always!)
\note{[P0] [Fig 2.10: The diagram shows a] "Visual representation of distance vs displacement"\\\\
[P1] "Short straight arrow is displacement magnitude"\\\\
- "Curved path length is distance"\\\\
- Distance is always greater than or equal to displacement magnitude\\\\
[THE CONNECTION - Digital Archetype] "Video games track both - map distance vs straight-line targeting"}
\end{frame}

\begin{frame}
\frametitle{2.1 The Round Trip}
\begin{figure}
\centering
\includegraphics[width=0.7\textwidth,height=0.5\textheight,keepaspectratio]{phys11-kinematics-fig2-6.jpg}
\caption{Drive to school (5 km) and back home (5 km)}
\end{figure}

\pause
\textbf{Distance:} 10 km

\pause
\textbf{Displacement:} 0 km (start and end at same position)
\note{[P0][Fig 2.6: The total distance that]  "Classic example: drive to school and back"\\\\
[P1] "Distance: 10 km total - odometer reading"\\\\
[P2] "Displacement: zero - ended where you started"\\\\
[THE REVELATION] Displacement cares about net change in position\\\\
- If forward is positive, backward is negative\\\\
- They cancel: +5 km + (-5 km) = 0}
\end{frame}

\begin{frame}
\frametitle{2.1 Scalars vs. Vectors}
\begin{columns}[T]
\column{0.48\textwidth}
\begin{block}{Scalar}
Magnitude only
\end{block}

\pause
\textbf{Examples:}
\begin{itemize}
\item Distance
\item Speed
\item Time
\item Temperature
\item Mass
\end{itemize}

\pause
\column{0.48\textwidth}
\begin{block}{Vector}
Magnitude AND direction
\end{block}

\pause
\textbf{Examples:}
\begin{itemize}
\item Displacement
\item Velocity
\item Force
\item Acceleration
\end{itemize}
\end{columns}
\note{[P0] "Two types of quantities in physics"\\\\
[P1] "Scalars have magnitude only - just a number with units"\\\\
[P2] "Vectors have magnitude and direction - size and where"\\\\
[P3] "Examples: temperature is scalar, velocity is vector"\\\\
- Vectors will be bold or have arrows above them\\\\
[THE CONNECTION - Digital Archetype] "Game engines track position vectors constantly"}
\end{frame}

\begin{frame}
\frametitle{2.1 Displacement Equation}
\begin{block}{Calculating Displacement}
\begin{center}
\Large $\boxed{\Delta \mathbf{d} = \mathbf{d}_{\mathrm{f}} - \mathbf{d}_{0}}$
\end{center}
Change in position equals final position minus initial position.
\end{block}

\pause
\vspace{0.3cm}

\textbf{Symbols:}
\begin{itemize}
\item $\Delta$ (delta) = change in
\item $\mathbf{d}_{0}$ = initial position
\item $\mathbf{d}_{\mathrm{f}}$ = final position
\end{itemize}
\note{[P0] [THE REVELATION] "The displacement equation - foundation of kinematics"\\\\
- Delta means change in\\\\
- d-naught is initial position, d-f is final position\\\\
[P1] "Subtract where you started from where you ended"\\\\
[ALGEBRA] "Delta d equals d-f minus d-naught"\\\\
[THE WONDER] Same equation for ant crawling on desk or spacecraft going to Mars}
\end{frame}

\begin{frame}
\frametitle{2.1 Choosing Your Axis}
\textbf{You must define:}
\begin{enumerate}
\item Origin (zero point) \pause
\item Positive direction \pause
\item Negative direction (opposite)
\end{enumerate}

\pause
\vspace{0.3cm}

\begin{exampleblock}{Smart Choices}
Choose origin and direction to make calculations easiest.
\begin{itemize}
\item Often: starting position = 0
\item Often: forward/right = positive
\end{itemize}
\end{exampleblock}
\note{[P0] "Before calculating, set up your coordinate system"\\\\
[P1] "Pick an origin - often your starting point"\\\\
[P2] "Pick positive direction - often forward or right"\\\\
[P3] "Negative is opposite direction"\\\\
[P4] [THE CONNECTION] "Like setting spawn point in video game"\\\\
- Make life easy: start at zero\\\\
- Be consistent throughout problem}
\end{frame}

\begin{frame}
\frametitle{Attempt: The Cyclist}
\begin{exampleblock}{The Challenge (3 min, silent)}
A cyclist rides 3 km west and then turns around and rides 2 km east.

\vspace{0.3cm}

\textbf{Find:}
\begin{itemize}
\item (a) Displacement
\item (b) Distance traveled
\item (c) Magnitude of displacement
\end{itemize}

\vspace{0.3cm}

\textit{Hint: Choose your positive direction first. Work silently.}
\end{exampleblock}
\note{[THE CHALLENGE] Can you track motion in two directions?\\\\
[SAY] "Try this on your own. Choose east or west as positive - your choice."\\\\
[TIMING] 3-4 min SILENT individual work\\\\
[CIRCULATE] Note who chooses which direction as positive\\\\
[WATCH FOR] Students forgetting about direction, adding distances instead\\\\
[DON'T HELP] Let them struggle with vector concept}
\end{frame}

\begin{frame}
\frametitle{Compare: The Cyclist}
\textbf{Turn and talk (2 min):}

\vspace{0.3cm}

\begin{enumerate}
\item Which direction did you choose as positive?
\item What was your displacement value?
\item Did you get a negative displacement? What does that mean?
\item How is distance different from displacement?
\end{enumerate}

\vspace{0.5cm}

\pause
\alert{Name wheel:} One pair share your approach.
\note{[TIMING] 2-3 min pair discussion\\\\
[CIRCULATE] Listen for understanding of positive/negative\\\\
[CHECK] Name wheel: call a pair to share\\\\
[EXPECTED APPROACH] Choose east positive, west negative (or vice versa), subtract positions\\\\
[COMMON ERROR] Adding both distances, forgetting direction matters}
\end{frame}

\begin{frame}
\frametitle{Reveal: The Cyclist Solution}
\textbf{Self-correct in a different color:}

\vspace{0.3cm}

\textbf{Setup:} Choose east = positive, west = negative

\pause

\textbf{(a) Displacement:}
$$\Delta \mathbf{d} = \mathbf{d}_{\mathrm{f}} - \mathbf{d}_{0} = -1 \text{ km}$$
(or 1 km west)

\pause

\textbf{(b) Distance traveled:}
$$\text{Distance} = 3 \text{ km} + 2 \text{ km} = 5 \text{ km}$$

\pause

\textbf{(c) Magnitude of displacement:}
$$|\Delta \mathbf{d}| = 1 \text{ km}$$
\note{[P0] "Self-correct in a different color"\\\\
[P1] [ALGEBRA] "Displacement equals final minus initial: negative 1 km"\\\\
- Negative because net motion is west (negative direction)\\\\
[P2] "Distance: add path lengths - direction doesn't matter: 5 km"\\\\
[P3] [ANSWER] "Magnitude is absolute value: 1 km"\\\\
[THE WONDER] GPS uses displacement. Fitness tracker uses distance. Different tools for different jobs.}
\end{frame}

\begin{frame}
\frametitle{2.1 The Mars Probe Disaster}
\begin{figure}
\centering
\includegraphics[width=0.6\textwidth,height=0.45\textheight,keepaspectratio]{phys11-kinematics-fig2-18.jpg}
\caption{Mars Climate Orbiter, 1998}
\end{figure}

\pause
\textbf{Cost:} \$125 million

\textbf{Mistake:} Calculations in English units, not SI units

\textbf{Result:} Probe crashed into Mars atmosphere
\note{[P0] [Fig 2.18: The graph shows the] "Real-world consequence of unit mistakes"\\\\
[P1] "125 million dollar satellite destroyed"\\\\
- American scientists forgot to convert feet to meters\\\\
- Probe flew too close to Mars, disintegrated\\\\
[THE HUMILITY] Even NASA makes mistakes\\\\
[THE REVELATION] Always use SI units in physics: meters, kilograms, seconds}
\end{frame}

\section{Speed and Velocity}

\begin{frame}
\frametitle{Learning Objectives}
\begin{block}{By the end of this section, you will be able to:}
\begin{itemize}
\item \textbf{2.2:} Calculate the average speed of an object \pause
\item \textbf{2.2:} Relate displacement and average velocity
\end{itemize}
\end{block}
\note{[P0] "Two objectives for section 2.2"\\\\
[P1] "First: calculate average speed - how fast"\\\\
[P2] "Second: average velocity - how fast AND which direction"\\\\
- These build on distance and displacement\\\\
- Assessment: practice problems}
\end{frame}

\begin{frame}
\frametitle{2.2 What Is Speed?}
\begin{block}{The Rate of Motion}
Speed is the rate at which an object changes its location. (Scalar)
\end{block}

\pause
\vspace{0.3cm}

\textbf{SI unit:} meters per second (m/s)

\textbf{Other units:} km/h, mph

\pause
\vspace{0.3cm}

\begin{exampleblock}{The Mental Model}
Speed is your odometer. It tells you how fast, but not where you're going.
\end{exampleblock}
\note{[P0] "Speed is the rate of motion - how fast position changes"\\\\
[P1] "SI unit: meters per second"\\\\
- Also use km/h for cars, mph in some countries\\\\
[P2] [THE CONNECTION - Kinetic Archetype] "Speedometer shows instantaneous speed"\\\\
[THE REVELATION] Speed is scalar - no direction, just how fast}
\end{frame}

\begin{frame}
\frametitle{2.2 Average Speed}
\begin{block}{Universal Law: Rate of Motion}
\begin{center}
\Large $\boxed{v_{\text{avg}} = \frac{\text{distance}}{\text{time}}}$
\end{center}
Average speed equals total distance divided by total time.
\end{block}

\pause
\vspace{0.3cm}

\textbf{Rearranged:}
\begin{align*}
\text{distance} &= v_{\text{avg}} \times \text{time}\\
\text{time} &= \frac{\text{distance}}{v_{\text{avg}}}
\end{align*}
\note{[P0] [THE REVELATION] "The speed equation - first kinematics formula"\\\\
[ALGEBRA] "v-average equals distance over time"\\\\
- Can rearrange to solve for distance or time\\\\
[P1] "Multiply both sides by time to get distance"\\\\
- Divide both sides by v-avg to get time\\\\
[THE WONDER] Same equation for walking to class or light traveling from stars}
\end{frame}

\begin{frame}
\frametitle{2.2 Average vs. Instantaneous Speed}
\textbf{Average speed:} Total distance divided by total time

\pause
\vspace{0.3cm}

\textbf{Example:} Car travels 150 km in 3.2 hours
$$v_{\text{avg}} = \frac{150 \text{ km}}{3.2 \text{ h}} = 47 \text{ km/h}$$

\pause
\vspace{0.3cm}

\textbf{Instantaneous speed:} Speed at a specific instant

\begin{exampleblock}{Real-World}
Your car's speedometer shows instantaneous speed, not average.
\end{exampleblock}
\note{[P0] "Two types of speed"\\\\
[P1] "Average: total distance over total time - 47 km/h for whole trip"\\\\
- Car sped up and slowed down during journey\\\\
[P2] "Instantaneous: speed at one moment"\\\\
[P3] [THE CONNECTION] "Speedometer shows instantaneous speed right now"\\\\
- Changes constantly as you drive}
\end{frame}

\begin{frame}
\frametitle{2.2 Round Trip Speed Problem}
\begin{figure}
\centering
\includegraphics[width=0.7\textwidth,height=0.5\textheight,keepaspectratio]{phys11-kinematics-fig2-8.jpg}
\caption{30-minute round trip, 6 km total}
\end{figure}

\pause
\textbf{Average speed:} $\frac{6 \text{ km}}{0.5 \text{ h}} = 12 \text{ km/h}$

\textbf{Displacement:} 0 km (round trip)
\note{[P0][Fig 2.8: During a 30 -minute]  "Round trip example"\\\\
[P1] "Total distance 6 km in half hour = 12 km/h average speed"\\\\
- But displacement is zero - ended where started\\\\
[THE CONFLICT] Speed and velocity will give different answers\\\\
- Speed uses distance, velocity uses displacement}
\end{frame}

\begin{frame}
\frametitle{Attempt: The Marble}
\begin{exampleblock}{The Challenge (3 min, silent)}
A marble rolls 5.2 m in 1.8 s.

\vspace{0.3cm}

\textbf{Find:} The marble's average speed

\vspace{0.3cm}

\textit{Work silently. Show your units.}
\end{exampleblock}
\note{[THE CHALLENGE] Can you calculate rate of motion?\\\\
[SAY] "Try this on your own. Use the speed equation."\\\\
[TIMING] 3 min SILENT individual work\\\\
[CIRCULATE] Note who forgets units\\\\
[WATCH FOR] Division errors, unit mistakes\\\\
[DON'T HELP] Let them apply formula themselves}
\end{frame}

\begin{frame}
\frametitle{Compare: The Marble}
\textbf{Turn and talk (2 min):}

\vspace{0.3cm}

\begin{enumerate}
\item What formula did you use?
\item What did you divide?
\item What units did you get?
\item Does your answer make sense? How fast is it?
\end{enumerate}

\vspace{0.5cm}

\pause
\alert{Name wheel:} One pair share your answer and reasoning.
\note{[TIMING] 2-3 min pair discussion\\\\
[CIRCULATE] Listen for correct formula use\\\\
[CHECK] Name wheel: call a pair to share\\\\
[EXPECTED APPROACH] v-avg = distance/time = 5.2/1.8\\\\
[COMMON ERROR] Forgetting units, reversing division}
\end{frame}

\begin{frame}
\frametitle{Reveal: The Marble Solution}
\textbf{Self-correct in a different color:}

\vspace{0.3cm}

\textbf{Given:}
\begin{itemize}
\item Distance = 5.2 m
\item Time = 1.8 s
\end{itemize}

\pause

\textbf{Equation:}
$$v_{\text{avg}} = \frac{\text{distance}}{\text{time}}$$

\pause

\textbf{Substitute:}
$$v_{\text{avg}} = \frac{5.2 \text{ m}}{1.8 \text{ s}} = 2.9 \text{ m/s}$$

\pause

\textbf{Check:} About 3 m/s - speed of brisk walk. Reasonable!
\note{[P0] "Self-correct in a different color"\\\\
[P1] [ALGEBRA] "v-average equals distance over time"\\\\
[P2] "Substitute: 5.2 meters divided by 1.8 seconds"\\\\
[P3] [ANSWER] "2.9 meters per second"\\\\
[P4] "Reality check: brisk walking speed - makes sense for rolling marble"\\\\
[THE WONDER] You just described motion the way Galileo did 400 years ago}
\end{frame}

\begin{frame}
\frametitle{2.2 What Is Velocity?}
\begin{block}{Speed Plus Direction}
Velocity is the vector version of speed. It includes both magnitude (how fast) and direction (which way).
\end{block}

\pause
\vspace{0.3cm}

\begin{exampleblock}{The Mental Model}
Velocity is your speedometer with a compass attached.\\
Speed tells you how fast. Velocity adds where.
\end{exampleblock}

\pause
\vspace{0.3cm}

\textbf{Bold vs. italics:}
\begin{itemize}
\item $\mathbf{v}$ = velocity (vector, includes direction)
\item $v$ = speed (scalar, magnitude only)
\end{itemize}
\note{[P0] "Velocity is vector version of speed"\\\\
[P1] [THE CONNECTION] "Speedometer shows speed. GPS shows velocity - speed AND direction"\\\\
[P2] "Notation: bold v is vector, italic v is scalar"\\\\
- Vectors can have arrows above them too\\\\
[THE REVELATION] Velocity is displacement over time, just like speed is distance over time}
\end{frame}

\begin{frame}
\frametitle{2.2 Average Velocity}
\begin{block}{Universal Law: Rate of Displacement}
\begin{center}
\Large $\boxed{\mathbf{v}_{\text{avg}} = \frac{\Delta \mathbf{d}}{\Delta t} = \frac{\mathbf{d}_{\mathrm{f}} - \mathbf{d}_{0}}{t_{\mathrm{f}} - t_{0}}}$
\end{center}
Average velocity equals displacement divided by time interval.
\end{block}

\pause
\vspace{0.3cm}

\textbf{SI unit:} m/s (same as speed, but includes direction)

\pause
\vspace{0.3cm}

\begin{alertblock}{The Paradox}
Round trip: Speed = 12 km/h, but velocity = 0 km/h!\\
(Displacement is zero)
\end{alertblock}
\note{[P0] [THE REVELATION] "The velocity equation"\\\\
[ALGEBRA] "v-average equals delta d over delta t"\\\\
- Displacement divided by time interval\\\\
[P1] "Same units as speed: meters per second"\\\\
[P2] [THE CONFLICT] "Round trip: speed is 12 km/h but velocity is zero"\\\\
- Because displacement is zero\\\\
[THE HUMILITY] This confuses everyone at first}
\end{frame}

\begin{frame}
\frametitle{2.2 Speed vs. Velocity}
\begin{columns}[T]
\column{0.48\textwidth}
\begin{block}{Speed (Scalar)}
\begin{itemize}
\item Uses distance
\item Magnitude only
\item Always positive
\item Example: 50 km/h
\end{itemize}
\end{block}

\pause
\column{0.48\textwidth}
\begin{block}{Velocity (Vector)}
\begin{itemize}
\item Uses displacement
\item Magnitude + direction
\item Can be negative
\item Example: 50 km/h north
\end{itemize}
\end{block}
\end{columns}

\pause
\vspace{0.3cm}

\begin{alertblock}{Civilian View vs. Reality}
\textbf{Civilian:} "Speed and velocity are the same thing."\\
\textbf{Physicist:} "Speed is how fast. Velocity is how fast AND which way."
\end{alertblock}
\note{[P0] "Two sides of the same coin"\\\\
[P1] "Speed: scalar, distance, always positive"\\\\
[P2] "Velocity: vector, displacement, includes direction"\\\\
[P3] [THE CONFLICT] "Everyday language: interchangeable. Physics: different concepts"\\\\
[THE CONNECTION - Digital Archetype] "Game engines track velocity vectors for every object"}
\end{frame}

\begin{frame}
\frametitle{Attempt: The Student}
\begin{exampleblock}{The Challenge (3 min, silent)}
A student has a displacement of 304 m north in 180 s.

\vspace{0.3cm}

\textbf{Find:} The student's average velocity

\vspace{0.3cm}

\textit{Remember: velocity needs direction! Work silently.}
\end{exampleblock}
\note{[THE CHALLENGE] Can you calculate velocity with direction?\\\\
[SAY] "Try this on your own. Don't forget the direction in your answer."\\\\
[TIMING] 3 min SILENT individual work\\\\
[CIRCULATE] Note who forgets to include direction\\\\
[WATCH FOR] Missing direction, wrong units\\\\
[DON'T HELP] Let them practice vector thinking}
\end{frame}

\begin{frame}
\frametitle{Compare: The Student}
\textbf{Turn and talk (2 min):}

\vspace{0.3cm}

\begin{enumerate}
\item What formula did you use?
\item Did you include direction in your answer?
\item How many significant figures should the answer have?
\end{enumerate}

\vspace{0.5cm}

\pause
\alert{Name wheel:} One pair share your complete answer with direction.
\note{[TIMING] 2-3 min pair discussion\\\\
[CIRCULATE] Listen for understanding of vectors\\\\
[CHECK] Name wheel: call a pair to share\\\\
[EXPECTED APPROACH] v-avg = displacement/time, include "north"\\\\
[COMMON ERROR] Forgetting direction, wrong sig figs}
\end{frame}

\begin{frame}
\frametitle{Reveal: The Student Solution}
\textbf{Self-correct in a different color:}

\vspace{0.3cm}

\textbf{Given:}
\begin{itemize}
\item Displacement = 304 m north
\item Time = 180 s
\end{itemize}

\pause

\textbf{Equation:}
$$\mathbf{v}_{\text{avg}} = \frac{\Delta \mathbf{d}}{\Delta t}$$

\pause

\textbf{Substitute:}
$$\mathbf{v}_{\text{avg}} = \frac{304 \text{ m}}{180 \text{ s}} = 1.7 \text{ m/s north}$$

\pause

\textbf{Note:} 2 sig figs from time (180 s)
\note{[P0] "Self-correct in a different color"\\\\
[P1] [ALGEBRA] "v-average equals delta d over delta t"\\\\
[P2] "304 meters divided by 180 seconds"\\\\
[P3] [ANSWER] "1.7 meters per second north - direction is essential"\\\\
- Time has only 2 sig figs, so answer has 2 sig figs\\\\
[THE WONDER] GPS satellites do this calculation millions of times per second to track your phone}
\end{frame}

\begin{frame}
\frametitle{2.2 Instantaneous Velocity}
\textbf{Average velocity:} Over a time interval

\pause
\vspace{0.3cm}

\textbf{Instantaneous velocity:} At a specific instant in time

\pause
\vspace{0.3cm}

\begin{figure}
\centering
\includegraphics[width=0.7\textwidth,height=0.4\textheight,keepaspectratio]{phys11-kinematics-fig2-9.jpg}
\caption{Airplane passenger velocity at different segments}
\end{figure}

\pause
If velocity is constant, instantaneous = average.
\note{[P0][Fig 2.9: The diagram shows a]  "Two types of velocity, like speed"\\\\
[P1] "Average: over entire time interval"\\\\
[P2] "Instantaneous: at one specific moment"\\\\
- Diagram shows smaller time segments\\\\
[P3] "Special case: constant velocity means instantaneous equals average throughout"\\\\
[THE REVELATION] Calculus finds instantaneous velocity from infinitesimally small intervals}
\end{frame}

\section{Summary}

\begin{frame}
\frametitle{What You Now Know}
\begin{block}{The Revelations}
\begin{enumerate}
\item Motion is relative - depends on reference frame \pause
\item Distance (scalar) vs. displacement (vector) \pause
\item Speed (scalar) vs. velocity (vector) \pause
\item Displacement = final position - initial position \pause
\item Average speed = distance / time \pause
\item Average velocity = displacement / time
\end{enumerate}
\end{block}
\note{[P0] "Six revelations today"\\\\
[P1] "Motion is relative to reference frame"\\\\
[P2] "Distance is path length, displacement is net change"\\\\
[P3] "Speed is how fast, velocity is how fast plus direction"\\\\
[P4] "Delta d equals d-f minus d-naught"\\\\
[P5] "v-avg equals distance over time"\\\\
[P6] "v-avg vector equals displacement over time"\\\\
[THE WONDER] You now speak the language of kinematics\\\\
- Name wheel: which concept was hardest?}
\end{frame}

\begin{frame}[shrink]
\frametitle{Key Equations}
\begin{align}
\Delta \mathbf{d} &= \mathbf{d}_{\mathrm{f}} - \mathbf{d}_{0} \\
v_{\text{avg}} &= \frac{\text{distance}}{\text{time}} \\
\mathbf{v}_{\text{avg}} &= \frac{\Delta \mathbf{d}}{\Delta t} = \frac{\mathbf{d}_{\mathrm{f}} - \mathbf{d}_{0}}{t_{\mathrm{f}} - t_{0}}
\end{align}

\vspace{0.3cm}

\textbf{Remember:}
\begin{itemize}
\item Bold = vector (includes direction)
\item Italic = scalar (magnitude only)
\end{itemize}
\note{- Three foundational equations for kinematics\\\\
- Displacement equation: change in position\\\\
- Speed equation: distance over time\\\\
- Velocity equation: displacement over time\\\\
- Know when to use each\\\\
- Questions before we end?}
\end{frame}

\begin{frame}
\frametitle{Homework}
\begin{center}
\Large
Complete the assigned problems\\[0.3cm]
posted on the LMS
\end{center}
\note{[SAY] "Homework is posted on the LMS"\\\\
[TIMING] Due date: check LMS\\\\
[CHECK] Questions before we end?\\\\
[TRANSITION] Next class: Section 2.3 Position vs Time Graphs\\\\
- We'll visualize motion using graphs\\\\
- Bring graph paper and ruler}
\end{frame}

\end{document}
