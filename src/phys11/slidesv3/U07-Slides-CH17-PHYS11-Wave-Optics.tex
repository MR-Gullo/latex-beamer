\documentclass{beamer}
\usepackage{../../../shared/templates/ds9_theme}
\usepackage[overridenote]{pdfpc}
\graphicspath{{../images/}{../../shared/images/}}

\title[Wave Behavior of Light]{PHYS11 CH:17 When Light Reveals Its Waves}
\subtitle{Diffraction, Interference, and the Hidden Structure of Reality}
\author[Mr. Gullo]{Mr. Gullo}
\date[December 2025]{December 2025}

\begin{document}

\frame{\titlepage
\note{[THE HOOK] Today we discover light's secret identity.\\\\
- Light doesn't just travel in straight lines - it bends, spreads, interferes\\\\
- CD rainbows, laser patterns, telescope limits - all from wave behavior\\\\
[THE WONDER] Same equations explain colors on soap bubbles and resolution of Hubble Space Telescope.\\\\
- By end of class, you'll see light as both particle AND wave}
}

\begin{frame}
\frametitle{Outline}
\tableofcontents
\end{frame}

\section{Introduction}

\begin{frame}
\frametitle{The Mystery of the Rainbow Disc}
\begin{figure}
\centering
\includegraphics[width=0.7\textwidth,height=0.5\textheight,keepaspectratio]{phys11-wave-optics-fig17-1.jpg}
\caption{CD showing rainbow colors from white light}
\end{figure}

\pause
\begin{center}
\Large How does straw-colored plastic produce a rainbow?
\end{center}
\note{[Fig 17.1: CD diffraction producing rainbow] "Physical demo: hold CD to overhead light"\\\\
[P0] "Hold up a CD to light - you see rainbows"\\\\
[P1] [THE HOOK] "How does straw-colored plastic produce a rainbow?"\\\\
- Not pigments - the CD has no dye\\\\
- Answer: wave behavior of light\\\\
[THE WONDER] Direct evidence light is a wave\\\\
[TEACHING HINT] Point to grooves on CD surface - spacing similar to light wavelength causes diffraction}
\end{frame}

\begin{frame}
\frametitle{The Dual Identity}
\textbf{Light behaves as both:}
\begin{itemize}
\item \textbf{Ray:} Travels in straight lines, reflects, refracts \pause
\item \textbf{Wave:} Bends around corners, interferes, diffracts
\end{itemize}

\pause
\vspace{0.3cm}

\begin{alertblock}{What Your Eyes Miss}
You see light's ray behavior every day. Its wave behavior is hidden - until objects become tiny.
\end{alertblock}
\note{[P0] "Light has dual identity"\\\\
[P1] "Ray: straight lines, mirrors, lenses - we covered that last chapter"\\\\
[P2] [THE CONFLICT] "Wave behavior hidden until objects become tiny"\\\\
- When light interacts with objects about same size as wavelength\\\\
- Then wave nature becomes visible\\\\
[THE CONNECTION - Digital Archetype] "Like seeing source code vs compiled program"}
\end{frame}

\section{Understanding Diffraction and Interference}

\begin{frame}
\frametitle{Learning Objectives}
\begin{block}{By the end of this section, you will be able to:}
\begin{itemize}
\item \textbf{17.1:} Explain wave behavior of light, including diffraction and interference \pause
\item \textbf{17.1:} Describe constructive and destructive interference in single-slit and double-slit experiments \pause
\item \textbf{17.1:} Calculate wavelength of light using two-slit interference data
\end{itemize}
\end{block}
\note{[P0] "Three objectives for section 17.1"\\\\
[P1] "First: understand diffraction and interference as wave behaviors"\\\\
[P2] "Second: distinguish constructive from destructive interference"\\\\
[P3] "Third: calculate wavelength from interference patterns"\\\\
- Young's double-slit experiment proved light is a wave}
\end{frame}

\begin{frame}
\frametitle{17.1 The Source Code of Light}
\begin{block}{Nature's Law for Light}
\begin{center}
\Large $\boxed{c = f\lambda}$
\end{center}
Speed equals frequency times wavelength
\end{block}

\pause
\vspace{0.3cm}

\textbf{For visible light in vacuum:}
\begin{itemize}
\item Speed: $c = 3.00 \times 10^8$ m/s (constant) \pause
\item Wavelength: $\lambda = 380$ to 750 nm \pause
\item Frequency: $f = 4.0 \times 10^{14}$ to $7.9 \times 10^{14}$ Hz
\end{itemize}
\note{[P0] [THE REVELATION] "c equals f lambda - fundamental equation for electromagnetic waves"\\\\
[P1] "Speed of light: 3 times 10 to the 8th meters per second"\\\\
[P2] "Visible wavelengths: 380 to 750 nanometers"\\\\
[P3] "Frequency stays constant even when light changes medium"\\\\
[THE WONDER] Same equation governs radio waves, microwaves, X-rays\\\\
[THE CONNECTION - Harmonic Archetype] "Frequency is like pitch - it defines the color"}
\end{frame}

\begin{frame}
\frametitle{17.1 Light as Both Ray and Wave}
\begin{figure}
\centering
\includegraphics[width=0.85\textwidth,height=0.65\textheight,keepaspectratio]{phys11-wave-optics-fig17-2.jpg}
\caption{Laser beam as ray (straight line) and wave (interference pattern after slits)}
\end{figure}
\note{[Fig 17.2: Paranal laser as ray + interference pattern] "Dual nature of light in one image"\\\\
- Left: laser beam from observatory - perfect ray behavior\\\\
- Right: same laser through narrow slits - wave behavior revealed\\\\
- Horizontal spreading into bright and dark regions\\\\
- Systematic constructive and destructive interference\\\\
[THE REVELATION] Interference is signature of wave behavior\\\\
[TEACHING HINT] Ask: "Why do we see ray behavior in daily life but need special setup for wave behavior?" Answer: everyday objects much larger than wavelength}
\end{frame}

\begin{frame}
\frametitle{17.1 Water Waves Show the Way}
\begin{figure}
\centering
\includegraphics[width=0.7\textwidth,height=0.55\textheight,keepaspectratio]{phys11-wave-optics-fig17-3.jpg}
\caption{Water waves passing through gaps in rocks}
\end{figure}

\pause
\textbf{Key observation:} Gap width similar to wavelength causes interference pattern
\note{[Fig 17.3: Water waves through rock gaps] "Use analogy to build mental model"\\\\
[P0] "Water waves easier to see than light waves"\\\\
[P1] [THE CONNECTION - Kinetic Archetype] "Gaps in rocks like slits in Young's experiment"\\\\
- Cross-section shows crests and troughs\\\\
- Same physics governs water, sound, and light\\\\
[THE HUMILITY] Water helps us visualize what happens to light\\\\
[TEACHING HINT] Trace with finger: incoming straight waves become curved after passing through gap - this is diffraction in action}
\end{frame}

\begin{frame}
\frametitle{17.1 Light in Different Media}
\textbf{When light enters a medium:}
\begin{itemize}
\item Speed changes: $v = \frac{c}{n}$ \pause
\item Wavelength changes: $\lambda_n = \frac{\lambda}{n}$ \pause
\item Frequency stays constant \pause
\item Color stays constant (color linked to frequency)
\end{itemize}

\pause
\vspace{0.3cm}

\begin{exampleblock}{Real-World: Light in Water}
Water has $n = 1.333$, so visible wavelengths compress to 285-570 nm
\end{exampleblock}
\note{[P0] "When light enters medium like water"\\\\
[P1] "Speed decreases by factor n"\\\\
[P2] "Wavelength also decreases by factor n"\\\\
[P3] "Frequency stays constant"\\\\
[P4] "Color stays constant because color determined by frequency"\\\\
[P5] "In water: wavelengths shrink but colors don't change"\\\\
[THE WONDER] Light slows down but never stops}
\end{frame}

\begin{frame}
\frametitle{17.1 Huygens's Principle}
\begin{block}{Nature's Rule for Wave Propagation}
Every point on a wavefront is a source of wavelets that spread forward at wave speed. New wavefront is tangent to all wavelets.
\end{block}

\pause
\begin{figure}
\centering
\includegraphics[width=0.6\textwidth,height=0.4\textheight,keepaspectratio]{phys11-wave-optics-fig17-4.jpg}
\caption{Wavefront emitting semicircular wavelets}
\end{figure}
\note{[Fig 17.4: Huygens wavelets propagating straight wavefront] "Geometric construction reveals wave mechanics"\\\\
[P0] [THE REVELATION] "Huygens's principle - 17th century Dutch scientist"\\\\
[P1] "Every point on wavefront creates new semicircular wavelet"\\\\
- Draw tangent to wavelets - that's new wavefront position\\\\
- Works for all waves: water, sound, light\\\\
[THE WONDER] Simple geometric principle explains complex wave behavior\\\\
[TEACHING HINT] Point to semicircles: "Each is like a mini-wave source. The tangent line shows where crests align - that's the new wavefront"}
\end{frame}

\begin{frame}
\frametitle{17.1 The Bending of Light}
\begin{exampleblock}{The Mental Model}
Sound bends around doorways. Light seems to travel straight. Why?
\end{exampleblock}

\pause
\vspace{0.3cm}

\textbf{Answer:} Wavelength compared to opening size
\begin{itemize}
\item Sound wavelength: $\sim$1 m (comparable to door width) \pause
\item Visible light wavelength: $\sim$500 nm (much smaller than door)
\end{itemize}

\pause
\begin{alertblock}{The Paradox}
Light DOES bend - but only around objects comparable to its wavelength
\end{alertblock}
\note{[P0] "Sound bends around corners - you hear it everywhere"\\\\
[P1] [THE CONFLICT] "Light seems to go straight - sharp shadows"\\\\
[P2] "Sound wavelength about 1 meter - comparable to door"\\\\
[P3] "Light wavelength 500 nanometers - 2 million times smaller"\\\\
[P4] "Light bends too, just need tiny openings to see it"\\\\
[THE REVELATION] Wavelength determines what counts as obstacle}
\end{frame}

\begin{frame}
\frametitle{17.1 Diffraction Revealed}
\begin{block}{Nature's Definition}
\textbf{Diffraction:} Bending of wave around edges of opening or obstacle
\end{block}

\pause
\begin{figure}
\centering
\includegraphics[width=0.7\textwidth,height=0.5\textheight,keepaspectratio]{phys11-wave-optics-fig17-5.jpg}
\caption{Huygens's principle applied to slit - edges bend}
\end{figure}
\note{[Fig 17.5: Huygens's principle to opening, edges bend]\\\\
[P0] [THE REVELATION] "Diffraction: wave characteristic that occurs for all waves"\\\\
[P1] "Apply Huygens's principle to opening"\\\\
- Edges of wavefront bend after passing through\\\\
- Smaller opening, more extreme bending\\\\
[THE WONDER] If diffraction observed, phenomenon is produced by waves}
\end{frame}

\begin{frame}
\frametitle{17.1 Ocean Waves Through Reef}
\begin{figure}
\centering
\includegraphics[width=0.8\textwidth,height=0.65\textheight,keepaspectratio]{phys11-wave-optics-fig17-6.jpg}
\caption{Ocean waves diffracting through opening - visible interference pattern}
\end{figure}
\note{[Fig 17.6: Ocean waves through reef opening]\\\\
- Incoming waves at top pass through gaps\\\\
- Foreground shows interference pattern\\\\
- Opening similar in width to wavelength\\\\
- Same principle as light through slits\\\\
[THE CONNECTION - Kinetic Archetype] "Surfers know waves bend around jetties"}
\end{frame}

\begin{frame}
\frametitle{17.1 Young's Revolutionary Experiment}
\begin{figure}
\centering
\includegraphics[width=0.7\textwidth,height=0.5\textheight,keepaspectratio]{phys11-wave-optics-fig17-7.jpg}
\caption{Double-slit experiment setup (1801)}
\end{figure}

\pause
\textbf{Result:} Vertical light and dark lines spread horizontally
\note{[Fig 17.7: Young's double-slit experiment]\\\\
[P0] "Thomas Young, 1801 - proved light is wave"\\\\
[P1] "Single-wavelength light through two narrow slits"\\\\
- Without diffraction: would see two bright lines\\\\
- With diffraction: see many bright and dark bands\\\\
- This pattern can ONLY be explained if light is wave\\\\
[THE HUMILITY] People resisted because it contradicted Newton}
\end{frame}

\begin{frame}
\frametitle{17.1 The Interference Pattern}
\begin{figure}
\centering
\includegraphics[width=0.85\textwidth,height=0.6\textheight,keepaspectratio]{phys11-wave-optics-fig17-8.jpg}
\caption{Double-slit interference: light diffracts from each slit, waves overlap and interfere}
\end{figure}

\pause
\begin{itemize}
\item \textbf{Constructive interference:} Crest meets crest $\rightarrow$ bright
\item \textbf{Destructive interference:} Crest meets trough $\rightarrow$ dark
\end{itemize}
\note{[Fig 17.8: Double slits with overlapping circular wavefronts] "Critical diagram showing interference mechanism"\\\\
[P0] "Light spreads out from each slit"\\\\
[P1] "Waves overlap and interfere"\\\\
- Constructive: crest meets crest - bright regions\\\\
- Destructive: crest meets trough - dark regions\\\\
- Pattern depends on wavelength and slit separation\\\\
[THE WONDER] Light interfering with itself - wave signature\\\\
[TEACHING HINT] Point to overlapping circles: "Where solid meets solid = bright. Where solid meets dashed = dark. This proves light is a wave"}
\end{frame}

\begin{frame}
\frametitle{17.1 The Math of Double-Slit Interference}
\begin{block}{Universal Law: Constructive Interference}
\begin{center}
\Large $\boxed{d \sin \theta = m\lambda}$
\end{center}
For $m = 0, \pm 1, \pm 2, \pm 3, \ldots$ (order of maximum)
\end{block}

\pause
\vspace{0.3cm}

\begin{block}{Universal Law: Destructive Interference}
\begin{center}
\Large $\boxed{d \sin \theta = \left(m + \frac{1}{2}\right)\lambda}$
\end{center}
For $m = 0, \pm 1, \pm 2, \ldots$ (order of minimum)
\end{block}
\note{[P0] [THE REVELATION] "d is distance between slits, theta is angle to bright/dark band"\\\\
[P1] "Constructive: path difference equals whole wavelengths"\\\\
- m equals 0: central maximum\\\\
- m equals 1: first-order maximum\\\\
[P2] "Destructive: path difference equals half-wavelengths"\\\\
[THE WONDER] Geometry of interference encoded in sine function}
\end{frame}

\begin{frame}
\frametitle{17.1 Path Difference Geometry}
\begin{figure}
\centering
\includegraphics[width=0.7\textwidth,height=0.55\textheight,keepaspectratio]{phys11-wave-optics-fig17-9.jpg}
\caption{Path difference $\Delta L = d \sin \theta$}
\end{figure}

\pause
\textbf{Key insight:} Waves start in phase, end in or out of phase depending on path difference
\note{[Fig 17.9: Path difference geometry with right triangle] "Geometric derivation of interference equation"\\\\
[P0] "Two waves travel from slits to screen"\\\\
[P1] "Different path lengths"\\\\
- If paths differ by whole wavelength: constructive\\\\
- If paths differ by half wavelength: destructive\\\\
- Trigonometry gives delta-L equals d sine theta\\\\
[THE CONNECTION - Digital Archetype] "Like packets arriving at router - timing matters"\\\\
[TEACHING HINT] Trace triangle: "Extra distance ΔL determines if waves arrive in phase. Right triangle makes this calculable with trig"}
\end{frame}

\begin{frame}
\frametitle{17.1 Intensity Pattern}
\begin{figure}
\centering
\includegraphics[width=0.7\textwidth,height=0.55\textheight,keepaspectratio]{phys11-wave-optics-fig17-10.jpg}
\caption{Intensity decreases with angle from center}
\end{figure}

\pause
\textbf{Observation:} Central maximum brightest, intensity falls off to sides
\note{[Fig 17.10: Interference pattern intensity vs angle]\\\\
[P0] "Graph shows intensity vs angle"\\\\
[P1] "Central maximum brightest"\\\\
- Higher-order maxima dimmer\\\\
- Intensity falls with increasing angle\\\\
- Photo shows actual pattern: bright and dark fringes\\\\
[THE WONDER] Beautiful mathematical regularity in nature}
\end{frame}

\begin{frame}
\frametitle{17.1 Single-Slit Diffraction}
\begin{figure}
\centering
\includegraphics[width=0.8\textwidth,height=0.6\textheight,keepaspectratio]{phys11-wave-optics-fig17-11.jpg}
\caption{Single slit produces wider central maximum with dimmer side maxima}
\end{figure}

\pause
\textbf{Key difference:} Central maximum is 6 times wider than side maxima
\note{[Fig 17.11: Single-slit pattern showing wide central maximum] "Distinguish single vs double slit patterns"\\\\
[P0] "Single slit different from double slit"\\\\
[P1] "Central maximum much wider and brighter"\\\\
- Intensity decreases rapidly on sides\\\\
- Minima occur at D sine theta equals m lambda\\\\
- D is slit width, not separation\\\\
[THE CONFLICT] Single slit still produces pattern - diffraction at edges\\\\
[TEACHING HINT] Compare to Fig 17.10: "Single slit has ONE wide bright band. Double slit has MANY equally-spaced bands. Don't confuse them!"}
\end{frame}

\begin{frame}
\frametitle{17.1 Single-Slit Geometry}
\begin{figure}
\centering
\includegraphics[width=0.7\textwidth,height=0.5\textheight,keepaspectratio]{phys11-wave-optics-fig17-12.jpg}
\caption{Ray diagram showing destructive interference for single slit}
\end{figure}

\pause
\begin{block}{Universal Law: Single-Slit Minima}
\begin{center}
$\boxed{D \sin \theta = m\lambda}$ \quad or \quad $\boxed{\frac{Dy}{L} = m\lambda}$
\end{center}
For $m = \pm 1, \pm 2, \pm 3, \ldots$ (not zero)
\end{block}
\note{[Fig 17.12: Single-slit diffraction equations]\\\\
[P0] "Ray from center travels lambda over 2 farther than ray from top"\\\\
[P1] "Arrive out of phase - destructive interference"\\\\
- Every ray from top half paired with ray from bottom half\\\\
- Cancel in pairs at minima\\\\
- D is slit width, y is distance on screen, L is screen distance\\\\
[THE REVELATION] Width of slit determines diffraction pattern}
\end{frame}

\begin{frame}
\frametitle{Attempt: Decoding the Double Slit}
\begin{exampleblock}{The Challenge (3 min, silent)}
Light from a He-Ne laser passes through two slits separated by 0.0100 mm. The third bright line forms at angle $10.95^\circ$ relative to incident beam.

\vspace{0.3cm}

\textbf{Given:}
\begin{itemize}
\item $d = 0.0100$ mm $= 1.00 \times 10^{-5}$ m
\item $\theta = 10.95^\circ$
\item $m = 3$ (third bright line)
\end{itemize}

\textbf{Find:} Wavelength $\lambda$ in nm

\vspace{0.3cm}

\textit{Can you decode the wavelength? Work silently.}
\end{exampleblock}
\note{[THE CHALLENGE] Can they extract wavelength from interference pattern?\\\\
[SAY] "Try this on your own. It's okay to get stuck."\\\\
[TIMING] 3-4 min SILENT individual work\\\\
[CIRCULATE] Note who finishes early, who struggles with unit conversion\\\\
[WATCH FOR] Using wrong equation, forgetting to convert mm to m\\\\
[DON'T HELP] Let them struggle - learning happens in Compare}
\end{frame}

\begin{frame}
\frametitle{Compare: Double-Slit Strategy}
\textbf{Turn and talk (2 min):}

\vspace{0.3cm}

\begin{enumerate}
\item Which equation did you choose for constructive interference?
\item How did you rearrange it to solve for $\lambda$?
\item What units did you get for wavelength?
\end{enumerate}

\vspace{0.5cm}

\pause
\alert{Name wheel:} One pair share your approach (not your answer).
\note{[TIMING] 2-3 min pair discussion\\\\
[CIRCULATE] Listen for common approaches\\\\
[CHECK] Name wheel: call a pair to share approach\\\\
[EXPECTED APPROACH] Use d sine theta equals m lambda, solve for lambda\\\\
[COMMON ERROR] Forgetting to convert mm to meters, using destructive formula}
\end{frame}

\begin{frame}
\frametitle{Reveal: The Wavelength of Light}
\textbf{Self-correct in a different color:}

\vspace{0.3cm}

\textbf{Equation:} $d \sin \theta = m\lambda$

\pause
\vspace{0.2cm}

\textbf{Rearrange:} $\lambda = \frac{d \sin \theta}{m}$

\pause
\vspace{0.2cm}

\textbf{Substitute:} $\lambda = \frac{(1.00 \times 10^{-5} \text{ m})(\sin 10.95^\circ)}{3}$

\pause
\vspace{0.2cm}

$$\lambda = \frac{(1.00 \times 10^{-5})(0.190)}{3} = 6.33 \times 10^{-7} \text{ m}$$

\pause
$$\boxed{\lambda = 633 \text{ nm}}$$

\pause
\textbf{Check:} 633 nm is red light - wavelength of He-Ne laser. Perfect!
\note{[P0] "Self-correct in different color"\\\\
[P1] [ALGEBRA] "Start with d sine theta equals m lambda"\\\\
[P2] "Solve for lambda: divide both sides by m"\\\\
[P3] "Substitute: d equals 1 times 10 to negative 5, sine 10.95 equals 0.190"\\\\
[P4] "Calculate: 6.33 times 10 to negative 7 meters"\\\\
[P5] [ANSWER] "633 nanometers - red light from He-Ne laser"\\\\
[P6] "Not coincidence - interference patterns measure wavelength"\\\\
[THE WONDER] Young used this to measure visible wavelengths - analytical technique still used today}
\end{frame}

\begin{frame}
\frametitle{Attempt: Single-Slit Width}
\begin{exampleblock}{The Challenge (3 min, silent)}
Visible light of wavelength 550 nm falls on single slit and produces second diffraction minimum at angle $45.0^\circ$.

\vspace{0.3cm}

\textbf{Given:}
\begin{itemize}
\item $\lambda = 550$ nm $= 550 \times 10^{-9}$ m
\item $\theta = 45.0^\circ$
\item $m = 2$ (second minimum)
\end{itemize}

\textbf{Find:} Slit width $D$ in micrometers

\vspace{0.3cm}

\textit{Can you decode the slit width? Work silently.}
\end{exampleblock}
\note{[THE CHALLENGE] Can they reverse-engineer slit width?\\\\
[SAY] "Single slit this time - different equation"\\\\
[TIMING] 3-4 min SILENT work\\\\
[CIRCULATE] Note who uses double-slit equation by mistake\\\\
[WATCH FOR] Using constructive instead of destructive formula\\\\
[DON'T HELP] Productive struggle builds understanding}
\end{frame}

\begin{frame}
\frametitle{Compare: Single-Slit Strategy}
\textbf{Turn and talk (2 min):}

\vspace{0.3cm}

\begin{enumerate}
\item What's the difference between single-slit and double-slit equations?
\item How did you solve for $D$?
\item What units did you get?
\end{enumerate}

\vspace{0.5cm}

\pause
\alert{Name wheel:} One pair share your approach.
\note{[TIMING] 2-3 min pair discussion\\\\
[CIRCULATE] Listen for understanding of single vs double slit\\\\
[CHECK] Name wheel: call a pair\\\\
[EXPECTED APPROACH] Use D sine theta equals m lambda, solve for D\\\\
[COMMON ERROR] Using d instead of D, using constructive interference formula}
\end{frame}

\begin{frame}
\frametitle{Reveal: The Narrow Slit}
\textbf{Self-correct in a different color:}

\vspace{0.3cm}

\textbf{Equation:} $D \sin \theta = m\lambda$

\pause
\vspace{0.2cm}

\textbf{Rearrange:} $D = \frac{m\lambda}{\sin \theta}$

\pause
\vspace{0.2cm}

\textbf{Substitute:} $D = \frac{2(550 \times 10^{-9} \text{ m})}{\sin 45.0^\circ}$

\pause
\vspace{0.2cm}

$$D = \frac{1100 \times 10^{-9}}{0.707} = 1.56 \times 10^{-6} \text{ m}$$

\pause
$$\boxed{D = 1.56 \text{ }\mu\text{m}}$$

\pause
\textbf{Check:} Only few times wavelength - consistent with significant wave effects!
\note{[P0] "Self-correct in different color"\\\\
[P1] [ALGEBRA] "D sine theta equals m lambda"\\\\
[P2] "Solve for D: divide by sine theta"\\\\
[P3] "Substitute: m equals 2, lambda equals 550 nanometers, sine 45 equals 0.707"\\\\
[P4] "Calculate: 1.56 times 10 to negative 6 meters"\\\\
[P5] [ANSWER] "1.56 micrometers - incredibly narrow"\\\\
[P6] "Slit only few times wavelength - that's why wave effects visible"\\\\
[THE WONDER] Light must interact with objects comparable to wavelength to show wave nature}
\end{frame}

\section{Applications of Diffraction, Interference, and Coherence}

\begin{frame}
\frametitle{Learning Objectives}
\begin{block}{By the end of this section, you will be able to:}
\begin{itemize}
\item \textbf{17.2:} Explain wave behaviors including diffraction, interference, and coherence \pause
\item \textbf{17.2:} Describe applications based on wave properties of light \pause
\item \textbf{17.2:} Perform calculations for diffraction gratings and resolution limits
\end{itemize}
\end{block}
\note{[P0] "Three objectives for section 17.2"\\\\
[P1] "First: understand coherence - in-phase light"\\\\
[P2] "Second: see real-world applications - lasers, spectroscopes"\\\\
[P3] "Third: calculate diffraction grating patterns and resolution"\\\\
- Applications connect theory to technology}
\end{frame}

\begin{frame}
\frametitle{17.2 The Birth of the Laser}
\textbf{Einstein's idea (1917):}
\begin{itemize}
\item Photon hits excited atom \pause
\item Atom emits second photon with same energy \pause
\item Two photons in phase = \textbf{coherent light} \pause
\item Chain reaction: stream of in-phase photons
\end{itemize}

\pause
\vspace{0.3cm}

\begin{block}{The Acronym}
\textbf{L}ight \textbf{A}mplification by \textbf{S}timulated \textbf{E}mission of \textbf{R}adiation
\end{block}
\note{[P0] "1917: Einstein thinking about photons and excited atoms"\\\\
[P1] "Photon with right energy hits excited atom"\\\\
[P2] "Atom emits matching photon - same energy, same phase"\\\\
[P3] "Two photons go on to hit more atoms - chain reaction"\\\\
[P4] "Result: coherent light - all waves in sync"\\\\
[P5] "Took 40 years to build first laser in 1960"\\\\
[THE WONDER] Einstein predicted lasers before technology existed to build them}
\end{frame}

\begin{frame}
\frametitle{17.2 Laser Properties and Uses}
\textbf{Properties:}
\begin{itemize}
\item Directional (doesn't spread much) \pause
\item Very intense \pause
\item Narrow (about 0.5 mm diameter) \pause
\item Monochromatic (one wavelength)
\end{itemize}

\pause
\vspace{0.3cm}

\textbf{Applications:}
\begin{itemize}
\item Read CDs and DVDs \pause
\item Cut steel in industry \pause
\item Eye surgery (minimal bleeding) \pause
\item Measure Earth-Moon distance \pause
\item Create holograms
\end{itemize}
\note{[P0] "Laser beams directional - stay narrow over long distances"\\\\
[P1] "Very intense - concentrated energy"\\\\
[P2] "Narrow - about half millimeter"\\\\
[P3] "Monochromatic - one pure color"\\\\
[P4] "Applications countless"\\\\
[P5] "CDs: interpret reflections from surface pits"\\\\
[P6] "Industry: cut through steel"\\\\
[P7] "Surgery: cauterizes as it cuts - little bleeding"\\\\
[P8] "Moon: bounce off reflectors astronauts left"\\\\
[P9] "Holograms: 3D images from interference"\\\\
[THE CONNECTION - Digital Archetype] "Your phone has multiple lasers"}
\end{frame}

\begin{frame}
\frametitle{17.2 Diffraction Gratings}
\begin{block}{Nature's Definition}
\textbf{Diffraction grating:} Large number of evenly-spaced parallel slits
\end{block}

\pause
\begin{figure}
\centering
\includegraphics[width=0.75\textwidth,height=0.5\textheight,keepaspectratio]{phys11-wave-optics-fig17-13.jpg}
\caption{Light through grating produces sharper pattern than double slit}
\end{figure}
\note{[Fig 17.13: Diffraction grating with parallel slits]\\\\
[P0] [THE REVELATION] "Diffraction grating: many slits instead of two"\\\\
[P1] "Pattern similar to double slit but sharper"\\\\
- Bright regions narrower and brighter\\\\
- Dark regions darker\\\\
- White light dispersed into rainbow\\\\
- Central maximum white, higher orders show spectrum\\\\
[THE WONDER] CDs, DVDs, butterfly wings, opals all act as gratings}
\end{frame}

\begin{frame}
\frametitle{17.2 Natural Diffraction Gratings}
\begin{figure}
\centering
\includegraphics[width=0.8\textwidth,height=0.65\textheight,keepaspectratio]{phys11-wave-optics-fig17-14.jpg}
\caption{Australian opal and butterfly wings - natural reflection gratings}
\end{figure}
\note{[Fig 17.14: Opal and butterfly - natural reflection gratings] "Nature as physics laboratory"\\\\
- Opal: rows of reflectors act like grating\\\\
- Butterfly: fingerlike structures in regular patterns\\\\
- Produce iridescence - colors from interference, not pigments\\\\
- Same physics as CD rainbows\\\\
[THE WONDER] Nature invented diffraction gratings long before humans\\\\
[TEACHING HINT] "These colors change with viewing angle - evidence of interference, not pigment. Crush the wing scales and color disappears"}
\end{frame}

\begin{frame}
\frametitle{17.2 Grating vs Double Slit}
\begin{figure}
\centering
\includegraphics[width=0.8\textwidth,height=0.6\textheight,keepaspectratio]{phys11-wave-optics-fig17-15.jpg}
\caption{Intensity comparison: double slit (a) vs grating (b)}
\end{figure}

\pause
\textbf{Key difference:} More slits = narrower, brighter maxima
\note{[Fig 17.15: Intensity graphs double slit vs grating]\\\\
[P0] "Compare intensity graphs"\\\\
[P1] "Grating produces much sharper pattern"\\\\
- Maxima at same angles\\\\
- But grating maxima narrower and brighter\\\\
- Darker regions between\\\\
- More slits, better definition\\\\
[THE REVELATION] Sharpness makes gratings useful for spectroscopy}
\end{frame}

\begin{frame}
\frametitle{17.2 The CD as Diffraction Grating}
\begin{figure}
\centering
\includegraphics[width=0.6\textwidth,height=0.45\textheight,keepaspectratio]{phys11-wave-optics-fig17-16.jpg}
\caption{CD holds data in spiral groove with 1,600 grooves per mm}
\end{figure}

\pause
\vspace{0.3cm}

\textbf{How it works:}
\begin{itemize}
\item Grooves act as reflection grating \pause
\item Laser tracks along spiral \pause
\item Pits encode binary data (0s and 1s) \pause
\item Reflected beam goes to photodiode detector
\end{itemize}
\note{[Fig 17.16: CD data storage and grooves]\\\\
[P0] "CD grooves 1,600 per millimeter - similar to visible wavelengths"\\\\
[P1] "That's why you see rainbows on CDs"\\\\
[P2] "One continuous spiral groove from center out"\\\\
[P3] "Data in pits - binary code"\\\\
[P4] "Laser tracks groove, reads pits"\\\\
[P5] "Thin aluminum coating makes pits reflective"\\\\
[THE CONNECTION - Digital Archetype] "Physics enables digital storage"}
\end{frame}

\begin{frame}
\frametitle{17.2 Spectroscopes}
\begin{figure}
\centering
\includegraphics[width=0.75\textwidth,height=0.55\textheight,keepaspectratio]{phys11-wave-optics-fig17-18.jpg}
\caption{Diffraction grating separates light into component wavelengths}
\end{figure}

\pause
\textbf{Uses:}
\begin{itemize}
\item Identify chemical elements by spectrum
\item Measure wavelengths of light from stars
\item Analyze laser output
\end{itemize}
\note{[Fig 17.18: Spectroscope with diffraction grating]\\\\
[P0] "Spectroscope uses grating to separate wavelengths"\\\\
[P1] "Applications numerous"\\\\
- Each element emits unique spectrum when heated\\\\
- Astronomers identify star composition from light\\\\
- Chemists identify unknown substances\\\\
- Works for wavelengths beyond visible\\\\
[THE WONDER] Light carries information about what emitted it}
\end{frame}

\begin{frame}
\frametitle{17.2 The Resolution Limit}
\begin{block}{Nature's Constraint}
Diffraction limits detail we can observe in images
\end{block}

\pause
\begin{figure}
\centering
\includegraphics[width=0.7\textwidth,height=0.45\textheight,keepaspectratio]{phys11-wave-optics-fig17-19.jpg}
\caption{Light through circular aperture produces fuzzy spot with rings}
\end{figure}

\pause
\begin{alertblock}{The Paradox}
Even perfect lens produces fuzzy images due to wave nature of light
\end{alertblock}
\note{[Fig 17.19: Airy disk - circular aperture diffraction] "Fundamental limit of optical systems"\\\\
[P0] [THE REVELATION] "Diffraction unavoidable consequence of wave nature"\\\\
[P1] "Light through small aperture spreads"\\\\
[P2] [THE CONFLICT] "Perfect optics can't overcome wave physics"\\\\
- Central bright spot with fuzzy edge\\\\
- Surrounded by rings\\\\
- Smaller aperture, more spreading\\\\
[THE HUMILITY] Nature sets fundamental limits on vision\\\\
[TEACHING HINT] "This is what stars ACTUALLY look like through telescope - not points but fuzzy disks. Physics, not engineering, sets the limit"}
\end{frame}

\begin{frame}
\frametitle{17.2 The Rayleigh Criterion}
\begin{block}{Universal Law: Resolution Limit}
Two images just resolvable when center of one diffraction pattern falls on first minimum of other
\end{block}

\pause
\begin{figure}
\centering
\includegraphics[width=0.7\textwidth,height=0.4\textheight,keepaspectratio]{phys11-wave-optics-fig17-20.jpg}
\caption{Rayleigh criterion for just-resolvable point sources}
\end{figure}

\pause
$$\boxed{\theta = 1.22 \frac{\lambda}{D}}$$
where $\theta$ is minimum resolvable angle (in radians), $D$ is aperture diameter
\note{[Fig 17.20: Rayleigh criterion - overlapping Airy disks] "Quantifying resolution limit"\\\\
[P0] [THE REVELATION] "Lord Rayleigh, 19th century - criterion for resolution"\\\\
[P1] "Graph shows intensity of diffraction pattern"\\\\
[P2] "Equation: theta equals 1.22 lambda over D"\\\\
- Smaller wavelength: better resolution\\\\
- Larger aperture: better resolution\\\\
- Eye pupil about 3 mm diameter\\\\
[THE WONDER] Limits vision of all optical instruments - telescopes, microscopes, eyes\\\\
[TEACHING HINT] Point to middle panel: "Two sources just barely distinguishable. Closer = blurred together. This defines resolution limit"}
\end{frame}

\begin{frame}
\frametitle{17.2 Real-World Limits}
\textbf{Diffraction limits:}
\begin{itemize}
\item \textbf{Human eye:} Pupil diameter limits acuity \pause
\item \textbf{Telescopes:} Mirror diameter limits detail \pause
\item \textbf{Microscopes:} Wavelength limits smallest visible object \pause
\item \textbf{Cameras:} Lens diameter affects sharpness
\end{itemize}

\pause
\vspace{0.3cm}

\begin{exampleblock}{The Trade-off}
Larger aperture = better resolution but heavier, more expensive
\end{exampleblock}
\note{[P0] "Every optical system limited by diffraction"\\\\
[P1] "Eye: pupil about 3 mm - sets acuity limit"\\\\
[P2] "Telescopes: why bigger mirrors see more detail"\\\\
[P3] "Microscopes: can't see smaller than wavelength"\\\\
[P4] "Cameras: why professional lenses so large"\\\\
[P5] "Engineers must balance resolution, size, cost"\\\\
[THE CONNECTION - Digital Archetype] "Like pixel limits on display"}
\end{frame}

\begin{frame}
\frametitle{Attempt: Wavelength in Water}
\begin{exampleblock}{The Challenge (3 min, silent)}
A monochromatic laser beam of green light with wavelength 550 nm in air enters water. Refractive index of water is 1.33.

\vspace{0.3cm}

\textbf{Given:}
\begin{itemize}
\item $\lambda = 550$ nm (in vacuum/air)
\item $n = 1.33$ (water)
\end{itemize}

\textbf{Find:} Wavelength $\lambda_n$ in water

\vspace{0.3cm}

\textit{Can you predict the wavelength shift? Work silently.}
\end{exampleblock}
\note{[THE CHALLENGE] Can they calculate wavelength change in medium?\\\\
[SAY] "What happens to light entering water?"\\\\
[TIMING] 3-4 min SILENT work\\\\
[CIRCULATE] Note who remembers wavelength changes but frequency doesn't\\\\
[WATCH FOR] Multiplying by n instead of dividing\\\\
[DON'T HELP] Simple equation but easy to invert}
\end{frame}

\begin{frame}
\frametitle{Compare: Medium Strategy}
\textbf{Turn and talk (2 min):}

\vspace{0.3cm}

\begin{enumerate}
\item What happens to speed, wavelength, and frequency when light enters water?
\item Which equation relates wavelength in medium to wavelength in vacuum?
\item Does wavelength increase or decrease in water?
\end{enumerate}

\vspace{0.5cm}

\pause
\alert{Name wheel:} One pair share your reasoning.
\note{[TIMING] 2-3 min pair discussion\\\\
[CIRCULATE] Listen for understanding of medium effects\\\\
[CHECK] Name wheel: call a pair\\\\
[EXPECTED APPROACH] Use lambda-n equals lambda over n\\\\
[COMMON ERROR] Using lambda-n equals n lambda (backwards)}
\end{frame}

\begin{frame}
\frametitle{Reveal: Light Slows and Compresses}
\textbf{Self-correct in a different color:}

\vspace{0.3cm}

\textbf{Equation:} $\lambda_n = \frac{\lambda}{n}$

\pause
\vspace{0.2cm}

\textbf{Substitute:} $\lambda_n = \frac{550 \text{ nm}}{1.33}$

\pause
\vspace{0.2cm}

$$\boxed{\lambda_n = 414 \text{ nm}}$$

\pause
\vspace{0.3cm}

\textbf{Check:} Wavelength decreased (550 $\rightarrow$ 414 nm). Color stays green because frequency constant!
\note{[P0] "Self-correct in different color"\\\\
[P1] [ALGEBRA] "Lambda-n equals lambda over n"\\\\
[P2] "Substitute: 550 divided by 1.33"\\\\
[P3] [ANSWER] "414 nanometers"\\\\
[P4] "Wavelength compressed by factor 1.33"\\\\
- Speed also decreased by same factor\\\\
- Frequency unchanged\\\\
- Color unchanged because color determined by frequency\\\\
[THE WONDER] Light slows down but never stops - even in diamond}
\end{frame}

\begin{frame}
\frametitle{Attempt: Diffraction Grating Angle}
\begin{exampleblock}{The Challenge (3 min, silent)}
A diffraction grating has 2,000 lines per centimeter. Green light with wavelength 520 nm passes through.

\vspace{0.3cm}

\textbf{Given:}
\begin{itemize}
\item 2,000 lines/cm $\rightarrow$ $d = \frac{1 \text{ cm}}{2000} = 5.00 \times 10^{-4}$ cm
\item $\lambda = 520$ nm $= 520 \times 10^{-9}$ m
\item $m = 1$ (first-order maximum)
\end{itemize}

\textbf{Find:} Angle $\theta$ for first-order maximum

\vspace{0.3cm}

\textit{Can you decode the angle? Work silently.}
\end{exampleblock}
\note{[THE CHALLENGE] Can they find angle from grating spacing?\\\\
[SAY] "Grating problem - first calculate d"\\\\
[TIMING] 3-4 min SILENT work\\\\
[CIRCULATE] Note who struggles with unit conversions\\\\
[WATCH FOR] Forgetting to convert cm to m, using wrong m value\\\\
[DON'T HELP] Multi-step problem builds confidence}
\end{frame}

\begin{frame}
\frametitle{Compare: Grating Strategy}
\textbf{Turn and talk (2 min):}

\vspace{0.3cm}

\begin{enumerate}
\item How did you calculate $d$ from lines per cm?
\item Which equation relates $d$, $\theta$, and $\lambda$ for grating?
\item How did you solve for $\theta$?
\end{enumerate}

\vspace{0.5cm}

\pause
\alert{Name wheel:} One pair share your approach.
\note{[TIMING] 2-3 min pair discussion\\\\
[CIRCULATE] Listen for calculation of d\\\\
[CHECK] Name wheel: call a pair\\\\
[EXPECTED APPROACH] d equals 1 over 2000 cm, convert to m, use d sine theta equals m lambda\\\\
[COMMON ERROR] Unit mismatch, forgetting arcsine}
\end{frame}

\begin{frame}
\frametitle{Reveal: The Grating Disperses Light}
\textbf{Self-correct in a different color:}

\vspace{0.3cm}

\textbf{Find d:} $d = \frac{1 \text{ cm}}{2000} = 5.00 \times 10^{-4}$ cm $= 5.00 \times 10^{-6}$ m

\pause
\vspace{0.2cm}

\textbf{Equation:} $d \sin \theta = m\lambda$

\pause
\vspace{0.2cm}

\textbf{Rearrange:} $\theta = \sin^{-1}\left(\frac{m\lambda}{d}\right)$

\pause
\vspace{0.2cm}

\textbf{Substitute:} $\theta = \sin^{-1}\left(\frac{(1)(520 \times 10^{-9})}{5.00 \times 10^{-6}}\right) = \sin^{-1}(0.104)$

\pause
$$\boxed{\theta = 5.97^\circ}$$

\pause
\textbf{Check:} Small angle - reasonable for first maximum!
\note{[P0] "Self-correct in different color"\\\\
[P1] "First find d: 1 over 2000 equals 5 times 10 to negative 6 meters"\\\\
[P2] [ALGEBRA] "d sine theta equals m lambda"\\\\
[P3] "Solve for theta: take arcsine of m lambda over d"\\\\
[P4] "Substitute: m equals 1, lambda equals 520 nanometers"\\\\
[P5] [ANSWER] "5.97 degrees"\\\\
[P6] "Small angle for first maximum - makes sense"\\\\
[THE WONDER] Gratings precisely control where each wavelength goes}
\end{frame}

\begin{frame}
\frametitle{Attempt: Laser Beam Spread}
\begin{exampleblock}{The Challenge (3 min, silent)}
A He-Ne laser beam (633 nm wavelength) is originally 1.00 mm in diameter.

\vspace{0.3cm}

\textbf{Given:}
\begin{itemize}
\item $\lambda = 633$ nm $= 633 \times 10^{-9}$ m
\item $D = 1.00$ mm $= 1.00 \times 10^{-3}$ m
\end{itemize}

\textbf{Find:} Minimum angular spread $\theta$ in radians and degrees

\vspace{0.3cm}

\textit{Can you predict the spreading? Work silently.}
\end{exampleblock}
\note{[THE CHALLENGE] Can they calculate diffraction spreading?\\\\
[SAY] "Even laser beams spread due to diffraction"\\\\
[TIMING] 3-4 min SILENT work\\\\
[CIRCULATE] Note who uses Rayleigh criterion\\\\
[WATCH FOR] Forgetting factor 1.22, unit errors\\\\
[DON'T HELP] Resolution equation applies to all apertures}
\end{frame}

\begin{frame}
\frametitle{Compare: Beam Spread Strategy}
\textbf{Turn and talk (2 min):}

\vspace{0.3cm}

\begin{enumerate}
\item Which equation gives minimum angular spread?
\item What does the diameter $D$ represent?
\item How did you convert radians to degrees?
\end{enumerate}

\vspace{0.5cm}

\pause
\alert{Name wheel:} One pair share your approach.
\note{[TIMING] 2-3 min pair discussion\\\\
[CIRCULATE] Listen for Rayleigh criterion understanding\\\\
[CHECK] Name wheel: call a pair\\\\
[EXPECTED APPROACH] Use theta equals 1.22 lambda over D\\\\
[COMMON ERROR] Forgetting 1.22 factor, wrong units}
\end{frame}

\begin{frame}
\frametitle{Reveal: Even Lasers Spread}
\textbf{Self-correct in a different color:}

\vspace{0.3cm}

\textbf{Equation:} $\theta = \frac{1.22\lambda}{D}$

\pause
\vspace{0.2cm}

\textbf{Substitute:} $\theta = \frac{(1.22)(633 \times 10^{-9} \text{ m})}{1.00 \times 10^{-3} \text{ m}}$

\pause
\vspace{0.2cm}

$$\theta = 7.72 \times 10^{-4} \text{ rad}$$

\pause
\vspace{0.2cm}

\textbf{Convert:} $\theta = (7.72 \times 10^{-4})(57.3^\circ/\text{rad}) = \boxed{0.0442^\circ}$

\pause
\textbf{Check:} Tiny spread - barely noticeable over short distances!
\note{[P0] "Self-correct in different color"\\\\
[P1] [ALGEBRA] "Theta equals 1.22 lambda over D"\\\\
[P2] "Substitute: 1.22 times 633 nanometers over 1 millimeter"\\\\
[P3] [ANSWER] "7.72 times 10 to negative 4 radians"\\\\
[P4] "Convert: multiply by 57.3 degrees per radian"\\\\
[P5] "0.0442 degrees - very small spread"\\\\
[THE WONDER] Even laser beams obey diffraction - wave nature inescapable}
\end{frame}

\section{Summary}

\begin{frame}[squeeze]
\frametitle{What You Now Know}
\begin{block}{The Revelations}
\begin{enumerate}
\item Light exhibits wave behavior: diffraction and interference \pause
\item Huygens's principle: every point on wavefront creates wavelets \pause
\item Double-slit interference proves wave nature: $d \sin \theta = m\lambda$ \pause
\item Single slit produces diffraction pattern: $D \sin \theta = m\lambda$ \pause
\item Lasers produce coherent light via stimulated emission \pause
\item Diffraction gratings separate wavelengths sharply \pause
\item Resolution fundamentally limited by wave nature: $\theta = 1.22\lambda/D$ \pause
\item Wavelength changes in media: $\lambda_n = \lambda/n$, but frequency constant
\end{enumerate}
\end{block}
\note{[P0] "Eight revelations today"\\\\
[P1] "Light exhibits wave behavior"\\\\
[P2] "Huygens's principle explains wave propagation"\\\\
[P3] "Double-slit: d sine theta equals m lambda"\\\\
[P4] "Single-slit: D sine theta equals m lambda"\\\\
[P5] "Lasers: coherent, directional, intense"\\\\
[P6] "Gratings: sharp spectral separation"\\\\
[P7] "Resolution: theta equals 1.22 lambda over D"\\\\
[P8] "Wavelength changes in media, frequency doesn't"\\\\
[THE WONDER] Wave nature hidden in everyday life, revealed in careful experiments\\\\
- Name wheel: which revelation most surprising?}
\end{frame}

\begin{frame}
\frametitle{Key Equations}
\begin{align}
c &= f\lambda \quad \text{(light in vacuum)} \\
\lambda_n &= \frac{\lambda}{n} \quad \text{(wavelength in medium)} \\
d \sin \theta &= m\lambda \quad \text{(double-slit constructive)} \\
d \sin \theta &= \left(m + \frac{1}{2}\right)\lambda \quad \text{(double-slit destructive)} \\
D \sin \theta &= m\lambda \quad \text{(single-slit minima)} \\
\theta &= 1.22\frac{\lambda}{D} \quad \text{(Rayleigh criterion)}
\end{align}
\note{- Six fundamental equations\\\\
- c equals f lambda: speed-frequency-wavelength relationship\\\\
- lambda-n: wavelength in medium\\\\
- d sine theta: double-slit interference\\\\
- D sine theta: single-slit diffraction\\\\
- Rayleigh criterion: resolution limit\\\\
- Know when to use each equation\\\\
- Questions before we end?}
\end{frame}

\begin{frame}
\frametitle{Homework}
\begin{center}
\Large
Complete the assigned problems\\[0.3cm]
posted on the LMS
\end{center}
\note{[SAY] "Homework posted on LMS"\\\\
[TIMING] Due date: check LMS\\\\
[CHECK] Questions before we end?\\\\
[TRANSITION] Next unit: Modern Physics\\\\
- Today you joined 200-year journey understanding light\\\\
- Young proved wave nature in 1801\\\\
- Einstein explained lasers in 1917\\\\
- You now understand both}
\end{frame}

\end{document}
