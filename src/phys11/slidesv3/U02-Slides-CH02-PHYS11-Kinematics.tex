\documentclass{beamer}
\usepackage{../../../shared/templates/ds9_theme}
\usepackage[overridenote]{pdfpc}
\graphicspath{{../images/}{../../shared/images/}}

\title[Reading Motion]{PHYS11 CH:2 Reading the Story of Motion}
\subtitle{Position and Velocity Graphs}
\author[Mr. Gullo]{Mr. Gullo}
\date[December 2025]{December 2025}

\begin{document}

\frame{\titlepage
\note{[THE HOOK] Today we learn to read the language of motion.\\\\
- Graphs are physics's storytelling tool\\\\
- Two chapters: position graphs and velocity graphs\\\\
[THE WONDER] By end of class, you'll decode motion like reading a book\\\\
- Same tools NASA uses to track spacecraft}
}

\begin{frame}
\frametitle{Outline}
\tableofcontents
\end{frame}

\section{Introduction}

\begin{frame}
\frametitle{The Mystery}
\begin{center}
\Large How can a single line\\
\textit{tell the complete story of motion?}
\end{center}

\pause
\vspace{0.5cm}
Position, velocity, acceleration—all encoded in curves and slopes.

\pause
\vspace{0.3cm}
\alert{A graph is worth a thousand equations.}
\note{[P0] "How can a single line tell the complete story of motion?"\\\\
[P1] "Position, velocity, acceleration - all encoded in curves and slopes"\\\\
[P2] [THE WONDER] "A graph is worth a thousand equations. Today you learn to read them"\\\\
[THE CONNECTION - Digital Archetype] Like reading computer code - every pixel has meaning}
\end{frame}

\section{Position vs. Time Graphs}

\begin{frame}
\frametitle{Learning Objectives}
\begin{block}{By the end of this lesson, you will be able to:}
\begin{itemize}
\item \textbf{2.3:} Explain the meaning of slope in position vs. time graphs \pause
\item \textbf{2.3:} Solve problems using position vs. time graphs
\end{itemize}
\end{block}
\note{[P0] "Two objectives for position graphs"\\\\
[P1] "First: understand what slope means. Second: solve real problems"\\\\
- Assessment: practice problems and quiz}
\end{frame}

\begin{frame}
\frametitle{2.3 The Language of Graphs}
\begin{exampleblock}{The Mental Model}
A graph is like a picture—worth a thousand words. It reveals relationships between physical quantities.
\end{exampleblock}

\pause
\vspace{0.3cm}

\textbf{Standard graph anatomy:}
\begin{itemize}
\item Horizontal axis = independent variable (usually time) \pause
\item Vertical axis = dependent variable (position, velocity) \pause
\item Straight line: $y = mx + b$ where $m$ = slope, $b$ = y-intercept
\end{itemize}
\note{[P0] [THE REVELATION] "Graphs are the visual language of physics"\\\\
[P1] "Horizontal axis is independent - usually time"\\\\
[P2] "Vertical axis is dependent - position, velocity"\\\\
[P3] "Straight line: y equals m x plus b"\\\\
- m is slope, b is y-intercept\\\\
- We'll give these physical meaning}
\end{frame}

\begin{frame}
\frametitle{2.3 Drive to School}
\begin{center}
\includegraphics[width=0.7\textwidth,height=0.5\textheight,keepaspectratio]{phys11-kinematics-fig2-11.jpg}

\small Graph of position vs. time for 5 km drive to school
\end{center}

\pause
\textbf{What does this line tell us?}
\begin{itemize}
\item Starts at home ($d_0 = 0$)
\item Ends at school ($d_f = 5$ km)
\item Takes 10 minutes
\end{itemize}
\note{[P0] "Position versus time for drive to school"\\\\
[P1] "Starts at home - zero position. Ends at school - 5 km. Takes 10 minutes"\\\\
- Straight line means constant velocity\\\\
- Name wheel: what is the slope?}
\end{frame}

\begin{frame}
\frametitle{2.3 Reading the Slope}
\begin{block}{Universal Law: Slope is Velocity}
In a position vs. time graph:
$$\text{slope} = \frac{\text{rise}}{\text{run}} = \frac{\Delta d}{\Delta t} = v_{\text{avg}}$$
\end{block}

\pause
\vspace{0.3cm}

\textbf{For the drive to school:}
$$v_{\text{avg}} = \frac{5 \text{ km}}{10 \text{ min}} = 0.5 \text{ km/min}$$

\pause
\begin{exampleblock}{The Anchor}
Steeper slope = faster motion. Flat line = at rest.
\end{exampleblock}
\note{[P0] [THE REVELATION] "Slope is velocity - rise over run"\\\\
[P1] "Drive to school: 5 km divided by 10 min equals 0.5 km per min"\\\\
[P2] [THE CONNECTION - Kinetic Archetype] "Steeper slope means faster motion. Athletes know this instinctively"\\\\
[THE WONDER] Slope connects position to velocity}
\end{frame}

\begin{frame}
\frametitle{2.3 Round Trip}
\begin{center}
\includegraphics[width=0.7\textwidth,height=0.5\textheight,keepaspectratio]{phys11-kinematics-fig2-15.jpg}

\small What does the graph look like with the return trip?
\end{center}

\pause
\textbf{Second leg:}
\begin{itemize}
\item Negative slope = moving backward
\item Returns to $d = 0$ (back home)
\item Net displacement = 0 km
\end{itemize}
\note{[P0] "Add the return trip to the graph"\\\\
[P1] "Second leg has negative slope - moving backward. Returns to zero - back home"\\\\
- Total distance: 10 km\\\\
- Net displacement: 0 km\\\\
- Direction matters!}
\end{frame}

\begin{frame}
\frametitle{2.3 Jet Car on Salt Flats}
\begin{center}
\includegraphics[width=0.7\textwidth,height=0.5\textheight,keepaspectratio]{phys11-kinematics-fig2-4.jpg}

\small Position vs. time for jet-powered car
\end{center}

\pause
\textbf{Reading the graph:}
\begin{itemize}
\item At $t = 0$ s: $d = 400$ m
\item At $t = 1$ s: $d = 650$ m
\item Slope = velocity = $250$ m/s
\end{itemize}
\note{[P0][Fig 2.4: Your total change in]  "Jet car on Bonneville Salt Flats"\\\\
[P1] "At t equals 0, position is 400 m. At t equals 1 s, position is 650 m"\\\\
- Calculate slope: 250 m divided by 1 s\\\\
- Velocity is 250 m/s\\\\
- That's 900 km/h! Incredibly fast}
\end{frame}

\begin{frame}
\frametitle{2.3 The Position Equation}
\begin{block}{Universal Law: Linear Motion}
From the graph equation $y = mx + b$, we get:
$$\boxed{d = v t + d_0}$$
or equivalently
$$\boxed{d = d_0 + vt}$$
\end{block}

\pause
\vspace{0.3cm}

Where:
\begin{itemize}
\item $m$ (slope) = velocity $v$
\item $b$ (y-intercept) = initial position $d_0$
\end{itemize}
\note{[P0] [THE REVELATION] "From graph anatomy, we derive motion equation"\\\\
[P1] "Slope m becomes velocity v. Y-intercept b becomes initial position"\\\\
- d equals d-zero plus v t\\\\
- This is foundational kinematics\\\\
[THE WONDER] Graph and equation are two languages for same truth}
\end{frame}

\begin{frame}
\frametitle{2.3 Curved Position Graphs}
\begin{center}
\includegraphics[width=0.6\textwidth,height=0.45\textheight,keepaspectratio]{phys11-kinematics-fig2-5.jpg}

\small Jet car speeding up - curved graph
\end{center}

\pause
\begin{alertblock}{The Conflict}
When the graph curves, velocity is changing. Slope is not constant!
\end{alertblock}

\pause
\textbf{Solution:} Use tangent line to find instantaneous velocity at any point.
\note{[P0] [Fig 2.5: A short line separates] "When object speeds up, graph curves"\\\\
[P1] [THE CONFLICT] "Slope is not constant - velocity is changing"\\\\
[P2] "Draw tangent line at any point - its slope is instantaneous velocity"\\\\
- Tangent touches curve at only one point\\\\
- Steeper curve means faster acceleration}
\end{frame}

\begin{frame}
\frametitle{2.3 Instantaneous Velocity from Tangent}
\begin{center}
\includegraphics[width=0.7\textwidth,height=0.5\textheight,keepaspectratio]{phys11-kinematics-fig2-6.jpg}

\small Slope of tangent line = instantaneous velocity
\end{center}

\pause
\textbf{At point Q ($t = 25$ s):}
$$v_Q = \frac{3120 - 1300 \text{ m}}{32 - 19 \text{ s}} = \frac{1820 \text{ m}}{13 \text{ s}} = 140 \text{ m/s}$$
\note{[P0] [Fig 2.6: The total distance that] "Two tangent lines shown at different points"\\\\
[P1] "At point Q, t equals 25 s. Calculate slope of tangent"\\\\
- Rise: 1820 m. Run: 13 s\\\\
- Instantaneous velocity: 140 m/s\\\\
- This is how we find velocity at any instant}
\end{frame}

\begin{frame}
\frametitle{Attempt: Reading a Position Graph}
\begin{exampleblock}{The Challenge (3 min, silent)}
\begin{center}
\includegraphics[width=0.6\textwidth,height=0.35\textheight,keepaspectratio]{phys11-kinematics-fig2-7.jpg}
\end{center}

\textbf{Given:} The graph above

\textbf{Find:} Average velocity over entire time interval (0 to 16 s)

\vspace{0.2cm}
\textit{Can you decode this motion? Work silently.}
\end{exampleblock}
\note{[THE CHALLENGE] Can they extract velocity from a graph?\\\\
[SAY] "Try this on your own. It's okay to get stuck."\\\\
[TIMING] 3-4 min SILENT individual work\\\\
[CIRCULATE] Note who identifies initial and final positions\\\\
[WATCH FOR] Students trying to use every point instead of endpoints\\\\
[DON'T HELP] Let them struggle - learning happens in Compare}
\end{frame}

\begin{frame}
\frametitle{Compare: Graph Reading Strategy}
\textbf{Turn and talk (2 min):}

\vspace{0.3cm}

\begin{enumerate}
\item What two points did you choose?
\item How did you calculate the slope?
\item What units did you get?
\end{enumerate}

\vspace{0.5cm}

\pause
\alert{Name wheel:} One pair share your approach (not your answer).
\note{[TIMING] 2-3 min pair discussion\\\\
[CIRCULATE] Listen for common approaches\\\\
[CHECK] Name wheel: call a pair to share approach\\\\
[EXPECTED APPROACH] Use endpoints - initial and final position\\\\
[COMMON ERROR] Adding up all segments instead of using net displacement}
\end{frame}

\begin{frame}
\frametitle{Reveal: Slope Equals Velocity}
\textbf{Self-correct in a different color:}

\vspace{0.3cm}

\textbf{Step 1:} Identify endpoints: $(t_0, d_0) = (0, 0)$ m and $(t_f, d_f) = (16, 5)$ m

\pause
\vspace{0.2cm}

\textbf{Step 2:} Calculate slope
$$v_{\text{avg}} = \frac{\Delta d}{\Delta t} = \frac{d_f - d_0}{t_f - t_0}$$

\pause
\vspace{0.2cm}

\textbf{Step 3:} Substitute
$$v_{\text{avg}} = \frac{5 - 0 \text{ m}}{16 - 0 \text{ s}} = \boxed{0.31 \text{ m/s}}$$

\pause
\textbf{Check:} About 1 km/h - a slow walk. Reasonable!
\note{[P0] [Fig 2.7: The Mars Climate Orbiter] "Self-correct in a different color"\\\\
[P1] [ALGEBRA] "Identify endpoints - zero and 16 s, zero and 5 m"\\\\
[P2] "Velocity equals change in d over change in t"\\\\
[P3] "5 m divided by 16 s equals 0.31 m/s"\\\\
[P4] [ANSWER] "0.31 m/s - about 1 km/h, a slow walk"\\\\
[THE WONDER] Slope unlocks velocity - graph becomes data}
\end{frame}

\section{Velocity vs. Time Graphs}

\begin{frame}
\frametitle{Learning Objectives}
\begin{block}{By the end of this lesson, you will be able to:}
\begin{itemize}
\item \textbf{2.4:} Explain the meaning of slope and area in velocity vs. time graphs \pause
\item \textbf{2.4:} Solve problems using velocity vs. time graphs
\end{itemize}
\end{block}
\note{[P0] "Two objectives for velocity graphs"\\\\
[P1] "First: understand slope AND area. Second: solve problems"\\\\
- Velocity graphs are more powerful\\\\
- Can extract both displacement and acceleration}
\end{frame}

\begin{frame}
\frametitle{2.4 From Position to Velocity Graph}
\begin{columns}[T]
\column{0.48\textwidth}
\begin{center}
\includegraphics[width=\linewidth,height=0.45\textheight,keepaspectratio]{phys11-kinematics-fig2-8.jpg}

\small Position graph: drive to and from school
\end{center}

\pause
\column{0.48\textwidth}
\begin{center}
\includegraphics[width=\linewidth,height=0.45\textheight,keepaspectratio]{phys11-kinematics-fig2-9.jpg}

\small Velocity graph: two constant velocities
\end{center}
\end{columns}

\vspace{0.3cm}
\pause
\textbf{Key insight:} Slope of position graph becomes height of velocity graph!
\note{[P0][Fig 2.9: The diagram shows a]  [Fig 2.8: During a 30 -minute] "Position graph on left - V-shaped"\\\\
[P1] "Velocity graph on right - two horizontal lines"\\\\
[P2] "Slope of position graph becomes height of velocity graph"\\\\
- Forward: +0.5 km/min\\\\
- Backward: -0.5 km/min\\\\
- Constant velocity means horizontal line}
\end{frame}

\begin{frame}
\frametitle{2.4 Reading Velocity Graphs}
\begin{block}{Universal Law: The Dual Nature}
In a velocity vs. time graph:
\begin{enumerate}
\item \textbf{Slope} = acceleration (rate of velocity change)
\item \textbf{Area under curve} = displacement
\end{enumerate}
\end{block}

\pause
\vspace{0.3cm}

\begin{exampleblock}{The Anchor}
Position graphs give velocity. Velocity graphs give acceleration AND displacement.
\end{exampleblock}
\note{[P0] [THE REVELATION] "Velocity graphs have two powers"\\\\
[P1] "Slope tells acceleration. Area tells displacement"\\\\
[P2] [THE CONNECTION - Digital Archetype] "Like function returns two values"\\\\
[THE WONDER] More information packed into one graph}
\end{frame}

\begin{frame}
\frametitle{2.4 Area Equals Displacement}
\begin{center}
\includegraphics[width=0.7\textwidth,height=0.5\textheight,keepaspectratio]{phys11-kinematics-fig2-10.jpg}

\small Velocity graph for drive to school
\end{center}

\pause
\textbf{Calculate displacement:}
$$d = v \times t = 0.5 \text{ km/min} \times 10 \text{ min} = 5 \text{ km}$$

\pause
\textbf{Return trip:}
$$d = (-0.5 \text{ km/min}) \times 10 \text{ min} = -5 \text{ km}$$

\pause
\textbf{Net:} $5 + (-5) = 0$ km. Back where you started!
\note{[P0] [Fig 2.10: The diagram shows a] "Calculate area under velocity curve"\\\\
[P1] "Forward leg: 0.5 km/min times 10 min equals 5 km"\\\\
[P2] "Return leg: negative 0.5 times 10 equals negative 5 km"\\\\
[P3] [THE WONDER] "Add them: zero displacement. Math confirms you're home"\\\\
- Units cancel: km/min times min equals km}
\end{frame}

\begin{frame}
\frametitle{2.4 The Velocity Equation}
\begin{block}{Universal Law: Velocity with Acceleration}
From the velocity graph equation $y = mx + b$, we get:
$$\boxed{v = v_0 + at}$$
\end{block}

\pause
\vspace{0.3cm}

Where:
\begin{itemize}
\item $m$ (slope) = acceleration $a$
\item $b$ (y-intercept) = initial velocity $v_0$
\end{itemize}

\pause
\vspace{0.3cm}

\textbf{And from area:}
$$\boxed{d = vt} \quad \text{(for constant velocity)}$$
\note{[P0] [THE REVELATION] "From graph anatomy, we derive velocity equation"\\\\
[P1] "Slope m becomes acceleration a. Y-intercept b becomes initial velocity"\\\\
[P2] "Area under curve gives displacement"\\\\
- v equals v-zero plus a t\\\\
- Connects velocity to acceleration\\\\
[THE WONDER] Same graph pattern, different physical meaning}
\end{frame}

\begin{frame}
\frametitle{2.4 Jet Car Velocity Graph}
\begin{center}
\includegraphics[width=0.7\textwidth,height=0.5\textheight,keepaspectratio]{phys11-kinematics-fig2-6.jpg}

\small Jet car speeding up - straight line with positive slope
\end{center}

\pause
\textbf{What we can read:}
\begin{itemize}
\item Starts at $v_0 = 20$ m/s at $t = 0$
\item Ends at $v_f = 160$ m/s at $t = 30$ s
\item Slope = acceleration (constant)
\end{itemize}
\note{[P0][Fig 2.6: The total distance that]  "Straight line means constant acceleration"\\\\
[P1] "Starts at 20 m/s, ends at 160 m/s over 30 s"\\\\
- Positive slope means speeding up\\\\
- Can calculate acceleration from slope\\\\
- Can calculate displacement from area}
\end{frame}

\begin{frame}
\frametitle{2.4 Zero Slope Means Constant Velocity}
\begin{center}
\includegraphics[width=0.7\textwidth,height=0.5\textheight,keepaspectratio]{phys11-kinematics-fig2-12.jpg}

\small Horizontal line in velocity graph
\end{center}

\pause
\begin{alertblock}{Key Insight}
Slope = 0 means acceleration = 0. Object moves at constant velocity.
\end{alertblock}

\pause
This is what we saw in the drive to school example!
\note{[P0][Fig 2.12: The diagram shows a]  [Fig 2.6: The total distance that] "Horizontal line in velocity graph"\\\\
[P1] "Slope is zero, so acceleration is zero"\\\\
[P2] "Constant velocity - no speeding up or slowing down"\\\\
- Position graph would be straight diagonal line\\\\
- This is uniform motion}
\end{frame}

\begin{frame}
\frametitle{Attempt: Calculating from Velocity Graph}
\begin{exampleblock}{The Challenge (3 min, silent)}
\begin{center}
\includegraphics[width=0.5\textwidth,height=0.3\textheight,keepaspectratio]{phys11-kinematics-fig2-13.jpg}
\end{center}

\textbf{Given:} Velocity graph above (jet car from 0 to 30 s)

\textbf{Find:}
\begin{itemize}
\item (a) Displacement
\item (b) Acceleration
\end{itemize}

\vspace{0.2cm}
\textit{Use both slope and area. Work silently.}
\end{exampleblock}
\note{[THE CHALLENGE] Can they extract two quantities from one graph?\\\\
[SAY] "Find displacement AND acceleration. Use what you learned."\\\\
[TIMING] 3-4 min SILENT individual work\\\\
[CIRCULATE] Note who recognizes area vs. slope\\\\
[WATCH FOR] Confusing which is which\\\\
[DON'T HELP] Productive struggle builds understanding}
\end{frame}

\begin{frame}
\frametitle{Compare: Dual Extraction}
\textbf{Turn and talk (2 min):}

\vspace{0.3cm}

\begin{enumerate}
\item How did you find displacement? (Hint: area)
\item How did you find acceleration? (Hint: slope)
\item Did you break the area into shapes?
\end{enumerate}

\vspace{0.5cm}

\pause
\alert{Name wheel:} One pair share your approach (not your answer).
\note{[TIMING] 2-3 min pair discussion\\\\
[CIRCULATE] Listen for strategy\\\\
[CHECK] Name wheel: call a pair to share approach\\\\
[EXPECTED APPROACH] Area = rectangle + triangle. Slope = rise/run\\\\
[COMMON ERROR] Using slope for displacement or area for acceleration}
\end{frame}

\begin{frame}
\frametitle{Reveal: Area and Slope}
\textbf{Self-correct in a different color:}

\vspace{0.2cm}

\textbf{(a) Displacement = Area under curve}

\pause
Rectangle: $20 \text{ m/s} \times 30 \text{ s} = 600$ m

\pause
Triangle: $\frac{1}{2} \times 30 \text{ s} \times 140 \text{ m/s} = 2100$ m

\pause
Total: $\boxed{d = 2700 \text{ m}}$

\pause
\vspace{0.2cm}

\textbf{(b) Acceleration = Slope}
\pause
$$a = \frac{\Delta v}{\Delta t} = \frac{140 \text{ m/s}}{30 \text{ s}} = \boxed{4.67 \text{ m/s}^2}$$
\note{[P0] [Fig 2.13: The diagram shows a] "Part a: displacement equals area"\\\\
[P1] "Rectangle: 20 m/s times 30 s equals 600 m"\\\\
[P2] "Triangle: half times base times height equals 2100 m"\\\\
[P3] [ANSWER] "Total displacement: 2700 m"\\\\
[P4] "Part b: acceleration equals slope"\\\\
[P5] "Change in v divided by change in t equals 4.67 m/s squared"\\\\
[THE WONDER] One graph, two insights - displacement and acceleration}
\end{frame}

\begin{frame}
\frametitle{2.4 Curved Velocity Graphs}
\begin{center}
\includegraphics[width=0.7\textwidth,height=0.5\textheight,keepaspectratio]{phys11-kinematics-fig2-14.jpg}

\small More realistic jet car - curved velocity graph
\end{center}

\pause
\begin{alertblock}{The Complication}
When velocity graph curves, acceleration is changing! Use tangent for instantaneous acceleration.
\end{alertblock}
\note{[P0] [Fig 2.14: A U.S. Air Force] "More realistic graph - velocity doesn't jump instantly"\\\\
[P1] [THE CONFLICT] "Curved means acceleration is changing"\\\\
- Draw tangent line for instantaneous acceleration\\\\
- Estimate area by breaking into sections\\\\
- Real motion is usually curved, not straight}
\end{frame}

\begin{frame}
\frametitle{2.4 Negative Velocity}
\begin{center}
\includegraphics[width=0.7\textwidth,height=0.5\textheight,keepaspectratio]{phys11-kinematics-fig2-15.jpg}

\small Velocity graph going below zero
\end{center}

\pause
\textbf{Interpretation:}
\begin{itemize}
\item Positive velocity = moving forward
\item Negative velocity = moving backward
\item Zero crossing = turning point (changes direction)
\end{itemize}
\note{[P0] [Fig 2.15: A graph of position] "Velocity can be positive or negative"\\\\
[P1] "Positive means forward, negative means backward"\\\\
- Zero crossing is turning point\\\\
- Area below axis is negative displacement\\\\
- Net displacement = positive area minus negative area}
\end{frame}

\begin{frame}
\frametitle{2.4 Position from Velocity Graph}
\begin{exampleblock}{The Connection}
\textbf{From position graph:} slope $\rightarrow$ velocity

\textbf{From velocity graph:} area $\rightarrow$ displacement
\end{exampleblock}

\pause
\vspace{0.3cm}

\begin{block}{Circular Relationship}
\begin{center}
Position $\xrightarrow{\text{slope}}$ Velocity $\xrightarrow{\text{slope}}$ Acceleration

\vspace{0.2cm}

Position $\xleftarrow{\text{area}}$ Velocity $\xleftarrow{\text{?}}$ Acceleration
\end{center}
\end{block}

\pause
We'll learn about acceleration graphs in the next chapter!
\note{[P0] [THE REVELATION] "Graphs connect position, velocity, acceleration"\\\\
[P1] "Slope gives rate of change. Area gives accumulation"\\\\
[P2] "Next chapter: acceleration graphs complete the picture"\\\\
[THE WONDER] Three quantities, linked by calculus - slope and area are inverse operations}
\end{frame}

\section{Summary}

\begin{frame}
\frametitle{What You Now Know}
\begin{block}{The Revelations}
\begin{enumerate}
\item Graphs are the visual language of motion \pause
\item Position graph: slope = velocity \pause
\item Velocity graph: slope = acceleration, area = displacement \pause
\item Tangent lines extract instantaneous values \pause
\item Negative slopes/areas show direction \pause
\item One graph encodes multiple quantities
\end{enumerate}
\end{block}
\note{[P0] "Six revelations today"\\\\
[P1] "Graphs are visual language of motion"\\\\
[P2] "Position graph slope is velocity"\\\\
[P3] "Velocity graph: slope is acceleration, area is displacement"\\\\
[P4] "Tangent lines extract instantaneous values"\\\\
[P5] "Negative shows direction"\\\\
[P6] "One graph, multiple insights"\\\\
[THE WONDER] You can now read motion like a book\\\\
- Name wheel: what surprised you most?}
\end{frame}

\begin{frame}[shrink]
\frametitle{Key Equations}
\begin{align}
\text{Position graph slope} &= \frac{\Delta d}{\Delta t} = v_{\text{avg}} \\
\text{Position equation} &= d = d_0 + vt \\
\text{Velocity graph slope} &= \frac{\Delta v}{\Delta t} = a \\
\text{Velocity equation} &= v = v_0 + at \\
\text{Displacement from velocity} &= \text{area under } v\text{-}t \text{ curve}
\end{align}
\note{- Five foundational equations from graphs\\\\
- First two: position graphs\\\\
- Last three: velocity graphs\\\\
- Know when to use each\\\\
- Questions before we end?}
\end{frame}

\begin{frame}
\frametitle{Homework}
\begin{center}
\Large
Complete the assigned problems\\[0.3cm]
posted on the LMS
\end{center}
\note{[SAY] "Homework is posted on the LMS"\\\\
[TIMING] Due date: check LMS\\\\
[CHECK] Questions before we end?\\\\
[TRANSITION] Next class: Chapter 2 sections 2.5-2.6 - Acceleration and equations of motion}
\end{frame}

\end{document}
