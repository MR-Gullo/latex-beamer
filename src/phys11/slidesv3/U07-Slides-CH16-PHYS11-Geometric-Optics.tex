\documentclass{beamer}
\usepackage{../../../shared/templates/ds9_theme}
\usepackage{../../../shared/templates/semantic-physics-colors}
\usepackage[overridenote]{pdfpc}
\graphicspath{{../images/}{../../shared/images/}}

\title[Geometric Optics]{PHYS11 CH:16 How Light Bends Reality}
\subtitle{Mirrors, Refraction, and Lenses}
\author[Mr. Gullo]{Mr. Gullo}
\date[December 2025]{December 2025}

\begin{document}

\frame{\titlepage
\note{[THE HOOK] Today we decode how light creates illusions.\\\\
- Mirrors that make you appear behind walls\\\\
- Pencils that bend in water\\\\
- Lenses that focus sunlight to burn paper\\\\
[THE WONDER] Same physics that corrects your vision explains telescopes and microscopes.\\\\
- Three sections: Reflection, Refraction, Lenses}
}

\begin{frame}
\frametitle{Outline}
\tableofcontents
\end{frame}

\section{Introduction}

\begin{frame}
\frametitle{The Illusion}
\begin{center}
\Large What if you could see yourself\\
\textit{standing behind a solid wall?}
\end{center}

\pause
\vspace{0.5cm}
Mirrors create images where nothing exists. Your brain is fooled.

\pause
\vspace{0.3cm}
\alert{Yet cameras capture the same illusion. This is geometric optics.}
\note{[P0] "What if you could see yourself standing behind a solid wall?"\\\\
[P1] "Mirrors create images where nothing exists - your brain is fooled"\\\\
[P2] [THE REVELATION] "Yet cameras capture the same illusion - this is real physics"\\\\
[THE CONNECTION - Digital Archetype] "Same ray tracing math used in video game engines"}
\end{frame}

\begin{frame}
\frametitle{Alice Through the Looking Glass}
\begin{figure}
\centering
\includegraphics[width=0.7\textwidth,height=0.5\textheight,keepaspectratio]{phys11-geometric-optics-fig16-1.jpg}
\end{figure}

\pause
In Lewis Carroll's story, Alice steps through a mirror into a virtual world.

\pause
Today we explore the optical meaning of real versus virtual.
\note{[P0] [Fig 16.1: Mirror reflection] "Alice Through the Looking Glass - stepping through mirrors into virtual worlds"\\\\
[P1] "In Lewis Carroll's story, Alice enters a virtual world"\\\\
[P2] "Today we explore real versus virtual in physics"\\\\
[THE WONDER] What your eyes see, your brain interprets - physics explains both}
\end{frame}

\begin{frame}
\frametitle{Geometric Optics}
\begin{block}{Definition: Geometric Optics}
When light interacts with objects much larger than its wavelength, it behaves like rays traveling in straight lines.
\end{block}

\pause
\vspace{0.3cm}

\textbf{Light as rays:}
\begin{itemize}
\item Travels in straight lines through a medium \pause
\item Changes direction at boundaries \pause
\item Predictable using geometry and trigonometry
\end{itemize}
\note{[P0] "Geometric optics: light behaves like rays"\\\\
[P1] "Light travels in straight lines through a medium"\\\\
[P2] "Changes direction at boundaries between materials"\\\\
[P3] "Predictable using geometry - angles, triangles, trig functions"\\\\
[THE CONNECTION - Kinetic Archetype] "Like billiard balls bouncing off walls - angles predict everything"}
\end{frame}

\section{Reflection}

\begin{frame}
\frametitle{Learning Objectives}
\begin{block}{By the end of this section, you will be able to:}
\begin{itemize}
\item \textbf{16.1:} Explain reflection from mirrors and describe image formation \pause
\item \textbf{16.1:} Apply ray diagrams to predict image locations \pause
\item \textbf{16.1:} Perform calculations using the law of reflection and curved mirror equations
\end{itemize}
\end{block}
\note{[P0] "Three objectives for reflection"\\\\
[P1] "First: explain reflection and image formation"\\\\
[P2] "Second: use ray diagrams to predict where images appear"\\\\
[P3] "Third: calculate image properties using equations"\\\\
- Real-world applications: security mirrors, telescopes, makeup mirrors}
\end{frame}

\begin{frame}
\frametitle{16.1 Three Paths for Light}
\begin{figure}
\centering
\includegraphics[width=0.8\textwidth,height=0.6\textheight,keepaspectratio]{phys11-geometric-optics-fig16-2.jpg}
\end{figure}

\pause
Light can travel: (a) through empty space, (b) through media, or (c) by reflection.
\note{[P0] [Fig 16.2: Light travel paths] "Three ways light travels from source to destination"\\\\
[P1] "Through empty space like sunlight to Earth, through media like air and glass, or by reflection like mirrors"\\\\
- All three cases: light modeled as traveling in straight lines\\\\
- Changes direction at boundaries between materials}
\end{frame}

\begin{frame}
\frametitle{16.1 The Law of Reflection}
\begin{figure}
\centering
\includegraphics[width=0.7\textwidth,height=0.5\textheight,keepaspectratio]{phys11-geometric-optics-fig16-3.jpg}
\end{figure}

\pause
\begin{block}{Universal Law: The Mirror's Rule}
\begin{center}
\Large $\boxed{\angle{\theta_r} = \angle{\theta_i}}$
\end{center}
\angle{Angle} of reflection equals \angle{angle} of incidence.
\end{block}
\note{[P0] [Fig 16.3: Law of reflection] "The law of reflection - governs all wave behavior at smooth surfaces"\\\\
[P1] [THE REVELATION] "Theta-r equals theta-i - angle of reflection equals angle of incidence"\\\\
- Angles measured from the normal line - perpendicular to surface\\\\
[THE WONDER] Same law for sound waves, water waves, light waves - universal}
\end{frame}

\begin{frame}
\frametitle{16.1 Smooth vs Rough Surfaces}
\begin{figure}
\centering
\includegraphics[width=0.7\textwidth,height=0.5\textheight,keepaspectratio]{phys11-geometric-optics-fig16-4.jpg}
\end{figure}

\pause
\textbf{Smooth surface:} Specular reflection - rays reflect at same angle

\pause
\textbf{Rough surface:} Diffuse reflection - rays scatter in many directions
\note{[P0] [Fig 16.4: Diffuse reflection] "Two types of reflection"\\\\
[P1] "Smooth surface like mirror: specular reflection - all rays reflect at same angle"\\\\
[P2] "Rough surface like paper: diffuse reflection - rays scatter"\\\\
[THE CONNECTION - Kinetic Archetype] "Why you can read paper from any angle - light scatters to reach your eyes"}
\end{frame}

\begin{frame}
\frametitle{16.1 Virtual Images in Plane Mirrors}
\begin{figure}
\centering
\includegraphics[width=0.7\textwidth,height=0.5\textheight,keepaspectratio]{phys11-geometric-optics-fig16-5.jpg}
\end{figure}

\pause
\begin{block}{Definition: Virtual Image}
An image formed when light rays \textit{appear} to diverge from a point without actually doing so.
\end{block}
\note{[P0] [Fig 16.5: Virtual image] "How flat mirrors create images"\\\\
[P1] "Virtual image: rays appear to diverge from behind mirror"\\\\
- Image distance d-i equals object distance d-o\\\\
- Image is same size as object\\\\
[THE CONFLICT] Image appears behind solid wall - impossible, yet cameras capture it}
\end{frame}

\begin{frame}
\frametitle{16.1 Curved Mirrors}
\begin{exampleblock}{The Mental Model}
\textbf{Concave:} Caves inward (like a spoon bowl)

\textbf{Convex:} Curves outward (like a spoon back)
\end{exampleblock}

\pause
\vspace{0.3cm}

\textbf{Focal point (F):} Where parallel rays converge or appear to converge

\textbf{Focal length (\disp{f}):} \disp{Distance} from mirror to focal point
\note{[P0] "Two types of curved mirrors"\\\\
[P1] "Concave caves inward, convex curves outward"\\\\
- Focal point: where parallel rays meet\\\\
- Focal length: distance from mirror to focal point\\\\
- Concave f is positive, convex f is negative}
\end{frame}

\begin{frame}
\frametitle{16.1 Concave vs Convex Focal Points}
\begin{figure}
\centering
\includegraphics[width=0.8\textwidth,height=0.6\textheight,keepaspectratio]{phys11-geometric-optics-fig16-9.jpg}
\end{figure}

\pause
Concave: rays \textbf{converge} ($\disp{f}$ positive). Convex: rays \textbf{diverge} ($\disp{f}$ negative).
\note{[P0] [Fig 16.9: Focal points] "Focal points for curved mirrors"\\\\
[P1] "Concave: rays converge to real focal point in front. Convex: rays diverge, appear to come from virtual focal point behind"\\\\
- Sign convention matters for calculations\\\\
- Parabolic mirrors focus all parallel rays to one point}
\end{frame}

\begin{frame}
\frametitle{16.1 Concave Mirror Image Formation}
\begin{figure}
\centering
\includegraphics[width=0.8\textwidth,height=0.6\textheight,keepaspectratio]{phys11-geometric-optics-fig16-10.jpg}
\end{figure}

\pause
\textbf{Object beyond F:} Real, inverted image

\pause
\textbf{Object inside F:} Virtual, upright, magnified image
\note{[P0] [Fig 16.10: Mirror images] "Ray diagrams show image formation"\\\\
[P1] "Object beyond focal point: real inverted image forms in front"\\\\
[P2] "Object inside focal point: virtual upright magnified image appears behind"\\\\
[THE CONNECTION - Harmonic Archetype] "Makeup mirrors use this - place face inside focal point for magnification"}
\end{frame}

\begin{frame}
\frametitle{16.1 Applications: Car Headlights}
\begin{figure}
\centering
\includegraphics[width=0.7\textwidth,height=0.5\textheight,keepaspectratio]{phys11-geometric-optics-fig16-13.jpg}
\end{figure}

\pause
\textbf{Parabolic concave mirror:} Bulb at focal point $\rightarrow$ parallel rays exit

\pause
Same principle: spotlights, solar collectors, satellite dishes
\note{[P0] [Fig 16.13: Parabolic headlight] "Parabolic mirrors have special property - light bulb surrounded by parabolic mirror creates parallel beam"\\\\
[P1] "Light source at focal point creates parallel rays - perfect for headlights and spotlights"\\\\
[P2] "Reverse works too: parallel rays from Sun focus to one point, generate heat for solar collectors"\\\\
[THE WONDER] Same geometry explains car headlights and solar power plants - optics works both directions}
\end{frame}

\begin{frame}
\frametitle{16.1 Applications: Security Mirrors}
\begin{figure}
\centering
\includegraphics[width=0.7\textwidth,height=0.5\textheight,keepaspectratio]{phys11-geometric-optics-fig16-11.jpg}
\end{figure}

\pause
\textbf{Convex mirrors:} Create smaller, upright images $\rightarrow$ wider field of view
\note{[P0] [Fig 16.11: Security mirror] "Convex mirrors in stores"\\\\
[P1] "Create smaller upright images - wider field of view for security"\\\\
- Also used in car side mirrors\\\\
- Trade-off: wider view but objects appear smaller and farther\\\\
- "Objects in mirror are closer than they appear"}
\end{frame}

\begin{frame}
\frametitle{16.1 Mirror Equations}
\begin{block}{Universal Laws: The Source Code}
\textbf{Lens/Mirror equation:}
$$\frac{1}{\disp{f}} = \frac{1}{\disp{d_i}} + \frac{1}{\disp{d_o}}$$

\pause
\textbf{Magnification:}
$$m = \frac{\disp{h_i}}{\disp{h_o}} = -\frac{\disp{d_i}}{\disp{d_o}}$$

\pause
\textbf{Radius of curvature:}
$$\disp{R} = 2\disp{f}$$
\end{block}
\note{[P0] "Three equations govern all curved mirrors"\\\\
[P1] "Lens-mirror equation relates focal length, object distance, image distance"\\\\
[P2] "Magnification relates image size to object size"\\\\
[P3] "Radius of curvature is twice focal length"\\\\
[ALGEBRA] "One over f equals one over d-i plus one over d-o"}
\end{frame}

\begin{frame}
\frametitle{16.1 Sign Conventions}
\begin{alertblock}{The Paradox: What Negative Means}
\begin{itemize}
\item Negative $\disp{d_i}$ $\rightarrow$ virtual image \pause
\item Negative $\disp{h_i}$ $\rightarrow$ inverted image \pause
\item Concave: $\disp{f}$ positive; Convex: $\disp{f}$ negative
\end{itemize}
\end{alertblock}

\pause
\vspace{0.3cm}
\textbf{Key insight:} Signs tell you where and how the image appears.
\note{[P0] "Sign conventions decode the results"\\\\
[P1] "Negative d-i means virtual image - appears behind mirror"\\\\
[P2] "Negative h-i means inverted image - upside down"\\\\
[P3] "Concave f is positive, convex f is negative"\\\\
[THE HUMILITY] Easy to mix up signs - check carefully every time}
\end{frame}

\begin{frame}
\frametitle{Attempt: Security Mirror}
\begin{exampleblock}{The Challenge (3 min, silent)}
A person stands 6.0 m from a convex security mirror. The virtual image appears 1.0 m behind the mirror.

\vspace{0.3cm}

\textbf{Given:}
\begin{itemize}
\item $\disp{d_o} = 6.0$ m
\item $\disp{d_i} = -1.0$ m (virtual)
\end{itemize}

\textbf{Find:} Focal length $\disp{f}$

\vspace{0.3cm}

\textit{Can you decode the mirror's geometry? Work silently.}
\end{exampleblock}
\note{[THE CHALLENGE] Can they apply the lens-mirror equation?\\\\
[SAY] "Try this on your own. It's okay to get stuck."\\\\
[TIMING] 3-4 min SILENT individual work\\\\
[CIRCULATE] Note who forgets negative sign for d-i\\\\
[WATCH FOR] Students using magnification equation instead\\\\
[DON'T HELP] Let them struggle with equation choice}
\end{frame}

\begin{frame}
\frametitle{Compare: Mirror Strategy}
\textbf{Turn and talk (2 min):}

\vspace{0.3cm}

\begin{enumerate}
\item Which equation did you choose? Why?
\item What sign did you use for $\disp{d_i}$? Why negative?
\item How did you rearrange to solve for $\disp{f}$?
\end{enumerate}

\vspace{0.5cm}

\pause
\alert{Name wheel:} One pair share your approach (not your answer).
\note{[TIMING] 2-3 min pair discussion\\\\
[CIRCULATE] Listen for equation choice\\\\
[CHECK] Name wheel: call a pair to share approach\\\\
[EXPECTED APPROACH] Use lens-mirror equation, d-i is negative because virtual, rearrange to f equals d-i times d-o over d-o plus d-i\\\\
[COMMON ERROR] Forgetting negative sign or using wrong equation}
\end{frame}

\begin{frame}
\frametitle{Reveal: Mirror Math}
\textbf{Self-correct in a different color:}

\vspace{0.3cm}

\textbf{Equation:} Lens/mirror equation
$$\disp{f} = \frac{\disp{d_i} \disp{d_o}}{\disp{d_o} + \disp{d_i}}$$

\pause
\vspace{0.2cm}

\textbf{Substitute:} $\disp{d_i} = -1.0$ m, $\disp{d_o} = 6.0$ m

\pause
\vspace{0.2cm}

$$\disp{f} = \frac{(-1.0)(6.0)}{6.0 + (-1.0)} = \frac{-6.0}{5.0}$$

\pause
\vspace{0.2cm}

$$\boxed{\disp{f} = -1.2 \text{ m}}$$

\pause
\textbf{Check:} Negative $\disp{f}$ confirms convex mirror!
\note{[P0] "Self-correct in different color"\\\\
[P1] [ALGEBRA] "f equals d-i times d-o over d-o plus d-i"\\\\
[P2] "Substitute: d-i negative 1.0 m, d-o 6.0 m"\\\\
[P3] "f equals negative 6.0 over 5.0"\\\\
[P4] [ANSWER] "f equals negative 1.2 m"\\\\
[P5] [THE WONDER] "Negative confirms convex - physics encoded in sign"}
\end{frame}

\section{Refraction}

\begin{frame}
\frametitle{Learning Objectives}
\begin{block}{By the end of this section, you will be able to:}
\begin{itemize}
\item \textbf{16.2:} Explain refraction at media boundaries and predict light paths \pause
\item \textbf{16.2:} Describe the index of refraction and explain total internal reflection \pause
\item \textbf{16.2:} Perform calculations using Snell's law
\end{itemize}
\end{block}
\note{[P0] "Three objectives for refraction"\\\\
[P1] "First: explain refraction and predict how light bends"\\\\
[P2] "Second: understand index of refraction and total internal reflection"\\\\
[P3] "Third: calculate angles using Snell's law"\\\\
- Applications: fiber optics, diamonds, rainbows}
\end{frame}

\begin{frame}
\frametitle{16.2 The Mystery of the Bent Pencil}
\begin{center}
\Large Why does a pencil appear broken\\
\textit{when placed in water?}
\end{center}

\pause
\vspace{0.5cm}
Light changes direction when moving between air and water.

\pause
\vspace{0.3cm}
\alert{This is refraction - the bending of light at boundaries.}
\note{[P0] "Classic observation: pencil in glass of water looks bent"\\\\
[P1] "Light changes direction when crossing from air to water"\\\\
[P2] [THE REVELATION] "This is refraction - bending at boundaries between media"\\\\
[THE CONNECTION - Harmonic Archetype] "Like sound changing pitch when you hear it through wall"}
\end{frame}

\begin{frame}
\frametitle{16.2 Why Light Bends}
\begin{exampleblock}{The Mental Model: Lawnmower Analogy}
\begin{itemize}
\item Lawnmower from sidewalk to grass: right wheel slows first, mower turns \pause
\item Light from air to glass: slows down, bends toward normal \pause
\item Light from glass to air: speeds up, bends away from normal
\end{itemize}
\end{exampleblock}
\note{[P0] "Why does light change direction?"\\\\
[P1] "Lawnmower from sidewalk to grass - one wheel slows, mower turns"\\\\
[P2] "Light entering denser medium slows, bends toward normal"\\\\
[P3] "Light entering less dense medium speeds up, bends away from normal"\\\\
[THE CONNECTION - Kinetic Archetype] "Like running from track onto sand - you slow and veer"}
\end{frame}

\begin{frame}
\frametitle{16.2 Index of Refraction}
\begin{block}{Nature's Speed Limit Code}
$$\pConst{n} = \frac{\vel{c}}{\vel{v}}$$
\end{block}

\pause
\vspace{0.3cm}

\textbf{Where:}
\begin{itemize}
\item $\pConst{n}$ = index of refraction (dimensionless) \pause
\item $\vel{c} = 3.00 \times 10^8$ m/s (\vel{speed} of light in vacuum) \pause
\item $\vel{v}$ = \vel{speed} of light in the material
\end{itemize}

\pause
Because $\vel{c} > \vel{v}$ always, $\pConst{n} \geq 1$ always.
\note{[P0] "Index of refraction: n equals c over v"\\\\
[P1] "n is index of refraction - no units"\\\\
[P2] "c is speed of light in vacuum - 3.0 times 10 to 8 m/s"\\\\
[P3] "v is speed of light in material - always less than c"\\\\
[P4] "Because c greater than v, n is always greater than or equal to 1"\\\\
[THE WONDER] Light slows in matter - interacts with atoms}
\end{frame}

\begin{frame}
\frametitle{16.2 Snell's Law}
\begin{figure}
\centering
\includegraphics[width=0.6\textwidth,height=0.4\textheight,keepaspectratio]{phys11-geometric-optics-fig16-17.jpg}
\end{figure}

\pause
\begin{block}{Universal Law: The Bending Rule}
\begin{center}
\Large $\boxed{\pConst{n_1} \sin\angle{\theta_1} = \pConst{n_2} \sin\angle{\theta_2}}$
\end{center}
This predicts exactly how light bends at any boundary.
\end{block}
\note{[P0] [Fig 16.17: Lawnmower analogy] "Understanding refraction through lawnmower crossing from sidewalk to grass - wheels slow at different times causing turn"\\\\
[P1] [THE REVELATION] "Snell's law: n-1 sin-theta-1 equals n-2 sin-theta-2 - mathematical prediction of bending"\\\\
[ALGEBRA] "n-one sine-theta-one equals n-two sine-theta-two"\\\\
- Theta measured from normal to surface\\\\
- Greater n means slower light, more bending toward normal\\\\
[THE WONDER] Works for any two materials - air-water, glass-diamond, anything. Same mechanism as lawnmower turning}
\end{frame}

\begin{frame}
\frametitle{16.2 Dispersion: Rainbows}
\begin{figure}
\centering
\includegraphics[width=0.7\textwidth,height=0.5\textheight,keepaspectratio]{phys11-geometric-optics-fig16-18.jpg}
\end{figure}

\pause
\textbf{Dispersion:} Index of refraction varies slightly with \wavelen{wavelength}

\pause
White light separates into colors: red bends least, violet bends most
\note{[P0] [Fig 16.18: Prism dispersion] "Prism separating white light into rainbow"\\\\
[P1] "Dispersion: index of refraction varies with wavelength"\\\\
[P2] "Red light bends least, violet most - different speeds in glass"\\\\
[THE CONNECTION - Harmonic Archetype] "Each color is different frequency - prism separates them like sorting musical notes"}
\end{frame}

\begin{frame}
\frametitle{16.2 Total Internal Reflection}
\begin{figure}
\centering
\includegraphics[width=0.8\textwidth,height=0.6\textheight,keepaspectratio]{phys11-geometric-optics-fig16-20.jpg}
\end{figure}

\pause
When $\angle{\theta_1} > \angle{\theta_c}$, \textbf{all} light reflects back - no refraction!
\note{[P0] [Fig 16.20: Critical angle] "Total internal reflection: 100 percent reflection"\\\\
[P1] "When incident angle exceeds critical angle, all light reflects - none escapes"\\\\
- Only occurs when going from high n to low n\\\\
- Critical angle: angle that produces 90-degree refraction\\\\
[THE WONDER] Perfect mirrors using refraction instead of reflection}
\end{frame}

\begin{frame}
\frametitle{16.2 Critical Angle}
\begin{block}{The Escape Threshold}
$$\angle{\theta_c} = \sin^{-1}\left(\frac{\pConst{n_2}}{\pConst{n_1}}\right)$$
for $\pConst{n_1} > \pConst{n_2}$
\end{block}

\pause
\vspace{0.3cm}

\textbf{Example:} Water to air

$\angle{\theta_c} = \sin^{-1}(1.00/1.33) = 48.6^\circ$
\note{[P0] "Critical angle equation"\\\\
[P1] [ALGEBRA] "Theta-c equals arcsine of n-2 over n-1, only when n-1 greater than n-2"\\\\
- Arcsine means inverse sine function\\\\
- Water to air: critical angle 48.6 degrees\\\\
- Beyond this angle, light cannot escape water}
\end{frame}

\begin{frame}
\frametitle{16.2 Applications: Diamonds}
\begin{figure}
\centering
\includegraphics[width=0.6\textwidth,height=0.45\textheight,keepaspectratio]{phys11-geometric-optics-fig16-21.jpg}
\end{figure}

\pause
Diamond critical angle: only $24.4^\circ$!

\pause
Light enters easily but struggles to exit $\rightarrow$ sparkle!
\note{[P0] [Fig 16.21: Diamond ray path] "Why diamonds sparkle - light ray entering diamond and bouncing off internal facets"\\\\
[P1] "Diamond has very high index of refraction (n=2.42), critical angle only 24.4 degrees - much smaller than water or glass"\\\\
[P2] "Light enters easily but trapped inside by total internal reflection, exits only at specific angles creating concentrated flashes"\\\\
- Facets designed to maximize internal reflections before exit\\\\
- Concentrated exits create brilliant sparkle and fire\\\\
[THE WONDER] Geometry explains beauty - diamond cutters use physics to maximize brilliance}
\end{frame}

\begin{frame}
\frametitle{16.2 Applications: Fiber Optics}
\begin{figure}
\centering
\includegraphics[width=0.7\textwidth,height=0.5\textheight,keepaspectratio]{phys11-geometric-optics-fig16-23.jpg}
\end{figure}

\pause
\textbf{Total internal reflection in thin fibers:}
\begin{itemize}
\item Light enters at large angle
\item Reflects repeatedly inside fiber
\item Carries signals around corners!
\end{itemize}
\note{[P0] [Fig 16.23: Fiber optics] "Fiber optics use total internal reflection"\\\\
[P1] "Light enters fiber at large angle, exceeds critical angle at walls, trapped inside"\\\\
- Used for internet, telephone, cable TV\\\\
- Light pipes carry signals around corners\\\\
- Faster than electrical signals in copper\\\\
[THE WONDER] Your internet travels as trapped light bouncing through glass}
\end{frame}

\begin{frame}
\frametitle{Attempt: Light Bending Through Glass}
\begin{exampleblock}{The Challenge (3 min, silent)}
Light enters glass ($n = 1.50$) from air ($n = 1.00$) at $45.0^\circ$ to the normal.

\vspace{0.3cm}

\textbf{Given:}
\begin{itemize}
\item $\pConst{n_1} = 1.00$ (air)
\item $\pConst{n_2} = 1.50$ (glass)
\item $\angle{\theta_1} = 45.0^\circ$
\end{itemize}

\textbf{Find:} \angle{Angle} of refraction $\angle{\theta_2}$

\vspace{0.3cm}

\textit{Can you predict the bend? Work silently.}
\end{exampleblock}
\note{[THE CHALLENGE] Can they apply Snell's law?\\\\
[SAY] "Try this on your own. It's okay to get stuck."\\\\
[TIMING] 3-4 min SILENT individual work\\\\
[CIRCULATE] Note who struggles with sine and arcsine\\\\
[WATCH FOR] Students using degrees in calculator set to radians\\\\
[DON'T HELP] Let them wrestle with trig functions}
\end{frame}

\begin{frame}
\frametitle{Compare: Refraction Strategy}
\textbf{Turn and talk (2 min):}

\vspace{0.3cm}

\begin{enumerate}
\item Which law did you use?
\item How did you rearrange for $\angle{\theta_2}$?
\item Did you use arcsine? Why?
\end{enumerate}

\vspace{0.5cm}

\pause
\alert{Name wheel:} One pair share your approach (not your answer).
\note{[TIMING] 2-3 min pair discussion\\\\
[CIRCULATE] Listen for Snell's law application\\\\
[CHECK] Name wheel: call a pair to share\\\\
[EXPECTED APPROACH] Use Snell's law, solve for sin-theta-2, use arcsine to find theta-2\\\\
[COMMON ERROR] Calculator in radians instead of degrees}
\end{frame}

\begin{frame}
\frametitle{Reveal: Snell's Law Solution}
\textbf{Self-correct in a different color:}

\vspace{0.3cm}

\textbf{Snell's Law:} $\pConst{n_1} \sin\angle{\theta_1} = \pConst{n_2} \sin\angle{\theta_2}$

\pause
\textbf{Rearrange:} $\sin\angle{\theta_2} = \frac{\pConst{n_1} \sin\angle{\theta_1}}{\pConst{n_2}}$

\pause
\textbf{Substitute:} $\sin\angle{\theta_2} = \frac{(1.00)(\sin 45.0^\circ)}{1.50} = \frac{0.707}{1.50}$

\pause
$$\sin\angle{\theta_2} = 0.471$$

\pause
$$\boxed{\angle{\theta_2} = \sin^{-1}(0.471) = 28.1^\circ}$$

\pause
\textbf{Check:} Light bends toward normal entering denser medium!
\note{[P0] "Self-correct in different color"\\\\
[P1] [ALGEBRA] "Snell's law: n-1 sine-theta-1 equals n-2 sine-theta-2"\\\\
[P2] "Rearrange: sine-theta-2 equals n-1 sine-theta-1 over n-2"\\\\
[P3] "Substitute: 1.00 times 0.707 over 1.50"\\\\
[P4] "Sine-theta-2 equals 0.471"\\\\
[P5] [ANSWER] "Theta-2 equals arcsine of 0.471 equals 28.1 degrees"\\\\
[P6] [THE WONDER] "Light bent from 45 to 28 degrees - toward normal entering denser glass"}
\end{frame}

\section{Lenses}

\begin{frame}
\frametitle{Learning Objectives}
\begin{block}{By the end of this section, you will be able to:}
\begin{itemize}
\item \textbf{16.3:} Describe image formation by convex and concave lenses \pause
\item \textbf{16.3:} Explain how the human eye works using geometric optics \pause
\item \textbf{16.3:} Perform calculations using the thin-lens equation
\end{itemize}
\end{block}
\note{[P0] "Three objectives for lenses"\\\\
[P1] "First: understand image formation by lenses"\\\\
[P2] "Second: explain how your eye works"\\\\
[P3] "Third: calculate image properties using thin-lens equation"\\\\
- Applications: eyeglasses, cameras, microscopes, telescopes}
\end{frame}

\begin{frame}
\frametitle{16.3 The Power to Focus Sunlight}
\begin{center}
\Large What if you could concentrate sunlight\\
\textit{to ignite paper?}
\end{center}

\pause
\vspace{0.5cm}
A converging lens bends all parallel rays to one focal point.

\pause
\vspace{0.3cm}
\alert{Enough concentrated light energy = fire!}
\note{[P0] "The power of converging lenses"\\\\
[P1] "Converging lens bends all parallel rays to single focal point"\\\\
[P2] "Concentrated light energy can produce enough heat to ignite paper"\\\\
[THE CONNECTION - Digital Archetype] "Like Death Star focusing laser - concentrate energy at one point"\\\\
[SAFETY] "Never look at Sun through lens - can cause permanent eye damage"}
\end{frame}

\begin{frame}
\frametitle{16.3 Converging vs Diverging Lenses}
\begin{figure}
\centering
\includegraphics[width=0.8\textwidth,height=0.6\textheight,keepaspectratio]{phys11-geometric-optics-fig16-25.jpg}
\end{figure}

\pause
\textbf{Convex (converging):} Parallel rays converge to focal point ($\disp{f}$ positive)

\pause
\textbf{Concave (diverging):} Parallel rays diverge from focal point ($\disp{f}$ negative)
\note{[P0] [Fig 16.25: Converging lens] "Two types of lenses"\\\\
[P1] "Convex or converging: parallel rays converge to focal point, f positive"\\\\
[P2] "Concave or diverging: parallel rays diverge, appear to come from focal point, f negative"\\\\
- Thicker in middle: converging\\\\
- Thicker at edges: diverging}
\end{frame}

\begin{frame}
\frametitle{16.3 Lens Power}
\begin{block}{The Focusing Strength}
$$\power{P} = \frac{1}{\disp{f}}$$
\end{block}

\pause
\vspace{0.3cm}

\textbf{Where:}
\begin{itemize}
\item $\power{P}$ = power in diopters (D) or m$^{-1}$ \pause
\item $\disp{f}$ = focal length in meters
\end{itemize}

\pause
\vspace{0.3cm}

Shorter \disp{focal length} = stronger lens = higher \power{power}
\note{[P0] "Power of lens: P equals one over f"\\\\
[P1] "P is power in diopters or reciprocal meters"\\\\
[P2] "f is focal length in meters"\\\\
[P3] "Shorter focal length means stronger lens, higher power"\\\\
[ALERT] "Power in diopters different from power in watts - same word, different meanings"}
\end{frame}

\begin{frame}
\frametitle{16.3 Ray Tracing Rules for Lenses}
\textbf{Converging lens:}
\begin{enumerate}
\item Ray parallel to axis $\rightarrow$ passes through far focal point \pause
\item Ray through center $\rightarrow$ continues straight \pause
\item Ray through near focal point $\rightarrow$ exits parallel to axis
\end{enumerate}

\pause
\vspace{0.3cm}

\textbf{Diverging lens:}
\begin{enumerate}
\item Ray parallel to axis $\rightarrow$ appears from near focal point \pause
\item Ray through center $\rightarrow$ continues straight \pause
\item Ray toward far focal point $\rightarrow$ exits parallel to axis
\end{enumerate}
\note{[P0] "Ray tracing rules for converging lenses"\\\\
[P1] "Ray parallel to axis passes through far focal point"\\\\
[P2] "Ray through center continues straight"\\\\
[P3] "Ray through near focal point exits parallel"\\\\
[P4] "Diverging lens rules similar but rays diverge"\\\\
- Only need two rays to locate image}
\end{frame}

\begin{frame}
\frametitle{16.3 Converging Lens Image Formation}
\begin{figure}
\centering
\includegraphics[width=0.8\textwidth,height=0.6\textheight,keepaspectratio]{phys11-geometric-optics-fig16-27.jpg}
\end{figure}

\pause
Object beyond $\disp{f}$: Real, inverted image on far side

\pause
Object inside $\disp{f}$: Virtual, upright, magnified image on near side
\note{[P0] [Fig 16.27: Ray tracing] "Ray diagram for converging lens"\\\\
[P1] "Object beyond focal length: real inverted image forms on opposite side"\\\\
[P2] "Object inside focal length: virtual upright magnified image on same side"\\\\
- Real images can be projected on screen\\\\
- Virtual images cannot\\\\
[THE CONNECTION - Kinetic Archetype] "Movie projectors use this - film beyond f creates real image on screen"}
\end{frame}

\begin{frame}
\frametitle{16.3 Thin Lens Equation}
\begin{block}{Same Equation as Mirrors}
$$\frac{1}{\disp{f}} = \frac{1}{\disp{d_i}} + \frac{1}{\disp{d_o}}$$
\end{block}

\pause
\vspace{0.3cm}

\textbf{Magnification:}
$$m = \frac{\disp{h_i}}{\disp{h_o}} = -\frac{\disp{d_i}}{\disp{d_o}}$$

\pause
\vspace{0.3cm}

\textbf{Sign conventions:}
\begin{itemize}
\item Converging lens: $\disp{f}$ positive; Diverging lens: $\disp{f}$ negative
\item Negative $\disp{d_i}$ $\rightarrow$ virtual image
\item Negative $m$ $\rightarrow$ inverted image
\end{itemize}
\note{[P0] "Thin lens equation identical to mirror equation"\\\\
[P1] "One over f equals one over d-i plus one over d-o"\\\\
[P2] "Magnification m equals h-i over h-o equals negative d-i over d-o"\\\\
[P3] "Sign conventions tell you image type"\\\\
[THE WONDER] Same math for mirrors and lenses - universal geometric optics}
\end{frame}

\begin{frame}
\frametitle{16.3 The Human Eye}
\begin{figure}
\centering
\includegraphics[width=0.7\textwidth,height=0.5\textheight,keepaspectratio]{phys11-geometric-optics-fig16-32.jpg}
\end{figure}

\pause
\textbf{Cornea and lens:} Form real image on retina

\pause
\textbf{Ciliary muscles:} Change lens shape to adjust \disp{focal length}

\pause
Eye adjusts \power{power} to keep image \disp{distance} constant for all object \disp{distances}!
\note{[P0] [Fig 16.32: Eye cross-section] "The eye as optical instrument - schematic showing all internal components"\\\\
[P1] "Cornea and lens act as single converging lens system, form real inverted image on retina"\\\\
[P2] "Ciliary muscles change lens shape to adjust focal length - this is accommodation"\\\\
[P3] "Eye adjusts power to focus objects at different distances - keeps image distance constant on retina"\\\\
- Retina is fixed distance from lens - unlike camera which moves lens\\\\
- Most refraction (about 70 percent) occurs at cornea air-cornea boundary, not inside lens}
\end{frame}

\begin{frame}
\frametitle{16.3 Vision Defects}
\begin{figure}
\centering
\includegraphics[width=0.8\textwidth,height=0.6\textheight,keepaspectratio]{phys11-geometric-optics-fig16-34.jpg}
\end{figure}

\pause
\textbf{Nearsighted (myopia):} Eye too strong, image forms in front of retina

\pause
\textbf{Farsighted (hyperopia):} Eye too weak, image forms behind retina
\note{[P0] [Fig 16.34: Myopia and hyperopia] "Two common vision defects - schematic cross-sections showing where image forms"\\\\
[P1] "Nearsighted (myopia): eye overconverges rays due to elongated eyeball or overpowered lens, image forms in front of retina - distant objects blurry"\\\\
[P2] "Farsighted (hyperopia): eye underconverges rays due to short eyeball or underpowered lens, image would form behind retina - close objects blurry"\\\\
- Nearsighted: can see close clearly, distant blurry\\\\
- Farsighted: can see distant clearly, close blurry\\\\
[THE CONNECTION] Eye shape determines optical power - biology creates physics problem}
\end{frame}

\begin{frame}
\frametitle{16.3 Correcting Vision Defects}
\begin{figure}
\centering
\includegraphics[width=0.8\textwidth,height=0.6\textheight,keepaspectratio]{phys11-geometric-optics-fig16-35.jpg}
\end{figure}

\pause
\textbf{Nearsighted:} Diverging lens (concave) reduces power

\pause
\textbf{Farsighted:} Converging lens (convex) increases power
\note{[P0] [Fig 16.35: Vision correction] "Correcting vision with eyeglasses - schematic showing corrective lens placement and ray paths"\\\\
[P1] "Nearsighted correction: diverging (concave) lens placed in front of eye reduces overall power, diverges rays before they enter eye, moves image back to retina"\\\\
[P2] "Farsighted correction: converging (convex) lens placed in front of eye increases overall power, converges rays before they enter eye, moves image forward to retina"\\\\
[THE WONDER] Physics corrects biology - external lenses compensate for eye shape defects. Eyeglasses are optical system working with eye's optics}
\end{frame}

\begin{frame}
\frametitle{16.3 Applications: Microscope}
\begin{figure}
\centering
\includegraphics[width=0.7\textwidth,height=0.5\textheight,keepaspectratio]{phys11-geometric-optics-fig16-30.jpg}
\end{figure}

\pause
\textbf{Two converging lenses:}
\begin{itemize}
\item Objective: Creates magnified real image
\item Eyepiece: Further magnifies that image
\end{itemize}
\note{[P0] [Fig 16.30: Microscope] "Compound microscope uses two lenses"\\\\
[P1] "Objective lens creates magnified real image"\\\\
- Eyepiece acts as magnifying glass on that image\\\\
- Total magnification is product of both\\\\
- Invented early 1600s by eyeglass makers}
\end{frame}

\begin{frame}
\frametitle{16.3 Applications: Telescope}
\begin{figure}
\centering
\includegraphics[width=0.8\textwidth,height=0.6\textheight,keepaspectratio]{phys11-geometric-optics-fig16-29.jpg}
\end{figure}

\pause
\textbf{Galileo design:} Convex objective + concave eyepiece = upright image

\pause
\textbf{Astronomical:} Two convex lenses = inverted image (doesn't matter for stars!)
\note{[P0] [Fig 16.29: Telescope] "Telescope designs using lenses"\\\\
[P1] "Galileo design: convex objective, concave eyepiece, produces upright image"\\\\
[P2] "Astronomical design: two convex lenses, inverted image fine for celestial objects"\\\\
[THE WONDER] Same physics lets us see cells and galaxies}
\end{frame}

\begin{frame}
\frametitle{Attempt: Magnifying Glass Power}
\begin{exampleblock}{The Challenge (3 min, silent)}
A magnifying glass focuses sunlight to a bright spot 8.00 cm from the lens.

\vspace{0.3cm}

\textbf{Given:}
\begin{itemize}
\item $\disp{f} = 8.00$ cm $= 0.0800$ m
\end{itemize}

\textbf{Find:} \power{Power} $\power{P}$ in diopters

\vspace{0.3cm}

\textit{Can you calculate the lens strength? Work silently.}
\end{exampleblock}
\note{[THE CHALLENGE] Can they use the power equation?\\\\
[SAY] "Try this on your own. It's okay to get stuck."\\\\
[TIMING] 3-4 min SILENT individual work\\\\
[CIRCULATE] Note who forgets to convert cm to m\\\\
[WATCH FOR] Students confusing power with energy\\\\
[DON'T HELP] Let them discover unit conversion need}
\end{frame}

\begin{frame}
\frametitle{Compare: Power Strategy}
\textbf{Turn and talk (2 min):}

\vspace{0.3cm}

\begin{enumerate}
\item Which equation relates \power{power} and \disp{focal length}?
\item What units did you use for \disp{focal length}?
\item Did you convert centimeters to meters?
\end{enumerate}

\vspace{0.5cm}

\pause
\alert{Name wheel:} One pair share your approach (not your answer).
\note{[TIMING] 2-3 min pair discussion\\\\
[CIRCULATE] Listen for unit conversion\\\\
[CHECK] Name wheel: call a pair to share\\\\
[EXPECTED APPROACH] Use P equals one over f, convert cm to m first\\\\
[COMMON ERROR] Forgetting to convert units}
\end{frame}

\begin{frame}
\frametitle{Reveal: Lens Power}
\textbf{Self-correct in a different color:}

\vspace{0.3cm}

\textbf{Given:} $\disp{f} = 8.00$ cm

\pause
\textbf{Convert to meters:} $\disp{f} = 0.0800$ m

\pause
\vspace{0.2cm}

\textbf{Power equation:} $\power{P} = \frac{1}{\disp{f}}$

\pause
\vspace{0.2cm}

$$\power{P} = \frac{1}{0.0800 \text{ m}}$$

\pause
\vspace{0.2cm}

$$\boxed{\power{P} = 12.5 \text{ D}}$$

\pause
\textbf{Check:} This is a relatively powerful lens!
\note{[P0] "Self-correct in different color"\\\\
[P1] "Given: f equals 8.00 cm"\\\\
[P2] "Convert: f equals 0.0800 m"\\\\
[P3] "Power equation: P equals one over f"\\\\
[P4] "P equals one over 0.0800 m"\\\\
[P5] [ANSWER] "P equals 12.5 diopters"\\\\
[P6] [THE WONDER] "Powerful lens - can focus sunlight to ignite paper"}
\end{frame}

\section{Summary}

\begin{frame}
\frametitle{What You Now Know}
\begin{block}{The Revelations}
\begin{enumerate}
\item Law of Reflection: $\angle{\theta_r} = \angle{\theta_i}$ (mirrors create virtual images) \pause
\item Curved mirrors use focal points to form real or virtual images \pause
\item Snell's Law: $\pConst{n_1}\sin\angle{\theta_1} = \pConst{n_2}\sin\angle{\theta_2}$ (light bends at boundaries) \pause
\item Total internal reflection traps light (fiber optics, diamonds) \pause
\item Lenses use refraction to converge or diverge rays \pause
\item Your eye is a variable-\power{power} converging lens \pause
\item Same equations govern mirrors and lenses
\end{enumerate}
\end{block}
\note{[P0] "Seven revelations today"\\\\
[P1] "Law of Reflection creates virtual images in mirrors"\\\\
[P2] "Curved mirrors use focal points"\\\\
[P3] "Snell's Law predicts refraction"\\\\
[P4] "Total internal reflection traps light"\\\\
[P5] "Lenses converge or diverge"\\\\
[P6] "Eye adjusts focal length"\\\\
[P7] "Same math for mirrors and lenses"\\\\
[THE WONDER] You now understand how light creates illusions and corrects vision}
\end{frame}

\begin{frame}[shrink]
\frametitle{Key Equations}
\begin{align}
\text{Law of Reflection:} \quad &\angle{\theta_r} = \angle{\theta_i} \\
\text{Index of refraction:} \quad &\pConst{n} = \frac{\vel{c}}{\vel{v}} \\
\text{Snell's Law:} \quad &\pConst{n_1}\sin\angle{\theta_1} = \pConst{n_2}\sin\angle{\theta_2} \\
\text{Critical angle:} \quad &\angle{\theta_c} = \sin^{-1}\left(\frac{\pConst{n_2}}{\pConst{n_1}}\right) \\
\text{Lens/Mirror:} \quad &\frac{1}{\disp{f}} = \frac{1}{\disp{d_i}} + \frac{1}{\disp{d_o}} \\
\text{Magnification:} \quad &m = \frac{\disp{h_i}}{\disp{h_o}} = -\frac{\disp{d_i}}{\disp{d_o}} \\
\text{Lens power:} \quad &\power{P} = \frac{1}{\disp{f}}
\end{align}
\note{- Seven key equations for geometric optics\\\\
- Law of reflection for mirrors\\\\
- Index of refraction and Snell's law for refraction\\\\
- Critical angle for total internal reflection\\\\
- Lens-mirror equation and magnification\\\\
- Lens power\\\\
- Master these and you can solve any optics problem}
\end{frame}

\begin{frame}
\frametitle{Homework}
\begin{center}
\Large
Complete the assigned problems\\[0.3cm]
posted on the LMS
\end{center}
\note{[SAY] "Homework posted on LMS"\\\\
[TIMING] Due date: check LMS\\\\
[CHECK] Questions before we end?\\\\
[TRANSITION] Next: Wave Optics - interference and diffraction\\\\
[THE WONDER] Today geometric optics - next we see wave nature of light}
\end{frame}

\end{document}
