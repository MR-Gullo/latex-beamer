\documentclass{beamer}
\usepackage{../../../shared/templates/ds9_theme}
\usepackage{../../../shared/templates/semantic-physics-colors}
\usepackage[overridenote]{pdfpc}
\graphicspath{{../images/}{../../shared/images/}}

\title[Invisible Rainbow]{PHYS11 CH:15 The Invisible Rainbow}
\subtitle{From Radio Waves to Gamma Rays}
\author[Mr. Gullo]{Mr. Gullo}
\date[December 2025]{December 2025}

\begin{document}

\frame{\titlepage
\note{[THE HOOK] Today we discover the invisible universe.\\\\
- Light you see is only tiny fraction of what exists\\\\
- Same physics from radio to gamma rays\\\\
[THE WONDER] By end of class, you'll see the invisible\\\\
- This is how we decode the cosmos}
}

\begin{frame}
\frametitle{Outline}
\tableofcontents
\end{frame}

\section{Introduction}

\begin{frame}
\frametitle{The Mystery}
\begin{center}
\Large What if everything you see\\
\textit{is only 0.0035\% of what exists?}
\end{center}

\pause
\vspace{0.5cm}
Visible light is a narrow sliver of electromagnetic radiation.

\pause
\vspace{0.3cm}
\alert{The universe broadcasts in frequencies we cannot see.}
\note{[P0] "What if everything you see is only a fraction of what exists?"\\\\
[P1] "Visible light is a narrow sliver of electromagnetic radiation"\\\\
[P2] [THE WONDER] "The universe broadcasts in frequencies we cannot see. Today we learn to detect the invisible"}
\end{frame}

\begin{frame}
\frametitle{Seeing the Unseeable}
\begin{figure}
\centering
\includegraphics[width=0.7\textwidth,height=0.5\textheight,keepaspectratio]{phys11-em-spectrum-fig15-1.jpg}
\end{figure}

\pause
\begin{exampleblock}{The Mental Model}
Fish eyes detect visible light. Our instruments detect the rest.
\end{exampleblock}
\note{[P0][Fig 15.1: Colorful fish swimming around coral reef] "Fish detect visible light wavelengths in their environment - evolution shaped sensory range"\\\\
[P1] [THE REVELATION] "Our eyes evolved for narrow 400-700nm band because atmosphere transmits these wavelengths - survival advantage"\\\\
[TEACHING HINT] Start with familiar biology (vision) then expand to invisible spectrum - scaffolds from known to unknown\\\\
[THE CONNECTION - Digital Archetype] "Like unlocking hidden channels on receiver - same EM waves, different frequencies"\\\\
[THE WONDER] Same Maxwell equations govern all EM radiation from radio to gamma}
\end{frame}

\section{15.1 The Electromagnetic Spectrum}

\begin{frame}
\frametitle{Learning Objectives}
\begin{block}{By the end of this lesson, you will be able to:}
\begin{itemize}
\item \textbf{15.1:} Define the electromagnetic spectrum in terms of frequencies and wavelengths \pause
\item \textbf{15.1:} Describe the differences and similarities of each section of the EM spectrum \pause
\item \textbf{15.1:} Explain applications of radiation from each section
\end{itemize}
\end{block}
\note{[P0] "Three objectives today"\\\\
[P1] "First: what is the EM spectrum"\\\\
[P2] "Second: how each type differs"\\\\
[P3] "Third: real-world applications"\\\\
- Assessment: quiz next week}
\end{frame}

\begin{frame}
\frametitle{15.1 The Source: Oscillating Charge}
\begin{block}{Nature's Broadcasting System}
Electromagnetic radiation is generated by a moving electric charge.
\end{block}

\pause
\vspace{0.3cm}

\textbf{What makes an EM wave:}
\begin{itemize}
\item Electric current creates electric field E \pause
\item Electric current creates magnetic field B \pause
\item E and B perpendicular to each other \pause
\item When charge oscillates, wave propagates
\end{itemize}

\note{[P0] [THE REVELATION] "EM radiation is generated by moving charge - oscillating current"\\\\
[P1] "Electric current creates electric field E"\\\\
[P2] "Electric current creates magnetic field B"\\\\
[P3] "E and B perpendicular to each other"\\\\
[P4] "When charge oscillates, wave propagates outward"\\\\
[THE CONNECTION - Kinetic Archetype] "Like shaking a rope sends wave down its length"}
\end{frame}

\begin{frame}
\frametitle{15.1 Anatomy of an EM Wave}
\begin{figure}
\centering
\includegraphics[width=0.7\textwidth,height=0.5\textheight,keepaspectratio]{phys11-em-spectrum-fig15-2.jpg}
\end{figure}

\pause
E and B fields oscillate in phase, perpendicular to each other and to direction of propagation.
\note{[P0][Fig 15.2: Intersecting E and B field planes with oscillating waves] "EM wave emerges from oscillating charge - perpendicular E (blue, vertical) and B (red, horizontal) fields"\\\\
[P1] "E and B oscillate in phase - synchronized peaks and troughs - mutually generating each other"\\\\
- E and B perpendicular to each other (90° spatial relationship)\\\\
- Both perpendicular to propagation direction c (transverse wave)\\\\
- Wave self-propagates through space - no medium required\\\\
[TEACHING HINT] Use right-hand rule: fingers point E direction, curl toward B direction, thumb shows propagation - helps visualize 3D geometry\\\\
[THE WONDER] This is how radio towers broadcast - accelerating electrons in antenna generate propagating EM fields}
\end{frame}

\begin{frame}
\frametitle{15.1 Wave Properties Review}
\textbf{All waves share these features:}
\begin{itemize}
\item \textbf{\wavelen{Wavelength}} $\wavelen{\lambda}$: Distance between two crests (meters) \pause
\item \textbf{\freq{Frequency}} $\freq{f}$: Number of crests passing per second (Hz) \pause
\item \textbf{\amplitude{Amplitude}}: Height of crest above null point
\end{itemize}

\pause

\begin{block}{Universal Law: The Speed of Light}
\begin{center}
\Large $\boxed{\vel{c} = \freq{f}\wavelen{\lambda}}$
\end{center}
\vel{Speed} equals \freq{frequency} times \wavelen{wavelength}. $\vel{c} = 3.00 \times 10^{8}$ m/s.
\end{block}
\note{[P0] "Three wave properties apply to all waves"\\\\
[P1] "Wavelength lambda: distance between crests"\\\\
[P2] "Frequency f: crests per second"\\\\
[P3] "Amplitude: height of crest"\\\\
[P4] [THE REVELATION] "c equals f lambda - speed of light is constant"\\\\
[THE WONDER] Same speed for all EM radiation in vacuum}
\end{frame}

\begin{frame}
\frametitle{15.1 The Full Spectrum}
\begin{figure}
\centering
\includegraphics[width=0.9\textwidth,height=0.7\textheight,keepaspectratio]{phys11-em-spectrum-fig15-3.jpg}
\end{figure}
\note{[Fig 15.3: Full EM spectrum with frequency and wavelength scales] "Complete electromagnetic spectrum - radio waves (10^0 Hz) to gamma rays (>10^24 Hz)"\\\\
- Visible light (400-700nm) is tiny sliver in middle - only 0.0035% of total spectrum\\\\
- Low frequency (long wavelength) on left: radio, infrared\\\\
- High frequency (short wavelength) on right: UV, X-ray, gamma\\\\
- Wavelength inversely proportional to frequency: lambda = c/f\\\\
- Energy directly proportional to frequency: E = hf (Planck relation)\\\\
[TEACHING HINT] Have students locate familiar technologies (WiFi ~10^9 Hz, dental X-rays ~10^18 Hz) to anchor abstract scale\\\\
- Name wheel: which types have you heard of? Build on prior knowledge\\\\
[THE WONDER] All governed by same physics - only frequency differs}
\end{frame}

\begin{frame}
\frametitle{15.1 Decoding the Spectrum}
\begin{columns}[T]
\column{0.48\textwidth}
\textbf{Low Frequency (IR):}
\begin{itemize}
\item Radio waves
\item Microwaves
\item Infrared (heat)
\end{itemize}

\pause
\column{0.48\textwidth}
\textbf{High Frequency (UV):}
\begin{itemize}
\item Ultraviolet
\item X-rays
\item Gamma rays
\end{itemize}
\end{columns}

\pause
\vspace{0.3cm}

\begin{exampleblock}{The Mental Model}
IR = below red. UV = beyond violet. Visible light in the middle.
\end{exampleblock}
\note{[P0] [Fig 15.3: The electromagnetic spectrum, showing] "Two categories: infrared and ultraviolet"\\\\
[P1] "High frequency includes UV, X-rays, gamma rays"\\\\
[P2] "IR means below red. UV means beyond violet"\\\\
- Visible light is narrow band between them\\\\
- Names tell you position relative to visible}
\end{frame}

\begin{frame}
\frametitle{15.1 The Intuition Trap}
\begin{alertblock}{What Your Brain Gets Wrong}
\textbf{Misconception:} Visible light is somehow different from other EM radiation.\\
\textbf{Reality:} All EM radiation is identical except for frequency and wavelength.
\end{alertblock}

\pause
\vspace{0.3cm}

\textbf{Why we see visible light:}
\begin{itemize}
\item Our eyes evolved to detect 400-700 nm wavelengths
\item This is the frequency range that penetrates atmosphere
\item Has nothing to do with the radiation itself
\end{itemize}
\note{[P0] [THE CONFLICT] "Visible light is not special - it's just what our eyes detect"\\\\
[P1] "Eyes evolved for 400-700 nm wavelengths"\\\\
- This is frequency that penetrates atmosphere\\\\
- Survival advantage to see this range\\\\
[THE HUMILITY] Our perception is limited - physics reveals the rest}
\end{frame}

\begin{frame}
\frametitle{15.1 Radio Waves}
\begin{exampleblock}{Real-World: Broadcasting}
\begin{itemize}
\item AM/FM radio, TV signals
\item Cell phones, Wi-Fi
\item Longest wavelengths, lowest frequencies
\end{itemize}
\end{exampleblock}

\pause
\vspace{0.3cm}

\textbf{AM vs FM:}
\begin{itemize}
\item AM: Amplitude Modulation (varies amplitude)
\item FM: Frequency Modulation (varies frequency)
\end{itemize}
\note{- Radio waves have longest wavelengths\\\\
- Low frequency, low energy\\\\
- Used for broadcasting because they travel far\\\\
- AM varies amplitude, FM varies frequency\\\\
- Information encoded in the wave properties\\\\
- Turn and talk: what devices use radio waves?}
\end{frame}

\begin{frame}
\frametitle{15.1 Microwaves}
\begin{exampleblock}{Real-World: Cooking and Radar}
\begin{itemize}
\item Microwave ovens: frequency $2.45 \times 10^{9}$ Hz
\item Cause polar molecules (water) to rotate
\item Rotational energy becomes heat
\item Radar: detect location and speed of objects
\end{itemize}
\end{exampleblock}

\pause
\vspace{0.3cm}

\textbf{Doppler radar:} Measures speed using frequency shift of reflected waves.
\note{- Microwave ovens use 2.45 GHz frequency\\\\
- Right amount of energy to rotate water molecules\\\\
- Polar molecules have partial charge separation\\\\
- Rotation creates heat through friction\\\\
- Radar reflects microwaves off objects\\\\
- Doppler shift measures speed - same as with sound}
\end{frame}

\begin{frame}
\frametitle{15.1 Infrared Radiation}
\begin{exampleblock}{Real-World: Heat}
\begin{itemize}
\item What we feel as radiant heat
\item Night-vision goggles detect body heat
\item Remote controls use IR signals
\end{itemize}
\end{exampleblock}

\pause
\vspace{0.3cm}

\begin{alertblock}{Misconception Alert}
Heat waves are no different from other EM waves. We feel them as heat because their frequency interacts with our bodies to create thermal energy.
\end{alertblock}
\note{- Infrared means below red frequency\\\\
- All warm objects radiate IR\\\\
- We evolved to feel this as heat\\\\
- Night-vision goggles detect IR from body heat\\\\
- Remote controls use IR pulses as signals\\\\
[THE CONNECTION - Kinetic Archetype] "Sun's warmth on your skin is IR radiation"}
\end{frame}

\begin{frame}
\frametitle{15.1 Visible Light}
\begin{figure}
\centering
\includegraphics[width=0.8\textwidth,height=0.55\textheight,keepaspectratio]{phys11-em-spectrum-fig15-5.jpg}
\end{figure}

\pause
\textbf{Wavelengths:} 400-700 nm

\textbf{Frequencies:} $4.0 \times 10^{14}$ to $7.9 \times 10^{14}$ Hz
\note{[P0][Fig 15.5: Visible spectrum band from infrared to ultraviolet with wavelength scale] "Visible spectrum from red (760nm) to violet (420nm) - continuous gradient"\\\\
[P1] "Wavelength range 400-700nm corresponds to frequency range 4.3×10^14 to 7.5×10^14 Hz"\\\\
- Calculate: f = c/λ for red: 3×10^8 / 700×10^-9 = 4.3×10^14 Hz\\\\
- Very narrow range - only 0.0035% of full EM spectrum\\\\
- All wavelengths combined produce white light (superposition)\\\\
- Prism disperses white light into rainbow by wavelength-dependent refraction (n varies with λ)\\\\
[TEACHING HINT] Show actual prism demo - connects abstract spectrum to tangible phenomenon - students see physics in action\\\\
[THE CONNECTION] Rainbow after rain shows same dispersion - water droplets act as prisms}
\end{frame}

\begin{frame}
\frametitle{15.1 The Color Wheels}
\begin{figure}
\centering
\includegraphics[width=0.7\textwidth,height=0.5\textheight,keepaspectratio]{phys11-em-spectrum-fig15-6.jpg}
\end{figure}

\pause
\textbf{Subtractive (pigments):} Cyan, Magenta, Yellow primaries $\rightarrow$ Black

\textbf{Additive (light):} Red, Green, Blue primaries $\rightarrow$ White
\note{[P0][Fig 15.6: Overlapping circles show subtractive and additive color] "Two different color mixing systems - left panel shows pigment, right shows light"\\\\
[P1] "Subtractive: pigments absorb wavelengths. CMY primaries overlap to black because all light absorbed"\\\\
- Additive: light sources combine wavelengths. RGB primaries overlap to white because all frequencies present\\\\
[TEACHING HINT] Students confuse paint mixing with light mixing - emphasize absorption vs emission\\\\
[THE CONNECTION - Harmonic Archetype] "Like mixing sound frequencies vs filtering them"}
\end{frame}

\begin{frame}
\frametitle{15.1 Animal Color Perception}
\begin{figure}
\centering
\includegraphics[width=0.7\textwidth,height=0.5\textheight,keepaspectratio]{phys11-em-spectrum-fig15-7.jpg}
\end{figure}

\pause
\textbf{Human vision:} Three cones (red, green, blue)

\pause
\vspace{0.3cm}

\textbf{Other species:}
\begin{itemize}
\item Most primates: 3 cones (like humans) \pause
\item Dogs: 2 cones (blue and yellow, colorblind to red/green) \pause
\item Birds, reptiles, insects: 4-5 cones (more hues) \pause
\item Bees: See UV; rattlesnakes: See IR
\end{itemize}

\pause
\vspace{0.3cm}

\begin{alertblock}{The Illusion}
Bulls are colorblind to red. The matador's cape color doesn't matter - it's the motion!
\end{alertblock}
\note{[P0][Fig 15.7: Color perception bands for humans, bees, dogs] "Different species detect different wavelength ranges - evolution shapes sensory systems"\\\\
[P1] "Humans have three cones: red, green, blue - trichromatic vision"\\\\
[P2] "Most primates same as us - evolved for fruit detection"\\\\
[P3] "Dogs have two cones - dichromatic vision, colorblind to red/green"\\\\
[P4] "Birds, reptiles, insects have 4-5 cones - tetrachromatic/pentachromatic, see more hues than humans"\\\\
[P5] "Bees see UV for flower navigation, rattlesnakes see IR for warm prey detection"\\\\
[P6] [THE CONFLICT] "Bulls are colorblind to red - matador's cape movement triggers aggression, not color"\\\\
[TEACHING HINT] Use this to challenge anthropocentric view of perception - physics is universal but biology is specialized\\\\
[THE WONDER] Evolution shapes perception to survival needs}
\end{frame}

\begin{frame}
\frametitle{15.1 Ultraviolet Radiation}
\begin{exampleblock}{Real-World: Sun and Sterilization}
\begin{itemize}
\item Sunlight contains UV (causes sunburn)
\item Kills bacteria (UV sterilization)
\item Black lights, counterfeit detection
\end{itemize}
\end{exampleblock}

\pause
\vspace{0.3cm}

\begin{alertblock}{Health Hazard}
UV radiation damages cells. Higher energy than visible light. Always use sunscreen!
\end{alertblock}
\note{- UV means beyond violet frequency\\\\
- Higher energy than visible light\\\\
- Penetrates skin cells and damages DNA\\\\
- Causes sunburn and skin cancer\\\\
- Also kills bacteria - used for sterilization\\\\
[THE CONNECTION - Kinetic Archetype] "Sun feels warm but UV is invisible danger"}
\end{frame}

\begin{frame}
\frametitle{15.1 X-Rays}
\begin{figure}
\centering
\includegraphics[width=0.6\textwidth,height=0.45\textheight,keepaspectratio]{phys11-em-spectrum-fig15-8.jpg}
\end{figure}

\pause
\textbf{Very high energy, very penetrating}

\textbf{Applications:}
\begin{itemize}
\item Medical imaging (see bones)
\item Airport security scanners
\end{itemize}
\note{[P0][Fig 15.8: X-ray image of chest showing artificial heart valves, pacemaker, wires] "Medical X-ray showing high-density implants - demonstrates differential absorption"\\\\
[P1] "Very high energy (10^16-10^19 Hz), very penetrating - ionizing radiation"\\\\
- X-rays pass through low-density soft tissue (muscle, organs) - appears dark\\\\
- High-density materials (bone, metal implants) absorb X-rays - appears white/bright\\\\
- Medical imaging exploits density differences to diagnose fractures, tumors, foreign objects\\\\
- Airport security scanners use same principle to inspect luggage contents\\\\
- Ionizing radiation damages DNA - exposure minimized (ALARA principle: As Low As Reasonably Achievable)\\\\
- Lead aprons (high atomic number, dense) block X-rays - protect reproductive organs during dental/medical imaging\\\\
[TEACHING HINT] Emphasize risk-benefit analysis - medical value vs radiation exposure - develops critical thinking about technology trade-offs\\\\
[THE CONNECTION] Same EM waves as visible light, just higher frequency - only difference is photon energy}
\end{frame}

\begin{frame}
\frametitle{15.1 Gamma Rays}
\textbf{Highest energy, most penetrating EM radiation}

\pause
\vspace{0.3cm}

\textbf{Sources:}
\begin{itemize}
\item Radioactive decay
\item Nuclear reactions
\item Cosmic rays from space
\end{itemize}

\pause
\vspace{0.3cm}

\textbf{Applications:}
\begin{itemize}
\item Cancer treatment (radiation therapy)
\item Sterilization of medical equipment
\end{itemize}

\pause
\vspace{0.3cm}

\alert{Extremely dangerous - ionizing radiation damages DNA}
\note{- Gamma rays are highest energy EM radiation\\\\
- Produced by radioactive decay and nuclear reactions\\\\
- Extremely penetrating - pass through most materials\\\\
- Used to kill cancer cells in radiation therapy\\\\
- Also sterilize medical equipment\\\\
- Ionizing radiation - removes electrons from atoms\\\\
- Very dangerous at high doses}
\end{frame}

\begin{frame}
\frametitle{15.1 Maxwell's Unification}
\begin{center}
\includegraphics[width=0.5\textwidth,height=0.35\textheight,keepaspectratio]{phys11-em-spectrum-fig15-4.jpg}

\small James Clerk Maxwell (1831-1879)
\end{center}

\pause
\textbf{Achievement:} Unified electricity and magnetism into electromagnetism

\pause
\vspace{0.3cm}

\alert{Electric and magnetic forces are two manifestations of the same thing - the electromagnetic force}
\note{[P0][Fig 15.4: Maxwell and Sutton's photograph of tartan ribbon bow] "First permanent color photograph (1861) - Maxwell pioneered both EM theory and color photography technology"\\\\
[P1] "Maxwell unified electricity and magnetism into single electromagnetic theory (1865) - one of greatest achievements in physics"\\\\
[P2] "Electric and magnetic forces are two aspects of same fundamental interaction - electromagnetic force"\\\\
- Maxwell's four equations completely describe EM phenomena - predict wave propagation at speed c\\\\
- Equations predicted EM waves before Hertz experimentally verified radio waves (1887)\\\\
- First unification of fundamental forces - inspired Einstein's quest for unified field theory\\\\
- Also developed Maxwell-Boltzmann distribution for kinetic theory of gases - versatile genius\\\\
[TEACHING HINT] Emphasize predictive power of mathematics - theory preceded experimental confirmation by decades\\\\
[THE WONDER] Math is language of nature - four elegant equations unite electricity, magnetism, optics, and reveal hidden EM spectrum}
\end{frame}

\section{15.2 Behavior of EM Radiation}

\begin{frame}
\frametitle{Learning Objectives}
\begin{block}{By the end of this section, you will be able to:}
\begin{itemize}
\item \textbf{15.2:} Describe the behavior of electromagnetic radiation \pause
\item \textbf{15.2:} Solve quantitative problems involving EM radiation
\end{itemize}
\end{block}
\note{- Two objectives for behavior of EM radiation\\\\
- First: understand how EM waves behave\\\\
- Second: solve problems with c equals f lambda\\\\
- These are practical skills for physics and astronomy}
\end{frame}

\begin{frame}
\frametitle{15.2 The Universal Speed Limit}
\begin{block}{Nature's Law: Speed of Light}
\begin{center}
\Large $\boxed{\vel{c} = 3.00 \times 10^{8} \text{ m/s}}$
\end{center}
All EM radiation travels at this \vel{speed} in a vacuum. 671 million mph. Constant everywhere in the universe.
\end{block}

\pause
\vspace{0.3cm}

\textbf{Cosmic distances:}
\begin{itemize}
\item Sun to Earth: 8.3 minutes
\item Nearest star: 4.2 years
\item Nearest galaxy: 25,000 years
\end{itemize}
\note{[P0] [THE REVELATION] "c equals 3 times 10 to the 8 meters per second. 671 million mph"\\\\
- All EM radiation travels at this speed in vacuum\\\\
- Fundamental physical constant\\\\
[P1] "Light from Sun takes 8.3 minutes to reach Earth"\\\\
- Nearest star takes 4.2 years\\\\
- Nearest galaxy takes 25,000 years\\\\
[THE WONDER] When you see distant stars, you see the past}
\end{frame}

\begin{frame}
\frametitle{15.2 Light in Different Media}
\textbf{In vacuum:} $\vel{c} = 3.00 \times 10^{8}$ m/s

\pause
\vspace{0.3cm}

\textbf{In other materials (slower):}
\begin{itemize}
\item Air: 99.97\% of $\vel{c}$
\item Water: 75\% of $\vel{c}$
\item Diamond: 41\% of $\vel{c}$
\end{itemize}

\pause
\vspace{0.3cm}

When light changes \vel{speed} at boundary, it changes direction. This is called \textbf{refraction}.
\note{- In vacuum, light travels at c\\\\
- In materials, light slows down\\\\
- Air: barely slower\\\\
- Water: three-fourths speed\\\\
- Diamond: less than half speed\\\\
- Change of speed causes bending - refraction\\\\
- Greater speed difference, more bending}
\end{frame}

\begin{frame}
\frametitle{15.2 Thin-Film Interference}
\begin{figure}
\centering
\includegraphics[width=0.6\textwidth,height=0.45\textheight,keepaspectratio]{phys11-em-spectrum-fig15-9.jpg}
\end{figure}

\pause
\textbf{Rainbow colors from:} Soap bubbles, oil slicks, CDs

\textbf{Cause:} Light reflects from top and bottom of thin film, waves interfere
\note{[P0][Fig 15.9: Diagram shows incident light, two reflection paths, and observer] "Light through thin film - multiple reflection paths create interference"\\\\
[P1] "Rainbow colors from constructive/destructive interference between paths"\\\\
- Ray 1 reflects from top surface of film (n2)\\\\
- Ray 2 penetrates, reflects from bottom surface, exits\\\\
- Path difference creates phase shift - wavelength-dependent\\\\
- Constructive interference for some colors, destructive for others\\\\
- Film thickness must be comparable to wavelength (nanometers)\\\\
[TEACHING HINT] Connect to wave superposition from earlier chapters - same principle, different medium\\\\
[THE CONNECTION] Soap bubbles show physics in everyday beauty}
\end{frame}

\begin{frame}
\frametitle{15.2 Polarization}
\begin{figure}
\centering
\includegraphics[width=0.7\textwidth,height=0.5\textheight,keepaspectratio]{phys11-em-spectrum-fig15-10.jpg}
\end{figure}

\pause
\textbf{Polarized light:} Electric field vibrates in only one direction

\textbf{Polarizing filter:} Transmits one direction, blocks others
\note{[P0][Fig 15.10: Two panels showing hand-generated waves passing through slits] "Mechanical analog - vertical slit passes vertical oscillations, horizontal slit passes horizontal oscillations"\\\\
[P1] "Polarized light: E field vibrates in single plane (linear polarization)"\\\\
- Panel (a): Vertical slit transmits vertically-polarized wave - oscillation direction aligned with slit\\\\
- Panel (b): Horizontal slit transmits horizontally-polarized wave\\\\
- Perpendicular polarizer blocks wave completely - no transmission\\\\
- For EM waves, polarization defined by E field oscillation direction (B always perpendicular to E)\\\\
- Polarizing filters contain aligned molecular chains - act as electromagnetic "slits"\\\\
[TEACHING HINT] Demo with rope through cardboard slit - kinesthetic learning makes abstract concept concrete\\\\
- Polarizing sunglasses exploit this: block horizontal glare, pass vertical information\\\\
[THE CONNECTION - Digital Archetype] "Like filter allowing only packets with specific header orientation - selective transmission"}
\end{frame}

\begin{frame}
\frametitle{15.2 Polarized Sunglasses}
\begin{figure}
\centering
\includegraphics[width=0.7\textwidth,height=0.5\textheight,keepaspectratio]{phys11-em-spectrum-fig15-11.jpg}
\end{figure}

\pause
\textbf{How they work:} Block horizontally polarized light (glare from water/glass)

\textbf{Result:} Reduced glare, clearer vision
\note{[P0][Fig 15.11: Two photos of river bed - left hazy with glare, right clear with filter] "Dramatic comparison showing polarization effect on reflected glare"\\\\
[P1] "Polarizing filter blocks horizontally-polarized reflected light - left panel shows hazy interference, right shows clear pebbles"\\\\
- Light reflecting off water/glass surfaces becomes preferentially polarized horizontally (Brewster's angle)\\\\
- Polarizing lenses oriented vertically block this horizontal polarization\\\\
- Vertical component of scattered light passes through - preserves useful information\\\\
- Result: glare eliminated, underwater features visible, colors enhanced\\\\
[TEACHING HINT] Demonstrate with actual polarizing filters - rotate to show intensity change, stack two at 90° to show complete blocking\\\\
- Especially effective on water, snow, road surfaces, car windshields\\\\
[THE CONNECTION] Same principle as antenna orientation in radio communication}
\end{frame}

\begin{frame}
\frametitle{Attempt: Decoding Yellow Light}
\begin{exampleblock}{The Challenge (3 min, silent)}
Yellow light has a \wavelen{wavelength} of $\wavelen{\lambda} = 6.00 \times 10^{-7}$ m.

\vspace{0.3cm}

\textbf{Given:}
\begin{itemize}
\item $\wavelen{\lambda} = 6.00 \times 10^{-7}$ m
\item $\vel{c} = 3.00 \times 10^{8}$ m/s
\end{itemize}

\textbf{Find:} \freq{Frequency} $\freq{f}$ in Hz

\vspace{0.3cm}

\textit{Can you calculate the frequency? Work silently.}
\end{exampleblock}
\note{[THE CHALLENGE] Can they decode the frequency of yellow light?\\\\
[SAY] "Try this on your own. It's okay to get stuck."\\\\
[TIMING] 3-4 min SILENT individual work\\\\
[CIRCULATE] Note who rearranges equation correctly\\\\
[WATCH FOR] Students mixing up c, f, and lambda\\\\
[DON'T HELP] Let them struggle - learning happens in Compare}
\end{frame}

\begin{frame}
\frametitle{Compare: Wave Equation}
\textbf{Turn and talk (2 min):}

\vspace{0.3cm}

\begin{enumerate}
\item What equation relates $\vel{c}$, $\freq{f}$, and $\wavelen{\lambda}$?
\item How did you rearrange to solve for $\freq{f}$?
\item Did you divide or multiply?
\end{enumerate}

\vspace{0.5cm}

\pause
\alert{Name wheel:} One pair share your approach (not your answer).
\note{[TIMING] 2-3 min pair discussion\\\\
[CIRCULATE] Listen for common approaches\\\\
[CHECK] Name wheel: call a pair to share approach\\\\
[EXPECTED APPROACH] c equals f lambda, rearrange to f equals c over lambda\\\\
[COMMON ERROR] Multiplying instead of dividing}
\end{frame}

\begin{frame}
\frametitle{Reveal: Frequency of Yellow Light}
\textbf{Self-correct in a different color:}

\vspace{0.3cm}

\textbf{Equation:} $\vel{c} = \freq{f}\wavelen{\lambda}$

\pause
\vspace{0.2cm}

\textbf{Rearrange:} $\freq{f} = \frac{\vel{c}}{\wavelen{\lambda}}$

\pause
\vspace{0.2cm}

\textbf{Substitute:} $\freq{f} = \frac{3.00 \times 10^{8} \text{ m/s}}{6.00 \times 10^{-7} \text{ m}}$

\pause
\vspace{0.2cm}

$$\boxed{\freq{f} = 5.00 \times 10^{14} \text{ Hz}}$$

\pause
\textbf{Check:} $10^{14}$ Hz is in visible range. Reasonable!
\note{[P0] [Fig 15.5: A small part of] "Self-correct in a different color"\\\\
[P1] [ALGEBRA] "c equals f lambda"\\\\
[P2] "Rearrange: f equals c over lambda"\\\\
[P3] "Substitute: 3 times 10 to the 8 divided by 6 times 10 to the negative 7"\\\\
[P4] [ANSWER] "f equals 5 times 10 to the 14 Hz - visible light frequency"\\\\
[THE WONDER] Same equation works for all EM radiation - radio to gamma}
\end{frame}

\begin{frame}
\frametitle{15.2 Illuminance: Light Intensity}
\textbf{Luminous flux \power{P}:} Rate light radiates from source (lumens, lm)

\pause
\vspace{0.3cm}

\textbf{Illuminance:} Lumens per square meter (lux, lx)

\pause
\vspace{0.3cm}

\begin{block}{Universal Law: Inverse Square Law}
\begin{center}
\Large $\boxed{\text{Illuminance} = \frac{\power{P}}{4\pi \disp{r}^{2}}}$
\end{center}
Light \intensity{intensity} decreases with square of \disp{distance}.
\end{block}
\note{[P0] "Luminous flux P: rate light radiates - measured in lumens"\\\\
[P1] "Illuminance: lumens per square meter - measured in lux"\\\\
[P2] [THE REVELATION] "Illuminance equals P over 4 pi r squared"\\\\
- Light spreads in all directions\\\\
- Double distance, one-fourth intensity\\\\
- Triple distance, one-ninth intensity\\\\
[THE WONDER] Same inverse-square law as gravity}
\end{frame}

\begin{frame}
\frametitle{Attempt: Reading Light}
\begin{exampleblock}{The Challenge (3 min, silent)}
A floor lamp has luminous flux of 2000 lm. You hold a book 2.00 m from the bulb.

\vspace{0.3cm}

\textbf{Given:}
\begin{itemize}
\item $\power{P} = 2000$ lm
\item $\disp{r} = 2.00$ m
\item $\pi = 3.14$
\end{itemize}

\textbf{Find:} Illuminance in lux

\vspace{0.3cm}

\textit{Can you calculate the illuminance? Work silently.}
\end{exampleblock}
\note{[THE CHALLENGE] Can they calculate illuminance for reading?\\\\
[SAY] "Try this on your own. It's okay to get stuck."\\\\
[TIMING] 3-4 min SILENT individual work\\\\
[CIRCULATE] Note who squares radius correctly\\\\
[WATCH FOR] Students forgetting to square r or multiply by 4 pi\\\\
[DON'T HELP] Let them struggle}
\end{frame}

\begin{frame}
\frametitle{Compare: Inverse Square Law}
\textbf{Turn and talk (2 min):}

\vspace{0.3cm}

\begin{enumerate}
\item What equation did you use?
\item What goes in the denominator?
\item Did you square the \disp{radius}?
\end{enumerate}

\vspace{0.5cm}

\pause
\alert{Name wheel:} One pair share your approach (not your answer).
\note{[TIMING] 2-3 min pair discussion\\\\
[CIRCULATE] Listen for common approaches\\\\
[CHECK] Name wheel: call a pair to share approach\\\\
[EXPECTED APPROACH] Illuminance equals P over 4 pi r squared\\\\
[COMMON ERROR] Forgetting to square radius}
\end{frame}

\begin{frame}
\frametitle{Reveal: Illuminance Calculation}
\textbf{Self-correct in a different color:}

\vspace{0.3cm}

\textbf{Equation:} Illuminance $= \frac{\power{P}}{4\pi \disp{r}^{2}}$

\pause
\vspace{0.2cm}

\textbf{Substitute:} Illuminance $= \frac{2000 \text{ lm}}{4(3.14)(2.00)^{2} \text{ m}^{2}}$

\pause
\vspace{0.2cm}

\textbf{Calculate:} Illuminance $= \frac{2000}{50.24}$

\pause
\vspace{0.2cm}

$$\boxed{\text{Illuminance} = 39.8 \text{ lx}}$$

\pause
\textbf{Check:} At 3 m, illuminance drops to 17.7 lx. Light fades rapidly!
\note{[P0] "Self-correct in a different color"\\\\
[P1] [ALGEBRA] "Illuminance equals P over 4 pi r squared"\\\\
[P2] "Substitute: 2000 over 4 times 3.14 times 2 squared"\\\\
[P3] "2000 over 50.24"\\\\
[P4] [ANSWER] "39.8 lux - decent reading light"\\\\
[THE WONDER] Move twice as far, get one-fourth the light - parents tell kids don't read in dim light}
\end{frame}

\section{Summary}

\begin{frame}
\frametitle{What You Now Know}
\begin{block}{The Revelations}
\begin{enumerate}
\item EM spectrum = radio to gamma rays (same physics) \pause
\item Visible light = tiny sliver of what exists \pause
\item E and B fields oscillate perpendicular to propagation \pause
\item $\vel{c} = 3.00 \times 10^{8}$ m/s in vacuum (constant) \pause
\item Higher \freq{frequency} = higher energy = more penetrating \pause
\item Polarization = E field vibrates in one direction \pause
\item Illuminance decreases with inverse square of \disp{distance}
\end{enumerate}
\end{block}
\note{[P0] "Seven revelations today"\\\\
[P1] "EM spectrum from radio to gamma rays - same physics"\\\\
[P2] "Visible light is tiny sliver of what exists"\\\\
[P3] "E and B fields oscillate perpendicular to propagation"\\\\
[P4] "c equals 3 times 10 to the 8 meters per second"\\\\
[P5] "Higher frequency means higher energy and more penetrating"\\\\
[P6] "Polarization: E field vibrates in one direction"\\\\
[P7] "Illuminance decreases with inverse square of distance"\\\\
[THE WONDER] You now see the invisible universe - same laws everywhere}
\end{frame}

\begin{frame}[shrink]
\frametitle{Key Equations}
\begin{align}
\vel{c} &= \freq{f}\wavelen{\lambda} \\
\vel{c} &= 3.00 \times 10^{8} \text{ m/s} \\
\freq{f} &= \frac{\vel{c}}{\wavelen{\lambda}} \\
\wavelen{\lambda} &= \frac{\vel{c}}{\freq{f}} \\
\text{Illuminance} &= \frac{\power{P}}{4\pi \disp{r}^{2}}
\end{align}
\note{- Five foundational formulas\\\\
- c equals f lambda: relate speed, frequency, wavelength\\\\
- c equals 3 times 10 to the 8 meters per second: speed of light\\\\
- Rearrange to solve for f or lambda\\\\
- Illuminance: inverse square law\\\\
- Know when to use each}
\end{frame}

\begin{frame}
\frametitle{Homework}
\begin{center}
\Large
Complete the assigned problems\\[0.3cm]
posted on the LMS
\end{center}
\note{[SAY] "Homework is posted on the LMS"\\\\
[TIMING] Due date: check LMS\\\\
[CHECK] Questions before we end?\\\\
[TRANSITION] Next class: Applications of EM radiation}
\end{frame}

\end{document}
