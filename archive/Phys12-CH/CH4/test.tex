\documentclass{beamer}
\usepackage[utf8]{inputenc}
\usepackage{amsmath}
\usepackage{graphicx}
\usepackage{xcolor}
\usepackage{tikz}

\usetheme{Madrid}
\usecolortheme{default}

% Define custom colors inspired by Star Trek DS9
\definecolor{ds9blue}{RGB}{25,25,112} % Midnight Blue
\definecolor{ds9gold}{RGB}{218,165,32} % Goldenrod
\definecolor{ds9grey}{RGB}{105,105,105} % Dim Gray
\definecolor{ds9red}{RGB}{178,34,34} % Firebrick

% Customize the colors
\setbeamercolor{title}{fg=ds9gold}
\setbeamercolor{frametitle}{bg=ds9blue, fg=white}
\setbeamercolor{block title}{bg=ds9gold, fg=black}
\setbeamercolor{block body}{bg=ds9grey!20, fg=black}
\setbeamercolor{section in toc}{fg=ds9gold}
\setbeamercolor{subsection in toc}{fg=ds9gold!70}
\setbeamercolor{footline}{bg=ds9blue, fg=white}
\setbeamercolor{author in head/foot}{fg=white}
\setbeamercolor{date in head/foot}{fg=white}
\setbeamercolor{title in head/foot}{fg=white}

% Title page configuration
\title{Lesson CH:4 Problems and Solutions}
\subtitle{Newton's Laws of Motion}
\author{Mr. Gullo}
\date{September 2024}

% Table of contents at the beginning of each section
\AtBeginSection[]
{
  \begin{frame}
    \frametitle{Table of Contents}
    \tableofcontents[currentsection]
  \end{frame}
}

\begin{document}

\frame{\titlepage}

\section{Overview}

\begin{frame}
\frametitle{Newton's Three Laws of Motion}
\begin{itemize}
    \item \textbf{First Law: The Law of Inertia}
    \begin{itemize}
        \item An object at rest stays still.
        \item An object in motion keeps moving.
        \item This continues unless an outside force acts on the object.
        \item Example: A book on a table stays still until you push it.
    \end{itemize}
    
    \item \textbf{Second Law: Force, Mass, and Acceleration}
    \begin{itemize}
        \item Force = Mass × Acceleration
        \item More force → More acceleration
        \item More mass → Less acceleration
        \item The direction of acceleration is the same as the force.
        \item Example: Pushing a shopping cart (try it with empty vs. full cart)
    \end{itemize}
    
    \item \textbf{Third Law: Action and Reaction}
    \begin{itemize}
        \item When one object pushes another, the second object pushes back.
        \item These forces are equal in strength but opposite in direction.
        \item Example: When you jump, you push down on the ground, and the ground pushes you up.
    \end{itemize}
\end{itemize}
\end{frame}

\begin{frame}
\frametitle{Key Concepts}
\begin{itemize}
    \item Acceleration and its relationship to force and mass
    \item External forces vs. internal forces
    \item Weight and free fall
    \item Friction and its effects on motion
    \item Normal force and tension
    \item Problem-solving strategies:
    \begin{itemize}
        \item Drawing free-body diagrams
        \item Resolving forces into components
        \item Applying Newton's second law
    \end{itemize}
    \item Applications in various real-world scenarios
\end{itemize}
\end{frame}

\begin{frame}
\frametitle{Newton's Second Law in Detail}
\begin{itemize}
    \item Acceleration ($\mathbf{a}$) is defined as a change in velocity, either in magnitude or direction, or both.
    \item Newton's second law of motion states that the acceleration of a system is directly proportional to and in the same direction as the net external force acting on the system, and inversely proportional to its mass.
    \item In equation form:
    \[\mathbf{a}=\frac{\mathbf{F}_{\text{net}}}{m}\]
    \item Often written in the more familiar form:
    \[\mathbf{F}_{\text{net}}=m\mathbf{a}\]
    \item $\mathbf{F}_{\text{net}}$ represents the net external force acting on the system.
    \item $m$ is the mass of the system.
\end{itemize}
\end{frame}

\begin{frame}
\frametitle{Weight and Free Fall}
\begin{itemize}
    \item Weight ($\mathbf{w}$) is defined as the force of gravity acting on an object of mass $m$.
    \item The object experiences an acceleration due to gravity $\mathbf{g}$:
    \[\mathbf{w}=m\mathbf{g}\]
    \item If the only force acting on an object is due to gravity, the object is in free fall.
    \item On Earth, the acceleration due to gravity is approximately 9.8 m/s² downward.
\end{itemize}
\end{frame}

\begin{frame}
\frametitle{Forces on an Inclined Plane}
When objects rest on an inclined plane that makes an angle $\theta$ with the horizontal surface:
\begin{itemize}
    \item The weight of the object can be resolved into components that act perpendicular ($\mathbf{w}_{\perp}$) and parallel ($\mathbf{w}_{\|}$) to the surface of the plane.
    \item These components can be calculated using:
    \begin{align*}
    w_{\|} &= w \sin(\theta) = mg \sin(\theta) \\
    w_{\perp} &= w \cos(\theta) = mg \cos(\theta)
    \end{align*}
    \item $w_{\|}$ is the component causing the object to slide down the plane.
    \item $w_{\perp}$ is the component balanced by the normal force from the plane.
\end{itemize}
\end{frame}

\begin{frame}
\frametitle{Tension and Normal Force}
\begin{itemize}
    \item Tension ($T$) is the pulling force that acts along a stretched flexible connector, such as a rope or cable.
    \item When a rope supports the weight of an object at rest:
    \[T = mg\]
    \item Normal force ($\mathbf{N}$) is the supporting force applied by a surface to an object that is at rest on the surface.
    \item On a horizontal, non-accelerating surface:
    \[N = mg\]
    \item The normal force is always perpendicular to the surface.
\end{itemize}
\end{frame}

\begin{frame}
\frametitle{Problem-Solving Strategy}
To solve problems involving Newton's laws of motion:
\begin{enumerate}
    \item Draw a sketch of the problem.
    \item Identify known and unknown quantities, and the system of interest.
    \item Draw a free-body diagram:
    \begin{itemize}
        \item Represent the object as a dot.
        \item Draw vectors for all forces acting on the object.
        \item Resolve non-horizontal/vertical vectors into components.
    \end{itemize}
    \item Apply Newton's second law in the horizontal and vertical directions:
    \begin{itemize}
        \item If no acceleration in a direction: $F_{\text{net}} = 0$
        \item If acceleration present: $F_{\text{net}} = ma$
    \end{itemize}
    \item Solve the resulting equations.
    \item Check your answer: Is it reasonable? Are the units correct?
\end{enumerate}
\end{frame}

\section{4.3 NEWTON'S SECOND LAW OF MOTION: CONCEPT OF A SYSTEM}

\begin{frame}
\frametitle{Problem 4}
\begin{itemize}
    \item Since astronauts in orbit are apparently weightless, a clever method of measuring their masses is needed to monitor their mass gains or losses to adjust diets. One way to do this is to exert a known force on an astronaut and measure the acceleration produced. Suppose a net external force of 50.0 N is exerted and the astronaut's acceleration is measured to be $0.893 \mathrm{~m} / \mathrm{s}^{2}$.
    \item (a) Calculate her mass.
    \item (b) By exerting a force on the astronaut, the vehicle in which they orbit experiences an equal and opposite force. Discuss how this would affect the measurement of the astronaut's acceleration. Propose a method in which recoil of the vehicle is avoided.
\end{itemize}
\end{frame}

\begin{frame}
\frametitle{Problem 4 - Solution}
Solution:
\begin{itemize}
    \item[(a)] $m = \frac{\text{net} F}{a} = \frac{50.0 \mathrm{~N}}{0.893 \mathrm{~m} / \mathrm{s}^{2}} = 56.0 \mathrm{kg}$
    \item[(b)] $a_{\text{meas}} = a_{\text{astro}} + a_{\text{ship}}$, where: $a_{\text{ship}} = \frac{m_{\text{astro}} a_{\text{astro}}}{m_{\text{ship}}}$
    \item If the force could be exerted on the astronaut by another source (other than the spaceship), then the spaceship would not experience a recoil.
\end{itemize}
\end{frame}

\begin{frame}
\frametitle{Derivation of Ship's Acceleration}
\small
\begin{itemize}
    \item[1)] Newton's Third Law: Force on astronaut equals negative force on ship
    $$F_{\text{on astro}} = -F_{\text{on ship}}$$
    
    \item[2)] Express forces using Newton's Second Law (F = ma):
    $$m_{\text{astro}} a_{\text{astro}} = -m_{\text{ship}} a_{\text{ship}}$$
    
    \item[3)] Rearrange to isolate $a_{\text{ship}}$:
    \begin{align*}
        -m_{\text{ship}} a_{\text{ship}} &= m_{\text{astro}} a_{\text{astro}} \\
        m_{\text{ship}} a_{\text{ship}} &= -m_{\text{astro}} a_{\text{astro}} \\
        a_{\text{ship}} &= -\frac{m_{\text{astro}} a_{\text{astro}}}{m_{\text{ship}}}
    \end{align*}
    
    \item[4)] Drop negative sign for magnitude:
    $$a_{\text{ship}} = \frac{m_{\text{astro}} a_{\text{astro}}}{m_{\text{ship}}}$$
    
    \item Interpretation: Ship's acceleration is proportional to astronaut's mass and acceleration, inverse to ship's mass.
\end{itemize}
\end{frame}

\section{4.4 NEWTON'S THIRD LAW OF MOTION: SYMMETRY IN FORCES}

\begin{frame}
\frametitle{Problem 16}
A rugby player is being pushed backward by an opposing player who is exerting a force of 800 N on him. The mass of the losing player plus equipment is 90.0 kg, and he is accelerating at $1.20 \mathrm{~m} / \mathrm{s}^{2}$ backward.
\begin{itemize}
    \item[(a)] What is the force of friction between the losing player's feet and the grass?
    \item[(b)] What force does the winning player exert on the ground to move forward if his mass plus equipment is 110 kg?
    \item[(c)] Draw a sketch of the situation showing the system of interest used to solve each part. For this situation, draw a free-body diagram and write the net force equation.
\end{itemize}
\end{frame}

\begin{frame}
\frametitle{Problem 16 - Part (a)}
\begin{block}{Question}
What is the force of friction between the losing player's feet and the grass?
\end{block}
\begin{block}{Knowns and Unknowns}
\textbf{Knowns:}
\begin{itemize}
    \item $F_{\text{opposing}} = 800$ N
    \item $m_{\text{losing}} = 90.0$ kg
    \item $a = 1.20$ m/s² backward
\end{itemize}
\textbf{Unknown:}
\begin{itemize}
    \item $F_{\text{friction}}$ (force of friction)
\end{itemize}
\end{block}
\begin{block}{Solution}
net $F = F - f = ma$
\begin{equation*}
f = F - ma = 800 \mathrm{~N} - (90.0 \mathrm{kg})(1.20 \mathrm{~m} / \mathrm{s}^{2}) = \underline{692 \mathrm{~N}}
\end{equation*}
\end{block}
\end{frame}

\begin{frame}
\frametitle{Problem 16 - Part (b)}
\begin{block}{Question}
What force does the winning player exert on the ground to move forward if his mass plus equipment is 110 kg?
\end{block}
\begin{block}{Knowns and Unknowns}
\textbf{Knowns:}
\begin{itemize}
    \item $m_{\text{winning}} = 110$ kg
    \item $a = 1.20$ m/s² (same as losing player, in opposite direction)
    \item $F_{\text{friction}} = 692$ N (calculated in part a)
\end{itemize}
\textbf{Unknown:}
\begin{itemize}
    \item $F_{\text{ground}}$ (force exerted on the ground)
\end{itemize}
\end{block}
\begin{block}{Solution}
\begin{equation*}
F = ma + f = (110 \mathrm{kg} + 90.0 \mathrm{kg})(1.20 \mathrm{~m} / \mathrm{s}^{2}) + 692 \mathrm{~N} = 932 \mathrm{~N}
\end{equation*}
\end{block}
\end{frame}


\begin{frame}
\frametitle{Problem 16 - Solution (c)}
\begin{itemize}
    \item[(c)]
\end{itemize}
\includegraphics[width=0.45\textwidth]{https://cdn.mathpix.com/cropped/2024_10_18_cf9b0039bf1ee1cceb9dg-06.jpg?height=549&width=476&top_left_y=1493&top_left_x=429}
\includegraphics[width=0.45\textwidth]{https://cdn.mathpix.com/cropped/2024_10_18_cf9b0039bf1ee1cceb9dg-06.jpg?height=581&width=375&top_left_y=1474&top_left_x=994}
\end{frame}


\begin{frame}
\frametitle{Problem 17}
Two teams of nine members each engage in a tug of war. Each of the first team's members has an average mass of 68 kg and exerts an average force of 1350 N horizontally. Each of the second team's members has an average mass of 73 kg and exerts an average force of 1365 N horizontally.
\begin{itemize}
    \item[(a)] What is the acceleration of the two teams?
    \item[(b)] What is the tension in the section of rope between the teams?
\end{itemize}
\end{frame}

\section{4.5 Normal, Tension, and Other Examples of Forces}

\begin{frame}
\frametitle{Problem 17 - Solution (a)}
\includegraphics[width=0.8\textwidth]{https://cdn.mathpix.com/cropped/2024_10_18_cf9b0039bf1ee1cceb9dg-07.jpg?height=448&width=1359&top_left_y=600&top_left_x=405}

net $F = Ma$; $f_1 = 1350 \mathrm{~N}$; $f_2 = 1365 \mathrm{~N}$

$9(f_2 - f_1) = 9(m_1 + m_2)a$; $m_1 = 68 \mathrm{~kg}$; $m_2 = 73 \mathrm{~kg}$

\begin{equation*}
a = \frac{f_2 - f_1}{m_1 + m_2} = \frac{1365 \mathrm{~N} - 1350 \mathrm{~N}}{68 \mathrm{~kg} + 73 \mathrm{~kg}} = 0.1064 \mathrm{~m} / \mathrm{s}^{2} = \underline{0.11 \mathrm{~m} / \mathrm{s}^{2}}
\end{equation*}

Thus, the heavy team wins. Note that the difference $1365 \mathrm{~N} - 1350 \mathrm{~N} = 15 \mathrm{~N}$ limits the answer to two significant figures.
\end{frame}

\begin{frame}
\frametitle{Problem 17 - Solution (b)}
\includegraphics[width=0.8\textwidth]{https://cdn.mathpix.com/cropped/2024_10_18_cf9b0039bf1ee1cceb9dg-07.jpg?height=172&width=806&top_left_y=1502&top_left_x=388}

$T - 9f_1 = 9m_1a \Rightarrow T = 9m_1a + 9f_1$

\begin{equation*}
T = 9(68 \mathrm{~kg})(0.1064 \mathrm{~m} / \mathrm{s}^{2}) + 9(1350 \mathrm{~N}) = \underline{1.2 \times 10^{4} \mathrm{~N}}
\end{equation*}
\end{frame}

\begin{frame}
\frametitle{Problem 17 - Explanation}
Explanation:
\begin{itemize}
    \item[(a)] We use Newton's Second Law for the entire system: F = Ma
    \begin{itemize}
        \item The net force is the difference between the forces of the two teams
        \item We divide by the total mass to find acceleration
    \end{itemize}
    \item[(b)] We consider the forces on one team
    \begin{itemize}
        \item Use Newton's Second Law: T - 9f_1 = 9m_1a
        \item Solve for T and substitute known values
    \end{itemize}
    \item The algebra involves simple addition, subtraction, multiplication, and division. There's no trigonometry in this problem as all forces are in one dimension.
\end{itemize}
\end{frame}

\begin{frame}
\frametitle{Problem 28}
Commercial airplanes are sometimes pushed out of the passenger loading area by a tractor.
\begin{itemize}
    \item[(a)] An 1800-kg tractor exerts a force of $1.75 \times 10^{4} \mathrm{~N}$ backward on the pavement, and the system experiences forces resisting motion that total 2400 N. If the acceleration is $0.150 \mathrm{~m} / \mathrm{s}^{2}$, what is the mass of the airplane?
    \item[(b)] Calculate the force exerted by the tractor on the airplane, assuming 2200 N of the friction is experienced by the airplane.
    \item[(c)] Draw two sketches showing the systems of interest used to solve each part, including the free-body diagrams for each.
\end{itemize}
\end{frame}

\begin{frame}
\frametitle{Problem 28 - Solution (a)}
Solution:
\begin{itemize}
    \item[(a)] net $F = Ma = (m_a + m_t)a = F - f$, so that: $m_a = \frac{F - f}{a} - m_t$
    \begin{equation*}
    m_a = \frac{1.75 \times 10^{4} \mathrm{~N} - 2400 \mathrm{~N}}{0.150 \mathrm{~m} / \mathrm{s}^{2}} - 1800 \mathrm{kg} = \underline{9.89 \times 10^{4} \mathrm{kg}}
    \end{equation*}
\end{itemize}
\end{frame}

\section{4.6 PROBLEM-SOLVING STRATEGIES}

\begin{frame}
\frametitle{Problem 28 - Solution (b) and (c)}
\begin{itemize}
    \item[(b)] net $F = F' - f' = m_a a$
    \begin{equation*}
    F' = m_a a + f' = (9.89 \times 10^{4} \mathrm{kg})(0.150 \mathrm{~m} / \mathrm{s}^{2}) + 2200 \mathrm{~N} = \underline{1.70 \times 10^{4} \mathrm{~N}}
    \end{equation*}
    \item[(c)]
\end{itemize}
\includegraphics[width=0.8\textwidth]{https://cdn.mathpix.com/cropped/2024_10_18_cf9b0039bf1ee1cceb9dg-13.jpg?height=346&width=1012&top_left_y=716&top_left_x=421}
\end{frame}

\begin{frame}
\frametitle{Problem 28 - Explanation}
Explanation:
\begin{itemize}
    \item[(a)] We use Newton's Second Law for the entire system: F = Ma
    \begin{itemize}
        \item Substitute the given force, friction, and acceleration
        \item Solve for the airplane's mass
    \end{itemize}
    \item[(b)] We use Newton's Second Law for just the airplane: F' - f' = m_a a
    \begin{itemize}
        \item Solve for F' and substitute known values
    \end{itemize}
    \item The algebra involves simple addition, subtraction, multiplication, and division. There's no trigonometry in this problem as all forces are in one dimension.
\end{itemize}
\end{frame}

\begin{frame}

\frametitle{Conclusion}
\begin{itemize}
\item We've explored various applications of Newton's Laws of Motion:
\begin{itemize}
\item Measuring mass in weightless environments
\item Analyzing forces in sports (rugby and tug of war)
\item Calculating forces in aircraft towing
\end{itemize}
\item Key takeaways:
\begin{itemize}
\item Newton's Second Law (F = ma) is crucial for solving these problems
\item Consider all forces acting on a system
\item Break down complex situations into simpler components
\item Pay attention to vector directions and sign conventions
\end{itemize}
\item Practice solving similar problems to reinforce your understanding of Newton's Laws
\end{itemize}
\end{frame}
\end{document}