\documentclass[12pt]{article}
\usepackage{geometry}
\usepackage{graphicx}
\usepackage{enumitem}
\usepackage{booktabs}
\usepackage{hyperref}
\usepackage{float}
\usepackage{multirow}
\usepackage{array}
\usepackage[table]{xcolor}
\usepackage{colortbl}

\geometry{margin=1in}

\title{Physics in Place: A Cultural and Scientific Analysis\\
\large Written Research Assignment}
\author{Nanmo Physics}
\date{March, 2025}

\begin{document}
\maketitle

\section*{Purpose}
To explore the intersection of physics principles and cultural significance in your local environment through written analysis and visual documentation. This assignment recognizes that understanding place—as both a physical and cultural entity—is fundamental to Indigenous knowledge systems and perspectives, particularly First Peoples' holistic approaches to understanding the natural world.

\section*{Required Sections}

\subsection*{1. Introduction }
\begin{itemize}
    \item Overview of chosen place/structure/system
    \item Brief historical context
    \item Thesis statement connecting physics concepts from our course to cultural significance
    \item Acknowledgment of whose traditional territories the place exists within
\end{itemize}

\subsection*{2. Physical Analysis }
\begin{itemize}
    \item Detailed explanation of at least two physics principles from our course present
    \item Supporting calculations and measurements
    \item Technical diagrams or photographs with detailed captions explaining physics concepts
    \item Analysis of how physical properties contribute to the structure's function
\end{itemize}

\subsection*{3. Cultural Significance }
\begin{itemize}
    \item Historical background and importance to local community
    \item Traditional knowledge and practices associated with the place
    \item Documentation of community perspectives (with appropriate permissions)
    \item Analysis of how physics understanding contributed to traditional practices
    \item Discussion of First Peoples' relationships to this place (where applicable)
\end{itemize}

\subsection*{4. Integration Analysis }
\begin{itemize}
    \item Discussion of how physical properties and cultural significance intersect
    \item Examination of traditional knowledge systems and modern physics understanding
    \item Analysis of how place contributes to community identity
    \item Reflection on different ways of knowing and understanding the natural world
\end{itemize}

\subsection*{5. Reflection }
\begin{itemize}
    \item Personal insights on the relationship between physics and culture
    \item Discussion of learning experience
    \item Broader implications for understanding physics in cultural context
    \item How this project has influenced your own connection to place
\end{itemize}


\newpage

\section*{Community Consultation Guidelines}

\subsection*{Approaching Community Members}
\begin{enumerate}
    \item \textbf{Research Before Contacting}: Learn about appropriate protocols for the specific community you wish to consult with before making contact.
    
    \item \textbf{Proper Introductions}: Begin by introducing yourself, your school, the purpose of your project, and how their knowledge will be used and credited.
    
    \item \textbf{Respect for Knowledge Keepers}: When approaching Elders or knowledge keepers, follow community protocols which may include:
    \begin{itemize}
        \item Bringing a small gift as a token of respect (e.g., tea, tobacco, or other culturally appropriate offerings)
        \item Speaking in a respectful manner and listening more than speaking
        \item Being patient and allowing time for stories and context
    \end{itemize}
    
    \item \textbf{Informed Consent}: Clearly explain how any information shared will be used in your project and obtain written permission using the provided consent forms.
    
    \item \textbf{Review and Approval}: Offer to share your draft with contributors before final submission to ensure accurate representation of their knowledge.
\end{enumerate}


\section*{Format Requirements}
\begin{itemize}
    \item Length: 1500-2000 words
    \item Structure: Research paper format with clear sections
    \item Visuals: Minimum 4 original photographs/diagrams with detailed captions
    \item Citations: APA format
    \item Font: 12-point Times New Roman, double-spaced
    \item Digital submission in PDF format
\end{itemize}

\section*{Research Requirements}
\begin{itemize}
    \item Minimum 4 academic sources
    \item Minimum 2 community sources (interviews, local documents, webistes)
    \item Proper documentation of all consultations
\end{itemize}



\newpage

% First page of tables
\begin{table}[t]
\renewcommand{\arraystretch}{1.5}
\begin{tabular}{>{\raggedright\arraybackslash}p{2cm}|>{\raggedright\arraybackslash}p{14cm}}
\toprule
\multicolumn{2}{l}{\textbf{Emerging}} \\
\midrule
Description & Student demonstrates basic understanding of physics concepts and cultural research methods, requiring significant guidance to complete tasks. Work shows initial attempts to connect physical and cultural elements but lacks depth and independence. \\
\midrule
Skills and Abilities & 
\begin{itemize}
    \item Identifies basic physics principles in chosen location with substantial guidance and support
    \item Conducts preliminary research using provided sources and basic documentation methods
    \item Creates simple visual documentation with basic captions that need significant revision
    \item Makes surface-level connections between physical properties and cultural significance
    \item Submits work more than 3 days late without communication, uses incorrect file formats, and leaves significant work incomplete
\end{itemize} \\
\bottomrule
\end{tabular}
\end{table}

\begin{table}[b]
\renewcommand{\arraystretch}{1.5}
\begin{tabular}{>{\raggedright\arraybackslash}p{2cm}|>{\raggedright\arraybackslash}p{14cm}}
\toprule
\multicolumn{2}{l}{\textbf{Developing}} \\
\midrule
Description & Student shows growing comprehension of both physics concepts and cultural analysis, requiring moderate guidance. Work demonstrates increasing ability to make connections and conduct independent research, though analysis remains somewhat superficial. \\
\midrule
Skills and Abilities & 
\begin{itemize}
    \item Explains fundamental physics principles with some accuracy, occasionally needing correction
    \item Conducts research using recommended sources and follows documentation guidelines with reminders
    \item Produces clear visual documentation with descriptive captions that address both physics and culture
    \item Draws meaningful connections between physical properties and cultural significance with some guidance
    \item Submits 1-2 days late or requests last-minute extensions while generally following format requirements and completing most work, though it may be rushed
\end{itemize} \\
\bottomrule
\end{tabular}
\end{table}

\clearpage % Force a page break

% Second page of tables
\begin{table}[t]
\renewcommand{\arraystretch}{1.5}
\begin{tabular}{>{\raggedright\arraybackslash}p{2cm}|>{\raggedright\arraybackslash}p{14cm}}
\toprule
\multicolumn{2}{l}{\textbf{Proficient}} \\
\midrule
Description & Student demonstrates solid understanding of physics concepts and cultural research methods, working independently with minimal guidance. Work shows thorough analysis and clear connections between physical and cultural elements. \\
\midrule
Skills and Abilities & 
\begin{itemize}
    \item Accurately analyzes and explains physics principles present in chosen location with supporting evidence
    \item Independently conducts comprehensive research using diverse academic and community sources
    \item Creates high-quality visual documentation with detailed, informative captions that integrate concepts
    \item Develops clear, well-supported connections between physical properties and cultural significance
    \item Submits on time with proper formatting, communicates about potential delays, and completes all components thoroughly
\end{itemize} \\
\bottomrule
\end{tabular}
\end{table}

\begin{table}[b]
\renewcommand{\arraystretch}{1.5}
\begin{tabular}{>{\raggedright\arraybackslash}p{2cm}|>{\raggedright\arraybackslash}p{14cm}}
\toprule
\multicolumn{2}{l}{\textbf{Extending}} \\
\midrule
Description & Student exhibits exceptional understanding of physics principles and cultural research methods, working independently and showing initiative. Work demonstrates sophisticated analysis, original insights, and seamless integration of physical and cultural elements. \\
\midrule
Skills and Abilities & 
\begin{itemize}
    \item Provides sophisticated analysis of physics principles with innovative applications and insights
    \item Conducts extensive research that exceeds requirements, incorporating unique perspectives and sources
    \item Produces outstanding visual documentation with comprehensive captions that enhance understanding
    \item Develops complex, nuanced connections between physical properties and cultural significance
    \item Submits quality work ahead of deadlines, maintains clear communication, prepares well for known absences, and creates systems for tracking requirements
\end{itemize} \\
\bottomrule
\end{tabular}
\end{table}
\end{document}