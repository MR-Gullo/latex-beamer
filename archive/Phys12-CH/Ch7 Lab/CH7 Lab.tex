\documentclass[12pt]{article}
\usepackage{amsmath}
\usepackage{graphicx}
\usepackage{booktabs}
\usepackage{siunitx}
\usepackage[colorlinks=true,linkcolor=blue]{hyperref}

\title{Laboratory Manual: Work and Energy}
\author{Physics Department}
\date{}

\begin{document}
\maketitle

\section{Theoretical Background}

The relationship between work and energy stands as one of physics' most elegant principles. When work is done on an object, that energy must go somewhere - typically manifesting as changes in the object's motion (kinetic energy) or position (potential energy).

\begin{quote}
\textbf{Work-Energy Theorem}: The net work done on an object equals its change in kinetic energy

$$W_{net} = \Delta KE = \frac{1}{2}mv_2^2 - \frac{1}{2}mv_1^2$$
\end{quote}

This experiment investigates this fundamental relationship using a cart-and-track system where we can precisely measure both the work done and the resulting change in kinetic energy.

\section{Equipment}
\begin{itemize}
\item Multi-purpose mechanical track system with leveling adjustment
\item Two photogate sensors
\item Cart with light blocking sheet (width = \SI{0.002}{\meter})
\item Calibrated mass set and pulley system
\item Digital timer (integrated with photogates)
\item Level indicator
\end{itemize}

\section{Experimental Setup}

\subsection{Key Parameters}
\begin{itemize}
\item Fixed distance between photogates: \SI{0.30}{\meter}
\item Cart mass (including attachments): \SI{0.225}{\kilogram}
\item Light blocking sheet width: \SI{0.002}{\meter}
\end{itemize}

\subsection{Setup Procedure}
1. Level the track using the adjustment screws and level indicator
2. Mount photogates securely at \SI{0.30}{\meter} separation
3. Attach pulley system to track end
4. Verify light blocking sheet triggers both photogates cleanly

\section{Data Collection}

For each trial:
\begin{enumerate}
\item Record the pulling force ($F$) from the hanging mass
\item Release the cart from rest
\item Record timing data ($t_1$, $t_2$) from both photogates
\item Repeat with different pulling forces (minimum 3 trials)
\end{enumerate}

\section{Analysis}

\subsection{Calculations}
For each trial, calculate:

1. \textbf{Work Done}:
   $$W = F \cdot d = F \cdot \SI{0.30}{\meter}$$

2. \textbf{Change in Kinetic Energy}:
   $$\Delta KE = \frac{m}{2}\left(\left(\frac{0.002}{t_2}\right)^2 - \left(\frac{0.002}{t_1}\right)^2\right)$$

\subsection{Data Analysis Table}
Below is a sample data table showing the measurements and calculations:

\begin{table}[htbp]
\centering
\begin{tabular}{ccccccc}
\toprule
$t_1$ (s) & $t_2$ (s) & $M$ (kg) & $s$ (m) & $F=9.8m_1$ (N) & $W=Fs$ (J) & $\Delta KE$ (J) \\
\midrule
0.07207 & 0.04399 & 0.225 & 0.30 & 0.0490 & 0.0147 & 0.0146 \\
0.04964 & 0.03095 & 0.225 & 0.30 & 0.0980 & 0.0294 & 0.0287 \\
0.04083 & 0.02542 & 0.225 & 0.30 & 0.1470 & 0.0441 & 0.0426 \\
0.03259 & 0.02140 & 0.225 & 0.30 & 0.1960 & 0.0588 & 0.0559 \\
0.03040 & 0.01965 & 0.225 & 0.30 & 0.2450 & 0.0735 & 0.0679 \\
\bottomrule
\end{tabular}
\caption{Experimental measurements and calculated values}
\end{table}

Where:
\begin{itemize}
\item $t_1$, $t_2$ are photogate timing measurements
\item $M$ is cart mass
\item $m_1$ is hanging mass
\item $s$ is distance between photogates
\item $F$ is pulling force
\item $W$ is work done
\item $KE$ is change in kinetic energy

\end{itemize}

\section{Discussion Points}

Consider these questions while conducting your analysis:

\begin{enumerate}
\item List potential sources of discrepancy between work ($W$) and change in kinetic energy ($\Delta KE$):
    \begin{itemize}
    \item Rolling friction effects
    \item Pulley system friction
    \item Air resistance
    \item Mechanical vibrations
    \item String elasticity
    \end{itemize}

\item Consider how friction would affect:
    \begin{itemize}
    \item The measured $\Delta KE$ versus calculated work
    \item Results at different velocities
    \item Measurements over varying distances
    \end{itemize}

\item Explain the advantages of photogate timing over direct velocity measurements:
    \begin{itemize}
    \item Precision considerations
    \item Measurement errors
    \item Practical limitations
    \item Data reliability
    \end{itemize}
\end{enumerate}

\section{Common Sources of Error}

\begin{itemize}
\item Track leveling imperfections
\item Friction in the pulley system
\item Air resistance
\item Timing uncertainties in photogate measurements
\item Cart wobble or vibration
\end{itemize}

\section{Extensions}

For advanced investigation:
\begin{itemize}
\item Plot $W$ vs $\Delta KE$ and analyze the slope
\item Investigate the effect of track angle on results
\item Model the impact of friction quantitatively
\end{itemize}

\end{document}