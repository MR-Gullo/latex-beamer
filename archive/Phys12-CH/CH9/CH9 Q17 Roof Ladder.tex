\documentclass{article}
\usepackage{amsmath}
\usepackage{physics}
\usepackage{graphicx}
\usepackage[margin=1in]{geometry}
\usepackage{siunitx}

\title{Physics Problem: Forces on a Ladder}
\author{Solution Guide}
\date{\today}

\begin{document}

\maketitle

\section*{Problem Statement}
To get up on the roof, a person (mass \SI{70.0}{\kilogram}) places a \SI{6.00}{\meter} aluminum ladder (mass \SI{10.0}{\kilogram}) against the house on a concrete pad with the base of the ladder \SI{2.00}{\meter} from the house. The ladder rests against a plastic rain gutter, which we can assume to be frictionless. The center of mass of the ladder is \SI{2}{\meter} from the bottom. The person is standing \SI{3}{\meter} from the bottom. What are the magnitudes of the forces on the ladder at the top and bottom?

\section*{Solution}
The forces involved are:
\begin{itemize}
    \item The weight of the man ($w$)
    \item The weight of the ladder ($W$)
    \item The normal force of the ground on the ladder bottom ($N$)
    \item The normal force of the gutter on the ladder top ($N'$)
    \item Friction between the ground and ladder bottom ($f$)
\end{itemize}

The condition of no net force horizontally leads to $f = N' \sin \theta$, where $\theta$ is the angle between the ladder and the ground:

\begin{equation}
    \theta = \arccos\left(\frac{2}{6}\right) = 70.5^\circ
\end{equation}

The condition of no net force vertically leads to $w + W = N + N' \cos \theta$, which combines with the previous condition to give $f = (w + W - N)\tan \theta$. 

The condition of no torque about the ladder bottom leads to $3w\cos \theta + 2W\cos \theta = 6N'$, which combines with the first condition to give:

\begin{equation}
    f = \left(\frac{1}{2}w + \frac{1}{3}W\right)\sin \theta \cos \theta
\end{equation}

Combining these last two conditions, we can solve for $N$:

\begin{align*}
    f &= \left(\frac{1}{2}w + \frac{1}{3}W\right)\sin \theta \cos \theta = (w + W - N)\tan \theta \\
    N &= \left(1 - \frac{\cos^2 \theta}{2}\right)w + \left(1 - \frac{\cos^2 \theta}{3}\right)W \\
    &= (0.944)(\SI{9.80}{\meter\per\second\squared})(\SI{70.0}{\kilogram}) + (0.963)(\SI{9.80}{\meter\per\second\squared})(\SI{10.0}{\kilogram}) \\
    &= \SI{742}{\newton}
\end{align*}

We can use this value to solve for $f$ and $N'$:

\begin{align*}
    f &= (w + W - N)\tan \theta = \SI{119}{\newton} \\
    N' &= \frac{f}{\sin \theta} = \SI{126}{\newton}
\end{align*}

\section*{Final Answer}
The magnitude of the force at the top is $N' = \boxed{\SI{126}{\newton}}$

The force at the bottom is the sum of friction and the normal force, with a magnitude of:
\[\sqrt{f^2 + N^2} = \boxed{\SI{751}{\newton}}\]

\end{document}