\documentclass[11pt]{article}
\usepackage[margin=1in]{geometry}
\usepackage{enumitem}
\usepackage{tcolorbox}
\usepackage{fontawesome}
\usepackage{microtype}
\usepackage{xcolor}
\usepackage{titlesec}

% Custom colors
\definecolor{highlight}{RGB}{70,130,180}
\definecolor{boxcolor}{RGB}{240,248,255}

% Custom box style for key concepts
\newtcolorbox{conceptbox}{
    colback=boxcolor,
    colframe=highlight,
    boxrule=0.5pt,
    arc=2mm,
    beforeafter skip=10pt,
    enhanced,
    fontupper=\small
}

% Section styling
\titleformat{\section}
{\normalfont\Large\bfseries\color{highlight}}
{\thesection}{1em}{}

\titleformat{\subsection}
{\normalfont\large\bfseries}
{\thesubsection}{1em}{}

\begin{document}

\begin{center}
\LARGE\textbf{Experimental Photo Journal Guide}
\end{center}

\vspace{1em}

\begin{conceptbox}
\textbf{Photo Journal:} A visual record that documents your scientific experiment as it happens, showing both the technical process and human elements.
\end{conceptbox}

\section*{Required Photo Sequences}

\subsection*{1. Setup Photos}
\begin{itemize}[leftmargin=*]
    \item Full equipment setup
    \item Important calibrations
    \item Size references
    \item Safety equipment
\end{itemize}

\subsection*{2. Process Photos}
\begin{itemize}[leftmargin=*]
    \item Starting condition
    \item Key changes during experiment
    \item Any unexpected events
    \item Adjustments made
\end{itemize}

\subsection*{3. People Photos}
\begin{itemize}[leftmargin=*]
    \item Team working with equipment
    \item Proper handling techniques
    \item Problem-solving moments
    \item Safety practices in action
\end{itemize}

\subsection*{4. Team Photo}
\begin{itemize}[leftmargin=*]
    \item All members in lab setting
    \item Include main equipment
    \item Show proper safety gear
    \item Natural poses around experiment
\end{itemize}

\begin{conceptbox}
\textbf{Technical Tips:}
\begin{itemize}[leftmargin=*]
    \item Take both wide shots (full setup) and close-ups (important details)
    \item Include objects for scale comparison
    \item Focus on critical alignments and connections
    \item Document each major step
\end{itemize}
\end{conceptbox}

\section*{Proficiency Rubric}

\begin{conceptbox}
\textbf{Assessment Framework:} This rubric evaluates both technical execution and narrative coherence in experimental documentation, recognizing that effective scientific communication requires both precision and storytelling.
\end{conceptbox}

\subsection*{Emerging}
\textbf{Description:} Beginning to grasp fundamental concepts of experimental documentation, requiring significant guidance.

\noindent\textbf{Skills and Abilities:}
\begin{itemize}[leftmargin=*]
    \item Captures basic equipment photos and group portrait with minimal attention to composition
    \item Documents experimental steps with inconsistent detail or focus
    \item Shows limited awareness of scale references and environmental context
\end{itemize}

\subsection*{Developing}
\textbf{Description:} Shows growing understanding of documentation principles, but needs support in execution.

\noindent\textbf{Skills and Abilities:}
\begin{itemize}[leftmargin=*]
    \item Creates clear equipment photos and group portrait with basic compositional awareness
    \item Records major experimental transitions with adequate detail
    \item Includes basic scale references and some environmental context
\end{itemize}

\subsection*{Proficient}
\textbf{Description:} Demonstrates solid comprehension of documentation methods, working independently.

\noindent\textbf{Skills and Abilities:}
\begin{itemize}[leftmargin=*]
    \item Produces well-composed equipment photos and engaged group portrait
    \item Captures complete experimental progression with appropriate detail
    \item Effectively integrates scale references and environmental context
\end{itemize}

\subsection*{Extending}
\textbf{Description:} Shows advanced understanding and creative capability in documentation.

\noindent\textbf{Skills and Abilities:}
\begin{itemize}[leftmargin=*]
    \item Creates compelling visual narrative through thoughtful composition and sequencing
    \item Documents subtle experimental details and transitional moments
    \item Innovatively incorporates context and scale while maintaining technical precision
\end{itemize}

\vspace{1em}
\noindent\emph{Remember: Each photo should help someone else understand what you did and how you did it.}

\end{document}