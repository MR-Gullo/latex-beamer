\documentclass{article}
\usepackage{graphicx} % Required for inserting images

\title{CS CH4 Exercise Review}
\author{Mr. Gullo}
\date{September 2024}

\begin{document}

\maketitle

\section{Introduction}
# Radioactive Decay Program: Code Review and Walkthrough

This document provides a line-by-line review of a C++ program that calculates the amount of radioactive material remaining after a given time period. Let's go through each line, explaining its purpose and how it contributes to the overall functionality of the program.

$$
\text{Remaining Amount} = \text{Initial Amount} \times e^{-\lambda t}
$$

Where:
- $e$ is Euler's number (approximately 2.71828)
- $\lambda$ is the decay constant (0.00012 in this program)
- $t$ is the time in years

## Program Review

1. `#include <iostream>`
   - This line includes the Input/Output stream library, which provides functionality for console input and output.

2. `#include <cmath>`
   - This includes the C++ mathematical functions library, which we'll need for the exponential function.

3. `using namespace std;`
   - This line allows us to use elements of the standard namespace without prefixing them with `std::`.

4. `/*` to `*/`
   - This is a multi-line comment describing the purpose of the program.

5. `const float E = 2.7182818;`
   - Defines a constant `E` representing Euler's number. However, note that this constant is not used in the final calculation.

6. `int main(){`
   - This line begins the main function, the entry point of our program.

7. `float startingAmount, numYears;`
   - Declares two float variables to store the initial amount of radioactive material and the number of years.

8. `cout << "Starting amount (grams): ";`
   - Prompts the user to enter the starting amount of radioactive material.

9. `cin >> startingAmount;`
   - Reads the user's input for the starting amount and stores it in `startingAmount`.

10. `cout << "Number of years: ";`
    - Prompts the user to enter the number of years.

11. `cin >> numYears;`
    - Reads the user's input for the number of years and stores it in `numYears`.

12. `cout << "The amount left after " << numYears << " years is "`
    - Begins the output statement, inserting the number of years.

13. `<< startingAmount * pow(E, -0.00012 * numYears)`
    - Calculates the remaining amount using the formula $A = A_0 e^{-\lambda t}$
    - Note: This line uses the `pow()` function from `<cmath>` instead of the defined `E` constant.

14. `<< " grams\n";`
    - Completes the output statement, adding "grams" and a newline.

15. `return 0;`
    - Indicates that the program has executed successfully.

16. `}`
    - Closes the main function.

## Observations and Potential Improvements

1. The program correctly implements the radioactive decay formula.
2. It uses `pow(E, ...)` instead of the predefined `E` constant. Using `exp()` function would be more appropriate.
3. The decay constant (0.00012) is hardcoded. It could be defined as a constant for better readability and maintainability.
4. Error checking for input validation is not implemented.
5. The program doesn't specify the units for the decay constant, which could lead to misinterpretation.

By understanding each line of this program, we can see how it implements the radioactive decay formula and interacts with the user to provide a useful calculation tool.
\end{document}
