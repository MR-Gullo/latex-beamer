\documentclass{beamer}
\usepackage[utf8]{inputenc}
\usepackage{amsmath}
\usepackage{graphicx}
\usepackage{xcolor}
\usepackage{tikz}

\usetheme{Madrid}
\usecolortheme{default}

% Define custom colors inspired by Star Trek DS9
\definecolor{ds9blue}{RGB}{25,25,112} % Midnight Blue
\definecolor{ds9gold}{RGB}{218,165,32} % Goldenrod
\definecolor{ds9grey}{RGB}{105,105,105} % Dim Gray
\definecolor{ds9red}{RGB}{178,34,34} % Firebrick

% Customize the colors
\setbeamercolor{title}{fg=ds9gold}
\setbeamercolor{frametitle}{bg=ds9blue, fg=white}
\setbeamercolor{block title}{bg=ds9gold, fg=black}
\setbeamercolor{block body}{bg=ds9grey!20, fg=black}
\setbeamercolor{section in toc}{fg=ds9gold}
\setbeamercolor{subsection in toc}{fg=ds9gold!70}
\setbeamercolor{footline}{bg=ds9blue, fg=white}
\setbeamercolor{author in head/foot}{fg=white}
\setbeamercolor{date in head/foot}{fg=white}
\setbeamercolor{title in head/foot}{fg=white}

% Title page configuration
\title[Kinematic Equations]{Kinematic Equations as Quadratic Analogies}
\subtitle{ }
\author[Mr. Gullo]{Mr. Gullo}
\date[Sept 2024]{September 2024}

% Table of contents at the beginning of each section
\AtBeginSection[]
{
  \begin{frame}
    \frametitle{Table of Contents}
    \tableofcontents[currentsection]
  \end{frame}
}

\begin{document}

\frame{\titlepage}
\section{Standard Form}

\begin{frame}
\frametitle{Standard Form of Quadratic Equation}
\begin{itemize}
    \item Standard form of a quadratic equation:
    $$ax^2 + bx + c = 0$$
    \item Analogous kinematic equation:
    $$x = x_0 + v_0t + \frac{1}{2}at^2$$
    \item Where:
    \begin{itemize}
        \item $x$ is the position (analogous to $y$ in the quadratic)
        \item $t$ is time (analogous to $x$ in the quadratic)
        \item $x_0$ is the initial position (analogous to $c$)
        \item $v_0$ is the initial velocity (analogous to $b$)
        \item $\frac{1}{2}a$ is half the acceleration (analogous to $a$)
    \end{itemize}
\end{itemize}
\end{frame}

\begin{frame}
\frametitle{Standard Form - Interpretation}
\begin{itemize}
    \item This equation describes the position of an object at any given time
    \item It considers:
    \begin{itemize}
        \item Initial position
        \item Initial velocity
        \item Acceleration
    \end{itemize}
    \item Useful for analyzing motion in one dimension
\end{itemize}
\end{frame}

\section{Vertex Form}

\begin{frame}
\frametitle{Vertex Form of Quadratic Equation}
\begin{itemize}
    \item Vertex form of a quadratic equation:
    $$y = a(x - h)^2 + k$$
    \item Analogous kinematic equation:
    $$x = x_0 + v_0t + \frac{1}{2}a(t - t_p)^2$$
    \item Where:
    \begin{itemize}
        \item $t_p$ is the time at which the position reaches its peak (analogous to $h$)
        \item $x_0 + v_0t$ represents the position at the peak (analogous to $k$)
    \end{itemize}
\end{itemize}
\end{frame}

\begin{frame}
\frametitle{Vertex Form - Applications}
\begin{itemize}
    \item Particularly useful for describing projectile motion
    \item $t_p$ represents the time at which the projectile reaches its highest point
    \item Helps in analyzing:
    \begin{itemize}
        \item Maximum height
        \item Time of flight
        \item Range of the projectile
    \end{itemize}
\end{itemize}
\end{frame}

\section{Factored Form}

\begin{frame}
\frametitle{Factored Form of Quadratic Equation}
\begin{itemize}
    \item Factored form of a quadratic equation:
    $$y = a(x - r_1)(x - r_2)$$
    \item Analogous kinematic equation:
    $$x - x_0 = v_0(t - t_1)(t - t_2)$$
    \item Where:
    \begin{itemize}
        \item $t_1$ and $t_2$ are the times when the object is at its initial position $x_0$ (analogous to roots $r_1$ and $r_2$)
        \item $v_0$ is a scaling factor (analogous to $a$)
    \end{itemize}
\end{itemize}
\end{frame}

\begin{frame}
\frametitle{Factored Form - Applications}
\begin{itemize}
    \item Less common in kinematics
    \item Useful in specific scenarios:
    \begin{itemize}
        \item Describing an object that returns to its starting position twice
        \item Example: A ball thrown vertically upward
    \end{itemize}
    \item Helps in analyzing:
    \begin{itemize}
        \item Time of flight
        \item Return times to initial position
    \end{itemize}
\end{itemize}
\end{frame}

\section{Conclusion}

\begin{frame}
\frametitle{Conclusion}
\begin{itemize}
    \item These analogies illustrate mathematical similarities between:
    \begin{itemize}
        \item Quadratic equations
        \item Motion in one dimension
    \end{itemize}
    \item Provides a different perspective on both topics
    \item Helps in understanding:
    \begin{itemize}
        \item The mathematical nature of motion
        \item The physical interpretation of quadratic equations
    \end{itemize}
    \item Encourages interdisciplinary thinking in mathematics and physics
\end{itemize}
\end{frame}

\end{document}
