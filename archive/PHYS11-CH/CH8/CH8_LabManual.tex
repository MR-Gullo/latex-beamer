\documentclass[12pt]{article}
\usepackage[margin=1in]{geometry}
\usepackage{amsmath}
\usepackage{graphicx}
\usepackage{float}
\usepackage{enumitem}
\usepackage{booktabs}
\usepackage{tikz}
\usepackage{fancyhdr}
\usepackage[rgb]{xcolor}
\usepackage{tcolorbox}
\usepackage{wrapfig}

% Define colors
\definecolor{labblue}{RGB}{51,101,138}
\definecolor{safetyyellow}{RGB}{255,240,150}
\definecolor{conceptgreen}{RGB}{200,230,201}

% Logo configuration
\newcommand{\headerlogo}[1]{%
    \includegraphics[height=20pt]{#1}%
}

% Configure page style
\pagestyle{fancy}
\fancyhead{}
\fancyhead[L]{\headerlogo{cinec_logo.png} \hspace{1em} Physics Laboratory}
\fancyhead[R]{Momentum in Sports}
\fancyfoot{}
\fancyfoot[C]{\thepage}

\begin{document}

\begin{center}
{\Large \textbf{Laboratory Investigation S1:}}\\[0.5cm]
{\LARGE \textbf{Momentum in Sports Physics}}\\[1cm]
\rule{\textwidth}{0.4pt}
\end{center}

\section*{Experimental Overview}

\textbf{Key Concept}:\\
Momentum is the product of an object's mass and velocity, expressed mathematically as: $$\vec{p} = m\vec{v}$$

In this investigation, you'll analyze the momentum of sports equipment in motion, connecting classroom physics to real-world athletics. By capturing slow-motion video of sports equipment in action, you'll discover how mass and velocity combine to create the dynamic interactions we see in sports.

\begin{tcolorbox}[colback=conceptgreen!10,colframe=conceptgreen,title=\textbf{Investigation Strategy}]
Choose a sport where an object's momentum plays a crucial role. Excellent choices include:
\begin{itemize}
\item A Frisbee catch
\item A soccer ball being kicked
\item A volleyball serve
\item A basketball jump shot
\item A Badminton serve
\end{itemize}
Consider which motion will be easiest to capture clearly on video while also being safe to measure.
\end{tcolorbox}

\section*{Materials and Equipment}
\begin{itemize}
\item Smartphone with video capability (120fps or higher recommended but not needed)
\item Electric balance (accurate to at least 0.1g)
\item Measuring tape or meter stick
\item Your chosen sports equipment
\item Spreadsheet software
\item Ruler (for measuring distances on video playback)
\item Reference object of known size (for video scale calibration)
\end{itemize}

\section*{Safety Considerations}
\begin{tcolorbox}[colback=safetyyellow!30,colframe=safetyyellow!80,title=\textbf{Safety Protocol}]
Before beginning your investigation:
\begin{itemize}
\item Choose a safe, open area appropriate for your sport
\item Ensure no people or valuable objects are in the path of your sports equipment
\item Follow all standard safety procedures for your chosen sport
\item Have a spotter present during measurements and filming
\item Secure loose clothing and remove jewelry that could interfere with movement
\end{itemize}
\end{tcolorbox}

\section*{Experimental Procedure}

\subsection*{Part 1: Mass Measurement}
\begin{enumerate}[label=\arabic*.]
\item Use the electric balance to measure your sports object's mass in kilograms
\item Record this value with appropriate significant figures
\item Measure and record any relevant dimensions of your object
\end{enumerate}

\subsection*{Part 2: Velocity Analysis}
\begin{enumerate}[label=\arabic*.]
\item Place your reference object (of known size) in the same plane as your intended motion
\item Record your sports action in slow motion (try to keep the camera perpendicular to the motion)
\item Repeat for at least 5 measurements
\item Transfer the video to a computer for analysis

\end{enumerate}

\subsection*{Video Analysis}
\begin{enumerate}[label=\arabic*.]
\item Count the total frames for your object's motion
\item Note your camera's frame rate (frames per second)
\item Use your reference object to establish scale in the video
\item Measure the distance your object travels
\item Calculate velocity using: $$v = \frac{\text{distance}}{\text{time}} = \frac{\text{distance}}{\text{number of frames}/\text{frames per second}}$$
\end{enumerate}

\begin{tcolorbox}[colback=labblue!5,colframe=labblue,title=\textbf{Sample Calculation}]
Let's analyze a basketball free throw:

Given:
\begin{itemize}
\item Distance traveled: 4.2 meters (from player to hoop)
\item Number of frames: 30 frames
\item Camera frame rate: 240 frames per second
\end{itemize}

Time interval = $\frac{30 \text{ frames}}{240 \text{ frames/second}} = 0.125 \text{ seconds}$

Velocity = $\frac{4.2 \text{ meters}}{0.125 \text{ seconds}} = 33.6 \text{ meters/second}$

If the basketball's mass is 0.62 kg, its momentum would be:

Momentum = $0.62 \text{ kg} \times 33.6 \text{ m/s} = 20.8 \text{ kg}\cdot\text{m/s}$

\textit{Note: This example illustrates why precision in measurement is crucial - small errors in distance or frame counting can significantly affect your final momentum calculation.}
\end{tcolorbox}

\section*{Data Collection}
Record your measurements in this format:

\begin{table}[H]
\centering
\begin{tabular}{ccc}
\toprule
Measurement & Value & Units \\
\midrule
Object Mass & & kg \\
Distance Traveled & & m \\
Frame Count & & frames \\
Time Interval & & s \\
Calculated Velocity & & m/s \\
Calculated Momentum & & kg⋅m/s \\
\bottomrule
\end{tabular}
\end{table}

\begin{tcolorbox}[colback=conceptgreen!10,colframe=conceptgreen,title=\textbf{Video Recording Best Practices}]
\begin{itemize}
\item Use the highest frame rate available
\item Keep the camera steady (tripod recommended)
\item Ensure good lighting
\item Position the camera perpendicular to the motion
\item Include a reference object of known size in the frame
\item Record multiple trials
\end{itemize}
\end{tcolorbox}

\section*{Analysis Questions}
\begin{enumerate}[label=\arabic*.]
\item How does your object's momentum compare to other sports? Research typical values for comparison.
\item What factors might affect the accuracy of your measurements?
\item How might air resistance impact your results?
\item Why is momentum important in your chosen sport?
\item How could your measurement technique be improved?
\end{enumerate}

\section*{Additional Insights}
Consider how momentum conservation applies in your sport. For example, in a collision between a bat and baseball, total momentum is conserved even as it transfers between objects. How does this principle manifest in your chosen sport?

\textit{Optional Extension}: Compare the momentum of different techniques in your sport (e.g., different types of serves in tennis or different types of kicks in soccer).
\section*{Assessment Rubric}

\begin{tcolorbox}[colback=conceptgreen!10,colframe=conceptgreen,title=\textbf{Emerging}]
\textbf{Description:} Beginning to grasp fundamental momentum concepts and video analysis methods, requiring significant guidance.

\textbf{Skills and Abilities:}
\begin{itemize}
\item Can identify basic lab equipment and follow safety protocols with supervision
\item Understands the basic momentum equation ($\vec{p} = m\vec{v}$) but struggles to apply it
\item Can record mass measurements and frame counts but needs help with calculations
\item Has difficulty converting video measurements to real-world velocities
\item Requires assistance with video recording and analysis techniques
\end{itemize}
\end{tcolorbox}

\begin{tcolorbox}[colback=conceptgreen!10,colframe=conceptgreen,title=\textbf{Developing}]
\textbf{Description:} Shows growing understanding of momentum principles and video analysis techniques, but needs support applying them.

\textbf{Skills and Abilities:}
\begin{itemize}
\item Sets up video recording equipment with some assistance, following safety guidelines
\item Records mass and video data systematically but may have inconsistent significant figures
\item Can perform basic velocity calculations from frame counts with support
\item Makes simple momentum calculations but may struggle with units
\item Creates basic data tables but needs help organizing multiple trials
\item Recognizes the relationship between mass and velocity in momentum but has difficulty explaining it
\end{itemize}
\end{tcolorbox}

\begin{tcolorbox}[colback=conceptgreen!10,colframe=conceptgreen,title=\textbf{Proficient}]
\textbf{Description:} Demonstrates solid comprehension of momentum concepts and video analysis methods, working independently with minimal support.

\textbf{Skills and Abilities:}
\begin{itemize}
\item Independently sets up video recording equipment and conducts trials safely
\item Records precise measurements with appropriate units and significant figures
\item Correctly analyzes video footage to determine velocities
\item Calculates momentum values accurately with consistent units
\item Creates clear data tables with all required measurements
\item Identifies common sources of experimental error in video analysis
\item Makes meaningful connections between calculated momentum and sports performance
\end{itemize}
\end{tcolorbox}

\begin{tcolorbox}[colback=conceptgreen!10,colframe=conceptgreen,title=\textbf{Extending}]
\textbf{Description:} Shows advanced understanding and analytical capability, exploring momentum concepts beyond basic requirements.

\textbf{Skills and Abilities:}
\begin{itemize}
\item Designs improvements to video analysis procedure to enhance accuracy
\item Considers advanced factors like air resistance and rotation in analysis
\item Provides sophisticated comparison of momentum values across different sports
\item Makes insightful connections between momentum conservation and game strategies
\item Proposes creative extensions to investigate related concepts like impulse
\item Analyzes how different techniques in their chosen sport affect momentum values
\item Critically evaluates limitations of video analysis methods and suggests improvements
\end{itemize}
\end{tcolorbox}


\end{document}