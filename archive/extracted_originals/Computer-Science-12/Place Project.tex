\documentclass[12pt]{article}
\usepackage{geometry}
\usepackage{graphicx}
\usepackage{enumitem}
\usepackage{booktabs}
\usepackage{hyperref}
\usepackage{float}
\usepackage{multirow}
\usepackage{array}
\usepackage[table]{xcolor}
\usepackage{colortbl}

\geometry{margin=1in}

\title{Computational Thinking in Context: A Technical and Cultural Analysis\\
\large Written Research Assignment}
\author{Computer Science 12}
\date{March, 2025}

\begin{document}
\maketitle

\section*{Purpose}
To explore the intersection of computational thinking and cultural significance in your local environment through technical analysis and implementation documentation. This assignment recognizes the importance of Indigenous perspectives on place-based knowledge and encourages students to consider First Peoples' approaches to problem-solving and knowledge systems in relation to computational thinking.

\section*{Required Sections}

\subsection*{1. Introduction }
\begin{itemize}
    \item Overview of chosen computational problem/system from the course content
    \item Brief technical context
    \item Thesis statement connecting computational concepts to cultural significance
    \item Recognition of First Peoples perspectives on the relationship between technology, place, and knowledge systems
\end{itemize}

\subsection*{2. Technical Analysis }
\begin{itemize}
    \item Detailed explanation of at least two computational principles present
    \item Supporting code implementation and complexity analysis
    \item Technical diagrams or flowcharts with detailed documentation explaining algorithms
    \item Analysis of how computational properties contribute to the system's function
\end{itemize}

\subsection*{3. Cultural Integration }
\begin{itemize}
    \item Historical background of computational needs in the local community
    \item Traditional problem-solving approaches and practices, including First Peoples methodologies where relevant
    \item Documentation of community technological perspectives
    \item Analysis of how computational understanding enhanced traditional practices
    \item Consideration of Indigenous knowledge systems and their relationship to computational thinking
\end{itemize}

\subsection*{4. Implementation Analysis }
\begin{itemize}
    \item Discussion of how computational methods and cultural practices intersect
    \item Examination of traditional problem-solving and modern algorithmic approaches
    \item Analysis of how computational solutions contribute to community development
    \item Reflection on how Indigenous ways of knowing might inform computational approaches
\end{itemize}

\subsection*{5. Reflection }
\begin{itemize}
    \item Personal insights on the relationship between computation and culture
    \item Discussion of development experience
    \item Broader implications for understanding computer science in cultural context
    \item Reflections on how First Peoples perspectives have influenced your understanding of computational thinking
\end{itemize}



\newpage


\section*{Format Requirements}
\begin{itemize}
    \item Length: 1500-2000 words
    \item Structure: Technical documentation format with clear sections
    \item Code: Minimum 4 original implementations with documentation
    \item Citations: APA format
    \item Font: 12-point Times New Roman, double-spaced
    \item Digital submission in PDF format
\end{itemize}


\section*{Research Requirements}
\begin{itemize}
    \item Minimum 4 technical sources
    \item Minimum 2 community sources (interviews, local systems, website)
    \item Proper documentation of all implementations
    \item Balance of technical and cultural references
    \item When including First Peoples knowledge, ensure appropriate attribution to specific nations or knowledge keepers
\end{itemize}
\section*{Technical Documentation Requirements}
Each technical element must include:
\begin{itemize}
    \item Clear, well-documented code or algorithm
    \item Detailed documentation (50-100 words) explaining:
    \begin{itemize}
        \item What the implementation accomplishes and source attribution and licensing
        \item Relevant computational concepts illustrated
        \item Cultural significance and impact
    \end{itemize}
    \end{itemize}

\newpage

\section*{Community Consultation Guidelines}
When engaging with community members, especially Elders or Indigenous knowledge keepers, please observe these protocols:
\begin{itemize}
    \item Approach with respect and humility, acknowledging the value of their time and wisdom
    \item Clearly explain the purpose of your research and how their knowledge will be used
    \item Obtain informed consent before recording or documenting conversations
    \item Offer appropriate gifts or honoraria in accordance with local cultural protocols
    \item Share your final work with those who contributed to validate your representation of their knowledge
    \item Recognize that some knowledge may not be appropriate to include in academic work and respect boundaries
    \item Acknowledge the specific First Nation, community, or individual who shared knowledge in your final work
\end{itemize}

\newpage
% First page of tables
\begin{table}[t]
\renewcommand{\arraystretch}{1.5}
\begin{tabular}{>{\raggedright\arraybackslash}p{2cm}|>{\raggedright\arraybackslash}p{14cm}}
\toprule
\multicolumn{2}{l}{\textbf{Emerging}} \\
\midrule
Description & Student demonstrates basic understanding of computational concepts and research methods, requiring significant guidance to complete tasks. Work shows initial attempts to connect technical and cultural elements but lacks depth and independence. \\
\midrule
Skills and Abilities & 
\begin{itemize}
    \item Implements basic algorithms with substantial guidance and support
    \item Conducts preliminary research using provided sources and basic documentation methods
    \item Creates simple technical documentation with basic comments that need significant revision
    \item Makes surface-level connections between computational methods and cultural significance
    \item Submits work more than 3 days late without communication, uses incorrect file formats, and leaves significant work incomplete
\end{itemize} \\
\bottomrule
\end{tabular}
\end{table}

\begin{table}[b]
\renewcommand{\arraystretch}{1.5}
\begin{tabular}{>{\raggedright\arraybackslash}p{2cm}|>{\raggedright\arraybackslash}p{14cm}}
\toprule
\multicolumn{2}{l}{\textbf{Developing}} \\
\midrule
Description & Student shows growing comprehension of both computational concepts and cultural analysis, requiring moderate guidance. Work demonstrates increasing ability to implement solutions and conduct independent research, though analysis remains somewhat superficial. \\
\midrule
Skills and Abilities & 
\begin{itemize}
    \item Implements fundamental algorithms with some accuracy, occasionally needing correction
    \item Conducts research using recommended sources and follows documentation guidelines with reminders
    \item Produces clear technical documentation with descriptive comments that address both implementation and cultural context
    \item Draws meaningful connections between computational methods and cultural significance with some guidance
    \item Submits 1-2 days late or requests last-minute extensions while generally following format requirements and completing most work, though it may be rushed
\end{itemize} \\
\bottomrule
\end{tabular}
\end{table}

\clearpage % Force a page break

% Second page of tables
\begin{table}[t]
\renewcommand{\arraystretch}{1.5}
\begin{tabular}{>{\raggedright\arraybackslash}p{2cm}|>{\raggedright\arraybackslash}p{14cm}}
\toprule
\multicolumn{2}{l}{\textbf{Proficient}} \\
\midrule
Description & Student demonstrates solid understanding of computational concepts and research methods, working independently with minimal guidance. Work shows thorough analysis and clear connections between technical and cultural elements. \\
\midrule
Skills and Abilities & 
\begin{itemize}
    \item Accurately implements and analyzes algorithms with supporting documentation
    \item Independently conducts comprehensive research using diverse technical and community sources
    \item Creates high-quality technical documentation with detailed, informative comments that integrate concepts
    \item Develops clear, well-supported connections between computational methods and cultural significance
    \item Submits on time with proper formatting, communicates about potential delays, and completes all components thoroughly
\end{itemize} \\
\bottomrule
\end{tabular}
\end{table}

\begin{table}[b]
\renewcommand{\arraystretch}{1.5}
\begin{tabular}{>{\raggedright\arraybackslash}p{2cm}|>{\raggedright\arraybackslash}p{14cm}}
\toprule
\multicolumn{2}{l}{\textbf{Extending}} \\
\midrule
Description & Student exhibits exceptional understanding of computational principles and research methods, working independently and showing initiative. Work demonstrates sophisticated analysis, original insights, and seamless integration of technical and cultural elements. \\
\midrule
Skills and Abilities & 
\begin{itemize}
    \item Provides sophisticated implementation of algorithms with innovative approaches and insights
    \item Conducts extensive research that exceeds requirements, incorporating unique perspectives and sources
    \item Produces outstanding technical documentation with comprehensive comments that enhance understanding
    \item Develops complex, nuanced connections between computational methods and cultural significance
    \item Submits quality work ahead of deadlines, maintains clear communication, prepares well for known absences, and creates systems for tracking requirements
\end{itemize} \\
\bottomrule
\end{tabular}
\end{table}
\vspace{1cm}

\end{document}