\documentclass[12pt]{article}

\usepackage{geometry}
\usepackage{titlesec}
\usepackage{enumitem}
\usepackage{xcolor}
\usepackage{tcolorbox}
\usepackage{fancyhdr}
\usepackage{hyperref}
\usepackage{tabularx}
\usepackage{booktabs}

% Custom colors
\definecolor{bcblue}{RGB}{0,85,140}
\definecolor{bcgreen}{RGB}{0,111,59}
\definecolor{bcgold}{RGB}{255,197,47}

% Custom formatting
\titleformat{\section}
{\color{bcblue}\Large\bfseries}
{\thesection}{1em}{}
\titleformat{\subsection}
{\color{bcgreen}\large\bfseries}
{\thesubsection}{1em}{}

% Header and footer
\pagestyle{fancy}
\fancyhead[L]{Computer Science 12}
\fancyhead[R]{Algorithm Analysis Unit}
\fancyfoot[C]{Page \thepage}

\begin{document}

\begin{center}
{\Large\bfseries Computer Science 12: Algorithm Analysis}\\[2ex]
{\large\bfseries Lesson Plan: Introduction to Algorithmic Efficiency}\\[2ex]
\end{center}

\section*{Curriculum Connections}

\subsection*{Big Ideas}
\begin{itemize}
    \item Design can be approached systematically
    \item Computational thinking enables us to solve problems
    \item Computer programming requires logic, creativity, and scientific thinking
\end{itemize}

\subsection*{Core Competencies}
\begin{description}
    \item[Communication] 
        Students will articulate their understanding of algorithmic efficiency through verbal, written, and visual means
    \item[Thinking]
        Students will analyze algorithms critically and develop creative solutions
    \item[Personal \& Social]
        Students will collaborate effectively and reflect on their learning process
\end{description}

\subsection*{Curricular Competencies}
\begin{itemize}
    \item Understanding context
    \item Defining problems and making decisions
    \item Engaging in computational thinking processes
    \item Testing, debugging, and documenting
\end{itemize}

\section{Lesson Overview}

\begin{tcolorbox}[colback=gray!5,colframe=bcblue,title=Lesson Framework]
Duration: 80 minutes (2 × 40-minute periods)\\
Focus: Algorithm Efficiency and Big O Notation\\
Learning Modalities: Visual, Auditory, Kinesthetic, Reading/Writing
\end{tcolorbox}

\section{Learning Objectives}
By the end of this lesson, students will be able to:
\begin{itemize}
    \item Define algorithmic efficiency using proper terminology
    \item Analyze and compare different time complexities
    \item Apply Big O notation to real-world scenarios
    \item Evaluate algorithm performance using systematic approaches
\end{itemize}

\section{Required Resources}
\begin{itemize}
    \item Digital presentation system
    \item Student devices for video viewing
    \item Online quiz platform access
\end{itemize}

\section{Detailed Lesson Structure}

\subsection{Part 1 (40 minutes)}

\subsubsection*{Opening (5 minutes)}
\begin{tcolorbox}[colback=gray!5,colframe=bcgreen]
Engage students with real-world analogies:
\begin{itemize}
    \item Direct access (O(1)): Finding a book on your desk
    \item Linear search (O(n)): Searching through a line of students
    \item Quadratic complexity (O(n²)): Checking every possible pair in class
\end{itemize}
\end{tcolorbox}

\subsubsection*{Direct Instruction (20 minutes)}
\begin{itemize}
    \item Present foundational concepts using interactive slides
    \item Demonstrate growth patterns visually
    \item Facilitate interactive examples and discussions
    \item Connect to Applied Design, Skills, and Technologies curriculum
\end{itemize}

\subsubsection*{Guided Practice (15 minutes)}
\begin{itemize}
    \item Work through example algorithms
    \item Implement think-pair-share strategies
    \item Practice complexity analysis techniques
\end{itemize}

\subsection{Part 2 (40 minutes)}

\subsubsection*{Video Series Introduction (5 minutes)}
Content overview and viewing strategy instruction

\subsubsection*{Active Learning (20 minutes)}
\begin{itemize}
    \item Video segments 1 \& 2 with guided notes
    \item Collaborative comprehension checks
    \item Real-time concept application
\end{itemize}

\subsubsection*{Independent Work Time (15 minutes)}
\begin{itemize}
    \item Homework introduction
    \item Initial quiz preparation
    \item Individual support and clarification
\end{itemize}

\section{Assessment Strategy}

\subsection*{Formative Assessment}
\begin{itemize}
    \item In-class participation
    \item Comprehension check responses
    \item Peer discussions
\end{itemize}

\subsection*{Summative Assessment}
\begin{tcolorbox}[colback=gray!5,colframe=bcblue,title=Quiz Structure (20 points)]
\begin{description}
    \item[Multiple Choice Practice Questions] Growth pattern identification, concept comprehension
    \item Students are allowed multiple attempts 
\end{description}
\end{tcolorbox}

\section{Differentiation Strategies}

\begin{itemize}
    \item Provide multiple representation modes
    \item Offer choice in demonstration of learning
    \item Scale complexity of examples
    \item Support diverse learning needs
\end{itemize}

\section{Extensions and Connections}

\begin{itemize}
    \item Link to Mathematics 12 concepts
    \item Connect to real-world applications
    \item Prepare for advanced computing concepts
    \item Bridge to post-secondary pathways
\end{itemize}

\section{Reflection and Feedback}

Teachers should consider:
\begin{itemize}
    \item Student engagement levels
    \item Concept mastery indicators
    \item Pacing effectiveness
    \item Areas for improvement
\end{itemize}

\end{document}