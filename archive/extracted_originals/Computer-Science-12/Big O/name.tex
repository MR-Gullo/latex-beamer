\documentclass{beamer}

% Required packages
\usepackage{tikz}
\usepackage{listings}
\usepackage{xcolor}

% Define custom colors
\definecolor{deepblue}{RGB}{0,51,102}
\definecolor{lightblue}{RGB}{235,244,255}
\definecolor{darkgray}{RGB}{64,64,64}
\definecolor{codegreen}{RGB}{104,180,104}

% Theme configuration
\usetheme{Madrid}
\usecolortheme{whale}
\setbeamercolor{structure}{fg=deepblue}
\setbeamercolor{block title}{bg=deepblue,fg=white}
\setbeamercolor{block body}{bg=lightblue,fg=black}

% Title page configuration
\title[Algorithm Analysis]{Think-Pair-Share: Algorithmic Detective Work}
\subtitle{Discovering Patterns in Code Together}
\author[Student Guide]{Collaborative Learning Session}
\date{}

\begin{document}

% Title page
\begin{frame}
    \titlepage
\end{frame}

% Our Mystery Today
\begin{frame}{The Mystery We'll Solve Together}
    \begin{block}{Our Challenge}
        Today, we're going to be code detectives. We'll examine a mysterious function and uncover:
        \begin{itemize}
            \item What it does
            \item How efficiently it works
            \item Where we might use it in real life
        \end{itemize}
    \end{block}
    
    \begin{alertblock}{How We'll Work}
        \begin{enumerate}
            \item Think: Solo investigation (3-5 minutes)
            \item Pair: Partner detective work (5-7 minutes)
            \item Share: Team conference (7-10 minutes)
        \end{enumerate}
    \end{alertblock}
\end{frame}

% The Mystery Function
\begin{frame}[fragile]{The Mystery Function}
    \begin{block}{Code to Analyze}
        \begin{lstlisting}[language=Python,basicstyle=\ttfamily\small]
def mystery_function(arr):
    result = 0
    for i in range(len(arr)):
        for j in range(i + 1, len(arr)):
            if arr[i] + arr[j] == 10:
                result += 1
    return result
        \end{lstlisting}
    \end{block}
    
    \begin{example}
        Test Array: [2, 8, 5, 3, 7]
    \end{example}
\end{frame}

% Think Phase
\begin{frame}{Think Phase: Solo Investigation}
    \begin{block}{Your Detective Notes}
        On the paper provided, investigate:
        \begin{enumerate}
            \item What does each loop accomplish?
            \item Trace the first three iterations
            \item What patterns do you notice?
        \end{enumerate}
    \end{block}
    
    \begin{alertblock}{Investigation Tools}
        \begin{itemize}
            \item Use arrows to track the loops
            \item Circle pairs that add to 10
            \item Note how many comparisons you make
        \end{itemize}
    \end{alertblock}
\end{frame}

% Pair Phase
\begin{frame}{Pair Phase: Compare Notes}
    \begin{block}{Discussion Guide}
        With your partner:
        \begin{itemize}
            \item Show how you traced the code
            \item Compare your findings
            \item Discuss any differences
        \end{itemize}
    \end{block}
    
    \begin{alertblock}{Key Questions}
        \begin{enumerate}
            \item Why does j start at i + 1?
            \item How many times does the inner loop run?
            \item What's the largest possible number of comparisons?
        \end{enumerate}
    \end{alertblock}
\end{frame}

% Share Phase Setup
\begin{frame}{Share Phase: Team Conference}
    \begin{block}{Discussion Format}
        We'll build understanding together:
        \begin{itemize}
            \item Different pairs will share their insights
            \item We'll compare approaches
            \item Together, we'll reach conclusions
        \end{itemize}
    \end{block}
    
    \begin{alertblock}{Prepare to Share}
        Be ready to discuss:
        \begin{itemize}
            \item Your method for counting operations
            \item Any patterns you discovered
            \item Real-world connections
        \end{itemize}
    \end{alertblock}
\end{frame}



% Common Pitfalls
\begin{frame}{Detective's Tips}
    \begin{block}{Watch Out For}
        Common misunderstandings:
        \begin{itemize}
            \item Thinking loops always mean O(n²)
            \item Missing the i + 1 pattern
            \item Counting only successful matches
        \end{itemize}
    \end{block}
    
    \begin{alertblock}{Investigation Tips}
        \begin{itemize}
            \item Count ALL comparisons
            \item Think about worst-case scenario
            \item Consider what happens as array grows
        \end{itemize}
    \end{alertblock}
\end{frame}

% Extension Challenges
\begin{frame}{Bonus Investigations}
    \begin{block}{For Expert Detectives}
        If you finish early, try:
        \begin{enumerate}
            \item Can we make this function faster?
            \item What if we wanted ALL pairs that sum to 10?
            \item How would sorting help/hurt?
        \end{enumerate}
    \end{block}
    
    
\end{frame}

% Wrap Up
\begin{frame}{Case Closed: What We Learned}
    \begin{block}{Key Discoveries}
        \begin{itemize}
            \item How nested loops affect complexity
            \item Why starting position matters
            \item Real-world applications of pair finding
        \end{itemize}
    \end{block}
    
    \begin{alertblock}{Next Steps}
        \begin{itemize}
            \item Apply these techniques to new problems
            \item Look for similar patterns in other code
            \item Share your detective skills!
        \end{itemize}
    \end{alertblock}
\end{frame}

\end{document}