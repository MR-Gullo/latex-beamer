\documentclass{beamer}
% Required packages
\usepackage{amsmath}
\usepackage{physics}
\usepackage{graphicx}
\usepackage{siunitx}
\usepackage{xcolor}
% Set image search paths
\graphicspath{{../images/}{../../shared/images/}}

% Define custom colors for DS9 theme
\definecolor{ds9blue}{RGB}{25,25,112}
\definecolor{ds9gold}{RGB}{218,165,32}
\definecolor{ds9grey}{RGB}{105,105,105}
\definecolor{ds9red}{RGB}{178,34,34}
% Set up the Madrid theme with custom colors
\usetheme{Madrid}
\usecolortheme{whale}
\setbeamercolor{palette primary}{bg=ds9blue,fg=white}
\setbeamercolor{palette secondary}{bg=ds9grey,fg=white}
\setbeamercolor{palette tertiary}{bg=ds9gold,fg=black}
\setbeamercolor{palette quaternary}{bg=ds9red,fg=white}
\setbeamercolor{structure}{fg=ds9blue}
\setbeamercolor{title}{fg=ds9gold}
\setbeamercolor{subtitle}{fg=ds9gold}
\setbeamercolor{frametitle}{bg=ds9blue,fg=white}
\setbeamercolor{block title}{bg=ds9blue,fg=white}
\setbeamercolor{block body}{bg=ds9grey!20,fg=black}

\title[Physics Final Strategy]{PHYS11/12: Final Exam Strategy}
\author[Mr. Gullo]{Mr. Gullo}
\date[June 2025]{June 2025}

\begin{document}

\frame{\titlepage}

\section{Introduction and Proficiency Scale}

\begin{frame}{Learning Objectives}
By the end of this presentation, you will be able to:
\begin{itemize}
\item Understand the proficiency scale used for grading open-ended questions
\item Apply strategic time management techniques during the exam
\item Use the G.U.E.S.S. method systematically for problem-solving
\item Distinguish between Proficient and Extending level responses
\item Demonstrate clear, logical thinking in your solutions
\end{itemize}
\end{frame}

\begin{frame}{Understanding the Proficiency Scale}
\begin{block}{Four Levels of Understanding}
\begin{itemize}
\item \textcolor{ds9red}{\textbf{Emerging:}} Limited understanding, incomplete solutions
\item \textcolor{orange}{\textbf{Developing:}} Partial understanding, some correct elements
\item \textcolor{green}{\textbf{Proficient:}} Clear understanding, systematic approach
\item \textcolor{ds9blue}{\textbf{Extending:}} Deep understanding, sophisticated reasoning
\end{itemize}
\end{block}

\vspace{0.5cm}

\textbf{Key Point:} \emph{How} you solve a problem is as important as your final answer!
\end{frame}

\section{Strategic Test-Taking}

\begin{frame}{The First Five Minutes: Strategic Setup}
\begin{enumerate}
\item \textbf{Brain Dump (30 seconds):} Write essential formulas, constants, and rules on scrap paper
\item \textbf{Survey the Battlefield (2 minutes):} Scan the entire exam, identify open-response questions
\item \textbf{Internalize the Goal:} Remember to be clear, systematic, and logical
\end{enumerate}

\vspace{0.5cm}

\begin{block}{Pro Tip}
Get worried formulas out of your head and onto paper immediately!
\end{block}
\end{frame}

\begin{frame}{The Three-Pass Strategy}
\textbf{Don't do the exam in order!} Maximize your points.

\begin{block}{Pass 1: Quick Wins (10-15 min)}
Answer all easy multiple-choice questions you know instantly. Circle harder questions and move on.
\end{block}
\begin{block}{Pass 2: The Deep Dive (45-50 min)}
Focus on open-response questions using the G.U.E.S.S. method. Spend quality time here.
\end{block}
\begin{block}{Pass 3: The Finish Line (10-15 min)}
Attempt remaining questions, review work, transfer answers carefully.
\end{block}


\end{frame}

\section{G.U.E.S.S. Problem-Solving Method}

\begin{frame}{Introducing the G.U.E.S.S. Method}
Your roadmap to \textbf{Proficient} level responses:

\vspace{0.5cm}
\begin{left}

\textbf{G} - Givens \& Diagram \\
\textbf{U} - Unknowns \& Plan \\
\textbf{E} - Equations \\
\textbf{S} - Substitute \& Solve \\
\textbf{S} - Solution \& Statement \\
\end{left}

\vspace{0.5cm}

\textbf{Remember:} This systematic approach demonstrates your thinking process!
\end{frame}

\begin{frame}{G - Givens \& Diagram}
\begin{block}{What to Include}
\begin{itemize}
\item List all knowns and unknowns clearly
\item Example: $V = 16\,\text{V}$, $I_{\text{off}} = 0.40\,\text{A}$, $I_{\text{on}} = 0.90\,\text{A}$, $P_{\text{screen}} = ?$
\item Make a clear, labeled diagram
\end{itemize}
\end{block}

\begin{block}{Diagram Types}
\begin{itemize}
\item \textbf{Circuits:} Redraw the circuit clearly
\item \textbf{Forces:} Draw free-body diagrams
\item \textbf{Kinematics:} Sketch the motion path
\end{itemize}
\end{block}

\textbf{Key:} This directly addresses the "Proficient" criterion for clear organization.
\end{frame}

\begin{frame}{U - Unknowns \& Plan}
\begin{block}{State Your Strategy}
\begin{itemize}
\item Clearly state what you need to find
\item Explain how you will solve it
\end{itemize}
\end{block}

\begin{block}{Example Planning Statement}
"To find the power of the screen, I will first find the current used by the screen alone. Then I will use the power formula $P = IV$."
\end{block}

\textbf{Impact:} This single sentence elevates your response from "Developing" to "Proficient" by showing logical planning!

\end{frame}

\begin{frame}{E - Equations}
\begin{block}{Best Practices}
\begin{itemize}
\item Write base formulas \emph{before} plugging in numbers
\item Show the physics principles you're using
\item Examples: $P = IV$, $V = IR$, $F_{\text{net}} = ma$
\end{itemize}
\end{block}

\begin{block}{Why This Matters}
Demonstrates you understand the relevant physics concepts, not just arithmetic.
\end{block}

\end{frame}

\begin{frame}{S - Substitute \& Solve}
\begin{block}{Show Logical Steps}
\begin{itemize}
\item Work line-by-line, vertically down the page
\item Use units consistently in every step
\item Don't show scattered calculations
\end{itemize}
\end{block}

\begin{columns}
\begin{column}{0.5\textwidth}
\textcolor{green}{\textbf{Good Example:}}
\begin{align}
I_{\text{screen}} &= I_{\text{on}} - I_{\text{off}} \\
&= 0.90\,\text{A} - 0.40\,\text{A} \\
&= 0.50\,\text{A}
\end{align}
\end{column}
\begin{column}{0.5\textwidth}
\textcolor{red}{\textbf{Bad Example:}}
\begin{align}
0.9 - 0.4 = 0.5
\end{align}
\end{column}
\end{columns}

\vspace{1cm}

\textbf{Units are a major differentiator between "Developing" and "Proficient"!}
\end{frame}

\begin{frame}{S - Solution \& Statement}
\begin{block}{Final Presentation}
\begin{itemize}
\item \textbf{Box your final answer}
\item Write a concluding statement with units
\item Example: "Therefore, the power used by the screen alone is 8.0 W."
\end{itemize}
\end{block}

\begin{block}{Complete Solution Format}
\vspace{0.5cm}
\boxed{P_{\text{screen}} = 8.0\,\text{W}}
\\
\vspace{0.5cm}
\\
"Therefore, the power used by the screen alone is 8.0 W."
\end{block}

\end{frame}

\section{Achieving Proficient and Extending Levels}

\begin{frame}{Reaching the 'Extending' Level}
\textbf{Extending} = Sophisticated understanding and complete command of physics

\begin{block}{Three Key Strategies}
\begin{enumerate}
\item \textbf{Find Hidden Details:} State implied information explicitly
\item \textbf{Explain the Physics:} Don't just show math, explain why
\item \textbf{Check Your Answer:} Sense-check, units, alternative methods
\end{enumerate}
\end{block}

\begin{block}{Hidden Details Examples}
\begin{itemize}
\item "starts from rest" $\rightarrow$ $v_i = 0$
\item "on the moon" $\rightarrow$ use $g_{\text{moon}}$
\item "no atmosphere" $\rightarrow$ no air resistance
\end{itemize}
\end{block}
\end{frame}
% Add this to your preamble if you don't have it already


\begin{frame}{Proficient vs. Exemplary: Capacitor Circuit}

\begin{columns}[T] % Aligns the tops of the columns
    \begin{column}{0.3\textwidth}
        \textbf{Question:} For a given circuit, what is the voltage across and current through the capacitor a long time after the switch is closed?

    \end{column}
    \begin{column}{0.65\textwidth}
        \begin{block}{Proficient Answer}
            After a long time, the capacitor is fully charged. \\
            $I_C = 0$ A     $V_C = 12$ V
        \end{block}

       \begin{alertblock}{Extending Answer}
    After a long time, the capacitor is fully charged and acts like an \textbf{open circuit}.
    
    \begin{itemize}
        \item \textbf{Current ($I_C$):} Because the circuit is open, no current can flow. Therefore, $\mathbf{I_C = 0}$ \textbf{A}.
        
        \item \textbf{Voltage ($V_C$):} With zero current, there is no voltage drop across the resistor ($V_R = I \cdot R = 0$). All of the battery's voltage must be across the capacitor. Therefore, $\mathbf{V_C = 12}$ \textbf{V}.
    \end{itemize}
\end{alertblock}

    \end{column}
\end{columns}

\end{frame}

\section{Practice Examples}

\begin{frame}{I Do: Complete G.U.E.S.S. Example}
\textbf{Problem:} A laptop uses 0.40 A when off and 0.90 A when on. If the voltage is 16 V, what power does the screen use?
\pause
\vspace{0.5cm}

\textbf{G - Givens:} $V = 16\,\text{V}$, $I_{\text{off}} = 0.40\,\text{A}$, $I_{\text{on}} = 0.90\,\text{A}$ \\
\pause
\textbf{U - Plan:} Find current used by screen alone, then use $P = IV$ \\
\pause
\textbf{E - Equations:} $I_{\text{screen}} = I_{\text{on}} - I_{\text{off}}$, $P = IV$ \\
\pause
\textbf{S - Solve:} 
$I_{\text{screen}} = 0.90\,\text{A} - 0.40\,\text{A} = 0.50\,\text{A}$
$P = (16\,\text{V})(0.50\,\text{A}) = 8.0\,\text{W}$ \\
\pause
\textbf{S - Statement:} $\boxed{\therefore P_{\text{screen}} = 8.0\,\text{W}}$
\end{frame}


\section{Summary and Final Tips}

\begin{frame}{Summary: Your Path to Success}
\begin{block}{Remember the Strategy}
\begin{itemize}
\item \textbf{First 5 minutes:} Brain dump, survey, internalize goals
\item \textbf{Three-pass approach:} Quick wins → Deep dive → Finish line
\item \textbf{G.U.E.S.S. method:} Your systematic problem-solving framework
\end{itemize}
\end{block}

\begin{block}{Proficient → Extending}
\begin{itemize}
\item Find hidden details and state assumptions
\item Explain the physics, not just the math
\item Always check your answers multiple ways
\end{itemize}
\end{block}

\textbf{Final Reminder:} Show your thinking process clearly – that's what the rubric measures!
\end{frame}

\begin{frame}{Final Words of Encouragement}
\begin{center}
\Large
\textcolor{ds9gold}{\textbf{Stay Calm}} \\
\textcolor{ds9blue}{\textbf{Be Systematic}} \\
\textcolor{ds9red}{\textbf{Show What You Know}} \\
\end{center}



\begin{figure}
    \centering
    \includegraphics[width=0.5\linewidth]{addoil.jpg}
\end{figure}
\end{frame}

\end{document}