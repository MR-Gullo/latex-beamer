\documentclass[12pt]{article}

\usepackage{amsmath}
\usepackage{amssymb}
\usepackage{graphicx}
\usepackage[margin=1in]{geometry}
\usepackage{xcolor}
\usepackage{mdframed}
\usepackage{tcolorbox}
\usepackage{enumitem}
\usepackage{fancyhdr}
\usepackage{physics}

% Custom environments and commands
\newcommand{\conceptbox}[2]{
    \begin{tcolorbox}[colback=blue!5,colframe=blue!40,title={\textbf{#1}}]
        #2
    \end{tcolorbox}
}

\newcommand{\note}[1]{
    \begin{mdframed}[linecolor=gray!50]
        \textit{Note:} #1
    \end{mdframed}
}

\pagestyle{fancy}
\fancyhead[L]{Physics Laboratory Manual}
\fancyhead[R]{Conservation of Momentum}
\fancyfoot[C]{\thepage}

\begin{document}

\title{\textbf{Laboratory Investigation:\\ Conservation of Linear Momentum}}
\author{Physics Department}
\date{}
\maketitle

\section*{Introduction}
The conservation of linear momentum stands as one of physics' most profound and fundamental principles. This laboratory investigation guides students through a systematic exploration of momentum conservation in both elastic and inelastic collisions, providing hands-on experience with a principle that underlies much of classical mechanics.

\conceptbox{Core Learning Objectives}{
\begin{itemize}
    \item Verify the conservation of linear momentum experimentally
    \item Distinguish between elastic and inelastic collisions
    \item Develop precision measurement techniques
    \item Apply error analysis to real physical systems
\end{itemize}
}

\section{Theoretical Framework}

\subsection{Conservation of Linear Momentum}
In an isolated system, the total linear momentum remains constant regardless of the interactions between its components. Mathematically:

\begin{equation}
    \sum \vec{p}_{\text{initial}} = \sum \vec{p}_{\text{final}}
\end{equation}

For a two-body collision, this principle takes the specific form:
\begin{equation}
    m_1v_{1i} + m_2v_{2i} = m_1v_{1f} + m_2v_{2f}
\end{equation}

Where:
\begin{itemize}
    \item $m_1, m_2$ represent the masses of the colliding objects
    \item $v_{1i}, v_{2i}$ denote initial velocities
    \item $v_{1f}, v_{2f}$ represent final velocities
\end{itemize}

\subsection{Types of Collisions}
This investigation examines two fundamental types of collisions:

\conceptbox{Elastic Collisions}{
\begin{itemize}
    \item Both momentum and kinetic energy are conserved
    \item Objects separate after collision
    \item Implemented using magnetic bumpers
    \item Ideal case: perfect elasticity
\end{itemize}
}

\conceptbox{Inelastic Collisions}{
\begin{itemize}
    \item Only momentum is conserved
    \item Objects stick together after collision
    \item Implemented using nylon buttons
    \item Kinetic energy transforms to other forms
\end{itemize}
}

\section{Experimental Apparatus}

\subsection{Required Equipment}
\begin{itemize}
    \item llongwill\textsuperscript{\textregistered} photogate sensors (2)
    \item Multi-purpose mechanical track system
    \item Collision carts (2)
    \item Light blocking flags (width = 0.020 m)
    \item Digital balance (precision 0.1 g)
    \item Interchangeable bumpers (magnetic and nylon)
    \item Data acquisition system
\end{itemize}

\subsection{Measurement Technique}
Velocity measurements utilize photogate timing according to:
\begin{equation}
    v = \frac{\text{flag width}}{\text{gate time}} = \frac{0.020 \text{ m}}{t}
\end{equation}

\note{Proper alignment of photogates is crucial for accurate measurements. Ensure flags pass through gates perpendicularly.}

\section{Experimental Procedure}

\subsection{Part A: Elastic Collisions}
\begin{enumerate}
    \item Mount magnetic bumpers securely on both carts
    \item Measure and record masses: $m_1$ and $m_2$ (±0.1 g)
    \item Position photogates ensuring proper flag height alignment
    \item Place carts at opposite track ends
    \item Initialize data collection system
    \item Gently push carts toward collision point
    \item Record timing data for both pre- and post-collision passages
    \item Repeat measurement minimum three times
\end{enumerate}

\subsection{Part B: Inelastic Collisions}
\begin{enumerate}
    \item Replace magnetic bumpers with nylon buttons
    \item Position one cart between photogates
    \item Release second cart to achieve collision
    \item Record timing data for combined mass post-collision
    \item Repeat measurement minimum three times
\end{enumerate}

\section{Data Analysis}

\subsection{Required Calculations}
For each collision event, calculate:

\begin{enumerate}
    \item Initial momenta:
        \begin{equation}
            p_{1i} = m_1v_{1i}, \quad p_{2i} = m_2v_{2i}
        \end{equation}
    
    \item Final momenta:
        \begin{equation}
            p_{1f} = m_1v_{1f}, \quad p_{2f} = m_2v_{2f}
        \end{equation}
    
    \item Total momentum:
        \begin{equation}
            p_{\text{total}} = p_1 + p_2
        \end{equation}
    
    \item Relative error:
        \begin{equation}
            \eta = \frac{p_{\text{before}} - p_{\text{after}}}{p_{\text{before}}}
        \end{equation}
\end{enumerate}

\subsection{Error Analysis}
Consider and quantify these sources of uncertainty:
\begin{itemize}
    \item Mass measurements (±0.1 g)
    \item Timing precision (±0.001 s)
    \item Track friction effects
    \item Air resistance
    \item Alignment errors
\end{itemize}

\section{Expected Results}

\conceptbox{Anticipated Outcomes}{
\begin{itemize}
    \item Elastic collisions: momentum conservation within 3\% error
    \item Inelastic collisions: momentum conservation with combined mass
    \item Systematic variations due to friction and air resistance
    \item Greater uncertainty in elastic collision measurements
\end{itemize}
}

\section{Discussion Questions}
\begin{enumerate}
    \item How does the choice of bumper type affect energy conservation?
    \item What role does friction play in momentum conservation?
    \item Explain any systematic deviations from theoretical predictions
    \item Propose improvements to reduce experimental uncertainty
    \item How might results differ in a frictionless environment?
\end{enumerate}

\section{Laboratory Report Requirements}

Your report should include:

\begin{enumerate}
    \item Abstract summarizing methods and findings
    \item Complete data tables with all measurements
    \item Sample calculations showing analysis method
    \item Error analysis with uncertainty propagation
    \item Discussion of systematic errors
    \item Comparison of elastic and inelastic results
    \item Suggestions for experimental improvements
\end{enumerate}

\note{Include clear diagrams and graphs where appropriate. All figures should have captions and be referenced in the text.}
\newpage
\section*{Proficiency Levels}

\subsection*{Emerging}
\textbf{Description:} Beginning to grasp fundamental concepts of momentum conservation and basic experimental methods, requiring significant guidance.

\noindent\textbf{Skills and Abilities:}
\begin{itemize}[leftmargin=*]
    \item Demonstrates basic laboratory safety and equipment identification while requiring supervision for setup and operation
    \item Records raw timing data and performs simple momentum calculations ($\vec{p} = m\vec{v}$) with assistance
    \item Recognizes the conceptual difference between elastic and inelastic collisions
\end{itemize}

\subsection*{Developing}
\textbf{Description:} Shows growing understanding of momentum conservation principles and basic lab techniques, but needs support in application and analysis.

\noindent\textbf{Skills and Abilities:}
\begin{itemize}[leftmargin=*]
    \item Conducts experimental procedures with minimal guidance, including basic photogate alignment and data collection
    \item Calculates velocities and momenta from raw data, though may struggle with uncertainty analysis
    \item Interprets collision data qualitatively while beginning to apply quantitative analysis techniques
\end{itemize}

\subsection*{Proficient}
\textbf{Description:} Demonstrates solid comprehension of momentum conservation concepts and experimental methods, working independently with minimal support.

\noindent\textbf{Skills and Abilities:}
\begin{itemize}[leftmargin=*]
    \item Executes experimental procedures independently with proper technique and uncertainty analysis
    \item Analyzes collision data quantitatively, including momentum conservation calculations and error propagation
    \item Evaluates experimental validity through systematic error analysis and theoretical comparisons
\end{itemize}

\subsection*{Extending}
\textbf{Description:} Shows advanced understanding and analytical capability, exploring momentum conservation concepts beyond basic requirements.

\noindent\textbf{Skills and Abilities:}
\begin{itemize}[leftmargin=*]
    \item Synthesizes experimental results with advanced physics concepts, including energy conservation and non-ideal effects
    \item Proposes and implements methodological improvements to enhance experimental precision
    \item Extends analysis beyond basic requirements through creative data visualization and theoretical connections
\end{itemize}


\end{document}