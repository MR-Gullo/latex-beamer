\documentclass{beamer}

\usepackage{xeCJK}
% 所需宏包
\usepackage{amsmath}
\usepackage{physics}
\usepackage{graphicx}
\usepackage{siunitx}
\usepackage{xcolor}

% 定义DS9主题自定义颜色
\definecolor{ds9blue}{RGB}{25,25,112}
\definecolor{ds9gold}{RGB}{218,165,32}
\definecolor{ds9grey}{RGB}{105,105,105}
\definecolor{ds9red}{RGB}{178,34,34}

% 设置马德里主题与自定义颜色
\usetheme{Madrid}
\usecolortheme{whale}

\setbeamercolor{palette primary}{bg=ds9blue,fg=white}
\setbeamercolor{palette secondary}{bg=ds9grey,fg=white}
\setbeamercolor{palette tertiary}{bg=ds9gold,fg=black}
\setbeamercolor{palette quaternary}{bg=ds9red,fg=white}
\setbeamercolor{structure}{fg=ds9blue}
\setbeamercolor{title}{fg=ds9gold}
\setbeamercolor{subtitle}{fg=ds9gold}
\setbeamercolor{frametitle}{bg=ds9blue,fg=white}
\setbeamercolor{block title}{bg=ds9blue,fg=white}
\setbeamercolor{block body}{bg=ds9grey!20,fg=black}

% 标题信息
\title[电势与电容器]{PHYS12 第19章:电势与电场}
\subtitle{电势、电场与电容器}
\author[Gullo先生]{Gullo先生}
\date[2025年2月]{2025年2月}
\institute{物理系}

\begin{document}

% 标题页
\begin{frame}
    \titlepage
\end{frame}

% 大纲页
\begin{frame}
    \frametitle{大纲}
    \tableofcontents
\end{frame}

\section{引言}

\begin{frame}
    \frametitle{学习目标}
    通过本课程,您将能够:
    \begin{itemize}
        \item 定义电势并解释其与电势能的关系
        \item 计算电场中两点间的电势差
        \item 建立电场强度与电势梯度的联系
        \item 计算点电荷产生的电势
        \item 理解等势线的概念
        \item 解释电容器工作原理并计算电容值
        \item 确定串联/并联组合的等效电容
        \item 计算电容器存储的能量
    \end{itemize}
\end{frame}

\section{电势能与电势差}

\begin{frame}
    \frametitle{电势能 vs 电势}
    
    \begin{block}{电势能}
        \begin{itemize}
            \item 电荷在电场中具有的能量
            \item 取决于电场和电荷量
            \item 单位:焦耳(J)
        \end{itemize}
    \end{block}
    
    \begin{block}{电势}
        \begin{itemize}
            \item 单位电荷的电势能
            \item 与测试电荷无关
            \item 单位:伏特(V),1 V = 1 J/C
            \item 表征电场中某点的性质
        \end{itemize}
    \end{block}
\end{frame}

\begin{frame}
    \frametitle{电势差与电压}
    
    \begin{block}{电势差}
        \begin{itemize}
            \item 单位电荷从A点移动到B点的电势能变化量
            \item 通常称为电压
        \end{itemize}
    \end{block}
    
    \begin{align}
        \Delta V &= \frac{\Delta PE}{q} \\
        \Delta PE &= q \Delta V
    \end{align}

    \begin{itemize}
        \item 正电荷从低电势移向高电势需要做功
        \item 正电荷自然从高电势移向低电势(释放能量)
        \item 负电荷自然从低电势移向高电势
    \end{itemize}
\end{frame}

\begin{frame}
    \frametitle{电子伏特}
    
    \begin{block}{定义}
        电子伏特(eV)是基本电荷通过1V电势差获得的能量。
    \end{block}
    
    \begin{align}
        1 \text{ eV} &= (1.60 \times 10^{-19} \text{ C})(1 \text{ V}) \\
        &= 1.60 \times 10^{-19} \text{ J}
    \end{align}
    
    \begin{itemize}
        \item 原子物理和核物理常用单位
        \item 常用倍数:keV, MeV, GeV
        \item 示例:12V电池可给予电子12eV能量
    \end{itemize}
\end{frame}

\begin{frame}
    \frametitle{电场中的能量守恒}
    
    \begin{block}{机械能}
        \begin{itemize}
            \item 系统动能与势能之和
            \item $E_{机械能} = \text{动能} + \text{势能}$
            \item 在保守场中守恒
        \end{itemize}
    \end{block}
    
    \begin{block}{应用}
        \begin{itemize}
            \item 电荷在电场中移动时能量形式转换
            \item $\Delta \text{动能} = -\Delta \text{势能} = -q\Delta V$
            \item 可计算通过电势差加速后带电粒子的末速度
        \end{itemize}
    \end{block}
\end{frame}

\section{匀强电场中的电势}

\begin{frame}
    \frametitle{匀强电场中的电势}
    
    在匀强电场中(如平行板间):
    \begin{align}
        V_{AB} &= Ed \\
        E &= \frac{V_{AB}}{d}
    \end{align}
    其中:
    \begin{itemize}
        \item $E$:电场强度(V/m或N/C)
        \item $d$:AB间距离(m)
        \item $V_{AB}$:电势差(V)
    \end{itemize}
\end{frame}

\begin{frame}
    \frametitle{电场与电势的关系}
    
    \begin{block}{通用关系式}
        \begin{align}
            E &= -\frac{\Delta V}{\Delta s}
        \end{align}
        $\Delta s$为电势变化$\Delta V$对应的距离
    \end{block}
    
    \begin{itemize}
        \item 负号表示E指向电势降低方向
        \item 电场是电势的梯度(斜率)
        \item 单位验证:$\text{V/m} = \text{N/C}$
        \item 电场越强,电势梯度越陡峭
    \end{itemize}
\end{frame}

\section{点电荷的电势}

\begin{frame}
    \frametitle{点电荷的电势}
    
    \begin{block}{点电荷电势公式}
        距点电荷$Q$为$r$处的电势:
        \begin{align}
            V = k\frac{Q}{r}
        \end{align}
        其中$k = 9.0 \times 10^9 \text{ N}\cdot\text{m}^2/\text{C}^2$
    \end{block}
    
    \begin{itemize}
        \item 正电荷电势为正,负电荷为负
        \item 随距离$r$增大而减小
        \item 参考点:$r=\infty$时$V=0$
        \item 注意:$E = k\frac{Q}{r^2}$,故$E = -\frac{dV}{dr}$
    \end{itemize}
\end{frame}

\begin{frame}
    \frametitle{电势叠加原理}
    
    \begin{block}{核心概念}
        电势是标量,多电荷系统的总电势为代数和:
        \begin{align}
            V_{总} = k\sum_i \frac{Q_i}{r_i}
        \end{align}
    \end{block}
    
    \begin{itemize}
        \item 比矢量形式的电场叠加更简便
        \item 便于解决复杂静电问题
        \item 先计算总电势,再通过梯度求电场
    \end{itemize}
\end{frame}

\section{等势线}

\begin{frame}
    \frametitle{等势线与等势面}
    
    \begin{block}{定义}
        等势线是电势恒定的连续曲线。
    \end{block}
    
    \begin{itemize}
        \item 等势面是三维版本的等势线
        \item 沿等势线移动电荷不做功
        \item 等势线始终垂直于电场线
        \item 点电荷的等势面为同心球面
        \item 匀强电场的等势面为平行平面
    \end{itemize}
\end{frame}

\begin{frame}
    \frametitle{等势线与接地}
    
    \begin{block}{等势面特性}
        \begin{itemize}
            \item 静电平衡时导体表面为等势面
            \item 导体内部电场为零
        \end{itemize}
    \end{block}

    \begin{block}{接地}
        \begin{itemize}
            \item 用良导体将设备与大地连接
            \item 使导体电势固定为零(大地参考)
            \item 电气安全的重要措施
        \end{itemize}
    \end{block}
\end{frame}

\section{电容器与电介质}

\begin{frame}
    \frametitle{电容器原理}
    
    \begin{block}{定义}
        电容器是存储电荷与电能的装置。
    \end{block}
    
    \begin{itemize}
        \item 典型结构:两导体板间夹绝缘介质
        \item 接电源时极板出现等量异号电荷
        \item 电场主要集中于极板间
        \item 常见类型:平行板、柱形、球形
    \end{itemize}
\end{frame}

\begin{frame}
    \frametitle{电容}
    
    \begin{block}{电容定义}
        电容(C)表示单位电压存储的电荷量:
        \begin{align}
            C = \frac{Q}{V}
        \end{align}
    \end{block}
    
    \begin{itemize}
        \item 单位:法拉(F),1 F = 1 C/V
        \item 典型值:pF到μF量级
        \item 仅取决于几何结构与材料
        \item 与电荷/电压无关(线性电容)
    \end{itemize}
\end{frame}

\begin{frame}
    \frametitle{平行板电容器}
    
    \begin{block}{电容公式}
        真空/空气中平行板电容:
        \begin{align}
            C = \epsilon_0 \frac{A}{d}
        \end{align}
        其中:
        \begin{itemize}
            \item $\epsilon_0 = 8.85 \times 10^{-12}$ F/m(真空介电常数)
            \item $A$:极板面积
            \item $d$:极板间距
        \end{itemize}
    \end{block}
    
    \begin{itemize}
        \item 面积越大 → 电容越大
        \item 间距越小 → 电容越大
    \end{itemize}
\end{frame}

\begin{frame}
    \frametitle{电介质的影响}
    
    \begin{block}{电介质效应}
        插入电介质后:
        \begin{align}
            C = \kappa\epsilon_0 \frac{A}{d}
        \end{align}
        $\kappa$为材料介电常数
    \end{block}
    
    \begin{block}{常见介电常数}
        \begin{itemize}
            \item 空气:$\kappa \approx 1.00059$
            \item 纸张:$\kappa \approx 2-4$
            \item 玻璃:$\kappa \approx 4-10$
            \item 特氟龙:$\kappa \approx 2.1$
            \item 水:$\kappa \approx 80$
        \end{itemize}
    \end{block}
\end{frame}

\section{电容器的串联与并联}

\begin{frame}
    \frametitle{串联电容器}
    
    \begin{block}{串联公式}
        串联电容器的等效电容:
        \begin{align}
            \frac{1}{C_S} = \sum_i \frac{1}{C_i}
        \end{align}
    \end{block}
    
    \begin{itemize}
        \item 各电容器电荷量相同
        \item 总电压按电容分配
        \item 等效电容小于任一电容
        \item 类比电阻并联
    \end{itemize}
\end{frame}

\begin{frame}
    \frametitle{并联电容器}
    
    \begin{block}{并联公式}
        并联电容器的等效电容:
        \begin{align}
            C_P = \sum_i C_i
        \end{align}
    \end{block}
    
    \begin{itemize}
        \item 各电容器电压相同
        \item 总电荷按电容分配
        \item 等效电容大于任一电容
        \item 类比电阻串联
    \end{itemize}
\end{frame}

\begin{frame}
    \frametitle{混联电路解法}
    
    \begin{block}{解题策略}
        \begin{enumerate}
            \item 识别串联/并联部分
            \item 分步计算等效电容
            \item 综合结果求总电容
        \end{enumerate}
    \end{block}
\end{frame}

\section{电容器储能}

\begin{frame}
    \frametitle{电容器储能公式}
    
    \begin{block}{能量表达式}
        三种等效形式:
        \begin{align}
            E_{储} &= \frac{QV}{2} \\
            E_{储} &= \frac{CV^2}{2} \\
            E_{储} &= \frac{Q^2}{2C}
        \end{align}
    \end{block}
    
    \begin{itemize}
        \item 能量存储在极板间电场中
        \item 单位:焦耳(J)
        \item 充电过程需要做功
        \item 放电时可释放能量
    \end{itemize}
\end{frame}

\begin{frame}
    \frametitle{电容器应用}
    
    \begin{block}{典型应用}
        \begin{itemize}
            \item 能量存储(后备电源)
            \item 电源滤波
            \item 定时电路
            \item 电子电路耦合/去耦
            \item 相机闪光灯
            \item 医疗除颤器
            \item 触摸屏传感器
        \end{itemize}
    \end{block}
    
    \begin{block}{能量密度}
        电场能量密度:
        \begin{align}
            u = \frac{1}{2}\epsilon_0 E^2 \quad (\text{J/m³})
        \end{align}
    \end{block}
\end{frame}

\section{例题解析}

\begin{frame}
    \frametitle{"教师示范"例题——电子加速}
    
    \begin{block}{题目}
        真空管使用40kV加速电压使电子撞击铜板产生X射线。计算电子最大速度(非相对论近似)。
    \end{block}
    
    \begin{block}{解答}
        能量守恒:电势能转化为动能
        \begin{align}
            v &= \sqrt{\frac{2qV}{m}} = 1.17 \times 10^8 \text{ m/s}
        \end{align}
    \end{block}
\end{frame}

\begin{frame}
    \frametitle{"师生共解"例题——含电介质电容}
    
    \begin{block}{题目}
        计算面积为5m²、间距0.1mm的特氟龙介质($\kappa=2.1$)平行板电容。
    \end{block}
    
    \begin{block}{解答}
        \begin{align}
            C &= (2.1)(8.85 \times 10^{-12})\frac{5}{0.1 \times 10^{-3}} \\
            &= 0.929 \text{ μF}
        \end{align}
    \end{block}
\end{frame}

\begin{frame}
    \frametitle{"自主练习"例题——电容储能}
    
    \begin{block}{题目}
        180μF电容器充电至120V:
        \begin{enumerate}
            \item 存储电荷量?
            \item 存储能量?
        \end{enumerate}
    \end{block}
    
    \begin{block}{提示}
        \begin{itemize}
            \item 使用$Q = CV$
            \item 使用$E = \frac{1}{2}CV^2$
            \item 注意单位换算(μF=$10^{-6}$F)
        \end{itemize}
    \end{block}
\end{frame}

\begin{frame}
    \frametitle{本章总结}
    
    \begin{block}{核心公式}
        \begin{itemize}
            \item 电势:$V = \frac{PE}{q}$
            \item 电势差:$\Delta V = \frac{\Delta PE}{q}$
            \item 匀强电场:$E = \frac{V}{d}$
            \item 场强-电势关系:$E = -\frac{\Delta V}{\Delta s}$
            \item 点电荷电势:$V = k\frac{Q}{r}$
            \item 电容定义:$C = \frac{Q}{V}$
            \item 平行板电容:$C = \kappa\epsilon_0 \frac{A}{d}$
            \item 串联电容:$\frac{1}{C_S} = \sum \frac{1}{C_i}$
            \item 并联电容:$C_P = \sum C_i$
            \item 储能公式:$E = \frac{1}{2}CV^2$
        \end{itemize}
    \end{block}
\end{frame}

\end{document}