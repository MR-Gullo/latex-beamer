\documentclass[10pt]{article}
\usepackage[utf8]{inputenc}
\usepackage[T1]{fontenc}
\usepackage{graphicx}
\usepackage[export]{adjustbox}
\graphicspath{ {./images/} }
\usepackage{caption}
\usepackage{amsmath}
\usepackage{amsfonts}
\usepackage{amssymb}
\usepackage[version=4]{mhchem}
\usepackage{stmaryrd}

\begin{document}
\captionsetup{singlelinecheck=false}
\begin{figure}[h]
\begin{center}
  \includegraphics[max width=\textwidth]{4a0380a2-146f-4ef5-b324-77a3ec44b850-01}
\captionsetup{labelformat=empty}
\caption{Figure 6.1 This Australian Grand Prix Formula 1 race car moves in a circular path as it makes the turn. Its wheels also spin rapidly. The same physical principles are involved in both of these motions. (Richard Munckton).}
\end{center}
\end{figure}

\section*{Chapter Outline}
6.1 Angle of Rotation and Angular Velocity

\subsection*{6.2 Uniform Circular Motion}
\subsection*{6.3 Rotational Motion}
\section*{Introduction}
\section*{Teacher Support}
Teacher Support Before students begin this chapter, they may wish to review the concepts of distance, displacement, speed, velocity, acceleration, force, and Newton's laws of motion. Address any misconceptions about centrifugal force.

Point out that we come across circular motion in our everyday lives; for instance, a car tire spinning, a fan rotating, and so forth. This chapter is about the quantities that describe rotational motion and the relationships between them.

You may recall learning about various aspects of motion along a straight line: kinematics (where we learned about displacement, velocity, and acceleration),\\
projectile motion (a special case of two-dimensional kinematics), force, and Newton's laws of motion. In some ways, this chapter is a continuation of Newton's laws of motion. Recall that Newton's first law tells us that objects move along a straight line at constant speed unless a net external force acts on them. Therefore, if an object moves along a circular path, such as the car in the photo, it must be experiencing an external force. In this chapter, we explore both circular motion and rotational motion.

\subsection*{6.1 Angle of Rotation and Angular Velocity}
\section*{Section Learning Objectives}
By the end of this section, you will be able to do the following:

\begin{itemize}
  \item Describe the angle of rotation and relate it to its linear counterpart
  \item Describe angular velocity and relate it to its linear counterpart
  \item Solve problems involving angle of rotation and angular velocity
\end{itemize}

\section*{Teacher Support}
Teacher Support The learning objectives in this section will help your students master the following standards:

\begin{itemize}
  \item (4) Science concepts. The student knows and applies the laws governing motion in a variety of situations. The student is expected to:
  \item (C) analyze and describe accelerated motion in two dimensions using equations, including projectile and circular examples.
\end{itemize}

\section*{Section Key Terms}
\section*{Angle of Rotation}
What exactly do we mean by circular motion or rotation? Rotational motion is the circular motion of an object about an axis of rotation. We will discuss specifically circular motion and spin. Circular motion is when an object moves in a circular path. Examples of circular motion include a race car speeding around a circular curve, a toy attached to a string swinging in a circle around your head, or the circular loop-the-loop on a roller coaster. Spin is rotation about an axis that goes through the center of mass of the object, such as Earth rotating on its axis, a wheel turning on its axle, the spin of a tornado on its path of destruction, or a figure skater spinning during a performance at the Olympics. Sometimes, objects will be spinning while in circular motion, like the Earth spinning on its axis while revolving around the Sun, but we will focus on these two motions separately.

\section*{Teacher Support}
Teacher Support [BL][OL] Explain the difference between circular and rotational motions by using the Earth's rotation about its axis and its revolution about the Sun. Explain that Earth's rotation is slightly elliptical, although it is very nearly circular.\\[0pt]
[OL][AL] Ask students to come up with examples of circular motion.\\
When solving problems involving rotational motion, we use variables that are similar to linear variables (distance, velocity, acceleration, and force) but take into account the curvature or rotation of the motion. Here, we define the angle of rotation, which is the angular equivalence of distance; and angular velocity, which is the angular equivalence of linear velocity.

When objects rotate about some axis-for example, when the CD in Figure 6.2 rotates about its center-each point in the object follows a circular path.\\
\includegraphics[max width=\textwidth, center]{4a0380a2-146f-4ef5-b324-77a3ec44b850-04(1)}

Figure 6.2 All points on a CD travel in circular paths. The pits (dots) along a line from the center to the edge all move through the same angle \(\Delta \theta\) in time \(\Delta t\) -

The arc length, , is the distance traveled along a circular path. The radius of curvature, \(\mathbf{r}\), is the radius of the circular path. Both are shown in Figure 6.3.

\begin{figure}[h]
\begin{center}
  \includegraphics[max width=\textwidth]{4a0380a2-146f-4ef5-b324-77a3ec44b850-04}
\captionsetup{labelformat=empty}
\caption{Figure 6.3 The radius ( \(r\) ) of a circle is rotated through an angle \(\Delta \theta\). The arc length, \(\Delta s\), is the distance covered along the circumference.}
\end{center}
\end{figure}

Consider a line from the center of the CD to its edge. In a given time, each pit (used to record information) on this line moves through the same angle. The angle of rotation is the amount of rotation and is the angular analog of distance. The angle of rotation \(\Delta \theta\) is the arc length divided by the radius of curvature.\\
\(\Delta \theta=\frac{\Delta s}{r}\)

The angle of rotation is often measured by using a unit called the radian. (Radians are actually dimensionless, because a radian is defined as the ratio of two distances, radius and arc length.) A revolution is one complete rotation, where every point on the circle returns to its original position. One revolution covers \(2 \pi\) radians (or 360 degrees), and therefore has an angle of rotation of \(2 \pi\) radians, and an arc length that is the same as the circumference of the circle. We can convert between radians, revolutions, and degrees using the relationship

1 revolution \(=2 \pi \mathrm{rad}=360^{\circ}\). See Table 6.1 for the conversion of degrees to radians for some common angles.

\[
\begin{aligned}
2 \pi \mathrm{rad} & =360^{\circ} \\
1 \mathrm{rad} & =\frac{360^{\circ}}{2 \pi} \approx 57.3^{\circ}
\end{aligned}
\]

6.1

Table 6.1 Commonly Used Angles in Terms of Degrees and Radians

\section*{Angular Velocity}
\section*{Teacher Support}
Teacher Support [BL] Review displacement, speed, velocity, acceleration.\\[0pt]
[AL] Ask students whether or not velocity changes in uniform circular motion. What about speed? What about acceleration?

How fast is an object rotating? We can answer this question by using the concept of angular velocity. Consider first the angular speed ( \(\omega\) ) is the rate at which the angle of rotation changes. In equation form, the angular speed is\\
\(\omega=\frac{\Delta \theta}{\Delta t}\),\\
6.2\\
which means that an angular rotation ( \(\Delta \theta\) ) occurs in a time, \(\Delta t\). If an object rotates through a greater angle of rotation in a given time, it has a greater angular speed. The units for angular speed are radians per second (rad/s).

Now let's consider the direction of the angular speed, which means we now must call it the angular velocity. The direction of the angular velocity is along the axis of rotation. For an object rotating clockwise, the angular velocity points away\\
from you along the axis of rotation. For an object rotating counterclockwise, the angular velocity points toward you along the axis of rotation.

Angular velocity ( ) is the angular version of linear velocity \(\mathbf{v}\). Tangential velocity is the instantaneous linear velocity of an object in rotational motion. To get the precise relationship between angular velocity and tangential velocity, consider again a pit on the rotating CD. This pit moves through an arc length\\
\((\Delta s)\)\\
in a short time\\
\((\Delta t)\)\\
so its tangential speed is\\
\(v=\frac{\Delta s}{\Delta t}\).\\
6.3

From the definition of the angle of rotation, \(\Delta \theta=\frac{\Delta s}{r}\), we see that \(\Delta s=r \Delta \theta\). Substituting this into the expression for \(v\) gives\\
\(v=\frac{\mathrm{r} \Delta \theta}{\Delta t}=r \omega\).\\
The equation \(v=r \omega\) says that the tangential speed \(v\) is proportional to the distance \(r\) from the center of rotation. Consequently, tangential speed is greater for a point on the outer edge of the CD (with larger \(r\) ) than for a point closer to the center of the CD (with smaller \(r\) ). This makes sense because a point farther out from the center has to cover a longer arc length in the same amount of time as a point closer to the center. Note that both points will still have the same angular speed, regardless of their distance from the center of rotation. See Figure 6.4.

\begin{figure}[h]
\begin{center}
  \includegraphics[max width=\textwidth]{4a0380a2-146f-4ef5-b324-77a3ec44b850-07}
\captionsetup{labelformat=empty}
\caption{Figure 6.4 Points 1 and 2 rotate through the same angle ( \(\Delta \theta\) ), but point 2 moves through a greater arc length ( \(\Delta s_{2}\) ) because it is farther from the center of rotation.}
\end{center}
\end{figure}

\section*{Teacher Support}
Teacher Support [AL] Explain that the time period \(\Delta t\) in the equation that defines tangential velocity ( \(\mathbf{v}=\frac{\Delta s}{\Delta t}\) ) must be short so that the arc described by the moving object can be approximated as a straight line. This allows us to define the direction of the tangential velocity as being tangent to the circle. This approximation becomes increasingly accurate as \(\Delta t\) becomes increasingly small.

Now, consider another example: the tire of a moving car (see Figure 6.5). The faster the tire spins, the faster the car moves-large \(\omega\) means large \(v\) because \(v=r \omega\). Similarly, a larger-radius tire rotating at the same angular velocity, , will produce a greater linear (tangential) velocity, \(\mathbf{v}\), for the car. This is because a larger radius means a longer arc length must contact the road, so the car must move farther in the same amount of time.

\begin{figure}[h]
\begin{center}
  \includegraphics[max width=\textwidth]{4a0380a2-146f-4ef5-b324-77a3ec44b850-08}
\captionsetup{labelformat=empty}
\caption{Figure 6.5 A car moving at a velocity, \(\mathbf{v}\), to the right has a tire rotating with angular velocity . The speed of the tread of the tire relative to the axle is \(v\), the same as if the car were jacked up and the wheels spinning without touching the road. Directly below the axle, where the tire touches the road, the tire tread moves backward with respect to the axle with tangential velocity \(v=r \omega\), where \(r\) is the tire radius. Because the road is stationary with respect to this point of the tire, the car must move forward at the linear velocity \(\mathbf{v}\). A larger angular velocity for the tire means a greater linear velocity for the car.}
\end{center}
\end{figure}

However, there are cases where linear velocity and tangential velocity are not equivalent, such as a car spinning its tires on ice. In this case, the linear velocity will be less than the tangential velocity. Due to the lack of friction under the tires of a car on ice, the arc length through which the tire treads move is greater than the linear distance through which the car moves. It's similar to running on a treadmill or pedaling a stationary bike; you are literally going nowhere fast.

\section*{Tips For Success}
Angular velocity and tangential velocity \(\mathbf{v}\) are vectors, so we must include magnitude and direction. The direction of the angular velocity is along the axis of rotation, and points away from you for an object rotating clockwise, and toward you for an object rotating counterclockwise. In mathematics this is described by the right-hand rule. Tangential velocity is usually described as up, down, left, right, north, south, east, or west, as shown in Figure 6.6.

\begin{figure}[h]
\begin{center}
  \includegraphics[max width=\textwidth]{4a0380a2-146f-4ef5-b324-77a3ec44b850-09}
\captionsetup{labelformat=empty}
\caption{Figure 6.6 As the fly on the edge of an old-fashioned vinyl record moves in a circle, its instantaneous velocity is always at a tangent to the circle. The direction of the angular velocity is into the page this case.}
\end{center}
\end{figure}

\section*{Watch Physics}
Relationship between Angular Velocity and Speed This video reviews the definition and units of angular velocity and relates it to linear speed. It also shows how to convert between revolutions and radians.

Click to view content\\
Click to view content\\
For an object traveling in a circular path at a constant angular speed, would the linear speed of the object change if the radius of the path increases?\\
a. Yes, because tangential speed is independent of the radius.\\
b. Yes, because tangential speed depends on the radius.\\
c. No, because tangential speed is independent of the radius.\\
d. No, because tangential speed depends on the radius.

\section*{Solving Problems Involving Angle of Rotation and Angular Velocity}
\section*{Snap Lab}
Measuring Angular Speed In this activity, you will create and measure uniform circular motion and then contrast it with circular motions with different radii.

\begin{itemize}
  \item One string ( 1 m long)
  \item One object (two-hole rubber stopper) to tie to the end
  \item One timer
\end{itemize}

Procedure

\begin{enumerate}
  \item Tie an object to the end of a string.
  \item Swing the object around in a horizontal circle above your head (swing from your wrist). It is important that the circle be horizontal!
  \item Maintain the object at uniform speed as it swings.
  \item Measure the angular speed of the object in this manner. Measure the time it takes in seconds for the object to travel 10 revolutions. Divide that time by 10 to get the angular speed in revolutions per second, which you can convert to radians per second.
  \item What is the approximate linear speed of the object?
  \item Move your hand up the string so that the length of the string is 90 cm . Repeat steps 2-5.
  \item Move your hand up the string so that its length is 80 cm . Repeat steps 2-5.
  \item Move your hand up the string so that its length is 70 cm . Repeat steps 2-5.
  \item Move your hand up the string so that its length is 60 cm . Repeat steps 2-5
  \item Move your hand up the string so that its length is 50 cm . Repeat steps 2-5
  \item Make graphs of angular speed vs. radius (i.e. string length) and linear speed vs. radius. Describe what each graph looks like.
\end{enumerate}

If you swing an object slowly, it may rotate at less than one revolution per second. What would be the revolutions per second for an object that makes one revolution in five seconds? What would be its angular speed in radians per second?\\
a. The object would spin at \(\backslash \operatorname{frac}\{1\}\{5\} \backslash, \backslash \operatorname{text}\{\mathrm{rev} / \mathrm{s}\}\). The angular speed of the object would be \(\backslash \operatorname{frac}\{2 \backslash \mathrm{pi}\}\{5\} \backslash, \backslash \operatorname{text}\{\mathrm{rad} / \mathrm{s}\}\).\\
b. The object would spin at \(\backslash \operatorname{frac}\{1\}\{5\} \backslash, \backslash \operatorname{text}\{\mathrm{rev} / \mathrm{s}\}\). The angular speed of the object would be \(\backslash\) frac \(\{\backslash \mathrm{pi}\}\{5\} \backslash, \backslash \operatorname{text}\{\mathrm{rad} / \mathrm{s}\}\).\\
c. The object would spin at \(5 \backslash, \backslash \operatorname{text}\{\mathrm{rev} / \mathrm{s}\}\). The angular speed of the object would be \(10 \backslash \mathrm{pi} \backslash, \backslash \operatorname{text}\{\mathrm{rad} / \mathrm{s}\}\).\\
d. The object would spin at \(5 \backslash, \backslash \operatorname{text}\{\mathrm{rev} / \mathrm{s}\}\). The angular speed of the object would be \(5 \backslash \mathrm{pi} \backslash, \backslash \operatorname{text}\{\mathrm{rad} / \mathrm{s}\}\).

Now that we have an understanding of the concepts of angle of rotation and angular velocity, we'll apply them to the real-world situations of a clock tower and a spinning tire.

\section*{Worked Example}
Angle of rotation at a Clock Tower The clock on a clock tower has a radius of 1.0 m . (a) What angle of rotation does the hour hand of the clock travel through when it moves from 12 p.m. to 3 p.m.? (b) What's the arc length along the outermost edge of the clock between the hour hand at these two times?

\section*{Strategy}
We can figure out the angle of rotation by multiplying a full revolution ( \(2 \pi\) radians) by the fraction of the 12 hours covered by the hour hand in going from 12 to 3 . Once we have the angle of rotation, we can solve for the arc length by rearranging the equation \(\Delta \theta=\frac{\Delta s}{r}\) since the radius is given.\\
Solution to (a)\\
In going from 12 to 3 , the hour hand covers \(1 / 4\) of the 12 hours needed to make a complete revolution. Therefore, the angle between the hour hand at 12 and at 3 is \(\frac{1}{4} \times 2 \mathrm{rad}=\frac{1}{2}\) (i.e., 90 degrees).\\
Solution to (b)\\
Rearranging the equation\\
\(\Delta \theta=\frac{\Delta s}{r}\),\\
6.4\\
we get\\
\(\Delta s=r \Delta \theta\).\\
6.5

Inserting the known values gives an arc length of

\[
\begin{aligned}
\Delta s & =(1.0 \mathrm{~m})\left(\frac{\pi}{2} \mathrm{rad}\right) \\
& =1.6 \mathrm{~m}
\end{aligned}
\]

6.6

Discussion\\
We were able to drop the radians from the final solution to part (b) because radians are actually dimensionless. This is because the radian is defined as the ratio of two distances (radius and arc length). Thus, the formula gives an answer in units of meters, as expected for an arc length.

\section*{Worked Example}
How Fast Does a Car Tire Spin? Calculate the angular speed of a 0.300 m radius car tire when the car travels at \(15.0 \mathrm{~m} / \mathrm{s}\) (about \(54 \mathrm{~km} / \mathrm{h}\) ). See Figure 6.5.

\section*{Strategy}
In this case, the speed of the tire tread with respect to the tire axle is the same as the speed of the car with respect to the road, so we have \(v=15.0 \mathrm{~m} / \mathrm{s}\). The radius of the tire is \(r=0.300 \mathrm{~m}\). Since we know \(v\) and \(r\), we can rearrange the equation \(v=r \omega\), to get \(\omega=\frac{v}{r}\) and find the angular speed.

Solution\\
To find the angular speed, we use the relationship: \(\omega=\frac{v}{r}\).\\
Inserting the known quantities gives

\[
\begin{aligned}
\omega & =\frac{15.0 \mathrm{~m} / \mathrm{s}}{0.300 \mathrm{~m}} \\
& =50.0 \mathrm{rad} / \mathrm{s} .
\end{aligned}
\]

\section*{6.7}
Discussion\\
When we cancel units in the above calculation, we get \(50.0 / \mathrm{s}\) (i.e., 50.0 per second, which is usually written as \(50.0 \mathrm{~s}^{-1}\) ). But the angular speed must have units of \(\mathrm{rad} / \mathrm{s}\). Because radians are dimensionless, we can insert them into the\\
answer for the angular speed because we know that the motion is circular. Also note that, if an earth mover with much larger tires, say 1.20 m in radius, were moving at the same speed of \(15.0 \mathrm{~m} / \mathrm{s}\), its tires would rotate more slowly. They would have an angular speed of

\[
\begin{aligned}
\omega & =\frac{15.0 \mathrm{~m} / \mathrm{s}}{1.20 \mathrm{~m}} \\
& =12.5 \mathrm{rad} / \mathrm{s}
\end{aligned}
\]

\section*{6.8}
\section*{Practice Problems}
1.

What is the angle in degrees between the hour hand and the minute hand of a clock showing 9:00 a.m.?\\
a. \(0^{\circ}\)\\
b. \(90^{\circ}\)\\
c. \(180^{\circ}\)\\
d. \(360^{\circ}\)\\
2.

What is the approximate value of the arc length between the hour hand and the minute hand of a clock showing 10:00 a.m if the radius of the clock is 0.2 m ?\\
a. 0.1 m\\
b. 0.2 m\\
c. 0.3 m\\
d. 0.6 m

\section*{Check Your Understanding}
3.

What is circular motion?\\
a. Circular motion is the motion of an object when it follows a linear path.\\
b. Circular motion is the motion of an object when it follows a zigzag path.\\
c. Circular motion is the motion of an object when it follows a circular path.\\
d. Option D is confusing as a distractor\\
4.

What is meant by radius of curvature when describing rotational motion?\\
a. The radius of curvature is the radius of a circular path.\\
b. The radius of curvature is the diameter of a circular path.\\
c. The radius of curvature is the circumference of a circular path.\\
d. The radius of curvature is the area of a circular path.\\
5.

What is angular velocity?\\
a. Angular velocity is the rate of change of the diameter of the circular path.\\
b. Angular velocity is the rate of change of the angle subtended by the circular path.\\
c. Angular velocity is the rate of change of the area of the circular path.\\
d. Angular velocity is the rate of change of the radius of the circular path.\\
6.

What equation defines angular velocity, when r is the radius of curvature, is the angle, and t is the time?\\
a. \(\backslash\) omega \(=\backslash\) frac \(\{\backslash\) Delta \(\backslash\) theta \(\}\{\backslash\) Delta \(\{\mathrm{t}\}\}\)\\
b. \(\backslash\) omega \(=\backslash\) frac \(\{\backslash\) Delta \(\{\mathrm{t}\}\}\{\backslash\) Delta \(\backslash\) theta \(\}\)\\
c. \(\backslash\) omega \(=\backslash\) frac \(\{\backslash\) Delta \(\{\mathrm{r}\}\}\{\backslash\) Delta \(\{\mathrm{t}\}\}\)\\
d. \(\backslash\) omega \(=\backslash\) frac \(\{\backslash\) Delta \(\{\mathrm{t}\}\}\{\backslash\) Delta \(\{\mathrm{r}\}\}\)\\
7.

Identify three examples of an object in circular motion.\\
a. an artificial satellite orbiting the Earth, a race car moving in the circular race track, and a top spinning on its axis\\
b. an artificial satellite orbiting the Earth, a race car moving in the circular race track, and a ball tied to a string being swung in a circle around a person's head\\
c. Earth spinning on its own axis, a race car moving in the circular race track, and a ball tied to a string being swung in a circle around a person's head\\
d. Earth spinning on its own axis, blades of a working ceiling fan, and a top spinning on its own axis\\
8.

What is the relative orientation of the radius and tangential velocity vectors of an object in uniform circular motion?\\
a. Tangential velocity vector is always parallel to the radius of the circular path along which the object moves.\\
b. Tangential velocity vector is always perpendicular to the radius of the circular path along which the object moves.\\
c. Tangential velocity vector is always at an acute angle to the radius of the circular path along which the object moves.\\
d. Tangential velocity vector is always at an obtuse angle to the radius of the circular path along which the object moves.

\section*{Teacher Support}
Teacher Support Use the Check Your Understanding questions to assess whether students master the learning objectives of this section. If students are struggling with a specific objective, the formative assessment will help identify which objective is causing the problem and direct students to the relevant content.

\subsection*{6.2 Uniform Circular Motion}
\section*{Section Learning Objectives}
By the end of this section, you will be able to do the following:

\begin{itemize}
  \item Describe centripetal acceleration and relate it to linear acceleration
  \item Describe centripetal force and relate it to linear force
  \item Solve problems involving centripetal acceleration and centripetal force
\end{itemize}

\section*{Teacher Support}
Teacher Support The learning objectives in this section will help your students master the following standards:

\begin{itemize}
  \item (4) Science concepts. The student knows and applies the laws governing motion in a variety of situations. The student is expected to:
  \item (C) analyze and describe accelerated motion in two dimensions using equations, including projectile and circular examples.
  \item (D) calculate the effect of forces on objects, including the law of inertia, the relationship between force and acceleration, and the nature of force pairs between objects.
\end{itemize}

In addition, the High School Physics Laboratory Manual addresses content in this section in the lab titled: Circular and Rotational Motion, as well as the following standards:

\begin{itemize}
  \item (4) Science concepts. The student knows and applies the laws governing motion in a variety of situations. The student is expected to:
  \item (C) analyze and describe accelerated motion in two dimensions using equations, including projectile and circular examples.
\end{itemize}

\section*{Section Key Terms}
\section*{Centripetal Acceleration}
\section*{Teacher Support}
Teacher Support [BL][OL] Review uniform circular motion. Ask students to give examples of circular motion. Review linear acceleration.

In the previous section, we defined circular motion. The simplest case of circular motion is uniform circular motion, where an object travels a circular path at a constant speed. Note that, unlike speed, the linear velocity of an object in circular motion is constantly changing because it is always changing direction. We know\\
from kinematics that acceleration is a change in velocity, either in magnitude or in direction or both. Therefore, an object undergoing uniform circular motion is always accelerating, even though the magnitude of its velocity is constant.

You experience this acceleration yourself every time you ride in a car while it turns a corner. If you hold the steering wheel steady during the turn and move at a constant speed, you are executing uniform circular motion. What you notice is a feeling of sliding (or being flung, depending on the speed) away from the center of the turn. This isn't an actual force that is acting on you-it only happens because your body wants to continue moving in a straight line (as per Newton's first law) whereas the car is turning off this straight-line path. Inside the car it appears as if you are forced away from the center of the turn. This fictitious force is known as the centrifugal force. The sharper the curve and the greater your speed, the more noticeable this effect becomes.

\section*{Teacher Support}
Teacher Support [BL][OL][AL] Demonstrate circular motion by tying a weight to a string and twirling it around. Ask students what would happen if you suddenly cut the string? In which direction would the object travel? Why? What does this say about the direction of acceleration? Ask students to give examples of when they have come across centripetal acceleration.

Figure 6.7 shows an object moving in a circular path at constant speed. The direction of the instantaneous tangential velocity is shown at two points along the path. Acceleration is in the direction of the change in velocity; in this case it points roughly toward the center of rotation. (The center of rotation is at the center of the circular path). If we imagine \(\Delta s\) becoming smaller and smaller, then the acceleration would point exactl toward the center of rotation, but this case is hard to draw. We call the acceleration of an object moving in uniform circular motion the centripetal acceleration \(\mathbf{a}_{\mathrm{c}}\) because centripetal means center seeking.

\begin{figure}[h]
\begin{center}
  \includegraphics[max width=\textwidth]{4a0380a2-146f-4ef5-b324-77a3ec44b850-18}
\captionsetup{labelformat=empty}
\caption{Figure 6.7 The directions of the velocity of an object at two different points are shown, and the change in velocity \(\Delta \mathbf{v}\) is seen to point approximately toward the center of curvature (see small inset). For an extremely small value of \(\Delta s\), \(\Delta \mathbf{v}\) points exactly toward the center of the circle (but this is hard to draw). Because \(\mathbf{a}_{\mathrm{c}}=\Delta \mathbf{v} / \Delta t\), the acceleration is also toward the center, so \(\mathbf{a}_{c}\) is called centripetal acceleration.}
\end{center}
\end{figure}

\section*{Teacher Support}
Teacher Support Consider Figure 6.7. The figure shows an object moving in a circular path at constant speed and the direction of the instantaneous velocity of two points along the path. Acceleration is in the direction of the change in velocity and points toward the center of rotation. This is strictly true only as \(\Delta s\) tends to zero.

Now that we know that the direction of centripetal acceleration is toward the center of rotation, let's discuss the magnitude of centripetal acceleration. For an object traveling at speed \(v\) in a circular path with radius \(r\), the magnitude of centripetal acceleration is\\
\(\mathbf{a}_{\mathrm{c}}=\frac{v^{2}}{r}\).\\
Centripetal acceleration is greater at high speeds and in sharp curves (smaller radius), as you may have noticed when driving a car, because the car actually pushes you toward the center of the turn. But it is a bit surprising that \(\mathbf{a}_{\mathrm{c}}\)\\
is proportional to the speed squared. This means, for example, that the acceleration is four times greater when you take a curve at \(100 \mathrm{~km} / \mathrm{h}\) than at 50 \(\mathrm{km} / \mathrm{h}\).

We can also express \(\mathbf{a}_{\mathrm{c}}\) in terms of the magnitude of angular velocity. Substituting \(v=r \omega\) into the equation above, we get \(a_{c}=\frac{(r \omega)^{2}}{r}=r \omega^{2}\). Therefore, the magnitude of centripetal acceleration in terms of the magnitude of angular velocity is\\
\(\mathbf{a}_{c}=r \omega^{2}\).\\
6.9

\section*{Tips For Success}
The equation expressed in the form \(a_{\mathrm{c}}=r^{2}\) is useful for solving problems where you know the angular velocity rather than the tangential velocity.

\section*{Virtual Physics}
Ladybug Motion in 2D In this simulation, you experiment with the position, velocity, and acceleration of a ladybug in circular and elliptical motion. Switch the type of motion from linear to circular and observe the velocity and acceleration vectors. Next, try elliptical motion and notice how the velocity and acceleration vectors differ from those in circular motion.

Click to view content

\section*{Grasp Check}
In uniform circular motion, what is the angle between the acceleration and the velocity? What type of acceleration does a body experience in the uniform circular motion?\\
a. The angle between acceleration and velocity is \(0^{\circ}\), and the body experiences linear acceleration.\\
b. The angle between acceleration and velocity is \(0^{\circ}\), and the body experiences centripetal acceleration.\\
c. The angle between acceleration and velocity is \(90^{\circ}\), and the body experiences linear acceleration.\\
d. The angle between acceleration and velocity is \(90^{\circ}\), and the body experiences centripetal acceleration.

\section*{Centripetal Force}
\section*{Teacher Support}
Teacher Support [BL][OL][AL] Using the same demonstration as before, ask\\
students to predict the relationships between the quantities of angular velocity, centripetal acceleration, mass, centripetal force. Invite students to experiment by using various lengths of string and different weights.

Because an object in uniform circular motion undergoes constant acceleration (by changing direction), we know from Newton's second law of motion that there must be a constant net external force acting on the object.

Any force or combination of forces can cause a centripetal acceleration. Just a few examples are the tension in the rope on a tether ball, the force of Earth's gravity on the Moon, the friction between a road and the tires of a car as it goes around a curve, or the normal force of a roller coaster track on the cart during a loop-the-loop.

Any net force causing uniform circular motion is called a centripetal force. The direction of a centripetal force is toward the center of rotation, the same as for centripetal acceleration. According to Newton's second law of motion, a net force causes the acceleration of mass according to \(\mathbf{F}_{\text {net }}=m \mathbf{a}\). For uniform circular motion, the acceleration is centripetal acceleration: \(\mathbf{a}=\mathbf{a}_{\mathrm{c}}\). Therefore, the magnitude of centripetal force, \(\mathbf{F}_{\mathrm{c}}\), is \(\mathbf{F}_{\mathrm{c}}=m \mathbf{a}_{\mathrm{c}}\).

By using the two different forms of the equation for the magnitude of centripetal acceleration, \(\mathbf{a}_{\mathrm{c}}=v^{2} / r\) and \(\mathbf{a}_{c}=r \omega^{2}\), we get two expressions involving the magnitude of the centripetal force \(\mathbf{F}_{\mathrm{c}}\). The first expression is in terms of tangential speed, the second is in terms of angular speed: \(\mathbf{F}_{\mathrm{c}}=m \frac{v^{2}}{r}\) and \(\mathbf{F}_{\mathrm{c}}=m r \omega^{2}\).

Both forms of the equation depend on mass, velocity, and the radius of the circular path. You may use whichever expression for centripetal force is more convenient. Newton's second law also states that the object will accelerate in the same direction as the net force. By definition, the centripetal force is directed towards the center of rotation, so the object will also accelerate towards the center. A straight line drawn from the circular path to the center of the circle will always be perpendicular to the tangential velocity. Note that, if you solve the first expression for \(r\), you get\\
\(r=\frac{m v^{2}}{\mathbf{F}_{\mathrm{c}}}\).\\
From this expression, we see that, for a given mass and velocity, a large centripetal force causes a small radius of curvature - that is, a tight curve.

\begin{figure}[h]
\begin{center}
  \includegraphics[max width=\textwidth]{4a0380a2-146f-4ef5-b324-77a3ec44b850-21}
\captionsetup{labelformat=empty}
\caption{Figure 6.8 In this figure, the frictional force \(\boldsymbol{f}\) serves as the centripetal force \(\mathbf{F}_{\mathrm{c}}\). Centripetal force is perpendicular to tangential velocity and causes uniform circular motion. The larger the centripetal force \(\mathbf{F}_{\mathrm{c}}\), the smaller is the radius of curvature \(r\) and the sharper is the curve. The lower curve has the same velocity \(\mathbf{v}\), but a larger centripetal force \(\mathbf{F}_{\mathrm{c}}\) produces a smaller radius \(r^{\prime}\).}
\end{center}
\end{figure}

\section*{Watch Physics}
Centripetal Force and Acceleration Intuition This video explains why a centripetal force creates centripetal acceleration and uniform circular motion.

It also covers the difference between speed and velocity and shows examples of uniform circular motion.

Click to view content

\section*{Teacher Support}
\section*{Teacher Support}
\section*{Misconception Alert}
Some students might be confused between centripetal force and centrifugal force. Centrifugal force is not a real force but the result of an accelerating reference frame, such as a turning car or the spinning Earth. Centrifugal force refers to a fictional center eeing force.

Click to view content\\
Imagine that you are swinging a yoyo in a vertical clockwise circle in front of you, perpendicular to the direction you are facing. If the string breaks just as the yoyo reaches its bottommost position, nearest the floor. What will happen to the yoyo after the string breaks?\\
a. The yoyo will fly inward in the direction of the centripetal force.\\
b. The yoyo will fly outward in the direction of the centripetal force.\\
c. The yoyo will fly to the left in the direction of the tangential velocity.\\
d. The yoyo will fly to the right in the direction of the tangential velocity.

\section*{Solving Centripetal Acceleration and Centripetal Force Problems}
To get a feel for the typical magnitudes of centripetal acceleration, we'll do a lab estimating the centripetal acceleration of a tennis racket and then, in our first Worked Example, compare the centripetal acceleration of a car rounding a curve to gravitational acceleration. For the second Worked Example, we'll calculate the force required to make a car round a curve.

\section*{Snap Lab}
Estimating Centripetal Acceleration In this activity, you will measure the swing of a golf club or tennis racket to estimate the centripetal acceleration of the end of the club or racket. You may choose to do this in slow motion. Recall that the equation for centripetal acceleration is \(\mathbf{a}_{\mathrm{c}}=\frac{v^{2}}{r}\) or \(\mathbf{a}_{c}=r \omega^{2}\).

\begin{itemize}
  \item One tennis racket or golf club
  \item One timer
  \item One ruler or tape measure
\end{itemize}

Procedure

\begin{enumerate}
  \item Work with a partner. Stand a safe distance away from your partner as he or she swings the golf club or tennis racket.
  \item Describe the motion of the swing-is this uniform circular motion? Why or why not?
  \item Try to get the swing as close to uniform circular motion as possible. What adjustments did your partner need to make?
  \item Measure the radius of curvature. What did you physically measure?
  \item By using the timer, find either the linear or angular velocity, depending on which equation you decide to use.
  \item What is the approximate centripetal acceleration based on these measurements? How accurate do you think they are? Why? How might you and your partner make these measurements more accurate?
\end{enumerate}

\section*{Teacher Support}
Teacher Support The swing of the golf club or racket can be made very close to uniform circular motion. For this, the person would have to move it at a constant speed, without bending their arm. The length of the arm plus the length of the club or racket is the radius of curvature. Accuracy of measurements of angular velocity and angular acceleration will depend on resolution of the timer used and human observational error. The swing of the golf club or racket can be made very close to uniform circular motion. For this, the person would have to move it at a constant speed, without bending their arm. The length of the arm plus the length of the club or racket is the radius of curvature. Accuracy of measurements of angular velocity and angular acceleration will depend on resolution of the timer used and human observational error.

\section*{Grasp Check}
Was it more useful to use the equation\\
\(a_{c}=\frac{v^{2}}{r}\)\\
or\\
\(a_{c}=r \omega^{2}\)\\
in this activity? Why?\\
a. It should be simpler to use

\begin{itemize}
  \item \(a_{c}=r \omega^{2}\)\\
because measuring angular velocity through observation would be easier.\\
b. It should be simpler to use
  \item \(a_{c}=\frac{v^{2}}{r}\)\\
because measuring tangential velocity through observation would be easier.\\
c. It should be simpler to use
  \item \(a_{c}=r \omega^{2}\)\\
because measuring angular velocity through observation would be difficult.\\
d. It should be simpler to use
  \item \(a_{c}=\frac{v^{2}}{r}\)\\
because measuring tangential velocity through observation would be difficult.
\end{itemize}

\section*{Worked Example}
Comparing Centripetal Acceleration of a Car Rounding a Curve with Acceleration Due to Gravity A car follows a curve of radius 500 m at a speed of \(25.0 \mathrm{~m} / \mathrm{s}\) (about \(90 \mathrm{~km} / \mathrm{h}\) ). What is the magnitude of the car's centripetal acceleration? Compare the centripetal acceleration for this fairly gentle curve taken at highway speed with acceleration due to gravity ( \(g\) ).\\
\includegraphics[max width=\textwidth, center]{4a0380a2-146f-4ef5-b324-77a3ec44b850-24}

\section*{Strategy}
Because linear rather than angular speed is given, it is most convenient to use the expression \(\mathbf{a}_{\mathrm{c}}=\frac{v^{2}}{r}\) to find the magnitude of the centripetal acceleration.

\section*{Solution}
Entering the given values of \(v=25.0 \mathrm{~m} / \mathrm{s}\) and \(r=500 \mathrm{~m}\) into the expression for \(\mathbf{a}_{\mathbf{c}}\) gives

\[
\begin{aligned}
\mathbf{a}_{\mathbf{c}} & =\frac{v^{2}}{r} \\
& =\frac{(25.0 \mathrm{~m} / \mathrm{s})^{2}}{500 \mathrm{~m}} \\
& =1.25 \mathrm{~m} / \mathrm{s}^{2} .
\end{aligned}
\]

Discussion\\
To compare this with the acceleration due to gravity ( \(g=9.80 \mathrm{~m} / \mathrm{s}^{2}\) ), we take the ratio \(\mathbf{a}_{\mathrm{c}} / g=\left(1.25 \mathrm{~m} / \mathrm{s}^{2}\right) /\left(9.80 \mathrm{~m} / \mathrm{s}^{2}\right)=0.128\). Therefore, \(\mathbf{a}_{\mathrm{c}}=0.128 g\), which means that the centripetal acceleration is about one tenth the acceleration due to gravity.

\section*{Worked Example}
\section*{Frictional Force on Car Tires Rounding a Curve}
a. Calculate the centripetal force exerted on a 900 kg car that rounds a 600m -radius curve on horizontal ground at \(25.0 \mathrm{~m} / \mathrm{s}\).\\
b. Static friction prevents the car from slipping. Find the magnitude of the frictional force between the tires and the road that allows the car to round the curve without sliding off in a straight line.\\
\includegraphics[max width=\textwidth, center]{4a0380a2-146f-4ef5-b324-77a3ec44b850-26}

\section*{Strategy and Solution for (a)}
We know that \(\mathbf{F}_{\mathrm{c}}=m \frac{v^{2}}{r}\). Therefore,

\[
\begin{aligned}
\mathbf{F}_{\mathrm{c}} & =m \frac{v^{2}}{r} \\
& =\frac{(900 \mathrm{~kg})(25.0 \mathrm{~m} / \mathrm{s})^{2}}{600 \mathrm{~m}} \\
& =938 \mathrm{~N} .
\end{aligned}
\]

\section*{Strategy and Solution for (b)}
The image above shows the forces acting on the car while rounding the curve. In this diagram, the car is traveling into the page as shown and is turning to the left. Friction acts toward the left, accelerating the car toward the center of the curve. Because friction is the only horizontal force acting on the car, it provides all of the centripetal force in this case. Therefore, the force of friction is the centripetal force in this situation and points toward the center of the curve.\\
\(f=\mathbf{F}_{\mathrm{c}}=938 \mathrm{~N}\)\\
Discussion\\
Since we found the force of friction in part (b), we could also solve for the coefficient of friction, since \(f=\mu_{\mathrm{s}} \mathrm{N}=\mu_{\mathrm{s}} m g\).

\section*{Practice Problems}
9.

What is the centripetal acceleration felt by the passengers of a car moving at \(12 \mathrm{~m} / \mathrm{s}\) along a curve with radius 2.0 m ?\\
a. \(3 \mathrm{~m} / \mathrm{s}^{2}\)\\
b. \(6 \mathrm{~m} / \mathrm{s}^{2}\)\\
c. \(36 \mathrm{~m} / \mathrm{s}^{2}\)\\
d. \(72 \mathrm{~m} / \mathrm{s}^{2}\)\\
10.

Calculate the centripetal acceleration of an object following a path with a radius of a curvature of 0.2 m and at an angular velocity of \(5 \mathrm{rad} / \mathrm{s}\).\\
a. \(1 \mathrm{~m} / \mathrm{s}\)\\
b. \(5 \mathrm{~m} / \mathrm{s}\)\\
c. \(1 \mathrm{~m} / \mathrm{s}^{2}\)\\
d. \(5 \mathrm{~m} / \mathrm{s}^{2}\)

\section*{Check Your Understanding}
11.

What is uniform circular motion?\\
a. Uniform circular motion is when an object accelerates on a circular path at a constantly increasing velocity.\\
b. Uniform circular motion is when an object travels on a circular path at a variable acceleration.\\
c. Uniform circular motion is when an object travels on a circular path at a constant speed.\\
d. Uniform circular motion is when an object travels on a circular path at a variable speed.\\
12.

What is centripetal acceleration?\\
a. The acceleration of an object moving in a circular path and directed radially toward the center of the circular orbit\\
b. The acceleration of an object moving in a circular path and directed tangentially along the circular path\\
c. The acceleration of an object moving in a linear path and directed in the direction of motion of the object\\
d. The acceleration of an object moving in a linear path and directed in the direction opposite to the motion of the object\\
13.

Is there a net force acting on an object in uniform circular motion?\\
a. Yes, the object is accelerating, so a net force must be acting on it.\\
b. Yes, because there is no acceleration.\\
c. No, because there is acceleration.\\
d. No, because there is no acceleration.\\
14.

Identify two examples of forces that can cause centripetal acceleration.\\
a. The force of Earth's gravity on the moon and the normal force\\
b. The force of Earth's gravity on the moon and the tension in the rope on an orbiting tetherball\\
c. The normal force and the force of friction acting on a moving car\\
d. The normal force and the tension in the rope on a tetherball

\section*{Teacher Support}
Teacher Support Use the Check Your Understanding questions to assess whether students master the learning objectives of this section. If students are struggling with a specific objective, the formative assessment will help identify which objective is causing the problem and direct students to the relevant content.

\subsection*{6.3 Rotational Motion}
\section*{Section Learning Objectives}
By the end of this section, you will be able to do the following:

\begin{itemize}
  \item Describe rotational kinematic variables and equations and relate them to their linear counterparts
  \item Describe torque and lever arm
  \item Solve problems involving torque and rotational kinematics
\end{itemize}

\section*{Teacher Support}
Teacher Support The learning objectives in this section will help your students master the following standards:

\begin{itemize}
  \item (4) Science concepts. The student knows and applies the laws governing motion in a variety of situations. The student is expected to:
  \item (C) analyze and describe accelerated motion in two dimensions using equations, including projectile and circular examples.
  \item (D) calculate the effect of forces on objects, including the law of inertia, the relationship between force and acceleration, and the nature of force pairs between objects.
\end{itemize}

In addition, the High School Physics Laboratory Manual addresses content in this section in the lab titled: Circular and Rotational Motion, as well as the following standards:

\begin{itemize}
  \item (4) Science concepts. The student knows and applies the laws governing motion in a variety of situations. The student is expected to:
  \item (D) calculate the effect of forces on objects, including the law of inertia, the relationship between force and acceleration, and the nature of force pairs between objects.
\end{itemize}

\section*{Section Key Terms}
\section*{Rotational Kinematics}
\section*{Teacher Support}
Teacher Support [BL][OL] Review linear kinematic equations.

\section*{Misconception Alert}
Students may get confused between deceleration and increasing acceleration in the negative direction.

In the section on uniform circular motion, we discussed motion in a circle at constant speed and, therefore, constant angular velocity. However, there are times when angular velocity is not constant-rotational motion can speed up, slow down, or reverse directions. Angular velocity is not constant when a spinning skater pulls in her arms, when a child pushes a merry-go-round to make it rotate, or when a CD slows to a halt when switched off. In all these cases, angular acceleration occurs because the angular velocity changes. The faster the change occurs, the greater is the angular acceleration. Angular acceleration is the rate of change of angular velocity. In equation form, angular acceleration is

\[
=\frac{\Delta}{\Delta t},
\]

where \(\Delta\) is the change in angular velocity and \(\Delta t\) is the change in time. The units of angular acceleration \(\operatorname{are}(\mathrm{rad} / \mathrm{s}) / \mathrm{s}\), or \(\mathrm{rad} / \mathrm{s}^{2}\). If increases, then is positive. If decreases, then is negative. Keep in mind that, by convention, counterclockwise is the positive direction and clockwise is the negative direction. For example, the skater in Figure 6.9 is rotating counterclockwise as seen from above, so her angular velocity is positive. Acceleration would be negative, for example, when an object that is rotating counterclockwise slows down. It would be positive when an object that is rotating counterclockwise speeds up.

\begin{figure}[h]
\begin{center}
  \includegraphics[max width=\textwidth]{4a0380a2-146f-4ef5-b324-77a3ec44b850-31}
\captionsetup{labelformat=empty}
\caption{Figure 6.9 A figure skater spins in the counterclockwise direction, so her angular velocity is normally considered to be positive. (Luu, Wikimedia Commons)}
\end{center}
\end{figure}

The relationship between the magnitudes of tangential acceleration, \(\mathbf{a}\), and angular acceleration,

\[
, \mathrm{is} \mathbf{a}=r \text { or }=\frac{\mathbf{a}}{r} .
\]

These equations mean that the magnitudes of tangential acceleration and angular acceleration are directly proportional to each other. The greater the angular acceleration, the larger the change in tangential acceleration, and vice versa. For example, consider riders in their pods on a Ferris wheel at rest. A Ferris wheel with greater angular acceleration will give the riders greater tangential acceleration because, as the Ferris wheel increases its rate of spinning, it also increases its tangential velocity. Note that the radius of the spinning object also matters. For example, for a given angular acceleration , a smaller Ferris wheel leads to a smaller tangential acceleration for the riders.

\section*{Tips For Success}
Tangential acceleration is sometimes denoted \(\mathbf{a}_{t}\). It is a linear acceleration in a direction tangent to the circle at the point of interest in circular or rotational motion. Remember that tangential acceleration is parallel to the tangential velocity (either in the same direction or in the opposite direction.) Centripetal acceleration is always perpendicular to the tangential velocity.

So far, we have defined three rotational variables: \(\theta\), , and . These are the angular versions of the linear variables \(x, \mathbf{v}\), and \(\mathbf{a}\). The following equations in the table represent the magnitude of the rotational variables and only when the radius is constant and perpendicular to the rotational variable. Table 6.2 shows how they are related.

Table 6.2 Rotational and Linear Variables\\
We can now begin to see how rotational quantities like \(\theta\), and are related to each other. For example, if a motorcycle wheel that starts at rest has a large angular acceleration for a fairly long time, it ends up spinning rapidly and rotates through many revolutions. Putting this in terms of the variables, if the wheel's angular acceleration is large for a long period of time \(t\), then the final angular velocity and angle of rotation \(\theta\) are large. In the case of linear motion, if an object starts at rest and undergoes a large linear acceleration, then it has a large final velocity and will have traveled a large distance.

The kinematics of rotational motion describes the relationships between the angle of rotation, angular velocity, angular acceleration, and time. It only describes motion - it does not include any forces or masses that may affect rotation (these are part of dynamics). Recall the kinematics equation for linear motion: \(\mathbf{v}=\mathbf{v}_{0}+\mathbf{a} t\) (constant \(\mathbf{a}\) ).

As in linear kinematics, we assume \(\mathbf{a}\) is constant, which means that angular acceleration is also a constant, because \(\mathbf{a}=r\). The equation for the kinematics relationship between , , and \(t\) is

\[
={ }_{0}+t(\text { constant }),
\]

where \({ }_{0}\) is the initial angular velocity. Notice that the equation is identical to the linear version, except with angular analogs of the linear variables. In fact, all of the linear kinematics equations have rotational analogs, which are given in Table 6.3. These equations can be used to solve rotational or linear kinematics problem in which \(\mathbf{a}\) and are constant.

\section*{Table 6.3 Equations for Rotational Kinematics}
In these equations, \({ }_{0}\) and \(\mathbf{v}_{0}\) are initial values, \(t_{0}\) is zero, and the average angular velocity - and average velocity \(\mathbf{v}\) are\\
\({ }^{-}=\frac{0^{+}}{2} \operatorname{and} \overline{\mathbf{v}}=\frac{\mathbf{v}_{0}+\mathbf{v}}{2}\).

\subsection*{6.11}
\section*{Fun In Physics}
\section*{Storm Chasing}
\begin{center}
\includegraphics[max width=\textwidth]{4a0380a2-146f-4ef5-b324-77a3ec44b850-33}
\end{center}

Figure 6.10 Tornadoes descend from clouds in funnel-like shapes that spin violently. (Daphne Zaras, U.S. National Oceanic and Atmospheric Administration)

Storm chasers tend to fall into one of three groups: Amateurs chasing tornadoes as a hobby, atmospheric scientists gathering data for research, weather watchers for news media, or scientists having fun under the guise of work. Storm chasing is a dangerous pastime because tornadoes can change course rapidly with little warning. Since storm chasers follow in the wake of the destruction left by tornadoes, changing flat tires due to debris left on the highway is common. The most active part of the world for tornadoes, called tornado alley, is in the central United States, between the Rocky Mountains and Appalachian Mountains.

Tornadoes are perfect examples of rotational motion in action in nature. They come out of severe thunderstorms called supercells, which have a column of air rotating around a horizontal axis, usually about four miles across. The difference in wind speeds between the strong cold winds higher up in the atmosphere in the jet stream and weaker winds traveling north from the Gulf of Mexico causes the column of rotating air to shift so that it spins around a vertical axis, creating a tornado.

Tornadoes produce wind speeds as high as \(500 \mathrm{~km} / \mathrm{h}\) (approximately 300 miles/h), particularly at the bottom where the funnel is narrowest because the rate of rotation increases as the radius decreases. They blow houses away as if they were made of paper and have been known to pierce tree trunks with pieces of straw.

What is the physics term for the eye of the storm? Why would winds be weaker at the eye of the tornado than at its outermost edge?\\
a. The eye of the storm is the center of rotation. Winds are weaker at the eye of a tornado because tangential velocity is directly proportional to radius of curvature.\\
b. The eye of the storm is the center of rotation. Winds are weaker at the eye of a tornado because tangential velocity is inversely proportional to radius of curvature.\\
c. The eye of the storm is the center of rotation. Winds are weaker at the eye of a tornado because tangential velocity is directly proportional to the square of the radius of curvature.\\
d. The eye of the storm is the center of rotation. Winds are weaker at the eye of a tornado because tangential velocity is inversely proportional to the square of the radius of curvature.

\section*{Torque}
If you have ever spun a bike wheel or pushed a merry-go-round, you know that force is needed to change angular velocity. The farther the force is applied from the pivot point (or fulcrum), the greater the angular acceleration. For example, a door opens slowly if you push too close to its hinge, but opens easily if you push far from the hinges. Furthermore, we know that the more massive the door is, the more slowly it opens; this is because angular acceleration is inversely proportional to mass. These relationships are very similar to the\\
relationships between force, mass, and acceleration from Newton's second law of motion. Since we have already covered the angular versions of distance, velocity and time, you may wonder what the angular version of force is, and how it relates to linear force.

The angular version of force is torque, which is the turning effectiveness of a force. See Figure 6.11. The equation for the magnitude of torque is\\
\(=r \mathbf{F} \sin \theta\),\\
where \(r\) is the magnitude of the lever arm, \(\mathbf{F}\) is the magnitude of the linear force, and \(\theta\) is the angle between the lever arm and the force. The lever arm is the vector from the point of rotation (pivot point or fulcrum) to the location where force is applied. Since the magnitude of the lever arm is a distance, its units are in meters, and torque has units of N m . Torque is a vector quantity and has the same direction as the angular acceleration that it produces.

\begin{figure}[h]
\begin{center}
  \includegraphics[max width=\textwidth]{4a0380a2-146f-4ef5-b324-77a3ec44b850-35}
\captionsetup{labelformat=empty}
\caption{Figure 6.11 A man pushes a merry-go-round at its edge and perpendicular to the lever arm to achieve maximum torque.}
\end{center}
\end{figure}

Applying a stronger torque will produce a greater angular acceleration. For example, the harder the man pushes the merry-go-round in Figure 6.11, the faster it accelerates. Furthermore, the more massive the merry-go-round is, the slower it accelerates for the same torque. If the man wants to maximize the effect of his force on the merry-go-round, he should push as far from the center as possible to get the largest lever arm and, therefore, the greatest torque and angular acceleration. Torque is also maximized when the force is applied perpendicular to the lever arm.

\section*{Teacher Support}
Teacher Support [BL][OL][AL] Demonstrate the physical relationships be-\\
tween torque, force, angle of application of force, and length of lever arm by using levers of different lengths. Help students make the connections between the physical observations and mathematical relationships. For instance, torque is maximum when the force is applied exactly perpendicular to the lever arm because \(\sin \theta=1\) for \(\theta=90\) degrees.

\section*{Solving Rotational Kinematics and Torque Problems}
Just as linear forces can balance to produce zero net force and no linear acceleration, the same is true of rotational motion. When two torques of equal magnitude act in opposing directions, there is no net torque and no angular acceleration, as you can see in the following video. If zero net torque acts on a system spinning at a constant angular velocity, the system will continue to spin at the same angular velocity.

\section*{Watch Physics}
Introduction to Torque This video defines torque in terms of moment arm (which is the same as lever arm). It also covers a problem with forces acting in opposing directions about a pivot point. (At this stage, you can ignore Sal's references to work and mechanical advantage.)

Click to view content\\
If the net torque acting on the ruler from the example was positive instead of zero, what would this say about the angular acceleration? What would happen to the ruler over time?\\
a. The ruler is in a state of rotational equilibrium so it will not rotate about its center of mass. Thus, the angular acceleration will be zero.\\
b. The ruler is not in a state of rotational equilibrium so it will not rotate about its center of mass. Thus, the angular acceleration will be zero.\\
c. The ruler is not in a state of rotational equilibrium so it will rotate about its center of mass. Thus, the angular acceleration will be non-zero.\\
d. The ruler is in a state of rotational equilibrium so it will rotate about its center of mass. Thus, the angular acceleration will be non-zero.

Now let's look at examples applying rotational kinematics to a fishing reel and the concept of torque to a merry-go-round.

\section*{Worked Example}
Calculating the Time for a Fishing Reel to Stop Spinning A deep-sea fisherman uses a fishing rod with a reel of radius 4.50 cm . A big fish takes the bait and swims away from the boat, pulling the fishing line from his fishing reel. As the fishing line unwinds from the reel, the reel spins at an angular velocity of \(220 \mathrm{rad} / \mathrm{s}\). The fisherman applies a brake to the spinning reel, creating an angular acceleration of \(-300 \mathrm{rad} / \mathrm{s}^{2}\). How long does it take the reel to come to a stop?\\
\includegraphics[max width=\textwidth, center]{4a0380a2-146f-4ef5-b324-77a3ec44b850-37}

\section*{Strategy}
We are asked to find the time \(t\) for the reel to come to a stop. The magnitude of the initial angular velocity is \({ }_{0}=220 \mathrm{rad} / \mathrm{s}\), and the magnitude of the final angular velocity \(=0\). The signed magnitude of the angular acceleration is \(=-300 \mathrm{rad} / \mathrm{s}^{2}\), where the minus sign indicates that it acts in the direction opposite to the angular velocity. Looking at the rotational kinematic equations, we see all quantities but \(t\) are known in the equation \(=_{0}+t\), making it the easiest equation to use for this problem.\\
Solution\\
The equation to use is \(={ }_{0}+t\).\\
We solve the equation algebraically for \(t\), and then insert the known values.

\[
\begin{aligned}
t & =\frac{\omega-\omega_{0}}{\alpha} \\
& =\frac{0-220 \mathrm{rad} / \mathrm{s}}{-300 \mathrm{rad} / \mathrm{s}^{2}} \\
& =0.733 \mathrm{~s}
\end{aligned}
\]

\subsection*{6.12}
Discussion\\
The time to stop the reel is fairly small because the acceleration is fairly large. Fishing lines sometimes snap because of the forces involved, and fishermen often let the fish swim for a while before applying brakes on the reel. A tired fish will be slower, requiring a smaller acceleration and therefore a smaller force.

\section*{Worked Example}
Calculating the Torque on a Merry-Go-Round Consider the man pushing the playground merry-go-round in Figure 6.11. He exerts a force of 250 N at the edge of the merry-go-round and perpendicular to the radius, which is\\
1.50 m . How much torque does he produce? Assume that friction acting on the merry-go-round is negligible.

\section*{Strategy}
To find the torque, note that the applied force is perpendicular to the radius and that friction is negligible.

Solution

\[
\begin{aligned}
\tau & =r \mathbf{F} \sin \theta \\
& =(1.50 \mathrm{~m})(250 \mathrm{~N}) \sin \left(\frac{\pi}{2}\right) \\
& =375 \mathrm{~N} \cdot \mathrm{~m}
\end{aligned}
\]

6.13

Discussion\\
The man maximizes the torque by applying force perpendicular to the lever arm, so that \(\theta=\frac{{ }_{2}}{2}\) and \(\sin \theta=1\). The man also maximizes his torque by pushing at the outer edge of the merry-go-round, so that he gets the largest-possible lever arm.

\section*{Practice Problems}
15.

How much torque does a person produce if he applies a \(12 \backslash, \backslash \operatorname{text}\{\mathrm{~N}\}\) force \(1.0 \backslash, \backslash \operatorname{text}\{\mathrm{~m}\}\) away from the pivot point, perpendicularly to the lever \(\operatorname{arm}\) ?\\
a. \(\backslash \operatorname{frac}\{1\}\{144\} \backslash, \backslash \operatorname{text}\{\mathrm{N}-\mathrm{m}\}\)\\
b. \(\backslash \operatorname{frac}\{1\}\{12\} \backslash, \backslash \operatorname{text}\{\mathrm{N}-\mathrm{m}\}\)\\
c. \(12 \backslash, \backslash \operatorname{text}\{\mathrm{~N}-\mathrm{m}\}\)\\
d. \(144 \backslash, \backslash \operatorname{text}\{\mathrm{~N}-\mathrm{m}\}\)\\
16.

An object's angular velocity changes from \(3 \mathrm{rad} / \mathrm{s}\) clockwise to \(8 \mathrm{rad} / \mathrm{s}\) clockwise in 5 s . What is its angular acceleration?\\
a. \(0.6 \mathrm{rad} / \mathrm{s}^{2}\)\\
b. \(1.6 \mathrm{rad} / \mathrm{s}^{2}\)\\
c. \(1 \mathrm{rad} / \mathrm{s}^{2}\)\\
d. \(5 \mathrm{rad} / \mathrm{s}^{2}\)

\section*{Check Your Understanding}
17.

What is angular acceleration?\\
a. Angular acceleration is the rate of change of the angular displacement.\\
b. Angular acceleration is the rate of change of the angular velocity.\\
c. Angular acceleration is the rate of change of the linear displacement.\\
d. Angular acceleration is the rate of change of the linear velocity.\\
18.

What is the equation for angular acceleration, ? Assume is the angle, is the angular velocity, and \(t\) is time.\\
a. \(\alpha=\frac{\Delta \omega}{\Delta t}\)\\
b. \(\alpha=\Delta \omega \Delta t\)\\
c. \(\alpha=\frac{\Delta \theta}{\Delta t}\)\\
d. \(\alpha=\Delta \theta \Delta t\)\\
19.

Which of the following best describes torque?\\
a. It is the rotational equivalent of a force.\\
b. It is the force that affects linear motion.\\
c. It is the rotational equivalent of acceleration.\\
d. It is the acceleration that affects linear motion.\\
20.

What is the equation for torque?\\
a. \(\langle\) tau \(=\{\mathrm{F} \backslash, \cos \backslash\) theta \(\} \backslash,\{\mathrm{r}\}\)\\
b. \(\backslash \mathrm{tau}=\backslash \operatorname{frac}\{\mathrm{F} \backslash \sin \backslash\) theta \(\}\{\mathrm{r}\}\)\\
c. \(\backslash \mathrm{tau}=\mathrm{rF} \backslash!\backslash \cos \backslash\) theta\\
d. \(\backslash\) tau \(=\mathrm{rF} \backslash!\backslash \sin \backslash\) theta

\section*{Teacher Support}
Teacher Support Use the Check Your Understanding questions to assess whether students master the learning objectives of this section. If students are struggling with a specific objective, these questions will help identify which objective is causing the problem and direct students to the relevant content.

\section*{Key Terms}
angle of rotation the ratio of the arc length to the radius of curvature of a circular path\\
angular acceleration the rate of change of angular velocity with time\\
angular velocity ( ) the rate of change in the angular position of an object following a circular path\\
arc length ( \(\Delta s\) ) the distance traveled by an object along a circular path\\
centrifugal force a fictitious force that acts in the direction opposite the centripetal acceleration\\
centripetal acceleration the acceleration of an object moving in a circle, directed toward the center of the circle\\
centripetal force any force causing uniform circular motion\\
circular motion the motion of an object along a circular path\\
kinematics of rotational motion the relationships between rotation angle, angular velocity, angular acceleration, and time\\
lever arm the distance between the point of rotation (pivot point) and the location where force is applied\\
radius of curvature the distance between the center of a circular path and the path\\
rotational motion the circular motion of an object about an axis of rotation\\
spin rotation about an axis that goes through the center of mass of the object\\
tangential acceleration the acceleration in a direction tangent to the circular path of motion and in the same direction or opposite direction as the tangential velocity\\
tangential velocity the instantaneous linear velocity of an object in circular or rotational motion\\
torque the effectiveness of a force to change the rotational speed of an object\\
uniform circular motion the motion of an object in a circular path at constant speed

\section*{Key Equations}
6.1 Angle of Rotation and Angular Velocity

\subsection*{6.2 Uniform Circular Motion}
\subsection*{6.3 Rotational Motion}
\section*{Section Summary}
\subsection*{6.1 Angle of Rotation and Angular Velocity}
\begin{itemize}
  \item Circular motion is motion in a circular path.
  \item The angle of rotation \(\Delta\) is defined as the ratio of the arc length to the radius of curvature.
  \item The arc length \(\Delta s\) is the distance traveled along a circular path and \(r\) is the radius of curvature of the circular path.
  \item The angle of rotation \(\Delta\) is measured in units of radians (rad), where \(2 \mathrm{rad}=360=1\) revolution.
  \item Angular velocity is the rate of change of an angle, where a rotation \(\Delta\) occurs in a time \(\Delta\).
  \item The units of angular velocity are radians per second (rad/s).
  \item Tangential speed \(v\) and angular speed are related by \(\quad=r\), and tangential velocity has units of \(\mathrm{m} / \mathrm{s}\).
  \item The direction of angular velocity is along the axis of rotation, toward (away) from you for clockwise (counterclockwise) motion.
\end{itemize}

\subsection*{6.2 Uniform Circular Motion}
\begin{itemize}
  \item Centripetal acceleration \(\mathbf{a}_{\mathrm{c}}\) is the acceleration experienced while in uniform circular motion.
  \item Centripetal acceleration force is a center-seeking force that always points toward the center of rotation, perpendicular to the linear velocity, in the same direction as the net force, and in the direction opposite that of the radius vector.
  \item The standard unit for centripetal acceleration is \(\mathrm{m} / \mathrm{s}^{2}\).
  \item Centripetal force \(\mathbf{F}_{\mathrm{c}}\) is any net force causing uniform circular motion.
\end{itemize}

\subsection*{6.3 Rotational Motion}
\begin{itemize}
  \item Kinematics is the description of motion.
  \item The kinematics of rotational motion describes the relationships between rotation angle, angular velocity, angular acceleration, and time.
  \item Torque is the effectiveness of a force to change the rotational speed of an object. Torque is the rotational analog of force.
  \item The lever arm is the distance between the point of rotation (pivot point) and the location where force is applied.
  \item Torque is maximized by applying force perpendicular to the lever arm and at a point as far as possible from the pivot point or fulcrum. If torque is zero, angular acceleration is zero.
\end{itemize}

\end{document}