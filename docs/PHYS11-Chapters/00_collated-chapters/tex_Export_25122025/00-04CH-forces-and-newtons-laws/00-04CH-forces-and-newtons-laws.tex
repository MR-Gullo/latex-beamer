\documentclass[10pt]{article}
\usepackage[utf8]{inputenc}
\usepackage[T1]{fontenc}
\usepackage{graphicx}
\usepackage[export]{adjustbox}
\graphicspath{ {./images/} }
\usepackage{caption}
\usepackage{amsmath}
\usepackage{amsfonts}
\usepackage{amssymb}
\usepackage[version=4]{mhchem}
\usepackage{stmaryrd}

\title{Applying Newton s Second Law }

\author{}
\date{}


\begin{document}
\maketitle
\captionsetup{singlelinecheck=false}
\begin{figure}[h]
\begin{center}
  \includegraphics[max width=\textwidth]{245726cb-221c-4c44-8efe-f3bac2e19c79-01}
\captionsetup{labelformat=empty}
\caption{Figure 4.1 Newton's laws of motion describe the motion of the dolphin's path. (Credit: Jin Jang)}
\end{center}
\end{figure}

\section*{Chapter Outline}
4.1 Force\\
4.2 Newton's First Law of Motion: Inertia\\
4.3 Newton's Second Law of Motion\\
4.4 Newton's Third Law of Motion

\section*{Introduction}
\section*{Teacher Support}
Teacher Support Before students begin this chapter, they may wish to review the concepts of distance, displacement, speed, velocity, acceleration, scalars, vectors, representing vectors, units of acceleration, and acceleration due to gravity. Explain that an object that is not moving is often described in physics as being at rest.

\section*{Teacher Support}
Teacher Support Point out that we come across motion in our everyday lives; for instance, a dolphin jumping out of water as shown in the photo. There\\
are simple laws of physics that govern motion. These laws are universal; that is, they apply to every object in the universe. Much of the work done in describing motion was done by Sir Isaac Newton. This chapter is about motion, the causes of motion, and the universal laws of motion.

Isaac Newton (1642-1727) was a natural philosopher; a great thinker who combined science and philosophy to try to explain the workings of nature on Earth and in the universe. His laws of motion were just one part of the monumental work that has made him legendary. The development of Newton's laws marks the transition from the Renaissance period of history to the modern era. This transition was characterized by a revolutionary change in the way people thought about the physical universe. Drawing upon earlier work by scientists Galileo Galilei and Johannes Kepler, Newton's laws of motion allowed motion on Earth and in space to be predicted mathematically. In this chapter you will learn about force as well as Newton's first, second, and third laws of motion.

\subsection*{4.1 Force}
\section*{Section Learning Objectives}
By the end of this section, you will be able to do the following:

\begin{itemize}
  \item Differentiate between force, net force, and dynamics
  \item Draw a free-body diagram
\end{itemize}

\section*{Teacher Support}
Teacher Support The learning objectives in this section will help your students master the following standards:

\begin{itemize}
  \item (4) Science concepts. The student knows and applies the laws governing motion in a variety of situations. The student is expected to:
  \item (C) analyze and describe accelerated motion in two dimensions using equations, including projectile and circular examples;
  \item (E) develop and interpret free-body diagrams.\\[0pt]
[BL][OL] Point out that objects at rest tend to stay at rest. A ball, for example, moves only when pushed or pulled. The action of pushing or pulling is the application of force. Force applied to an object changes its motion.\\[0pt]
[AL] Start a discussion about force and motion. Ask students what would happen if more than one force is applied to an object. Take a heavy object such as a desk for demonstration. Ask one student to push it from one side. Explain how force and motion work. Now ask a second student to push it in the opposite direction. Ask students why no motion occurs, even though the first student applies the same amount of force. Introduce the concept of adding forces.
\end{itemize}

\section*{Section Key Terms}
\section*{Defining Force and Dynamics}
\section*{Teacher Support}
Teacher Support [OL] Explain that the word dynamics comes from a Greek word meaning power. Also point out that the word dynamics is singular, like the word physics.\\[0pt]
[BL][OL] You may want to introduce the terms system, external force, and internal force.\\[0pt]
[AL] Explain that both magnitude and direction must be considered when talking about forces.

\section*{Teacher Demonstration}
By using physical objects, demonstrate how different forces acting together can be additive if they act in the same direction or cancel one another if they act in opposite directions. Explain the terms acting on and being acted on.

Force is the cause of motion, and motion draws our attention. Motion itself can be beautiful, such as a dolphin jumping out of the water, the flight of a bird, or the orbit of a satellite. The study of motion is called kinematics, but kinematics describes only the way objects move - their velocity and their acceleration. Dynamics considers the forces that affect the motion of moving objects and systems. Newton's laws of motion are the foundation of dynamics. These laws describe the way objects speed up, slow down, stay in motion, and interact with other objects. They are also universal laws: they apply everywhere on Earth as well as in space.

A force pushes or pulls an object. The object being moved by a force could be an inanimate object, a table, or an animate object, a person. The pushing or pulling may be done by a person, or even the gravitational pull of Earth. Forces have different magnitudes and directions; this means that some forces are stronger than others and can act in different directions. For example, a cannon exerts a strong force on the cannonball that is launched into the air. In contrast, a mosquito landing on your arm exerts only a small force on your arm.

When multiple forces act on an object, the forces combine. Adding together all of the forces acting on an object gives the total force, or net force. An external force is a force that acts on an object within the system from outside the system. This type of force is different than an internal force, which acts between two objects that are both within the system. The net external force combines these two definitions; it is the total combined external force. We discuss further details about net force, external force, and net external force in the coming sections.

In mathematical terms, two forces acting in opposite directions have opposite signs (positive or negative). By convention, the negative sign is assigned to any movement to the left or downward. If two forces pushing in opposite directions are added together, the larger force will be somewhat canceled out by the smaller force pushing in the opposite direction. It is important to be consistent with your chosen coordinate system within a problem; for example, if negative values are assigned to the downward direction for velocity, then distance, force, and acceleration should also be designated as being negative in the downward direction.

\section*{Free-Body Diagrams and Examples of Forces}
\section*{Teacher Support}
Teacher Support [BL] Review vectors and how they are represented. Review vector addition.\\[0pt]
[AL] Ask students to give everyday examples of situations where multiple forces act together. Draw free-body diagrams for some of these situations.

For our first example of force, consider an object hanging from a rope. This example gives us the opportunity to introduce a useful tool known as a free-body diagram. A free-body diagram represents the object being acted upon-that is, the free body - as a single point. Only the forces acting on the body (that is, external forces) are shown and are represented by vectors (which are drawn as arrows). These forces are the only ones shown because only external forces acting on the body affect its motion. We can ignore any internal forces within the body because they cancel each other out, as explained in the section on Newton's third law of motion. Free-body diagrams are very useful for analyzing forces acting on an object.\\
\includegraphics[max width=\textwidth, center]{245726cb-221c-4c44-8efe-f3bac2e19c79-06}

Figure 4.2 An object of mass, \(m\), is held up by the force of tension.\\
Figure 4.2 shows the force of tension in the rope acting in the upward direction, opposite the force of gravity. The forces are indicated in the free-body diagram by an arrow pointing up, representing tension, and another arrow pointing down, representing gravity. In a free-body diagram, the lengths of the arrows show the relative magnitude (or strength) of the forces. Because forces are vectors, they add just like other vectors. Notice that the two arrows have equal lengths in Figure 4.2, which means that the forces of tension and weight are of equal magnitude. Because these forces of equal magnitude act in opposite directions, they are perfectly balanced, so they add together to give a net force of zero.

Not all forces are as noticeable as when you push or pull on an object. Some forces act without physical contact, such as the pull of a magnet (in the case of magnetic force) or the gravitational pull of Earth (in the case of gravitational force).

In the next three sections discussing Newton's laws of motion, we will learn about three specific types of forces: friction, the normal force, and the gravitational force. To analyze situations involving forces, we will create free-body diagrams to organize the framework of the mathematics for each individual situation.

\section*{Tips For Success}
Correctly drawing and labeling a free-body diagram is an important first step for solving a problem. It will help you visualize the problem and correctly apply the mathematics to solve the problem.

\section*{Check Your Understanding}
\section*{Teacher Support}
Teacher Support Use the questions in Check Your Understanding to assess whether students have mastered the learning objectives of this section. If students are struggling with a specific objective, the Check Your Understanding assessment will help identify which objective is causing the problem and direct students to the relevant content.\\
1.

What is kinematics?\\
a. Kinematics is the study of motion.\\
b. Kinematics is the study of the cause of motion.\\
c. Kinematics is the study of dimensions.\\
d. Kinematics is the study of atomic structures.\\
2.

Do two bodies have to be in physical contact to exert a force upon one another?\\
a. No, the gravitational force is a field force and does not require physical contact to exert a force.\\
b. No, the gravitational force is a contact force and does not require physical contact to exert a force.\\
c. Yes, the gravitational force is a field force and requires physical contact to exert a force.\\
d. Yes, the gravitational force is a contact force and requires physical contact to exert a force.\\
3.

What kind of physical quantity is force?\\
a. Force is a scalar quantity.\\
b. Force is a vector quantity.\\
c. Force is both a vector quantity and a scalar quantity.\\
d. Force is neither a vector nor a scalar quantity.\\
4.

Which forces can be represented in a free-body diagram?\\
a. Internal forces\\
b. External forces\\
c. Both internal and external forces\\
d. A body that is not influenced by any force

\section*{4. Ne on' Fir La of Mo ion: Iner ia}
\section*{Section Learning Objectives}
By the end of this section, you will be able to do the following:

\begin{itemize}
  \item Describe Newton's first law and friction, and
  \item Discuss the relationship between mass and inertia.
\end{itemize}

\section*{Teacher Support}
Teacher Support The learning objectives in this section will help students master the following standards:

\begin{itemize}
  \item (4) Science concepts. The student knows and applies the laws governing motion in a variety of situations. The student is expected to:
  \item (D) calculate the effect of forces on objects, including the law of inertia, the relationship between force and acceleration, and the nature of force pairs between objects.
\end{itemize}

Before students begin this section, it is useful to review the concepts of force, external force, net external force, and addition of forces.\\[0pt]
[BL][OL][AL] Ask students to speculate what happens to objects when they are set in motion. Do they remain in motion or stop after some time? Why?

\section*{Misconception Alert}
Students may believe that objects that are in motion tend to slow down and stop. Explain the concept of friction. Talk about objects in outer space, where there is no atmosphere and no gravity. Ask students to describe the motion of such objects.

\section*{Section Key Terms}
\section*{Newton s First Law and Friction}
\section*{Teacher Support}
Teacher Support [BL][OL][AL] Discuss examples of Newton's first law seen in everyday life.\\[0pt]
[BL][OL][AL] Talk about different pairs of surfaces and how each exhibits different levels of friction. Ask students to give examples of smooth and rough\\
surfaces. Ask them where friction may be useful and where it may be undesirable.\\[0pt]
[OL][AL] Ask students to give different examples of systems where multiple forces occur. Draw free-body diagrams for these. Include the force of friction. Emphasize the direction of the force of friction.

Newton's first law of motion states the following:

\begin{enumerate}
  \item A body at rest tends to remain at rest.
  \item A body in motion tends to remain in motion at a constant velocity unless acted on by a net external force. (Recall that constant velocity means that the body moves in a straight line and at a constant speed.)
\end{enumerate}

At first glance, this law may seem to contradict your everyday experience. You have probably noticed that a moving object will usually slow down and stop unless some effort is made to keep it moving. The key to understanding why, for example, a sliding box slows down (seemingly on its own) is to first understand that a net external force acts on the box to make the box slow down. Without this net external force, the box would continue to slide at a constant velocity (as stated in Newton's first law of motion). What force acts on the box to slow it down? This force is called friction. Friction is an external force that acts opposite to the direction of motion (see Figure 4.3). Think of friction as a resistance to motion that slows things down.

Consider an air hockey table. When the air is turned off, the puck slides only a short distance before friction slows it to a stop. However, when the air is turned on, it lifts the puck slightly, so the puck experiences very little friction as it moves over the surface. With friction almost eliminated, the puck glides along with very little change in speed. On a frictionless surface, the puck would experience no net external force (ignoring air resistance, which is also a form of friction). Additionally, if we know enough about friction, we can accurately predict how quickly objects will slow down.

Now let's think about another example. A man pushes a box across a floor at constant velocity by applying a force of\\
+50 N . (The positive sign indicates that, by convention, the direction of motion is to the right.) What is the force of friction that opposes the motion? The force of friction must be -50 N . Why? According to Newton's first law of motion, any object moving at constant velocity has no net external force acting upon it, which means that the sum of the forces acting on the object must be zero. The mathematical way to say that no net external force acts on an object is \(\mathbf{F}_{\text {net }}=0\) or \(\mathbf{F}=0\). So if the man applies +50 N of force, then the force of friction must be -50 N for the two forces to add up to zero (that is, for the two forces to cancel each other). Whenever you encounter the phrase at constant velocity, Newton's first law tells you that the net external force is zero.

\begin{figure}[h]
\begin{center}
  \includegraphics[max width=\textwidth]{245726cb-221c-4c44-8efe-f3bac2e19c79-11}
\captionsetup{labelformat=empty}
\caption{Figure 4.3 For a box sliding across a floor, friction acts in the direction opposite to the velocity.}
\end{center}
\end{figure}

The force of friction depends on two factors: the coefficient of friction and the normal force. For any two surfaces that are in contact with one another, the coefficient of friction is a constant that depends on the nature of the surfaces. The normal force is the force exerted by a surface that pushes on an object in response to gravity pulling the object down on a horizontal surface. In equation form, the force of friction is\\
\(\mathbf{f}=\mu \mathbf{N}\),

\section*{4.1}
where is the coefficient of friction and \(\mathbf{N}\) is the normal force. (The coefficient of friction is discussed in more detail in another chapter, and the normal force is discussed in more detail in the section Newton's Third Law of Motion.)

Recall from the section on Force that a net external force acts from outside on the object of interest. A more precise definition is that it acts on the system of interest. A system is one or more objects that you choose to study. It is important to define the system at the beginning of a problem to figure out\\
which forces are external and need to be considered, and which are internal and can be ignored.

For example, in Figure 4.4 (a), two children push a third child in a wagon at a constant velocity. The system of interest is the wagon plus the small child, as shown in part (b) of the figure. The two children behind the wagon exert external forces on this system (F1, F2). Friction \(f\) acting at the axles of the wheels and at the surface where the wheels touch the ground two other external forces acting on the system. Two more external forces act on the system: the weight \(\mathbf{W}\) of the system pulling down and the normal force \(\mathbf{N}\) of the ground pushing up. Notice that the wagon is not accelerating vertically, so Newton's first law tells us that the normal force balances the weight. Because the wagon is moving forward at a constant velocity, the force of friction must have the same strength as the sum of the forces applied by the two children.

\begin{figure}[h]
\begin{center}
  \includegraphics[max width=\textwidth]{245726cb-221c-4c44-8efe-f3bac2e19c79-13}
\captionsetup{labelformat=empty}
\caption{Figure 4.4 (a) The wagon and rider form a system that is acted on by external forces. (b) The two children pushing the wagon and child provide two external}
\end{center}
\end{figure}

forces. Friction acting at the wheel axles and on the surface of the tires where they touch the ground provide an external force that act against the direction of motion. The weight \(\mathbf{W}\) and the normal force \(\mathbf{N}\) from the ground are two more external forces acting on the system. All external forces are represented in the figure by arrows. All of the external forces acting on the system add together, but because the wagon moves at a constant velocity, all of the forces must add up to zero.

\section*{Mass and Inertia}
\section*{Teacher Support}
Teacher Support [BL] Review Newton's first law. Explain that the property of objects to maintain their state of motion is called inertia.\\[0pt]
[OL][AL] Take two similar carts or trolleys with wheels. Place a heavy weight in one of them. Ask students which cart would require more force to change its state of motion. Ask students which would stay in motion longer if you were to set them in motion. Based on this discussion, have students speculate on what inertia may depend on.\\[0pt]
[BL][OL] Explain the concepts of mass and weight. Explain that these terms may be used interchangeably in everyday life but have different meanings in science.

Inertia is the tendency for an object at rest to remain at rest, or for a moving object to remain in motion in a straight line with constant speed. This key property of objects was first described by Galileo. Later, Newton incorporated the concept of inertia into his first law, which is often referred to as the law of inertia.

As we know from experience, some objects have more inertia than others. For example, changing the motion of a large truck is more difficult than changing the motion of a toy truck. In fact, the inertia of an object is proportional to the mass of the object. Mass is a measure of the amount of matter (or stuff) in an object. The quantity or amount of matter in an object is determined by the number and types of atoms the object contains. Unlike weight (which changes if the gravitational force changes), mass does not depend on gravity. The mass of an object is the same on Earth, in orbit, or on the surface of the moon. In practice, it is very difficult to count and identify all of the atoms and molecules in an object, so mass is usually not determined this way. Instead, the mass of an object is determined by comparing it with the standard kilogram. Mass is therefore expressed in kilograms.

\section*{Tips For Success}
In everyday language, people often use the terms weight and mass interchangeably-but this is not correct. Weight is actually a force. (We cover this topic in more detail in the section Newton's Second Law of Motion.)

\section*{Watch Physics}
Newton s First Law of Motion This video contrasts the way we thought about motion and force in the time before Galileo's concept of inertia and Newton's first law of motion with the way we understand force and motion now.

Click to view content

\section*{Grasp Check}
Before we understood that objects have a tendency to maintain their velocity in a straight line unless acted upon by a net force, people thought that objects had a tendency to stop on their own. This happened because a specific force was not yet understood. What was that force?\\
a. Gravitational force\\
b. Electrostatic force\\
c. Nuclear force\\
d. Frictional force

\section*{Virtual Physics}
Forces and Motion-Basics In this simulation, you will first explore net force by placing blue people on the left side of a tug-of-war rope and red people on the right side of the rope (by clicking people and dragging them with your mouse). Experiment with changing the number and size of people on each side to see how it affects the outcome of the match and the net force. Hit the "Go!" button to start the match, and the "reset all" button to start over.

Next, click on the Friction tab. Try selecting different objects for the person to push. Slide the applied force button to the right to apply force to the right, and to the left to apply force to the left. The force will continue to be applied as long as you hold down the button. See the arrow representing friction change in magnitude and direction, depending on how much force you apply. Try increasing or decreasing the friction force to see how this change affects the motion.

Click to view content

\section*{Grasp Check}
Click on the tab for the Acceleration Lab and check the Sum of Forces option. Push the box to the right and then release. Notice which direction the sum of forces arrow points after the person stops pushing the box and lets it continue\\
moving to the right on its own. At this point, in which direction is the net force, the sum of forces, pointing? Why?\\
a. The net force acts to the right because the applied external force acted to the right.\\
b. The net force acts to the left because the applied external force acted to the left.\\
c. The net force acts to the right because the frictional force acts to the right.\\
d. The net force acts to the left because the frictional force acts to the left.

\section*{Teacher Support}
Teacher Support Use the questions in Check Your Understanding to assess whether students have mastered the learning objectives of this section. If students are struggling with a specific objective, the Check Your Understanding assessment will help identify which objective is causing the problem and direct students to the relevant content.

\section*{Check Your Understanding}
5.

What does Newton's first law state?\\
a. A body at rest tends to remain at rest and a body in motion tends to remain in motion at a constant acceleration unless acted on by a net external force.\\
b. A body at rest tends to remain at rest and a body in motion tends to remain in motion at a constant velocity unless acted on by a net external force.\\
c. The rate of change of momentum of a body is directly proportional to the external force applied to the body.\\
d. The rate of change of momentum of a body is inversely proportional to the external force applied to the body.\\
6.

According to Newton's first law, a body in motion tends to remain in motion at a constant velocity. However, when you slide an object across a surface, the object eventually slows down and stops. Why?\\
a. The object experiences a frictional force exerted by the surface, which opposes its motion.\\
b. The object experiences the gravitational force exerted by Earth, which opposes its motion\\
c. The object experiences an internal force exerted by the body itself, which opposes its motion.\\
d. The object experiences a pseudo-force from the body in motion, which opposes its motion.\\
7.

What is inertia?\\
a. Inertia is an object's tendency to maintain its mass.\\
b. Inertia is an object's tendency to remain at rest.\\
c. Inertia is an object's tendency to remain in motion\\
d. Inertia is an object's tendency to remain at rest or, if moving, to remain in motion.\\
8.

What is mass? What does it depend on?\\
a. Mass is the weight of an object, and it depends on the gravitational force acting on the object.\\
b. Mass is the weight of an object, and it depends on the number and types of atoms in the object.\\
c. Mass is the quantity of matter contained in an object, and it depends on the gravitational force acting on the object.\\
d. Mass is the quantity of matter contained in an object, and it depends on the number and types of atoms in the object.

\section*{4. Ne on' econd La of Mo ion}
\section*{Section Learning Objectives}
By the end of this section, you will be able to do the following:

\begin{itemize}
  \item Describe Newton's second law, both verbally and mathematically
  \item Use Newton's second law to solve problems
\end{itemize}

\section*{Teacher Support}
Teacher Support The learning objectives in this section will help students master the following standards:

\begin{itemize}
  \item (4) Science concepts. The student knows and applies the laws governing motion in a variety of situations. The student is expected to:
  \item (D) calculate the effect of forces on objects, including the law of inertia, the relationship between force and acceleration, and the nature of force pairs between objects.
\end{itemize}

Before beginning this section, review forces, acceleration, acceleration due to gravity (g), friction, inertia, and Newton's first law.

\section*{Section Key Terms}
\section*{Describing Newton s Second Law of Motion}
\section*{Teacher Support}
Teacher Support [BL][OL] Review the concepts of inertia and Newton's first law. Explain that, according to Newton's first law, a change in motion is caused by an external force. For instance, a ball that is pitched changes its speed and direction when it is hit by a bat.\\[0pt]
[BL][OL][AL] Write the equation for Newton's second law and show how it can be solved for all three variables, \(F, m\), and \(a\). Explain the practical implications for each case. Ask students how the other two variables would behave if one quantity is held constant.

\section*{Misconception Alert}
Students might confuse the terms equal and proportional.\\
Newton's first law considered bodies at rest or bodies in motion at a constant velocity. The other state of motion to consider is when an object is moving with a changing velocity, which means a change in the speed and/or the direction of\\
motion. This type of motion is addressed by Newton's second law of motion, which states how force causes changes in motion. Newton's second law of motion is used to calculate what happens in situations involving forces and motion, and it shows the mathematical relationship between force, mass, and acceleration. Mathematically, the second law is most often written as\\
\(\mathrm{F}_{\text {net }}=m \mathrm{aor} \quad \mathrm{F}=m \mathrm{a}\),\\
4.2\\
where \(\mathbf{F}_{\text {net }}\) (or \(\mathbf{F}\) ) is the net external force, \(m\) is the mass of the system, and \(\mathbf{a}\) is the acceleration. Note that \(\mathbf{F}_{\text {net }}\) and \(\mathbf{F}\) are the same because the net external force is the sum of all the external forces acting on the system.

First, what do we mean by a change in motion? A change in motion is simply a change in velocity: the speed of an object can become slower or faster, the direction in which the object is moving can change, or both of these variables may change. A change in velocity means, by definition, that an acceleration has occurred. Newton's first law says that only a nonzero net external force can cause a change in motion, so a net external force must cause an acceleration. Note that acceleration can refer to slowing down or to speeding up. Acceleration can also refer to a change in the direction of motion with no change in speed, because acceleration is the change in velocity divided by the time it takes for that change to occur, and velocity is defined by speed and direction.

From the equation \(\mathrm{F}_{\text {net }}=m \mathrm{a}\), we see that force is directly proportional to both mass and acceleration, which makes sense. To accelerate two objects from rest to the same velocity, you would expect more force to be required to accelerate the more massive object. Likewise, for two objects of the same mass, applying a greater force to one would accelerate it to a greater velocity.

Now, let's rearrange Newton's second law to solve for acceleration. We get \(\mathrm{a}=\frac{\mathrm{F}_{\text {net }}}{m}\) ora \(=\frac{\mathrm{F}}{m}\).\\
4.3

In this form, we can see that acceleration is directly proportional to force, which we write as\\
\(\mathrm{a} \propto \mathrm{F}_{\text {net }}\),\\
4.4\\
where the symbol \(\propto\) means proportional to.\\
This proportionality mathematically states what we just said in words: acceleration is directly proportional to the net external force. When two variables are directly proportional to each other, then if one variable doubles, the other variable must double. Likewise, if one variable is reduced by half, the other variable must also be reduced by half. In general, when one variable is multiplied by a number, the other variable is also multiplied by the same number. It seems\\
reasonable that the acceleration of a system should be directly proportional to and in the same direction as the net external force acting on the system. An object experiences greater acceleration when acted on by a greater force.

It is also clear from the equation \(\mathbf{a}=\mathbf{F}_{\text {net }} / m\) that acceleration is inversely proportional to mass, which we write as\\
a \(\propto \frac{1}{m}\).\\
4.5

Inversely proportional means that if one variable is multiplied by a number, the other variable must be divided by the same number. Now, it also seems reasonable that acceleration should be inversely proportional to the mass of the system. In other words, the larger the mass (the inertia), the smaller the acceleration produced by a given force. This relationship is illustrated in Figure 4.5, which shows that a given net external force applied to a basketball produces a much greater acceleration than when applied to a car.

\begin{figure}[h]
\begin{center}
  \includegraphics[max width=\textwidth]{245726cb-221c-4c44-8efe-f3bac2e19c79-20}
\captionsetup{labelformat=empty}
\caption{Figure 4.5 The same force exerted on systems of different masses produces different accelerations. (a) A boy pushes a basketball to make a pass. The effect of gravity on the ball is ignored. (b) The same boy pushing with identical force on a stalled car produces a far smaller acceleration (friction is negligible). Note that the free-body diagrams for the ball and for the car are identical, which allows us to compare the two situations.}
\end{center}
\end{figure}

\section*{Teacher Support}
Teacher Support [BL] Review how to convert between units.\\[0pt]
[OL][AL] Ask students to give examples of Newton's second law.

\section*{Misconception Alert}
Students might confuse weight, which is a force, and \(\mathbf{g}\), which is the acceleration due to gravity.\\[0pt]
[BL][OL][AL] Ask students if they think an astronaut weighs the same on the moon as they do on Earth. Talk about the difference between mass and weight.

Before putting Newton's second law into action, it is important to consider units. The equation \(\mathbf{F}_{\text {net }}=m \mathbf{a}\) is used to define the units of force in terms of the three basic units of mass, length, and time (recall that acceleration has units of length divided by time squared). The SI unit of force is called the newton (abbreviated \(\mathrm{N})\) and is the force needed to accelerate a \(1-\mathrm{kg}\) system at the rate of \(1 \mathrm{~m} / \mathrm{s}\). That is, because \(\mathbf{F}_{\text {net }}=m \mathbf{a}\), we have\\
\(1 \mathrm{~N}=1 \mathrm{~kg} \times 1 \mathrm{~m} / \mathrm{s}^{2}=1 \frac{\mathrm{~kg} \cdot \mathrm{~m}}{\mathrm{~s}^{2}}\).\\
4.6

One of the most important applications of Newton's second law is to calculate weight (also known as the gravitational force), which is usually represented mathematically as \(\mathbf{W}\). When people talk about gravity, they don't always realize that it is an acceleration. When an object is dropped, it accelerates toward the center of Earth. Newton's second law states that the net external force acting on an object is responsible for the acceleration of the object. If air resistance is negligible, the net external force on a falling object is only the gravitational force (i.e., the weight of the object).

Weight can be represented by a vector because it has a direction. Down is defined as the direction in which gravity pulls, so weight is normally considered a downward force. By using Newton's second law, we can figure out the equation for weight.

Consider an object with mass \(m\) falling toward Earth. It experiences only the force of gravity (i.e., the gravitational force or weight), which is represented by W. Newton's second law states that \(\mathbf{F}_{\text {net }}=m \mathbf{a}\). Because the only force acting on the object is the gravitational force, we have \(\mathbf{F}_{\text {net }}=\mathbf{W}\). We know that the acceleration of an object due to gravity is \(\mathbf{g}\), so we have \(\mathbf{a}=\mathbf{g}\). Substituting these two expressions into Newton's second law gives\\
\(\mathrm{W}=m g\).\\
4.7

This is the equation for weight-the gravitational force on a mass \(m\). On Earth, \(\mathbf{g}=9.80 \mathrm{~m} / \mathrm{s}\), so the weight (disregarding for now the direction of the weight) of a \(1.0-\mathrm{kg}\) object on Earth is\\
\(\mathbf{W}=m \mathbf{g}=(1.0 \mathrm{~kg})(9.80 \mathrm{~m} / \mathrm{s})=9.8 \mathrm{~N}\).\\
4.8

Although most of the world uses newtons as the unit of force, in the United States the most familiar unit of force is the pound (lb), where \(1 \mathrm{~N}=0.225 \mathrm{lb}\).

Recall that although gravity acts downward, it can be assigned a positive or negative value, depending on what the positive direction is in your chosen coordinate system. Be sure to take this into consideration when solving problems with weight. When the downward direction is taken to be negative, as is often the case, acceleration due to gravity becomes\\
\(\mathbf{g}=-9.8 \mathrm{~m} / \mathrm{s}\).\\
When the net external force on an object is its weight, we say that it is in freefall. In this case, the only force acting on the object is the force of gravity. On the surface of Earth, when objects fall downward toward Earth, they are never truly in freefall because there is always some upward force due to air resistance that acts on the object (and there is also the buoyancy force of air, which is similar to the buoyancy force in water that keeps boats afloat).

Gravity varies slightly over the surface of Earth, so the weight of an object depends very slightly on its location on Earth. Weight varies dramatically away from Earth's surface. On the moon, for example, the acceleration due to gravity is only \(1.67 \mathrm{~m} / \mathrm{s}\). Because weight depends on the force of gravity, a \(1.0-\mathrm{kg}\) mass weighs 9.8 N on Earth and only about 1.7 N on the moon.

It is important to remember that weight and mass are very different, although they are closely related. Mass is the quantity of matter (how much stu ) in an object and does not vary, but weight is the gravitational force on an object and is proportional to the force of gravity. It is easy to confuse the two, because our experience is confined to Earth, and the weight of an object is essentially the same no matter where you are on Earth. Adding to the confusion, the terms mass and weight are often used interchangeably in everyday language; for example, our medical records often show our weight in kilograms, but never in the correct unit of newtons.

\section*{Snap Lab}
\section*{Mass and Weight}
\section*{Teacher Support}
Teacher Support Explain that even though a scale gives a mass, it actually measures weight. Scales are calibrated to show the correct mass on Earth. They\\
would give different results on the moon, because the force of gravity is weaker on the moon.

In this activity, you will use a scale to investigate mass and weight.

\begin{itemize}
  \item 1 bathroom scale
  \item 1 table
\end{itemize}

\begin{enumerate}
  \item What do bathroom scales measure?
  \item When you stand on a bathroom scale, what happens to the scale? It depresses slightly. The scale contains springs that compress in proportion to your weight-similar to rubber bands expanding when pulled.
  \item The springs provide a measure of your weight (provided you are not accelerating). This is a force in newtons (or pounds). In most countries, the measurement is now divided by 9.80 to give a reading in kilograms, which is a of mass. The scale detects weight but is calibrated to display mass.
  \item If you went to the moon and stood on your scale, would it detect the same mass as it did on Earth?
\end{enumerate}

\section*{Grasp Check}
While standing on a bathroom scale, push down on a table next to you. What happens to the reading? Why?\\
a. The reading increases because part of your weight is applied to the table and the table exerts a matching force on you that acts in the direction of your weight.\\
b. The reading increases because part of your weight is applied to the table and the table exerts a matching force on you that acts in the direction opposite to your weight.\\
c. The reading decreases because part of your weight is applied to the table and the table exerts a matching force on you that acts in the direction of your weight.\\
d. The reading decreases because part of your weight is applied to the table and the table exerts a matching force on you that acts in the direction opposite to your weight.

\section*{Tips For Success}
Only net external force impacts the acceleration of an object. If more than one force acts on an object and you calculate the acceleration by using only one of these forces, you will not get the correct acceleration for that object.

\section*{Watch Physics}
Newton s Second Law of Motion This video reviews Newton's second law of motion and how net external force and acceleration relate to one another and to mass. It also covers units of force, mass, and acceleration, and reviews a worked-out example.

Click to view content

\section*{Grasp Check}
True or False-If you want to reduce the acceleration of an object to half its original value, then you would need to reduce the net external force by half.\\
a. True\\
b. False

\section*{Worked Example}
What Acceleration Can a Person Produce when Pushing a Lawn Mower? Suppose that the net external force (push minus friction) exerted on a lawn mower is 51 N parallel to the ground. The mass of the mower is 240 kg. What is its acceleration?

\begin{figure}[h]
\begin{center}
  \includegraphics[max width=\textwidth]{245726cb-221c-4c44-8efe-f3bac2e19c79-24}
\captionsetup{labelformat=empty}
\caption{Figure 4.6}
\end{center}
\end{figure}

\section*{Strategy}
Because \(\mathbf{F}_{\text {net }}\) and \(m\) are given, the acceleration can be calculated directly from Newton's second law: \(\mathbf{F}_{\text {net }}=m \mathbf{a}\).

Solution\\
Solving Newton's second law for the acceleration, we find that the magnitude of the acceleration, \(\mathbf{a}\), is \(\mathbf{a}=\frac{\mathbf{F}_{\text {net }}}{m}\). Entering the given values for net external force\\
and mass gives\\
\(\mathbf{a}=\frac{51 \mathrm{~N}}{240 \mathrm{~kg}}\)\\
4.9

Inserting the units \(\mathrm{kg} \cdot \mathrm{m} / \mathrm{s}\) for N yields\\
\(\mathbf{a}=\frac{51 \mathrm{~kg} \cdot \mathrm{~m} / \mathrm{s}^{2}}{240 \mathrm{~kg}}=0.21 \mathrm{~m} / \mathrm{s}\).\\
4.10

Discussion\\
The acceleration is in the same direction as the net external force, which is parallel to the ground and to the right. There is no information given in this example about the individual external forces acting on the system, but we can say something about their relative magnitudes. For example, the force exerted by the person pushing the mower must be greater than the friction opposing the motion, because we are given that the net external force is in the direction in which the person pushes. Also, the vertical forces must cancel if there is no acceleration in the vertical direction (the mower is moving only horizontally). The acceleration found is reasonable for a person pushing a mower; the mower's speed must increase by \(0.21 \mathrm{~m} / \mathrm{s}\) every second, which is possible. The time during which the mower accelerates would not be very long because the person's top speed would soon be reached. At this point, the person could push a little less hard, because he only has to overcome friction.

\section*{Worked Example}
What Rocket Thrust Accelerates This Sled? Prior to manned space flights, rocket sleds were used to test aircraft, missile equipment, and physiological effects on humans at high accelerations. Rocket sleds consisted of a platform mounted on one or two rails and propelled by several rockets. Calculate the magnitude of force exerted by each rocket, called its thrust, \(\mathbf{T}\), for the four-rocket propulsion system shown below. The sled's initial acceleration is \(49 \mathrm{~m} / \mathrm{s}\), the mass of the system is \(2,100 \mathrm{~kg}\), and the force of friction opposing the motion is 650 N .

\begin{figure}[h]
\begin{center}
  \includegraphics[max width=\textwidth]{245726cb-221c-4c44-8efe-f3bac2e19c79-26}
\captionsetup{labelformat=empty}
\caption{Figure 4.7}
\end{center}
\end{figure}

\section*{Strategy}
The system of interest is the rocket sled. Although forces act vertically on the system, they must cancel because the system does not accelerate vertically. This leaves us with only horizontal forces to consider. We'll assign the direction to the right as the positive direction. See the free-body diagram in Figure 4.8.

\section*{Solution}
We start with Newton's second law and look for ways to find the thrust \(\mathbf{T}\) of the engines. Because all forces and acceleration are along a line, we need only consider the magnitudes of these quantities in the calculations. We begin with\\
\(\mathbf{F}_{\text {net }}=m \mathbf{a}\),\\
4.11\\
where \(\mathbf{F}_{\text {net }}\) is the net external force in the horizontal direction. We can see from Figure 4.8 that the engine thrusts are in the same direction (which we call the positive direction), whereas friction opposes the thrust. In equation form, the net external force is\\
\(\mathbf{F}_{\text {net }}=4 \mathbf{T}-\mathbf{f}\).\\
4.12

Newton's second law tells us that \(\mathbf{F}_{\text {net }}=m \mathbf{a}\), so we get\\
\(m \mathbf{a}=4 \mathbf{T}-\mathbf{f}\).\\
4.13

After a little algebra, we solve for the total thrust 4T:\\
\(4 \mathbf{T}=m \mathbf{a}+\mathbf{f}\),\\
4.14\\
which means that the individual thrust is\\
\(\mathbf{T}=\frac{m \mathbf{a}+\mathbf{f}}{4}\).\\
4.15

Inserting the known values yields\\
\(\mathbf{T}=\frac{(2100 \mathrm{~kg})\left(49 \mathrm{~m} / \mathrm{s}^{2}\right)+650 \mathrm{~N}}{4}=2.6 \times 10^{4} \mathrm{~N}\).\\
4.16

Discussion\\
The numbers are quite large, so the result might surprise you. Experiments such as this were performed in the early 1960s to test the limits of human endurance and to test the apparatus designed to protect fighter pilots in emergency ejections. Speeds of \(1000 \mathrm{~km} / \mathrm{h}\) were obtained, with accelerations of 45 g . (Recall that g , the acceleration due to gravity, is \(9.80 \mathrm{~m} / \mathrm{s}^{2}\). An acceleration of \(45 \mathbf{g}\) is \(45 \times 9.80 \mathrm{~m} / \mathrm{s}^{2}\), which is approximately \(440 \mathrm{~m} / \mathrm{s}^{2}\).) Living subjects are no longer used, and land speeds of \(10,000 \mathrm{~km} / \mathrm{h}\) have now been obtained with rocket sleds. In this example, as in the preceding example, the system of interest is clear. We will see in later examples that choosing the system of interest is crucial - and that the choice is not always obvious.

\section*{Practice Problems}
9.

If 1 N is equal to 0.225 lb , how many pounds is 5 N of force?\\
a. 0.045 lb\\
b. 1.125 lb\\
c. 2.025 lb\\
d. 5.000 lb\\
10.

How much force needs to be applied to a 5 -kg object for it to accelerate at 20 \(\mathrm{m} / \mathrm{s}\) ?\\
a. 1 N\\
b. 10 N\\
c. 100 N\\
d. \(1,000 \mathrm{~N}\)

\section*{Check Your Understanding}
\section*{Teacher Support}
Teacher Support Use the questions in Check Your Understanding to assess whether students have achieved the section learning objectives. If students are struggling with a specific objective, the Check Your Understanding assessment will help identify which is causing the problem and direct students to the relevant content.\\
11.

What is the mathematical statement for Newton's second law of motion?\\
a. \(\mathrm{F}=m \mathrm{a}\)\\
b. \(\mathrm{F}=2 m \mathrm{a}\)\\
c. \(\mathrm{F}=\frac{m}{\mathrm{a}}\)\\
d. \(\mathrm{F}=m \mathrm{a}\)\\
12.

Newton's second law describes the relationship between which quantities?\\
a. Force, mass, and time\\
b. Force, mass, and displacement\\
c. Force, mass, and velocity\\
d. Force, mass, and acceleration\\
13.

What is acceleration?\\
a. Acceleration is the rate at which displacement changes.\\
b. Acceleration is the rate at which force changes.\\
c. Acceleration is the rate at which velocity changes.\\
d. Acceleration is the rate at which mass changes.

\subsection*{4.4 Ne on' hird La of Mo ion}
\section*{Section Learning Objectives}
By the end of this section, you will be able to do the following:

\begin{itemize}
  \item Describe Newton's third law, both verbally and mathematically
  \item Use Newton's third law to solve problems
\end{itemize}

\section*{Teacher Support}
Teacher Support The learning objectives in this section will help your students master the following standards:

\begin{itemize}
  \item (4) Science concepts. The student knows and applies the laws governing motion in a variety of situations. The student is expected to:
  \item (D) calculate the effect of forces on objects, including the law of inertia, the relationship between force and acceleration, and the nature of force pairs between objects.
\end{itemize}

\section*{Section Key Terms}
\section*{Describing Newton s Third Law of Motion}
\section*{Teacher Support}
Teacher Support [BL][OL] Review Newton's first and second laws.\\[0pt]
[AL] Start a discussion about action and reaction by giving examples. Introduce the concepts of systems and systems of interest. Explain how forces can be classified as internal or external to the system of interest. Give examples of systems. Ask students which forces are internal and which are external in each scenario.

If you have ever stubbed your toe, you have noticed that although your toe initiates the impact, the surface that you stub it on exerts a force back on your toe. Although the first thought that crosses your mind is probably "ouch, that hurt" rather than "this is a great example of Newton's third law," both statements are true.

This is exactly what happens whenever one object exerts a force on anothereach object experiences a force that is the same strength as the force acting on the other object but that acts in the opposite direction. Everyday experiences, such as stubbing a toe or throwing a ball, are all perfect examples of Newton's third law in action.

Newton's third law of motion states that whenever a first object exerts a force on a second object, the first object experiences a force equal in magnitude but opposite in direction to the force that it exerts.

Newton's third law of motion tells us that forces always occur in pairs, and one object cannot exert a force on another without experiencing the same strength force in return. We sometimes refer to these force pairs as action-reaction pairs, where the force exerted is the action, and the force experienced in return is the reaction (although which is which depends on your point of view).

Newton's third law is useful for figuring out which forces are external to a system. Recall that identifying external forces is important when setting up a problem, because the external forces must be added together to find the net force.

We can see Newton's third law at work by looking at how people move about. Consider a swimmer pushing off from the side of a pool, as illustrated in Figure 4.8. She pushes against the pool wall with her feet and accelerates in the direction opposite to her push. The wall has thus exerted on the swimmer a force of equal magnitude but in the direction opposite that of her push. You might think that two forces of equal magnitude but that act in opposite directions would cancel, but they do not because they act on different systems.

In this case, there are two different systems that we could choose to investigate: the swimmer or the wall. If we choose the swimmer to be the system of interest, as in the figure, then F all on feet is an external force on the swimmer and affects her motion. Because acceleration is in the same direction as the net external force, the swimmer moves in the direction of F all on feet Because the swimmer is our system (or object of interest) and not the wall, we do not need to consider the force \(\mathrm{F}_{\text {feet on }}\) all because it originates from the swimmer rather than acting on the swimmer. Therefore, \(\mathrm{F}_{\text {feet on }}\) all does not directly affect the motion of the system and does not cancel F all on feet Note that the swimmer pushes in the direction opposite to the direction in which she wants to move.

\begin{figure}[h]
\begin{center}
  \includegraphics[max width=\textwidth]{245726cb-221c-4c44-8efe-f3bac2e19c79-30}
\captionsetup{labelformat=empty}
\caption{Figure 4.8 When the swimmer exerts a force \(\mathbf{F}_{\text {feet on }}\) all on the wall, she accelerates in the direction opposite to that of her push. This means that the net external force on her is in the direction opposite to \(\mathbf{F}_{\text {feet on }}\) alr This opposition}
\end{center}
\end{figure}

is the result of Newton's third law of motion, which dictates that the wall exerts a force \(\mathbf{F}\) all on feet on the swimmer that is equal in magnitude but that acts in the direction opposite to the force that the swimmer exerts on the wall.

Other examples of Newton's third law are easy to find. As a teacher paces in front of a whiteboard, he exerts a force backward on the floor. The floor exerts a reaction force in the forward direction on the teacher that causes him to accelerate forward. Similarly, a car accelerates because the ground pushes forward on the car's wheels in reaction to the car's wheels pushing backward on the ground. You can see evidence of the wheels pushing backward when tires spin on a gravel road and throw rocks backward.

Another example is the force of a baseball as it makes contact with the bat. Helicopters create lift by pushing air down, creating an upward reaction force. Birds fly by exerting force on air in the direction opposite that in which they wish to fly. For example, the wings of a bird force air downward and backward in order to get lift and move forward. An octopus propels itself forward in the water by ejecting water backward through a funnel in its body, which is similar to how a jet ski is propelled. In these examples, the octopus or jet ski push the water backward, and the water, in turn, pushes the octopus or jet ski forward.

\section*{Applying Newton s Third Law}
\section*{Teacher Support}
Teacher Support [BL] Review the concept of weight as a force.\\[0pt]
[OL] Ask students what happens when an object is dropped from a height. Why does it stop when it hits the ground? Introduce the term normal force.

\section*{Teacher Demonstration}
[BL][OL][AL] Demonstrate the concept of tension by using physical objects. Suspend an object such as an eraser from a peg by using a rubber band. Hang another rubber band beside the first but with no object attached. Ask students what the difference is between the two. What are the forces acting on the first peg? Explain how the rubber band (i.e., the connector) transmits force. Now ask students what the direction of the external forces acting on the connectoris. Also, ask what internal forces are acting on the connector. If you remove the eraser, in which direction will the rubber band move? This is the direction of the force the rubber band applied to the eraser.

Forces are classified and given names based on their source, how they are transmitted, or their effects. In previous sections, we discussed the forces called push, weight, and friction. In this section, applying Newton's third law of motion will allow us to explore three more forces: the normal force, tension, and thrust. However, because we haven't yet covered vectors in depth, we'll only consider one-dimensional situations in this chapter. Another chapter will consider forces acting in two dimensions.

The gravitational force (or weight) acts on objects at all times and everywhere on Earth. We know from Newton's second law that a net force produces an acceleration; so, why is everything not in a constant state of freefall toward the center of Earth? The answer is the normal force. The normal force is the force that a surface applies to an object to support the weight of that object; it acts perpendicular to the surface upon which the object rests. If an object on a flat surface is not accelerating, the net external force is zero, and the normal force has the same magnitude as the weight of the system but acts in the opposite direction. In equation form, we write that\\
\(\mathbf{N}=m \mathbf{g}\).

\subsection*{4.17}
Note that this equation is only true for a horizontal surface.\\
The word tension comes from the Latin word meaning to stretch. Tension is the force along the length of a flexible connector, such as a string, rope, chain, or cable. Regardless of the type of connector attached to the object of interest, one must remember that the connector can only pull (or exert tension) in the direction parallel to its length. Tension is a pull that acts parallel to the connector, and that acts in opposite directions at the two ends of the connector. This is possible because a flexible connector is simply a long series of actionreaction forces, except at the two ends where outside objects provide one member of the action-reaction forces.

Consider a person holding a mass on a rope, as shown in Figure 4.9.\\
\includegraphics[max width=\textwidth, center]{245726cb-221c-4c44-8efe-f3bac2e19c79-32}

Figure 4.9 When a perfectly flexible connector (one requiring no force to bend it) such as a rope transmits a force \(\mathbf{T}\), this force must be parallel to the length of the rope, as shown. The pull that such a flexible connector exerts is a tension. Note that the rope pulls with equal magnitude force but in opposite directions to the hand and to the mass (neglecting the weight of the rope). This is an example of Newton's third law. The rope is the medium that transmits forces of equal magnitude between the two objects but that act in opposite directions.

Tension in the rope must equal the weight of the supported mass, as we can prove by using Newton's second law. If the 5.00 kg mass in the figure is stationary, then its acceleration is zero, so \(\mathbf{F}_{\text {net }}=0\). The only external forces acting on the mass are its weight \(\mathbf{W}\) and the tension \(\mathbf{T}\) supplied by the rope. Summing the external forces to find the net force, we obtain\\
\(\mathbf{F}_{\text {net }}=\mathbf{T}-\mathbf{W}=0\),\\
4.18\\
where \(\mathbf{T}\) and \(\mathbf{W}\) are the magnitudes of the tension and weight, respectively, and their signs indicate direction, with up being positive. By substituting \(m \mathbf{g}\) for \(\mathbf{F}_{\text {net }}\) and rearranging the equation, the tension equals the weight of the supported mass, just as you would expect\\
\(\mathbf{T}=\mathbf{W}=m \mathbf{g}\).\\
4.19

For a \(5.00-\mathrm{kg}\) mass (neglecting the mass of the rope), we see that\\
\(\mathbf{T}=m \mathbf{g}=(5.00 \mathrm{~kg})\left(9.80 \mathrm{~m} / \mathrm{s}^{2}\right)=49.0 \mathrm{~N}\).\\
4.20

Another example of Newton's third law in action is thrust. Rockets move forward by expelling gas backward at a high velocity. This means that the rocket exerts a large force backward on the gas in the rocket combustion chamber, and the gas, in turn, exerts a large force forward on the rocket in response. This reaction force is called thrust.

\section*{Tips For Success}
A common misconception is that rockets propel themselves by pushing on the ground or on the air behind them. They actually work better in a vacuum, where they can expel exhaust gases more easily.

\section*{Links To Physics}
Math: Problem-Solving Strategy for Newton s Laws of Motion The basics of problem solving, presented earlier in this text, are followed here with\\
specific strategies for applying Newton's laws of motion. These techniques also reinforce concepts that are useful in many other areas of physics.

First, identify the physical principles involved. If the problem involves forces, then Newton's laws of motion are involved, and it is important to draw a careful sketch of the situation. An example of a sketch is shown in Figure 4.10. Next, as in Figure 4.10, use vectors to represent all forces. Label the forces carefully, and make sure that their lengths are proportional to the magnitude of the forces and that the arrows point in the direction in which the forces act.

\begin{figure}[h]
\begin{center}
  \includegraphics[max width=\textwidth]{245726cb-221c-4c44-8efe-f3bac2e19c79-34}
\captionsetup{labelformat=empty}
\caption{Figure 4.10 (a) A sketch of Tarzan hanging motionless from a vine. (b) Arrows are used to represent all forces. \(\mathbf{T}\) is the tension exerted on Tarzan by the vine, \(\mathbf{F}_{\mathrm{T}}\) is the force exerted on the vine by Tarzan, and \(\mathbf{W}\) is Tarzan's weight (i.e., the force exerted on Tarzan by Earth's gravity). All other forces, such as a nudge of a breeze, are assumed to be negligible. (c) Suppose we are given Tarzan's mass and asked to find the tension in the vine. We define the system of interest as shown and draw a free-body diagram, as shown in (d). \(\mathbf{F}_{\mathrm{T}}\) is no longer shown because it does not act on the system of interest; rather, \(\mathbf{F}_{\mathrm{T}}\) acts on the outside world. (d) The free-body diagram shows only the external forces}
\end{center}
\end{figure}

acting on Tarzan. For these to sum to zero, we must have \(\mathbf{T}=\mathbf{W}\).\\
Next, make a list of knowns and unknowns and assign variable names to the quantities given in the problem. Figure out which variables need to be calculated; these are the unknowns. Now carefully define the system: which objects are of interest for the problem. This decision is important, because Newton's second law involves only external forces. Once the system is identified, it's possible to see which forces are external and which are internal (see Figure 4.10).

If the system acts on an object outside the system, then you know that the outside object exerts a force of equal magnitude but in the opposite direction on the system.

A diagram showing the system of interest and all the external forces acting on it is called a free-body diagram. Only external forces are shown on free-body diagrams, not acceleration or velocity. Figure 4.10 shows a free-body diagram for the system of interest.

After drawing a free-body diagram, apply Newton's second law to solve the problem. This is done in Figure 4.10 for the case of Tarzan hanging from a vine. When external forces are clearly identified in the free-body diagram, translate the forces into equation form and solve for the unknowns. Note that forces acting in opposite directions have opposite signs. By convention, forces acting downward or to the left are usually negative.

\section*{Grasp Check}
If a problem has more than one system of interest, more than one free-body diagram is required to describe the external forces acting on the different systems.\\
a. True\\
b. False

\section*{Watch Physics}
Newton s Third Law of Motion This video explains Newton's third law of motion through examples involving push, normal force, and thrust (the force that propels a rocket or a jet).

Click to view content

\section*{Grasp Check}
If the astronaut in the video wanted to move upward, in which direction should he throw the object? Why?\\
a. He should throw the object upward because according to Newton's third law, the object will then exert a force on him in the same direction (i.e., upward).\\
b. He should throw the object upward because according to Newton's third law, the object will then exert a force on him in the opposite direction (i.e., downward).\\
c. He should throw the object downward because according to Newton's third law, the object will then exert a force on him in the opposite direction (i.e., upward).\\
d. He should throw the object downward because according to Newton's third law, the object will then exert a force on him in the same direction (i.e., downward).

\section*{Worked Example}
An Accelerating Equipment Cart A physics teacher pushes a cart of demonstration equipment to a classroom, as in Figure 4.11. Her mass is 65.0 kg , the cart's mass is 12.0 kg , and the equipment's mass is 7.0 kg . To push the cart forward, the teacher's foot applies a force of 150 N in the opposite direction (backward) on the floor. Calculate the acceleration produced by the teacher. The force of friction, which opposes the motion, is 24.0 N .

\begin{figure}[h]
\begin{center}
  \includegraphics[max width=\textwidth]{245726cb-221c-4c44-8efe-f3bac2e19c79-36}
\captionsetup{labelformat=empty}
\caption{Figure 4.11}
\end{center}
\end{figure}

\section*{Strategy}
Because they accelerate together, we define the system to be the teacher, the cart, and the equipment. The teacher pushes backward with a force \(\mathbf{F}_{\text {foot }}\) of 150 N. According to Newton's third law, the floor exerts a forward force \(\mathbf{F}\) oor 150 N on the system. Because all motion is horizontal, we can assume that no net\\
force acts in the vertical direction, and the problem becomes one dimensional. As noted in the figure, the friction f opposes the motion and therefore acts opposite the direction of \(\mathbf{F}\)

We should not include the forces \(\mathbf{F}_{\text {teacher }}\), \(\mathbf{F}_{\text {cart }}\), or \(\mathbf{F}_{\text {foot }}\) because these are exerted by the system, not on the system. We find the net external force by adding together the external forces acting on the system (see the free-body diagram in the figure) and then use Newton's second law to find the acceleration.

Solution\\
Newton's second law is\\
\(\mathbf{a}=\frac{\mathbf{F}_{\text {net }}}{m}\).\\
4.21

The net external force on the system is the sum of the external forces: the force of the floor acting on the teacher, cart, and equipment (in the horizontal direction) and the force of friction. Because friction acts in the opposite direction, we assign it a negative value. Thus, for the net force, we obtain\\
\(\mathbf{F}_{\text {net }}=\mathbf{F}_{\text {oor }}-\mathbf{f}=150 \mathrm{~N}-24.0 \mathrm{~N}=126 \mathrm{~N}\).\\
4.22

The mass of the system is the sum of the mass of the teacher, cart, and equipment.\\
\(m=(65.0+12.0+7.0) \mathrm{kg}=84 \mathrm{~kg}\)\\
4.23

Insert these values of net F and \(m\) into Newton's second law to obtain the acceleration of the system.\\
\(\mathbf{a}=\frac{\mathbf{F}_{\text {net }}}{m}\)\\
\(a=\frac{126 \mathrm{~N}}{84 \mathrm{~kg}}=1.5 \mathrm{~m} / \mathrm{s}^{2}\)\\
4.24\\
\(F_{1}<F_{2}\)\\
4.25

Discussion\\
None of the forces between components of the system, such as between the teacher's hands and the cart, contribute to the net external force because they are internal to the system. Another way to look at this is to note that the forces between components of a system cancel because they are equal in magnitude and opposite in direction. For example, the force exerted by the teacher on the cart is of equal magnitude but in the opposite direction of the force exerted by\\
the cart on the teacher. In this case, both forces act on the same system, so they cancel. Defining the system was crucial to solving this problem.

\section*{Practice Problems}
14.

What is the equation for the normal force for a body with mass \(m\) that is at rest on a horizontal surface?\\
a. \(\mathrm{N}=m\)\\
b. \(\mathrm{N}=m g\)\\
c. \(\mathrm{N}=m v\)\\
d. \(\mathrm{N}=g\)\\
15.

An object with mass \(m\) is at rest on the floor. What is the magnitude and direction of the normal force acting on it?\\
a. \(\mathrm{N}=m v\) in upward direction\\
b. \(\mathrm{N}=m g\) in upward direction\\
c. \(\mathrm{N}=m v\) in downward direction\\
d. \(\mathrm{N}=m g\) in downward direction

\section*{Check Your Understanding}
\section*{Teacher Support}
Teacher Support Use the questions in Check Your Understanding to assess whether students have mastered the learning objectives of this section. If students are struggling with a specific objective, the Check Your Understanding assessment will help identify which objective is causing the problem and direct students to the relevant content.\\
16.

What is Newton's third law of motion?\\
a. Whenever a first body exerts a force on a second body, the first body experiences a force that is twice the magnitude and acts in the direction of the applied force.\\
b. Whenever a first body exerts a force on a second body, the first body experiences a force that is equal in magnitude and acts in the direction of the applied force.\\
c. Whenever a first body exerts a force on a second body, the first body experiences a force that is twice the magnitude but acts in the direction opposite the direction of the applied force.\\
d. Whenever a first body exerts a force on a second body, the first body experiences a force that is equal in magnitude but acts in the direction opposite the direction of the applied force.\\
17.

Considering Newton's third law, why don't two equal and opposite forces cancel out each other?\\
a. Because the two forces act in the same direction\\
b. Because the two forces have different magnitudes\\
c. Because the two forces act on different systems\\
d. Because the two forces act in perpendicular directions

Ke erm\\
dynamics the study of how forces affect the motion of objects and systems\\
external force a force acting on an object or system that originates outside of the object or system\\
force a push or pull on an object with a specific magnitude and direction; can be represented by vectors; can be expressed as a multiple of a standard force\\
free-body diagram a diagram showing all external forces acting on a body\\
freefall a situation in which the only force acting on an object is the force of gravity\\
friction an external force that acts in the direction opposite to the direction of motion\\
inertia the tendency of an object at rest to remain at rest, or for a moving object to remain in motion in a straight line and at a constant speed\\
law of inertia Newton's first law of motion: a body at rest remains at rest or, if in motion, remains in motion at a constant speed in a straight line, unless acted on by a net external force; also known as the law of inertia\\
mass the quantity of matter in a substance; measured in kilograms\\
net external force the sum of all external forces acting on an object or system\\
net force the sum of all forces acting on an object or system\\
Newton s first law of motion a body at rest remains at rest or, if in motion, remains in motion at a constant speed in a straight line, unless acted on by a net external force; also known as the law of inertia

Newton s second law of motion the net external force, \(\mathbf{F}_{\text {net }}\), on an object is proportional to and in the same direction as the acceleration of the object, a, and also proportional to the object's mass, \(m\); defined mathematically as \(\mathbf{F}_{\text {net }}=m \mathbf{a}\) or \(\quad \mathrm{F}=m \mathrm{a}\).

Newton \(s\) third law of motion when one body exerts a force on a second body, the first body experiences a force that is equal in magnitude and opposite in direction to the force that it exerts\\
normal force the force that a surface applies to an object; acts perpendicular and away from the surface with which the object is in contact\\
system one or more objects of interest for which only the forces acting on them from the outside are considered, but not the forces acting between them or inside them\\
tension a pulling force that acts along a connecting medium, especially a stretched flexible connector, such as a rope or cable; when a rope sup-\\
ports the weight of an object, the force exerted on the object by the rope is called tension\\
thrust a force that pushes an object forward in response to the backward ejection of mass by the object; rockets and airplanes are pushed forward by a thrust reaction force in response to ejecting gases backward\\
weight the force of gravity, \(\mathbf{W}\), acting on an object of mass \(m\); defined mathematically as \(\mathbf{W}=m \mathbf{g}\), where \(\mathbf{g}\) is the magnitude and direction of the acceleration due to gravity

Ke Eq a ion\\
4.2 Newton's First Law of Motion: Inertia

\subsection*{4.3 Newton's Second Law of Motion}
4.4 Newton's Third Law of Motion

\section*{ec ion mmar}
\subsection*{4.1 Force}
\begin{itemize}
  \item Dynamics is the study of how forces affect the motion of objects and systems.
  \item Force is a push or pull that can be defined in terms of various standards. It is a vector and so has both magnitude and direction.
  \item External forces are any forces outside of a body that act on the body. A free-body diagram is a drawing of all external forces acting on a body.
\end{itemize}

\subsection*{4.2 Newton's First Law of Motion: Inertia}
\begin{itemize}
  \item Newton's first law states that a body at rest remains at rest or, if moving, remains in motion in a straight line at a constant speed, unless acted on by a net external force. This law is also known as the law of inertia.
  \item Inertia is the tendency of an object at rest to remain at rest or, if moving, to remain in motion at constant velocity. Inertia is related to an object's mass.
  \item Friction is a force that opposes motion and causes an object or system to slow down.
  \item Mass is the quantity of matter in a substance.
\end{itemize}

\subsection*{4.3 Newton's Second Law of Motion}
\begin{itemize}
  \item Acceleration is a change in velocity, meaning a change in speed, direction, or both.
  \item An external force acts on a system from outside the system, as opposed to internal forces, which act between components within the system.
  \item Newton's second law of motion states that the acceleration of a system is directly proportional to and in the same direction as the net external force acting on the system, and inversely proportional to the system's mass.
  \item In equation form, Newton's second law of motion is \(\mathbf{F}_{\text {net }}=m \mathbf{a}\) or \(\quad \mathbf{F}= m\). This is sometimes written as \(\mathbf{a}=\frac{\mathbf{F}_{\text {net }}}{m}\) or \(\mathbf{a}=\frac{\mathbf{F}}{m}\).
  \item The weight of an object of mass \(m\) is the force of gravity that acts on it. From Newton's second law, weight is given by \(\mathrm{W}=m \mathrm{~g}\).
  \item If the only force acting on an object is its weight, then the object is in freefall.
\end{itemize}

\subsection*{4.4 Newton's Third Law of Motion}
\begin{itemize}
  \item Newton's third law of motion states that when one body exerts a force on a second body, the first body experiences a force that is equal in magnitude and opposite in direction to the force that it exerts.
  \item When an object rests on a surface, the surface applies a force on the object that opposes the weight of the object. This force acts perpendicular to the surface and is called the normal force.
  \item The pulling force that acts along a stretched flexible connector, such as a rope or cable, is called tension. When a rope supports the weight of an object at rest, the tension in the rope is equal to the weight of the object.
  \item Thrust is a force that pushes an object forward in response to the backward ejection of mass by the object. Rockets and airplanes are pushed forward by thrust.
\end{itemize}

\end{document}