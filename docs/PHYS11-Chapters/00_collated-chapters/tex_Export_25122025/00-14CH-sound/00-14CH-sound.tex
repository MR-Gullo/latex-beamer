\documentclass[10pt]{article}
\usepackage[utf8]{inputenc}
\usepackage[T1]{fontenc}
\usepackage{graphicx}
\usepackage[export]{adjustbox}
\graphicspath{ {./images/} }
\usepackage{caption}
\usepackage{amsmath}
\usepackage{amsfonts}
\usepackage{amssymb}
\usepackage[version=4]{mhchem}
\usepackage{stmaryrd}
\usepackage{hyperref}
\hypersetup{colorlinks=true, linkcolor=blue, filecolor=magenta, urlcolor=cyan,}
\urlstyle{same}

\title{The Relationship Between the Speed of Sound and the Frequency and Wavelength of a Sound Wave }

\author{}
\date{}


\begin{document}
\maketitle
\captionsetup{singlelinecheck=false}
\begin{figure}[h]
\begin{center}
  \includegraphics[max width=\textwidth]{971f4937-33ce-4297-8938-cac113e8d1f3-01}
\captionsetup{labelformat=empty}
\caption{Figure 14.1 This tree fell some time ago. When it fell, particles in the air were disturbed by the energy of the tree hitting the ground. This disturbance of matter, which our ears have evolved to detect, is called sound. (B.A. Bowen Photography)}
\end{center}
\end{figure}

\section*{Chapter Outline}
\subsection*{14.1 Speed of Sound, Frequency, and Wavelength}
14.2 Sound Intensity and Sound Level

\subsection*{14.3 Doppler Effect and Sonic Booms}
\subsection*{14.4 Sound Interference and Resonance}
\section*{Introduction}
\section*{Teacher Support}
Teacher Support [BL][OL][AL] Begin by asking students the old philosophical question given at the start of the chapter, "If a tree falls in the forest and no one is there to hear it, does it make a sound?" Ask them to give reasons for their responses. Ask them if they have seen Star Trek or Star Wars. Show a clip where a battle scene is taking place and you are able to hear explosions. Ask students if it is possible to hear another spaceship explode in space. Why or why not? In the discussion, ask them to think about what defines sound. Explain that sound is a mechanical wave. Refresh their memory about different waves and their properties.

If a tree falls in a forest (see Figure 14.1) and no one is there to hear it, does it make a sound? The answer to this old philosophical question depends on how\\
you define sound. If sound only exists when someone is around to perceive it, then the falling tree produced no sound. However, in physics, we know that colliding objects can disturb the air, water or other matter surrounding them. As a result of the collision, the surrounding particles of matter began vibrating in a wave-like fashion. This is a sound wave. Consequently, if a tree collided with another object in space, no one would hear it, because no sound would be produced. This is because, in space, there is no air, water or other matter to be disturbed and produce sound waves. In this chapter, we'll learn more about the wave properties of sound, and explore hearing, as well as some special uses for sound.

\subsection*{14.1 Speed of Sound, Frequency, and Wavelength}
\section*{Section Learning Objectives}
By the end of this section, you will be able to do the following:

\begin{itemize}
  \item Relate the characteristics of waves to properties of sound waves
  \item Describe the speed of sound and how it changes in various media
  \item Relate the speed of sound to frequency and wavelength of a sound wave
\end{itemize}

\section*{Teacher Support}
Teacher Support The learning objectives in this section will help your students master the following standards:

\begin{itemize}
  \item (7) Science concepts. The student knows the characteristics and behavior of waves. The student is expected to:
  \item (A) examine and describe oscillatory motion and wave propagation in various types of media;
  \item (B) investigate and analyze characteristics of waves, including velocity, frequency, amplitude, and wavelength, and calculate using the relationship between wave speed, frequency, and wavelength;
  \item (C) compare characteristics and behaviors of transverse waves, including electromagnetic waves and the electromagnetic spectrum, and characteristics and behaviors of longitudinal waves, including sound waves;
  \item (F) describe the role of wave characteristics and behaviors in medical and industrial applications.
\end{itemize}

In addition, the High School Physics Laboratory Manual addresses content in this section in the lab titled: Waves, as well as the following standards:

\begin{itemize}
  \item (7) Science concepts. The student knows the characteristics and behavior of waves. The student is expected to:
  \item (B) investigate and analyze characteristics of waves, including velocity, frequency, amplitude, and wavelength, and calculate using the relationship between wave speed, frequency, and wavelength.
\end{itemize}

\section*{Section Key Terms}
\section*{Teacher Support}
Teacher Support [BL][OL] Review waves and types of waves-mechanical and non-mechanical, transverse and longitudinal, pulse and periodic. Review\\
properties of waves-amplitude, period, frequency, velocity and their interrelations.

\section*{Properties of Sound Waves}
Sound is a wave. More specifically, sound is defined to be a disturbance of matter that is transmitted from its source outward. A disturbance is anything that is moved from its state of equilibrium. Some sound waves can be characterized as periodic waves, which means that the atoms that make up the matter experience simple harmonic motion.

A vibrating string produces a sound wave as illustrated in Figure 14.2, Figure 14.3, and Figure 14.4. As the string oscillates back and forth, part of the string's energy goes into compressing and expanding the surrounding air. This creates slightly higher and lower pressures. The higher pressure... regions are compressions, and the low pressure regions are rarefactions. The pressure disturbance moves through the air as longitudinal waves with the same frequency as the string. Some of the energy is lost in the form of thermal energy transferred to the air. You may recall from the chapter on waves that areas of compression and rarefaction in longitudinal waves (such as sound) are analogous to crests and troughs in transverse waves.

\begin{figure}[h]
\begin{center}
  \includegraphics[max width=\textwidth]{971f4937-33ce-4297-8938-cac113e8d1f3-04(1)}
\captionsetup{labelformat=empty}
\caption{Figure 14.2 A vibrating string moving to the right compresses the air in front of it and expands the air behind it.}
\end{center}
\end{figure}

\begin{figure}[h]
\begin{center}
  \includegraphics[max width=\textwidth]{971f4937-33ce-4297-8938-cac113e8d1f3-04}
\captionsetup{labelformat=empty}
\caption{Figure 14.3 As the string moves to the left, it creates another compression and rarefaction as the particles on the right move away from the string.}
\end{center}
\end{figure}

\begin{figure}[h]
\begin{center}
  \includegraphics[max width=\textwidth]{971f4937-33ce-4297-8938-cac113e8d1f3-05}
\captionsetup{labelformat=empty}
\caption{Figure 14.4 After many vibrations, there is a series of compressions and rarefactions that have been transmitted from the string as a sound wave. The graph shows gauge pressure ( \(\mathrm{P}_{\text {gauge }}\) ) versus distance \(x\) from the source. Gauge pressure is the pressure relative to atmospheric pressure; it is positive for pressures above atmospheric pressure, and negative for pressures below it. For ordinary, everyday sounds, pressures vary only slightly from average atmospheric pressure.}
\end{center}
\end{figure}

The amplitude of a sound wave decreases with distance from its source, because the energy of the wave is spread over a larger and larger area. But some of the energy is also absorbed by objects, such as the eardrum in Figure 14.5, and some of the energy is converted to thermal energy in the air. Figure 14.4 shows a graph of gauge pressure versus distance from the vibrating string. From this figure, you can see that the compression of a longitudinal wave is analogous to the peak of a transverse wave, and the rarefaction of a longitudinal wave is analogous to the trough of a transverse wave. Just as a transverse wave alternates between peaks and troughs, a longitudinal wave alternates between compression and rarefaction.\\
\includegraphics[max width=\textwidth, center]{971f4937-33ce-4297-8938-cac113e8d1f3-05(1)}

Figure 14.5 Sound wave compressions and rarefactions travel up the ear canal and force the eardrum to vibrate. There is a net force on the eardrum, since the sound wave pressures differ from the atmospheric pressure found behind the eardrum. A complicated mechanism converts the vibrations to nerve impulses, which are then interpreted by the brain.

\section*{The Speed of Sound}
\section*{Teacher Support}
Teacher Support [BL] Review the fact that sound is a mechanical wave and requires a medium through which it is transmitted.\\[0pt]
[OL][AL] Ask students if they know the speed of sound and if not, ask them to take a guess. Ask them why the sound of thunder is heard much after the lightning is seen during storms. This phenomenon is also observed during a display of fireworks. Through this discussion, develop the concept that the speed of sound is finite and measurable and is much slower than that of light.

The speed of sound varies greatly depending upon the medium it is traveling through. The speed of sound in a medium is determined by a combination of the medium's rigidity (or compressibility in gases) and its density. The more rigid (or less compressible) the medium, the faster the speed of sound. The greater the density of a medium, the slower the speed of sound. The speed of sound in air is low, because air is compressible. Because liquids and solids are relatively rigid and very difficult to compress, the speed of sound in such media is generally greater than in gases. Table 14.1 shows the speed of sound in various media. Since temperature affects density, the speed of sound varies with the temperature of the medium through which it's traveling to some extent, especially for gases.

\section*{Teacher Support}
\section*{Teacher Support}
\section*{Misconception Alert}
Students might be confused between rigidity and density and how they affect the speed of sound. The speed of sound is slower in denser media. Solids are denser than gases. However, they are also very rigid, and hence sound travels faster in solids. Stress on the fact that the speed of sound always depends on a combination of these two properties of any medium.

Table 14.1 Speed of Sound in Various Media

\section*{Teacher Support}
Teacher Support [BL] Note that in the table, the speed of sound in very rigid materials such as glass, aluminum, and steel ... is quite high, whereas the speed in rubber, which is considerably less rigid, is quite low.

\begin{figure}[h]
\begin{center}
  \includegraphics[max width=\textwidth]{971f4937-33ce-4297-8938-cac113e8d1f3-08}
\captionsetup{labelformat=empty}
\caption{Figure 14.6 When fireworks explode in the sky, the light energy is perceived before the sound energy. Sound travels more slowly than light does. (Dominic Alves, Flickr)}
\end{center}
\end{figure}

Sound, like all waves, travels at certain speeds through different media and has the properties of frequency and wavelength. Sound travels much slower than light-you can observe this while watching a fireworks display (see Figure 14.6), since the flash of an explosion is seen before its sound is heard.

The relationship between the speed of sound, its frequency, and wavelength is the same as for all waves:\\
\(v=f \lambda\),\\
where \(v\) is the speed of sound (in units of \(\mathrm{m} / \mathrm{s}\) ), \(f\) is its frequency (in units of hertz), and \(\lambda\) is its wavelength (in units of meters). Recall that wavelength is defined as the distance between adjacent identical parts of a wave. The wavelength of a sound, therefore, is the distance between adjacent identical parts of a sound wave. Just as the distance between adjacent crests in a transverse wave is one wavelength, the distance between adjacent compressions in a sound wave is also one wavelength, as shown in Figure 14.7. The frequency of a sound wave is the same as that of the source. For example, a tuning fork vibrating at a given frequency would produce sound waves that oscillate at that same frequency. The frequency of a sound is the number of waves that pass a point per unit time.

\begin{figure}[h]
\begin{center}
  \includegraphics[max width=\textwidth]{971f4937-33ce-4297-8938-cac113e8d1f3-09}
\captionsetup{labelformat=empty}
\caption{Figure 14.7 A sound wave emanates from a source vibrating at a frequency \(f\), propagates at \(v\), and has a wavelength \(\lambda\).}
\end{center}
\end{figure}

\section*{Teacher Support}
Teacher Support [BL][OL][AL] In musical instruments, shorter strings vibrate faster and hence produce sounds at higher pitches. Fret placements on instruments such as guitars, banjos, and mandolins, are mathematically determined to give the correct interval or change in pitch. When the string is pushed against the fret wire, the string is effectively shortened, changing its pitch. Ask students to experiment with strings of different lengths and observe how the pitch changes in each case.

One of the more important properties of sound is that its speed is nearly independent of frequency. If this were not the case, and high-frequency sounds traveled faster, for example, then the farther you were from a band in a football stadium, the more the sound from the low-pitch instruments would lag behind the high-pitch ones. But the music from all instruments arrives in cadence independent of distance, and so all frequencies must travel at nearly the same speed.

Recall that \(v=f \lambda\), and in a given medium under fixed temperature and humidity, \(v\) is constant. Therefore, the relationship between \(f\) and \(\lambda\) is inverse: The higher the frequency, the shorter the wavelength of a sound wave.

\section*{Teacher Support}
\section*{Teacher Support}
\section*{Teacher Demonstration}
Hold a meter stick flat on a desktop, with about 80 cm sticking out over the edge of the desk. Make the meter stick vibrate by pulling the tip down and releasing, while holding the meter stick tight to the desktop. While it is vibrating, move the stick back onto the desktop, shortening the part that is sticking out. Students will see the shortening of the vibrating part of the meter stick, and hear the pitch or number of vibrations go up-an increase in frequency.

The speed of sound can change when sound travels from one medium to another. However, the frequency usually remains the same because it is like a driven oscillation and maintains the frequency of the original source. If \(v\) changes and \(f\) remains the same, then the wavelength \(\lambda\) must change. Since \(v=f \lambda\), the higher the speed of a sound, the greater its wavelength for a given frequency.

\section*{Teacher Support}
Teacher Support [AL] Ask students to predict what would happen if the speeds of sound in air varied by frequency.

\section*{Virtual Physics}
Sound Click to view content\\
This simulation lets you see sound waves. Adjust the frequency or amplitude (volume) and you can see and hear how the wave changes. Move the listener around and hear what she hears. Switch to the Two Source Interference tab or the Interference by Reflection tab to experiment with interference and reflection.

\section*{Tips For Success}
Make sure to have audio enabled and set to Listener rather than Speaker, or else the sound will not vary as you move the listener around.

PhET Explorations: Sound. This simulation lets you see sound waves. Adjust the frequency or volume and you can see and hear how the wave changes. Move the listener around and hear what she hears.

Click to view content\\
In the first tab, Listen to a Single Source, move the listener as far away from the speaker as possible, and then change the frequency of the sound wave. You may have noticed that there is a delay between the time when you change the\\
setting and the time when you hear the sound get lower or higher in pitch. Why is this?\\
a. Because, intensity of the sound wave changes with the frequency.\\
b. Because, the speed of the sound wave changes when the frequency is changed.\\
c. Because, loudness of the sound wave takes time to adjust after a change in frequency.\\
d. Because it takes time for sound to reach the listener, so the listener perceives the new frequency of sound wave after a delay.

Is there a difference in the amount of delay depending on whether you make the frequency higher or lower? Why?\\
a. Yes, the speed of propagation depends only on the frequency of the wave.\\
b. Yes, the speed of propagation depends upon the wavelength of the wave, and wavelength changes as the frequency changes.\\
c. No, the speed of propagation depends only on the wavelength of the wave.\\
d. No, the speed of propagation is constant in a given medium; only the wavelength changes as the frequency changes.

\section*{Snap Lab}
Voice as a Sound Wave In this lab you will observe the effects of blowing and speaking into a piece of paper in order to compare and contrast different sound waves.

\begin{itemize}
  \item sheet of paper
  \item tape
  \item table
\end{itemize}

Instructions\\
Procedure

\begin{enumerate}
  \item Suspend a sheet of paper so that the top edge of the paper is fixed and the bottom edge is free to move. You could tape the top edge of the paper to the edge of a table, for example.
  \item Gently blow air near the edge of the bottom of the sheet and note how the sheet moves.
  \item Speak softly and then louder such that the sounds hit the edge of the bottom of the paper, and note how the sheet moves.
  \item Interpret the results.
\end{enumerate}

\section*{Grasp Check}
Which sound wave property increases when you are speaking more loudly than softly?\\
a. amplitude of the wave\\
b. frequency of the wave\\
c. speed of the wave\\
d. wavelength of the wave

\section*{Worked Example}
What Are the Wavelengths of Audible Sounds? Calculate the wavelengths of sounds at the extremes of the audible range, 20 and \(20,000 \mathrm{~Hz}\), in conditions where sound travels at \(348.7 \mathrm{~m} / \mathrm{s}\).

\section*{Strategy}
To find wavelength from frequency, we can use \(v=f \lambda\).\\
Solution\\
(1) Identify the knowns. The values for \(v\) and \(f\) are given.\\
(2) Solve the relationship between speed, frequency and wavelength for \(\lambda\).\\
\(\lambda=\frac{v}{f}\).\\
14.2\\
(3) Enter the speed and the minimum frequency to give the maximum wavelength.\\
\(\lambda_{\text {max }}=\frac{348.7 \mathrm{~m} / \mathrm{s}}{20 \mathrm{~Hz}}=17 \mathrm{~m} \approx 20 \mathrm{~m}\) (1 sig. figure)\\
14.3\\
(4) Enter the speed and the maximum frequency to give the minimum wavelength.\\
\(\lambda_{\text {min }}=\frac{348.7 \mathrm{~m} / \mathrm{s}}{20,000 \mathrm{~Hz}}=0.017 \mathrm{~m} \approx 2 \mathrm{~cm}\) (1 sig. figure)\\
14.4

Discussion\\
Because the product of \(f\) multiplied by \(\lambda\) equals a constant velocity in unchanging conditions, the smaller \(f\) is, the larger \(\lambda\) must be, and vice versa. Note that you can also easily rearrange the same formula to find frequency or velocity.

\section*{Practice Problems}
1.

What is the speed of a sound wave with frequency \(2000 \backslash, \backslash \operatorname{text}\{\mathrm{~Hz}\}\) and wavelength \(0.4 \backslash, \backslash \operatorname{text}\{\mathrm{~m}\}\) ?\\
a. \(5 \backslash\) times \(10^{\wedge} 3 \backslash, \backslash \operatorname{text}\{\mathrm{~m}\} / \backslash \operatorname{text}\{\mathrm{s}\}\)\\
b. \(3.2 \backslash\) times \(10^{\wedge} 2 \backslash, \backslash \operatorname{text}\{\mathrm{~m}\} / \backslash \operatorname{text}\{\mathrm{s}\}\)\\
c. \(2 \backslash\) times \(10^{\wedge}\{-4\} \backslash, \backslash \operatorname{text}\{\mathrm{m} / \mathrm{s}\}\)\\
d. \(8 \backslash\) times \(10^{\wedge} 2 \backslash, \backslash \operatorname{text}\{\mathrm{~m}\} / \backslash \operatorname{text}\{\mathrm{s}\}\)

\section*{2.}
Dogs can hear frequencies of up to \(45 \backslash, \backslash \operatorname{text}\{\mathrm{kHz}\}\). What is the wavelength of a sound wave with this frequency traveling in air at \(0^{\wedge}\{\backslash \operatorname{circ}\} \backslash \operatorname{text}\{\mathrm{C}\}\) ?\\
a. \(2.0 \backslash\) times \(10^{\wedge} 7 \backslash, \backslash \operatorname{text}\{\mathrm{~m}\}\)\\
b. \(1.5 \backslash\) times \(10^{\wedge} 7 \backslash, \backslash \operatorname{text}\{\mathrm{~m}\}\)\\
c. \(1.4 \backslash\) times \(10^{\wedge} 2 \backslash, \backslash \operatorname{text}\{\mathrm{~m}\}\)\\
d. \(7.4 \backslash\) times \(10^{\wedge}\{-3\} \backslash, \backslash \operatorname{text}\{\mathrm{m}\}\)

\section*{Links To Physics}
\section*{Echolocation}
\begin{center}
\includegraphics[max width=\textwidth]{971f4937-33ce-4297-8938-cac113e8d1f3-13}
\end{center}

Figure 14.8 A bat uses sound echoes to find its way about and to catch prey. The time for the echo to return is directly proportional to the distance.

Echolocation is the use of reflected sound waves to locate and identify objects. It is used by animals such as bats, dolphins and whales, and is also imitated by humans in SONAR-Sound Navigation and Ranging-and echolocation technology.

Bats, dolphins and whales use echolocation to navigate and find food in their environment. They locate an object (or obstacle) by emitting a sound and then sensing the reflected sound waves. Since the speed of sound in air is constant, the time it takes for the sound to travel to the object and back gives the animal a sense of the distance between itself and the object. This is called ranging. Figure 14.8 shows a bat using echolocation to sense distances.

Echolocating animals identify an object by comparing the relative intensity of the sound waves returning to each ear to figure out the angle at which the sound waves were reflected. This gives information about the direction, size and shape of the object. Since there is a slight distance in position between the two ears of an animal, the sound may return to one of the ears with a bit of a delay, which also provides information about the position of the object. For example, if a bear is directly to the right of a bat, the echo will return to the bat's left ear later than to its right ear. If, however, the bear is directly ahead of the bat, the echo would return to both ears at the same time. For an animal without a sense of sight such as a bat, it is important to know where other animals are as well as what they are; their survival depends on it.

Principles of echolocation have been used to develop a variety of useful sensing technologies. SONAR, is used by submarines to detect objects underwater and measure water depth. Unlike animal echolocation, which relies on only one transmitter (a mouth) and two receivers (ears), manmade SONAR uses many transmitters and beams to get a more accurate reading of the environment. Radar technologies use the echo of radio waves to locate clouds and storm systems in weather forecasting, and to locate aircraft for air traffic control. Some new cars use echolocation technology to sense obstacles around the car, and warn the driver who may be about to hit something (or even to automatically parallel park). Echolocation technologies and training systems are being developed to help visually impaired people navigate their everyday environments.

If a predator is directly to the left of a bat, how will the bat know?\\
a. The echo would return to the left ear first.\\
b. The echo would return to the right ear first.

\section*{Check Your Understanding}
\section*{Teacher Support}
Teacher Support Use these questions to assess student achievement of the section's Learning Objectives. If students are struggling with a specific objective, these questions will help identify which and direct students to the relevant content.\\
3.

What is a rarefaction?\\
a. Rarefaction is the high-pressure region created in a medium when a longitudinal wave passes through it.\\
b. Rarefaction is the low-pressure region created in a medium when a longitudinal wave passes through it.\\
c. Rarefaction is the highest point of amplitude of a sound wave.\\
d. Rarefaction is the lowest point of amplitude of a sound wave.\\
4.

What sort of motion do the particles of a medium experience when a sound wave passes through it?\\
a. Simple harmonic motion\\
b. Circular motion\\
c. Random motion\\
d. Translational motion\\
5.

What does the speed of sound depend on?\\
a. The wavelength of the wave\\
b. The size of the medium\\
c. The frequency of the wave\\
d. The properties of the medium\\
6.

What property of a gas would affect the speed of sound traveling through it?\\
a. The volume of the gas\\
b. The flammability of the gas\\
c. The mass of the gas\\
d. The compressibility of the gas

\subsection*{14.2 Sound Intensity and Sound Level}
\section*{Section Learning Objectives}
By the end of this section, you will be able to do the following:

\begin{itemize}
  \item Relate amplitude of a wave to loudness and energy of a sound wave
  \item Describe the decibel scale for measuring sound intensity
  \item Solve problems involving the intensity of a sound wave
  \item Describe how humans produce and hear sounds
\end{itemize}

\section*{Teacher Support}
Teacher Support The learning objectives in this section will help your students master the following standards:

\begin{itemize}
  \item (7) Science concepts. The student knows the characteristics and behavior of waves. The student is expected to:
  \item (C) compare characteristics and behaviors of transverse waves, including electromagnetic waves and the electromagnetic spectrum, and characteristics and behaviors of longitudinal waves, including sound waves;
  \item (F) describe the role of wave characteristics and behaviors in medical and industrial applications.
\end{itemize}

\section*{Section Key Terms}
\section*{Teacher Support}
Teacher Support [BL] Review sound, properties of sound waves and characteristics of sound waves.

\section*{Amplitude, Loudness and Energy of a Sound Wave}
\begin{figure}[h]
\begin{center}
  \includegraphics[max width=\textwidth]{971f4937-33ce-4297-8938-cac113e8d1f3-17}
\captionsetup{labelformat=empty}
\caption{Figure 14.9 Noise on crowded roadways like this one in Delhi makes it hard to hear others unless they shout. (Lingaraj G J, Flickr)}
\end{center}
\end{figure}

\section*{Teacher Support}
\section*{Teacher Support}
\section*{Misconception Alert}
Students may be confused between amplitude and intensity. While sound intensity is proportional to amplitude, they are different physical quantities. Sound intensity is defined as the sound power per unit area, whereas amplitude is the distance between the resting position and the crest of a wave.

In a quiet forest, you can sometimes hear a single leaf fall to the ground. But in\\
a traffic jam filled with honking cars, you may have to shout just so the person next to you can hear Figure 14.9.The loudness of a sound is related to how energetically its source is vibrating. In cartoons showing a screaming person, the cartoonist often shows an open mouth with a vibrating uvula (the hanging tissue at the back of the mouth) to represent a loud sound coming from the throat. Figure 14.10 shows such a cartoon depiction of a bird loudly expressing its opinion.

A useful quantity for describing the loudness of sounds is called sound intensity. In general, the intensity of a wave is the power per unit area carried by the wave. Power is the rate at which energy is transferred by the wave. In equation form, intensity \(I\) is\\
\(I=\frac{P}{A}\),\\
14.5\\
where \(P\) is the power through an area \(A\). The SI unit for \(I\) is \(\mathrm{W} / \mathrm{m}^{2}\). The intensity of a sound depends upon its pressure amplitude. The relationship between the intensity of a sound wave and its pressure amplitude (or pressure variation \(\Delta p\) ) is\\
\(I=\frac{(\Delta p)^{2}}{2 \rho v_{w}}\),\\
14.6\\
where is the density of the material in which the sound wave travels, in units of \(\mathrm{kg} / \mathrm{m}^{3}\), and \(v\) is the speed of sound in the medium, in units of \(\mathrm{m} / \mathrm{s}\). Pressure amplitude has units of pascals ( Pa ) or \(\mathrm{N} / \mathrm{m}^{2}\). Note that \(\Delta p\) is half the difference between the maximum and minimum pressure in the sound wave.

We can see from the equation that the intensity of a sound is proportional to its amplitude squared. The pressure variation is proportional to the amplitude of the oscillation, and so \(I\) varies as \((\Delta p)^{2}\). This relationship is consistent with the fact that the sound wave is produced by some vibration; the greater its pressure amplitude, the more the air is compressed during the vibration. Because the power of a sound wave is the rate at which energy is transferred, the energy of a sound wave is also proportional to its amplitude squared.

\section*{Teacher Support}
Teacher Support [OL][AL] Note from the equation that the intensity of sound is also affected by the density of the material that it travels through. The denser the material, the lower the intensity of sound.

\section*{Tips For Success}
Pressure is usually denoted by capital \(P\), but we are using a lowercase \(p\) for pressure in this case to distinguish it from power \(P\) above.

\begin{figure}[h]
\begin{center}
  \includegraphics[max width=\textwidth]{971f4937-33ce-4297-8938-cac113e8d1f3-19}
\captionsetup{labelformat=empty}
\caption{Figure 14.10 Graphs of the pressures in two sound waves of different intensities. The more intense sound is produced by a source that has larger-amplitude oscillations and has greater pressure maxima and minima. Because pressures are higher in the greater-intensity sound, it can exert larger forces on the objects it encounters.}
\end{center}
\end{figure}

\section*{Teacher Support}
Teacher Support [BL][OL][AL] Ask students whether the pitch of both birds differs. How can they tell by looking at the graph?

\section*{The Decibel Scale}
You may have noticed that when people talk about the loudness of a sound, they describe it in units of decibels rather than watts per meter squared. While sound intensity (in \(\mathrm{W} / \mathrm{m}^{2}\) ) is the SI unit, the sound intensity level in decibels ( dB ) is more relevant for how humans perceive sounds. The way our ears perceive sound can be more accurately described by the logarithm of the intensity of a sound rather than the intensity of a sound directly. The sound intensity level is defined to be\\
\(\beta(\mathrm{dB})=10 \log _{10}\left(\frac{I}{I_{0}}\right)\),

\section*{14.7}
where \(I\) is sound intensity in watts per meter squared, and \(I=10^{12} \mathrm{~W} / \mathrm{m}^{2}\) is a reference intensity. \(I\) is chosen as the reference point because it is the lowest intensity of sound a person with normal hearing can perceive. The decibel level of a sound having an intensity of \(10^{12} \mathrm{~W} / \mathrm{m}^{2}\) is \(=0 \mathrm{~dB}\), because \(\log _{1} 1=0\). That is, the threshold of human hearing is 0 decibels.

Each factor of 10 in intensity corresponds to 10 dB . For example, a 90 dB sound compared with a 60 dB sound is 30 dB greater, or three factors of 10 (that is, \(10^{3}\) times) as intense. Another example is that if one sound is \(10^{7}\) as intense as another, it is 70 dB higher.

Since is defined in terms of a ratio, it is unit-less. The unit called decibel (dB) is used to indicate that this ratio is multiplied by 10 . The sound intensity level is not the same as sound intensity-it tells you the level of the sound relative to a reference intensity rather than the actual intensity.

\section*{Teacher Support}
Teacher Support [BL][OL][AL] Note that decibel is different from other units in that it is not an absolute measurement. It is a ratio of two measurements. It is useful, and more widely used because it is closer to how humans perceive sound.

\section*{Snap Lab}
Feeling Sound In this lab, you will play music with a heavy beat to literally feel the vibrations and explore what happens when the volume changes.

\begin{itemize}
  \item CD player or portable electronic device connected to speakers
  \item rock or rap music CD or mp3
  \item a lightweight table
\end{itemize}

Procedure

\begin{enumerate}
  \item Place the speakers on a light table, and start playing the CD or mp3.
  \item Place your hand gently on the table next to the speakers.
  \item Increase the volume and note the level when the table just begins to vibrate as the music plays.
  \item Increase the reading on the volume control until it doubles. What has happened to the vibrations?
\end{enumerate}

Do you think that when you double the volume of a sound wave you are doubling the sound intensity level (in dB) or the sound intensity (in \(\left.\backslash \operatorname{text}\{\mathrm{W}\} / \backslash \operatorname{text}\{\mathrm{m}\}^{\wedge}\{2\}\right)\) ? Why?\\
a. The sound intensity in \(\backslash \operatorname{text}\{\mathrm{W}\} / \backslash \operatorname{text}\{\mathrm{m}\}^{\wedge} 2\), because it is a closer measure of how humans perceive sound.\\
b. The sound intensity level in \(\backslash \operatorname{text}\{\mathrm{dB}\}\) because it is a closer measure of how humans perceive sound.\\
c. The sound intensity in \(\backslash \operatorname{text}\{\mathrm{W}\} / \backslash \operatorname{text}\{\mathrm{m}\}^{\wedge} 2\) because it is the only unit to express the intensity of sound.\\
d. The sound intensity level in \(\backslash \operatorname{text}\{\mathrm{dB}\}\) because it is the only unit to express the intensity of sound.

\section*{Solving Sound Wave Intensity Problems}
\section*{Worked Example}
Calculating Sound Intensity Levels: Sound Waves Calculate the sound intensity level in decibels for a sound wave traveling in air at \(0{ }^{\circ} \mathrm{C}\) and having a pressure amplitude of 0.656 Pa .

\section*{Strategy}
We are given \(\Delta p\), so we can calculate \(I\) using the equation \(I=\frac{(\Delta p)^{2}}{2 \rho v}\). Using \(I\), we can calculate straight from its definition in \(\beta(d \mathrm{~B})=10 \log _{10}\left(\frac{I}{I_{0}}\right) \beta(\mathrm{dB})=10 \log _{10}\left(\frac{I}{I_{0}}\right)\).\\
Solution\\
(1) Identify knowns:

Sound travels at \(331 \mathrm{~m} / \mathrm{s}\) in air at \(0^{\circ} \mathrm{C}\).\\
Air has a density of \(1.29 \mathrm{~kg} / \mathrm{m}^{3}\) at atmospheric pressure and \(0^{\circ} \mathrm{C}\).\\
(2) Enter these values and the pressure amplitude into \(I=\frac{(\Delta p)^{2}}{2 \rho v_{w}}\).\\
\(I=\frac{(\Delta p)^{2}}{2 \rho v_{w}}=\frac{(0.656 \mathrm{~Pa})^{2}}{2\left(1.29 \mathrm{~kg} / \mathrm{m}^{3}\right)(331 \mathrm{~m} / \mathrm{s})}=5.04 \times 10^{-4} \mathrm{~W} / \mathrm{m}^{2}\).\\
(3) Enter the value for \(I\) and the known value for \(I\) into \(\beta(\mathrm{dB})=10 \log _{10}\left(\frac{I}{I_{0}}\right)\) . Calculate to find the sound intensity level in decibels.\\
\(10 \log _{10}\left(5.04 \times 10^{8}\right)=10(8.70) \mathrm{dB}=87.0 \mathrm{~dB}\).\\
Discussion\\
This 87.0 dB sound has an intensity five times as great as an 80 dB sound. So a factor of five in intensity corresponds to a difference of 7 dB in sound intensity level. This value is true for any intensities differing by a factor of five.

\section*{Worked Example}
Change Intensity Levels of a Sound: What Happens to the Decibel Level? Show that if one sound is twice as intense as another, it has a sound level about 3 dB higher.

\section*{Strategy}
You are given that the ratio of two intensities is 2 to 1 , and are then asked to find the difference in their sound levels in decibels. You can solve this problem using of the properties of logarithms.

Solution\\
(1) Identify knowns:

The ratio of the two intensities is 2 to 1 , or:\\
\(\frac{I_{2}}{I_{1}}=2.00\).\\
We want to show that the difference in sound levels is about 3 dB . That is, we want to show\\
\(\beta_{2}-\beta_{1}=3 \mathrm{~dB}\).\\
14.8

Note that\\
\(\log _{10} b-\log _{10} a=\log _{10}\left(\frac{b}{a}\right)\).\\
14.9\\
(2) Use the definition of to get\\
\(\beta_{2}-\beta_{1}=10 \log _{10}\left(\frac{I_{2}}{I_{1}}\right)=10 \log _{10} 2.00=10(0.301) \mathrm{dB}\).\\
14.10

Therefore,\\
\(\beta_{2}-\beta_{1}=3.01 \mathrm{~dB}\).\\
Discussion\\
This means that the two sound intensity levels differ by 3.01 dB , or about 3 dB , as advertised. Note that because only the ratio \(I_{2} / I_{1}\) is given (and not the actual intensities), this result is true for any intensities that differ by a factor of two. For example, a 56.0 dB sound is twice as intense as a 53.0 dB sound, a 97.0 dB sound is half as intense as a 100 dB sound, and so on.

\section*{Practice Problems}
7.

Calculate the intensity of a wave if the power transferred is 10 W and the area through which the wave is transferred is 5 square meters.\\
a. \(200 \mathrm{~W} / \mathrm{m}^{2}\)\\
b. \(50 \mathrm{~W} / \mathrm{m}^{2}\)\\
c. \(0.5 \mathrm{~W} / \mathrm{m}^{2}\)\\
d. \(2 \mathrm{~W} / \mathrm{m}^{2}\)\\
8.

Calculate the sound intensity for a sound wave traveling in air at \(0^{\wedge}\{\backslash \operatorname{circ}\} \backslash \operatorname{text}\{\mathrm{C}\}\) and having a pressure amplitude of \(0.90 \backslash, \backslash \operatorname{text}\{\mathrm{~Pa}\}\).\\
a. \(1.8 \backslash\) times \(10^{\wedge}\{-3\} \backslash \operatorname{text}\{\mathrm{W}\} / \backslash \operatorname{text}\{\mathrm{m}\}^{\wedge} 2\)\\
b. \(4.2 \backslash\) times \(10^{\wedge}\{-3\} \backslash, \backslash \operatorname{text}\{\mathrm{W}\} / \backslash \operatorname{text}\{\mathrm{m}\}^{\wedge}\{2\}\)\\
c. \(1.1 \backslash\) times \(10^{\wedge}\{3\} \backslash, \backslash \operatorname{text}\{\mathrm{W}\} / \backslash \operatorname{text}\{\mathrm{m}\}^{\wedge} 2\)\\
d. \(9.5 \backslash\) times \(10^{\wedge}\{-4\} \backslash, \backslash \operatorname{text}\{\mathrm{W}\} / \backslash \operatorname{text}\{\mathrm{m}\}^{\wedge} 2\)

\section*{Hearing and Voice}
People create sounds by pushing air up through their lungs and through elastic folds in the throat called vocal cords. These folds open and close rhythmically, creating a pressure buildup. As air travels up and past the vocal cords, it causes them to vibrate. This vibration escapes the mouth along with puffs of air as sound. A voice changes in pitch when the muscles of the larynx relax or tighten, changing the tension on the vocal chords. A voice becomes louder when air flow from the lungs increases, making the amplitude of the sound pressure wave greater.

Hearing is the perception of sound. It can give us plenty of information - such as pitch, loudness, and direction. Humans can normally hear frequencies ranging from approximately 20 to \(20,000 \mathrm{~Hz}\). Other animals have hearing ranges different from that of humans. Dogs can hear sounds as high as \(45,000 \mathrm{~Hz}\), whereas bats and dolphins can hear up to \(110,000 \mathrm{~Hz}\) sounds. You may have noticed that dogs respond to the sound of a dog whistle which produces sound out of the range of human hearing.

Sounds below 20 Hz are called infrasound, whereas those above \(20,000 \mathrm{~Hz}\) are ultrasound. The perception of frequency is called pitch, and the perception of intensity is called loudness.

The way we hear involves some interesting physics. The sound wave that hits our ear is a pressure wave. The ear converts sound waves into electrical nerve impulses, similar to a microphone.

Figure 14.11 shows the anatomy of the ear with its division into three parts: the outer ear or ear canal; the middle ear, which runs from the eardrum to the cochlea; and the inner ear, which is the cochlea itself. The body part normally referred to as the ear is technically called the pinna.

\begin{figure}[h]
\begin{center}
  \includegraphics[max width=\textwidth]{971f4937-33ce-4297-8938-cac113e8d1f3-24}
\captionsetup{labelformat=empty}
\caption{Figure 14.11 The illustration shows the anatomy of the human ear.}
\end{center}
\end{figure}

The outer ear, or ear canal, carries sound to the eardrum protected inside of the ear. The middle ear converts sound into mechanical vibrations and applies these vibrations to the cochlea. The lever system of the middle ear takes the force exerted on the eardrum by sound pressure variations, amplifies it and transmits it to the inner ear via the oval window. Two muscles in the middle ear protect the inner ear from very intense sounds. They react to intense sound in a few milliseconds and reduce the force transmitted to the cochlea. This protective reaction can also be triggered by your own voice, so that humming during a fireworks display, for example, can reduce noise damage.

Figure 14.12 shows the middle and inner ear in greater detail. As the middle ear bones vibrate, they vibrate the cochlea, which contains fluid. This creates pressure waves in the fluid that cause the tectorial membrane to vibrate. The motion of the tectorial membrane stimulates tiny cilia on specialized cells called hair cells. These hair cells, and their attached neurons, transform the motion of the tectorial membrane into electrical signals that are sent to the brain.

The tectorial membrane vibrates at different positions based on the frequency of the incoming sound. This allows us to detect the pitch of sound. Additional processing in the brain also allows us to determine which direction the sound is coming from (based on comparison of the sound's arrival time and intensity between our two ears).

\begin{figure}[h]
\begin{center}
  \includegraphics[max width=\textwidth]{971f4937-33ce-4297-8938-cac113e8d1f3-25}
\captionsetup{labelformat=empty}
\caption{Figure 14.12 The inner ear, or cochlea, is a coiled tube about 3 mm in diameter and 3 cm in length when uncoiled. As the stapes vibrates against the oval window, it creates pressure waves that travel through fluid in the cochlea. These waves vibrate the tectorial membrane, which bends the cilia and stimulates nerves in the organ of Corti. These nerves then send information about the sound to the brain.}
\end{center}
\end{figure}

\section*{Fun In Physics}
\section*{Musical Instruments}
\begin{figure}[h]
\begin{center}
  \includegraphics[max width=\textwidth]{971f4937-33ce-4297-8938-cac113e8d1f3-26}
\captionsetup{labelformat=empty}
\caption{Figure 14.13 Playing music, also known as "rocking out", involves creating vibrations using musical instruments. (John Norton)}
\end{center}
\end{figure}

Yet another way that people make sounds is through playing musical instruments (see the previous figure). Recall that the perception of frequency is called pitch. You may have noticed that the pitch range produced by an instrument tends to depend upon its size. Small instruments, such as a piccolo, typically make high-pitch sounds, while larger instruments, such as a tuba, typically make low-pitch sounds. High-pitch means small wavelength, and the size of a musical instrument is directly related to the wavelengths of sound it produces. So a small instrument creates short-wavelength sounds, just as a large instrument creates long-wavelength sounds.

Most of us have excellent relative pitch, which means that we can tell whether one sound has a different frequency from another. We can usually distinguish one sound from another if the frequencies of the two sounds differ by as little as 1 Hz . For example, 500.0 and 501.5 Hz are noticeably different.

Musical notes are particular sounds that can be produced by most instruments, and are the building blocks of a song. In Western music, musical notes have particular names, such as A-sharp, C, or E-flat. Some people can identify musical notes just by listening to them. This rare ability is called perfect, or absolute, pitch.

When a violin plays middle C , there is no mistaking it for a piano playing the same note. The reason is that each instrument produces a distinctive set of frequencies and intensities. We call our perception of these combinations of frequencies and intensities the timbre of the sound. It is more difficult to quantify timbre than loudness or pitch. Timbre is more subjective. Evocative adjectives such as dull, brilliant, warm, cold, pure, and rich are used to describe the timbre of a sound rather than quantities with units, which makes for a difficult topic to dissect with physics. So the consideration of timbre takes us into the realm of perceptual psychology, where higher-level processes in the brain are dominant. This is also true for other perceptions of sound, such as music and noise. But as a teenager, you are likely already aware that one person's music may be another person's noise.

If you turn up the volume of your stereo, will the pitch change? Why or why not?\\
a. No, because pitch does not depend on intensity.\\
b. Yes, because pitch is directly related to intensity.

\section*{Check Your Understanding}
\section*{Teacher Support}
Teacher Support Use these questions to assess student achievement of the section's Learning Objectives. If students are struggling with a specific objective, these questions will help identify which and direct students to the relevant content.\\
9.

What is sound intensity?\\
a. Intensity is the energy per unit area carried by a wave.\\
b. Intensity is the energy per unit volume carried by a wave.\\
c. Intensity is the power per unit area carried by a wave.\\
d. Intensity is the power per unit volume carried by a wave.\\
10.

How is power defined with reference to a sound wave?\\
a. Power is the rate at which energy is transferred by a sound wave.\\
b. Power is the rate at which mass is transferred by a sound wave.\\
c. Power is the rate at which amplitude of a sound wave changes.\\
d. Power is the rate at which wavelength of a sound wave changes.\\
11.

What word or phrase is used to describe the loudness of sound?\\
a. frequency or oscillation\\
b. intensity level or decibel\\
c. timbre\\
d. pitch\\
12.

What is the mathematical expression for sound intensity level \textbackslash beta?\\
a. \(\backslash\) beta \(\backslash\) left \((\backslash\) text \(\{\mathrm{dB}\} \backslash\) right \()=10 \backslash \log \_\{10\} \backslash \operatorname{left}\left(\backslash \operatorname{frac}\left\{\mathrm{I} \_0\right\}\{\mathrm{I}\} \backslash\right.\) right \()\)\\
b. \(\backslash\) beta \(\backslash\) left \((\backslash \operatorname{text}\{\mathrm{dB}\} \backslash\) right \()=20 \backslash \log \_\{10\} \backslash \operatorname{left}\left(\backslash \operatorname{frac}\{\mathrm{I}\}\left\{\mathrm{I} \_0\right\} \backslash\right.\) right \()\)\\
c. \(\backslash\) beta \(\backslash \operatorname{left}(\backslash \operatorname{text}\{\mathrm{dB}\} \backslash\) right \()=20 \backslash \log \_\{10\} \backslash \operatorname{left}\left(\backslash \operatorname{frac}\left\{\mathrm{I} \_0\right\}\{\mathrm{I}\} \backslash\right.\) right \()\)\\
d. \(\backslash\) beta \(\backslash\) left \((\backslash \operatorname{text}\{\mathrm{dB}\} \backslash\) right \()=10 \backslash \log \_\{10\} \backslash \operatorname{left}\left(\backslash \operatorname{frac}\{\mathrm{I}\}\left\{\mathrm{I} \_0\right\} \backslash\right.\) right \()\)\\
13.

What is the range of frequencies that humans are capable of hearing?\\
a. 20 Hz to \(200,000 \mathrm{~Hz}\)\\
b. 2 Hz to \(50,000 \mathrm{~Hz}\)\\
c. 2 Hz to \(2,000 \mathrm{~Hz}\)\\
d. 20 Hz to \(20,000 \mathrm{~Hz}\)\\
14.

How do humans change the pitch of their voice?\\
a. Relaxing or tightening their glottis\\
b. Relaxing or tightening their uvula\\
c. Relaxing or tightening their tongue\\
d. Relaxing or tightening their larynx

\section*{References}
Nave, R. Vocal sound production-HyperPhysics. Retrieved from \href{http://hyperphysics.phyastr.gsu.edu/hbase/music/voice.html}{http://hyperphysics.phyastr.gsu.edu/hbase/music/voice.html}

\subsection*{14.3 Doppler E ect and Sonic Booms}
\section*{Section Learning Objectives}
By the end of this section, you will be able to do the following:

\begin{itemize}
  \item Describe the Doppler effect of sound waves
  \item Explain a sonic boom
  \item Calculate the frequency shift of sound from a moving object by the Doppler shift formula, and calculate the speed of an object by the Doppler shift formula
\end{itemize}

\section*{Teacher Support}
Teacher Support The learning objectives in this section will help your students master the following standards:

\begin{itemize}
  \item (7) Science concepts. The student knows the characteristics and behavior of waves. The student is expected to:
  \item (D) investigate behaviors of waves, including reflection, refraction, diffraction, interference, resonance, and the Doppler effect.
\end{itemize}

\section*{Section Key Terms}
\section*{Teacher Support}
Teacher Support [BL] Before the start of this section, it would be useful to review the properties of sound waves and how they are related to each other.

\section*{The Doppler Effect of Sound Waves}
The Doppler effect is a change in the observed pitch of a sound, due to relative motion between the source and the observer. An example of the Doppler effect due to the motion of a source occurs when you are standing still, and the sound of a siren coming from an ambulance shifts from high-pitch to low-pitch as it passes by. The closer the ambulance is to you, the more sudden the shift. The faster the ambulance moves, the greater the shift. We also hear this shift in frequency for passing race cars, airplanes, and trains. An example of the Doppler effect with a stationary source and moving observer is if you ride a train past a stationary warning bell, you will hear the bell's frequency shift from high to low as you pass by.

\section*{Teacher Support}
Teacher Support [BL][OL][AL] Ask students if they have ever experienced the phenomenon where a car horn or siren appears to change its pitch as the vehicle passes them by. If so, when did it appear to be higher? And when was it lower? You can do a demonstration of the Doppler effect in class using a buzzer and a string. Tie the buzzer to one end of a string. A buzzer produces a monotonous sound. However, when you swing it around your head, its pitch appears to change. Ask students how they think this happens. What could be the reason for the changing pitch?

Safety warning: Make sure the buzzer is secured tightly to the string before swinging.

What causes the Doppler effect? Let's compare three different scenarios: Sound waves emitted by a stationary source (Figure 14.14), sound waves emitted by a moving source (Figure 14.15), and sound waves emitted by a stationary source but heard by moving observers (Figure 14.16). In each case, the sound spreads out from the point where it was emitted.

If the source and observers are stationary, then observers on either side see the same wavelength and frequency as emitted by the source. But if the source is moving and continues to emit sound as it travels, then the air compressions (crests) become closer together in the direction in which it's traveling and farther apart in the direction it's traveling away from. Therefore, the wavelength is shorter in the direction the source is moving (on the right in Figure 14.15), and longer in the opposite direction (on the left in Figure 14.15).

Finally, if the observers move, as in Figure 14.16, the frequency at which they receive the compressions changes. The observer moving toward the source receives them at a higher frequency (and therefore shorter wavelength), and the person moving away from the source receives them at a lower frequency (and therefore longer wavelength).

\section*{Teacher Support}
\section*{Teacher Support}
\section*{Misconception Alert}
Be sure to point out that the Doppler effect is only experienced due to the relative motion between the source and the observer and does not depend on the actual speed of either.

Students might think that the Doppler effect occurs only with sound waves. This is not the case. It can occur with any kind of waves. In fact, we see it in the light waves that reach us from distant stars. Here, the effect is observed in the form of color change.

\begin{figure}[h]
\begin{center}
  \includegraphics[max width=\textwidth]{971f4937-33ce-4297-8938-cac113e8d1f3-31}
\captionsetup{labelformat=empty}
\caption{Figure 14.14 Sounds emitted by a source spread out in spherical waves. Because the source, observers, and air are stationary, the wavelength and frequency are the same in all directions and to all observers.}
\end{center}
\end{figure}

\begin{figure}[h]
\begin{center}
  \includegraphics[max width=\textwidth]{971f4937-33ce-4297-8938-cac113e8d1f3-31(1)}
\captionsetup{labelformat=empty}
\caption{Figure 14.15 Sounds emitted by a source moving to the right spread out from the points at which they were emitted. The wavelength is reduced and, consequently, the frequency is increased in the direction of motion, so that the observer on the right hears a higher-pitch sound. The opposite is true for the observer on the left, where the wavelength is increased and the frequency is reduced.}
\end{center}
\end{figure}

\begin{figure}[h]
\begin{center}
\texttt{https://cdn.mathpix.com/cropped/971f4937-33ce-4297-8938-cac113e8d1f3-31.jpg?height=345&width=521&top_left_y=1542&top_left_x=455}
\captionsetup{labelformat=empty}
\caption{Figure 14.16 The same effect is produced when the observers move relative to the source. Motion toward the source increases frequency as the observer on the right passes through more wave crests than she would if stationary. Motion away from the source decreases frequency as the observer on the left passes through fewer wave crests than he would if stationary.}
\end{center}
\end{figure}

We know that wavelength and frequency are related by \(v=f \lambda\), where \(v\) is the fixed speed of sound. The sound moves in a medium and has the same speed \(v\) in that medium whether the source is moving or not. Therefore, \(f\) multiplied by \(\lambda\) is a constant. Because the observer on the right in Figure 14.15 receives a\\
shorter wavelength, the frequency she perceives must be higher. Similarly, the observer on the left receives a longer wavelength and therefore perceives a lower frequency.

The same thing happens in Figure 14.16. A higher frequency is perceived by the observer moving toward the source, and a lower frequency is perceived by an observer moving away from the source. In general, then, relative motion of source and observer toward one another increases the perceived frequency. Relative motion apart decreases the perceived frequency. The greater the relative speed is, the greater the effect.

\section*{Watch Physics}
Introduction to the Doppler Effect This video explains the Doppler effect visually.

Click to view content

\section*{Grasp Check}
If you are standing on the sidewalk facing the street and an ambulance drives by with its siren blaring, at what point will the frequency that you observe most closely match the actual frequency of the siren?\\
a. when it is coming toward you\\
b. when it is going away from you\\
c. when it is in front of you

For a stationary observer and a moving source of sound, the frequency ( \(f_{\text {obs }}\) ) of sound perceived by the observer is\\
\(f_{o b s}=f_{s}\left(\frac{v_{w}}{v_{w} \pm v_{s}}\right)\),\\
14.11\\
where \(f_{\mathrm{s}}\) is the frequency of sound from a source, \(v_{\mathrm{s}}\) is the speed of the source along a line joining the source and observer, and \(v_{\mathrm{w}}\) is the speed of sound. The minus sign is used for motion toward the observer and the plus sign for motion away from the observer.

\section*{Tips For Success}
Rather than just memorizing rules, which are easy to forget, it is better to think about the rules of an equation intuitively. Using a minus sign in \(f_{\text {obs }}= f_{s}\left(\frac{v_{w}}{v_{w} \pm v_{s}}\right)\) will decrease the denominator and increase the observed frequency, which is consistent with the expected outcome of the Doppler effect when the source is moving toward the observer. Using a plus sign will increase the denominator and decrease the observed frequency, consistent with what you would expect for the source moving away from the observer. This may be more helpful to keep in mind rather than memorizing the fact that "the minus sign is used\\
for motion toward the observer and the plus sign for motion away from the observer."

Note that the greater the speed of the source, the greater the Doppler effect. Similarly, for a stationary source and moving observer, the frequency perceived by the observer \(f_{\text {obs }}\) is given by\\
\(f_{o b s}=f_{s}\left(\frac{v_{w} \pm v_{o b s}}{v_{w}}\right)\),\\
14.12\\
where \(v_{\text {obs }}\) is the speed of the observer along a line joining the source and observer. Here the plus sign is for motion toward the source, and the minus sign is for motion away from the source.

\section*{Sonic Booms}
What happens to the sound produced by a moving source, such as a jet airplane, that approaches or even exceeds the speed of sound? Suppose a jet airplane is coming nearly straight at you, emitting a sound of frequency \(f_{\mathrm{s}}\). The greater the plane's speed, \(v_{\mathrm{s}}\), the greater the Doppler shift and the greater the value of \(f_{\text {obs }}\). Now, as \(v_{\mathrm{s}}\) approaches the speed of sound, \(v_{\mathrm{w}}, f_{\text {obs }}\) approaches infinity, because the denominator in \(f_{\text {obs }}=f_{s}\left(\frac{v_{w}}{v_{w}-v_{s}}\right)\) approaches zero.

\section*{Teacher Support}
Teacher Support [BL][OL][AL]The equation shows that a sonic boom is created as the observed frequency approaches infinity. Ask students what happens to the amplitude of the sound wave at this time. The Doppler effect only changes the frequency of the sound. However, when all the waves are superimposed on one another, and their crests match, the amplitude will also tend to infinity. This is what increases the intensity of the wave, creating the boom.

This result means that at the speed of sound, in front of the source, each wave is superimposed on the previous one because the source moves forward at the speed of sound. The observer gets them all at the same instant, and so the frequency is theoretically infinite. If the source exceeds the speed of sound, no sound is received by the observer until the source has passed, so that the sounds from the source when it was approaching are stacked up with those from it when receding, creating a sonic boom. A sonic boom is a constructive interference of sound created by an object moving faster than sound.

An aircraft creates two sonic booms, one from its nose and one from its tail (see Figure 14.17). During television coverage of space shuttle landings, two distinct booms could often be heard. These were separated by exactly the time it would take the shuttle to pass by a point. Observers on the ground often do not observe the aircraft creating the sonic boom, because it has passed by before the shock wave reaches them. If the aircraft flies close by at low altitude, pressures in\\
the sonic boom can be destructive enough to break windows. Because of this, supersonic flights are banned over populated areas of the United States.\\
\includegraphics[max width=\textwidth, center]{971f4937-33ce-4297-8938-cac113e8d1f3-34}

Figure 14.17 Two sonic booms, created by the nose and tail of an aircraft, are observed on the ground after the plane has passed by.

\section*{Solving Problems Using the Doppler Shift Formula}
\section*{Watch Physics}
Doppler Effect Formula for Observed Frequency This video explains the Doppler effect formula for cases when the source is moving toward the observer.

Click to view content

\section*{Grasp Check}
Let's say that you have a rare phobia where you are afraid of the Doppler effect. If you see an ambulance coming your way, what would be the best Strategy to minimize the Doppler effect and soothe your Doppleraphobia?\\
a. Stop moving and become stationary till it passes by.\\
b. Run toward the ambulance.\\
c. Run alongside the ambulance.

\section*{Watch Physics}
Doppler Effect Formula When Source is Moving Away This video explains the Doppler effect formula for cases when the source is moving away from the observer.

Click to view content\\
Watch Physics: Doppler Effect Formula when Source is Moving Away. This video explains how the Doppler Effect formula differs when the source is moving\\
away.\\
Click to view content\\
Sal uses two different formulas for the Doppler effect-one for when the source is moving toward the observer and another for when the source is moving away. However, in this textbook we use only one formula. Explain.\\
a. The combined formula that can be used is, Use (+) when the source is moving toward the observer and (-) when the source is moving away from the observer.\\
b. The combined formula that can be used is, \(\mathrm{f} \_\{\mathrm{obs}\}=\mathrm{f} \_\mathrm{s} \backslash \operatorname{left}\left(\backslash \operatorname{frac}\left\{\mathrm{v} \_\mathrm{w} \backslash \mathrm{pm}\right.\right.\) v\_s \(\}\{\) v\_w \(\} \backslash\) right). Use \((+)\) when the source is moving away from the observer and ( - ) when the source is moving toward the observer.\\
c. The combined formula that can be used is, f\_\_\{obs\}=f\_s \(\backslash \operatorname{left}\left(\backslash \operatorname{frac}\left\{\mathrm{v} \_\mathrm{w}\right\}\left\{\mathrm{v} \_\mathrm{w} \backslash \mathrm{pm}\right.\right.\) v\_s \(\} \backslash\) right) . Use \((+)\) when the source is moving toward the observer and ( - ) when the source is moving away from the observer.\\
d. The combined formula that can be used is, \(\mathrm{f} \_\{\mathrm{obs}\}=\mathrm{f} \_\mathrm{s} \backslash \operatorname{left}( \backslash\) frac\{v\_w\}\{v\_\_w \textbackslash pm v\_\_s\} \textbackslash right). Use ( + ) when the source is moving away from the observer and (-) when the source is moving toward the observer.

\section*{Worked Example}
Calculate Doppler Shift: A Train Horn Suppose a train that has a 150 Hz horn is moving at \(35 \mathrm{~m} / \mathrm{s}\) in still air on a day when the speed of sound is \(340 \mathrm{~m} / \mathrm{s}\). What frequencies are observed by a stationary person at the side of the tracks as the train approaches and after it passes?

\section*{Strategy}
To find the observed frequency, \(f_{\text {obs }}=f_{s}\left(\frac{v_{w}}{v_{w} \pm v_{s}}\right)\) must be used because the source is moving. The minus sign is used for the approaching train, and the plus sign for the receding train.

Solution\\
(1) Enter known values into \(f_{\text {obs }}=f_{s}\left(\frac{v_{w}}{v_{w}-v_{s}}\right)\) to calculate the frequency observed by a stationary person as the train approaches:\\
\(f_{\text {obs }}=f_{s}\left(\frac{v_{w}}{v_{w}-v_{s}}\right)=(150 \mathrm{~Hz})\left(\frac{340 \mathrm{~m} / \mathrm{s}}{340 \mathrm{~m} / \mathrm{s}-35 \mathrm{~m} / \mathrm{s}}\right)\)\\
\(=167 \mathrm{~Hz} \approx 170 \mathrm{~Hz}\) (2 sig. figs.)\\
(2) Use the same equation but with the plus sign to find the frequency heard by a stationary person as the train recedes.\\
\(f_{\text {obs }}=f_{s}\left(\frac{v_{w}}{v_{w}+v_{s}}\right)=(150 \mathrm{~Hz})\left(\frac{340 \mathrm{~m} / \mathrm{s}}{340 \mathrm{~m} / \mathrm{s}+35 \mathrm{~m} / \mathrm{s}}\right)\)\\
\(=136 \mathrm{~Hz} \approx 140 \mathrm{~Hz}\) (2 sig. figs.)\\
Discussion\\
The numbers calculated are valid when the train is far enough away that the motion is nearly along the line joining the train and the observer. In both cases, the shift is significant and easily noticed. Note that the shift is approximately 20 Hz for motion toward and approximately 10 Hz for motion away. The shifts are not symmetric.

\section*{Practice Problems}
15.

What is the observed frequency when the source having frequency \(3.0 \backslash, \backslash \operatorname{text}\{\mathrm{kHz}\}\) is moving towards the observer at a speed of \(1.0 \backslash\) times \(10^{\wedge} 2 \backslash, \backslash \operatorname{text}\{\mathrm{~m}\} / \backslash \operatorname{text}\{\mathrm{s}\}\) and the speed of sound is \(331 \backslash, \backslash \operatorname{text}\{\mathrm{~m} / \mathrm{s}\}\) ?\\
a. \(3.0 \backslash, \backslash \operatorname{text}\{\mathrm{kHz}\}\)\\
b. \(3.5 \backslash, \backslash \operatorname{text}\{\mathrm{kHz}\}\)\\
c. \(2.3 \backslash, \backslash \operatorname{text}\{\mathrm{kHz}\}\)\\
d. \(4.3 \backslash, \backslash \operatorname{text}\{\mathrm{kHz}\}\)\\
16.

A train is moving away from you at a speed of \(50.0 \backslash, \backslash \operatorname{text}\{\mathrm{~m} / \mathrm{s}\}\). If you are standing still and hear the whistle at a frequency of \(305 \backslash, \backslash \operatorname{text}\{\mathrm{~Hz}\}\), what is the actual frequency of the produced whistle? (Assume speed of sound to be \(331 \backslash, \backslash \operatorname{text}\{\mathrm{~m} / \mathrm{s}\}\).)\\
a. \(259 \backslash, \backslash \operatorname{text}\{\mathrm{~Hz}\}\)\\
b. \(205 \backslash, \backslash \operatorname{text}\{\mathrm{~Hz}\}\)\\
c. \(405 \backslash, \backslash \operatorname{text}\{\mathrm{~Hz}\}\)\\
d. \(351 \backslash, \backslash \operatorname{text}\{\mathrm{~Hz}\}\)

\section*{Check Your Understanding}
\section*{Teacher Support}
Teacher Support Use these questions to assess student achievement of the section's Learning Objectives. If students are struggling with a specific objective, these questions will help identify which and direct students to the relevant content.\\
17.

What is the Doppler effect?\\
a. The Doppler effect is a change in the observed speed of a sound due to the relative motion between the source and the observer.\\
b. The Doppler effect is a change in the observed frequency of a sound due to the relative motion between the source and the observer.\\
c. The Doppler effect is a change in the observed intensity of a sound due to the relative motion between the source and the observer.\\
d. The Doppler effect is a change in the observed timbre of a sound, due to the relative motion between the source and the observer.\\
18.

Give an example of the Doppler effect caused by motion of the source.\\
a. The sound of a vehicle horn shifts from low-pitch to high-pitch as we move towards it.\\
b. The sound of a vehicle horn shifts from low-pitch to high-pitch as we move away from it.\\
c. The sound of a vehicle horn shifts from low-pitch to high-pitch as it passes by.\\
d. The sound of a vehicle horn shifts from high-pitch to low-pitch as it passes by.\\
19.

What is a sonic boom?\\
a. It is a destructive interference of sound created by an object moving faster than sound.\\
b. It is a constructive interference of sound created by an object moving faster than sound.\\
c. It is a destructive interference of sound created by an object moving slower than sound.\\
d. It is a constructive interference of sound created by an object moving slower than sound.\\
20.

What is the relation between speed of source and value of observed frequency when the source is moving towards the observer?\\
a. They are independent of each other.\\
b. The greater the speed, the greater the value of observed frequency.\\
c. The greater the speed, the smaller the value of observed frequency.\\
d. The speed of the sound is directly proportional to the square of the frequency observed.

\subsection*{14.4 Sound Interference and Resonance}
\section*{Section Learning Objectives}
By the end of this section, you will be able to do the following:

\begin{itemize}
  \item Describe resonance and beats
  \item Define fundamental frequency and harmonic series
  \item Contrast an open-pipe and closed-pipe resonator
  \item Solve problems involving harmonic series and beat frequency
\end{itemize}

\section*{Teacher Support}
Teacher Support The learning objectives in this section will help your students master the following standards:

\begin{itemize}
  \item (7) Science concepts. The student knows the characteristics and behavior of waves. The student is expected to:
  \item (D) investigate behaviors of waves, including reflection, refraction, diffraction, interference, resonance, and the Doppler effect.
\end{itemize}

In addition, the High School Physics Laboratory Manual addresses content in this section in the lab titled: Sound Waves, as well as the following standards:

\begin{itemize}
  \item (7) Science concepts. The student knows the characteristics and behavior of waves. The student is expected to:
  \item (D) investigate behaviors of waves, including reflection, refraction, diffraction, interference, resonance, and the Doppler effect.
\end{itemize}

\section*{Section Key Terms}
\section*{Teacher Support}
Teacher Support [BL]Before the start of this section, it would be useful to review the properties of sound waves and how they are related to each other, standing waves, superposition and interference of waves.

\section*{Resonance and Beats}
Sit in front of a piano sometime and sing a loud brief note at it while pushing down on the sustain pedal. It will sing the same note back at you - the strings that have the same frequencies as your voice, are resonating in response to the forces from the sound waves that you sent to them. This is a good example of\\
the fact that objects-in this case, piano strings-can be forced to oscillate but oscillate best at their natural frequency.

A driving force (such as your voice in the example) puts energy into a system at a certain frequency, which is not necessarily the same as the natural frequency of the system. Over time the energy dissipates, and the amplitude gradually reduces to zero- this is called damping. The natural frequency is the frequency at which a system would oscillate if there were no driving and no damping force. The phenomenon of driving a system with a frequency equal to its natural frequency is called resonance, and a system being driven at its natural frequency is said to resonate.

Most of us have played with toys where an object bobs up and down on an elastic band, something like the paddle ball suspended from a finger in Figure 14.18. At first you hold your finger steady, and the ball bounces up and down with a small amount of damping. If you move your finger up and down slowly, the ball will follow along without bouncing much on its own. As you increase the frequency at which you move your finger up and down, the ball will respond by oscillating with increasing amplitude. When you drive the ball at its natural frequency, the ball's oscillations increase in amplitude with each oscillation for as long as you drive it. As the driving frequency gets progressively higher than the resonant or natural frequency, the amplitude of the oscillations becomes smaller, until the oscillations nearly disappear and your finger simply moves up and down with little effect on the ball.

\begin{figure}[h]
\begin{center}
  \includegraphics[max width=\textwidth]{971f4937-33ce-4297-8938-cac113e8d1f3-39}
\captionsetup{labelformat=empty}
\caption{Figure 14.18 The paddle ball on its rubber band moves in response to the finger supporting it. If the finger moves with the natural frequency of the ball on the rubber band, then a resonance is achieved, and the amplitude of the ball's oscillations increases dramatically. At higher and lower driving frequencies, energy is transferred to the ball less efficiently, and it responds with lower-amplitude oscillations.}
\end{center}
\end{figure}

Another example is that when you tune a radio, you adjust its resonant frequency so that it oscillates only at the desired station's broadcast (driving) frequency. Also, a child on a swing is driven (pushed) by a parent at the swing's natural frequency to reach the maximum amplitude (height). In all of these cases, the efficiency of energy transfer from the driving force into the oscillator\\
is best at resonance.

\begin{figure}[h]
\begin{center}
  \includegraphics[max width=\textwidth]{971f4937-33ce-4297-8938-cac113e8d1f3-40}
\captionsetup{labelformat=empty}
\caption{Figure 14.19 Some types of headphones use the phenomena of constructive and destructive interference to cancel out outside noises.}
\end{center}
\end{figure}

\section*{Teacher Support}
Teacher Support [BL][OL][AL] Tuning forks and pipes may be used to demonstrate the concept of resonance. Use any pipe or tube closed at one end. Fix it so that it stands upright with the open end on top. Choose a tuning fork and strike it to make it vibrate. Place it near the mouth of the pipe and hear the sound. Now fill the pipe with some water and repeat. The changing water level changes the length of the resonating air column. Continue doing this. When a certain length is obtained, the sound of the tuning fork will resonate through the column.

All sound resonances are due to constructive and destructive interference. Only the resonant frequencies interfere constructively to form standing waves, while others interfere destructively and are absent. From the toot made by blowing over a bottle to the recognizability of a great singer's voice, resonance and standing waves play a vital role in sound.

Interference happens to all types of waves, including sound waves. In fact, one way to support that something is a wave is to observe interference effects. Figure 14.19 shows a set of headphones that employs a clever use of sound interference to cancel noise. To get destructive interference, a fast electronic analysis is performed, and a second sound is introduced with its maxima and minima exactly reversed from the incoming noise.

In addition to resonance, superposition of waves can also create beats. Beats are produced by the superposition of two waves with slightly different frequencies but the same amplitude. The waves alternate in time between constructive interference and destructive interference, giving the resultant wave an amplitude\\
that varies over time. (See the resultant wave in Figure 14.20).\\
This wave fluctuates in amplitude, or beats, with a frequency called the beat frequency. The equation for beat frequency is\\
\(f_{B}=\left|f_{1}-f_{2}\right|\),\\
14.13\\
where \(f_{1}\) and \(f_{2}\) are the frequencies of the two original waves. If the two frequencies of sound waves are similar, then what we hear is an average frequency that gets louder and softer at the beat frequency.

\section*{Tips For Success}
Don't confuse the beat frequency with the regular frequency of a wave resulting from superposition. While the beat frequency is given by the formula above, and describes the frequency of the beats, the actual frequency of the wave resulting from superposition is the average of the frequencies of the two original waves.\\
\includegraphics[max width=\textwidth, center]{971f4937-33ce-4297-8938-cac113e8d1f3-41}

Figure 14.20 Beats are produced by the superposition of two waves of slightly different frequencies but identical amplitudes. The waves alternate in time between constructive interference and destructive interference, giving the resulting wave a time-varying amplitude.

\section*{Virtual Physics}
Wave Interference Click to view content\\
For this activity, switch to the Sound tab. Turn on the Sound option, and experiment with changing the frequency and amplitude, and adding in a second speaker and a barrier.

According to the graph, what happens to the amplitude of pressure over time. What is this phenomenon called, and what causes it ?\\
a. The amplitude decreases over time. This phenomenon is called damping. It is caused by the dissipation of energy.\\
b. The amplitude increases over time. This phenomenon is called feedback. It is caused by the gathering of energy.\\
c. The amplitude oscillates over time. This phenomenon is called echoing. It is caused by fluctuations in energy.

\section*{Fundamental Frequency and Harmonics}
Suppose we hold a tuning fork near the end of a tube that is closed at the other end, as shown in Figure 14.21, Figure 14.22, and Figure 14.23. If the tuning fork has just the right frequency, the air column in the tube resonates loudly, but at most frequencies it vibrates very little. This means that the air column has only certain natural frequencies. The figures show how a resonance at the lowest of these natural frequencies is formed. A disturbance travels down the tube at the speed of sound and bounces off the closed end. If the tube is just the right length, the reflected sound arrives back at the tuning fork exactly half a cycle later, and it interferes constructively with the continuing sound produced by the tuning fork. The incoming and reflected sounds form a standing wave in the tube as shown.

\begin{figure}[h]
\begin{center}
\texttt{https://cdn.mathpix.com/cropped/971f4937-33ce-4297-8938-cac113e8d1f3-42.jpg?height=319&width=513&top_left_y=966&top_left_x=459}
\captionsetup{labelformat=empty}
\caption{Figure 14.21 Resonance of air in a tube closed at one end, caused by a tuning fork. A disturbance moves down the tube.}
\end{center}
\end{figure}

\begin{figure}[h]
\begin{center}
  \includegraphics[max width=\textwidth]{971f4937-33ce-4297-8938-cac113e8d1f3-42(1)}
\captionsetup{labelformat=empty}
\caption{Figure 14.22 Resonance of air in a tube closed at one end, caused by a tuning fork. The disturbance reflects from the closed end of the tube.}
\end{center}
\end{figure}

\begin{figure}[h]
\begin{center}
  \includegraphics[max width=\textwidth]{971f4937-33ce-4297-8938-cac113e8d1f3-42}
\captionsetup{labelformat=empty}
\caption{Figure 14.23 Resonance of air in a tube closed at one end, caused by a tuning fork. If the length of the tube \(L\) is just right, the disturbance gets back to the}
\end{center}
\end{figure}

tuning fork half a cycle later and interferes constructively with the continuing sound from the tuning fork. This interference forms a standing wave, and the air column resonates.

The standing wave formed in the tube has its maximum air displacement (an antinode) at the open end, and no displacement (a node) at the closed end. Recall from the last chapter on waves that motion is unconstrained at the antinode, and halted at the node. The distance from a node to an antinode is one-fourth of a wavelength, and this equals the length of the tube; therefore, \(\lambda=4 L\). This same resonance can be produced by a vibration introduced at or near the closed end of the tube, as shown in Figure 14.24.

\begin{figure}[h]
\begin{center}
  \includegraphics[max width=\textwidth]{971f4937-33ce-4297-8938-cac113e8d1f3-43}
\captionsetup{labelformat=empty}
\caption{Figure 14.24 The same standing wave is created in the tube by a vibration introduced near its closed end.}
\end{center}
\end{figure}

Since maximum air displacements are possible at the open end and none at the closed end, there are other, shorter wavelengths that can resonate in the tube see Figure 14.25). Here the standing wave has three-fourths of its wavelength in the tube, or \(L=(3 / 4) \lambda^{\prime}\), so that \(\lambda^{\prime}=4 L / 3\). There is a whole series of shorter-wavelength and higher-frequency sounds that resonate in the tube.

We use specific terms for the resonances in any system. The lowest resonant frequency is called the fundamental, while all higher resonant frequencies are called overtones. All resonant frequencies are multiples of the fundamental, and are called harmonics. The fundamental is the first harmonic, the first overtone is the second harmonic, and so on. Figure 14.26 shows the fundamental and the first three overtones (the first four harmonics) in a tube closed at one end.

\begin{figure}[h]
\begin{center}
  \includegraphics[max width=\textwidth]{971f4937-33ce-4297-8938-cac113e8d1f3-43(1)}
\captionsetup{labelformat=empty}
\caption{Figure 14.25 Another resonance for a tube closed at one end. This has maximum air displacements at the open end, and none at the closed end. The wavelength is}
\end{center}
\end{figure}

shorter, with three-fourths \(\lambda^{\prime}\) equaling the length of the tube, so that \(\lambda^{\prime}=4 L / 3\) . This higher-frequency vibration is the first overtone.

\begin{figure}[h]
\begin{center}
  \includegraphics[max width=\textwidth]{971f4937-33ce-4297-8938-cac113e8d1f3-44}
\captionsetup{labelformat=empty}
\caption{Figure 14.26 The fundamental and three lowest overtones for a tube closed at one end. All have maximum air displacements at the open end and none at the closed end.}
\end{center}
\end{figure}

The fundamental and overtones can be present at the same time in a variety of combinations. For example, the note middle C on a trumpet sounds very different from middle C on a clarinet, even though both instruments are basically modified versions of a tube closed at one end. The fundamental frequency is the same (and usually the most intense), but the overtones and their mix of intensities are different. This mix is what gives musical instruments (and human voices) their distinctive characteristics, whether they have air columns, strings, or drumheads. In fact, much of our speech is determined by shaping the cavity formed by the throat and mouth and positioning the tongue to adjust the fundamental and combination of overtones.

\section*{Open-Pipe and Closed-Pipe Resonators}
The resonant frequencies of a tube closed at one end (known as a closed-pipe resonator) are\\
\(f_{n}=n \frac{v}{4 L}, n=1,3,5 \ldots\),\\
where \(f_{1}\) is the fundamental, \(f_{3}\) is the first overtone, and so on. Note that the resonant frequencies depend on the speed of sound \(v\) and on the length of the tube \(L\).

Another type of tube is one that is open at both ends (known as an open-pipe resonator). Examples are some organ pipes, flutes, and oboes. The air columns in tubes open at both ends have maximum air displacements at both ends. (See Figure 14.27). Standing waves form as shown.

\begin{figure}[h]
\begin{center}
  \includegraphics[max width=\textwidth]{971f4937-33ce-4297-8938-cac113e8d1f3-45}
\captionsetup{labelformat=empty}
\caption{Figure 14.27 The resonant frequencies of a tube open at both ends are shown, including the fundamental and the first three overtones. In all cases the maximum air displacements occur at both ends of the tube, giving it different natural frequencies than a tube closed at one end.}
\end{center}
\end{figure}

The resonant frequencies of an open-pipe resonator are\\
\(f_{n}=n \frac{v}{2 L}, n=1,2,3 \ldots\),\\
where \(f_{1}\) is the fundamental, \(f_{2}\) is the first overtone, \(f_{3}\) is the second overtone, and so on. Note that a tube open at both ends has a fundamental frequency twice what it would have if closed at one end. It also has a different spectrum of overtones than a tube closed at one end. So if you had two tubes with the same fundamental frequency but one was open at both ends and the other was closed at one end, they would sound different when played because they have different overtones.

Middle C, for example, would sound richer played on an open tube since it has more overtones. An open-pipe resonator has more overtones than a closedpipe resonator because it has even multiples of the fundamental as well as odd, whereas a closed tube has only odd multiples.

In this section we have covered resonance and standing waves for wind instruments, but vibrating strings on stringed instruments also resonate and have fundamentals and overtones similar to those for wind instruments.

\section*{Teacher Support}
Teacher Support [BL][OL][AL] Other instruments also use air resonance in different ways to amplify sound. For instance, a violin and a guitar both have sounding boxes but with different shapes, resulting in different overtone structures. The vibrating string creates a sound that resonates in the sounding box, greatly amplifying the sound and creating overtones that give the instrument its characteristic flavor. The more complex the shape of the sounding box, the greater its ability to resonate over a wide range of frequencies. The type and\\
thickness of wood or other materials used to make the sounding box also affects the quality of sound. Ask students to give more examples of how different musical instruments use the phenomenon of resonance.

\section*{Solving Problems Involving Harmonic Series and Beat Frequency}
\section*{Worked Example}
Finding the Length of a Tube for a Closed-Pipe Resonator If sound travels through the air at a speed of \(344 \mathrm{~m} / \mathrm{s}\), what should be the length of a tube closed at one end to have a fundamental frequency of 128 Hz ?

\section*{Strategy}
The length \(L\) can be found by rearranging the equation \(f_{n}=n \frac{v}{4 L}\).\\
Solution\\
(1) Identify knowns.

\begin{itemize}
  \item The fundamental frequency is 128 Hz .
  \item The speed of sound is \(344 \mathrm{~m} / \mathrm{s}\).\\
(2) Use \(f_{n}=n \frac{v_{w}}{4 L}\) to find the fundamental frequency ( \(n=1\) ).\\
\(f_{1}=\frac{v}{4 L}\)\\
14.14\\
(3) Solve this equation for length.\\
\(L=\frac{v}{4 f_{1}}\)\\
14.15\\
(4) Enter the values of the speed of sound and frequency into the expression for L.\\
\(L=\frac{v}{4 f_{1}}=\frac{344 \mathrm{~m} / \mathrm{s}}{4(128 \mathrm{~Hz})}=0.672 \mathrm{~m}\)\\
14.16
\end{itemize}

Discussion\\
Many wind instruments are modified tubes that have finger holes, valves, and other devices for changing the length of the resonating air column and therefore, the frequency of the note played. Horns producing very low frequencies, such as tubas, require tubes so long that they are coiled into loops.

\section*{Worked Example}
Finding the Third Overtone in an Open-Pipe Resonator If a tube that's open at both ends has a fundamental frequency of 120 Hz , what is the frequency of its third overtone?

\section*{Strategy}
Since we already know the value of the fundamental frequency ( \(\mathrm{n}=1\) ), we can solve for the third overtone ( \(\mathrm{n}=4\) ) using the equation \(f_{n}=n \frac{v}{2 L}\).

Solution\\
Since fundamental frequency ( \(\mathrm{n}=1\) ) is\\
\(f_{1}=\frac{v}{2 L}\),\\
14.17\\
and\\
\(f_{4}=4 \frac{v}{2 L}, f_{4}=4 f_{1}=4(120 \mathrm{~Hz})=480 \mathrm{~Hz}\).\\
14.18

Discussion\\
To solve this problem, it wasn't necessary to know the length of the tube or the speed of the air because of the relationship between the fundamental and the third overtone. This example was of an open-pipe resonator; note that for a closed-pipe resonator, the third overtone has a value of \(\mathrm{n}=7\) (not \(\mathrm{n}=4\) ).

\section*{Worked Example}
Using Beat Frequency to Tune a Piano Piano tuners use beats routinely in their work. When comparing a note with a tuning fork, they listen for beats and adjust the string until the beats go away (to zero frequency). If a piano tuner hears two beats per second, and the tuning fork has a frequency of 256 Hz , what are the possible frequencies of the piano?

\section*{Strategy}
Since we already know that the beat frequency \(f_{B}\) is 2 , and one of the frequencies (let's say \(f_{2}\) ) is 256 Hz , we can use the equation \(f_{B}=\left|f_{1}-f_{2}\right|\) to solve for the frequency of the piano \(f_{1}\).

Solution\\
Since \(f_{B}=\left|f_{1}-f_{2}\right|\),\\
we know that either \(f_{B}=f_{1}-f_{2}\) or \(-f_{B}=f_{1}-f_{2}\).

Solving for \(f_{1}\),\\
\(f_{1}=f_{B}+f_{2}\) or \(f_{1}=-f_{B}+f_{2}\).\\
14.19

Substituting in values,\\
\(f_{1}=2+256 \mathrm{~Hz}\) or \(f_{1}=-2+256 \mathrm{~Hz}\)\\
14.20

So,\\
\(f_{1}=258 \mathrm{~Hz}\) or 254 Hz .\\
14.21

Discussion\\
The piano tuner might not initially be able to tell simply by listening whether the frequency of the piano is too high or too low and must tune it by trial and error, making an adjustment and then testing it again. If there are even more beats after the adjustment, then the tuner knows that he went in the wrong direction.

\section*{Practice Problems}
21.

Two sound waves have frequencies \(250 \backslash, \backslash \operatorname{text}\{\mathrm{~Hz}\}\) and \(280 \backslash, \backslash \operatorname{text}\{\mathrm{~Hz}\}\). What is the beat frequency produced by their superposition?\\
a. \(290 \backslash, \backslash \operatorname{text}\{\mathrm{~Hz}\}\)\\
b. \(265 \backslash, \backslash \operatorname{text}\{\mathrm{~Hz}\}\)\\
c. \(60 \backslash, \backslash \operatorname{text}\{\mathrm{~Hz}\}\)\\
d. \(30 \backslash, \backslash \operatorname{text}\{\mathrm{~Hz}\}\)\\
22.

What is the length of a pipe closed at one end with fundamental frequency \(350 \backslash, \backslash \operatorname{text}\{\mathrm{~Hz}\}\) ? (Assume the speed of sound in air is \(331 \backslash, \backslash \operatorname{text}\{\mathrm{~m} / \mathrm{s}\}\).)\\
a. \(26 \backslash, \backslash \operatorname{text}\{\mathrm{~cm}\}\)\\
b. \(26 \backslash, \backslash \operatorname{text}\{\mathrm{~m}\}\)\\
c. \(24 \backslash, \backslash \operatorname{text}\{\mathrm{~m}\}\)\\
d. \(24 \backslash, \backslash \operatorname{text}\{\mathrm{~cm}\}\)

\section*{Check Your Understanding}
\section*{Teacher Support}
Teacher Support Use these questions to assess student achievement of the section's Learning Objectives. If students are struggling with a specific objective,\\
these questions will help identify it and direct students to the relevant content.\\
23.

What is damping?\\
a. Over time the energy increases and the amplitude gradually reduces to zero. This is called damping.\\
b. Over time the energy dissipates and the amplitude gradually increases. This is called damping.\\
c. Over time the energy increases and the amplitude gradually increases. This is called damping.\\
d. Over time the energy dissipates and the amplitude gradually reduces to zero. This is called damping.\\
24.

What is resonance? When can you say that the system is resonating?\\
a. The phenomenon of driving a system with a frequency equal to its natural frequency is called resonance, and a system being driven at its natural frequency is said to resonate.\\
b. The phenomenon of driving a system with a frequency higher than its natural frequency is called resonance, and a system being driven at its natural frequency does not resonate.\\
c. The phenomenon of driving a system with a frequency equal to its natural frequency is called resonance, and a system being driven at its natural frequency does not resonate.\\
d. The phenomenon of driving a system with a frequency higher than its natural frequency is called resonance, and a system being driven at its natural frequency is said to resonate.\\
25.

In the tuning fork and tube experiment, in case a standing wave is formed, at what point on the tube is the maximum disturbance from the tuning fork observed? Recall that the tube has one open end and one closed end.\\
a. At the midpoint of the tube\\
b. Both ends of the tube\\
c. At the closed end of the tube\\
d. At the open end of the tube\\
26.

In the tuning fork and tube experiment, when will the air column produce the loudest sound?\\
a. If the tuning fork vibrates at a frequency twice that of the natural frequency of the air column.\\
b. If the tuning fork vibrates at a frequency lower than the natural frequency of the air column.\\
c. If the tuning fork vibrates at a frequency higher than the natural frequency of the air column.\\
d. If the tuning fork vibrates at a frequency equal to the natural frequency of the air column.\\
27.

What is a closed-pipe resonator?\\
a. A pipe or cylindrical air column closed at both ends\\
b. A pipe with an antinode at the closed end\\
c. A pipe with a node at the open end\\
d. A pipe or cylindrical air column closed at one end\\
28.

Give two examples of open-pipe resonators.\\
a. piano, violin\\
b. drum, tabla\\
c. rlectric guitar, acoustic guitar\\
d. flute, oboe

\section*{Key Terms}
amplitude the amount that matter is disrupted during a sound wave, as measured by the difference in height between the crests and troughs of the sound wave.\\
beat a phenomenon produced by the superposition of two waves with slightly different frequencies but the same amplitude\\
beat frequency the frequency of the amplitude fluctuations of a wave\\
damping the reduction in amplitude over time as the energy of an oscillation dissipates\\
decibel a unit used to describe sound intensity levels\\
Doppler effect an alteration in the observed frequency of a sound due to relative motion between the source and the observer\\
fundamental the lowest-frequency resonance\\
harmonics the term used to refer to the fundamental and its overtones\\
hearing the perception of sound\\
loudness the perception of sound intensity\\
natural frequency the frequency at which a system would oscillate if there were no driving and no damping forces\\
overtones all resonant frequencies higher than the fundamental\\
pitch the perception of the frequency of a sound\\
rarefaction a low-pressure region in a sound wave\\
resonance the phenomenon of driving a system with a frequency equal to the system's natural frequency\\
resonate to drive a system at its natural frequency\\
sonic boom a constructive interference of sound created by an object moving faster than sound\\
sound a disturbance of matter that is transmitted from its source outward by longitudinal waves\\
sound intensity the power per unit area carried by a sound wave\\
sound intensity level the level of sound relative to a fixed standard related to human hearing

\section*{Key Equations}
14.1 Speed of Sound, Frequency, and Wavelength\\
14.2 Sound Intensity and Sound Level

\subsection*{14.3 Doppler Effect and Sonic Booms}
\subsection*{14.4 Sound Interference and Resonance}
\section*{Section Summary}
\subsection*{14.1 Speed of Sound, Frequency, and Wavelength}
\begin{itemize}
  \item Sound is one type of wave.
  \item Sound is a disturbance of matter that is transmitted from its source outward in the form of longitudinal waves.
  \item The relationship of the speed of sound \(v\), its frequency \(f\), and its wavelength \(\lambda\) is given by \(v=f \lambda\), which is the same relationship given for all waves.
  \item The speed of sound depends upon the medium through which the sound wave is travelling.
  \item In a given medium at a specific temperature (or density), the speed of sound \(v\) is the same for all frequencies and wavelengths.
\end{itemize}

\subsection*{14.2 Sound Intensity and Sound Level}
\begin{itemize}
  \item The intensity of a sound is proportional to its amplitude squared.
  \item The energy of a sound wave is also proportional to its amplitude squared.
  \item Sound intensity level in decibels (dB) is more relevant for how humans perceive sounds than sound intensity (in \(\mathrm{W} / \mathrm{m}^{2}\) ), even though sound intensity is the SI unit.
  \item Sound intensity level is not the same as sound intensity-it tells you the level of the sound relative to a reference intensity rather than the actual intensity.
  \item Hearing is the perception of sound and involves that transformation of sound waves into vibrations of parts within the ear. These vibrations are then transformed into neural signals that are interpreted by the brain.
  \item People create sounds by pushing air up through their lungs and through elastic folds in the throat called vocal cords.
\end{itemize}

\subsection*{14.3 Doppler Effect and Sonic Booms}
\begin{itemize}
  \item The Doppler effect is a shift in the observed frequency of a sound due to motion of either the source or the observer.
  \item The observed frequency is greater than the actual source's frequency when the source and the observer are moving closer together, either by the source moving toward the observer or the observer moving toward the source.
  \item A sonic boom is constructive interference of sound created by an object moving faster than sound.
\end{itemize}

\subsection*{14.4 Sound Interference and Resonance}
\begin{itemize}
  \item A system's natural frequency is the frequency at which the system will oscillate if not affected by driving or damping forces.
  \item A periodic force driving a harmonic oscillator at its natural frequency produces resonance. The system is said to resonate.
  \item Beats occur when waves of slightly different frequencies are superimposed.
  \item In air columns, the lowest-frequency resonance is called the fundamental, whereas all higher resonant frequencies are called overtones. Collectively, they are called harmonics.
  \item The resonant frequencies of a tube closed at one end are \(f_{n}=n \frac{v}{4 L}, n= 1,3,5 \ldots\), where \(f_{1}\) is the fundamental and L is the length of the tube.
  \item The resonant frequencies of a tube open at both ends are \(f_{n}=n \frac{v}{2 L}, n= 1,2,3 \ldots\)
\end{itemize}

\end{document}