\documentclass[10pt]{article}
\usepackage[utf8]{inputenc}
\usepackage[T1]{fontenc}
\usepackage{graphicx}
\usepackage[export]{adjustbox}
\graphicspath{ {./images/} }
\usepackage{caption}
\usepackage{amsmath}
\usepackage{amsfonts}
\usepackage{amssymb}
\usepackage[version=4]{mhchem}
\usepackage{stmaryrd}

\begin{document}
\captionsetup{singlelinecheck=false}
\begin{figure}[h]
\begin{center}
  \includegraphics[max width=\textwidth]{9044bee6-b601-41f3-ac5a-eb9d55d1bb9c-01}
\captionsetup{labelformat=empty}
\caption{Figure 11.1 The welder's gloves and helmet protect the welder from the electric arc, which transfers enough thermal energy to melt the rod, spray sparks, and emit high-energy electromagnetic radiation that can burn the retina of an unprotected eye. The thermal energy can be felt on exposed skin a few meters away, and its light can be seen for kilometers (Kevin S. O'Brien, U.S. Navy)}
\end{center}
\end{figure}

\section*{Chapter Outline}
11.1 Temperature and Thermal Energy\\
11.2 Heat, Specific Heat, and Heat Transfer\\
11.3 Phase Change and Latent Heat

\section*{Introduction}
\section*{Teacher Support}
Teacher Support Review the concepts of energy, internal energy, and mass. Ask students how the welder feels the heat from the welding equipment. How does heat transfer thermal energy from the equipment to his hand? Is it the same manner in which heat transfers thermal energy to the material being welded? Explain that in this chapter you shall be learning about thermal energy, temperature, different ways heat affects matter, and different modes of heat transfer.

Heat is something familiar to all of us. We feel the warmth of the summer sun, the hot vapor rising up out of a cup of hot cocoa, and the cooling effect of our sweat. When we feel warmth, it means that heat is transferring energy to our bodies; when we feel cold, that means heat is transferring energy away from our bodies. Heat transfer is the movement of thermal energy from one place or material to another, and is caused by temperature differences. For example,\\
much of our weather is caused by Earth evening out the temperature across the planet through wind and violent storms, which are driven by heat transferring energy away from the equator towards the cold poles. In this chapter, we'll explore the precise meaning of heat, how it relates to temperature as well as to other forms of energy, and its connection to work.

\section*{Teacher Support}
Teacher Support Before the start of this chapter, it is useful to review the following concept:

\begin{itemize}
  \item Units of joules and calories and the interconversion of these
\end{itemize}

\subsection*{11.1 Temperature and Thermal Energy}
\section*{Section Learning Objectives}
By the end of this section, you will be able to do the following:

\begin{itemize}
  \item Explain that temperature is a measure of internal kinetic energy
  \item Interconvert temperatures between Celsius, Kelvin, and Fahrenheit scales
\end{itemize}

\section*{Teacher Support}
Teacher Support The Learning Objectives in this section will help your students master the following standards:

\begin{itemize}
  \item (6) Science concepts. The student knows that changes occur within a physical system and applies the laws of conservation of energy and momentum. The student is expected to:
  \item (E) describe how the macroscopic properties of a thermodynamic system such as temperature, specific heat, and pressure are related to the molecular level of matter, including kinetic or potential energy of atoms.
\end{itemize}

In addition, the High School Physics Laboratory Manual addresses content in this section in the lab titled: Thermodynamics, as well as the following standards:

\begin{itemize}
  \item (6) Science concepts. The student knows that changes occur within a physical system and applies the laws of conservation of energy and momentum. The student is expected to:
  \item (E) describe how the macroscopic properties of a thermodynamic system such as temperature, specific heat, and pressure are related to the molecular level of matter, including kinetic or potential energy of atoms;
  \item (G) analyze and explain everyday examples that illustrate the laws of thermodynamics, including the law of conservation of energy and the law of entropy.
\end{itemize}

\section*{Section Key Terms}
\section*{Teacher Support}
Teacher Support [BL] [OL][AL] Check prior knowledge of terms such as heat, temperature, and temperature scales.\\[0pt]
[OL] Ask students what contains more heat-a bucketful of warm water or a spoonful of boiling water. From this ask them to define heat. Ask them which one has a higher temperature.

\section*{Misconception Alert}
Dispel any notions that heat content is solely dependent on temperature.

\section*{Temperature}
What is temperature? It's one of those concepts so ingrained in our everyday lives that, although we know what it means intuitively, it can be hard to define. It is tempting to say that temperature measures heat, but this is not strictly true. Heat is the transfer of energy due to a temperature difference. Temperature is defined in terms of the instrument we use to tell us how hot or cold an object is, based on a mechanism and scale invented by people. Temperature is literally defined as what we measure on a thermometer.

Heat is often confused with temperature. For example, we may say that the heat was unbearable, when we actually mean that the temperature was high. This is because we are sensitive to the flow of energy by heat, rather than the temperature. Since heat, like work, transfers energy, it has the SI unit of joule (J).

Atoms and molecules are constantly in motion, bouncing off one another in random directions. Recall that kinetic energy is the energy of motion, and that it increases in proportion to velocity squared. Without going into mathematical detail, we can say that thermal energy-the energy associated with heat-is the average kinetic energy of the particles (molecules or atoms) in a substance. Faster moving molecules have greater kinetic energies, and so the substance has greater thermal energy, and thus a higher temperature. The total internal energy of a system is the sum of the kinetic and potential energies of its atoms and molecules. Thermal energy is one of the subcategories of internal energy, as is chemical energy.

\section*{Teacher Support}
\section*{Teacher Support}
\section*{Teacher Demonstration}
You can show that temperature is related to the kinetic energy of molecules by a simple demonstration. Take an eraser and rub it vigorously against any surface. Then feel it against your skin. Is it hot?

To measure temperature, some scale must be used as a standard of measurement. The three most commonly used temperature scales are the Fahrenheit, Celsius, and Kelvin scales. Both the Fahrenheit scale and Celsius scale are relative\\
temperature scales, meaning that they are made around a reference point. For example, the Celsius scale uses the freezing point of water as its reference point; all measurements are either lower than the freezing point of water by a given number of degrees (and have a negative sign), or higher than the freezing point of water by a given number of degrees (and have a positive sign). The boiling point of water is \(100^{\circ} \mathrm{C}\) for the Celsius scale, and its unit is the degree Celsius \(\left({ }^{\circ} \mathrm{C}\right)\).

On the Fahrenheit scale, the freezing point of water is at \(32{ }^{\circ} \mathrm{F}\), and the boiling point is at \(212^{\circ} \mathrm{F}\). The unit of temperature on this scale is the degree Fahrenheit ( \({ }^{\circ} \mathrm{F}\) ). Note that the difference in degrees between the freezing and boiling points is greater for the Fahrenheit scale than for the Celsius scale. Therefore, a temperature difference of one degree Celsius is greater than a temperature difference of one degree Fahrenheit. Since 100 Celsius degrees span the same range as 180 Fahrenheit degrees, one degree on the Celsius scale is 1.8 times larger than one degree on the Fahrenheit scale (because \(\frac{180}{100}=\frac{9}{5}=1.8\) ). This relationship can be used to convert between temperatures in Fahrenheit and Celsius (see Figure 11.2).

\begin{figure}[h]
\begin{center}
  \includegraphics[max width=\textwidth]{9044bee6-b601-41f3-ac5a-eb9d55d1bb9c-05}
\captionsetup{labelformat=empty}
\caption{Figure 11.2 Relationships between the Fahrenheit, Celsius, and Kelvin temperature scales, rounded to the nearest degree. The relative sizes of the scales are also shown.}
\end{center}
\end{figure}

The Kelvin scale is the temperature scale that is commonly used in science because it is an absolute temperature scale. This means that the theoretically lowest-possible temperature is assigned the value of zero. Zero degrees on the Kelvin scale is known as absolute zero; it is theoretically the point at which there is no molecular motion to produce thermal energy. On the original Kelvin scale first created by Lord Kelvin, all temperatures have positive values, making\\
it useful for scientific work. The official temperature unit on this scale is the kelvin, which is abbreviated as K . The freezing point of water is 273.15 K , and the boiling point of water is 373.15 K .

\section*{Teacher Support}
Teacher Support [BL][OL][AL] Ask students in which case each scale would be most convenient to use.

Although absolute zero is possible in theory, it cannot be reached in practice. The lowest temperature ever created and measured during a laboratory experiment was \(1.0 \times 10^{-10} \mathrm{~K}\), at Helsinki University of Technology in Finland. In comparison, the coldest recorded temperature for a place on Earth's surface was \(183 \mathrm{~K}\left(-89^{\circ} \mathrm{C}\right)\), at Vostok, Antarctica, and the coldest known place (outside the lab) in the universe is the Boomerang Nebula, with a temperature of 1 K . Luckily, most of us humans will never have to experience such extremes.

\section*{Teacher Support}
Teacher Support [AL]Ask why absolute zero has never been recorded. Discuss if atoms and molecules can ever be completely motionless.

The average normal body temperature is \(98.6^{\circ} \mathrm{F}\) ( \(37.0^{\circ} \mathrm{C}\) ), but people have been known to survive with body temperatures ranging from \(75^{\circ} \mathrm{F}\) to \(111^{\circ} \mathrm{F} \left(24^{\circ} \mathrm{C}\right.\) to \(\left.44^{\circ} \mathrm{C}\right)\).

\section*{Watch Physics}
Comparing Celsius and Fahrenheit Temperature Scales This video shows how the Fahrenheit and Celsius temperature scales compare to one another.

Click to view content\\
Watch Physics: Comparing Celsius and Fahrenheit Temperature Scales. This video makes a comparison between the Celsius and Fahrenheit temperature scales.

Click to view content\\
Even without the number labels on the thermometer, you could tell which side is marked Fahrenheit and which is Celsius by how the degree marks are spaced. Why?\\
a. The separation between two consecutive divisions on the Fahrenheit scale is greater than a similar separation on the Celsius scale, because each degree Fahrenheit is equal to 1.8 degrees Celsius.\\
b. The separation between two consecutive divisions on the Fahrenheit scale is smaller than the similar separation on the Celsius scale, because each degree Celsius is equal to 1.8 degrees Fahrenheit.\\
c. The separation between two consecutive divisions on the Fahrenheit scale is greater than a similar separation on the Celsius scale, because each degree Fahrenheit is equal to 3.6 degrees Celsius.\\
d. The separation between two consecutive divisions on the Fahrenheit scale is smaller than a similar separation on the Celsius scale, because each degree Celsius is equal to 3.6 degrees Fahrenheit.

\section*{Teacher Support}
Teacher Support Students can use the process described in this video as a means of comparing different temperature scales. Point out to them that all they need to know are the temperatures on each scale of a single property, such as the boiling and freezing points of a liquid, whether it be water, ethanol, or tetrachloromethane.

\section*{Converting Between Celsius, Kelvin, and Fahrenheit Scales}
While the Fahrenheit scale is still the most commonly used scale in the United States, the majority of the world uses Celsius, and scientists prefer Kelvin. It's often necessary to convert between these scales. For instance, if the TV meteorologist gave the local weather report in kelvins, there would likely be some confused viewers! Table 11.1 gives the equations for conversion between the three temperature scales.

\begin{table}[h]
\begin{center}
\begin{tabular}{|l|l|}
\hline
To Convert From... & Use This Equation \\
\hline
Celsius to Fahrenheit & \(T_{{ }^{\circ} \mathrm{F}}=\frac{9}{5} T^{\circ}{ }_{\mathrm{C}}+32\) \\
\hline
Fahrenheit to Celsius & \(T_{{ }_{\circ} \mathrm{C}}=\frac{5}{9}\left(T_{{ }_{\circ} \mathrm{F}}-32\right)\) \\
\hline
Celsius to Kelvin & \(T_{\mathrm{K}}=T_{{ }^{\circ} \mathrm{C}}+273.15\) \\
\hline
Kelvin to Celsius & \(T_{\mathrm{O}}{ }_{\mathrm{C}}=T_{\mathrm{K}}-273.15\) \\
\hline
Fahrenheit to Kelvin & \(T_{\mathrm{K}}=\frac{5}{9}\left(T_{{ }_{\mathrm{F}}}-32\right)+273.15\) \\
\hline
Kelvin to Fahrenheit & \(T_{{ }_{\mathrm{F}}}=\frac{9}{5}\left(T_{\mathrm{K}}-273.15\right)+32\) \\
\hline
\end{tabular}
\captionsetup{labelformat=empty}
\caption{Table 11.1 Temperature Conversions}
\end{center}
\end{table}

\section*{Teacher Support}
Teacher Support [BL][OL][AL] Ask students which is more - a difference of \(5{ }^{\circ} \mathrm{F}\) or a difference of \(5{ }^{\circ} \mathrm{C}\). Now ask them the same for \(5{ }^{\circ} \mathrm{C}\) and \(5{ }^{\circ} \mathrm{F}\). A difference in temperature for Kelvin and that for Celsius are the same. The same is not true for Celsius and Fahrenheit.

\section*{Worked Example}
Converting between Temperature Scales: Room Temperature Room temperature is generally defined to be \(25^{\circ} \mathrm{C}\). (a) What is room temperature in \({ }^{\circ} \mathrm{F}\) ? (b) What is it in K ?

\section*{Teacher Support}
Teacher Support In an alcohol-in-glass thermometer, alcohol molecules absorb energy by heat, and as the intermolecular distances increase, the bulk alcohol expands. Ask students the temperature range for which this thermometer shows an accurate reading. Why is this the case? Ask them if it is possible to design a thermometer with any other substance. Why or why not?

\section*{Strategy}
To answer these questions, all we need to do is choose the correct conversion equations and plug in the known values.

Solution for (a)

\begin{enumerate}
  \item Choose the right equation. To convert from \({ }^{\circ} \mathrm{C}\) to \({ }^{\circ} \mathrm{F}\), use the equation
\end{enumerate}

\begin{itemize}
  \item \(T_{{ }_{\mathrm{F}}}=\frac{9}{5} T_{{ }_{\mathrm{C}}}+32\).\\
11.1
\end{itemize}

\begin{enumerate}
  \setcounter{enumi}{1}
  \item Plug the known value into the equation and solve.
\end{enumerate}

\begin{itemize}
  \item \(T_{{ }^{\circ} \mathrm{F}}=\frac{9}{5} 25{ }^{\circ} \mathrm{C}+32=77{ }^{\circ} \mathrm{F}\)\\
11.2
\end{itemize}

Solution for (b)

\begin{enumerate}
  \item Choose the right equation. To convert from \({ }^{\circ} \mathrm{C}\) to K , use the equation
\end{enumerate}

\begin{itemize}
  \item \(T_{\mathrm{K}}=T_{{ }^{\circ} \mathrm{C}}+273.15\).\\
11.3
\end{itemize}

\begin{enumerate}
  \setcounter{enumi}{1}
  \item Plug the known value into the equation and solve.
\end{enumerate}

\begin{itemize}
  \item \(T_{\mathrm{K}}=25^{\circ} \mathrm{C}+273.15=298 \mathrm{~K}\)\\
11.4
\end{itemize}

Discussion\\
Living in the United States, you are likely to have more of a sense of what the temperature feels like if it's described as \(77^{\circ} \mathrm{F}\) than as \(25^{\circ} \mathrm{C}\) (or 298 K , for that matter).

\section*{Worked Example}
Converting Between Temperature Scales: The Reaumur Scale The Reaumur scale is a temperature scale that was used widely in Europe in the \(18^{\text {th }}\) and \(19^{\text {th }}\) centuries. On the Reaumur temperature scale, the freezing point of water is \(0^{\circ} \mathrm{R}\) and the boiling temperature is \(80^{\circ} \mathrm{R}\). If "room temperature" is \(25{ }^{\circ} \mathrm{C}\) on the Celsius scale, what is it on the Reaumur scale?

\section*{Strategy}
To answer this question, we must compare the Reaumur scale to the Celsius scale. The difference between the freezing point and boiling point of water on the Reaumur scale is \(80^{\circ} \mathrm{R}\). On the Celsius scale, it is \(100^{\circ} \mathrm{C}\). Therefore, \(100^{\circ} \mathrm{C}=80^{\circ} \mathrm{R}\). Both scales start at \(0^{\circ}\) for freezing, so we can create a simple formula to convert between temperatures on the two scales.

Solution

\begin{enumerate}
  \item Derive a formula to convert from one scale to the other.
\end{enumerate}

\begin{itemize}
  \item \(T_{\circ}{ }_{\mathrm{R}}=\frac{0.80^{\circ} \mathrm{R}}{{ }^{\circ} \mathrm{C}} \times T_{\circ}{ }^{\circ} \mathrm{C}\)\\
11.5
\end{itemize}

\begin{enumerate}
  \setcounter{enumi}{1}
  \item Plug the known value into the equation and solve.
\end{enumerate}

\begin{itemize}
  \item \(T^{\circ} \mathrm{R}=\frac{0.80^{\circ} \mathrm{R}}{{ }^{\circ} \mathrm{C}} \times 25{ }^{\circ} \mathrm{C}=20{ }^{\circ} \mathrm{R}\)\\
11.6
\end{itemize}

Discussion\\
As this example shows, relative temperature scales are somewhat arbitrary. If you wanted, you could create your own temperature scale!

\section*{Practice Problems}
1.

What is \(12.0^{\circ} \mathrm{C}\) in kelvins?\\
a. 112.0 K\\
b. 273.2 K\\
c. 12.0 K\\
d. 285.2 K\\
2.

What is \(32.0^{\circ} \mathrm{C}\) in degrees Fahrenheit?\\
a. \(57.6{ }^{\circ} \mathrm{F}\)\\
b. \(25.6{ }^{\circ} \mathrm{F}\)\\
c. \(305.2{ }^{\circ} \mathrm{F}\)\\
d. \(89.6{ }^{\circ} \mathrm{F}\)

\section*{Tips For Success}
Sometimes it is not so easy to guess the temperature of the air accurately. Why is this? Factors such as humidity and wind speed affect how hot or cold we feel. Wind removes thermal energy from our bodies at a faster rate than usual, making us feel colder than we otherwise would; on a cold day, you may have heard the TV weather person refer to the wind chill.

On humid summer days, people tend to feel hotter because sweat doesn't evaporate from the skin as efficiently as it does on dry days, when the evaporation of sweat cools us off.

\section*{Teacher Support}
Teacher Support Ask students how wind chill works. Will it work on any surface or just the human body? The rate of heat loss for any object depends on the temperature difference between the object and its surroundings. When heat transfers energy away from the object, it warms a layer of air around it. Wind disrupts this layer, replacing it with cooler air. This, in turn, increases the rate of heat loss.

\section*{Misconception Alert}
Wind chill can only increase the rate of cooling. It can never cool an object to a temperature below the ambient temperature.

\section*{Check Your Understanding}
\section*{Teacher Support}
Teacher Support Use these questions to assess student achievement of the section's learning objectives. If students are struggling with a specific objective, these questions will help identify which and direct students to the relevant content.\\
3.

What is thermal energy?\\
a. The thermal energy is the average potential energy of the particles in a system.\\
b. The thermal energy is the total sum of the potential energies of the particles in a system.\\
c. The thermal energy is the average kinetic energy of the particles due to the interaction among the particles in a system.\\
d. The thermal energy is the average kinetic energy of the particles in a system.\\
4.

What is used to measure temperature?\\
a. a galvanometer\\
b. a manometer\\
c. a thermometer\\
d. a voltmeter

\subsection*{11.2 Heat, Speci c Heat, and Heat Transfer}
\section*{Section Learning Objectives}
By the end of this section, you will be able to do the following:

\begin{itemize}
  \item Explain heat, heat capacity, and specific heat
  \item Distinguish between conduction, convection, and radiation
  \item Solve problems involving specific heat and heat transfer
\end{itemize}

\section*{Teacher Support}
Teacher Support The learning objectives in this section will help your students master the following standards:

\begin{itemize}
  \item (6) Science concepts. The student knows that changes occur within a physical system and applies the laws of conservation of energy and momentum.\\
The student is expected to:
  \item (F) contrast and give examples of different processes of thermal energy transfer, including conduction, convection, and radiation.
\end{itemize}

\section*{Section Key Terms}
\section*{Teacher Support}
Teacher Support [BL][OL][AL] Review concepts of heat, temperature, and mass.\\[0pt]
[AL] Check prior knowledge of conduction and convection.

\section*{Heat Transfer, Specific Heat, and Heat Capacity}
We learned in the previous section that temperature is proportional to the average kinetic energy of atoms and molecules in a substance, and that the average internal kinetic energy of a substance is higher when the substance's temperature is higher.

If two objects at different temperatures are brought in contact with each other, energy is transferred from the hotter object (that is, the object with the greater temperature) to the colder (lower temperature) object, until both objects are at the same temperature. There is no net heat transfer once the temperatures are equal because the amount of heat transferred from one object to the other is the same as the amount of heat returned. One of the major effects of heat transfer is temperature change: Heating increases the temperature while cooling decreases it. Experiments show that the heat transferred to or from a substance depends\\
on three factors - the change in the substance's temperature, the mass of the substance, and certain physical properties related to the phase of the substance.

The equation for heat transfer \(Q\) is\\
\(Q=m c \Delta T\),\\
11.7\\
where \(m\) is the mass of the substance and \(\Delta T\) is the change in its temperature, in units of Celsius or Kelvin. The symbol \(c\) stands for specific heat, and depends on the material and phase. The specific heat is the amount of heat necessary to change the temperature of 1.00 kg of mass by \(1.00^{\circ} \mathrm{C}\). The specific heat \(c\) is a property of the substance; its SI unit is \(\mathrm{J} /(\mathrm{kg} \cdot \mathrm{K})\) or \(\mathrm{J} /\left(\mathrm{kg} \cdot{ }^{\circ} \mathrm{C}\right)\). The temperature change ( \(\Delta T\) ) is the same in units of kelvins and degrees Celsius (but not degrees Fahrenheit). Specific heat is closely related to the concept of heat capacity. Heat capacity is the amount of heat necessary to change the temperature of a substance by \(1.00^{\circ} \mathrm{C}\). In equation form, heat capacity \(C\) is \(C=m c\), where \(m\) is mass and \(c\) is specific heat. Note that heat capacity is the same as specific heat, but without any dependence on mass. Consequently, two objects made up of the same material but with different masses will have different heat capacities. This is because the heat capacity is a property of an object, but specific heat is a property of any object made of the same material.

Values of specific heat must be looked up in tables, because there is no simple way to calculate them. Table 11.2 gives the values of specific heat for a few substances as a handy reference. We see from this table that the specific heat of water is five times that of glass, which means that it takes five times as much heat to raise the temperature of 1 kg of water than to raise the temperature of 1 kg of glass by the same number of degrees.

\section*{Teacher Support}
Teacher Support [BL][OL][AL]Explain that this formula only works when there is no change in phase of the substance. The transfer of thermal energy, heat, and phase change will be covered later in the chapter.

\section*{Misconception Alert}
The units of specific heat are \(\mathrm{J} /\left(\mathrm{kg} \cdot{ }^{\circ} \mathrm{C}\right)\) and \(\mathrm{J} /(\mathrm{kg} \cdot \mathrm{K})\). However, degrees Celsius and Kelvins are not always interchangeable. The formula for specific heat uses a difference in temperature and not absolute temperature. This is the reason that degrees Celsius may be used in place of Kelvins.

Table 11.2 Specific Heats of Various Substances.

\section*{Snap Lab}
Temperature Change of Land and Water What heats faster, land or water? You will answer this question by taking measurements to study differences in specific heat capacity.

\begin{itemize}
  \item Open flame-Tie back all loose hair and clothing before igniting an open flame. Follow all of your teacher's instructions on how to ignite the flame. Never leave an open flame unattended. Know the location of fire safety equipment in the laboratory.
  \item Sand or soil
  \item Water
  \item Oven or heat lamp
  \item Two small jars
  \item Two thermometers
\end{itemize}

Instructions\\
Procedure

\begin{enumerate}
  \item Place equal masses of dry sand (or soil) and water at the same temperature into two small jars. (The average density of soil or sand is about 1.6 times that of water, so you can get equal masses by using 50 percent more water by volume.)
  \item Heat both substances (using an oven or a heat lamp) for the same amount of time.
  \item Record the final temperatures of the two masses.
  \item Now bring both jars to the same temperature by heating for a longer period of time.
  \item Remove the jars from the heat source and measure their temperature every 5 minutes for about 30 minutes.
\end{enumerate}

Soil has an approximate specific heat of \(800 \mathrm{~J} / \mathrm{kg}{ }^{\circ} \mathrm{C}\). A farmer monitors both the soil temperature of his field and the temperature of a nearby pond as winter sets in. Will the field or the pond reach \(0{ }^{\circ} \mathrm{C}\) first and why?\\
a. The pond will reach \(0^{\circ} \mathrm{C}\) first because of water's greater specific heat.\\
b. The field will reach \(0{ }^{\circ} \mathrm{C}\) first because of soil's lower specific heat.\\
c. They will reach \(0^{\circ} \mathrm{C}\) at the same time because they are exposed to the same weather.\\
d. The water will take longer to heat as well as to cool. This tells us that the specific heat of water is greater than that of land.

\section*{Conduction, Convection, and Radiation}
Whenever there is a temperature difference, heat transfer occurs. Heat transfer may happen rapidly, such as through a cooking pan, or slowly, such as through the walls of an insulated cooler.

There are three different heat transfer methods: conduction, convection, and radiation. At times, all three may happen simultaneously. See Figure 11.3.

\begin{figure}[h]
\begin{center}
  \includegraphics[max width=\textwidth]{9044bee6-b601-41f3-ac5a-eb9d55d1bb9c-15}
\captionsetup{labelformat=empty}
\caption{Figure 11.3 In a fireplace, heat transfer occurs by all three methods: conduction, convection, and radiation. Radiation is responsible for most of the heat}
\end{center}
\end{figure}

transferred into the room. Heat transfer also occurs through conduction into the room, but at a much slower rate. Heat transfer by convection also occurs through cold air entering the room around windows and hot air leaving the room by rising up the chimney.

Conduction is heat transfer through direct physical contact. Heat transferred between the electric burner of a stove and the bottom of a pan is transferred by conduction. Sometimes, we try to control the conduction of heat to make ourselves more comfortable. Since the rate of heat transfer is different for different materials, we choose fabrics, such as a thick wool sweater, that slow down the transfer of heat away from our bodies in winter.

As you walk barefoot across the living room carpet, your feet feel relatively comfortable...until you step onto the kitchen's tile floor. Since the carpet and tile floor are both at the same temperature, why does one feel colder than the other? This is explained by different rates of heat transfer: The tile material removes heat from your skin at a greater rate than the carpeting, which makes it feel colder.

\section*{Teacher Support}
Teacher Support \([\mathrm{BL}][\mathrm{OL}][\mathrm{AL}]\) Ask students what the current temperature in the classroom is. Ask them if all the objects in the room are at the same temperature. Once this is established, ask them to place their hand on their desk or on a metal object. Does it feel colder? Why? If their desk is Formica laminate, then it will feel cool to their hand because the laminate is a good conductor of heat and draws heat from their hand creating a sensation of "cold" due to heat leaving the body.

Some materials simply conduct thermal energy faster than others. In general, metals (like copper, aluminum, gold, and silver) are good heat conductors, whereas materials like wood, plastic, and rubber are poor heat conductors.

Figure 11.4 shows particles (either atoms or molecules) in two bodies at different temperatures. The (average) kinetic energy of a particle in the hot body is higher than in the colder body. If two particles collide, energy transfers from the particle with greater kinetic energy to the particle with less kinetic energy. When two bodies are in contact, many particle collisions occur, resulting in a net flux of heat from the higher-temperature body to the lower-temperature body. The heat flux depends on the temperature difference \(\Delta T=T_{\text {hot }}-T_{\text {cold }}\) - Therefore, you will get a more severe burn from boiling water than from hot tap water.

\begin{figure}[h]
\begin{center}
  \includegraphics[max width=\textwidth]{9044bee6-b601-41f3-ac5a-eb9d55d1bb9c-17}
\captionsetup{labelformat=empty}
\caption{Figure 11.4 The particles in two bodies at different temperatures have different average kinetic energies. Collisions occurring at the contact surface tend to transfer energy from high-temperature regions to low-temperature regions. In this illustration, a particle in the lower-temperature region (right side) has low kinetic energy before collision, but its kinetic energy increases after colliding with the contact surface. In contrast, a particle in the higher-temperature region (left side) has more kinetic energy before collision, but its energy decreases after colliding with the contact surface.}
\end{center}
\end{figure}

Convection is heat transfer by the movement of a fluid. This type of heat transfer happens, for example, in a pot boiling on the stove, or in thunderstorms, where hot air rises up to the base of the clouds.

\section*{Tips For Success}
In everyday language, the term fluid is usually taken to mean liquid. For example, when you are sick and the doctor tells you to "push fluids," that only means to drink more beverages - not to breath more air. However, in physics, fluid means a liquid or a gas. Fluids move differently than solid material, and even have their own branch of physics, known as fluid dynamics, that studies how they move.

As the temperature of fluids increase, they expand and become less dense. For example, Figure 11.4 could represent the wall of a balloon with different temperature gases inside the balloon than outside in the environment. The hotter and thus faster moving gas particles inside the balloon strike the surface with more force than the cooler air outside, causing the balloon to expand. This decrease in density relative to its environment creates buoyancy (the tendency to rise). Convection is driven by buoyancy-hot air rises because it is less dense than the surrounding air.

Sometimes, we control the temperature of our homes or ourselves by controlling\\
air movement. Sealing leaks around doors with weather stripping keeps out the cold wind in winter. The house in Figure 11.5 and the pot of water on the stove in Figure 11.6 are both examples of convection and buoyancy by human design. Ocean currents and large-scale atmospheric circulation transfer energy from one part of the globe to another, and are examples of natural convection.

\begin{figure}[h]
\begin{center}
  \includegraphics[max width=\textwidth]{9044bee6-b601-41f3-ac5a-eb9d55d1bb9c-18(1)}
\captionsetup{labelformat=empty}
\caption{Figure 11.5 Air heated by the so-called gravity furnace expands and rises, forming a convective loop that transfers energy to other parts of the room. As the air is cooled at the ceiling and outside walls, it contracts, eventually becoming denser than room air and sinking to the floor. A properly designed heating system like this one, which uses natural convection, can be quite efficient in uniformly heating a home.}
\end{center}
\end{figure}

\begin{figure}[h]
\begin{center}
  \includegraphics[max width=\textwidth]{9044bee6-b601-41f3-ac5a-eb9d55d1bb9c-18}
\captionsetup{labelformat=empty}
\caption{Figure 11.6 Convection plays an important role in heat transfer inside this pot of water. Once conducted to the inside fluid, heat transfer to other parts of the pot is mostly by convection. The hotter water expands, decreases in density, and rises to transfer heat to other regions of the water, while colder water sinks to the bottom. This process repeats as long as there is water in the pot.}
\end{center}
\end{figure}

Radiation is a form of heat transfer that occurs when electromagnetic radiation is emitted or absorbed. Electromagnetic radiation includes radio waves, microwaves, infrared radiation, visible light, ultraviolet radiation, X-rays, and\\
gamma rays, all of which have different wavelengths and amounts of energy (shorter wavelengths have higher frequency and more energy).

\section*{Teacher Support}
Teacher Support [BL][OL] Electromagnetic waves are also often referred to as EM waves. We perceive EM waves of different frequencies differently. Just as we are able to see certain frequencies as visible light, we perceive certain others as heat.

You can feel the heat transfer from a fire and from the sun. Similarly, you can sometimes tell that the oven is hot without touching its door or looking insideit may just warm you as you walk by. Another example is thermal radiation from the human body; people are constantly emitting infrared radiation, which is not visible to the human eye, but is felt as heat.

Radiation is the only method of heat transfer where no medium is required, meaning that the heat doesn't need to come into direct contact with or be transported by any matter. The space between Earth and the sun is largely empty, without any possibility of heat transfer by convection or conduction. Instead, heat is transferred by radiation, and Earth is warmed as it absorbs electromagnetic radiation emitted by the sun.\\
\includegraphics[max width=\textwidth, center]{9044bee6-b601-41f3-ac5a-eb9d55d1bb9c-19}

Figure 11.7 Most of the heat transfer from this fire to the observers is through infrared radiation. The visible light transfers relatively little thermal energy. Since skin is very sensitive to infrared radiation, you can sense the presence of a fire without looking at it directly. (Daniel X. O'Neil)

All objects absorb and emit electromagnetic radiation (see Figure 11.7). The rate of heat transfer by radiation depends mainly on the color of the object. Black is the most effective absorber and radiator, and white is the least effective. People living in hot climates generally avoid wearing black clothing, for instance. Similarly, black asphalt in a parking lot will be hotter than adjacent patches of grass on a summer day, because black absorbs better than green. The reverse is also true-black radiates better than green. On a clear summer night, the black asphalt will be colder than the green patch of grass, because black radiates energy faster than green. In contrast, white is a poor absorber and also a poor radiator. A white object reflects nearly all radiation, like a mirror.

\section*{Teacher Support}
Teacher Support Ask students to give examples of conduction, convection, and radiation.

\section*{Virtual Physics}
Energy Forms and Changes Click to view content\\
In this animation, you will explore heat transfer with different materials. Experiment with heating and cooling the iron, brick, and water. This is done by dragging and dropping the object onto the pedestal and then holding the lever either to Heat or Cool. Drag a thermometer beside each object to measure its temperature - you can watch how quickly it heats or cools in real time.

Now let's try transferring heat between objects. Heat the brick and then place it in the cool water. Now heat the brick again, but then place it on top of the iron. What do you notice?

Selecting the fast forward option lets you speed up the heat transfers, to save time.

Compare how quickly the different materials are heated or cooled. Based on these results, what material do you think has the greatest specific heat? Why? Which has the smallest specific heat? Can you think of a real-world situation where you would want to use an object with large specific heat?\\
a. Water will take the longest, and iron will take the shortest time to heat, as well as to cool. Objects with greater specific heat would be desirable for insulation. For instance, woolen clothes with large specific heat would prevent heat loss from the body.\\
b. Water will take the shortest, and iron will take the longest time to heat, as well as to cool. Objects with greater specific heat would be desirable for insulation. For instance, woolen clothes with large specific heat would prevent heat loss from the body.\\
c. Brick will take shortest and iron will take longest time to heat up as well as to cool down. Objects with greater specific heat would be desirable for insulation. For instance, woolen clothes with large specific heat would prevent heat loss from the body.\\
d. Water will take shortest and brick will take longest time to heat up as well as to cool down. Objects with greater specific heat would be desirable for insulation. For instance, woolen clothes with large specific heat would prevent heat loss from the body.

\section*{Teacher Support}
Teacher Support Have students consider the differences in the interactive exercise results if different materials were used. For example, ask them whether\\
the temperature change would be greater or smaller if the brick were replaced with a block of iron with the same mass as the brick. Ask students to consider identical masses of the metals aluminum, gold, and copper. After they have stated whether the temperature change is greater or less for each metal, have them refer to Table 11.2 and check whether their predictions were correct.

\section*{Solving Heat Transfer Problems}
\section*{Worked Example}
Calculating the Required Heat: Heating Water in an Aluminum Pan A 0.500 kg aluminum pan on a stove is used to heat 0.250 L of water from 20.0 \({ }^{\circ} \mathrm{C}\) to \(80.0^{\circ} \mathrm{C}\). (a) How much heat is required? What percentage of the heat is used to raise the temperature of (b) the pan and (c) the water?

\section*{Strategy}
The pan and the water are always at the same temperature. When you put the pan on the stove, the temperature of the water and the pan is increased by the same amount. We use the equation for heat transfer for the given temperature change and masses of water and aluminum. The specific heat values for water and aluminum are given in the previous table.

Solution to (a)\\
Because the water is in thermal contact with the aluminum, the pan and the water are at the same temperature.

\begin{enumerate}
  \item Calculate the temperature difference.
\end{enumerate}

\begin{itemize}
  \item \(\Delta T=T_{f}-T_{i}=60.0^{\circ} \mathrm{C}\)\\
11.8
\end{itemize}

\begin{enumerate}
  \setcounter{enumi}{1}
  \item Calculate the mass of water using the relationship between density, mass, and volume. Density is mass per unit volume, or \(\rho=\frac{m}{V}\). Rearranging this equation, solve for the mass of water.
\end{enumerate}

\begin{itemize}
  \item \(m_{w}=\rho \cdot V=1000 \mathrm{~kg} / \mathrm{m}^{3} \times\left(0.250 \mathrm{~L} \times \frac{0.001 \mathrm{~m}^{3}}{1 \mathrm{~L}}\right)=0.250 \mathrm{~kg}\)\\
11.9
\end{itemize}

\begin{enumerate}
  \setcounter{enumi}{2}
  \item Calculate the heat transferred to the water. Use the specific heat of water in the previous table.
\end{enumerate}

\begin{itemize}
  \item \(Q_{w}=m_{w} c_{w} \Delta T=(0.250 \mathrm{~kg})\left(4186 \mathrm{~J} / \mathrm{kg}^{\circ} \mathrm{C}\right)\left(60.0^{\circ} \mathrm{C}\right)=62.8 \mathrm{~kJ}\) 11.10
\end{itemize}

\begin{enumerate}
  \setcounter{enumi}{3}
  \item Calculate the heat transferred to the aluminum. Use the specific heat for aluminum in the previous table.
\end{enumerate}

\begin{itemize}
  \item \(Q_{A l}=m_{A l} c_{A l} \Delta T=(0.500 \mathrm{~kg})\left(900 \mathrm{~J} / \mathrm{kg}^{\circ} \mathrm{C}\right)\left(60.0^{\circ} \mathrm{C}\right)=27.0 \times 10^{3} \mathrm{~J}=27.0 \mathrm{~kJ}\)\\
11.11
\end{itemize}

\begin{enumerate}
  \setcounter{enumi}{4}
  \item Find the total transferred heat.
\end{enumerate}

\begin{itemize}
  \item \(Q_{\text {Total }}=Q_{w}+Q_{A l}=62.8 \mathrm{~kJ}+27.0 \mathrm{~kJ}=89.8 \mathrm{~kJ}\)\\
11.12
\end{itemize}

Solution to (b)\\
The percentage of heat going into heating the pan is\\
\(\frac{27.0 \mathrm{~kJ}}{89.8 \mathrm{~kJ}} \times 100 \%=30.1 \%\)\\
11.13

Solution to (c)\\
The percentage of heat going into heating the water is\\
\(\frac{62.8 \mathrm{~kJ}}{89.8 \mathrm{~kJ}} \times 100 \%=69.9 \%\)\\
11.14

Discussion\\
In this example, most of the total heat transferred is used to heat the water, even though the pan has twice as much mass. This is because the specific heat of water is over four times greater than the specific heat of aluminum. Therefore, it takes a bit more than twice as much heat to achieve the given temperature change for the water than for the aluminum pan.

Water can absorb a tremendous amount of energy with very little resulting temperature change. This property of water allows for life on Earth because it stabilizes temperatures. Other planets are less habitable because wild temperature swings make for a harsh environment. You may have noticed that climates closer to large bodies of water, such as oceans, are milder than climates landlocked in the middle of a large continent. This is due to the climate-moderating effect of water's large heat capacity-water stores large amounts of heat during hot weather and releases heat gradually when it's cold outside.

\section*{Worked Example}
Calculating Temperature Increase: Truck Brakes Overheat on Downhill Runs When a truck headed downhill brakes, the brakes must do work to convert the gravitational potential energy of the truck to internal energy of the brakes. This conversion prevents the gravitational potential energy from being converted into kinetic energy of the truck, and keeps the truck from speeding up and losing control. The increased internal energy of the brakes raises their\\
temperature. When the hill is especially steep, the temperature increase may happen too quickly and cause the brakes to overheat.

Calculate the temperature increase of 100 kg of brake material with an average specific heat of \(800 \mathrm{~J} / \mathrm{kg} \cdot{ }^{\circ} \mathrm{C}\) from a \(10,000 \mathrm{~kg}\) truck descending 75.0 m (in vertical displacement) at a constant speed.\\
\includegraphics[max width=\textwidth, center]{9044bee6-b601-41f3-ac5a-eb9d55d1bb9c-23}

\section*{Strategy}
We first calculate the gravitational potential energy ( \(M g h\) ) of the truck, and then find the temperature increase produced in the brakes.

\section*{Solution}
\begin{enumerate}
  \item Calculate the change in gravitational potential energy as the truck goes downhill.
\end{enumerate}

\begin{itemize}
  \item \(M g h=(10,000 \mathrm{~kg})\left(9.80 \mathrm{~m} / \mathrm{s}^{2}\right)(75.0 \mathrm{~m})=7.35 \times 10^{6} \mathrm{~J}\)\\
11.15
\end{itemize}

\begin{enumerate}
  \setcounter{enumi}{1}
  \item Calculate the temperature change from the heat transferred by rearranging the equation \(Q=m c \Delta T\) to solve for \(\Delta T\).
\end{enumerate}

\begin{itemize}
  \item \(\Delta T=\frac{Q}{m c}\),\\
11.16\\
where \(m\) is the mass of the brake material (not the entire truck). Insert the values \(Q=7.35 \times 10^{6} \mathrm{~J}\) (since the heat transfer is equal to the change in gravitational potential energy), \(m=100 \mathrm{~kg}\), and \(c=800 \mathrm{~J} / \mathrm{kg} \cdot{ }^{\circ} \mathrm{C}\) to find
\end{itemize}

\[
\Delta T=\frac{7.35 \times 10^{6} \mathrm{~J}}{(100 \mathrm{~kg})\left(800 \mathrm{~J} / \mathrm{kg} \cdot{ }^{\circ} \mathrm{C}\right)}=91.9^{\circ} \mathrm{C}
\]

11.17

\section*{Discussion}
This temperature is close to the boiling point of water. If the truck had been traveling for some time, then just before the descent, the brake temperature would likely be higher than the ambient temperature. The temperature increase in the descent would likely raise the temperature of the brake material above the boiling point of water, which would be hard on the brakes. This is why truck drivers sometimes use a different technique for called "engine braking" to avoid burning their brakes during steep descents. Engine braking is using the slowing forces of an engine in low gear rather than brakes to slow down.

\section*{Practice Problems}
5.

How much heat does it take to raise the temperature of 10.0 kg of water by 1.0 \({ }^{\circ} \mathrm{C}\) ?\\
a. 84 J\\
b. 42 J\\
c. 84 kJ\\
d. 42 kJ\\
6.

Calculate the change in temperature of 1.0 kg of water that is initially at room temperature if 3.0 kJ of heat is added.\\
a. \(358{ }^{\circ} \mathrm{C}\)\\
b. \(716{ }^{\circ} \mathrm{C}\)\\
c. \(0.36{ }^{\circ} \mathrm{C}\)\\
d. \(0.72{ }^{\circ} \mathrm{C}\)

\section*{Check Your Understanding}
\section*{Teacher Support}
Teacher Support Use these questions to assess student achievement of the section's learning objectives. If students are struggling with a specific objective, these questions will help identify which and direct students to the relevant content.\\
7.

What causes heat transfer?\\
a. The mass difference between two objects causes heat transfer.\\
b. The density difference between two objects causes heat transfer.\\
c. The temperature difference between two systems causes heat transfer.\\
d. The pressure difference between two objects causes heat transfer.\\
8.

When two bodies of different temperatures are in contact, what is the overall direction of heat transfer?\\
a. The overall direction of heat transfer is from the higher-temperature object to the lower-temperature object.\\
b. The overall direction of heat transfer is from the lower-temperature object to the higher-temperature object.\\
c. The direction of heat transfer is first from the lower-temperature object to the higher-temperature object, then back again to the lower-temperature object, and so-forth, until the objects are in thermal equilibrium.\\
d. The direction of heat transfer is first from the higher-temperature object to the lower-temperature object, then back again to the higher-temperature object, and so-forth, until the objects are in thermal equilibrium.\\
9.

What are the different methods of heat transfer?\\
a. conduction, radiation, and reflection\\
b. conduction, reflection, and convection\\
c. convection, radiation, and reflection\\
d. conduction, radiation, and convection\\
10.

True or false-Conduction and convection cannot happen simultaneously\\
a. True\\
b. False

\subsection*{11.3 Phase Change and Latent Heat}
\section*{Section Learning Objectives}
By the end of this section, you will be able to do the following:

\begin{itemize}
  \item Explain changes in heat during changes of state, and describe latent heats of fusion and vaporization
  \item Solve problems involving thermal energy changes when heating and cooling substances with phase changes
\end{itemize}

\section*{Teacher Support}
Teacher Support The learning objectives in this section will help your students master the following standards:

\begin{itemize}
  \item (6) Science concepts. The student knows that changes occur within a physical system and applies the laws of conservation of energy and momentum. The student is expected to:
  \item (E) describe how the macroscopic properties of a thermodynamic system such as temperature, specific heat, and pressure are related to the molecular level of matter, including kinetic or potential energy of atoms;
  \item (F) contrast and give examples of different processes of thermal energy transfer, including conduction, convection, and radiation.
\end{itemize}

\section*{Section Key Terms}
\section*{Teacher Support}
Teacher Support Introduce this section by asking students to give examples of solids, liquids, and gases.

\section*{Phase Changes}
So far, we have learned that adding thermal energy by heat increases the temperature of a substance. But surprisingly, there are situations where adding energy does not change the temperature of a substance at all! Instead, the additional thermal energy acts to loosen bonds between molecules or atoms and causes a phase change. Because this energy enters or leaves a system during a phase change without causing a temperature change in the system, it is known as latent heat (latent means hidden).

The three phases of matter that you frequently encounter are solid, liquid and gas (see Figure 11.8). Solid has the least energetic state; atoms in solids are in close contact, with forces between them that allow the particles to vibrate but not change position with neighboring particles. (These forces can be thought of as springs that can be stretched or compressed, but not easily broken.)

Liquid has a more energetic state, in which particles can slide smoothly past one another and change neighbors, although they are still held together by their mutual attraction.

Gas has a more energetic state than liquid, in which particles are broken free of their bonds. Particles in gases are separated by distances that are large compared with the size of the particles.

The most energetic state of all is plasma. Although you may not have heard much about plasma, it is actually the most common state of matter in the universe - stars are made up of plasma, as is lightning. The plasma state is reached by heating a gas to the point where particles are pulled apart, separating the electrons from the rest of the particle. This produces an ionized gas that is a combination of the negatively charged free electrons and positively charged ions, known as plasma.

\begin{figure}[h]
\begin{center}
  \includegraphics[max width=\textwidth]{9044bee6-b601-41f3-ac5a-eb9d55d1bb9c-27}
\captionsetup{labelformat=empty}
\caption{Figure 11.8 (a) Particles in a solid always have the same neighbors, held close by forces represented here by springs. These particles are essentially in contact with one another. A rock is an example of a solid. This rock retains its shape because of the forces holding its atoms or molecules together. (b) Particles in a liquid are also in close contact but can slide over one another. Forces between them strongly resist attempts to push them closer together and also hold them in close contact. Water is an example of a liquid. Water can flow, but it also remains in an open container because of the forces between its molecules. (c) Particles in a gas are separated by distances that are considerably larger than the size of the particles themselves, and they move about freely. A gas must be held in a closed container to prevent it from moving out into its surroundings. (d) The atmosphere is ionized in the extreme heat of a lightning strike.}
\end{center}
\end{figure}

During a phase change, matter changes from one phase to another, either through the addition of energy by heat and the transition to a more energetic state, or from the removal of energy by heat and the transition to a less energetic\\
state.\\
Phase changes to a more energetic state include the following:

\begin{itemize}
  \item Melting-Solid to liquid
  \item Vaporization-Liquid to gas (included boiling and evaporation)
  \item Sublimation-Solid to gas
  \item IonizationGas to plasma
\end{itemize}

Phase changes to a less energetic state are as follows:

\begin{itemize}
  \item Condensation-Gas to liquid
  \item Freezing-Liquid to solid
  \item Recombination-Plasma to gas
  \item DepositionGas to solid
\end{itemize}

Energy is required to melt a solid because the bonds between the particles in the solid must be broken. Since the energy involved in a phase changes is used to break bonds, there is no increase in the kinetic energies of the particles, and therefore no rise in temperature. Similarly, energy is needed to vaporize a liquid to overcome the attractive forces between particles in the liquid. There is no temperature change until a phase change is completed. The temperature of a cup of soda and ice that is initially at \(0{ }^{\circ} \mathrm{C}\) stays at \(0{ }^{\circ} \mathrm{C}\) until all of the ice has melted. In the reverse of these processes-freezing and condensation-energy is released from the latent heat (see Figure 11.9).

\section*{Teacher Support}
Teacher Support [BL][OL] Ask students if the same amount of energy is absorbed or released in melting or freezing a particular quantity of a substance.\\[0pt]
[AL] Ask student how water is able to evaporate even when it is at room temperature and not at \(100^{\circ} \mathrm{C}\).

\begin{figure}[h]
\begin{center}
  \includegraphics[max width=\textwidth]{9044bee6-b601-41f3-ac5a-eb9d55d1bb9c-29}
\captionsetup{labelformat=empty}
\caption{Figure 11.9 (a) Energy is required to partially overcome the attractive forces between particles in a solid to form a liquid. That same energy must be removed for freezing to take place. (b) Particles are separated by large distances when changing from liquid to vapor, requiring significant energy to overcome}
\end{center}
\end{figure}

molecular attraction. The same energy must be removed for condensation to take place. There is no temperature change until a phase change is completed. (c) Enough energy is added that the liquid state is skipped over completely as a substance undergoes sublimation.

The heat, \(Q\), required to change the phase of a sample of mass \(m\) is\\
\(Q=m L_{f}\) (for melting/freezing),\\
\(Q=m L_{v}\) (for vaporization/condensation),\\
where \(L_{f}\) is the latent heat of fusion, and \(L_{v}\) is the latent heat of vaporization. The latent heat of fusion is the amount of heat needed to cause a phase change between solid and liquid. The latent heat of vaporization is the amount of heat needed to cause a phase change between liquid and gas. \(L_{f}\) and \(L_{v}\) are coefficients that vary from substance to substance, depending on the strength of intermolecular forces, and both have standard units of J/kg. See Table 11.3 for values of \(L_{f}\) and \(L_{v}\) of different substances.

Table 11.3 Latent Heats of Fusion and Vaporization, along with Melting and Boiling Points

Let's consider the example of adding heat to ice to examine its transitions through all three phases - solid to liquid to gas. A phase diagram indicating the temperature changes of water as energy is added is shown in Figure 11.10. The ice starts out at \(-20^{\circ} \mathrm{C}\), and its temperature rises linearly, absorbing heat at a constant rate until it reaches \(0^{\circ}\). Once at this temperature, the ice gradually melts, absorbing \(334 \mathrm{~kJ} / \mathrm{kg}\). The temperature remains constant at \(0{ }^{\circ} \mathrm{C}\) during this phase change. Once all the ice has melted, the temperature of the liquid\\
water rises, absorbing heat at a new constant rate. At \(100^{\circ} \mathrm{C}\), the water begins to boil and the temperature again remains constant while the water absorbs \(2256 \mathrm{~kJ} / \mathrm{kg}\) during this phase change. When all the liquid has become steam, the temperature rises again at a constant rate.

\begin{figure}[h]
\begin{center}
  \includegraphics[max width=\textwidth]{9044bee6-b601-41f3-ac5a-eb9d55d1bb9c-31}
\captionsetup{labelformat=empty}
\caption{Figure 11.10 A graph of temperature versus added energy. The system is constructed so that no vapor forms while ice warms to become liquid water, and so when vaporization occurs, the vapor remains in the system. The long stretches of constant temperature values at \(0{ }^{\circ} \mathrm{C}\) and \(100^{\circ} \mathrm{C}\) reflect the large latent heats of melting and vaporization, respectively.}
\end{center}
\end{figure}

We have seen that vaporization requires heat transfer to a substance from its surroundings. Condensation is the reverse process, where heat in transferred away from a substance to its surroundings. This release of latent heat increases the temperature of the surroundings. Energy must be removed from the condensing particles to make a vapor condense. This is why condensation occurs on cold surfaces: the heat transfers energy away from the warm vapor to the cold surface. The energy is exactly the same as that required to cause the phase change in the other direction, from liquid to vapor, and so it can be calculated from \(Q=m L_{v}\). Latent heat is also released into the environment when a liquid freezes, and can be calculated from \(Q=m L_{f}\).

\section*{Fun In Physics}
\section*{Making Ice Cream}
\begin{figure}[h]
\begin{center}
  \includegraphics[max width=\textwidth]{9044bee6-b601-41f3-ac5a-eb9d55d1bb9c-32}
\captionsetup{labelformat=empty}
\caption{Figure 11.11 With the proper ingredients, some ice and a couple of plastic bags, you could make your own ice cream in five minutes. (ElinorD, Wikimedia Commons)}
\end{center}
\end{figure}

Ice cream is certainly easy enough to buy at the supermarket, but for the hardcore ice cream enthusiast, that may not be satisfying enough. Going through the process of making your own ice cream lets you invent your own flavors and marvel at the physics firsthand (Figure 11.11).

The first step to making homemade ice cream is to mix heavy cream, whole milk, sugar, and your flavor of choice; it could be as simple as cocoa powder or vanilla extract, or as fancy as pomegranates or pistachios.

The next step is to pour the mixture into a container that is deep enough that you will be able to churn the mixture without it spilling over, and that is also freezer-safe. After placing it in the freezer, the ice cream has to be stirred vigorously every 45 minutes for four to five hours. This slows the freezing process and prevents the ice cream from turning into a solid block of ice. Most people prefer a soft creamy texture instead of one giant popsicle.

As it freezes, the cream undergoes a phase change from liquid to solid. By now, we're experienced enough to know that this means that the cream must experience a loss of heat. Where does that heat go? Due to the temperature difference between the freezer and the ice cream mixture, heat transfers thermal energy from the ice cream to the air in the freezer. Once the temperature in the freezer rises enough, the freezer is cooled by pumping excess heat outside into the kitchen.

A faster way to make ice cream is to chill it by placing the mixture in a plastic bag, surrounded by another plastic bag half full of ice. (You can also add a teaspoon of salt to the outer bag to lower the temperature of the ice/salt mixture.) Shaking the bag for five minutes churns the ice cream while cooling it evenly. In this case, the heat transfers energy out of the ice cream mixture and into the ice during the phase change.

This video gives a demonstration of how to make home-made ice cream using ice and plastic bags.

Why does the ice bag method work so much faster than the freezer method for making ice cream?\\
a. Ice has a smaller specific heat than the surrounding air in a freezer. Hence, it absorbs more energy from the ice-cream mixture.\\
b. Ice has a smaller specific heat than the surrounding air in a freezer. Hence, it absorbs less energy from the ice-cream mixture.\\
c. Ice has a greater specific heat than the surrounding air in a freezer. Hence, it absorbs more energy from the ice-cream mixture.\\
d. Ice has a greater specific heat than the surrounding air in a freezer. Hence, it absorbs less energy from the ice-cream mixture.

\section*{Solving Thermal Energy Problems with Phase Changes}
\section*{Worked Example}
Calculating Heat Required for a Phase Change Calculate a) how much energy is needed to melt 1.000 kg of ice at \(0^{\circ} \mathrm{C}\) (freezing point), and b) how much energy is required to vaporize 1.000 kg of water at \(100^{\circ} \mathrm{C}\) (boiling point).

\section*{Strategy FOR (A)}
Using the equation for the heat required for melting, and the value of the latent heat of fusion of water from the previous table, we can solve for part (a).

Solution to (a)\\
The energy to melt 1.000 kg of ice is\\
\(Q=m L_{f}=(1.000 \mathrm{~kg})(334 \mathrm{~kJ} / \mathrm{kg})=334 \mathrm{~kJ}\).\\
11.18

\section*{Strategy FOR (B)}
To solve part (b), we use the equation for heat required for vaporization, along with the latent heat of vaporization of water from the previous table.

Solution to (b)\\
The energy to vaporize 1.000 kg of liquid water is\\
\(Q=m L_{v}=(1.000 \mathrm{~kg})(2256 \mathrm{~kJ} / \mathrm{kg})=2256 \mathrm{~kJ}\).\\
11.19

Discussion\\
The amount of energy need to melt a kilogram of ice ( 334 kJ ) is the same amount of energy needed to raise the temperature of 1.000 kg of liquid water from \(0^{\circ} \mathrm{C}\) to \(79.8^{\circ} \mathrm{C}\). This example shows that the energy for a phase change is enormous compared to energy associated with temperature changes. It also demonstrates that the amount of energy needed for vaporization is even greater.

\section*{Worked Example}
Calculating Final Temperature from Phase Change: Cooling Soda with Ice Cubes Ice cubes are used to chill a soda at \(20^{\circ} \mathrm{C}\) and with a mass of \(m_{\text {soda }}=0.25 \mathrm{~kg}\). The ice is at \(0{ }^{\circ} \mathrm{C}\) and the total mass of the ice cubes is 0.018 kg . Assume that the soda is kept in a foam container so that heat loss can be ignored, and that the soda has the same specific heat as water. Find the final temperature when all of the ice has melted.

\section*{Strategy}
The ice cubes are at the melting temperature of \(0{ }^{\circ} \mathrm{C}\). Heat is transferred from the soda to the ice for melting. Melting of ice occurs in two steps: first, the phase change occurs and solid (ice) transforms into liquid water at the melting temperature; then, the temperature of this water rises. Melting yields water at \(0^{\circ} \mathrm{C}\), so more heat is transferred from the soda to this water until they are the same temperature. Since the amount of heat leaving the soda is the same as the amount of heat transferred to the ice.\\
\(Q_{\text {ice }}=-Q_{\text {soda }}\)\\
11.20

The heat transferred to the ice goes partly toward the phase change (melting), and partly toward raising the temperature after melting. Recall from the last section that the relationship between heat and temperature change is \(Q= m c \Delta T\). For the ice, the temperature change is \(T_{f}-0{ }^{\circ} \mathrm{C}\). The total heat transferred to the ice is therefore\\
\(Q_{i c e}=m_{i c e} L_{f}+m_{i c e} c_{w}\left(T_{f}-0{ }^{\circ} \mathrm{C}\right)\).\\
11.21

Since the soda doesn't change phase, but only temperature, the heat given off by the soda is\\
\(Q_{\text {soda }}=m_{\text {soda }} c_{w}\left(T_{f}-20^{\circ} \mathrm{C}\right)\).\\
11.22

Since \(Q_{\text {ice }}=-Q_{\text {soda }}\),\\
\(m_{i c e} L_{f}+m_{i c e} c_{w}\left(T_{f}-0^{\circ} \mathrm{C}\right)=-m_{\text {soda }} c_{w}\left(T_{f}-20^{\circ} \mathrm{C}\right)\).\\
11.23

Bringing all terms involving \(T_{f}\) to the left-hand-side of the equation, and all other terms to the right-hand-side, we can solve for \(T_{f}\).\\
\(T_{f}=\frac{m_{\text {soda }} c_{w}\left(20^{\circ} \mathrm{C}\right)-m_{\text {ice }} L_{f}}{\left(m_{\text {soda }}+m_{\text {ice }}\right) c_{w}}\)\\
11.24

Substituting the known quantities\\
\(T_{f}=\frac{(0.25 \mathrm{~kg})\left(4186 \mathrm{~J} / \mathrm{kg} \cdot{ }^{\circ} \mathrm{C}\right)\left(20^{\circ} \mathrm{C}\right)-(0.018 \mathrm{~kg})(334,000 \mathrm{~J} / \mathrm{kg})}{(0.25 \mathrm{~kg}+0.018 \mathrm{~kg})\left(4186 \mathrm{~K} / \mathrm{kg} \cdot{ }^{\circ} \mathrm{C}\right)}=13{ }^{\circ} \mathrm{C}\)\\
11.25

Discussion\\
This example shows the enormous energies involved during a phase change. The mass of the ice is about 7 percent the mass of the soda, yet it causes a noticeable change in the soda's temperature.

\section*{Tips For Success}
If the ice were not already at the freezing point, we would also have to factor in how much energy would go into raising its temperature up to \(0^{\circ} \mathrm{C}\), before the phase change occurs. This would be a realistic scenario, because the temperature of ice is often below \(0^{\circ} \mathrm{C}\).

\section*{Practice Problems}
11.

How much energy is needed to melt 2.00 kg of ice at \(0^{\circ} \mathrm{C}\) ?\\
a. 334 kJ\\
b. 336 kJ\\
c. 167 kJ\\
d. 668 kJ\\
12.

If \(2500 \backslash, \backslash \operatorname{text}\{\mathrm{~kJ}\}\) of energy is just enough to melt \(3.0 \backslash, \backslash \operatorname{text}\{\mathrm{~kg}\}\) of a substance, what is the substance's latent heat of fusion?\\
a. \(7500 \backslash, \backslash \operatorname{text}\{\mathrm{~kJ}\} \backslash \operatorname{cdot} \backslash \operatorname{text}\{\mathrm{kg}\}\)\\
b. \(7500 \backslash, \backslash \operatorname{text}\{\mathrm{~kJ} / \mathrm{kg}\}\)\\
c. \(830 \backslash, \backslash \operatorname{text}\{\mathrm{~kJ}\} \backslash \operatorname{cdot} \backslash \operatorname{text}\{\mathrm{kg}\}\)\\
d. \(830 \backslash, \backslash \operatorname{text}\{\mathrm{~kJ} / \mathrm{kg}\}\)

\section*{Check Your Understanding}
\section*{Teacher Support}
Teacher Support Use these questions to assess student achievement of the section's learning objectives. If students are struggling with a specific objective, these questions will help identify which and direct students to the relevant content.\\
13.

What is latent heat?\\
a. It is the heat that must transfer energy to or from a system in order to cause a mass change with a slight change in the temperature of the system.\\
b. It is the heat that must transfer energy to or from a system in order to cause a mass change without a temperature change in the system.\\
c. It is the heat that must transfer energy to or from a system in order to cause a phase change with a slight change in the temperature of the system.\\
d. It is the heat that must transfer energy to or from a system in order to cause a phase change without a temperature change in the system.\\
14.

In which phases of matter are molecules capable of changing their positions?\\
a. gas, liquid, solid\\
b. liquid, plasma, solid\\
c. liquid, gas, plasma\\
d. plasma, gas, solid

\section*{Key Terms}
absolute zero lowest possible temperature; the temperature at which all molecular motion ceases

Celsius scale temperature scale in which the freezing point of water is \(0^{\circ} \mathrm{C}\) and the boiling point of water is \(100^{\circ} \mathrm{C}\) at 1 atm of pressure\\
condensation phase change from gas to liquid\\
conduction heat transfer through stationary matter by physical contact\\
convection heat transfer by the movement of fluid\\
degree Celsius unit on the Celsius temperature scale\\
degree Fahrenheit unit on the Fahrenheit temperature scale\\
Fahrenheit scale temperature scale in which the freezing point of water is 32 \({ }^{\circ} \mathrm{F}\) and the boiling point of water is \(212{ }^{\circ} \mathrm{F}\)\\
freezing phase change from liquid to solid\\
heat transfer of thermal (or internal) energy due to a temperature difference\\
heat capacity amount of heat necessary to change the temperature of a substance by \(1.00^{\circ} \mathrm{C}\)

Kelvin unit on the Kelvin temperature scale; note that it is never referred to in terms of "degrees" Kelvin

Kelvin scale temperature scale in which 0 K is the lowest possible temperature, representing absolute zero\\
latent heat heat related to the phase change of a substance rather than a change of temperature\\
latent heat of fusion amount of heat needed to cause a phase change between solid and liquid\\
latent heat of vaporization amount of heat needed to cause a phase change between liquid and gas\\
melting phase change from solid to liquid\\
phase change transition between solid, liquid, or gas states of a substance\\
plasma ionized gas that is a combination of the negatively charged free electrons and positively charged ions\\
radiation energy transferred by electromagnetic waves\\
specific heat amount of heat necessary to change the temperature of 1.00 kg of a substance by \(1.00^{\circ} \mathrm{C}\)\\
sublimation phase change from solid to gas\\
temperature quantity measured by a thermometer thermal energy average random kinetic energy of a molecule or an atom vaporization phase change from liquid to gas

\section*{Key Equations}
\subsection*{11.1 Temperature and Thermal Energy}
\begin{center}
\begin{tabular}{ll}
\hline
Celsius to Fahrenheit conversion & \(T_{{ }^{\circ} \mathrm{F}}=\frac{9}{5} T_{{ }^{\circ} \mathrm{C}}+32\) \\
Fahrenheit to Celsius conversion & \(T_{{ }^{\circ} \mathrm{C}}=\frac{5}{9}\left(T_{{ }^{\circ} \mathrm{F}}-32\right)\) \\
Celsius to Kelvin conversion & \(T_{\mathrm{K}}=T_{{ }^{\circ} \mathrm{C}}+273.15\) \\
Kelvin to Celsius conversion & \(T_{{ }^{\circ} \mathrm{C}}=T_{\mathrm{K}}-273.15\) \\
Fahrenheit to Kelvin conversion & \(T_{\mathrm{K}}=\frac{5}{9}\left(T_{{ }^{\circ} \mathrm{F}}-32\right)+273.15\) \\
Kelvin to Fahrenheit conversion & \(T_{{ }^{\circ} \mathrm{F}}=\frac{9}{5}\left(T_{\mathrm{K}}-273.15\right)+32\) \\
\hline
\end{tabular}
\end{center}

\subsection*{11.2 Heat, Specific Heat, and Heat Transfer}
\subsection*{11.3 Phase Change and Latent Heat}
\section*{Section Summary}
\subsection*{11.1 Temperature and Thermal Energy}
\begin{itemize}
  \item Temperature is the quantity measured by a thermometer.
  \item Temperature is related to the average kinetic energy of atoms and molecules in a system.
  \item Absolute zero is the temperature at which there is no molecular motion.
  \item There are three main temperature scales: Celsius, Fahrenheit, and Kelvin.
  \item Temperatures on one scale can be converted into temperatures on another scale.
\end{itemize}

\subsection*{11.2 Heat, Specific Heat, and Heat Transfer}
\begin{itemize}
  \item Heat is thermal (internal) energy transferred due to a temperature difference.
  \item The transfer of heat \(Q\) that leads to a change \(\Delta T\) in the temperature of a body with mass m is \(Q=m c \Delta T\), where \(c\) is the specific heat of the material.
  \item Heat is transferred by three different methods: conduction, convection, and radiation.
  \item Heat conduction is the transfer of heat between two objects in direct contact with each other.
  \item Convection is heat transfer by the movement of mass.
  \item Radiation is heat transfer by electromagnetic waves.
\end{itemize}

\subsection*{11.3 Phase Change and Latent Heat}
\begin{itemize}
  \item Most substances have four distinct phases: solid, liquid, gas, and plasma.
  \item Gas is the most energetic state and solid is the least.
  \item During a phase change, a substance undergoes transition to a higher energy state when heat is added, or to a lower energy state when heat is removed.
  \item Heat is added to a substance during melting and vaporization.
  \item Latent heat is released by a substance during condensation and freezing.
  \item Phase changes occur at fixed temperatures called boiling and freezing (or melting) points for a given substance.
\end{itemize}

\end{document}