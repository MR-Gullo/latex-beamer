\documentclass[10pt]{article}
\usepackage[utf8]{inputenc}
\usepackage[T1]{fontenc}
\usepackage{graphicx}
\usepackage[export]{adjustbox}
\graphicspath{ {./images/} }
\usepackage{caption}
\usepackage{amsmath}
\usepackage{amsfonts}
\usepackage{amssymb}
\usepackage[version=4]{mhchem}
\usepackage{stmaryrd}

\DeclareUnicodeCharacter{00D7}{\ifmmode\times\else{$\times$}\fi}

\begin{document}
\captionsetup{singlelinecheck=false}
\begin{figure}[h]
\begin{center}
  \includegraphics[max width=\textwidth]{8484573f-7913-4d1d-981f-d54fe53a1e32-01}
\captionsetup{labelformat=empty}
\caption{Figure 1.1 Galaxies, such as the Andromeda galaxy pictured here, are immense in size. The small blue spots in this photo are also galaxies. The same physical laws apply to objects as large as galaxies or objects as small as atoms. The laws of physics are, therefore, surprisingly few in number. (NASA, JPL-Caltech, P. Barmby, Harvard-Smithsonian Center for Astrophysics).}
\end{center}
\end{figure}

\section*{Chapter Outline}
1.1 Physics: Definitions and Applications\\
1.2 The Scientific Methods\\
1.3 The Language of Physics: Physical Quantities and Units

\section*{Introduction}
\section*{Teacher Support}
Teacher Support Before students begin this chapter, it is useful to review the following concepts:

\begin{itemize}
  \item The definition of the atom and subatomic particles (electron, proton, and neutron)
  \item Metric units
  \item Using significant figures in calculations
\end{itemize}

\section*{Teacher Support}
Teacher Support The photo of the Andromeda galaxy and its subsequent mentioned in this chapter is meant to show students that the same laws of physics apply to extremely large systems, such as a galaxy, and apply also to smaller systems in our universe. The same laws that govern the movement of\\
the stars within the Andromeda galaxy also explain the gravitational forces on Earth that all humans experience and interact with every second of their lives.

Take a look at the image above of the Andromeda Galaxy (Figure 1.1), which contains billions of stars. This galaxy is the nearest one to our own galaxy (the Milky Way) but is still a staggering 2.5 million light years from Earth. (A light year is a measurement of the distance light travels in a year.) Yet, the primary force that affects the movement of stars within Andromeda is the same force that we contend with here on Earth-namely, gravity.

You may soon realize that physics plays a much larger role in your life than you thought. This section introduces you to the realm of physics, and discusses applications of physics in other disciplines of study. It also describes the methods by which science is done, and how scientists communicate their results to each other.

\subsection*{1.1 Physics: Definitions and Applications}
\section*{Section Learning Objectives}
By the end of this section, you will be able to do the following:

\begin{itemize}
  \item Describe the definition, aims, and branches of physics
  \item Describe and distinguish classical physics from modern physics and describe the importance of relativity, quantum mechanics, and relativistic quantum mechanics in modern physics
  \item Describe how aspects of physics are used in other sciences (e.g., biology, chemistry, geology, etc.) as well as in everyday technology
\end{itemize}

\section*{Teacher Support}
Teacher Support The learning objectives in this section will help your students master the following standards:

\begin{itemize}
  \item (2) Scientific processes. The student uses a systematic approach to answer scientific laboratory and field investigative questions. The student is expected to:
  \item (A) know the definition of science and understand that it has limitations, as specified in subsection (b)(2) of this section;
  \item (3) Scientific processes. The student uses critical thinking, scientific reasoning, and problem solving to make informed decisions within and outside the classroom. The student is expected to:
  \item (A) in all fields of science, analyze, evaluate, and critique scientific explanations by using empirical evidence, logical reasoning, and experimental and observational testing, including examining all sides of scientific evidence of those scientific explanations, so as to encourage critical thinking by the student.
  \item (B) communicate and apply scientific information extracted from various sources such as current events, news reports, published journal articles, and marketing materials;
  \item (C) draw inferences based on data related to promotional materials for products and services;
  \item (D) explain the impacts of the scientific contributions of a variety of historical and contemporary scientists on scientific thought and society.
\end{itemize}

\section*{Section Key Terms}
\section*{Teacher Support}
Teacher Support To help meet the multimodal needs of classrooms today, OpenStax Tutor's Physics provides Teacher Support tips for on-level [OL], below-level [BL], and above-level [AL] students.\\[0pt]
[OL]Pre-assessment for this section could involve asking students the definition of matter, atoms, electrons, protons, neutrons, subatomic particles, and energy. Students could also be asked to name some prominent classical and modern physicists and to describe some of their work in general terms.\\[0pt]
[OL]The introduction and opening picture are meant to show students that the physical laws governing their own everyday surroundings also govern the movement of stars in a galaxy. Teachers could ask students how gravity affects life on Earth. Students will likely mention how gravity keeps us on Earth's surface. Prompt them, if necessary, to also think about Earth's orbital motion around the sun. This motion allows Earth to bask in the warmth of the sun's light. Without the Sun's gravity, Earth would continue moving in a straight line and move away from the sun, while people would float off of Earth's surface. The orbit of the moon could also be brought into this discussion, because Earth's gravity keeps the moon moving around Earth rather than continuing in a straight path.

\section*{What Physics Is}
Think about all of the technological devices that you use on a regular basis. Computers, wireless internet, smart phones, tablets, global positioning system (GPS), MP3 players, and satellite radio might come to mind. Next, think about the most exciting modern technologies that you have heard about in the news, such as trains that levitate above their tracks, invisibility cloaks that bend light around them, and microscopic robots that fight diseased cells in our bodies. All of these groundbreaking advancements rely on the principles of physics.

Physics is a branch of science. The word science comes from a Latin word that means having knowledge, and refers the knowledge of how the physical world operates, based on objective evidence determined through observation and experimentation. A key requirement of any scientific explanation of a natural phenomenon is that it must be testable; one must be able to devise and conduct an experimental investigation that either supports or refutes the explanation. It is important to note that some questions fall outside the realm of science precisely because they deal with phenomena that are not scientifically testable. This need for objective evidence helps define the investigative process scientists follow, which will be described later in this chapter.

Physics is the science aimed at describing the fundamental aspects of our universe. This includes what things are in it, what properties of those things are noticeable, and what processes those things or their properties undergo. In simpler terms, physics attempts to describe the basic mechanisms that make our\\
universe behave the way it does. For example, consider a smart phone (Figure 1.2). Physics describes how electric current interacts with the various circuits inside the device. This knowledge helps engineers select the appropriate materials and circuit layout when building the smart phone. Next, consider a GPS. Physics describes the relationship between the speed of an object, the distance over which it travels, and the time it takes to travel that distance. When you use a GPS device in a vehicle, it utilizes these physics relationships to determine the travel time from one location to another.

Figure 1.2 Physics describes the way that electric charge flows through the circuits of this device. Engineers use their knowledge of physics to construct a smartphone with features that consumers will enjoy, such as a GPS function. GPS uses physics equations to determine the driving time between two locations on a map. (@gletham GIS, Social, Mobile Tech Images)

\section*{Teacher Support}
Teacher Support [AL]Ask what parts of a cell phone should contain conducting materials (wires, circuit boards, etc.) versus insulating materials (e.g., places where electrical insulation keeps humans from touching electrical circuits inside the phone).\\[0pt]
[AL]You can delve into GPS usage at this point by defining velocity \(=\) distance/time, discussing triangulation, and/or discussing line of sight.

As our technology evolved over the centuries, physics expanded into many branches. Ancient peoples could only study things that they could see with the naked eye or otherwise experience without the aid of scientific equipment. This included the study of kinematics, which is the study of moving objects. For example, ancient people often studied the apparent motion of objects in the sky, such as the sun, moon, and stars. This is evident in the construction of prehistoric astronomical observatories, such as Stonehenge in England (shown in Figure 1.3).

\begin{figure}[h]
\begin{center}
  \includegraphics[max width=\textwidth]{8484573f-7913-4d1d-981f-d54fe53a1e32-07}
\captionsetup{labelformat=empty}
\caption{Figure 1.3 Stonehenge is a monument located in England that was built between 3000 and 1000 B.C. It functions as an ancient astronomical observatory, with certain rocks in the monument aligning with the position of the sun during the summer and winter solstices. Other rocks align with the rising and setting of the moon during certain days of the year. (Citypeek, Wikimedia Commons)}
\end{center}
\end{figure}

Ancient people also studied statics and dynamics, which focus on how objects start moving, stop moving, and change speed and direction in response to forces\\
that push or pull on the objects. This early interest in kinematics and dynamics allowed humans to invent simple machines, such as the lever, the pulley, the ramp, and the wheel. These simple machines were gradually combined and integrated to produce more complicated machines, such as wagons and cranes. Machines allowed humans to gradually do more work more effectively in less time, allowing them to create larger and more complicated buildings and structures, many of which still exist today from ancient times.

As technology advanced, the branches of physics diversified even more. These include branches such as acoustics, the study of sound, and optics, the study of the light. In 1608, the invention of the telescope by a Germany spectacle maker, Hans Lippershey, led to huge breakthroughs in astronomy-the study of objects or phenomena in space. One year later, in 1609, Galileo Galilei began the first studies of the solar system and the universe using a telescope. During the Renaissance era, Isaac Newton used observations made by Galileo to construct his three laws of motion. These laws were the standard for studying kinematics and dynamics even today.

Another major branch of physics is thermodynamics, which includes the study of thermal energy and the transfer of heat. James Prescott Joule, an English physicist, studied the nature of heat and its relationship to work. Joule's work helped lay the foundation for the first of three laws of thermodynamics that describe how energy in our universe is transferred from one object to another or transformed from one form to another. Studies in thermodynamics were motivated by the need to make engines more efficient, keep people safe from the elements, and preserve food.

The \(18{ }^{\text {th }}\) and \(19{ }^{\text {th }}\) centuries also saw great strides in the study of electricity and magnetism. Electricity involves the study of electric charges and their movements. Magnetism had long ago been noticed as an attractive force between a magnetized object and a metal like iron, or between the opposite poles (North and South) of two magnetized objects. In 1820, Danish physicist Hans Christian Oersted showed that electric currents create magnetic fields. In 1831, English inventor Michael Faraday showed that moving a wire through a magnetic field could induce an electric current. These studies led to the inventions of the electric motor and electric generator, which revolutionized human life by bringing electricity and magnetism into our machines.

The end of the \(19^{\text {th }}\) century saw the discovery of radioactive substances by the scientists Marie and Pierre Curie. Nuclear physics involves studying the nuclei of atoms, the source of nuclear radiation. In the \(20^{\text {th }}\) century, the study of nuclear physics eventually led to the ability to split the nucleus of an atom, a process called nuclear fission. This process is the basis for nuclear power plants and nuclear weapons. Also, the field of quantum mechanics, which involves the mechanics of atoms and molecules, saw great strides during the \(20{ }^{\text {th }}\) century as our understanding of atoms and subatomic particles increased (see below).

Early in the \(20^{\text {th }}\) century, Albert Einstein revolutionized several branches of\\
physics, especially relativity. Relativity revolutionized our understanding of motion and the universe in general as described further in this chapter. Now, in the \(21^{\text {st }}\) century, physicists continue to study these and many other branches of physics.

By studying the most important topics in physics, you will gain analytical abilities that will enable you to apply physics far beyond the scope of what can be included in a single book. These analytical skills will help you to excel academically, and they will also help you to think critically in any career you choose to pursue.

\section*{Physics: Past and Present}
The word physics is thought to come from the Greek word phusis, meaning nature. The study of nature later came to be called natural philosophy. From ancient times through the Renaissance, natural philosophy encompassed many fields, including astronomy, biology, chemistry, mathematics, and medicine. Over the last few centuries, the growth of scientific knowledge has resulted in ever-increasing specialization and branching of natural philosophy into separate fields, with physics retaining the most basic facets. Physics, as it developed from the Renaissance to the end of the \(19{ }^{\text {th }}\) century, is called classical physics. Revolutionary discoveries starting at the beginning of the \(20^{\text {th }}\) century transformed physics from classical physics to modern physics.

\section*{Teacher Support}
Teacher Support [BL][EL]English learners may need philosophy and classical defined during this section. Relate the definition of classical physics to the use of the word classical in a context that is probably more familiar to students, such as classic films.

Classical physics is not an exact description of the universe, but it is an excellent approximation under the following conditions: (1) matter must be moving at speeds less than about 1 percent of the speed of light, (2) the objects dealt with must be large enough to be seen with the naked eye, and (3) only weak gravity, such as that generated by Earth, can be involved. Very small objects, such as atoms and molecules, cannot be adequately explained by classical physics. These three conditions apply to almost all of everyday experience. As a result, most aspects of classical physics should make sense on an intuitive level.

\section*{Teacher Support}
Teacher Support [OL]To better relate to student experience, express the speed of light in units used while driving a car, for example, 1.080 million \(\mathrm{km} / \mathrm{h}\) or 671 million miles per hour. Relate this to the approximately eight minute trip that light takes to travel 150 billion kilometers ( 93 billion miles) from the Sun to the Earth.

Many laws of classical physics have been modified during the \(20^{\text {th }}\) century, resulting in revolutionary changes in technology, society, and our view of the universe. As a result, many aspects of modern physics, which occur outside of the range of our everyday experience, may seem bizarre or unbelievable. So why is most of this textbook devoted to classical physics? There are two main reasons. The first is that knowledge of classical physics is necessary to understand modern physics. The second reason is that classical physics still gives an accurate description of the universe under a wide range of everyday circumstances.

Modern physics includes two revolutionary theories: relativity and quantum mechanics. These theories deal with the very fast and the very small, respectively. The theory of relativity was developed by Albert Einstein in 1905. By examining how two observers moving relative to each other would see the same phenomena, Einstein devised radical new ideas about time and space. He came to the startling conclusion that the measured length of an object travelling at high speeds (greater than about one percent of the speed of light) is shorter than the same object measured at rest. Perhaps even more bizarre is the idea the time for the same process to occur is different depending on the motion of the observer. Time passes more slowly for an object travelling at high speeds. A trip to the nearest star system, Alpha Centauri, might take an astronaut 4.5 Earth years if the ship travels near the speed of light. However, because time is slowed at higher speeds, the astronaut would age only 0.5 years during the trip. Einstein's ideas of relativity were accepted after they were confirmed by numerous experiments.

Gravity, the force that holds us to Earth, can also affect time and space. For example, time passes more slowly on Earth's surface than for objects farther from the surface, such as a satellite in orbit. The very accurate clocks on global positioning satellites have to correct for this. They slowly keep getting ahead of clocks at Earth's surface. This is called time dilation, and it occurs because gravity, in essence, slows down time.

\section*{Teacher Support}
Teacher Support [AL]By saying that time passes more slowly at near-light speeds or high gravity, it is important to mention that people in both locations perceive the second as the same length of time.

Large objects, like Earth, have strong enough gravity to distort space. To visualize this idea, think about a bowling ball placed on a trampoline. The bowling ball depresses or curves the surface of the trampoline. If you rolled a marble across the trampoline, it would follow the surface of the trampoline, roll into the depression caused by the bowling ball, and hit the ball. Similarly, the Earth curves space around it in the shape of a funnel. These curves in space due to the Earth cause objects to be attracted to Earth (i.e., gravity).

Because of the way gravity affects space and time, Einstein stated that gravity affects the space-time continuum, as illustrated in Figure 1.4. This is why time\\
proceeds more slowly at Earth's surface than in orbit. In black holes, whose gravity is hundreds of times that of Earth, time passes so slowly that it would appear to a far-away observer to have stopped!

\begin{figure}[h]
\begin{center}
  \includegraphics[max width=\textwidth]{8484573f-7913-4d1d-981f-d54fe53a1e32-11}
\captionsetup{labelformat=empty}
\caption{Figure 1.4 Einstein's theory of relativity describes space and time as an interweaved mesh. Large objects, such as a planet, distort space, causing objects to fall in toward the planet due to the action of gravity. Large objects also distort time, causing time to proceed at a slower rate near the surface of Earth compared with the area outside of the distorted region of space-time.}
\end{center}
\end{figure}

\section*{Teacher Support}
Teacher Support [AL]Black holes are much more dense and massive than Earth. The greater an object's mass, the stronger the gravitational field it produces, and the more that gravity slows down time.

In summary, relativity says that in describing the universe, it is important to realize that time, space and speed are not absolute. Instead, they can appear different to different observers. Einstein's ability to reason out relativity is even more amazing because we cannot see the effects of relativity in our everyday lives.

Quantum mechanics is the second major theory of modern physics. Quantum mechanics deals with the very small, namely, the subatomic particles that make up atoms. Atoms (Figure 1.5) are the smallest units of elements. However, atoms themselves are constructed of even smaller subatomic particles, such as protons, neutrons and electrons. Quantum mechanics strives to describe the properties and behavior of these and other subatomic particles. Often, these\\
particles do not behave in the ways expected by classical physics. One reason for this is that they are small enough to travel at great speeds, near the speed of light.

\begin{figure}[h]
\begin{center}
  \includegraphics[max width=\textwidth]{8484573f-7913-4d1d-981f-d54fe53a1e32-12}
\captionsetup{labelformat=empty}
\caption{Figure 1.5 Using a scanning tunneling microscope (STM), scientists can see the individual atoms that compose this sheet of gold. (Erwinrossen)}
\end{center}
\end{figure}

\section*{Teacher Support}
Teacher Support [OL][AL]Assess prior knowledge of subatomic particles by asking students if they have heard of protons, electrons, neutrons, as well as quarks, Higgs-Boson particles, and so on.\\[0pt]
[AL]Scanning electron microscopes generate highly-detailed surface views of objects such as that shown in Figure 1.5. They scan the object's surface with beams of electrons to detect the object's microscopic topography.

At particle colliders (Figure 1.6), such as the Large Hadron Collider on the France-Swiss border, particle physicists can make subatomic particles travel at very high speeds within a 27 kilometers ( 17 miles) long superconducting tunnel. They can then study the properties of the particles at high speeds, as well as collide them with each other to see how they exchange energy. This has led to many intriguing discoveries such as the Higgs-Boson particle, which gives matter the property of mass, and antimatter, which causes a huge energy release when it comes in contact with matter.

\begin{figure}[h]
\begin{center}
  \includegraphics[max width=\textwidth]{8484573f-7913-4d1d-981f-d54fe53a1e32-13}
\captionsetup{labelformat=empty}
\caption{Figure 1.6 Particle colliders such as the Large Hadron Collider in Switzerland or Fermilab in the United States (pictured here), have long tunnels that allows subatomic particles to be accelerated to near light speed. (Andrius.v )}
\end{center}
\end{figure}

Physicists are currently trying to unify the two theories of modern physics, relativity and quantum mechanics, into a single, comprehensive theory called relativistic quantum mechanics. Relating the behavior of subatomic particles to gravity, time, and space will allow us to explain how the universe works in a much more comprehensive way.

\section*{Application of Physics}
You need not be a scientist to use physics. On the contrary, knowledge of physics is useful in everyday situations as well as in nonscientific professions. For example, physics can help you understand why you shouldn't put metal in the microwave (Figure 1.7), why a black car radiator helps remove heat in a car engine, and why a white roof helps keep the inside of a house cool. The operation of a car's ignition system, as well as the transmission of electrical signals through our nervous system, are much easier to understand when you think about them in terms of the basic physics of electricity.

\begin{figure}[h]
\begin{center}
  \includegraphics[max width=\textwidth]{8484573f-7913-4d1d-981f-d54fe53a1e32-14}
\captionsetup{labelformat=empty}
\caption{Figure 1.7 Why can't you put metal in the microwave? Microwaves are highenergy radiation that increases the movement of electrons in metal. These moving electrons can create an electrical current, causing sparking that can lead to a fire. (= MoneyBlogNewz)}
\end{center}
\end{figure}

\section*{Teacher Support}
Teacher Support [AL]It is hazardous to put metal in the microwave because metal reflects microwaves, which, when free to bounce around the oven, can damage the oven. Also, the metal in the microwave oven gets very hot and begins generating an electrical field. This electrical field ionizes the air surrounding the metal, creating sparks.

Physics is the foundation of many important scientific disciplines. For example, chemistry deals with the interactions of atoms and molecules. Not surprisingly, chemistry is rooted in atomic and molecular physics. Most branches of engineering are also applied physics. In architecture, physics is at the heart of determining structural stability, acoustics, heating, lighting, and cooling for buildings. Parts of geology, the study of nonliving parts of Earth, rely heavily on physics; including radioactive dating, earthquake analysis, and heat trans-\\
fer across Earth's surface. Indeed, some disciplines, such as biophysics and geophysics, are hybrids of physics and other disciplines.

\section*{Teacher Support}
Teacher Support [BL][EL]Students may need acoustics to be explained as the properties of a room or structure that determine how sound is transmitted within it.

Physics also describes the chemical processes that power the human body. Physics is involved in medical diagnostics, such as x-rays, magnetic resonance imaging (MRI), and ultrasonic blood flow measurements (Figure 1.8). Medical therapy Physics also has many applications in biology, the study of life. For example, physics describes how cells can protect themselves using their cell walls and cell membranes (Figure 1.9). Medical therapy sometimes directly involves physics, such as in using X-rays to diagnose health conditions. Physics can also explain what we perceive with our senses, such as how the ears detect sound or the eye detects color.\\
\includegraphics[max width=\textwidth, center]{8484573f-7913-4d1d-981f-d54fe53a1e32-16}

Figure 1.8 Magnetic resonance imaging (MRI) uses electromagnetic waves to yield an image of the brain, which doctors can use to find diseased regions. (Rashmi Chawla, Daniel Smith, and Paul E. Marik)

\begin{figure}[h]
\begin{center}
  \includegraphics[max width=\textwidth]{8484573f-7913-4d1d-981f-d54fe53a1e32-17}
\captionsetup{labelformat=empty}
\caption{Figure 1.9 Physics, chemistry, and biology help describe the properties of cell walls in plant cells, such as the onion cells seen here. (Umberto Salvagnin)}
\end{center}
\end{figure}

\section*{Teacher Support}
Teacher Support [BL]Cell membranes (found in the cells of all organisms) control the transport of materials into and out of a cell. Cell walls (found in plant cells, fungus cells, bacteria, and plant-like microbes) mainly provide structure and support.\\[0pt]
[AL]X-rays easily penetrate skin and soft tissues but are absorbed to a far greater extent by bone. This produces an image where bones within the body are clearly visible while soft tissue is not. MRI scans for the magnetic properties of atoms within the body, allowing the solid versus empty areas within the body to be visualized. Ultrasonic blood flow measurements use sound waves and the Doppler effect to measure blood flow speed and volume.

\section*{Boundless Physics}
The Physics of Landing on a Comet On November 12, 2014, the European Space Agency's Rosetta spacecraft (shown in Figure 1.10) became the first ever to reach and orbit a comet. Shortly after, Rosetta's rover, Philae, landed on the comet, representing the first time humans have ever landed a space probe on a comet.

\begin{figure}[h]
\begin{center}
  \includegraphics[max width=\textwidth]{8484573f-7913-4d1d-981f-d54fe53a1e32-18}
\captionsetup{labelformat=empty}
\caption{Figure 1.10 The Rosetta spacecraft, with its large and revolutionary solar panels, carried the Philae lander to a comet. The lander then detached and landed on the comet's surface. (European Space Agency)}
\end{center}
\end{figure}

After traveling 6.4 billion kilometers starting from its launch on Earth, Rosetta landed on the comet \(67 \mathrm{P} /\) Churyumov-Gerasimenko, which is only 4 kilometers wide. Physics was needed to successfully plot the course to reach such a small, distant, and rapidly moving target. Rosetta's path to the comet was not straight forward. The probe first had to travel to Mars so that Mars's gravity could accelerate it and divert it in the exact direction of the comet.

This was not the first time humans used gravity to power our spaceships. Voyager 2, a space probe launched in 1977, used the gravity of Saturn to slingshot over to Uranus and Neptune (illustrated in Figure 1.11), providing the first pictures ever taken of these planets. Now, almost 40 years after its launch, Voyager 2 is at the very edge of our solar system and is about to enter interstellar space. Its sister ship, Voyager 1 (illustrated in Figure 1.11), which was also launched in 1977, is already there.

To listen to the sounds of interstellar space or see images that have been transmitted back from the Voyager I or to learn more about the Voyager mission, visit the Voyager's Mission website.

\begin{figure}[h]
\begin{center}
  \includegraphics[max width=\textwidth]{8484573f-7913-4d1d-981f-d54fe53a1e32-19}
\captionsetup{labelformat=empty}
\caption{Figure 1.11 a) Voyager 2, launched in 1977, used the gravity of Saturn to slingshot over to Uranus and Neptune. NASA b) A rendering of Voyager 1, the first space probe to ever leave our solar system and enter interstellar space. NASA}
\end{center}
\end{figure}

Both Voyagers have electrical power generators based on the decay of radioisotopes. These generators have served them for almost 40 years. Rosetta, on the other hand, is solar-powered. In fact, Rosetta became the first space probe to travel beyond the asteroid belt by relying only on solar cells for power generation.

At 800 million kilometers from the sun, Rosetta receives sunlight that is only 4 percent as strong as on Earth. In addition, it is very cold in space. Therefore, a lot of physics went into developing Rosetta's low-intensity low-temperature solar cells.

In this sense, the Rosetta project nicely shows the huge range of topics encompassed by physics: from modeling the movement of gigantic planets over huge distances within our solar systems, to learning how to generate electric power from low-intensity light. Physics is, by far, the broadest field of science.

\section*{Grasp Check}
What characteristics of the solar system would have to be known or calculated in order to send a probe to a distant planet, such as Jupiter?\\
a. the effects due to the light from the distant stars\\
b. the effects due to the air in the solar system\\
c. the effects due to the gravity from the other planets\\
d. the effects due to the cosmic microwave background radiation

\section*{Teacher Support}
Teacher Support This passage describes the physics behind getting the Rosetta and Voyager probes across the solar system using gravitational sling\\
shots. In addition, the physics behind the power systems of these probes is compared. This is meant to reinforce how physics applies over wide ranges, from the immense distances in our universe to the tiny sizes of subatomic particles.

Answers to the Grasp Check may vary. A sample answer: You would have to how the target planet moves to know when to launch the probe so it actually reaches the planet. You would also need to know and account for the effects of gravity from other planets during the path followed during its journey.

In summary, physics studies many of the most basic aspects of science. A knowledge of physics is, therefore, necessary to understand all other sciences. This is because physics explains the most basic ways in which our universe works. However, it is not necessary to formally study all applications of physics. A knowledge of the basic laws of physics will be most useful to you, so that you can use them to solve some everyday problems. In this way, the study of physics can improve your problem-solving skills.

\section*{Check Your Understanding}
1.

Which of the following is not an essential feature of scientific explanations?\\
a. They must be subject to testing.\\
b. They strictly pertain to the physical world.\\
c. Their validity is judged based on objective observations.\\
d. Once supported by observation, they can be viewed as a fact.\\
2.

Which of the following does not represent a question that can be answered by science?\\
a. How much energy is released in a given nuclear chain reaction?\\
b. Can a nuclear chain reaction be controlled?\\
c. Should uncontrolled nuclear reactions be used for military applications?\\
d. What is the half-life of a waste product of a nuclear reaction?\\
3.

What are the three conditions under which classical physics provides an excellent description of our universe?\\
a. 1. Matter is moving at speeds less than about 1 percent of the speed of light\\
2. Objects dealt with must be large enough to be seen with the naked eye.\\
3. Strong electromagnetic fields are involved.\\
b. 1. Matter is moving at speeds less than about 1 percent of the speed of light.\\
2. Objects dealt with must be large enough to be seen with the naked eye.\\
3. Only weak gravitational fields are involved.\\
c. 1. Matter is moving at great speeds, comparable to the speed of light.\\
2. Objects dealt with are large enough to be seen with the naked eye.\\
3. Strong gravitational fields are involved.\\
d. 1. Matter is moving at great speeds, comparable to the speed of light.\\
2. Objects are just large enough to be visible through the most powerful telescope.\\
3. Only weak gravitational fields are involved.\\
4.

Why is the Greek word for nature appropriate in describing the field of physics?\\
a. Physics is a natural science that studies life and living organism on habitable planets like Earth.\\
b. Physics is a natural science that studies the laws and principles of our universe.\\
c. Physics is a physical science that studies the composition, structure, and changes of matter in our universe.\\
d. Physics is a social science that studies the social behavior of living beings on habitable planets like Earth.\\
5.

Which aspect of the universe is studied by quantum mechanics?\\
a. objects at the galactic level\\
b. objects at the classical level\\
c. objects at the subatomic level\\
d. objects at all levels, from subatomic to galactic

\section*{Teacher Support}
Teacher Support Use the Check Your Understanding questions to assess students' mastery of the sections learning objectives. If students are struggling with a specific objective, the Check Your Understanding will help identify the source of the problem and direct students to the relevant content.

\subsection*{1.2 The Scientific Methods}
\section*{Section Learning Objectives}
By the end of this section, you will be able to do the following:

\begin{itemize}
  \item Explain how the methods of science are used to make scientific discoveries
  \item Define a scientific model and describe examples of physical and mathematical models used in physics
  \item Compare and contrast hypothesis, theory, and law
\end{itemize}

\section*{Teacher Support}
Teacher Support The learning objectives in this section will help your students master the following standards:

\begin{itemize}
  \item (2) Scientific processes. The student uses a systematic approach to answer scientific laboratory and field investigative questions. The student is expected to:
  \item (A) know the definition of science and understand that it has limitations, as specified in subsection (b)(2) of this section;
  \item (B) know that scientific hypotheses are tentative and testable statements that must be capable of being supported or not supported by observational evidence. Hypotheses of durable explanatory power which have been tested over a wide variety of conditions are incorporated into theories;
  \item (C) know that scientific theories are based on natural and physical phenomena and are capable of being tested by multiple independent researchers. Unlike hypotheses, scientific theories are well-established and highly-reliable explanations, but may be subject to change as new areas of science and new technologies are developed;
  \item (D) distinguish between scientific hypotheses and scientific theories.
\end{itemize}

\section*{Section Key Terms}
\section*{Teacher Support}
Teacher Support [OL]Pre-assessment for this section could involve students sharing or writing down an anecdote about when they used the methods of science. Then, students could label their thought processes in their anecdote with the appropriate scientific methods. The class could also discuss their definitions of theory and law, both outside and within the context of science.\\[0pt]
[OL]It should be noted and possibly mentioned that a scientist, as mentioned in this section, does not necessarily mean a trained scientist. It could be anyone using methods of science.

\section*{Scientific Methods}
Scientists often plan and carry out investigations to answer questions about the universe around us. These investigations may lead to natural laws. Such laws are intrinsic to the universe, meaning that humans did not create them and cannot change them. We can only discover and understand them. Their discovery is a very human endeavor, with all the elements of mystery, imagination, struggle, triumph, and disappointment inherent in any creative effort. The cornerstone of discovering natural laws is observation. Science must describe the universe as it is, not as we imagine or wish it to be.

We all are curious to some extent. We look around, make generalizations, and try to understand what we see. For example, we look up and wonder whether one type of cloud signals an oncoming storm. As we become serious about exploring nature, we become more organized and formal in collecting and analyzing data. We attempt greater precision, perform controlled experiments (if we can), and write down ideas about how data may be organized. We then formulate models, theories, and laws based on the data we have collected, and communicate those results with others. This, in a nutshell, describes the scientific method that scientists employ to decide scientific issues on the basis of evidence from observation and experiment.

An investigation often begins with a scientist making an observation. The scientist observes a pattern or trend within the natural world. Observation may generate questions that the scientist wishes to answer. Next, the scientist may perform some research about the topic and devise a hypothesis. A hypothesis is a testable statement that describes how something in the natural world works. In essence, a hypothesis is an educated guess that explains something about an observation.

\section*{Teacher Support}
Teacher Support [OL] An educated guess is used throughout this section in describing a hypothesis to combat the tendency to think of a theory as an educated guess.

Scientists may test the hypothesis by performing an experiment. During an experiment, the scientist collects data that will help them learn about the phenomenon they are studying. Then the scientists analyze the results of the experiment (that is, the data), often using statistical, mathematical, and/or graphical methods. From the data analysis, they draw conclusions. They may conclude that their experiment either supports or rejects their hypothesis. If the hypothesis is supported, the scientist usually goes on to test another hypothesis related to the first. If their hypothesis is rejected, they will often then test a\\
new and different hypothesis in their effort to learn more about whatever they are studying.

Scientific processes can be applied to many situations. Let's say that you try to turn on your car, but it will not start. You have just made an observation! You ask yourself, "Why won't my car start?" You can now use scientific processes to answer this question. First, you generate a hypothesis such as, "The car won't start because it has no gasoline in the gas tank." To test this hypothesis, you put gasoline in the car and try to start it again. If the car starts, then your hypothesis is supported by the experiment. If the car does not start, then your hypothesis is rejected. You will then need to think up a new hypothesis to test such as, "My car won't start because the fuel pump is broken." Hopefully, your investigations lead you to discover why the car won't start and enable you to fix it.

\section*{Modeling}
A model is a representation of something that is often too difficult (or impossible) to study directly. Models can take the form of physical models, equations, computer programs, or simulations-computer graphics/animations. Models are tools that are especially useful in modern physics because they let us visualize phenomena that we normally cannot observe with our senses, such as very small objects or objects that move at high speeds. For example, we can understand the structure of an atom using models, despite the fact that no one has ever seen an atom with their own eyes. Models are always approximate, so they are simpler to consider than the real situation; the more complete a model is, the more complicated it must be. Models put the intangible or the extremely complex into human terms that we can visualize, discuss, and hypothesize about.

Scientific models are constructed based on the results of previous experiments. Even still, models often only describe a phenomenon partially or in a few limited situations. Some phenomena are so complex that they may be impossible to model them in their entirety, even using computers. An example is the electron cloud model of the atom in which electrons are moving around the atom's center in distinct clouds (Figure 1.12), that represent the likelihood of finding an electron in different places. This model helps us to visualize the structure of an atom. However, it does not show us exactly where an electron will be within its cloud at any one particular time.

\begin{figure}[h]
\begin{center}
  \includegraphics[max width=\textwidth]{8484573f-7913-4d1d-981f-d54fe53a1e32-25}
\captionsetup{labelformat=empty}
\caption{Figure 1.12 The electron cloud model of the atom predicts the geometry and shape of areas where different electrons may be found in an atom. However, it cannot indicate exactly where an electron will be at any one time.}
\end{center}
\end{figure}

As mentioned previously, physicists use a variety of models including equations, physical models, computer simulations, etc. For example, three-dimensional models are often commonly used in chemistry and physics to model molecules. Properties other than appearance or location are usually modelled using mathematics, where functions are used to show how these properties relate to one another. Processes such as the formation of a star or the planets, can also be modelled using computer simulations. Once a simulation is correctly programmed based on actual experimental data, the simulation can allow us to view processes that happened in the past or happen too quickly or slowly for us to observe directly. In addition, scientists can also run virtual experiments using computer-based models. In a model of planet formation, for example, the scientist could alter the amount or type of rocks present in space and see how it affects planet formation.

Scientists use models and experimental results to construct explanations of observations or design solutions to problems. For example, one way to make a car more fuel efficient is to reduce the friction or drag caused by air flowing around the moving car. This can be done by designing the body shape of the car to be more aerodynamic, such as by using rounded corners instead of sharp ones. Engineers can then construct physical models of the car body, place them in a wind tunnel, and examine the flow of air around the model. This can also be done mathematically in a computer simulation. The air flow pattern can be analyzed for regions smooth air flow and for eddies that indicate drag. The model of the car body may have to be altered slightly to produce the smoothest pattern of air flow (i.e., the least drag). The pattern with the least drag may be the solution to increasing fuel efficiency of the car. This solution might then be incorporated into the car design.

\section*{Snap Lab}
Using Models and the Scientific Processes Be sure to secure loose items before opening the window or door.

In this activity, you will learn about scientific models by making a model of how air flows through your classroom or a room in your house.

\begin{itemize}
  \item One room with at least one window or door that can be opened
  \item Piece of single-ply tissue paper
\end{itemize}

\begin{enumerate}
  \item Work with a group of four, as directed by your teacher. Close all of the windows and doors in the room you are working in. Your teacher may assign you a specific window or door to study.
  \item Before opening any windows or doors, draw a to-scale diagram of your room. First, measure the length and width of your room using the tape measure. Then, transform the measurement using a scale that could fit on your paper, such as 5 centimeters \(=1\) meter.
  \item Your teacher will assign you a specific window or door to study air flow. On your diagram, add arrows showing your hypothesis (before opening any windows or doors) of how air will flow through the room when your assigned window or door is opened. Use pencil so that you can easily make changes to your diagram.
  \item On your diagram, mark four locations where you would like to test air flow in your room. To test for airflow, hold a strip of single ply tissue paper between the thumb and index finger. Note the direction that the paper moves when exposed to the airflow. Then, for each location, predict which way the paper will move if your air flow diagram is correct.
  \item Now, each member of your group will stand in one of the four selected areas. Each member will test the airflow Agree upon an approximate height at which everyone will hold their papers.
  \item When you teacher tells you to, open your assigned window and/or door. Each person should note the direction that their paper points immediately after the window or door was opened. Record your results on your diagram.
  \item Did the airflow test data support or refute the hypothetical model of air flow shown in your diagram? Why or why not? Correct your model based on your experimental evidence.
  \item With your group, discuss how accurate your model is. What limitations did it have? Write down the limitations that your group agreed upon.
\end{enumerate}

Your diagram is a model, based on experimental evidence, of how air flows through the room. Could you use your model to predict how air would flow through a new window or door placed in a different location in the classroom? Make a new diagram that predicts the room's airflow with the addition of a new window or door. Add a short explanation that describes how.\\
a. Yes, you could use your model to predict air flow through a new window. The earlier experiment of air flow would help you model the system more accurately.\\
b. Yes, you could use your model to predict air flow through a new window. The earlier experiment of air flow is not useful for modeling the new system.\\
c. No, you cannot model a system to predict the air flow through a new window. The earlier experiment of air flow would help you model the system more accurately.\\
d. No, you cannot model a system to predict the air flow through a new window. The earlier experiment of air flow is not useful for modeling the new system.

\section*{Teacher Support}
Teacher Support This Snap Lab! has students construct a model of how air flows in their classroom. Each group of four students will create a model of air flow in their classroom using a scale drawing of the room. Then, the groups will test the validity of their model by placing weathervanes that they have constructed around the room and opening a window or door. By observing the weather vanes, students will see how air actually flows through the room from a specific window or door. Students will then correct their model based on their experimental evidence. The following material list is given per group:

\begin{itemize}
  \item One room with at least one window or door that can be opened (An optimal configuration would be one window or door per group.)
  \item Several pieces of construction paper (at least four per group)
  \item Strips of single ply tissue paper
  \item One tape measure (long enough to measure the dimensions of the room)
  \item Straws
  \item Scissors
  \item tape
\end{itemize}

\begin{enumerate}
  \item Group size can vary depending on the number of windows/doors available and the number of students in the class.
  \item The room dimensions could be provided by the teacher. Also, students may need a brief introduction in how to make a drawing to scale.
  \item This is another opportunity to discuss controlled experiments in terms of why the students should hold the strips of tissue paper at the same height and in the same way. One student could also serve as a control and stand far away from the window/door or in another area that will not receive air flow from the window/door.
  \item You will probably need to coordinate this when multiple windows or doors are used. Only one window or door should be opened at a time for best results. Between openings, allow a short period ( 5 minutes) when all\\
windows and doors are closed, if possible.\\
Answers to the Grasp Check will vary, but the air flow in the new window or door should be based on what the students observed in their experiment.
\end{enumerate}

\section*{Scientific Laws and Theories}
A scientific law is a description of a pattern in nature that is true in all circumstances that have been studied. That is, physical laws are meant to be universal, meaning that they apply throughout the known universe. Laws are often also concise, whereas theories are more complicated. A law can be expressed in the form of a single sentence or mathematical equation. For example, Newton's second law of motion, which relates the motion of an object to the force applied ( \(F\) ), the mass of the object ( \(m\) ), and the object's acceleration ( \(a\) ), is simply stated using the equation\\
\(F=m a\).\\
Scientific ideas and explanations that are true in many, but not all situations in the universe are usually called principles. An example is Pascal's principle, which explains properties of liquids, but not solids or gases. However, the distinction between laws and principles is sometimes not carefully made in science.

A theory is an explanation for patterns in nature that is supported by much scientific evidence and verified multiple times by multiple researchers. While many people confuse theories with educated guesses or hypotheses, theories have withstood more rigorous testing and verification than hypotheses.

\section*{Teacher Support}
Teacher Support [OL]Explain to students that in informal, everyday English the word theory can be used to describe an idea that is possibly true but that has not been proven to be true. This use of the word theory often leads people to think that scientific theories are nothing more than educated guesses. This is not just a misconception among students, but among the general public as well.

As a closing idea about scientific processes, we want to point out that scientific laws and theories, even those that have been supported by experiments for centuries, can still be changed by new discoveries. This is especially true when new technologies emerge that allow us to observe things that were formerly unobservable. Imagine how viewing previously invisible objects with a microscope or viewing Earth for the first time from space may have instantly changed our scientific theories and laws! What discoveries still await us in the future? The constant retesting and perfecting of our scientific laws and theories allows our knowledge of nature to progress. For this reason, many scientists are reluctant to say that their studies prove anything. By saying support instead of prove, it keeps the door open for future discoveries, even if they won't occur for centuries or even millennia.

\section*{Teacher Support}
Teacher Support [OL]With regard to scientists avoiding using the word prove, the general public knows that science has proven certain things such as that the heart pumps blood and the Earth is round. However, scientists should shy away from using prove because it is impossible to test every single instance and every set of conditions in a system to absolutely prove anything. Using support or similar terminology leaves the door open for further discovery.

\section*{Check Your Understanding}
6.

Explain why scientists sometimes use a model rather than trying to analyze the behavior of the real system.\\
a. Models are simpler to analyze.\\
b. Models give more accurate results.\\
c. Models provide more reliable predictions.\\
d. Models do not require any computer calculations.\\
7.

Describe the difference between a question, generated through observation, and a hypothesis.\\
a. They are the same.\\
b. A hypothesis has been thoroughly tested and found to be true.\\
c. A hypothesis is a tentative assumption based on what is already known.\\
d. A hypothesis is a broad explanation firmly supported by evidence.\\
8.

What is a scientific model and how is it useful?\\
a. A scientific model is a representation of something that can be easily studied directly. It is useful for studying things that can be easily analyzed by humans.\\
b. A scientific model is a representation of something that is often too difficult to study directly. It is useful for studying a complex system or systems that humans cannot observe directly.\\
c. A scientific model is a representation of scientific equipment. It is useful for studying working principles of scientific equipment.\\
d. A scientific model is a representation of a laboratory where experiments are performed. It is useful for studying requirements needed inside the laboratory.\\
9.

Which of the following statements is correct about the hypothesis?\\
a. The hypothesis must be validated by scientific experiments.\\
b. The hypothesis must not include any physical quantity.\\
c. The hypothesis must be a short and concise statement.\\
d. The hypothesis must apply to all the situations in the universe.\\
10.

What is a scientific theory?\\
a. A scientific theory is an explanation of natural phenomena that is supported by evidence.\\
b. A scientific theory is an explanation of natural phenomena without the support of evidence.\\
c. A scientific theory is an educated guess about the natural phenomena occurring in nature.\\
d. A scientific theory is an uneducated guess about natural phenomena occurring in nature.\\
11.

Compare and contrast a hypothesis and a scientific theory.\\
a. A hypothesis is an explanation of the natural world with experimental support, while a scientific theory is an educated guess about a natural phenomenon.\\
b. A hypothesis is an educated guess about natural phenomenon, while a scientific theory is an explanation of natural world with experimental support.\\
c. A hypothesis is experimental evidence of a natural phenomenon, while a scientific theory is an explanation of the natural world with experimental support.\\
d. A hypothesis is an explanation of the natural world with experimental support, while a scientific theory is experimental evidence of a natural phenomenon.

\section*{Teacher Support}
Teacher Support Use the Check Your Understanding questions to assess students' achievement of the section's learning objectives. If students are struggling with a specific objective, the Check Your Understanding will help identify which objective and direct students to the relevant content.

\subsection*{1.3 The Language of Physics: Physical Quantities and Units}
\section*{Section Learning Objectives}
By the end of this section, you will be able to do the following:

\begin{itemize}
  \item Associate physical quantities with their International System of Units (SI)and perform conversions among SI units using scientific notation
  \item Relate measurement uncertainty to significant figures and apply the rules for using significant figures in calculations
  \item Correctly create, label, and identify relationships in graphs using mathematical relationships (e.g., slope, \(y\)-intercept, inverse, quadratic and logarithmic)
\end{itemize}

\section*{Teacher Support}
Teacher Support The learning objectives in this section will help your students master the following standards:

\begin{itemize}
  \item (2) Scientific processes. The student uses a systematic approach to answer scientific laboratory and field investigative questions. The student is expected to
  \item (H) make measurements with accuracy and precision and record data using scientific notation and International System (SI) units;
  \item (L) express and manipulate relationships among physical variables quantitatively, including the use of graphs, charts, and equations.
\end{itemize}

In addition, the High School Physics Laboratory Manual addresses content in this section in the lab titled: Measurement, Precision and Accuracy, as well as the following standards:

\begin{itemize}
  \item (2) Scientific processes. The student uses a systematic approach to answer scientific laboratory and field investigative questions. The student is expected to:
  \item (H) make measurements with accuracy and precision and record data using scientific notation and International System (SI) units;
  \item (I) identify and quantify causes and effects of uncertainties in measured data;
  \item (J) organize and evaluate data and make inferences from data, including the use of tables, charts, and graphs.
\end{itemize}

\section*{Section Key Terms}
\section*{Teacher Support}
Teacher Support [OL]Pre-assessment for this section could involve asking students what experience they have had with the four fundamental units in their daily lives. One could also poll the class for what they think accuracy, precision, and uncertainty refer to. For graphing, students could make a quick graph of some data and then edit their graph after reading to note ways they could improve the clarity of their graph.

\section*{The Role of Units}
Physicists, like other scientists, make observations and ask basic questions. For example, how big is an object? How much mass does it have? How far did it travel? To answer these questions, they make measurements with various instruments (e.g., meter stick, balance, stopwatch, etc.).

The measurements of physical quantities are expressed in terms of units, which are standardized values. For example, the length of a race, which is a physical quantity, can be expressed in meters (for sprinters) or kilometers (for long distance runners). Without standardized units, it would be extremely difficult for scientists to express and compare measured values in a meaningful way (Figure 1.13).

\begin{figure}[h]
\begin{center}
  \includegraphics[max width=\textwidth]{8484573f-7913-4d1d-981f-d54fe53a1e32-32}
\captionsetup{labelformat=empty}
\caption{Figure 1.13 Distances given in unknown units are maddeningly useless.}
\end{center}
\end{figure}

All physical quantities in the International System of Units (SI) are expressed in terms of combinations of seven fundamental physical units, which are units for: length, mass, time, electric current, temperature, amount of a substance, and luminous intensity.

\section*{SI Units: Fundamental and Derived Units}
In any system of units, the units for some physical quantities must be defined through a measurement process. These are called the base quantities for that system and their units are the system's base units. All other physical quantities can then be expressed as algebraic combinations of the base quantities. Each of these physical quantities is then known as a derived quantity and each unit is called a derived unit. The choice of base quantities is somewhat arbitrary, as long as they are independent of each other and all other quantities can be derived from them. Typically, the goal is to choose physical quantities that can be measured accurately to a high precision as the base quantities. The reason for this is simple. Since the derived units can be expressed as algebraic combinations of the base units, they can only be as accurate and precise as the base units from which they are derived.

\section*{Teacher Support}
Teacher Support [OL]As a clarification, certain countries use the British system for a few of their measurements. For example, Britain still uses the pint to measure beer, miles to measure road distances, and pounds to measure body weight (although weight must be reported in kg in British medical records). The British people still use the British system extensively in their everyday lives, but the metric system is the official standard for the government. Likewise, many oil-producing countries measure oil in British gallons.

Based on such considerations, the International Standards Organization recommends using seven base quantities, which form the International System of Quantities (ISQ). These are the base quantities used to define the SI base units. (Table 1.1) lists these seven ISQ base quantities and the corresponding SI base units.

Table 1.1 SI Base Units

The Meter The SI unit for length is the meter (m). The definition of the meter has changed over time to become more accurate and precise. The meter was first defined in 1791 as \(1 / 10,000,000\) of the distance from the equator to the North Pole. This measurement was improved in 1889 by redefining the meter to be the distance between two engraved lines on a platinum-iridium bar. (The bar\\
is now housed at the International Bureau of Weights and Meaures, near Paris). By 1960, some distances could be measured more precisely by comparing them to wavelengths of light. The meter was redefined as \(1,650,763.73\) wavelengths of orange light emitted by krypton atoms. In 1983, the meter was given its present definition as the distance light travels in a vacuum in \(1 / 299,792,458\) of a second (Figure 1.14).

\begin{figure}[h]
\begin{center}
  \includegraphics[max width=\textwidth]{8484573f-7913-4d1d-981f-d54fe53a1e32-34}
\captionsetup{labelformat=empty}
\caption{Figure 1.14 The meter is defined to be the distance light travels in \(1 / 299,792,458\) of a second through a vacuum. Distance traveled is speed multiplied by time.}
\end{center}
\end{figure}

The Kilogram The SI unit for mass is the kilogram (abbreviated kg); it was previously defined to be the mass of a platinum-iridium cylinder kept with the old meter standard at the International Bureau of Weights and Measures near Paris. Exact replicas of the previously defined kilogram are also kept at the United States' National Institute of Standards and Technology, or NIST, located in Gaithersburg, Maryland outside of Washington D.C., and at other locations around the world. The determination of all other masses could be ultimately traced to a comparison with the standard mass. Even though the platinumiridium cylinder was resistant to corrosion, airborne contaminants were able to adhere to its surface, slightly changing its mass over time. In May 2019, the scientific community adopted a more stable definition of the kilogram. The kilogram is now defined in terms of the second, the meter, and Planck's constant, \(h\) (a quantum mechanical value that relates a photon's energy to its frequency).

The Second The SI unit for time, the second (s) also has a long history. For many years it was defined as \(1 / 86,400\) of an average solar day. However, the average solar day is actually very gradually getting longer due to gradual slowing of Earth's rotation. Accuracy in the fundamental units is essential, since all other measurements are derived from them. Therefore, a new standard was adopted to define the second in terms of a non-varying, or constant, physical phenomenon. One constant phenomenon is the very steady vibration of Cesium atoms, which can be observed and counted. This vibration forms the basis of the cesium atomic clock. In 1967, the second was redefined as the time required for \(9,192,631,770\) Cesium atom vibrations (Figure 1.15).

\begin{figure}[h]
\begin{center}
  \includegraphics[max width=\textwidth]{8484573f-7913-4d1d-981f-d54fe53a1e32-35}
\captionsetup{labelformat=empty}
\caption{Figure 1.15 An atomic clock such as this one uses the vibrations of cesium atoms to keep time to a precision of one microsecond per year. The fundamental unit of time, the second, is based on such clocks. This image is looking down from the top of an atomic clock. (Steve Jurvetson/Flickr)}
\end{center}
\end{figure}

\section*{Teacher Support}
Teacher Support [BL]An average solar day was used to originally define the second because the length of a solar day varies throughout the year due to Earth's tilt of its axis as well as its elliptical orbit. The accumulation of these variations could result in a day length difference of up to 16 minutes during different seasons. Using an average solar day resolves these variations in day length.

The Ampere Electric current is measured in the ampere (A), named after Andre Ampere. You have probably heard of amperes, or amps, when people discuss electrical currents or electrical devices. Understanding an ampere requires a basic understanding of electricity and magnetism, something that will be explored in depth in later chapters of this book. Basically, two parallel wires with an electric current running through them will produce an attractive force on each other. One ampere is defined as the amount of electric current that will produce an attractive force of \(2.7 \times 10^{-7}\) newton per meter of separation between the two wires (the newton is the derived unit of force).

\section*{Teacher Support}
Teacher Support [BL]Some students may not know that a vacuum is a region of space that contains no air.

Kelvins The SI unit of temperature is the kelvin (or kelvins, but not degrees kelvin). This scale is named after physicist William Thomson, Lord Kelvin, who was the first to call for an absolute temperature scale. The Kelvin scale is based on absolute zero. This is the point at which all thermal energy has been removed from all atoms or molecules in a system. This temperature, 0 K , is equal to \(-273.15{ }^{\circ} \mathrm{C}\) and \(-459.67^{\circ} \mathrm{F}\). Conveniently, the Kelvin scale actually changes in the same way as the Celsius scale. For example, the freezing point ( \(0{ }^{\circ} \mathrm{C}\) ) and boiling points of water ( \(100^{\circ} \mathrm{C}\) ) are 100 degrees apart on the Celsius scale. These two temperatures are also 100 kelvins apart (freezing point \(=273.15 \mathrm{~K}\); boiling point \(=373.15 \mathrm{~K}\) ).

Metric Prefixes Physical objects or phenomena may vary widely. For example, the size of objects varies from something very small (like an atom) to something very large (like a star). Yet the standard metric unit of length is the meter. So, the metric system includes many prefixes that can be attached to a unit. Each prefix is based on factors of \(10(10,100,1,000\), etc., as well as \(0.1,0.01,0.001\), etc.). Table 1.2 gives the metric prefixes and symbols used to denote the different various factors of 10 in the metric system.

Table 1.2 Metric Prefixes for Powers of 10 and Their Symbols [1]See Appendix A for a discussion of powers of 10 .\\
Note-Some examples are approximate.\\
The metric system is convenient because conversions between metric units can be done simply by moving the decimal place of a number. This is because the metric prefixes are sequential powers of 10 . There are 100 centimeters in a meter, 1000 meters in a kilometer, and so on. In nonmetric systems, such as U.S. customary units, the relationships are less simple - there are 12 inches in a foot, 5,280 feet in a mile, 4 quarts in a gallon, and so on. Another advantage of the metric system is that the same unit can be used over extremely large ranges of values simply by switching to the most-appropriate metric prefix. For example, distances in meters are suitable for building construction, but kilometers are used to describe road construction. Therefore, with the metric system, there is no need to invent new units when measuring very small or very large objectsyou just have to move the decimal point (and use the appropriate prefix).

Known Ranges of Length, Mass, and Time Table 1.3 lists known lengths, masses, and time measurements. You can see that scientists use a range of measurement units. This wide range demonstrates the vastness and complexity of the universe, as well as the breadth of phenomena physicists study. As you examine this table, note how the metric system allows us to discuss and compare an enormous range of phenomena, using one system of measurement (Figure 1.16 and Figure 1.17).

\begin{figure}[h]
\begin{center}
\captionsetup{labelformat=empty}
\caption{Table 1.3 Approximate Values of Length, Mass, and Time [1] More precise values are in parentheses.}
  \includegraphics[max width=\textwidth]{8484573f-7913-4d1d-981f-d54fe53a1e32-38}
\end{center}
\end{figure}

Figure 1.16 Tiny phytoplankton float among crystals of ice in the Antarctic Sea. They range from a few micrometers to as much as 2 millimeters in length. (Prof.

\begin{figure}[h]
\begin{center}
\captionsetup{labelformat=empty}
\caption{Gordon T. Taylor, Stony Brook University; NOAA Corps Collections)}
  \includegraphics[max width=\textwidth]{8484573f-7913-4d1d-981f-d54fe53a1e32-39}
\end{center}
\end{figure}

Figure 1.17 Galaxies collide 2.4 billion light years away from Earth. The tremendous range of observable phenomena in nature challenges the imagination. (NASA/CXC/UVic./A. Mahdavi et al. Optical/lensing: CFHT/UVic./H. Hoekstra et al.)

\section*{Using Scientific Notation with Physical Measurements}
Scientific notation is a way of writing numbers that are too large or small to be conveniently written as a decimal. For example, consider the number \(840,000,000,000,000\). It's a rather large number to write out. The scientific notation for this number is \(8.40 \times 10^{14}\). Scientific notation follows this general format\\
\(x \times 10^{y}\).\\
In this format \(x\) is the value of the measurement with all placeholder zeros removed. In the example above, \(x\) is 8.4. The \(x\) is multiplied by a factor, \(10^{y}\), which indicates the number of placeholder zeros in the measurement. Placeholder zeros are those at the end of a number that is 10 or greater, and at the beginning of a decimal number that is less than 1 . In the example above, the factor is \(10^{14}\). This tells you that you should move the decimal point 14 positions to the right, filling in placeholder zeros as you go. In this case, moving the decimal point 14 places creates only 13 placeholder zeros, indicating that the actual measurement value is \(840,000,000,000,000\).

Numbers that are fractions can be indicated by scientific notation as well. Consider the number 0.0000045 . Its scientific notation is \(4.5 \times 10^{-6}\). Its scientific notation has the same format\\
\(x \times 10^{y}\).\\
Here, \(x\) is 4.5. However, the value of \(y\) in the \(10^{y}\) factor is negative, which indicates that the measurement is a fraction of 1 . Therefore, we move the decimal place to the left, for a negative \(y\). In our example of \(4.5 \times 10^{-6}\), the decimal point would be moved to the left six times to yield the original number, which would be 0.0000045 .

The term order of magnitude refers to the power of 10 when numbers are expressed in scientific notation. Quantities that have the same power of 10 when expressed in scientific notation, or come close to it, are said to be of the same order of magnitude. For example, the number 800 can be written as \(8 \times 10^{2}\), and the number 450 can be written as \(4.5 \times 10^{2}\). Both numbers have the same value for \(y\). Therefore, 800 and 450 are of the same order of magnitude. Similarly, 101 and 99 would be regarded as the same order of magnitude, \(10^{2}\). Order of magnitude can be thought of as a ballpark estimate for the scale of a value. The diameter of an atom is on the order of \(10^{-9} \mathrm{~m}\), while the diameter of the sun is on the order of \(10^{9} \mathrm{~m}\). These two values are 18 orders of magnitude apart.

Scientists make frequent use of scientific notation because of the vast range of physical measurements possible in the universe, such as the distance from Earth to the moon (Figure 1.18), or to the nearest star.

\begin{figure}[h]
\begin{center}
  \includegraphics[max width=\textwidth]{8484573f-7913-4d1d-981f-d54fe53a1e32-41}
\captionsetup{labelformat=empty}
\caption{Figure 1.18 The distance from Earth to the moon may seem immense, but it is just a tiny fraction of the distance from Earth to our closest neighboring star. (NASA)}
\end{center}
\end{figure}

Unit Conversion and Dimensional Analysis It is often necessary to convert from one type of unit to another. For example, if you are reading a European cookbook in the United States, some quantities may be expressed in liters and you need to convert them to cups. A Canadian tourist driving through the United States might want to convert miles to kilometers, to have a sense of how far away his next destination is. A doctor in the United States might convert a patient's weight in pounds to kilograms.

Let's consider a simple example of how to convert units within the metric system. How can we convert 1 hour to seconds?

First, we need to determine a conversion factor. A conversion factor is a ratio expressing how many of one unit are equal to another unit. A conversion factor is simply a fraction which equals 1 . You can multiply any number by 1 and get the same value. When you multiply a number by a conversion factor, you are simply multiplying it by one. For example, the following are conversion factors: \((1\) foot \() /(12\) inches \()=1\) to convert inches to feet, ( 1 meter) \(/(100\) centimeters) \(=1\) to convert centimeters to meters, \((1\) minute \() /(60\) seconds \()=1\) to convert seconds to minutes.

Now we can set up our unit conversion. We will write the units that we have and then multiply them by the conversion factor \((1 \mathrm{~km} / 1,000 \mathrm{~m})=1\), so we are simply multiplying 80 m by 1 :\\
\(1 \mathrm{~h} \times \frac{60 \mathrm{~min}}{1 \mathrm{~h}} \times \frac{60 \mathrm{~s}}{1 \mathrm{~min}}=3600 \mathrm{~s}=3.6 \times 10^{3} \mathrm{~s}\)\\
1.1

When there is a unit in the original number, and a unit in the denominator (bottom) of the conversion factor, the units cancel. In this case, hours and minutes cancel and the value in seconds remains.

You can use this method to convert between any types of unit, including between the U.S. customary system and metric system. Notice also that, although you can multiply and divide units algebraically, you cannot add or subtract different units. An expression like \(10 \mathrm{~km}+5 \mathrm{~kg}\) makes no sense. Even adding two lengths in different units, such as \(10 k m+20 m\) does not make sense. You express both lengths in the same unit. See Appendix C for a more complete list of conversion factors.

\section*{Worked Example}
Unit Conversions: A Short Drive Home Suppose that you drive the 10.0 km from your university to home in 20.0 min . Calculate your average speed (a) in kilometers per hour ( \(\mathrm{km} / \mathrm{h}\) ) and (b) in meters per second (m/s). (NoteAverage speed is distance traveled divided by time of travel.)

\section*{Strategy}
First we calculate the average speed using the given units. Then we can get the average speed into the desired units by picking the correct conversion factor and multiplying by it. The correct conversion factor is the one that cancels the unwanted unit and leaves the desired unit in its place.

Solution for (a)

\begin{enumerate}
  \item Calculate average speed. Average speed is distance traveled divided by time of travel. (Take this definition as a given for now-average speed and other motion concepts will be covered in a later module.) In equation form,
\end{enumerate}

\begin{itemize}
  \item average speed \(=\frac{\text { distance }}{\text { time }}\).
\end{itemize}

\begin{enumerate}
  \setcounter{enumi}{1}
  \item Substitute the given values for distance and time.
\end{enumerate}

\begin{itemize}
  \item average speed \(=\frac{10.0 \mathrm{~km}}{20.0 \mathrm{~min}}=0.500 \frac{\mathrm{~km}}{\mathrm{~min}}\)
\end{itemize}

\begin{enumerate}
  \setcounter{enumi}{2}
  \item Convert \(\mathrm{km} / \min\) to \(\mathrm{km} / \mathrm{h}\) : multiply by the conversion factor that will cancel minutes and leave hours. That conversion factor is \(60 \mathrm{~min} / 1 \mathrm{~h}\). Thus,
\end{enumerate}

\begin{itemize}
  \item average speed \(=0.500 \frac{\mathrm{~km}}{\mathrm{~min}} \times \frac{60 \mathrm{~min}}{1 \mathrm{~h}}=30.0 \frac{\mathrm{~km}}{\mathrm{~h}}\).
\end{itemize}

\section*{Discussion for (a)}
To check your answer, consider the following:

\begin{enumerate}
  \item Be sure that you have properly cancelled the units in the unit conversion. If you have written the unit conversion factor upside down, the units will not cancel properly in the equation. If you accidentally get the ratio upside down, then the units will not cancel; rather, they will give you the wrong units as follows
\end{enumerate}

\begin{itemize}
  \item \(\frac{\mathrm{km}}{\min } \times \frac{1 \mathrm{hr}}{60 \mathrm{~min}}=\frac{1}{60} \frac{\mathrm{~km} \cdot \mathrm{~h}}{\mathrm{~min}^{2}}\),\\
which are obviously not the desired units of \(\mathrm{km} / \mathrm{h}\).
\end{itemize}

\begin{enumerate}
  \setcounter{enumi}{1}
  \item Check that the units of the final answer are the desired units. The problem asked us to solve for average speed in units of \(\mathrm{km} / \mathrm{h}\) and we have indeed obtained these units.
  \item Check the significant figures. Because each of the values given in the problem has three significant figures, the answer should also have three significant figures. The answer \(30.0 \mathrm{~km} / \mathrm{h}\) does indeed have three significant figures, so this is appropriate. Note that the significant figures in the conversion factor are not relevant because an hour is defined to be 60 min , so the precision of the conversion factor is perfect.
  \item Next, check whether the answer is reasonable. Let us consider some information from the problem-if you travel 10 km in a third of an hour ( 20 min), you would travel three times that far in an hour. The answer does seem reasonable.
\end{enumerate}

\section*{Solution (b)}
There are several ways to convert the average speed into meters per second.

\begin{enumerate}
  \item Start with the answer to (a) and convert \(\mathrm{km} / \mathrm{h}\) to \(\mathrm{m} / \mathrm{s}\). Two conversion factors are needed-one to convert hours to seconds, and another to convert kilometers to meters.
  \item Multiplying by these yields
\end{enumerate}

\begin{itemize}
  \item Averagespeed \(=30.0 \frac{\mathrm{~km}}{\mathrm{~h}} \times \frac{1 \mathrm{~h}}{3,600 \mathrm{~s}} \times \frac{1,000 \mathrm{~m}}{1 \mathrm{~km}}\)
\end{itemize}

Averagespeed \(=8.33 \frac{\mathrm{~m}}{\mathrm{~s}}\)

\section*{Discussion for (b)}
If we had started with \(0.500 \mathrm{~km} / \mathrm{min}\), we would have needed different conversion factors, but the answer would have been the same: \(8.33 \mathrm{~m} / \mathrm{s}\).

You may have noted that the answers in the worked example just covered were given to three digits. Why? When do you need to be concerned about the number of digits in something you calculate? Why not write down all the digits your calculator produces?

\section*{Worked Example}
Using Physics to Evaluate Promotional Materials A commemorative coin that is 2 in diameter is advertised to be plated with 15 mg of gold. If the density of gold is \(19.3 \mathrm{~g} / \mathrm{cc}\), and the amount of gold around the edge of the coin can be ignored, what is the thickness of the gold on the top and bottom faces of the coin?

\section*{Strategy}
To solve this problem, the volume of the gold needs to be determined using the gold's mass and density. Half of that volume is distributed on each face of the coin, and, for each face, the gold can be represented as a cylinder that is 2 in diameter with a height equal to the thickness. Use the volume formula for a cylinder to determine the thickness.

Solution\\
The mass of the gold is given by the formula\\
\(m=\rho V=15 \times 10^{-3} \mathrm{~g}\),\\
where\\
\(\rho=19.3 \mathrm{~g} / \mathrm{cc}\)\\
and \(V\) is the volume. Solving for the volume gives\\
\(V=\frac{m}{\rho}=\frac{15 \times 10^{-3} \mathrm{~g}}{19.3 \mathrm{~g} / \mathrm{cc}} \cong 7.8 \times 10^{-4} \mathrm{cc}\).\\
If \(t\) is the thickness, the volume corresponding to half the gold is\\
\(\frac{1}{2}\left(7.8 \times 10^{-4}\right)=\pi r^{2} t=\pi(2.54)^{2} t\),\\
where the 1 radius has been converted to cm . Solving for the thickness gives\\
\(t=\frac{\left(3.9 \times 10^{-4}\right)}{\pi(2.54)^{2}} \cong 1.9 \times 10^{-5} \mathrm{~cm}=0.00019 \mathrm{~mm}\).\\
Discussion\\
The amount of gold used is stated to be 15 mg , which is equivalent to a thickness of about 0.00019 mm . The mass figure may make the amount of gold sound larger, both because the number is much bigger ( 15 versus 0.00019 ), and because people may have a more intuitive feel for how much a millimeter is than for how much a milligram is. A simple analysis of this sort can clarify the significance of claims made by advertisers.

\section*{Teacher Support}
Teacher Support Ask students to find other promotional materials that make claims that can be analyzed using physics principles. Compile any items\\
that come in for later use at appropriate points in the course. For example, after covering power consumption in electric circuits, compare the performance of electric fireplaces advertised as revolutionary to the performance of standard space heaters.

\section*{Accuracy, Precision and Significant Figures}
Science is based on experimentation that requires good measurements. The validity of a measurement can be described in terms of its accuracy and its precision (see Figure 1.19 and Figure 1.20). Accuracy is how close a measurement is to the correct value for that measurement. For example, let us say that you are measuring the length of standard piece of printer paper. The packaging in which you purchased the paper states that it is 11 inches long, and suppose this stated value is correct. You measure the length of the paper three times and obtain the following measurements: 11.1 inches, 11.2 inches, and 10.9 inches. These measurements are quite accurate because they are very close to the correct value of 11.0 inches. In contrast, if you had obtained a measurement of 12 inches, your measurement would not be very accurate. This is why measuring instruments are calibrated based on a known measurement. If the instrument consistently returns the correct value of the known measurement, it is safe for use in finding unknown values.

\begin{figure}[h]
\begin{center}
  \includegraphics[max width=\textwidth]{8484573f-7913-4d1d-981f-d54fe53a1e32-45}
\captionsetup{labelformat=empty}
\caption{Figure 1.19 A double-pan mechanical balance is used to compare different masses. Usually an object with unknown mass is placed in one pan and objects}
\end{center}
\end{figure}

of known mass are placed in the other pan. When the bar that connects the two pans is horizontal, then the masses in both pans are equal. The known masses are typically metal cylinders of standard mass such as 1 gram, 10 grams, and 100 grams. (Serge Melki)

\begin{figure}[h]
\begin{center}
  \includegraphics[max width=\textwidth]{8484573f-7913-4d1d-981f-d54fe53a1e32-46}
\captionsetup{labelformat=empty}
\caption{Figure 1.20 Whereas a mechanical balance may only read the mass of an object to the nearest tenth of a gram, some digital scales can measure the mass of an object up to the nearest thousandth of a gram. As in other measuring devices, the precision of a scale is limited to the last measured figures. This is the hundredths place in the scale pictured here. (Splarka, Wikimedia Commons)}
\end{center}
\end{figure}

Precision states how well repeated measurements of something generate the same or similar results. Therefore, the precision of measurements refers to how close together the measurements are when you measure the same thing several times. One way to analyze the precision of measurements would be to determine the range, or difference between the lowest and the highest measured values. In the case of the printer paper measurements, the lowest value was 10.9 inches and the highest value was 11.2 inches. Thus, the measured values deviated from each other by, at most, 0.3 inches. These measurements were reasonably precise because they varied by only a fraction of an inch. However, if the measured values had been 10.9 inches, 11.1 inches, and 11.9 inches, then the measurements would not be very precise because there is a lot of variation\\
from one measurement to another.\\
The measurements in the paper example are both accurate and precise, but in some cases, measurements are accurate but not precise, or they are precise but not accurate. Let us consider a GPS system that is attempting to locate the position of a restaurant in a city. Think of the restaurant location as existing at the center of a bull's-eye target. Then think of each GPS attempt to locate the restaurant as a black dot on the bull's eye.

In Figure 1.21, you can see that the GPS measurements are spread far apart from each other, but they are all relatively close to the actual location of the restaurant at the center of the target. This indicates a low precision, high accuracy measuring system. However, in Figure 1.22, the GPS measurements are concentrated quite closely to one another, but they are far away from the target location. This indicates a high precision, low accuracy measuring system. Finally, in Figure 1.23, the GPS is both precise and accurate, allowing the restaurant to be located.

\begin{figure}[h]
\begin{center}
  \includegraphics[max width=\textwidth]{8484573f-7913-4d1d-981f-d54fe53a1e32-48}
\captionsetup{labelformat=empty}
\caption{Figure 1.21 A GPS system attempts to locate a restaurant at the center of the bull's-eye. The black dots represent each attempt to pinpoint the location of the restaurant. The dots are spread out quite far apart from one another, indicating low precision, but they are each rather close to the actual location of the restaurant, indicating high accuracy. (Dark Evil)}
\end{center}
\end{figure}

\begin{figure}[h]
\begin{center}
  \includegraphics[max width=\textwidth]{8484573f-7913-4d1d-981f-d54fe53a1e32-49}
\captionsetup{labelformat=empty}
\caption{Figure 1.22 In this figure, the dots are concentrated close to one another, indicating high precision, but they are rather far away from the actual location of the restaurant, indicating low accuracy. (Dark Evil)}
\end{center}
\end{figure}

\begin{figure}[h]
\begin{center}
  \includegraphics[max width=\textwidth]{8484573f-7913-4d1d-981f-d54fe53a1e32-50}
\captionsetup{labelformat=empty}
\caption{Figure 1.23 In this figure, the dots are concentrated close to one another, indicating high precision, and they are very close to the actual location of the restaurant, indicating high accuracy. (Dark Evil)}
\end{center}
\end{figure}

Uncertainty The accuracy and precision of a measuring system determine the uncertainty of its measurements. Uncertainty is a way to describe how\\
much your measured value deviates from the actual value that the object has. If your measurements are not very accurate or precise, then the uncertainty of your values will be very high. In more general terms, uncertainty can be thought of as a disclaimer for your measured values. For example, if someone asked you to provide the mileage on your car, you might say that it is 45,000 miles, plus or minus 500 miles. The plus or minus amount is the uncertainty in your value. That is, you are indicating that the actual mileage of your car might be as low as 44,500 miles or as high as 45,500 miles, or anywhere in between. All measurements contain some amount of uncertainty. In our example of measuring the length of the paper, we might say that the length of the paper is 11 inches plus or minus 0.2 inches or \(11.0 \pm 0.2\) inches. The uncertainty in a measurement, \(A\), is often denoted as \(A\) ("delta \(A\) "),

The factors contributing to uncertainty in a measurement include the following:

\begin{enumerate}
  \item Limitations of the measuring device
  \item The skill of the person making the measurement
  \item Irregularities in the object being measured
  \item Any other factors that affect the outcome (highly dependent on the situation)
\end{enumerate}

In the printer paper example uncertainty could be caused by: the fact that the smallest division on the ruler is 0.1 inches, the person using the ruler has bad eyesight, or uncertainty caused by the paper cutting machine (e.g., one side of the paper is slightly longer than the other.) It is good practice to carefully consider all possible sources of uncertainty in a measurement and reduce or eliminate them,

Percent Uncertainty One method of expressing uncertainty is as a percent of the measured value. If a measurement, \(A\), is expressed with uncertainty, \(A\), the percent uncertainty is\\
\(\%\) uncertainty \(=\frac{\delta \mathrm{A}}{\mathrm{A}} \times 100 \%\).\\
1.2

\section*{Worked Example}
Calculating Percent Uncertainty: A Bag of Apples A grocery store sells 5 -lb bags of apples. You purchase four bags over the course of a month and weigh the apples each time. You obtain the following measurements:

\begin{itemize}
  \item Week 1 weight: 4.8 lb
  \item Week 2 weight: 5.3 lb
  \item Week 3 weight: 4.9 lb
  \item Week 4 weight: 5.4 lb
\end{itemize}

You determine that the weight of the 5 lb bag has an uncertainty of \(\pm 0.4 \mathrm{lb}\). What is the percent uncertainty of the bag's weight?

\section*{Strategy}
First, observe that the expected value of the bag's weight, \(A\), is 5 lb . The uncertainty in this value, \(\delta A\), is 0.4 lb . We can use the following equation to determine the percent uncertainty of the weight\\
\(\%\) uncertainty \(=\frac{\delta \mathrm{A}}{\mathrm{A}} \times 100 \%\).\\
Solution\\
Plug the known values into the equation\\
\(\%\) uncertainty \(=\frac{0.4 \mathrm{lb}}{5 \mathrm{lb}} \times 100 \%=8 \%\).\\
Discussion\\
We can conclude that the weight of the apple bag is \(5 \mathrm{lb} \pm 8\) percent. Consider how this percent uncertainty would change if the bag of apples were half as heavy, but the uncertainty in the weight remained the same. Hint for future calculations: when calculating percent uncertainty, always remember that you must multiply the fraction by 100 percent. If you do not do this, you will have a decimal quantity, not a percent value.

Uncertainty in Calculations There is an uncertainty in anything calculated from measured quantities. For example, the area of a floor calculated from measurements of its length and width has an uncertainty because the both the length and width have uncertainties. How big is the uncertainty in something you calculate by multiplication or division? If the measurements in the calculation have small uncertainties (a few percent or less), then the method of adding percents can be used. This method says that the percent uncertainty in a quantity calculated by multiplication or division is the sum of the percent uncertainties in the items used to make the calculation. For example, if a floor has a length of 4.00 m and a width of 3.00 m , with uncertainties of 2 percent and 1 percent, respectively, then the area of the floor is \(12.0 \mathrm{~m}^{2}\) and has an uncertainty of 3 percent (expressed as an area this is \(0.36 \mathrm{~m}^{2}\), which we round to \(0.4 \mathrm{~m}^{2}\) since the area of the floor is given to a tenth of a square meter).

For more information on the accuracy, precision, and uncertainty of measurements based upon the units of measurement, visit this website.

Precision of Measuring Tools and Significant Figures An important factor in the accuracy and precision of measurements is the precision of the measuring tool. In general, a precise measuring tool is one that can measure values in very small increments. For example, consider measuring the thickness of a coin. A standard ruler can measure thickness to the nearest millimeter, while a micrometer can measure the thickness to the nearest 0.005 millimeter. The micrometer is a more precise measuring tool because it can measure extremely small differences in thickness. The more precise the measuring tool, the more precise and accurate the measurements can be.

When we express measured values, we can only list as many digits as we initially measured with our measuring tool (such as the rulers shown in Figure 1.24). For example, if you use a standard ruler to measure the length of a stick, you may measure it with a decimeter ruler as 3.6 cm . You could not express this value as 3.65 cm because your measuring tool was not precise enough to measure a hundredth of a centimeter. It should be noted that the last digit in a measured value has been estimated in some way by the person performing the measurement. For example, the person measuring the length of a stick with a ruler notices that the stick length seems to be somewhere in between 36 mm and 37 mm. He or she must estimate the value of the last digit. The rule is that the last digit written down in a measurement is the first digit with some uncertainty. For example, the last measured value 36.5 mm has three digits, or three significant figures. The number of significant figures in a measurement indicates the precision of the measuring tool. The more precise a measuring tool is, the greater the number of significant figures it can report.

\section*{0.3 decimeters}
\begin{center}
\includegraphics[max width=\textwidth]{8484573f-7913-4d1d-981f-d54fe53a1e32-53(1)}
\end{center}

\section*{3.6 centimeters}
\begin{center}
\includegraphics[max width=\textwidth]{8484573f-7913-4d1d-981f-d54fe53a1e32-53}
\end{center}

\section*{36.5 millimeters}
\texttt{https://cdn.mathpix.com/cropped/8484573f-7913-4d1d-981f-d54fe53a1e32-53.jpg?height=140&width=1223&top_left_y=1680&top_left_x=464}

Figure 1.24 Three metric rulers are shown. The first ruler is in decimeters and can measure point three decimeters. The second ruler is in centimeters long and can measure three point six centimeters. The last ruler is in millimeters and can measure thirty-six point five millimeters.

Zeros Special consideration is given to zeros when counting significant figures. For example, the zeros in 0.053 are not significant because they are only placeholders that locate the decimal point. There are two significant figures in 0.053 -the 5 and the 3 . However, if the zero occurs between other significant figures, the zeros are significant. For example, both zeros in 10.053 are significant, as these zeros were actually measured. Therefore, the 10.053 placeholder\\
has five significant figures. The zeros in 1300 may or may not be significant, depending on the style of writing numbers. They could mean the number is known to the last zero, or the zeros could be placeholders. So 1300 could have two, three, or four significant figures. To avoid this ambiguity, write 1300 in scientific notation as \(1.3 \times 10^{3}\). Only significant figures are given in the \(x\) factor for a number in scientific notation (in the form \(x \times 10^{y}\) ). Therefore, we know that 1 and 3 are the only significant digits in this number. In summary, zeros are significant except when they serve only as placeholders. Table 1.4 provides examples of the number of significant figures in various numbers.

Table 1.4

Significant Figures in Calculations When combining measurements with different degrees of accuracy and precision, the number of significant digits in the final answer can be no greater than the number of significant digits in the least precise measured value. There are two different rules, one for multiplication and division and another rule for addition and subtraction, as discussed below.

\begin{enumerate}
  \item For multiplication and division: The answer should have the same number of significant figures as the starting value with the fewest significant figures. For example, the area of a circle can be calculated from its radius using \(A=\pi r^{2}\). Let us see how many significant figures the area will have if the radius has only two significant figures, for example, \(r=\) 2.0 m . Then, using a calculator that keeps eight significant figures, you would get
\end{enumerate}

\begin{itemize}
  \item \(A=r^{2}=(3.1415927 \ldots) \times(2.0 \mathrm{~m})^{2}=4.5238934 \mathrm{~m}^{2}\).
\end{itemize}

But because the radius has only two significant figures, the area calculated is meaningful only to two significant figures or\\
\(A=4.5 \mathrm{~m}^{2}\)\\
even though the value of \(\pi\) is meaningful to at least eight digits.\\
2. For addition and subtraction: The answer should have the same number places (e.g. tens place, ones place, tenths place, etc.) as the leastprecise starting value. Suppose that you buy 7.56 kg of potatoes in a\\
grocery store as measured with a scale having a precision of 0.01 kg . Then you drop off 6.052 kg of potatoes at your laboratory as measured by a scale with a precision of 0.001 kg . Finally, you go home and add 13.7 kg of potatoes as measured by a bathroom scale with a precision of 0.1 kg. How many kilograms of potatoes do you now have, and how many significant figures are appropriate in the answer? The mass is found by simple addition and subtraction:

\[
\begin{array}{rl}
7.56 & \mathrm{~kg} \\
-6.052 & \mathrm{~kg} \\
+13.7 & \mathrm{~kg} \\
\hline
\end{array}
\]

. 15.208 kg\\
The least precise measurement is 13.7 kg . This measurement is expressed to the 0.1 decimal place, so our final answer must also be expressed to the 0.1 decimal place. Thus, the answer should be rounded to the tenths place, giving 15.2 kg . The same is true for non-decimal numbers. For example, \(6527.23+2=6529.23=6529\).\\
We cannot report the decimal places in the answer because 2 has no decimal places that would be significant. Therefore, we can only report to the ones place.\\
It is a good idea to keep extra significant figures while calculating, and to round off to the correct number of significant figures only in the final answers. The reason is that small errors from rounding while calculating can sometimes produce significant errors in the final answer. As an example, try calculating \(5,098-(5.000) \times(1,010)\) to obtain a final answer to only two significant figures. Keeping all significant during the calculation gives 48 . Rounding to two significant figures in the middle of the calculation changes it to \(5,100-(5.000) \times(1,000)=100\), which is way off. You would similarly avoid rounding in the middle of the calculation in counting and in doing accounting, where many small numbers need to be added and subtracted accurately to give possibly much larger final numbers.

\section*{Teacher Support}
Teacher Support Remind students that they will be expected to report the proper number of significant figures on assignment and test problems.

Significant Figures in this Text In this textbook, most numbers are assumed to have three significant figures. Furthermore, consistent numbers of significant figures are used in all worked examples. You will note that an answer given to three digits is based on input good to at least three digits. If the input has fewer significant figures, the answer will also have fewer significant figures. Care is also taken that the number of significant figures is reasonable for the situation posed. In some topics, such as optics, more than three significant\\
figures will be used. Finally, if a number is exact, such as the 2 in the formula, \(c=2 \pi r\), it does not affect the number of significant figures in a calculation.

\section*{Worked Example}
Approximating Vast Numbers: a Trillion Dollars The U.S. federal deficit in the 2008 fiscal year was a little greater than \(\$ 10\) trillion. Most of us do not have any concept of how much even one trillion actually is. Suppose that you were given a trillion dollars in \(\$ 100\) bills. If you made 100 -bill stacks, like that shown in Figure 1.25, and used them to evenly cover a football field (between the end zones), make an approximation of how high the money pile would become. (We will use feet/inches rather than meters here because football fields are measured in yards.) One of your friends says 3 in ., while another says 10 ft . What do you think?

\begin{figure}[h]
\begin{center}
  \includegraphics[max width=\textwidth]{8484573f-7913-4d1d-981f-d54fe53a1e32-56}
\captionsetup{labelformat=empty}
\caption{Figure 1.25 A bank stack contains one hundred \(\$ 100\) bills, and is worth \(\$ 10,000\). How many bank stacks make up a trillion dollars? (Andrew Magill)}
\end{center}
\end{figure}

\section*{Strategy}
When you imagine the situation, you probably envision thousands of small stacks of 100 wrapped \(\$ 100\) bills, such as you might see in movies or at a bank. Since this is an easy-to-approximate quantity, let us start there. We can find the volume of a stack of 100 bills, find out how many stacks make up one trillion dollars, and then set this volume equal to the area of the football field multiplied by the unknown height.

Solution

\begin{enumerate}
  \item Calculate the volume of a stack of 100 bills. The dimensions of a single bill are approximately 3 in . by 6 in . A stack of 100 of these is about 0.5 in. thick. So the total volume of a stack of 100 bills is\\
volume of stack \(=\) length × width × height,\\
volume of stack \(=6 \mathrm{in} . \times 3 \mathrm{in} . \times 0.5 \mathrm{in}\).,
\end{enumerate}

\begin{itemize}
  \item volume of stack \(=9 \mathrm{in} .^{3}\).
\end{itemize}

\begin{enumerate}
  \setcounter{enumi}{1}
  \item Calculate the number of stacks. Note that a trillion dollars is equal to \(\$ 1 \times 10^{12}\), and a stack of one-hundred \(\$ 100\) bills is equal to \(\$ 10,000\), or \(\$ 1 \times 10^{4}\). The number of stacks you will have is
\end{enumerate}

\begin{itemize}
  \item \(\$ 1 \times 10^{12}\) (a trillion dollars) / \(\$ 1 \times 10^{4}\) per stack \(=1 \times 10^{8}\) stacks.\\
1.3
\end{itemize}

\begin{enumerate}
  \setcounter{enumi}{2}
  \item Calculate the area of a football field in square inches. The area of a football field is \(100 \mathrm{yd} \times 50 \mathrm{yd}\), which gives \(5,000 \mathrm{yd}^{2}\). Because we are working in inches, we need to convert square yards to square inches\\
Area \(=5,000 \mathrm{yd}^{2} \times \frac{3 \mathrm{ft}}{1 \mathrm{yd}} \times \frac{3 \mathrm{ft}}{1 \mathrm{yd}} \times \frac{12 \mathrm{in} .}{1 \text { foot }} \times \frac{12 \mathrm{in} .}{1 \text { foot }}=6,480,000 \mathrm{in} .^{2}\),
\end{enumerate}

\begin{itemize}
  \item Area \(\approx 6 \times 10^{6}\) in. \({ }^{2}\).
\end{itemize}

This conversion gives us \(6 \times 10^{6} \mathrm{in}^{2}\) for the area of the field. (Note that we are using only one significant figure in these calculations.)\\
4. Calculate the total volume of the bills. The volume of all the \(\$ 100\)-bill stacks is

\begin{itemize}
  \item 9 in. \({ }^{3} /\) stack \(\times 10^{8}\) stacks \(=9 \times 10^{8}\) in. \(^{3}\)
\end{itemize}

\begin{enumerate}
  \setcounter{enumi}{4}
  \item Calculate the height. To determine the height of the bills, use the following equation\\
volume of bills \(=\) area of field × height of money\\
Height of money \(=\quad \frac{\text { volume of bills }}{\text { area of field }}\)\\
Height of money \(=\quad \frac{9 \times 10^{8} \mathrm{in} .^{3}}{6 \times 10^{6} \mathrm{in} .^{2}}=1.33 \times 10^{2} \mathrm{in}\).\\
. Height of money \(=\quad 1 \times 10^{2}\) in. \(=100\) in.
\end{enumerate}

The height of the money will be about 100 in . high. Converting this value to feet gives

\[
100 \mathrm{in} . \times \frac{1 \mathrm{ft}}{12 \mathrm{in} .}=8.33 \mathrm{ft} \approx 8 \mathrm{ft} .
\]

Discussion\\
The final approximate value is much higher than the early estimate of 3 in ., but the other early estimate of \(10 \mathrm{ft}(120 \mathrm{in}\).\() was roughly correct. How did the\) approximation measure up to your first guess? What can this exercise tell you in terms of rough guesstimates versus carefully calculated approximations?

In the example above, the final approximate value is much higher than the first friend's early estimate of 3 in . However, the other friend's early estimate of 10 ft . ( 120 in .) was roughly correct. How did the approximation measure up to your first guess? What can this exercise suggest about the value of rough guesstimates versus carefully calculated approximations?

\section*{Teacher Support}
Teacher Support In [link], point out to students the importance of precision in their measurements. Greater precision allows measurements to be less uncertain, and therefore, a close approximation rather than a guessitimate.

\section*{Graphing in Physics}
Most results in science are presented in scientific journal articles using graphs. Graphs present data in a way that is easy to visualize for humans in general, especially someone unfamiliar with what is being studied. They are also useful for presenting large amounts of data or data with complicated trends in an easily-readable way.

One commonly-used graph in physics and other sciences is the line graph, probably because it is the best graph for showing how one quantity changes in response to the other. Let's build a line graph based on the data in Table 1.5, which shows the measured distance that a train travels from its station versus time. Our two variables, or things that change along the graph, are time in minutes, and distance from the station, in kilometers. Remember that measured data may not have perfect accuracy.

\section*{Table 1.5}
\begin{enumerate}
  \item Draw the two axes. The horizontal axis, or \(x\)-axis, shows the independent variable, which is the variable that is controlled or manipulated. The vertical axis, or \(y\)-axis, shows the dependent variable, the non-manipulated variable that changes with (or is dependent on) the value of the independent variable. In the data above, time is the independent variable and should be plotted on the \(x\)-axis. Distance from the station is the dependent variable and should be plotted on the \(y\)-axis.
  \item Label each axes on the graph with the name of each variable, followed by the symbol for its units in parentheses. Be sure to leave room so that you can number each axis. In this example, use Time (min) as the label for the \(x\)-axis.
  \item Next, you must determine the best scale to use for numbering each axis. Because the time values on the \(x\)-axis are taken every 10 minutes, we could easily number the \(x\)-axis from 0 to 70 minutes with a tick mark every 10 minutes. Likewise, the \(y\)-axis scale should start low enough and continue high enough to include all of the distance from station values. A scale from 0 km to 160 km should suffice, perhaps with a tick mark every 10 km .
\end{enumerate}

\begin{itemize}
  \item In general, you want to pick a scale for both axes that 1 ) shows all of your data, and 2) makes it easy to identify trends in your data. If you make your scale too large, it will be harder to see how your data change. Likewise, the smaller and more fine you make your scale, the more space you will need to make the graph. The number of significant figures in the axis values should be coarser than the number of significant figures in the measurements.
\end{itemize}

\begin{enumerate}
  \setcounter{enumi}{3}
  \item Now that your axes are ready, you can begin plotting your data. For the first data point, count along the \(x\)-axis until you find the 10 min tick mark. Then, count up from that point to the 10 km tick mark on the \(y\)-axis, and approximate where 22 km is along the \(y\)-axis. Place a dot at this location. Repeat for the other six data points (Figure 1.26).
\end{enumerate}

\begin{figure}[h]
\begin{center}
  \includegraphics[max width=\textwidth]{8484573f-7913-4d1d-981f-d54fe53a1e32-60}
\captionsetup{labelformat=empty}
\caption{Figure 1.26 The graph of the train's distance from the station versus time from the exercise above.}
\end{center}
\end{figure}

\begin{enumerate}
  \setcounter{enumi}{4}
  \item Add a title to the top of the graph to state what the graph is describing, such as the \(y\)-axis parameter vs. the \(x\)-axis parameter. In the graph shown here, the title is train motion. It could also be titled distance of the train from the station vs. time.
  \item Finally, with data points now on the graph, you should draw a trend line (Figure 1.27). The trend line represents the dependence you think the graph represents, so that the person who looks at your graph can see how close it is to the real data. In the present case, since the data points look like they ought to fall on a straight line, you would draw a straight line as the trend line. Draw it to come closest to all the points. Real data may have some inaccuracies, and the plotted points may not all fall on the trend line. In some cases, none of the data points fall exactly on the trend line.
\end{enumerate}

\begin{figure}[h]
\begin{center}
  \includegraphics[max width=\textwidth]{8484573f-7913-4d1d-981f-d54fe53a1e32-61}
\captionsetup{labelformat=empty}
\caption{Figure 1.27 The completed graph with the trend line included.}
\end{center}
\end{figure}

\section*{Teacher Support}
Teacher Support [OL]The importance of bar graphs should also be mentioned as a useful way to show data relations when one variable is not continuous, such as in a frequency histogram, which compares how many data points fall into discrete categories.\\[0pt]
[OL]If students have difficulty understanding the difference between dependent and independent variables in the train example, explain that time is independent because it will continue to move forward at the same rate whether the train leaves the station or not.

Analyzing a Graph Using Its Equation One way to get a quick snapshot of a dataset is to look at the equation of its trend line. If the graph produces a straight line, the equation of the trend line takes the form\\
\(y=m x+b\).\\
The \(b\) in the equation is the \(y\)-intercept while the \(m\) in the equation is the slope. The \(y\)-intercept tells you at what \(y\) value the line intersects the \(y\)-axis. In the case of the graph above, the \(y\)-intercept occurs at 0 , at the very beginning of the graph. The \(y\)-intercept, therefore, lets you know immediately where on the \(y\)-axis the plot line begins.

The \(m\) in the equation is the slope. This value describes how much the line on\\
the graph moves up or down on the \(y\)-axis along the line's length. The slope is found using the following equation\\
\(m=\frac{Y_{2}-Y_{1}}{X_{2}-X_{1}}\).\\
In order to solve this equation, you need to pick two points on the line (preferably far apart on the line so the slope you calculate describes the line accurately). The quantities \(Y_{2}\) and \(Y_{1}\) represent the \(y\)-values from the two points on the line (not data points) that you picked, while \(X_{2}\) and \(X_{1}\) represent the two \(x\)-values of the those points.

What can the slope value tell you about the graph? The slope of a perfectly horizontal line will equal zero, while the slope of a perfectly vertical line will be undefined because you cannot divide by zero. A positive slope indicates that the line moves up the \(y\)-axis as the \(x\)-value increases while a negative slope means that the line moves down the \(y\)-axis. The more negative or positive the slope is, the steeper the line moves up or down, respectively. The slope of our graph in Figure 1.26 is calculated below based on the two endpoints of the line\\
\(m=\frac{Y_{2}-Y_{1}}{X_{2}-X_{1}}\)\\
\(m=\frac{(80 \mathrm{~km})-(20 \mathrm{~km})}{(40 \mathrm{~min})-(10 \mathrm{~min})}\)\\
\(m=\quad \frac{60 \mathrm{~km}}{30 \mathrm{~min}}\)\\
\(m=2.0 \mathrm{~km} / \min\).\\
Equation of line: \(y=(2.0 \mathrm{~km} / \min ) x+0\)\\
Because the \(x\) axis is time in minutes, we would actually be more likely to use the time \(t\) as the independent ( \(x\)-axis) variable and write the equation as\\
\(y=(2.0 \mathrm{~km} / \min ) t+0\).\\
1.4

The formula \(y=m x+b\) only applies to linear relationships, or ones that produce a straight line. Another common type of line in physics is the quadratic relationship, which occurs when one of the variables is squared. One quadratic relationship in physics is the relation between the speed of an object its centripetal acceleration, which is used to determine the force needed to keep an object moving in a circle. Another common relationship in physics is the inverse relationship, in which one variable decreases whenever the other variable increases. An example in physics is Coulomb's law. As the distance between two charged objects increases, the electrical force between the two charged objects decreases. Inverse proportionality, such the relation between \(x\) and \(y\) in the equation\\
\(y=k / x\),\\
for some number \(k\), is one particular kind of inverse relationship. A third\\
commonly-seen relationship is the exponential relationship, in which a change in the independent variable produces a proportional change in the dependent variable. As the value of the dependent variable gets larger, its rate of growth also increases. For example, bacteria often reproduce at an exponential rate when grown under ideal conditions. As each generation passes, there are more and more bacteria to reproduce. As a result, the growth rate of the bacterial population increases every generation (Figure 1.28).

\begin{figure}[h]
\begin{center}
  \includegraphics[max width=\textwidth]{8484573f-7913-4d1d-981f-d54fe53a1e32-63}
\captionsetup{labelformat=empty}
\caption{Figure 1.28 Examples of (a) linear, (b) quadratic, (c) inverse, and (d) exponential relationship graphs.}
\end{center}
\end{figure}

Using Logarithmic Scales in Graphing Sometimes a variable can have a very large range of values. This presents a problem when you're trying to figure out the best scale to use for your graph's axes. One option is to use a logarithmic (log) scale. In a logarithmic scale, the value each mark labels is the previous mark's value multiplied by some constant. For a log base 10 scale, each mark labels a value that is 10 times the value of the mark before it. Therefore, a base 10 logarithmic scale would be numbered: \(0,10,100,1,000\), etc. You can see how the logarithmic scale covers a much larger range of values than the corresponding linear scale, in which the marks would label the values \(0,10,20\), 30 , and so on.

If you use a logarithmic scale on one axis of the graph and a linear scale on the other axis, you are using a semi-log plot. The Richter scale, which measures the strength of earthquakes, uses a semi-log plot. The degree of ground movement is plotted on a logarithmic scale against the assigned intensity level of the earthquake, which ranges linearly from 1-10 (Figure 1.29 (a)).

If a graph has both axes in a logarithmic scale, then it is referred to as a log-log plot. The relationship between the wavelength and frequency of electromagnetic radiation such as light is usually shown as a log-log plot (Figure 1.29 (b)). Loglog plots are also commonly used to describe exponential functions, such as radioactive decay.

\begin{figure}[h]
\begin{center}
  \includegraphics[max width=\textwidth]{8484573f-7913-4d1d-981f-d54fe53a1e32-64}
\captionsetup{labelformat=empty}
\caption{Figure 1.29 (a) The Richter scale uses a log base 10 scale on its \(y\)-axis (microns of amplified maximum ground motion). (b) The relationship between the frequency and wavelength of electromagnetic radiation can be plotted as a straight line if a \(\log -\log\) plot is used.}
\end{center}
\end{figure}

\section*{Worked Example}
Method of Adding Percents: Shingling Your Roof A series of shingles are used to protect the roof of a home. Using a measuring tape, you measure one shingle and find its dimensions to be 44 cm by 100 cm . Knowing that your measurements are not perfect, you estimate an uncertainty of \(\pm 0.5 \mathrm{~cm}\). Following the method of adding percents, what is the area of the shingle, including uncertainty?

\section*{Strategy}
While calculating the area of the shingle is straightforward ( \(44 \mathrm{~cm} \times 100 \mathrm{~cm}= 4400 \mathrm{~cm}^{2}\) ), determining the percent uncertainty is more challenging. In order to use the method of adding percents, you must first calculate the percent uncertainty of each measurement.

Solution\\
Length \% Uncertainty: A/A x 100\% \(=0.5 / 44 \times 100 \%=1.1 \%\)\\
Width \% Uncertainty: A/A x 100\% \(=0.5 / 100 \times 100 \%=0.5 \%\)\\
Adding Percents: \(1.1 \%+0.5 \%=1.6 \%\) uncertainty\\
Area of the Shingle: \(4400 \mathrm{~cm}^{2} \pm 1.6 \%\)\\
Note that this uncertainty can also be expressed in metric terms.\\
\(1.6 \% \times 4400 \mathrm{~cm}^{2}=70.4 \mathrm{~cm}^{2}\)\\
Area of the Shingle: \(4400 \pm 70.4 \mathrm{~cm}^{2}\)\\
Discussion\\
Knowing the percent uncertainty of a shingle can help a contractor determine the number of shingles needed, and therefore the cost, of roofing a new home. Consider how using smaller shingles would affect this uncertainty, and what role this would play in the cost estimation process.

\section*{Virtual Physics}
Graphing Lines In this simulation you will examine how changing the slope and \(y\)-intercept of an equation changes the appearance of a plotted line. Select slope-intercept form and drag the blue circles along the line to change the line's characteristics. Then, play the line game and see if you can determine the slope or \(y\)-intercept of a given line.

Click to view content

\section*{Grasp Check}
How would the following changes affect a line that is neither horizontal nor vertical and has a positive slope?

\begin{enumerate}
  \item increase the slope but keeping the \(y\)-intercept constant
  \item increase the \(y\)-intercept but keeping the slope constant\\
a. Increasing the slope will cause the line to rotate clockwise around the \(y\)-intercept. Increasing the \(y\)-intercept will cause the line to move vertically up on the graph without changing the line's slope.\\
b. Increasing the slope will cause the line to rotate counter-clockwise around the \(y\)-intercept. Increasing the \(y\)-intercept will cause the line to move vertically up on the graph without changing the line's slope.\\
c. Increasing the slope will cause the line to rotate clockwise around the \(y\)-intercept. Increasing the \(y\)-intercept will cause the line to move horizontally right on the graph without changing the line's slope.\\
d. Increasing the slope will cause the line to rotate counter-clockwise around the \(y\)-intercept. Increasing the \(y\)-intercept will cause the line to move horizontally right on the graph without changing the line's slope.
\end{enumerate}

\section*{Check Your Understanding}
12.

Identify some advantages of metric units.\\
a. Conversion between units is easier in metric units.\\
b. Comparison of physical quantities is easy in metric units.\\
c. Metric units are more modern than English units.\\
d. Metric units are based on powers of 2.\\
13.

The length of an American football field is \(100 \backslash, \backslash \operatorname{text}\{\mathrm{yd}\}\), excluding the end zones. How long is the field in meters? Round to the nearest \(0.1 \backslash, \backslash \operatorname{text}\{\mathrm{~m}\}\).\\
a. \(10.2 \backslash, \backslash \operatorname{text}\{\mathrm{~m}\}\)\\
b. \(91.4 \backslash, \backslash \operatorname{text}\{\mathrm{~m}\}\)\\
c. \(109.4 \backslash, \backslash \operatorname{text}\{\mathrm{~m}\}\)\\
d. \(328.1 \backslash, \backslash \operatorname{text}\{\mathrm{~m}\}\)\\
14.

The speed limit on some interstate highways is roughly \(100 \backslash, \backslash \operatorname{text}\{\mathrm{~km} / \mathrm{h}\}\). How many miles per hour is this if \(1.0 \backslash, \backslash \operatorname{text}\{\) mile \(\}\) is about \(1.609 \backslash, \backslash \operatorname{text}\{\mathrm{~km}\}\) ?\\
a. \(0.1 \mathrm{mi} / \mathrm{h}\)\\
b. \(27.8 \mathrm{mi} / \mathrm{h}\)\\
c. \(62 \mathrm{mi} / \mathrm{h}\)\\
d. \(160 \mathrm{mi} / \mathrm{h}\)\\
15.

Briefly describe the target patterns for accuracy and precision and explain the differences between the two.\\
a. Precision states how much repeated measurements generate the same or closely similar results, while accuracy states how close a measurement is to the true value of the measurement.\\
b. Precision states how close a measurement is to the true value of the measurement, while accuracy states how much repeated measurements generate the same or closely similar result.\\
c. Precision and accuracy are the same thing. They state how much repeated measurements generate the same or closely similar results.\\
d. Precision and accuracy are the same thing. They state how close a measurement is to the true value of the measurement.

\section*{Teacher Support}
Teacher Support Use the Check Your Understanding questions to assess students' achievement of the sections learning objectives. If students are struggling with a specific objective, the Check Your Understanding will help identify which and direct students to the relevant content.

\section*{Key Terms}
accuracy how close a measurement is to the correct value for that measurement ampere the SI unit for electrical current\\
atom smallest and most basic units of matter\\
base quantity physical quantity chosen by convention and practical considerations such that all other physical quantities can be expressed as algebraic combinations of them\\
base unit standard for expressing the measurement of a base quantity within a particular system of units; defined by a particular procedure used to measure the corresponding base quantity\\
classical physics physics, as it developed from the Renaissance to the end of the nineteenth century\\
constant a quantity that does not change\\
conversion factor a ratio expressing how many of one unit are equal to another unit\\
dependent variable the vertical, or \(y\)-axis, variable, which changes with (or is dependent on) the value of the independent variable\\
derived quantity physical quantity defined using algebraic combinations of base quantities\\
derived units units that are derived by combining the fundamental physical units\\
experiment process involved with testing a hypothesis\\
exponential relationship relation between variables in which a constant change in the independent variable is accompanied by change in the dependent variable that is proportional to the value it already had\\
fundamental physical units the seven fundamental physical units in the SI system of units are length, mass, time, electric current, temperature, amount of a substance, and luminous intensity\\
hypothesis testable statement that describes how something in the natural world works\\
independent variable the horizontal, or \(x\)-axis, variable, which is not influence by the second variable on the graph, the dependent variable\\
inverse proportionality a relation between two variables expressible by an equation of the form \(y=k / x\) where \(k\) stays constant when \(x\) and \(y\) change; the special form of inverse relationship that satisfies this equation\\
inverse relationship any relation between variables where one variable decreases as the other variable increases\\
kilogram the SI unit for mass, abbreviated (kg)\\
linear relationships relation between variables that produce a straight line when graphed\\
log-log plot a plot that uses a logarithmic scale in both axes\\
logarithmic scale a graphing scale in which each tick on an axis is the previous tick multiplied by some value\\
meter the SI unit for length, abbreviated (m)\\
method of adding percents calculating the percent uncertainty of a quantity in multiplication or division by adding the percent uncertainties in the quantities being added or divided\\
model system that is analogous to the real system of interest in essential ways but more easily analyzed\\
modern physics physics as developed from the twentieth century to the present, involving the theories of relativity and quantum mechanics\\
observation step where a scientist observes a pattern or trend within the natural world\\
order of magnitude the size of a quantity in terms of its power of 10 when expressed in scientific notation\\
physics science aimed at describing the fundamental aspects of our universeenergy, matter, space, motion, and time\\
precision how well repeated measurements generate the same or closely similar results\\
principle description of nature that is true in many, but not all situations\\
quadratic relationship relation between variables that can be expressed in the form \(y=a x^{2}+b x+c\), which produces a curved line when graphed\\
quantum mechanics major theory of modern physics which describes the properties and nature of atoms and their subatomic particles\\
science the study or knowledge of how the physical world operates, based on objective evidence determined through observation and experimentation\\
scientific law pattern in nature that is true in all circumstances studied thus far\\
scientific methods techniques and processes used in the constructing and testing of scientific hypotheses, laws, and theories, and in deciding issues on the basis of experiment and observation\\
scientific notation way of writing numbers that are too large or small to be conveniently written in simple decimal form; the measurement is multiplied by a power of 10, which indicates the number of placeholder zeros in the measurement\\
second the SI unit for time, abbreviated (s)\\
semi-log plot A plot that uses a logarithmic scale on one axis of the graph and a linear scale on the other axis.\\
significant figures when writing a number, the digits, or number of digits, that express the precision of a measuring tool used to measure the number\\
slope the ratio of the change of a graph on the \(y\) axis to the change along the \(x\)-axis, the value of \(m\) in the equation of a line, \(y=m x+b\)\\
theory explanation of patterns in nature that is supported by much scientific evidence and verified multiple times by various groups of researchers\\
theory of relativity theory constructed by Albert Einstein which describes how space, time and energy are different for different observers in relative motion\\
uncertainty a quantitative measure of how much measured values deviate from a standard or expected value\\
universal applies throughout the known universe\\
\(\boldsymbol{y}\)-intercept the point where a plot line intersects the \(y\)-axis

\section*{Key Equations}
1.3 The Language of Physics: Physical Quantities and Units

\section*{Section Summary}
\subsection*{1.1 Physics: Definitions and Applications}
\begin{itemize}
  \item Physics is the most fundamental of the sciences, concerning itself with energy, matter, space and time, and their interactions.
  \item Modern physics involves the theory of relativity, which describes how time, space and gravity are not constant in our universe can be different for different observers, and quantum mechanics, which describes the behavior of subatomic particles.
  \item Physics is the basis for all other sciences, such as chemistry, biology and geology, because physics describes the fundamental way in which the universe functions.
\end{itemize}

\subsection*{1.2 The Scientific Methods}
\begin{itemize}
  \item Science seeks to discover and describe the underlying order and simplicity in nature.
  \item The processes of science include observation, hypothesis, experiment, and conclusion.
  \item Theories are scientific explanations that are supported by a large body experimental results.
  \item Scientific laws are concise descriptions of the universe that are universally true.
\end{itemize}

\subsection*{1.3 The Language of Physics: Physical Quantities and Units}
\begin{itemize}
  \item Physical quantities are a characteristic or property of an object that can be measured or calculated from other measurements.
  \item The four fundamental units we will use in this textbook are the meter (for length), the kilogram (for mass), the second (for time), and the ampere (for electric current). These units are part of the metric system, which uses powers of 10 to relate quantities over the vast ranges encountered in nature.
  \item Unit conversions involve changing a value expressed in one type of unit to another type of unit. This is done by using conversion factors, which are ratios relating equal quantities of different units.
  \item Accuracy of a measured value refers to how close a measurement is to the correct value. The uncertainty in a measurement is an estimate of the amount by which the measurement result may differ from this value.
  \item Precision of measured values refers to how close the agreement is between repeated measurements.
  \item Significant figures express the precision of a measuring tool.
  \item When multiplying or dividing measured values, the final answer can contain only as many significant figures as the least precise value.
  \item When adding or subtracting measured values, the final answer cannot contain more decimal places than the least precise value.
\end{itemize}

\end{document}