\documentclass[10pt]{article}
\usepackage[utf8]{inputenc}
\usepackage[T1]{fontenc}
\usepackage{graphicx}
\usepackage[export]{adjustbox}
\graphicspath{ {./images/} }
\usepackage{caption}
\usepackage{amsmath}
\usepackage{amsfonts}
\usepackage{amssymb}
\usepackage[version=4]{mhchem}
\usepackage{stmaryrd}

\begin{document}
\captionsetup{singlelinecheck=false}
\begin{figure}[h]
\begin{center}
  \includegraphics[max width=\textwidth]{07aafe2d-011d-4a78-9a20-85415115affd-01}
\captionsetup{labelformat=empty}
\caption{Figure 10.1 Special relativity explains why travel to other star systems, such as these in the Orion Nebula, is unlikely using our current level of technology. (s58y, Flickr)}
\end{center}
\end{figure}

\section*{Chapter Outline}
10.1 Postulates of Special Relativity

\subsection*{10.2 Consequences of Special Relativity}
\section*{Introduction}
\section*{Teacher Support}
Teacher Support Start a discussion that taps into the longing of humans to explore worlds beyond our planet. Is this basic human nature? Perhaps it is; humans have now been almost everywhere there is to go on this planet. Ask students why we have not traveled to other star systems yet. Is it just a matter waiting a few years for technological advances, or is there a more daunting problem? If no one knows, tell them it all has to do with achievable speeds, and use this as a lead-in to Einstein's postulate related to the speed of light.

Have you ever dreamed of traveling to other planets in faraway star systems? The trip might seem possible by traveling fast enough, but you will read in this chapter why it is not. In 1905, Albert Einstein developed the theory of special relativity. Einstein developed the theory to help explain inconsistencies between the equations describing electromagnetism and Newtonian mechanics, and to explain why the ether did not exist. This theory explains the limit on an object's speed among other implications.

Relativity is the study of how different observers moving with respect to one another measure the same events. Galileo and Newton developed the first correct version of classical relativity. Einstein developed the modern theory of relativity. Modern relativity is divided into two parts. Special relativity deals\\
with observers moving at constant velocity. General relativity deals with observers moving at constant acceleration. Einstein's theories of relativity made revolutionary predictions. Most importantly, his predictions have been verified by experiments.

In this chapter, you learn how experiments and puzzling contradictions in existing theories led to the development of the theory of special relativity. You will also learn the simple postulates on which the theory was based; a postulate is a statement that is assumed to be true for the purposes of reasoning in a scientific or mathematic argument.

\section*{Teacher Support}
Teacher Support Before students begin this chapter, it is useful to review the following concepts:

\begin{itemize}
  \item Using significant figures in calculations-Demonstrate how to use the proper number of significant figures when adding and multiplying.
  \item Using scientific notation in calculations-Demonstrate how to use the proper scientific notation and operations in scientific notation (e.g., addition/subtraction, multiplication/division).
  \item Converting units-Demonstrate how to convert from \(\mathrm{km} / \mathrm{h}\) to \(\mathrm{m} / \mathrm{s}\).
  \item Calculating average - Demonstrate how to average two numbers by dividing their sum by 2 .
  \item Reviewing the difference between mass and weight.
  \item Commonly used terms-Explain that constant means unchanging. Constant speed refers to speed that is not changing. Explain that initial means starting. Initial time is the time at which the action of a problem begins. Explain that an object that is not moving is often described in physics as being at rest.
\end{itemize}

To reinforce this description, and to open the door for a discussion of frame of reference, take an object, place it in front of the class, and ask someone to describe its motion. Students will likely respond that the object is at rest. Explain that this correct, but it is not the only correct answer. Help students to understand that the object is sitting still but also moving at a high rate of speed as the earth rotates, orbits the sun, etc. It all depends on how you define the frame of reference.

Initiate a discussion aimed at making relativity theory less intimidating. Dispel the misconception that "Only three people in the world understand Einstein's theories." Stories like this come about because Einstein's second relativity theory, called general relativity, was more difficult to understand. In this chapter, we will only learn about special relativity.

\subsection*{10.1 Postulates of Special Relativity}
\section*{Section Learning Objectives}
By the end of this section, you will be able to do the following:

\begin{itemize}
  \item Describe the experiments and scientific problems that led Albert Einstein to develop the special theory of relativity
  \item Understand the postulates on which the special theory of relativity was based
\end{itemize}

\section*{Teacher Support}
Teacher Support The learning objectives in this section will help your students master the following student standards:

\begin{itemize}
  \item (2) Scientific processes. The student uses a systematic approach to answer scientific laboratory and field investigative questions. The student is expected to:
  \item (C) Know that scientific theories are based on natural and physical phenomena and are capable of being tested by multiple independent researchers. Unlike hypotheses, scientific theories are well-established and well-tested explanations, but may be subject to change as new areas of science and new technologies are developed.
  \item (3) Scientific processes. The student uses critical thinking, scientific reasoning, and problem solving to make informed decisions within and outside the classroom. The student is expected to:
  \item (D) Explain the impacts of the scientific contributions of a variety of historical and contemporary scientists on scientific thought and society.
  \item (4) Science concepts. The student knows and applies the laws governing motion in a variety of situations. The student is expected to:
  \item (F) Identify and describe motion relative to different frames of reference.
\end{itemize}

\section*{Section Key Terms}
\section*{Teacher Support}
Teacher Support [AL] Discuss the history of the concept of the ether. Explain that it more of a philosophical concept that was important prior to the\\
development of modern science. It arose from the belief that matter was continuous and vacuums were impossible. Ether was sometimes considered to be one of the elements.\\[0pt]
[BL][OL] Mention that electromagnetic waves are unique among wavepropagated energy forms, in that they can travel across empty space. This was difficult to believe and caused scientists to doggedly hang onto the idea that there must be an ether permeating space. Ask students what they know about Einstein and dispel any misconceptions. Explain what thought experiments and postulates are.

\section*{Scientific Experiments and Problems}
Relativity is not new. Way back around the year 1600, Galileo explained that motion is relative. Wherever you happen to be, it seems like you are at a fixed point and that everything moves with respect to you. Everyone else feels the same way. Motion is always measured with respect to a fixed point. This is called establishing a frame of reference. But the choice of the point is arbitrary, and all frames of reference are equally valid. A passenger in a moving car is not moving with respect to the driver, but they are both moving from the point of view of a person on the sidewalk waiting for a bus. They are moving even faster as seen by a person in a car coming toward them. It is all relative.

\section*{Teacher Support}
Teacher Support [OL][AL] Focus students' thinking on the speed of light. Can the students think of anything else that has a maximum allowable value and is also a universal constant? Most constants are just numbers, like the value of pi. Most properties, such as mass and volume, have no fixed upper limit. Why does speed have a limit?

\section*{Tips For Success}
A frame of reference is not a complicated concept. It is just something you decide is a fixed point or group of connected points. It is completely up to you. For example, when you look up at celestial objects in the sky, you choose the earth as your frame of reference, and the sun, moon, etc., seem to move across the sky.

Light is involved in the discussion of relativity because theories related to electromagnetism are inconsistent with Galileo's and Newton's explanation of relativity. The true nature of light was a hot topic of discussion and controversy in the late 19th century. At the time, it was not generally believed that light could travel across empty space. It was known to travel as waves, and all other types of energy that propagated as waves needed to travel though a material medium. It was believed that space was filled with an invisible medium that\\
light waves traveled through. This imaginary (as it turned out) material was called the ether (also spelled aether). It was thought that everything moved through this mysterious fluid. In other words, ether was the one fixed frame of reference. The Michelson-Morley experiment proved it was not.

In 1887, Albert Michelson and Edward Morley designed the interferometer shown in Figure 10.2 to measure the speed of Earth through the ether. A light beam is split into two perpendicular paths and then recombined. Recombining the waves produces an inference pattern, with a bright fringe at the locations where the two waves arrive in phase; that is, with the crests of both waves arriving together and the troughs arriving together. A dark fringe appears where the crest of one wave coincides with a trough of the other, so that the two cancel. If Earth is traveling through the ether as it orbits the sun, the peaks in one arm would take longer than in the other to reach the same location. The places where the two waves arrive in phase would change, and the interference pattern would shift. But, using the interferometer, there was no shift seen! This result led to two conclusions: that there is no ether and that the speed of light is the same regardless of the relative motion of source and observer. The Michelson-Morley investigation has been called the most famous failed experiment in history.

\begin{figure}[h]
\begin{center}
  \includegraphics[max width=\textwidth]{07aafe2d-011d-4a78-9a20-85415115affd-05}
\captionsetup{labelformat=empty}
\caption{Figure 10.2 This is a diagram of the instrument used in the Michelson-Morley experiment.}
\end{center}
\end{figure}

\section*{Teacher Support}
Teacher Support [BL][OL] Explain the geometry of the Michelson-Morley experiment. Explain why failure in this case was actually a success. Discuss how accepting unexpected results is an important ability for scientists. Ask students to memorize the value of the speed of light in \(\mathrm{m} / \mathrm{s}\) to three significant figures.

To see what Michelson and Morley expected to find when they measured the speed of light in two directions, watch this animation. In the video, two people swimming in a lake are represented as an analogy to light beams leaving Earth as it moves through the ether (if there were any ether). The swimmers swim away from and back to a platform that is moving through the water. The swimmers swim in different directions with respect to the motion of the platform. Even though they swim equal distances at the same speed, the motion of the platform causes them to arrive at different times.

\section*{Teacher Support}
Teacher Support [AL]Be sure students understand that this animation does not explain how light behaves. It shows what Michelson and Morley expected to observe. It may work best to just introduce the Michelson-Morley experiment briefly and then watch the animation.

\section*{Einstein's Postulates}
The results described above left physicists with some puzzling and unsettling questions such as, why doesn't light emitted by a fast-moving object travel faster than light from a street lamp? A radical new theory was needed, and Albert Einstein, shown in Figure 10.3, was about to become everyone's favorite genius. Einstein began with two simple postulates based on the two things we have discussed so far in this chapter.

\begin{enumerate}
  \item The laws of physics are the same in all inertial reference frames.
  \item The speed of light is the same in all inertial reference frames and is not affected by the speed of its source.
\end{enumerate}

\begin{figure}[h]
\begin{center}
  \includegraphics[max width=\textwidth]{07aafe2d-011d-4a78-9a20-85415115affd-06}
\captionsetup{labelformat=empty}
\caption{Figure 10.3 Albert Einstein (1879-1955) developed modern relativity and also made fundamental contributions to the foundations of quantum mechanics.}
\end{center}
\end{figure}

\section*{(The Library of Congress)}
The speed of light is given the symbol \(c\) and is equal to exactly \(299,792,458 \mathrm{m} / \mathrm{s}\). This is the speed of light in vacuum; that is, in the absence of air. For most purposes, we round this number off to \(3.00 \times 10^{8} \mathrm{~m} / \mathrm{s}\) The term inertial reference frame simply refers to a frame of reference where all objects follow Newton's first law of motion: Objects at rest remain at rest, and objects in motion remain in motion at a constant velocity in a straight line, unless acted upon by an external force. The inside of a car moving along a road at constant velocity and the inside of a stationary house are inertial reference frames.

\section*{Teacher Support}
Teacher Support [BL] Ask students to round off the value given for \(c\) to 3 significant figures and express in scientific notation. Stress the units of measurements.\\[0pt]
[OL] Explain the postulates carefully. Note that, although they both seem true, they lead to problems with the classical mechanics of Newton. Explain the concept of reference frame and ask students to think of examples of reference frames that are moving relative to one other. Use vehicles and celestial bodies. Explain that the understanding of relative motion goes back hundreds of years and did not begin with relativity theory.\\[0pt]
[AL] Explain that it is the combination of these two postulates that leads to unusual results that will follow in the next section and that it is the combination of these postulates that forces us to abandon some aspects of Newtonian physics in some scenarios.

\section*{Misconception Alert}
Note that the very precise value for the speed of light only applies to light traveling through a vacuum and that in all transparent material media it is slower.

\section*{Watch Physics}
The Speed of Light This lecture on light summarizes the most important facts about the speed of light. If you are interested, you can watch the whole video, but the parts relevant to this chapter are found between \(3: 25\) and 5:10, which you find by running your cursor along the bottom of the video.

Click to view content

\section*{Teacher Support}
Teacher Support Only the section from 3:25 to 5:10 minutes is completely relevant to an understanding of the speed of light. If students watch it just after the text above, they will be adequately prepared.

\section*{Grasp Check}
An airliner traveling at \(200 \mathrm{~m} / \mathrm{s}\) emits light from the front of the plane. Which statement describes the speed of the light?\\
a. It travels at a speed of \(c+200 \mathrm{~m} / \mathrm{s}\).\\
b. It travels at a speed of \(c-200 \mathrm{~m} / \mathrm{s}\).\\
c. It travels at a speed \(c\), like all light.\\
d. It travels at a speed slightly less than \(c\).

\section*{Snap Lab}
Measure the Speed of Light In this experiment, you will measure the speed of light using a microwave oven and a slice of bread. The waves generated by a microwave oven are not part of the visible spectrum, but they are still electromagnetic radiation, so they travel at the speed of light. If we know the wavelength, , and frequency, \(f\), of a wave, we can calculate its speed, \(v\), using the equation \(v=f\). You can measure the wavelength. You will find the frequency on a label on the back of a microwave oven. The wave in a microwave is a standing wave with areas of high and low intensity. The high intensity sections are one-half wavelength apart.

\begin{itemize}
  \item High temperature: Very hot temperatures are encountered in this lab. These can cause burns.
  \item a microwave oven
  \item one slice of plain white bread
  \item a centimeter ruler
  \item a calculator
\end{itemize}

\begin{enumerate}
  \item Work with a partner.
  \item Turn off the revolving feature of the microwave oven or remove the wheels under the microwave dish that make it turn. It is important that the dish does not turn.
  \item Place the slice of bread on the dish, set the microwave on high, close the door, run the microwave for about 15 seconds.
  \item A row of brown or black marks should appear on the bread. Stop the microwave as soon as they appear. Measure the distance between two adjacent burn marks and multiply the result by 2 . This is the wavelength.
  \item The frequency of the waves is written on the back of the microwave. Look for something like " \(2,450 \mathrm{MHz}\)." Hz is the unit hertz, which means per\\
second. The M represents mega, which stands for million, so multiply the number by \(10^{6}\).
  \item Express the wavelength in meters and multiply it times the frequency. If you did everything correctly, you will get a number very close to the speed of light. Do not eat the bread. It is a general laboratory safety rule never to eat anything in the lab.
\end{enumerate}

\section*{Teacher Support}
Teacher Support This experiment could be demonstrated to the class if a microwave is available in the classroom. It might be better to let the students do the experiment at home so they can get more of a hands-on experience. Before the lab you might have them watch this video.

To continue the discussion, you could tell them about how light is refracted when it changes speed as it passes from one medium to another. Because light speed in air is slower than in a vacuum, light is refracted as it enters Earth's atmosphere.

This link takes you to a video demonstrating how to measure the speed of light using a microwave, a ruler, and a bar of chocolate. There is also an accompanying article with background information on measuring the speed of light.

\section*{Grasp Check}
How does your measured value of the speed of light compare to the accepted value (\% error)?\\
a. The measured value of speed will be equal to \(c\).\\
b. The measured value of speed will be slightly less than \(c\).\\
c. The measured value of speed will be slightly greater than \(c\).\\
d. The measured value of speed will depend on the frequency of the microwave.

\section*{Teacher Support}
Teacher Support [AL] This will be difficult to grasp completely for some students. The fact that the observers see different things is the result of the two postulates being true. If light speed is a constant and the two frames of reference are both valid, then simultaneity is not the same for all observers. Ask them to try the thought experiment in their own head to grasp what is being shown here. Sometimes students have fun sketching a cartoon that explains or demonstrates difficult concepts. They can then show the cartoon to the class and explain their reasoning.\\[0pt]
[OL] If students are struggling with this explanation of simultaneity, let them know that there will be an animation in the next section that should make it clearer.\\[0pt]
[OL][BL] Point out that the relationship between special relativity and Newton's mechanics is an excellent example of how science advances. Explain that new theories rarely reverse old theories. It is more common that new theories extend and expand on old theories. Ask students if they can think of other examples from the history of science.

Einstein's postulates were carefully chosen, and they both seemed very likely to be true. Einstein proceeded despite realizing that these two ideas taken together and applied to extreme conditions led to results that contradict Newtonian mechanics. He just took the ball and ran with it.

In the traditional view, velocities are additive. If you are running at \(3 \mathrm{~m} / \mathrm{s}\) and you throw a ball forward at a speed of \(10 \mathrm{~m} / \mathrm{s}\), the ball should have a net speed of \(13 \mathrm{~m} / \mathrm{s}\). However, according to relativity theory, the speed of a moving light source is not added to the speed of the emitted light.

In addition, Einstein's theory shows that if you were moving forward relative to Earth at nearly \(c\) (the speed of light) and could throw a ball forward at \(c\), an observer at rest on the earth would not see the ball moving at nearly twice the speed of light. The observer would see it moving at a speed that is still less than \(c\). This result conforms to both of Einstein's postulates: The speed of light has a fixed maximum and neither reference frame is privileged.

Consider how we measure elapsed time. If we use a stopwatch, for example, how do we know when to start and stop the watch? One method is to use the arrival of light from the event, such as observing a light turn green to start a drag race. The timing will be more accurate if some sort of electronic detection is used, avoiding human reaction times and other complications.

Now suppose we use this method to measure the time interval between two flashes of light produced by flash lamps on a moving train. (See Figure 10.4)

\begin{figure}[h]
\begin{center}
  \includegraphics[max width=\textwidth]{07aafe2d-011d-4a78-9a20-85415115affd-11}
\captionsetup{labelformat=empty}
\caption{Figure 10.4 Light arriving to observer A as seen by two different observers.}
\end{center}
\end{figure}

A woman (observer A) is seated in the center of a rail car, with two flash lamps at opposite sides equidistant from her. Multiple light rays that are emitted from the flash lamps move towards observer A , as shown with arrows. A velocity vector arrow for the rail car is shown towards the right. A man (observer B) standing on the platform is facing the woman and also observes the flashes of light.

Observer A moves with the lamps on the rail car as the rail car moves towards the right of observer B . Observer B receives the light flashes simultaneously, and sees the bulbs as both having flashed at the same time. However, he sees observer A receive the flash from the right first. Because the pulse from the right reaches her first, in her frame of reference she sees the bulbs as not having flashed simultaneously. Here, a relative velocity between observers affects whether two\\
events at well-separated locations are observed to be simultaneous. Simultaneity, or whether different events occur at the same instant, depends on the frame of reference of the observer. Remember that velocity equals distance divided by time, so \(t=d / \mathbf{v}\). If velocity appears to be different, then duration of time appears to be different.

This illustrates the power of clear thinking. We might have guessed incorrectly that, if light is emitted simultaneously, then two observers halfway between the sources would see the flashes simultaneously. But careful analysis shows this not to be the case. Einstein was brilliant at this type of thought experiment (in German, Gedankenexperiment). He very carefully considered how an observation is made and disregarded what might seem obvious. The validity of thought experiments, of course, is determined by actual observation. The genius of Einstein is evidenced by the fact that experiments have repeatedly confirmed his theory of relativity. No experiments after that of Michelson and Morley were able to detect any ether medium. We will describe later how experiments also confirmed other predictions of special relativity, such as the distance between two objects and the time interval of two events being different for two observers moving with respect to each other.

In summary: Two events are defined to be simultaneous if an observer measures them as occurring at the same time (such as by receiving light from the events). Two events are not necessarily simultaneous to all observers.

The discrepancies between Newtonian mechanics and relativity theory illustrate an important point about how science advances. Einstein's theory did not replace Newton's but rather extended it. It is not unusual that a new theory must be developed to account for new information. In most cases, the new theory is built on the foundation of older theory. It is rare that old theories are completely replaced.

In this chapter, you will learn about the theory of special relativity, but, as mentioned in the introduction, Einstein developed two relativity theories: special and general. Table 10.1 summarizes the differences between the two theories.

Table 10.1 Comparing Special Relativity and General Relativity

\section*{Worked Example}
Calculating the Time it Takes Light to Travel a Given Distance The sun is \(1.50 \times 10^{8} \mathrm{~km}\) from Earth. How long does it take light to travel from the sun to Earth in minutes and seconds?

\section*{Strategy}
Identify knowns.\\
Distance \(=1.50 \times 10^{8} \mathrm{~km}\)\\
Speed \(=3.00 \times 10^{8} \mathrm{~km}\)\\
10.1

Identify unknowns.\\
Time\\
Find the equation that relates knowns and unknowns.\\
\(v=\frac{d}{t} ; \quad t=\frac{d}{v}\)\\
10.2

Be sure to use consistent units.\\
Solution

\[
\begin{gathered}
t=\frac{d}{v}=\frac{\left(1.50 \times 10^{8} \mathrm{~km}\right) \times \frac{10^{3} \mathrm{~m}}{\mathrm{~km}}}{3.00 \times 10^{8} \frac{\mathrm{~m}}{\mathrm{~s}}}=5.00 \times 10^{2} s \\
500 \mathrm{~s}=8 \text { min and } 20 \mathrm{~s}
\end{gathered}
\]

\section*{Discussion}
The answer is written as \(5.00 \times 10^{2}\) rather than 500 in order to show that there are three significant figures. When astronomers witness an event on the sun, such as a sunspot, it actually happened minutes earlier. Compare 8 light minutes to the distance to stars, which are light years away. Any events on other stars happened years ago.

\section*{Teacher Support}
Teacher Support Identify the three variables and choose the relevant equation. In physics, calculations are usually done using units of meters and seconds. Use scientific notation to keep track of significant figures.

\section*{Practice Problems}
1.

Light travels through 1.00 m of water in \(4.42 \times 10^{-9} \mathrm{~s}\). What is the speed of light in water?\\
a. \(4.42 \times 10^{-9} \mathrm{~m} / \mathrm{s}\)\\
b. \(4.42 \times 10^{9} \mathrm{~m} / \mathrm{s}\)\\
c. \(2.26 \times 10^{8} \mathrm{~m} / \mathrm{s}\)\\
d. \(226 \times 10^{8} \mathrm{~m} / \mathrm{s}\)\\
2.

An astronaut on the moon receives a message from mission control on Earth. The signal is sent by a form of electromagnetic radiation and takes 1.28 s to travel the distance between Earth and the moon. What is the distance from Earth to the moon?\\
a. \(2.34 \times 10^{5} \mathrm{~km}\)\\
b. \(2.34 \times 10^{8} \mathrm{~km}\)\\
c. \(3.84 \times 10^{5} \mathrm{~km}\)\\
d. \(3.84 \times 10^{8} \mathrm{~km}\)

\section*{Check Your Understanding}
\section*{Teacher Support}
Teacher Support Use the Check Your Understanding questions to assess students' achievement of the section's learning objectives. If students are struggling with a specific objective, the Check Your Understanding will help identify which and direct students to the relevant content.\\
3.

Explain what is meant by a frame of reference.\\
a. A frame of reference is a graph plotted between distance and time.\\
b. A frame of reference is a graph plotted between speed and time.\\
c. A frame of reference is the velocity of an object through empty space without regard to its surroundings.\\
d. A frame of reference is an arbitrarily fixed point with respect to which motion of other points is measured.\\
4.

Two people swim away from a raft that is floating downstream. One swims upstream and returns, and the other swims across the current and back. If this scenario represents the Michelson-Morley experiment, what do (i) the water, (ii) the swimmers, and (iii) the raft represent?\\
a. the ether rays of light Earth\\
b. rays of light the ether Earth\\
c. the ether Earth rays of light\\
d. Earth rays of light the ether\\
5.

If Michelson and Morley had observed the interference pattern shift in their interferometer, what would that have indicated?\\
a. The speed of light is the same in all frames of reference.\\
b. The speed of light depends on the motion relative to the ether.\\
c. The speed of light changes upon reflection from a surface.\\
d. The speed of light in vacuum is less than \(3.00 \times 10^{8} \mathrm{~m} / \mathrm{s}\).\\
6.

If you designate a point as being fixed and use that point to measure the motion of surrounding objects, what is the point called?\\
a. An origin\\
b. A frame of reference\\
c. A moving frame\\
d. A coordinate system

\subsection*{10.2 Consequences of Special Relativity}
\section*{Section Learning Objectives}
By the end of this section, you will be able to do the following:

\begin{itemize}
  \item Describe the relativistic effects seen in time dilation, length contraction, and conservation of relativistic momentum
  \item Explain and perform calculations involving mass-energy equivalence
\end{itemize}

\section*{Teacher Support}
Teacher Support The learning objectives in this section will help your students master the following student standards:

\begin{itemize}
  \item (4) Science concepts. The student knows and applies the laws governing motion in a variety of situations. The student is expected to:
  \item (F) Identify and describe motion relative to different frames of reference.
  \item (8) Science concepts. The student knows simple examples of atomic, nuclear, and quantum phenomena. The student is expected to:
  \item (C) Describe the significance of mass-energy equivalence and apply it in explanations of phenomena such as nuclear stability, fission, and fusion.
\end{itemize}

\section*{Section Key Terms}
\section*{Teacher Support}
Teacher Support In this section, you will see how the postulates lead to the theory of special relativity and see how that theory predicts effects on time, distance, momentum, and energy at velocities approaching the speed of light.\\[0pt]
[BL] Begin a discussion by asking if students have ever seen a science fiction movie where space travelers age more slowly than the people left behind on Earth. Tell them there is some basis in fact to these stories. Discuss nuclear power. Ask if they know the basic difference between the nuclear power and combustion power.\\[0pt]
[OL] Explain that Newton's laws are valid for everyday mechanics but break down at speeds approaching the speed of light. Discuss the relationship between relativity theory and Newton's laws. Briefly describe the changes predicted for\\
measurements of time, length, momentum, and energy. See how much they know about energy derived from nuclear reactions.\\[0pt]
[AL] Ask what the students already know about relativity theory. See if they know that relative motion is an old idea and ask for examples of relative motion in everyday situations. Explain that special relativity is similar but describes unexpected results at speeds approaching the speed of light. Ask if anyone can explain why this statement is true: "The original source of all the energy we use is the conversion of matter into energy."

\section*{Relativistic Effects on Time, Distance, and Momentum}
Consideration of the measurement of elapsed time and simultaneity leads to an important relativistic effect. Time dilation is the phenomenon of time passing more slowly for an observer who is moving relative to another observer.

For example, suppose an astronaut measures the time it takes for light to travel from the light source, cross her ship, bounce off a mirror, and return. (See Figure 10.5.) How does the elapsed time the astronaut measures compare with the elapsed time measured for the same event by a person on the earth? Asking this question (another thought experiment) produces a profound result. We find that the elapsed time for a process depends on who is measuring it. In this case, the time measured by the astronaut is smaller than the time measured by the earth bound observer. The passage of time is different for the two observers because the distance the light travels in the astronaut's frame is smaller than in the earth bound frame. Light travels at the same speed in each frame, and so it will take longer to travel the greater distance in the earth bound frame.

\section*{Teacher Support}
Teacher Support [OL] Discuss the expression for the relativistic factor. Explain that this is involved in all relativistic effects. Show how to tell when relativistic effects are significant and when they are negligible by plugging in values of \(v\) and \(c\).

\begin{figure}[h]
\begin{center}
  \includegraphics[max width=\textwidth]{07aafe2d-011d-4a78-9a20-85415115affd-18}
\captionsetup{labelformat=empty}
\caption{Figure 10.5 (a) An astronaut measures the time \(\Delta t_{0}\) for light to cross her ship using an electronic timer. Light travels a distance \(2 D\) in the astronaut's frame. (b) A person on the earth sees the light follow the longer path \(2 s\) and take a longer time \(\Delta t\).}
\end{center}
\end{figure}

\section*{Teacher Support}
Teacher Support [AL]Figure 10.5, like Figure 10.4, may be hard for some students to grasp. Refer back to the previous figure. The animation in the discussion of length contraction further on should also be some help.

The relationship between \(\Delta t\) and \(\Delta t_{\mathrm{o}}\) is given by\\
\(\Delta t=\gamma \Delta t_{0}\),\\
where \(\gamma\) is the relativistic factor given by\\
\(\gamma=\frac{1}{\sqrt{1-\frac{v^{2}}{c^{2}}}}\),\\
and \(v\) and \(c\) are the speeds of the moving observer and light, respectively.

\section*{Tips For Success}
Try putting some values for \(v\) into the expression for the relativistic factor ( \(\gamma\) ). Observe at which speeds this factor will make a difference and when is so close to 1 that it can be ignored. Try \(225 \mathrm{~m} / \mathrm{s}\), the speed of an airliner; \(2.98 \times 10^{4} \mathrm{~m} / \mathrm{s}\), the speed of Earth in its orbit; and \(2.990 \times 10^{8} \mathrm{~m} / \mathrm{s}\), the speed of a particle in an accelerator.

\section*{Teacher Support}
Teacher Support Try putting some values for \(v\) into the expression for the relativistic factor. Observe at which speeds this factor will make a difference and when it is so close to 1 that it can be ignored.

Notice that when the velocity \(v\) is small compared to the speed of light \(c\), then \(v / c\) becomes small, and \(\gamma\) becomes close to 1 . When this happens, time measurements are the same in both frames of reference. Relativistic effects, meaning those that have to do with special relativity, usually become significant when speeds become comparable to the speed of light. This is seen to be the case for time dilation.

You may have seen science fiction movies in which space travelers return to Earth after a long trip to find that the planet and everyone on it has aged much more than they have. This type of scenario is a based on a thought experiment, known as the twin paradox, which imagines a pair of twins, one of whom goes on a trip into space while the other stays home. When the space traveler returns, she finds her twin has aged much more than she. This happens because the traveling twin has been in two frames of reference, one leaving Earth and one returning.

Time dilation has been confirmed by comparing the time recorded by an atomic clock sent into orbit to the time recorded by a clock that remained on Earth. GPS satellites must also be adjusted to compensate for time dilation in order to give accurate positioning.

Have you ever driven on a road, like that shown in Figure 10.6, that seems like it goes on forever? If you look ahead, you might say you have about 10 km left to go. Another traveler might say the road ahead looks like it is about 15 km long. If you both measured the road, however, you would agree. Traveling at everyday speeds, the distance you both measure would be the same. You will read in this section, however, that this is not true at relativistic speeds. Close to the speed of light, distances measured are not the same when measured by different observers moving with respect to one other.

\begin{figure}[h]
\begin{center}
  \includegraphics[max width=\textwidth]{07aafe2d-011d-4a78-9a20-85415115affd-20}
\captionsetup{labelformat=empty}
\caption{Figure 10.6 People might describe distances differently, but at relativistic speeds, the distances really are different. (Corey Leopold, Flickr)}
\end{center}
\end{figure}

\section*{Teacher Support}
Teacher Support [OL] Discuss the relationship between time dilation and length contraction. If observers agree on speed, but not on time, they must also disagree on length because \(v=d / t\).

One thing all observers agree upon is their relative speed. When one observer is traveling away from another, they both see the other receding at the same speed, regardless of whose frame of reference is chosen. Remember that speed equals distance divided by time: \(v=d / t\). If the observers experience a difference in elapsed time, they must also observe a difference in distance traversed. This is because the ratio \(d / t\) must be the same for both observers.

The shortening of distance experienced by an observer moving with respect to the points whose distance apart is measured is called length contraction. Proper length, \(L_{0}\), is the distance between two points measured in the reference frame where the observer and the points are at rest. The observer in motion with respect to the points measures \(L\). These two lengths are related by the equation \(L=\frac{L_{0}}{\gamma}\).\\
Because is the same expression used in the time dilation equation above, the equation becomes\\
\(L=L_{0} \sqrt{1-\frac{v^{2}}{c^{2}}}\).\\
To see how length contraction is seen by a moving observer, go to this simulation. Here you can also see that simultaneity, time dilation, and length contraction are interrelated phenomena.

This link is to a simulation that illustrates the relativity of simultaneous events.

In classical physics, momentum is a simple product of mass and velocity. When special relativity is taken into account, objects that have mass have a speed limit. What effect do you think mass and velocity have on the momentum of objects moving at relativistic speeds; i.e., speeds close to the speed of light?

Momentum is one of the most important concepts in physics. The broadest form of Newton's second law is stated in terms of momentum. Momentum is conserved in classical mechanics whenever the net external force on a system is zero. This makes momentum conservation a fundamental tool for analyzing collisions. We will see that momentum has the same importance in modern physics. Relativistic momentum is conserved, and much of what we know about subatomic structure comes from the analysis of collisions of accelerator-produced relativistic particles.

One of the postulates of special relativity states that the laws of physics are the same in all inertial frames. Does the law of conservation of momentum survive this requirement at high velocities? The answer is yes, provided that the momentum is defined as follows.

Relativistic momentum, \(\mathbf{p}\), is classical momentum multiplied by the relativistic factor \(\gamma\).\\
\(\mathbf{p}=\gamma m \mathbf{u}\),\\
10.3\\
where \(m\) is the rest mass of the object (that is, the mass measured at rest, without any \(\gamma\) factor involved), \(\mathbf{u}\) is its velocity relative to an observer, and\\
\(\gamma\),\\
as before, is the relativistic factor. We use the mass of the object as measured at rest because we cannot determine its mass while it is moving.

Note that we use \(\mathbf{u}\) for velocity here to distinguish it from relative velocity \(\mathbf{v}\) between observers. Only one observer is being considered here. With \(\mathbf{p}\) defined in this way, \(\mathbf{p}_{\text {tot }}\) is conserved whenever the net external force is zero, just as in classical physics. Again we see that the relativistic quantity becomes virtually the same as the classical at low velocities. That is, relativistic momentum \(\gamma m \mathbf{u}\) becomes the classical \(m \mathbf{u}\) at low velocities, because \(\gamma\) is very nearly equal to 1 at low velocities.

Relativistic momentum has the same intuitive feel as classical momentum. It is greatest for large masses moving at high velocities. Because of the factor \(\gamma\), however, relativistic momentum behaves differently from classical momentum by approaching infinity as \(\mathbf{u}\) approaches \(c\). (See Figure 10.7.) This is another indication that an object with mass cannot reach the speed of light. If it did, its momentum would become infinite, which is an unreasonable value.

\begin{figure}[h]
\begin{center}
  \includegraphics[max width=\textwidth]{07aafe2d-011d-4a78-9a20-85415115affd-22}
\captionsetup{labelformat=empty}
\caption{Figure 10.7 Relativistic momentum approaches infinity as the velocity of an object approaches the speed of light.}
\end{center}
\end{figure}

\section*{Teacher Support}
Teacher Support [OL] Discuss the graph. Explain how it shows that objects that have mass cannot reach the speed of light. Have the students analyze the equation for relativistic momentum and see how this supports this conclusion. Explain that light can travel at the speed of light because it has no rest mass.

Relativistic momentum is defined in such a way that the conservation of momentum will hold in all inertial frames. Whenever the net external force on a system is zero, relativistic momentum is conserved, just as is the case for classical momentum. This has been verified in numerous experiments.

\section*{Mass-Energy Equivalence}
Let us summarize the calculation of relativistic effects on objects moving at speeds near the speed of light. In each case we will need to calculate the relativistic factor, given by\\
\(\gamma=\frac{1}{\sqrt{1-\frac{\mathrm{v}^{2}}{c^{2}}}}\),\\
where \(\mathbf{v}\) and \(c\) are as defined earlier. We use \(\mathbf{u}\) as the velocity of a particle or an object in one frame of reference, and \(\mathbf{v}\) for the velocity of one frame of reference with respect to another.

Time Dilation Elapsed time on a moving object, \(\Delta t_{0}\), as seen by a stationary observer is given by \(\Delta t=\gamma \Delta t_{0}\), where \(\Delta t_{0}\) is the time observed on the moving object when it is taken to be the frame or reference.

Length Contraction Length measured by a person at rest with respect to a moving object, \(L\), is given by\\
\(L=\frac{L_{0}}{\gamma}\),\\
where \(L_{0}\) is the length measured on the moving object.

Relativistic Momentum Momentum, \(\mathbf{p}\), of an object of mass, \(m\), traveling at relativistic speeds is given by \(\mathbf{p}=\gamma m \mathbf{u}\), where \(\mathbf{u}\) is velocity of a moving object as seen by a stationary observer.

Relativistic Energy The original source of all the energy we use is the conversion of mass into energy. Most of this energy is generated by nuclear reactions in the sun and radiated to Earth in the form of electromagnetic radiation, where it is then transformed into all the forms with which we are familiar. The remaining energy from nuclear reactions is produced in nuclear power plants and in Earth's interior. In each of these cases, the source of the energy is the conversion of a small amount of mass into a large amount of energy. These sources are shown in Figure 10.8.

\begin{figure}[h]
\begin{center}
  \includegraphics[max width=\textwidth]{07aafe2d-011d-4a78-9a20-85415115affd-23}
\captionsetup{labelformat=empty}
\caption{Figure 10.8 The sun (a) and the Susquehanna Steam Electric Station (b) both convert mass into energy. ((a) NASA/Goddard Space Flight Center, Scientific Visualization Studio; (b) U.S. government)\\
The first postulate of relativity states that the laws of physics are the same in all inertial frames. Einstein showed that the law of conservation of energy is valid relativistically, if we define energy to include a relativistic factor. The result of his analysis is that a particle or object of mass \(m\) moving at velocity \(\mathbf{u}\) has relativistic energy given by}
\end{center}
\end{figure}

\(E=\gamma m c^{2}\).\\
This is the expression for the total energy of an object of mass \(m\) at any speed \(\mathbf{u}\) and includes both kinetic and potential energy. Look back at the equation for \(\gamma\) and you will see that it is equal to 1 when \(\mathbf{u}\) is 0 ; that is, when an object is at rest. Then the rest energy, \(E_{0}\), is simply\\
\(E_{0}=m c^{2}\).\\
This is the correct form of Einstein's famous equation.\\
This equation is very useful to nuclear physicists because it can be used to calculate the energy released by a nuclear reaction. This is done simply by subtracting the mass of the products of such a reaction from the mass of the reactants. The difference is the \(m\) in \(E_{0}=m c^{2}\). Here is a simple example:

A positron is a type of antimatter that is just like an electron, except that it has a positive charge. When a positron and an electron collide, their masses are completely annihilated and converted to energy in the form of gamma rays. Because both particles have a rest mass of \(9.11 \times 10^{-31} \mathrm{~kg}\), we multiply the \(m c^{2}\) term by 2 . So the energy of the gamma rays is

\[
\begin{aligned}
E_{0} & =2\left(9.11 \times 10^{-31} \mathrm{~kg}\right)\left(3.00 \times 10^{8} \frac{\mathrm{~m}}{\mathrm{~s}}\right)^{2} \\
& =1.64 \times 10^{-13} \frac{\mathrm{~kg} \cdot \mathrm{~m}^{2}}{\mathrm{~s}^{2}} \\
& =1.64 \times 10^{-13} \mathrm{~J}
\end{aligned}
\]

10.4\\
where we have the expression for the joule ( J ) in terms of its SI base units of \(\mathrm{kg}, \mathrm{m}\), and s . In general, the nuclei of stable isotopes have less mass then their constituent subatomic particles. The energy equivalent of this difference is called the binding energy of the nucleus. This energy is released during the formation of the isotope from its constituent particles because the product is more stable than the reactants. Expressed as mass, it is called the mass defect. For example, a helium nucleus is made of two neutrons and two protons and has a mass of 4.0003 atomic mass units \((u)\). The sum of the masses of two protons and two neutrons is 4.0330 u . The mass defect then is 0.0327 u . Converted to kg , the mass defect is \(5.0442 \times 10^{-30} \mathrm{~kg}\). Multiplying this mass times \(c^{2}\) gives a binding energy of \(4.540 \times 10^{-12} \mathrm{~J}\). This does not sound like much because it is only one atom. If you were to make one gram of helium out of neutrons and protons, it would release \(683,000,000,000 \mathrm{~J}\). By comparison, burning one gram of coal releases about 24 J .

\section*{Teacher Support}
Teacher Support [BL] In regards to the change in the law of conservation of energy to the law of conservation of mass-energy, it may help to think of mass as simply a very concentrated form of energy.\\[0pt]
[OL] Impress upon the students the enormous amount of energy derived from the conversion of a small amount of mass. Have them note that \(c^{2}\) is a very large number. Students try to understand new concepts by using previous knowledge, and that may result in a misconception here. They are comfortable with chemical reactions and may try to relate this to the burning of a piece of wood. Tell them that burning the wood chemically might provide energy for a single room in a house, but converting the mass of the wood completely to energy according to \(E=m c^{2}\) would provide power for thousands of houses.\\[0pt]
[AL] Ask students if they know the difference between fission and fusion and where examples of each of these occur.

\section*{Boundless Physics}
The RHIC Collider Figure 10.9 shows the Brookhaven National Laboratory in Upton, NY. The circular structure houses a particle accelerator called the RHIC, which stands for Relativistic Heavy Ion Collider. The heavy ions in the name are gold nuclei that have been stripped of their electrons. Streams of ions are accelerated in several stages before entering the big ring seen in the figure. Here, they are accelerated to their final speed, which is about 99.7 percent the speed of light. Such high speeds are called relativistic. All the relativistic phenomena we have been discussing in this chapter are very pronounced in this case. At this speed \(\gamma=12.9\), so that relativistic time dilates by a factor of about 13, and relativistic length contracts by the same factor.

\begin{figure}[h]
\begin{center}
  \includegraphics[max width=\textwidth]{07aafe2d-011d-4a78-9a20-85415115affd-26}
\captionsetup{labelformat=empty}
\caption{Figure 10.9 Brookhaven National Laboratory. The circular structure houses the RHIC. (energy.gov, Wikimedia Commons)}
\end{center}
\end{figure}

Two ion beams circle the 2.4 -mile long track around the big ring in opposite directions. The paths can then be made to cross, thereby causing ions to collide. The collision event is very short-lived but amazingly intense. The temperatures\\
and pressures produced are greater than those in the hottest suns. At 4 trillion degrees Celsius, this is the hottest material ever created in a laboratory

But what is the point of creating such an extreme event? Under these conditions, the neutrons and protons that make up the gold nuclei are smashed apart into their components, which are called quarks and gluons. The goal is to recreate the conditions that theorists believe existed at the very beginning of the universe. It is thought that, at that time, matter was a sort of soup of quarks and gluons. When things cooled down after the initial bang, these particles condensed to form protons and neutrons.

Some of the results have been surprising and unexpected. It was thought the quark-gluon soup would resemble a gas or plasma. Instead, it behaves more like a liquid. It has been called a perfect liquid because it has virtually no viscosity, meaning that it has no resistance to flow.

\section*{Teacher Support}
Teacher Support Discuss particle colliders such as the relatively new Large Hadron Collider built by CERN. Students may want to know more about this project and the God particle. Explain why there are so many applications of special relativity theory in the field of particle physics.

\section*{Grasp Check}
Calculate the relativistic factor , for a particle traveling at 99.7 percent of the speed of light.\\
a. 0.08\\
b. 0.71\\
c. 1.41\\
d. 12.9

\section*{Worked Example}
The Speed of Light One night you are out looking up at the stars and an extraterrestrial spaceship flashes across the sky. The ship is 50 meters long and is travelling at 95 percent of the speed of light. What would the ship's length be when measured from your earthbound frame of reference?

\section*{Strategy}
List the knowns and unknowns.\\
Knowns: proper length of the ship, \(L_{0}=50 \mathrm{~m}\); velocity, \(\mathbf{v},=0.95 c\)\\
Unknowns: observed length of the ship accounting for relativistic length contraction, \(L\).

Choose the relevant equation.\\
\(L=\frac{L_{0}}{\gamma}=L_{0} \sqrt{1-\frac{\mathbf{u}^{2}}{c^{2}}}\)\\
Solution\\
\(L=50 \mathrm{~m} \sqrt{1-\frac{(0.95)^{2} c^{2}}{c^{2}}}=50 \mathrm{~m} \sqrt{1-(0.95)^{2}}=16 \mathrm{~m}\)\\
Discussion\\
Calculations of can usually be simplified in this way when \(v\) is expressed as a percentage of \(c\) because the \(c^{2}\) terms cancel. Be sure to also square the decimal representing the percentage before subtracting from 1 . Note that the aliens will still see the length as \(L_{0}\) because they are moving with the frame of reference that is the ship.

\section*{Teacher Support}
Teacher Support Identify the variables, the knowns and unknowns, and the relevant equation. Understand clearly which length applies to your frame of reference and which applies to the ship's frame of reference; that is, which is proper length.

\section*{Practice Problems}
7.

Calculate the relativistic factor, , for an object traveling at \(2.00 \times 10^{8} \mathrm{~m} / \mathrm{s}\).\\
a. 0.74\\
b. 0.83\\
c. 1.2\\
d. 1.34\\
8.

The distance between two points, called the proper length, L0, is 1.00 km . An observer in motion with respect to the frame of reference of the two points measures 0.800 km , which is L . What is the relative speed of the frame of reference with respect to the observer?\\
a. \(1.80 \times 10^{8} \mathrm{~m} / \mathrm{s}\)\\
b. \(2.34 \times 10^{8} \mathrm{~m} / \mathrm{s}\)\\
c. \(3.84 \times 10^{8} \mathrm{~m} / \mathrm{s}\)\\
d. \(5.00 \times 10^{8} \mathrm{~m} / \mathrm{s}\)\\
9.

Consider the nuclear fission reaction \(\mathrm{n}+\mathrm{U} 92235 \rightarrow \mathrm{C} 55137 \mathrm{~s}+\mathrm{R} 3797 \mathrm{b}+2 \mathrm{n}+\mathrm{E}\). If a neutron has a rest mass of \(1.009 \mathrm{u}, \mathrm{U} 92235\) has a rest mass of\\
235.044u, C 55137 s has rest mass of 136.907 u , and R 3797 b has a rest mass of 96.937 u , what is the value of \(E\) in joules?\\
a. \(1.8 \times 10^{-11} \mathrm{~J}\)\\
b. \(2.9 \times 10^{-11} \mathrm{~J}\)\\
c. \(1.8 \times 10^{-10} \mathrm{~J}\)\\
d. \(2.9 \times 10^{-10} \mathrm{~J}\)

Solution The correct answer is (b). The mass deficit in the reaction is \(235.044 \mathrm{u}-(136.907+96.937+1.009) \mathrm{u}\), or 0.191 u . Converting that mass to kg and applying \(E=m c^{2}\) to find the energy equivalent of the mass deficit gives \((0.191 \mathrm{u})\left(1.66 \times 10^{-27} \mathrm{~kg} / \mathrm{u}\right)\left(3.00 \times 10^{8} \mathrm{~m} / \mathrm{s}\right)^{2} \cong 2.85 \times 10^{-11} \mathrm{~J}\).\\
10.

Consider the nuclear fusion reaction H \(12+\mathrm{H} 12 \rightarrow \mathrm{H} 13+\mathrm{H} 11+\mathrm{E}\). If H 12 has a rest mass of \(2.014 \mathrm{u}, \mathrm{H} 13\) has a rest mass of 3.016 u , and H 11 has a rest mass of 1.008 u , what is the value of \(E\) in joules?\\
a. \(6 \times 10^{-13} \mathrm{~J}\)\\
b. \(6 \times 10^{-12} \mathrm{~J}\)\\
c. \(6 \times 10^{-11} \mathrm{~J}\)\\
d. \(6 \times 10^{-10} \mathrm{~J}\)

Solution The correct answer is (a). The mass deficit in the reaction is \(2(2.014 \mathrm{u})-(3.016+1.008) \mathrm{u}\), or 0.004 u . Converting that mass to kg and applying \(E=m c^{2}\) to find the energy equivalent of the mass deficit gives \((0.004 \mathrm{u})\left(1.66 \times 10^{-27} \mathrm{~kg} / \mathrm{u}\right)\left(3.00 \times 10^{8} \mathrm{~m} / \mathrm{s}\right)^{2} \cong 5.98 \times 10^{-13} \mathrm{~J}\).

\section*{Check Your Understanding}
11.

Describe time dilation and state under what conditions it becomes significant.\\
a. When the speed of one frame of reference past another reaches the speed of light, a time interval between two events at the same location in one frame appears longer when measured from the second frame.\\
b. When the speed of one frame of reference past another becomes comparable to the speed of light, a time interval between two events at the same location in one frame appears longer when measured from the second frame.\\
c. When the speed of one frame of reference past another reaches the speed of light, a time interval between two events at the same location in one frame appears shorter when measured from the second frame.\\
d. When the speed of one frame of reference past another becomes comparable to the speed of light, a time interval between two events at the same\\
location in one frame appears shorter when measured from the second frame.\\
12.

The equation used to calculate relativistic momentum is \(p=\quad m \quad u\) Define the terms to the right of the equal sign and state how \(m\) and \(u\) are measured.\\
a. is the relativistic factor, \(m\) is the rest mass measured when the object is at rest in the frame of reference, and \(u\) is the velocity of the frame.\\
b. is the relativistic factor, \(m\) is the rest mass measured when the object is at rest in the frame of reference, and \(u\) is the velocity relative to an observer.\\
c. is the relativistic factor, \(m\) is the relativistic mass ( i.e., \(\frac{m}{\sqrt{1-\frac{u^{2}}{c^{2}}}}\) ) measured when the object is moving in the frame of reference, and \(u\) is the velocity of the frame.\\
d. is the relativistic factor, \(m\) is the relativistic mass ( i.e., \(\frac{m}{\sqrt{1-\frac{u^{2}}{c^{2}}}}\) ) measured when the object is moving in the frame of reference, and \(u\) is the velocity relative to an observer.\\
13.

Describe length contraction and state when it occurs.\\
a. When the speed of an object becomes the speed of light, its length appears to shorten when viewed by a stationary observer.\\
b. When the speed of an object approaches the speed of light, its length appears to shorten when viewed by a stationary observer.\\
c. When the speed of an object becomes the speed of light, its length appears to increase when viewed by a stationary observer.\\
d. When the speed of an object approaches the speed of light, its length appears to increase when viewed by a stationary observer.

\section*{Teacher Support}
Teacher Support Use the Check Your Understanding questions to assess students' achievement of the section's learning objectives. If students are struggling with a specific objective, the Check Your Understanding will help identify which and direct students to the relevant content.

\section*{Key Terms}
binding energy the energy equivalent of the difference between the mass of a nucleus and the masses of its nucleons\\
ether scientists once believed there was a medium that carried light waves; eventually, experiments proved that ether does not exist\\
frame of reference the point or collection of points arbitrarily chosen, which motion is measured in relation to\\
general relativity the theory proposed to explain gravity and acceleration\\
inertial reference frame a frame of reference where all objects follow Newton's first law of motion\\
length contraction the shortening of an object as seen by an observer who is moving relative to the frame of reference of the object\\
mass defect the difference between the mass of a nucleus and the masses of its nucleons\\
postulate a statement that is assumed to be true for the purposes of reasoning in a scientific or mathematic argument\\
proper length the length of an object within its own frame of reference, as opposed to the length observed by an observer moving relative to that frame of reference\\
relativistic having to do with modern relativity, such as the effects that become significant only when an object is moving close enough to the speed of light for to be significantly greater than 1\\
relativistic energy the total energy of a moving object or particle \(E=\gamma m c^{2}\), which includes both its rest energy \(m c^{2}\) and its kinetic energy\\
relativistic factor \(\gamma=\frac{1}{\sqrt{1-\frac{\mathbf{u}^{2}}{c^{2}}}}\), where \(\mathbf{u}\) is the velocity of a moving object and \(c\) is the speed of light\\
relativistic momentum \(\mathbf{p}=m \mathbf{u}\), where is the relativistic factor, \(m\) is rest mass of an object, and \(\mathbf{u}\) is the velocity relative to an observer\\
relativity the explanation of how objects move relative to one another\\
rest mass the mass of an object that is motionless with respect to its frame of reference\\
simultaneity the property of events that occur at the same time\\
special relativity the theory proposed to explain the consequences of requiring the speed of light and the laws of physics to be the same in all inertial frames\\
time dilation the contraction of time as seen by an observer in a frame of reference that is moving relative to the observer

\section*{Key Equations}
10.1 Postulates of Special Relativity

\subsection*{10.2 Consequences of Special Relativity}
\section*{Section Summary}
\subsection*{10.1 Postulates of Special Relativity}
\begin{itemize}
  \item One postulate of special relativity theory is that the laws of physics are the same in all inertial frames of reference.
  \item The other postulate is that the speed of light in a vacuum is the same in all inertial frames.
  \item Einstein showed that simultaneity, or lack of it, depends on the frame of reference of the observer.
\end{itemize}

\subsection*{10.2 Consequences of Special Relativity}
\begin{itemize}
  \item Time dilates, length contracts, and momentum increases as an object approaches the speed of light.
  \item Energy and mass are interchangeable, according to the relationship \(E= m c 2\). The laws of conservation of mass and energy are combined into the law of conservation of mass-energy.
\end{itemize}

\end{document}