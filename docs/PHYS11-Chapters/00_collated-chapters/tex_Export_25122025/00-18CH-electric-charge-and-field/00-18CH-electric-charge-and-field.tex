\documentclass[10pt]{article}
\usepackage[utf8]{inputenc}
\usepackage[T1]{fontenc}
\usepackage{graphicx}
\usepackage[export]{adjustbox}
\graphicspath{ {./images/} }
\usepackage{caption}
\usepackage{amsmath}
\usepackage{amsfonts}
\usepackage{amssymb}
\usepackage[version=4]{mhchem}
\usepackage{stmaryrd}

\begin{document}
\captionsetup{singlelinecheck=false}
\begin{figure}[h]
\begin{center}
  \includegraphics[max width=\textwidth]{b13713c9-2fd4-4196-9ad9-05e4dd292cb0-01}
\captionsetup{labelformat=empty}
\caption{Figure 18.1 This child's hair contains an imbalance of electrical charge (commonly called static electricity), which causes it to stand on end. The sliding motion stripped electrons away from the child's body, leaving him with an excess of positive charges, which repel each other along each strand of hair. (credit: Ken Bosma, Wikimedia Commons)}
\end{center}
\end{figure}

\section*{Chapter Outline}
18.1 Electrical Charges, Conservation of Charge, and Transfer of Charge\\
18.2 Coulomb's law\\
18.3 Electric Field\\
18.4 Electric Potential\\
18.5 Capacitors and Dielectrics

\section*{Introduction}
\section*{Teacher Support}
\section*{Teacher Support}
\begin{itemize}
  \item Ask students what they know about static electricity and about electric charge. Explain that static electricity is a field of study that focuses on the forces and electrical pressure between objects that have an imbalance of electric charge. Although the word static implies that nothing is moving, this is often not the case. For example, separating positive from negative\\
charges involves charges in motion but creates static electricity. Likewise, a lightning strike is a huge electric current but is created by static electricity. Normally, static electricity phenomena involve high electric pressure and low, noncontinuous electric currents.
  \item Review Newton's law of universal gravitation. Discuss how the force of gravity varies inversely with the distance squared between masses and how it is always an attractive force. Point out how the quantity that determines how strongly gravity acts on an object is its mass. Explain that electric charge is the analogous quantity that determines how strong electric forces are between objects that have nonzero electric charge.
\end{itemize}

\section*{Teacher Support}
Teacher Support [BL][OL]Review the notion of repulsive and attractive forces. Discuss how forces can have the same magnitude but act in opposite directions.\\[0pt]
[AL]Explain that all the macroscopic forces that we experience directly, such as the sensations of touch and the tension in a rope, are due to the electromagnetic force, one of the four fundamental forces in nature. Electrostatic forces are a subset of electromagnetic forces, which are covered further in later chapters. The gravitational force, another fundamental force, is actually sensed through the electromagnetic interaction of molecules, such as between those in our feet and those on the top of a bathroom scale. The other two fundamental forces, the strong nuclear force and the weak nuclear force, cannot be sensed on the human scale.

You may have been introduced to static electricity like the child sliding down the slide in the opening photograph (Figure 18.1). The zap that he is likely to receive if he touches a playmate or parent tends to bring home the lesson. But static electricity is more than just fun and games - it is put to use in many industries. The forces between electrically charged particles are used in technologies such as printers, pollution filters, and spray guns used for painting cars and trucks. Static electricity is the study of phenomena that involve an imbalance of electrical charge. Although creating this imbalance typically requires moving charge around, once the imbalance is created, it often remains static for a long time. The study of charge in motion is called electromagnetism and will be covered in a later chapter. What is electrical charge, how is it associated with objects, and what forces does it create? These are just some of the questions that this chapter addresses.

\section*{Teacher Support}
Teacher Support Before students begin this chapter, it would be useful to review the following concepts:

\begin{itemize}
  \item Newton's universal law of gravitation-Review its mathematical form and how it depends on the masses of the two objects involved and on the inverse square of the distance between the objects. Tell students that this chapter presents a similar inverse-square law that describes the force created by electric effects.
  \item Review vectors and vector addition-Remind students that vectors are defined by a magnitude and a direction and that adding vectors of the same magnitude but opposite direction results in zero (i.e., they cancel each other).
\end{itemize}

\subsection*{18.1 Electrical Charges, Conservation of Charge, and Transfer of Charge}
\section*{Section Learning Objectives}
By the end of this section, you will be able to do the following:

\begin{itemize}
  \item Describe positive and negative electric charges
  \item Use conservation of charge to calculate quantities of charge transferred between objects
  \item Characterize materials as conductors or insulators based on their electrical properties
  \item Describe electric polarization and charging by induction
\end{itemize}

\section*{Teacher Support}
Teacher Support The learning objectives in this section will help your students master the following standards

\begin{itemize}
  \item (5) The student knows the nature of forces in the physical world. The student is expected to:
  \item (C) describe and calculate how the magnitude of the electrical force between two objects depends on their charges and the distance between them; and
  \item (E) characterize materials as conductors or insulators based on their electrical properties.
\end{itemize}

In addition, the High School Physics Laboratory Manual addresses content in this section in the lab titled Electric Charge as well as the following standards:

\begin{itemize}
  \item (5) The student knows the nature of forces in the physical world. The student is expected to:
  \item (C) describe and calculate how the magnitude of the electrical force between two objects depends on their charges and the distance between them; and
  \item (E) characterize materials as conductors or insulators based on their electrical properties.
\end{itemize}

\section*{Section Key Terms}
\section*{Electric Charge}
You may know someone who has an electric personality, which usually means that other people are attracted to this person. This saying is based on electric\\
charge, which is a property of matter that causes objects to attract or repel each other. Electric charge comes in two varieties, which we call positive and negative. Like charges repel each other, and unlike charges attract each other. Thus, two positive charges repel each other, as do two negative charges. A positive charge and a negative charge attract each other.

How do we know there are two types of electric charge? When various materials are rubbed together in controlled ways, certain combinations of materials always result in a net charge of one type on one material and a net charge of the opposite type on the other material. By convention, we call one type of charge positive and the other type negative. For example, when glass is rubbed with silk, the glass becomes positively charged and the silk negatively charged. Because the glass and silk have opposite charges, they attract one another like clothes that have rubbed together in a dryer. Two glass rods rubbed with silk in this manner will repel one another, because each rod has positive charge on it. Similarly, two silk cloths rubbed in this manner will repel each other, because both cloths have negative charge. Figure 18.2 shows how these simple materials can be used to explore the nature of the force between charges.

\begin{figure}[h]
\begin{center}
  \includegraphics[max width=\textwidth]{b13713c9-2fd4-4196-9ad9-05e4dd292cb0-05}
\captionsetup{labelformat=empty}
\caption{Figure 18.2 A glass rod becomes positively charged when rubbed with silk, whereas the silk becomes negatively charged. (a) The glass rod is attracted to the silk, because their charges are opposite. (b) Two similarly charged glass rods repel. (c) Two similarly charged silk cloths repel.}
\end{center}
\end{figure}

\section*{Teacher Support}
\section*{Teacher Support}
\section*{Teacher Demonstration}
Prepare a demonstration of static electricity. A simple demonstration may be to charge a glass rod or comb by rubbing it with wool, silk, or other cloth and then charge an inflated balloon by rubbing it on your shirt or hair. Place the balloon on a nonconducting tabletop, and use the glass rod or comb to repel the balloon and make it roll across the tabletop. Amuse the students by pushing the balloon first in one direction and then quickly moving the glass rod or comb to the opposite side of the balloon to make it decelerate and then move in the opposite direction. Ask which type of force is at work between the balloon and\\
the glass rod or comb (a repulsive force).\\[0pt]
It took scientists a long time to discover what lay behind these two types of charges. The word electric itself comes from the Greek word elektron for amber, because the ancient Greeks noticed that amber, when rubbed by fur, attracts dry straw. Almost 2,000 years later, the English physicist William Gilbert proposed a model that explained the effect of electric charge as being due to a mysterious electrical fluid that would pass from one object to another. This model was debated for several hundred years, but it was finally put to rest in 1897 by the work of the English physicist J. J. Thomson and French physicist Jean Perrin. Along with many others, Thomson and Perrin were studying the mysterious cathode rays that were known at the time to consist of particles smaller than the smallest atom. Perrin showed that cathode rays actually carried negative electrical charge. Later, Thomson's work led him to declare, "I can see no escape from the conclusion that [cathode rays] are charges of negative electricity carried by particles of matter."

It took several years of further experiments to confirm Thomson's interpretation of the experiments, but science had in fact discovered the particle that carries the fundamental unit of negative electrical charge. We now know this particle as the electron.

Atoms, however, were known to be electrically neutral, which means that they carry the same amount of positive and negative charge, so their net charge is zero. Because electrons are negative, some other part of the atom must contain positive charge. Thomson put forth what is called the plum pudding model, in which he described atoms as being made of thousands of electrons swimming around in a nebulous mass of positive charge, as shown by the left-side image of Figure 18.3. His student, Ernest Rutherford, originally believed that this model was correct and used it (along with other models) to try to understand the results of his experiments bombarding gold foils with alpha particles (i.e., helium atoms stripped of their electrons). The results, however, did not confirm Thomson's model but rather destroyed it! Rutherford found that most of the space occupied by the gold atoms was actually empty and that almost all of the matter of each atom was concentrated into a tiny, extremely dense nucleus, as shown by the right-side image of Figure 18.3. The atomic nucleus was later found to contain particles called protons, each of which carries a unit of positive electric charge. 1

1\\
Protons were later found to contain sub particles called quarks, which have fractional electric charge. But that is another story that we leave for subsequent physics courses.

\begin{figure}[h]
\begin{center}
  \includegraphics[max width=\textwidth]{b13713c9-2fd4-4196-9ad9-05e4dd292cb0-07}
\captionsetup{labelformat=empty}
\caption{Figure 18.3 The left drawing shows Thompson's plum-pudding model, in which the electrons swim around in a nebulous mass of positive charge. The right drawing shows Rutherford's model, in which the electrons orbit around a tiny, massive nucleus. Note that the size of the nucleus is vastly exaggerated in this drawing. Were it drawn to scale with respect to the size of the electron orbits, the nucleus would not be visible to the naked eye in this drawing. Also, as far as science can currently detect, electrons are point particles, which means that they have no size at all!}
\end{center}
\end{figure}

Protons and electrons are thus the fundamental particles that carry electric charge. Each proton carries one unit of positive charge, and each electron carries one unit of negative charge. To the best precision that modern technology can provide, the charge carried by a proton is exactly the opposite of that carried by an electron. The SI unit for electric charge is the coulomb (abbreviated as "C"), which is named after the French physicist Charles Augustin de Coulomb, who studied the force between charged objects. The proton carries \(+1.602 \times 10^{-19} \mathrm{C}\). and the electron carries \(-1.602 \times 10^{-19} \mathrm{C}\), . The number \(n\) of protons required to make +1.00 C is\\
\(n=1.00 \mathrm{C} \times \frac{1 \text { proton }}{1.602 \times 10^{-19} \mathrm{C}}=6.25 \times 10^{18}\) protons.

\section*{18.1}
The same number of electrons is required to make -1.00 C of electric charge. The fundamental unit of charge is often represented as \(e\). Thus, the charge on a proton is \(e\), and the charge on an electron is \(-e\). Mathematically, \(e= +1.602 \times 10^{-19} \mathrm{C}\).

\section*{Links To Physics}
Measuring the Fundamental Electric Charge The American physicist Robert Millikan (1868-1953) and his student Harvey Fletcher (1884-1981) were the first to make a relatively accurate measurement of the fundamental unit of charge on the electron. They designed what is now a classic experiment performed by students. The Millikan oil-drop experiment is shown in Figure 18.4. The experiment involves some concepts that will be introduced later, but the basic idea is that a fine oil mist is sprayed between two plates that can be charged with a known amount of opposite charge. Some oil drops accumulate\\
some excess negative charge when being sprayed and are attracted to the positive charge of the upper plate and repelled by the negative charge on the lower plate. By tuning the charge on these plates until the weight of the oil drop is balanced by the electric forces, the net charge on the oil drop can be determined quite precisely.

\begin{figure}[h]
\begin{center}
  \includegraphics[max width=\textwidth]{b13713c9-2fd4-4196-9ad9-05e4dd292cb0-08}
\captionsetup{labelformat=empty}
\caption{Figure 18.4 The oil-drop experiment involved spraying a fine mist of oil between two metal plates charged with opposite charges. By knowing the mass of the oil droplets and adjusting the electric charge on the plates, the charge on the oil drops can be determined with precision.}
\end{center}
\end{figure}

Millikan and Fletcher found that the drops would accumulate charge in discrete units of about \(-1.59 \times 10^{-19} \mathrm{C}\), which is within 1 percent of the modern value of \(-1.60 \times 10^{-19} \mathrm{C}\). Although this difference may seem quite small, it is actually five times greater than the possible error Millikan reported for his results!

Because the charge on the electron is a fundamental constant of nature, determining its precise value is very important for all of science. This created pressure on Millikan and others after him that reveals some equally important aspects of human nature.

First, Millikan took sole credit for the experiment and was awarded the 1923 Nobel Prize in physics for this work, although his student Harvey Fletcher apparently contributed in significant ways to the work. Just before his death in 1981, Fletcher divulged that Millikan coerced him to give Millikan sole credit for the work, in exchange for which Millikan promoted Fletcher's career at Bell Labs.

Another great scientist, Richard Feynman, points out that many scientists who measured the fundamental charge after Millikan were reluctant to report values that differed much from Millikan's value. History shows that later measurements slowly crept up from Millikan's value until settling on the modern value. Why did they not immediately find the error and correct the value, asks Feynman. Apparently, having found a value higher than the much-respected value found\\
by Millikan, scientists would look for possible mistakes that might lower their value to make it agree better with Millikan's value. This reveals the important psychological weight carried by preconceived notions and shows how hard it is to refute them. Scientists, however devoted to logic and data they may be, are apparently just as vulnerable to this aspect of human nature as everyone else. The lesson here is that, although it is good to be skeptical of new results, you should not discount them just because they do not agree with conventional wisdom. If your reasoning is sound and your data are reliable, the conclusion demanded by the data must be seriously considered, even if that conclusion disagrees with the commonly accepted truth.

\section*{Grasp Check}
Suppose that Millikan observed an oil drop carrying three fundamental units of charge. What would be the net charge on this oil drop?\\
a. \(-4.81 \times 10^{-19} \mathrm{C}\)\\
b. \(-1.602 \times 10^{-19} \mathrm{C}\)\\
c. \(1.602 \times 10^{-19} \mathrm{C}\)\\
d. \(4.81 \times 10^{-19} \mathrm{C}\)

\section*{Snap Lab}
Like and Unlike Charges This activity investigates the repulsion and attraction caused by static electrical charge.

\begin{itemize}
  \item Adhesive tape
  \item Nonconducting surface, such as a plastic table or chair
\end{itemize}

Instructions\\
Procedure for Part (a)

\begin{enumerate}
  \item Prepare two pieces of tape about 4 cm long. To make a handle, double over about 0.5 cm at one end so that the sticky side sticks together.
  \item Attach the pieces of tape side by side onto a nonmetallic surface, such as a tabletop or the seat of a chair, as shown in Figure 18.5(a).
  \item Peel off both pieces of tape and hang them downward, holding them by the handles, as shown in Figure 18.5(b). If the tape bends upward and sticks to your hand, try using a shorter piece of tape, or simply shake the tape so that it no longer sticks to your hand.
  \item Now slowly bring the two pieces of tape together, as shown in Figure 18.5(c). What happens?
\end{enumerate}

\begin{figure}[h]
\begin{center}
  \includegraphics[max width=\textwidth]{b13713c9-2fd4-4196-9ad9-05e4dd292cb0-10(1)}
\captionsetup{labelformat=empty}
\caption{Figure 18.5}
\end{center}
\end{figure}

Procedure for Part (b)\\
5. Stick one piece of tape on the nonmetallic surface, and stick the second piece of tape on top of the first piece, as shown in Figure 18.6(a).\\
6. Slowly peel off the two pieces by pulling on the handle of the bottom piece.\\
7. Gently stroke your finger along the top of the second piece of tape (i.e., the nonsticky side), as shown in Figure 18.6(b).\\
8. Peel the two pieces of tape apart by pulling on their handles, as shown in Figure 18.6(c).\\
9. Slowly bring the two pieces of tape together. What happens?

\begin{figure}[h]
\begin{center}
\texttt{https://cdn.mathpix.com/cropped/b13713c9-2fd4-4196-9ad9-05e4dd292cb0-10.jpg?height=209&width=225&top_left_y=1194&top_left_x=459}
\captionsetup{labelformat=empty}
\caption{(a)}
\end{center}
\end{figure}

\begin{figure}[h]
\begin{center}
\texttt{https://cdn.mathpix.com/cropped/b13713c9-2fd4-4196-9ad9-05e4dd292cb0-10.jpg?height=209&width=218&top_left_y=1194&top_left_x=760}
\captionsetup{labelformat=empty}
\caption{(b)}
\end{center}
\end{figure}

\begin{figure}[h]
\begin{center}
  \includegraphics[max width=\textwidth]{b13713c9-2fd4-4196-9ad9-05e4dd292cb0-10}
\captionsetup{labelformat=empty}
\caption{(c)}
\end{center}
\end{figure}

Figure 18.6

\section*{Grasp Check}
In step 4, why did the two pieces of tape repel each other? In step 9, why did they attract each other?\\
a. Like charges attract, while unlike charges repel each other.\\
b. Like charges repel, while unlike charges attract each other.\\
c. Tapes having positive charge repel, while tapes having negative charge attract each other.\\
d. Tapes having negative charge repel, while tapes having positive charge attract each other.

\section*{Conservation of Charge}
\section*{Teacher Support}
Teacher Support [BL][OL]Discuss what is meant by conservation in the physics sense. Point out how conservation laws serve as accounting rules that allow us to keep track of certain quantities. This is similar to knowing how many students are on a field trip and using that information to ensure that no\\
students go missing. Because students cannot vanish into thin air, counting the students allows the teacher to know whether any students are not present. If they are not present, then they must be elsewhere, and a search can begin.\\[0pt]
[AL]Ask what other laws of conservation they have encountered in physics, and discuss how these laws are used.

Because the fundamental positive and negative units of charge are carried on protons and electrons, we would expect that the total charge cannot change in any system that we define. In other words, although we might be able to move charge around, we cannot create or destroy it. This should be true provided that we do not create or destroy protons or electrons in our system. In the twentieth century, however, scientists learned how to create and destroy electrons and protons, but they found that charge is still conserved. Many experiments and solid theoretical arguments have elevated this idea to the status of a law. The law of conservation of charge says that electrical charge cannot be created or destroyed.

The law of conservation of charge is very useful. It tells us that the net charge in a system is the same before and after any interaction within the system. Of course, we must ensure that no external charge enters the system during the interaction and that no internal charge leaves the system. Mathematically, conservation of charge can be expressed as\\
\(q_{\text {initial }}=q_{\text {final }}\).\\
18.2\\
where \(q_{\text {initial }}\) is the net charge of the system before the interaction, and \(q_{\text {final }}\), is the net charge after the interaction.

\section*{Worked Example}
What is the missing charge? Figure 18.7 shows two spheres that initially have +4 C and +8 C of charge. After an interaction (which could simply be that they touch each other), the blue sphere has +10 C of charge, and the red sphere has an unknown quantity of charge. Use the law of conservation of charge to find the final charge on the red sphere.

\section*{Strategy}
The net initial charge of the system is \(q_{\text {initial }}=+4 \mathrm{C}+8 \mathrm{C}=+12 \mathrm{C}\). The net final charge of the system is \(q_{\text {final }}=+10 \mathrm{C}+q_{\text {red }}\), where \(q_{\text {red }}\) is the final charge on the red sphere. Conservation of charge tells us that \(q_{\text {initial }}=q_{\text {final }}\), so we can solve for \(q_{\text {red }}\).

Solution\\
Equating \(q_{\text {initial }}\) and \(q_{\text {final }}\) and solving for \(q_{\text {red }}\) gives

\[
\begin{aligned}
q_{\text {initial }} & =q_{\text {final }} \\
+12 \mathrm{C} & =+10 \mathrm{C}+q_{\text {red. }} \\
+2 \mathrm{C} & =q_{\text {red. }}
\end{aligned}
\]

18.3

The red sphere has +2 C of charge.\\
\includegraphics[max width=\textwidth, center]{b13713c9-2fd4-4196-9ad9-05e4dd292cb0-12}

Figure 18.7 Two spheres, one blue and one red, initially have +4 C and +8 C of charge, respectively. After the two spheres interact, the blue sphere has a charge of +10 C . The law of conservation of charge allows us to find the final charge \(q_{\text {red }}\) on the red sphere.

Discussion\\
Like all conservation laws, conservation of charge is an accounting scheme that helps us keep track of electric charge.

\section*{Practice Problems}
1.

Which equation describes conservation of charge?\\
a. \(q_{\text {initial }}=q_{\text {final }}=\) constant\\
b. \(q_{\text {initial }}=q_{\text {final }}=0\)\\
c. \(q_{\text {initial }}-q_{\text {final }}=0\)\\
d. \(q_{\text {initial }} / q_{\text {final }}=\) constant\\
2.

An isolated system contains two objects with charges \(\mathrm{q} \_\{1\}\) and \(\mathrm{q} \_\{2\}\). If object 1 loses half of its charge, what is the final charge on object 2 ?\\
a. \(\backslash\) frac \(\left\{\mathrm{q} \_2\right\}\{2\}\)\\
b. \(\backslash \operatorname{frac}\left\{3 q \_2\right\}\{2\}\)\\
c. \(\mathrm{q} \_2-\backslash\) frac \(\left\{\mathrm{q} \_1\right\}\{2\}\)\\
d. q\_2 + \textbackslash frac\{q\_1\}\{2\}

\section*{Conductors and Insulators}
\section*{Teacher Support}
Teacher Support [BL]Have students define the meaning of conductor and insulator. Explain how these terms are used in physics to mean materials that allow a quantity to pass through and those that do not.\\[0pt]
[OL]Ask students whether they have encountered conductors and insulators in their everyday lives. What are the properties of these materials? Be prepared to discuss and differentiate thermal conductors and insulators.\\[0pt]
[AL]Ask whether students recall other conductors and insulators in physics. Discuss how thermal insulators and conductors function with regard to thermal energy.

Materials can be classified depending on whether they allow charge to move. If charge can easily move through a material, such as metals, then these materials are called conductors. This means that charge can be conducted (i.e., move) through the material rather easily. If charge cannot move through a material, such as rubber, then this material is called an insulator.

Most materials are insulators. Their atoms and molecules hold on more tightly to their electrons, so it is difficult for electrons to move between atoms. However, it is not impossible. With enough energy, it is possible to force electrons to move through an insulator. However, the insulator is often physically destroyed in the process. In metals, the outer electrons are loosely bound to their atoms, so not much energy is required to make electrons move through metal. Such metals as copper, silver, and aluminum are good conductors. Insulating materials include plastics, glass, ceramics, and wood.

The conductivity of some materials is intermediate between conductors and insulators. These are called semiconductors. They can be made conductive under the right conditions, which can involve temperature, the purity of the material, and the force applied to push electrons through them. Because we can control whether semiconductors are conductors or insulators, these materials are used extensively in computer chips. The most commonly used semiconductor is silicon. Figure 18.8 shows various materials arranged according to their ability to conduct electrons.

\begin{figure}[h]
\begin{center}
  \includegraphics[max width=\textwidth]{b13713c9-2fd4-4196-9ad9-05e4dd292cb0-14(1)}
\captionsetup{labelformat=empty}
\caption{Figure 18.8 Materials can be arranged according to their ability to conduct electric charge. The slashes on the arrow mean that there is a very large gap in conducting ability between conductors, semiconductors, and insulators, but the drawing is compressed to fit on the page. The numbers below the materials give their resistivity in \(\Omega \bullet \mathrm{m}\) (which you will learn about below). The resistivity is a measure of how hard it is to make charge move through a given material.}
\end{center}
\end{figure}

\section*{Teacher Support}
Teacher Support Point out that the scale is not linear, which means that the conductivity of the insulators is much, much less than that of conductors. Also point out that semiconductors are often made to act as insulators or as conductors, but not as materials with a conductivity that is between that of insulators and conductors.

What happens if an excess negative charge is placed on a conducting object? Because like charges repel each other, they will push against each other until they are as far apart as they can get. Because the charge can move in a conductor, it moves to the outer surfaces of the object. Figure 18.9(a) shows schematically how an excess negative charge spreads itself evenly over the outer surface of a metal sphere.

What happens if the same is done with an insulating object? The electrons still repel each other, but they are not able to move, because the material is an insulator. Thus, the excess charge stays put and does not distribute itself over the object. Figure 18.9(b) shows this situation.

\begin{figure}[h]
\begin{center}
  \includegraphics[max width=\textwidth]{b13713c9-2fd4-4196-9ad9-05e4dd292cb0-14}
\captionsetup{labelformat=empty}
\caption{(a)}
\end{center}
\end{figure}

\begin{figure}[h]
\begin{center}
\texttt{https://cdn.mathpix.com/cropped/b13713c9-2fd4-4196-9ad9-05e4dd292cb0-14.jpg?height=246&width=248&top_left_y=2017&top_left_x=726}
\captionsetup{labelformat=empty}
\caption{(b)}
\end{center}
\end{figure}

Figure 18.9 (a) A conducting sphere with excess negative charge (i.e., electrons). The electrons repel each other and spread out to cover the outer surface of the sphere. (b) An insulating sphere with excess negative charge. The electrons cannot move, so they remain in their original positions.

\section*{Teacher Support}
Teacher Support Point out that static buildup does not remain forever on an object. Ask students how a static charge may escape from an object. Point out that this static buildup is dissipated faster on humid days than on dry days.

\section*{Transfer and Separation of Charge}
\section*{Teacher Support}
Teacher Support [BL][OL]Ask how the concept of static electricity can be compatible with transfer of charge. Isn't transfer of charge the movement of charge, which contradicts being static?\\[0pt]
[AL]Ask students to define separation of charge. Prepare to explain why this does not mean splitting electrons apart.

Most objects we deal with are electrically neutral, which means that they have the same amount of positive and negative charge. However, transferring negative charge from one object to another is fairly easy to do. When negative charge is transferred from one object to another, an excess of positive charge is left behind. How do we know that the negative charge is the mobile charge? The positive charge is carried by the proton, which is stuck firmly in the nucleus of atoms, and the atoms are stuck in place in solid materials. Electrons, which carry the negative charge, are much easier to remove from their atoms or molecules and can therefore be transferred more easily.

Electric charge can be transferred in several manners. One of the simplest ways to transfer charge is charging by contact, in which the surfaces of two objects made of different materials are placed in close contact. If one of the materials holds electrons more tightly than the other, then it takes some electrons with it when the materials are separated. Rubbing two surfaces together increases the transfer of electrons, because it creates a closer contact between the materials. It also serves to present fresh material with a full supply of electrons to the other material. Thus, when you walk across a carpet on a dry day, your shoes rub against the carpet, and some electrons are removed from the carpet by your shoes. The result is that you have an excess of negative charge on your shoes. When you then touch a doorknob, some of your excess of electrons transfer to the neutral doorknob, creating a small spark.

Touching the doorknob with your hand demonstrates a second way to transfer electric charge, which is charging by conduction. This transfer happens because like charges repel, and so the excess electrons that you picked up from the carpet want to be as far away from each other as possible. Some of them move to the\\
doorknob, where they will distribute themselves over the outer surface of the metal. Another example of charging by conduction is shown in the top row of Figure 18.10. A metal sphere with 100 excess electrons touches a metal sphere with 50 excess electrons, so 25 electrons from the first sphere transfer to the second sphere. Each sphere finishes with 75 excess electrons.

The same reasoning applies to the transfer of positive charge. However, because positive charge essentially cannot move in solids, it is transferred by moving negative charge in the opposite direction. For example, consider the bottom row of Figure 18.10. The first metal sphere has 100 excess protons and touches a metal sphere with 50 excess protons, so the second sphere transfers 25 electrons to the first sphere. These 25 extra electrons will electrically cancel 25 protons so that the first metal sphere is left with 75 excess protons. This is shown in the bottom row of Figure 18.10. The second metal sphere lost 25 electrons so it has 25 more excess protons, for a total of 75 excess protons. The end result is the same if we consider that the first ball transferred a net positive charge equal to that of 25 protons to the first ball.

\begin{figure}[h]
\begin{center}
  \includegraphics[max width=\textwidth]{b13713c9-2fd4-4196-9ad9-05e4dd292cb0-16}
\captionsetup{labelformat=empty}
\caption{Figure 18.10 In the top row, a metal sphere with 100 excess electrons transfers 25 electrons to a metal sphere with an excess of 50 electrons. After the transfer, both spheres have 75 excess electrons. In the bottom row, a metal sphere with 100 excess protons receives 25 electrons from a ball with 50 excess protons. After the transfer, both spheres have 75 excess protons.}
\end{center}
\end{figure}

\section*{Teacher Support}
Teacher Support Point out how the total charge at each instant is the same. Discuss how moving electrons to the right is equivalent to moving the same magnitude of positive charge to the left, but be sure to clarify that, in most situations, only negative charges actually move in solids.\\[0pt]
[BL][OL]Discuss the meaning of polarization in everyday language. For example, discuss what is meant by a polarizing debate or a polarized Congress. Compare and contrast the everyday meaning with the physics meaning.\\[0pt]
[AL]Ask what other examples of polarization they can think of from everyday life.

In this discussion, you may wonder how the excess electrons originally got from your shoes to your hand to create the spark when you touched the doorknob. The answer is that no electrons actually traveled from your shoes to your hands. Instead, because like charges repel each other, the excess electrons on your shoe simply pushed away some of the electrons in your feet. The electrons thus dislodged from your feet moved up into your leg and in turn pushed away some electrons in your leg. This process continued through your whole body until a distribution of excess electrons covered the extremities of your body. Thus your head, your hands, the tip of your nose, and so forth all received their doses of excess electrons that had been pushed out of their normal positions. All this was the result of electrons being pushed out of your feet by the excess electrons on your shoes.

This type of charge separation is called polarization. As soon as the excess electrons leave your shoes (by rubbing off onto the floor or being carried away in humid air), the distribution of electrons in your body returns to normal. Every part of your body is again electrically neutral (i.e., zero excess charge).\\
The phenomenon of polarization is seen in Figure 18.1. The child has accumulated excess positive charge by sliding on the slide. This excess charge repels itself and so becomes distributed over the extremities of the child's body, notably in his hair. As a result, the hair stands on end, because the excess negative charge on each strand repels the excess negative charge on neighboring strands.

Polarization can be used to charge objects. Consider the two metallic spheres shown in Figure 18.11. The spheres are electrically neutral, so they carry the same amounts of positive and negative charge. In the top picture (Figure 18.11(a)), the two spheres are touching, and the positive and negative charge is evenly distributed over the two spheres. We then approach a glass rod that carries an excess positive charge, which can be done by rubbing the glass rod with silk, as shown in Figure 18.11(b). Because opposite charges attract each other, the negative charge is attracted to the glass rod, leaving an excess positive charge on the opposite side of the right sphere. This is an example of charging by induction, whereby a charge is created by approaching a charged object with a second object to create an unbalanced charge in the second object. If we then separate the two spheres, as shown in Figure 18.11(c), the excess charge is stuck on each sphere. The left sphere now has an excess negative charge, and the right sphere has an excess positive charge. Finally, in the bottom picture, the rod is removed, and the opposite charges attract each other, so they move as close together as they can get.

\begin{figure}[h]
\begin{center}
  \includegraphics[max width=\textwidth]{b13713c9-2fd4-4196-9ad9-05e4dd292cb0-18}
\captionsetup{labelformat=empty}
\caption{Figure 18.11 (a) Two neutral conducting spheres are touching each other, so the charge is evenly spread over both spheres. (b) A positively charged rod approaches, which attracts negative charges, leaving excess positive charge on the right sphere. (c) The spheres are separated. Each sphere now carries an equal magnitude of excess charge. (d) When the positively charged rod is removed, the excess negative charge on the left sphere is attracted to the excess positive charge on the right sphere.}
\end{center}
\end{figure}

\section*{Teacher Support}
Teacher Support Discuss the analogous situation with insulating spheres. Point out how the spheres remain neutral despite being polarized in panels (b) and (c).

\section*{Fun In Physics}
Create a Spark in a Science Fair Van de Graaff generators are devices that are used not only for serious physics research but also for demonstrating\\
the physics of static electricity at science fairs and in classrooms. Because they deliver relatively little electric current, they can be made safe for use in such environments. The first such generator was built by Robert Van de Graaff in 1931 for use in nuclear physics research. Figure 18.12 shows a simplified sketch of a Van de Graaff generator.

Van de Graaff generators use smooth and pointed surfaces and conductors and insulators to generate large static charges. In the version shown in Figure 18.12, electrons are "sprayed" from the tips of the lower comb onto a moving belt, which is made of an insulating material like, such as rubber. This technique of charging the belt is akin to charging your shoes with electrons by walking across a carpet. The belt raises the charges up to the upper comb, where they transfer again, akin to your touching the doorknob and transferring your charge to it. Because like charges repel, the excess electrons all rush to the outer surface of the globe, which is made of metal (a conductor). Thus, the comb itself never accumulates too much charge, because any charge it gains is quickly depleted by the charge moving to the outer surface of the globe.

\begin{figure}[h]
\begin{center}
  \includegraphics[max width=\textwidth]{b13713c9-2fd4-4196-9ad9-05e4dd292cb0-19}
\captionsetup{labelformat=empty}
\caption{Figure 18.12 Van de Graaff generators transfer electrons onto a metallic sphere, where the electrons distribute themselves uniformly over the outer surface.}
\end{center}
\end{figure}

Van de Graaff generators are used to demonstrate many interesting effects caused by static electricity. By touching the globe, a person gains excess charge, so his or her hair stands on end, as shown in Figure 18.13. You can also create mini lightning bolts by moving a neutral conductor toward the globe. Another favorite is to pile up aluminum muffin tins on top of the uncharged globe, then turn on the generator. Being made of conducting material, the tins accumulate excess charge. They then repel each other and fly off the globe one by one. A quick Internet search will show many examples of what you can do with a Van de Graaff generator.

\begin{figure}[h]
\begin{center}
  \includegraphics[max width=\textwidth]{b13713c9-2fd4-4196-9ad9-05e4dd292cb0-20}
\captionsetup{labelformat=empty}
\caption{Figure 18.13 The man touching the Van de Graaff generator has excess charge, which spreads over his hair and repels hair strands from his neighbors. (credit: Jon "ShakataGaNai" Davis)}
\end{center}
\end{figure}

\section*{Grasp Check}
Why don't the electrons stay on the rubber belt when they reach the upper comb?\\
a. The upper comb has no excess electrons, and the excess electrons in the rubber belt get transferred to the comb by contact.\\
b. The upper comb has no excess electrons, and the excess electrons in the rubber belt get transferred to the comb by conduction.\\
c. The upper comb has excess electrons, and the excess electrons in the rubber belt get transferred to the comb by conduction.\\
d. The upper comb has excess electrons, and the excess electrons in the rubber belt get transferred to the comb by contact.

\section*{Virtual Physics}
Balloons and Static Electricity Click to view content\\
This simulation allows you to observe negative charge accumulating on a balloon as you rub it against a sweater. You can then observe how two charged balloons interact and how they cause polarization in a wall.

\section*{Grasp Check}
Click the reset button, and start with two balloons. Charge a first balloon by rubbing it on the sweater, and then move it toward the second balloon. Why does the second balloon not move?\\
a. The second balloon has an equal number of positive and negative charges.\\
b. The second balloon has more positive charges than negative charges.\\
c. The second balloon has more negative charges than positive charges.\\
d. The second balloon is positively charged and has polarization.

\section*{Snap Lab}
Polarizing Tap Water This lab will demonstrate how water molecules can easily be polarized.

\begin{itemize}
  \item Plastic object of small dimensions, such as comb or plastic stirrer
  \item Source of tap water
\end{itemize}

Instructions\\
Procedure

\begin{enumerate}
  \item Thoroughly rub the plastic object with a dry cloth.
  \item Open the faucet just enough to let a smooth filament of water run from the tap.
  \item Move an edge of the charged plastic object toward the filament of running water.
\end{enumerate}

What do you observe? What happens when the plastic object touches the water filament? Can you explain your observations?

Why does the water curve around the charged object?\\
a. The charged object induces uniform positive charge on the water molecules.\\
b. The charged object induces uniform negative charge on the water molecules.\\
c. The charged object attracts the polarized water molecules and ions that are dissolved in the water.\\
d. The charged object depolarizes the water molecules and the ions dissolved in the water.

\section*{Worked Example}
Charging Ink Droplets Electrically neutral ink droplets in an ink-jet printer pass through an electron beam created by an electron gun, as shown in Figure 18.14. Some electrons are captured by the ink droplet, so that it becomes charged. After passing through the electron beam, the net charge of the ink droplet is \(q_{\text {inkdrop }}=-1 \times 10^{-10} \mathrm{C}\). How many electrons are captured by the ink droplet?

\begin{figure}[h]
\begin{center}
  \includegraphics[max width=\textwidth]{b13713c9-2fd4-4196-9ad9-05e4dd292cb0-22}
\captionsetup{labelformat=empty}
\caption{Figure 18.14 Electrons from an electron gun charge a passing ink droplet.}
\end{center}
\end{figure}

\section*{Strategy}
A single electron carries a charge of \(q_{e^{-}}=-1.602 \times 10^{-19} \mathrm{C}\). Dividing the net charge of the ink droplet by the charge \(q_{e^{-}}\)of a single electron will give the number of electrons captured by the ink droplet.

Solution\\
The number \(n\) of electrons captured by the ink droplet are\\
\(n=\frac{q_{\text {inkdrop }}}{q_{e^{-}}}=\frac{-1 \times 10^{-10} \mathrm{C}}{-1.602 \times 10^{-19} \mathrm{C}}=6 \times 10^{8}\).\\
18.4

Discussion\\
This is almost a billion electrons! It seems like a lot, but it is quite small compared to the number of atoms in an ink droplet, which number about \(10^{16}\). Thus, each extra electron is shared between about \(10^{16} /\left(6 \times 10^{8}\right) \approx 10^{7}\) atoms.

\section*{Practice Problems}
3.

How many protons are needed to make 1 nC of charge? \(1 \mathrm{nC}=10-9 \mathrm{C}\)\\
a. \(1.6 \times 10^{-28}\)\\
b. \(1.6 \times 10^{-10}\)\\
c. \(3 \times 10^{9}\)\\
d. \(6 \times 10^{9}\)\\
4.

In a physics lab, you charge up three metal spheres, two with \(+3 \backslash, \backslash \operatorname{text}\{\mathrm{nC}\}\) and one with - \(5 \backslash, \backslash \operatorname{text}\{\mathrm{nC}\}\). When you bring all three spheres together so that they all touch one another, what is the total charge on the three spheres?\\
a. \(+1 \backslash, \backslash \operatorname{text}\{\mathrm{nC}\}\)\\
b. \(+3 \backslash, \backslash \operatorname{text}\{\mathrm{nC}\}\)\\
c. \(+5 \backslash, \backslash \operatorname{text}\{\mathrm{nC}\}\)\\
d. \(+6 \backslash, \backslash \operatorname{text}\{\mathrm{nC}\}\)

\section*{Check Your Understanding}
5.

How many types of electric charge exist?\\
a. one type\\
b. two types\\
c. three types\\
d. four types\\
6.

Which are the two main electrical classifications of materials based on how easily charges can move through them?\\
a. conductor and insulator\\
b. semiconductor and insulator\\
c. conductor and superconductor\\
d. conductor and semiconductor\\
7.

True or false - A polarized material must have a nonzero net electric charge.\\
a. true\\
b. false\\
8.

Describe the force between two positive point charges that interact.\\
a. The force is attractive and acts along the line joining the two point charges.\\
b. The force is attractive and acts tangential to the line joining the two point charges.\\
c. The force is repulsive and acts along the line joining the two point charges.\\
d. The force is repulsive and acts tangential to the line joining the two point charges.\\
9.

How does a conductor differ from an insulator?\\
a. Electric charges move easily in an insulator but not in a conducting material.\\
b. Electric charges move easily in a conductor but not in an insulator.\\
c. A conductor has a large number of electrons.\\
d. More charges are in an insulator than in a conductor.\\
10.

True or false-Charging an object by polarization requires touching it with an object carrying excess charge.\\
a. true\\
b. false

\subsection*{18.2 Coulomb's la}
\section*{Section Learning Objectives}
By the end of this section, you will be able to do the following:

\begin{itemize}
  \item Describe Coulomb's law verbally and mathematically
  \item Solve problems involving Coulomb's law
\end{itemize}

\section*{Teacher Support}
Teacher Support The learning objectives in this section will help your students master the following standards:

\begin{itemize}
  \item (5) The student knows the nature of forces in the physical world. The student is expected to:
  \item (C) describe and calculate how the magnitude of the electrical force between two objects depends on their charges and the distance between them.
\end{itemize}

\section*{Section Key Terms}
\section*{Teacher Support}
Teacher Support This section presents Coulomb's law and points out its similarities and differences with respect to Newton's law of universal gravitation. The similarities include the inverse-square nature of the two laws and the analogous roles of mass and charge. The differences include the restriction of positive mass versus positive or negative charge.\\[0pt]
[BL][OL]Discuss how Coulomb described this law long after Newton described the law of universal gravitation.\\[0pt]
[AL]Ask why the law of force between electrostatic charge was discovered after that of gravity if gravity is weak compared to electrostatic forces.

More than 100 years before Thomson and Rutherford discovered the fundamental particles that carry positive and negative electric charges, the French scientist Charles-Augustin de Coulomb mathematically described the force between charged objects. Doing so required careful measurements of forces between charged spheres, for which he built an ingenious device called a torsion balance.

This device, shown in Figure 18.15, contains an insulating rod that is hanging by a thread inside a glass-walled enclosure. At one end of the rod is the metallic sphere A . When no charge is on this sphere, it touches sphere B. Coulomb would touch the spheres with a third metallic ball (shown at the bottom of the\\
diagram) that was charged. An unknown amount of charge would distribute evenly between spheres A and B , which would then repel each other, because like charges repel. This force would cause sphere A to rotate away from sphere B , thus twisting the wire until the torsion in the wire balanced the electrical force. Coulomb then turned the knob at the top, which allowed him to rotate the thread, thus bringing sphere A closer to sphere B . He found that bringing sphere A twice as close to sphere B required increasing the torsion by a factor of four. Bringing the sphere three times closer required a ninefold increase in the torsion. From this type of measurement, he deduced that the electrical force between the spheres was inversely proportional to the distance squared between the spheres. In other words,\\
\(F \propto \frac{1}{r^{2}}\)\\
18.5\\
where \(r\) is the distance between the spheres.\\
An electrical charge distributes itself equally between two conducting spheres of the same size. Knowing this allowed Coulomb to divide an unknown charge in half. Repeating this process would produce a sphere with one quarter of the initial charge, and so on. Using this technique, he measured the force between spheres A and B when they were charged with different amounts of charge. These measurements led him to deduce that the force was proportional to the charge on each sphere, or\\
\(F \propto q_{\mathrm{A}} q_{\mathrm{B}}\),\\
18.6\\
where \(q_{\mathrm{A}}\) is the charge on sphere A , and \(q_{\mathrm{B}}\) is the charge on sphere B .\\
\includegraphics[max width=\textwidth, center]{b13713c9-2fd4-4196-9ad9-05e4dd292cb0-27}\\
\includegraphics[max width=\textwidth, center]{b13713c9-2fd4-4196-9ad9-05e4dd292cb0-27(1)}

Figure 18.15 A drawing of Coulomb's torsion balance, which he used to measure the electrical force between charged spheres. (credit: Charles-Augustin de Coulomb)

Combining these two proportionalities, he proposed the following expression to describe the force between the charged spheres.\\
\(F=\frac{k q_{1} q_{2}}{r^{2}}\)\\
18.7

This equation is known as Coulomb's law, and it describes the electrostatic force between charged objects. The constant of proportionality \(k\) is called Coulomb \(s\) constant. In SI units, the constant \(k\) has the value \(k=8.99 \times 10^{9} \mathrm{~N} \cdot \mathrm{~m}^{2} / \mathrm{C}^{2}\).

The direction of the force is along the line joining the centers of the two objects. If the two charges are of opposite signs, Coulomb's law gives a negative result. This means that the force between the particles is attractive. If the two charges have the same signs, Coulomb's law gives a positive result. This means that the force between the particles is repulsive. For example, if both \(q_{1}\) and \(q_{2}\) are negative or if both are positive, the force between them is repulsive. This is shown in Figure 18.16(a). If \(q_{1}\) is a negative charge and \(q_{2}\) is a positive charge (or vice versa), then the charges are different, so the force between them is attractive. This is shown in Figure 18.16(b).

\begin{figure}[h]
\begin{center}
  \includegraphics[max width=\textwidth]{b13713c9-2fd4-4196-9ad9-05e4dd292cb0-28}
\captionsetup{labelformat=empty}
\caption{Figure 18.16 The magnitude of the electrostatic force \(F\) between point charges \(q_{1}\) and \(q_{2}\) separated by a distance \(r\) is given by Coulomb's law. Note that Newton's third law (every force exerted creates an equal and opposite force) applies as usual-the force ( \(F_{1,2}\) ) on \(q_{1}\) is equal in magnitude and opposite in direction to the force ( \(F_{2,1}\) ) it exerts on \(q_{2}\). (a) Like charges. (b) Unlike charges.}
\end{center}
\end{figure}

\section*{Teacher Support}
Teacher Support Point out how the subscripts \({ }_{1},{ }_{2}\) means the force on object 1 due to object 2 (and vice versa).

Note that Coulomb's law applies only to charged objects that are not moving with respect to each other. The law says that the force is proportional to the amount of charge on each object and inversely proportional to the square of the distance between the objects. If we double the charge \(q_{1}\), for instance, then the force is doubled. If we double the distance between the objects, then the force between them decreases by a factor of \(2^{2}=4\). Although Coulomb's law is true in general, it is easiest to apply to spherical objects or to objects that are much\\
smaller than the distance between the objects (in which case, the objects can be approximated as spheres).

Coulomb's law is an example of an inverse-square law, which means the force depends on the square of the denominator. Another inverse-square law is Newton's law of universal gravitation, which is \(F=G m_{1} m_{2} / r^{2}\). Although these laws are similar, they differ in two important respects: (i) The gravitational constant \(G\) is much, much smaller than \(k\)\\
( \(G=6.67 \times 10^{-11} \mathrm{~m}^{3} / \mathrm{kg} \cdot \mathrm{s}^{2}\) ); and (ii) only one type of mass exists, whereas two types of electric charge exist. These two differences explain why gravity is so much weaker than the electrostatic force and why gravity is only attractive, whereas the electrostatic force can be attractive or repulsive.

Finally, note that Coulomb measured the distance between the spheres from the centers of each sphere. He did not explain this assumption in his original papers, but it turns out to be valid. From outside a uniform spherical distribution of charge, it can be treated as if all the charge were located at the center of the sphere.

\section*{Watch Physics}
Electrostatics (part 1): Introduction to charge and Coulomb's law This video explains the basics of Coulomb's law. Note that the lecturer uses \(d\) for the distance between the center of the particles instead of \(r\).

Click to view content

\section*{Grasp Check}
True or false-If one particle carries a positive charge and another carries a negative charge, then the force between them is attractive.\\
a. true\\
b. false

\section*{Snap Lab}
Hovering plastic In this lab, you will use electrostatics to hover a thin piece of plastic in the air.

\begin{itemize}
  \item Balloon
  \item Light plastic bag (e.g., produce bag from grocery store)
\end{itemize}

Instructions\\
Procedure

\begin{enumerate}
  \item Cut the plastic bag to make a plastic loop about 2 inches wide.
  \item Inflate the balloon.
  \item Charge the balloon by rubbing it on your clothes.
  \item Charge the plastic loop by placing it on a nonmetallic surface and rubbing it with a cloth.
  \item Hold the balloon in one hand, and in the other hand hold the plastic loop above the balloon. If the loop clings too much to your hand, recruit a friend to hold the strip above the balloon with both hands. Now let go of the plastic loop, and maneuver the balloon under the plastic loop to keep it hovering in the air above the balloon.
\end{enumerate}

\section*{Grasp Check}
How does the balloon keep the plastic loop hovering?\\
a. The balloon and the loop are both negatively charged. This will help the balloon keep the plastic loop hovering.\\
b. The balloon is charged, while the plastic loop is neutral.This will help the balloon keep the plastic loop hovering.\\
c. The balloon and the loop are both positively charged. This will help the balloon keep the plastic loop hovering.\\
d. The balloon is positively charged, while the plastic loop is negatively charged. This will help the balloon keep the plastic loop hovering.

\section*{Worked Example}
Using Coulomb s law to find the force between charged objects Suppose Coulomb measures a force of \(20 \times 10^{-6} \mathrm{~N}\) between the two charged spheres when they are separated by 5.0 cm . By turning the dial at the top of the torsion balance, he approaches the spheres so that they are separated by 3.0 cm . Which force does he measure now?

\section*{Strategy}
Apply Coulomb's law to the situation before and after the spheres are brought closer together. Although we do not know the charges on the spheres, we do know that they remain the same. We call these unknown but constant charges \(q_{1}\) and \(q_{2}\). Because these charges appear as a product in Coulomb's law, they form a single unknown. We thus have two equations and two unknowns, which we can solve. The first unknown is the force (which we call \(F_{\mathrm{f}}\) ) when the spheres are 3.0 cm apart, and the second is \(q_{1} q_{2}\).

Use the following notation: When the charges are 5.0 cm apart, the force is \(F_{\mathrm{i}}=20 \times 10^{-6} \mathrm{~N}\) and \(r_{\mathrm{i}}=5.0 \mathrm{~cm}=0.050 \mathrm{~m}\), where the subscript \(i\) means initial. Once the charges are brought closer together, we know \(r_{\mathrm{f}}=3.0 \mathrm{~cm}=0.030 \mathrm{~m}\), where the subscript \(f\) means nal

Solution

Coulomb's law applied to the spheres in their initial positions gives\\
\(F_{\mathrm{i}}=\frac{k q_{1} q_{2}}{r_{\mathrm{i}}^{2}}\).\\
18.8

Coulomb's law applied to the spheres in their final positions gives\\
\(F_{\mathrm{f}}=\frac{k q_{1} q_{2}}{r_{\mathrm{f}}^{2} .}\).\\
18.9

Dividing the second equation by the first and solving for the final force \(F_{\mathrm{f}}\) leads to

\[
\begin{aligned}
\frac{F_{\mathrm{f}}}{F_{\mathrm{i}}} & =\frac{k q_{1} q_{2} / r_{\mathrm{f}}^{2}}{k q_{1} q_{2} / r_{\mathrm{i}}^{2}} \\
& =\frac{r_{\mathrm{i}}^{2}}{r_{\mathrm{f}}^{2}} \\
F_{\mathrm{f}} & =F_{\mathrm{i}} \frac{r_{\mathrm{i}}^{2}}{r_{\mathrm{f}}^{2}}
\end{aligned}
\]

18.10

Inserting the known quantities yields

\[
\begin{aligned}
F_{\mathrm{f}} & =\quad F_{\mathrm{i}} \frac{r_{\mathrm{i}}^{2}}{r_{\mathrm{f}}^{2}} \\
& =\left(20 \times 10^{-6} \mathrm{~N}\right) \frac{(0.050 \mathrm{~m})^{2}}{(0.030 \mathrm{~m})^{2}} \\
& =56 \times 10^{-6} \mathrm{~N} \\
& =5.6 \times 10^{-5} \mathrm{~N}
\end{aligned}
\]

18.11

The force acts along the line joining the centers of the spheres. Because the same type of charge is on each sphere, the force is repulsive.

Discussion\\
As expected, the force between the charges is greater when they are 3.0 cm apart than when they are 5.0 cm apart. Note that although it is a good habit to convert cm to m (because the constant \(k\) is in SI units), it is not necessary in this problem, because the distances cancel out.

We can also solve for the second unknown \(\left|q_{1} q_{2}\right|\). By using the first equation, we find

\[
\begin{aligned}
F_{\mathrm{f}} & =\frac{k q_{1} q_{2}}{r_{\mathrm{i}}^{2}} \\
q_{1} q_{2} & =\frac{F_{\mathrm{i}} r_{\mathrm{i}}^{2}}{k} \\
& =\frac{\left(20 \times 10^{-6} \mathrm{~N}\right)(0.050 \mathrm{~m})^{2}}{8.99 \times 10^{9} \mathrm{~N} \cdot \mathrm{~m}^{2} / \mathrm{C}^{2}} \\
& =5.6 \times 10^{-18} \mathrm{C}^{2}
\end{aligned}
\]

18.12

Note how the units cancel in the second-to-last line. Had we not converted cm to m , this would not occur, and the result would be incorrect. Finally, because the charge on each sphere is the same, we can further deduce that\\
\(q_{1}=q_{2}= \pm \sqrt{5.6 \times 10^{-18} \mathrm{C}^{2}}= \pm 2.4 \mathrm{nC}\).\\
18.13

\section*{Worked Example}
Using Coulomb s law to find the distance between charged objects An engineer measures the force between two ink drops by measuring their acceleration and their diameter. She finds that each member of a pair of ink drops exerts a repulsive force of \(F=5.5 \mathrm{mN}\) on its partner. If each ink drop carries a charge \(q_{\text {inkdrop }}=-1 \times 10^{-10} \mathrm{C}\), how far apart are the ink drops?

\section*{Strategy}
We know the force and the charge on each ink drop, so we can solve Coulomb's law for the distance \(r\) between the ink drops. Do not forget to convert the force into SI units: \(F=5.5 \mathrm{mN}=5.5 \times 10^{-3} \mathrm{~N}\).

Solution\\
The charges in Coulomb's law are \(q_{1}=q_{2}=q_{\text {inkdrop }}\), so the numerator in Coulomb's law takes the form \(q_{1} q_{2}=q_{\text {inkdrop }}^{2}\). Inserting this into Coulomb's law and solving for the distance \(r\) gives

\[
\begin{aligned}
F & =\frac{k q_{\text {inkdrop }}^{2}}{r^{2}} \\
r & = \pm \sqrt{\frac{k q_{\text {inkdrop }}^{2}}{F}} \\
& = \pm \sqrt{\frac{\left(8.99 \times 10^{9} \mathrm{~N} \cdot \mathrm{~m}^{2} / \mathrm{C}^{2}\right)\left(-1 \times 10^{-10} \mathrm{C}\right)^{2}}{5.5 \times 10^{-3} \mathrm{~N}}} \\
& = \pm 1.3 \times 10^{-4} \mathrm{~m}
\end{aligned}
\]

18.14\\
or 130 microns (about one-tenth of a millimeter).\\
Discussion\\
The plus-minus sign means that we do not know which ink drop is to the right and which is to the left, but that is not important, because both ink drops are the same.

\section*{Practice Problems}
11.

A charge of \(-4 \times 10^{-9} \mathrm{C}\) is a distance of 3 cm from a charge of \(3 \times 10^{-9} \mathrm{C}\). What is the magnitude and direction of the force between them?\\
a. \(1.2 \times 10^{-4} \mathrm{~N}\), and the force is attractive\\
b. \(1.2 \times 10^{14} \mathrm{~N}\), and the force is attractive\\
c. \(6.74 \times 10^{23} \mathrm{~N}\), and the force is attractive\\
d. - , and the force is attractive\\
12.

Two charges are repelled by a force of 2.0 N . If the distance between them triples, what is the force between the charges?\\
a. 0.22 N\\
b. 0.67 N\\
c. 2.0 N\\
d. 18.0 N

\section*{Check Your Understanding}
13.

How are electrostatic force and charge related?\\
a. The force is proportional to the product of two charges.\\
b. The force is inversely proportional to the product of two charges.\\
c. The force is proportional to any one of the charges between which the force is acting.\\
d. The force is inversely proportional to any one of the charges between which the force is acting.\\
14.

Why is Coulomb's law called an inverse-square law?\\
a. because the force is proportional to the inverse of the distance squared between charges\\
b. because the force is proportional to the product of two charges\\
c. because the force is proportional to the inverse of the product of two charges\\
d. because the force is proportional to the distance squared between charges

\subsection*{18.3 Electric Field}
\section*{Section Learning Objectives}
By the end of this section, you will be able to do the following:

\begin{itemize}
  \item Calculate the strength of an electric field
  \item Create and interpret drawings of electric fields
\end{itemize}

\section*{Teacher Support}
Teacher Support The learning objectives in this section will help your students master the following standards:

\begin{itemize}
  \item (5) The student knows the nature of forces in the physical world. The student is expected to:
  \item (C) describe and calculate how the magnitude of the electrical force between two objects depends on their charges and the distance between them.
\end{itemize}

\section*{Section Key Terms}
\section*{Teacher Support}
Teacher Support Ask students whether they have seen movies that use the concept of elds as in force elds Have them describe how such fields work. Describe how gravity can be thought of as a field that surrounds a mass and with which other masses interact. Explain that electric fields are very similar to gravitational fields.

You may have heard of a force eld in science fiction movies, where such fields apply forces at particular positions in space to keep a villain trapped or to protect a spaceship from enemy fire. The concept of a eld is very useful in physics, although it differs somewhat from what you see in movies.

A eld is a way of conceptualizing and mapping the force that surrounds any object and acts on another object at a distance without apparent physical connection. For example, the gravitational field surrounding Earth and all other masses represents the gravitational force that would be experienced if another mass were placed at a given point within the field. Michael Faraday, an English physicist of the nineteenth century, proposed the concept of an electric field. If you know the electric field, then you can easily calculate the force (magnitude and direction) applied to any electric charge that you place in the field.

An electric field is generated by electric charge and tells us the force per unit charge at all locations in space around a charge distribution. The charge distribution could be a single point charge; a distribution of charge over, say, a flat plate; or a more complex distribution of charge. The electric field extends into space around the charge distribution. Now consider placing a test charge in the field. A test charge is a positive electric charge whose charge is so small that it does not significantly disturb the charges that create the electric field. The electric field exerts a force on the test charge in a given direction. The force exerted is proportional to the charge of the test charge. For example, if we double the charge of the test charge, the force exerted on it doubles. Mathematically, saying that electric field is the force per unit charge is written as\\
\(\vec{E}=\frac{\vec{F}}{q_{\text {test }}}\)\\
18.15\\
where we are considering only electric forces. Note that the electric field is a vector field that points in the same direction as the force on the positive test charge. The units of electric field are N/C.

If the electric field is created by a point charge or a sphere of uniform charge, then the magnitude of the force between this point charge \(Q\) and the test charge is given by Coulomb's law\\
\(F=\frac{k\left|Q q_{\text {test }}\right|}{r^{2}}\)\\
where the absolute value is used, because we only consider the magnitude of the force. The magnitude of the electric field is then\\
\(E=\frac{F}{q_{\text {test }}}=\frac{k|Q|}{r^{2}}\).\\
18.16

This equation gives the magnitude of the electric field created by a point charge \(Q\). The distance \(r\) in the denominator is the distance from the point charge, \(Q\), or from the center of a spherical charge, to the point of interest.

If the test charge is removed from the electric field, the electric field still exists. To create a three-dimensional map of the electric field, imagine placing the test charge in various locations in the field. At each location, measure the force on the charge, and use the vector equation \(\vec{E}=\vec{F} / q_{\text {test }}\) to calculate the electric field. Draw an arrow at each point where you place the test charge to represent the strength and the direction of the electric field. The length of the arrows should be proportional to the strength of the electric field. If you join together these arrows, you obtain lines. Figure 18.17 shows an image of the three-dimensional electric field created by a positive charge.

\begin{figure}[h]
\begin{center}
  \includegraphics[max width=\textwidth]{b13713c9-2fd4-4196-9ad9-05e4dd292cb0-37}
\captionsetup{labelformat=empty}
\caption{Figure 18.17 Three-dimensional representation of the electric field generated by a positive charge.}
\end{center}
\end{figure}

\section*{Teacher Support}
Teacher Support [BL][OL]Point out that all electric field lines originate from the charge.\\[0pt]
[AL]Point out that the number of lines crossing an imaginary sphere surrounding the charge is the same no matter what size sphere you choose. Ask whether students can use this to show that the number of field lines crossing a surface per unit area shows that the electric field strength decreases as the inverse square of the distance.

Just drawing the electric field lines in a plane that slices through the charge gives the two-dimensional electric-field maps shown in Figure 18.18. On the left is the electric field created by a positive charge, and on the right is the electric field created by a negative charge.

Notice that the electric field lines point away from the positive charge and toward the negative charge. Thus, a positive test charge placed in the electric field of the positive charge will be repelled. This is consistent with Coulomb's law, which says that like charges repel each other. If we place the positive charge in the electric field of the negative charge, the positive charge is attracted to the negative charge. The opposite is true for negative test charges. Thus, the direction of the electric field lines is consistent with what we find by using Coulomb's law.

The equation \(E=k|Q| / r^{2}\) says that the electric field gets stronger as we approach the charge that generates it. For example, at 2 cm from the charge \(Q\) ( \(r =2 \mathrm{~cm}\) ), the electric field is four times stronger than at 4 cm from the charge \((r=4 \mathrm{~cm})\). Looking at Figure 18.17 and Figure 18.18 again, we see that the electric field lines become denser as we approach the charge that generates it. In fact, the density of the electric field lines is proportional to the strength of the electric field!

\begin{figure}[h]
\begin{center}
  \includegraphics[max width=\textwidth]{b13713c9-2fd4-4196-9ad9-05e4dd292cb0-38}
\captionsetup{labelformat=empty}
\caption{Figure 18.18 Electric field lines from two point charges. The red point on the left carries a charge of +1 nC , and the blue point on the right carries a charge of -1 nC . The arrows point in the direction that a positive test charge would move. The field lines are denser as you approach the point charge.}
\end{center}
\end{figure}

Electric-field maps can be made for several charges or for more complicated charge distributions. The electric field due to multiple charges may be found by adding together the electric field from each individual charge. Because this sum can only be a single number, we know that only a single electric-field line can go through any given point. In other words, electric-field lines cannot cross each other.

Figure 18.19(a) shows a two-dimensional map of the electric field generated by a charge of \(+q\) and a nearby charge of \(-q\). The three-dimensional version of this map is obtained by rotating this map about the axis that goes through both charges. A positive test charge placed in this field would experience a force in the direction of the field lines at its location. It would thus be repelled from the positive charge and attracted to the negative charge. Figure 18.19(b) shows the electric field generated by two charges of \(-q\). Note how the field lines tend to repel each other and do not overlap. A positive test charge placed in this field would be attracted to both charges. If you are far from these two charges, where far means much farther than the distance between the charges, the electric field looks like the electric field from a single charge of \(-2 q\).

\begin{figure}[h]
\begin{center}
  \includegraphics[max width=\textwidth]{b13713c9-2fd4-4196-9ad9-05e4dd292cb0-39}
\captionsetup{labelformat=empty}
\caption{Figure 18.19 (a) The electric field generated by a positive point charge (left) and a negative point charge of the same magnitude (right). (b) The electric field generated by two equal negative charges.}
\end{center}
\end{figure}

\section*{Teacher Support}
Teacher Support Ask students to interpret the electric field maps. Where is the field strongest? Where is the field weakest? In which direction is the field increasing or decreasing? Where is the field the most uniform? Can they verify that the magnitude of the charges is the same in a given panel? How does the field for the two negative charges differ from that of the positive and negative charges?

\section*{Virtual Physics}
Probing an Electric Field Click to view content\\
This simulation shows you the electric field due to charges that you place on the screen. Start by clicking the top checkbox in the options panel on the right-hand side to show the electric field. Drag charges from the buckets onto the screen, move them around, and observe the electric field that they form. To see more precisely the magnitude and direction of the electric field, drag an electric-field sensor, or \(E\) - eldsensor from the bottom bucket, and move it around the screen.

PhET Explorations: Charges and Fields. Move point charges around on the playing field and then view the electric field, voltages, equipotential lines, and more.

Click to view content\\
Two positive charges are placed on a screen. Which statement describes the electric field produced by the charges?\\
a. It is constant everywhere.\\
b. It is zero near each charge.\\
c. It is zero halfway between the charges.\\
d. It is strongest halfway between the charges.

\section*{Watch Physics}
Electrostatics (part 2): Interpreting electric field This video explains how to calculate the electric field of a point charge and how to interpret electricfield maps in general. Note that the lecturer uses \(d\) for the distance between particles instead of \(r\). Note that the point charges are infinitesimally small, so all their charges are focused at a point. When larger charged objects are considered, the distance between the objects must be measured between the center of the objects.

Click to view content

\section*{Grasp Check}
True or false-If a point charge has electric field lines that point into it, the charge must be positive.\\
a. true\\
b. false

\section*{Worked Example}
What is the charge? Look at the drawing of the electric field in Figure 18.20. What is the relative strength and sign of the three charges?

\begin{figure}[h]
\begin{center}
  \includegraphics[max width=\textwidth]{b13713c9-2fd4-4196-9ad9-05e4dd292cb0-41}
\captionsetup{labelformat=empty}
\caption{Figure 18.20 Map of electric field due to three charged particles.}
\end{center}
\end{figure}

\section*{Strategy}
We know the electric field extends out from positive charge and terminates on negative charge. We also know that the number of electric field lines that touch a charge is proportional to the charge. Charge 1 has 12 fields coming out of it. Charge 2 has six field lines going into it. Charge 3 has 12 field lines going into it.

\section*{Solution}
The electric-field lines come out of charge 1, so it is a positive charge. The electric-field lines go into charges 2 and 3, so they are negative charges. The ratio of the charges is \(q_{1}: q_{2}: q_{3}=+12:-6:-12\). Thus, magnitude of charges 1 and 3 is twice that of charge 2.

Discussion\\
Although we cannot determine the precise charge on each particle, we can get a lot of information from the electric field regarding the magnitude and sign of the charges and where the force on a test charge would be greatest (or least).

\section*{Worked Example}
Electric field from doorknob A doorknob, which can be taken to be a spherical metal conductor, acquires a static electricity charge of \(q=-1.5 \mathrm{nC}\). What is the electric field 1.0 cm in front of the doorknob? The diameter of the doorknob is 5.0 cm .

\section*{Strategy}
Because the doorknob is a conductor, the entire charge is distributed on the outside surface of the metal. In addition, because the doorknob is assumed to be perfectly spherical, the charge on the surface is uniformly distributed, so we can treat the doorknob as if all the charge were located at the center of the doorknob. The validity of this simplification will be proved in a later physics\\
course. Now sketch the doorknob, and define your coordinate system. Use \(+x\) to indicate the outward direction perpendicular to the door, with \(x=0\) at the center of the doorknob (as shown in the figure below).\\
\includegraphics[max width=\textwidth, center]{b13713c9-2fd4-4196-9ad9-05e4dd292cb0-42}

If the diameter of the doorknob is 5.0 cm , its radius is 2.5 cm . We want to know the electric field 1.0 cm from the surface of the doorknob, which is a distance \(r=2.5 \mathrm{~cm}+1.0 \mathrm{~cm}=3.5 \mathrm{~cm}\) from the center of the doorknob. We can use the equation \(E=\frac{k|Q|}{r^{2}}\) to find the magnitude of the electric field. The direction of the electric field is determined by the sign of the charge, which is negative in this case.

Solution\\
Inserting the charge \(Q=-1.5 \mathrm{nC}=-1.5 \times 10^{-9} \mathrm{C}\) and the distance \(r= 3.5 \mathrm{~cm}=0.035 \mathrm{~m}\) into the equation \(E=\frac{k|Q|}{r^{2}}\) gives

\[
\begin{aligned}
E & =\frac{k|Q|}{r^{2}} \\
& =\frac{\left(8.99 \times 10^{9} \mathrm{~N} \cdot \mathrm{~m}^{2} / \mathrm{C}^{2}\right)\left|-1.5 \times 10^{-9} \mathrm{C}\right|}{(0.035 \mathrm{~m})^{2}} \\
& =1.1 \times 10^{4} \mathrm{~N} / \mathrm{C}
\end{aligned}
\]

\subsection*{18.17}
Because the charge is negative, the electric-field lines point toward the center of the doorknob. Thus, the electric field at \(x=3.5 \mathrm{~cm}\) is \(\left(-1.1 \times 10^{4} \mathrm{~N} / \mathrm{C}\right) \widehat{x}\).

Discussion\\
This seems like an enormous electric field. Luckily, it takes an electric field roughly 100 times stronger ( \(3 \times 10^{6} \mathrm{~N} / \mathrm{C}\) ) to cause air to break down and conduct electricity. Also, the weight of an adult is about \(70 \mathrm{~kg} \times 9.8 \mathrm{~m} / \mathrm{s}^{2} \approx\) 700 N , so why don't you feel a force on the protons in your hand as you reach for the doorknob? The reason is that your hand contains an equal amount of negative charge, which repels the negative charge in the doorknob. A very small\\
force might develop from polarization in your hand, but you would never notice it.

\section*{Practice Problems}
15.

What is the magnitude of the electric field from 20 cm from a point charge of \(q =33 \mathrm{nC}\) ?\\
a. \(7.4 \times 10^{3} \mathrm{~N} / \mathrm{C}\)\\
b. \(1.48 \times 10^{3} \mathrm{~N} / \mathrm{C}\)\\
c. \(7.4 \times 10^{12} \mathrm{~N} / \mathrm{C}\)\\
d. 0\\
16.\\
\(\mathrm{A}-10 \mathrm{nC}\) charge is at the origin. In which direction does the electric field from the charge point at \(x+10 \mathrm{~cm}\) ?\\
a. The electric field points away from negative charges.\\
b. The electric field points toward negative charges.\\
c. The electric field points toward positive charges.\\
d. The electric field points away from positive charges.

\section*{Check Your Understanding}
17.

When electric field lines get closer together, what does that tell you about the electric field?\\
a. The electric field is inversely proportional to the density of electric field lines.\\
b. The electric field is directly proportional to the density of electric field lines.\\
c. The electric field is not related to the density of electric field lines.\\
d. The electric field is inversely proportional to the square root of density of electric field lines.\\
18.

If five electric-field lines come out of a +5 nC charge, how many electric-field lines should come out of a +20 nC charge?\\
a. five field lines\\
b. 10 field lines\\
c. 15 field lines\\
d. 20 field lines

\subsection*{18.4 Electric Potential}
\section*{Section Learning Objectives}
By the end of this section, you will be able to do the following:

\begin{itemize}
  \item Explain the similarities and differences between electric potential energy and gravitational potential energy
  \item Calculate the electric potential difference between two point charges and in a uniform electric field
\end{itemize}

\section*{Teacher Support}
Teacher Support The learning objectives in this section will help your students master the following standards:

\begin{itemize}
  \item (5) The student knows the nature of forces in the physical world. The student is expected to:
  \item (C) describe and calculate how the magnitude of the electrical force between two objects depends on their charges and the distance between them; and
  \item (D) identify examples of electric and magnetic forces in everyday life.
\end{itemize}

\section*{Section Key Terms}
\section*{Teacher Support}
Teacher Support This section again uses the analogy between gravity and electric force to present electric potential energy and electric potential.\\[0pt]
[BL]Ask students to define how potential is used in everyday life. Draw the analogy between this and potential energy.\\[0pt]
[OL][AL]Review the concept of potential energy as being the potential an object has to do work by virtue of its position with respect to another object.\\[0pt]
[OL]Remind students that gravitational potential energy is only meaningful when we deal with differences in energy. In other words, a reference position is chosen, and all other potential energies are compared with the potential energy at the reference position.\\[0pt]
[AL]Point out that sometimes the reference position is at infinity, in which case this may not be explicitly stated.

As you learned in studying gravity, a mass in a gravitational field has potential energy, which means it has the potential to accelerate and thereby increase\\
its kinetic energy. This kinetic energy can be used to do work. For example, imagine you want to use a stone to pound a nail into a piece of wood. You first lift the stone high above the nail, which increases the potential energy of the stone-Earth system-because Earth is so large, it does not move, so we usually shorten this by saying simply that the potential energy of the stone increases. When you drop the stone, gravity converts the potential energy into kinetic energy. When the stone hits the nail, it does work by pounding the nail into the wood. The gravitational potential energy is the work that a mass can potentially do by virtue of its position in a gravitational field. Potential energy is a very useful concept, because it can be used with conservation of energy to calculate the motion of masses in a gravitational field.

Electric potential energy works much the same way, but it is based on the electric field instead of the gravitational field. By virtue of its position in an electric field, a charge has an electric potential energy. If the charge is free to move, the force due to the electric field causes it to accelerate, so its potential energy is converted to kinetic energy, just like a mass that falls in a gravitational field. This kinetic energy can be used to do work. The electric potential energy is the work that a charge can do by virtue of its position in an electric field.

The analogy between gravitational potential energy and electric potential energy is depicted in Figure 18.21. On the left, the ball-Earth system gains gravitational potential energy when the ball is higher in Earth's gravitational field. On the right, the two-charge system gains electric potential energy when the positive charge is farther from the negative charge.

\begin{figure}[h]
\begin{center}
  \includegraphics[max width=\textwidth]{b13713c9-2fd4-4196-9ad9-05e4dd292cb0-45}
\captionsetup{labelformat=empty}
\caption{Figure 18.21 On the left, the gravitational field points toward Earth. The higher the ball is in the gravitational field, the higher the potential energy is of the Earth-ball system. On the right, the electric field points toward a negative charge. The farther the positive charge is from the negative charge, the higher the potential energy is of the two-charge system.}
\end{center}
\end{figure}

Let's use the symbol \(U_{G}\) to denote gravitational potential energy. When a mass falls in a gravitational field, its gravitational potential energy decreases. Conservation of energy tells us that the work done by the gravitational field to make the mass accelerate must equal the loss of potential energy of the mass. If we use the symbol \(W_{\text {donebygravity }}\) to denote this work, then\\
\(-\Delta U_{\mathrm{G}}=W_{\text {donebygravity, }}\)\\
18.18\\
where the minus sign reflects the fact that the potential energy of the ball decreases.

The work done by gravity on the mass is\\
\(W_{\text {donebygravity }}=-F\left(y_{\mathrm{f}}-y_{\mathrm{i}}\right)\),\\
18.19\\
where \(F\) is the force due to gravity, and \(y_{\mathrm{i}}\) and \(y_{\mathrm{f}}\) are the initial and final positions of the ball, respectively. The negative sign is because gravity points down, which we consider to be the negative direction. For the constant gravitational field near Earth's surface, \(F=m g\). The change in gravitational potential energy of the mass is\\
\(-\Delta U_{\mathrm{G}}=W_{\text {donebygravity }}=-F\left(y_{\mathrm{f}}-y_{\mathrm{i}}\right)=-m g\left(y_{\mathrm{f}}-y_{\mathrm{i}}\right)\), or \(\Delta U_{\mathrm{G}}= m g\left(y_{\mathrm{f}}-y_{\mathrm{i}}\right)\).\\
18.20

Note that \(y_{\mathrm{f}}-y_{\mathrm{i}}\) is just the negative of the height \(h\) from which the mass falls, so we usually just write \(\Delta U_{\mathrm{G}}=-m g h\).

We now apply the same reasoning to a charge in an electric field to find the electric potential energy. The change \(\Delta U_{\mathrm{E}}\) in electric potential energy is the work done by the electric field to move a charge \(q\) from an initial position \(x_{\mathrm{i}}\) to a final position \(x_{\mathrm{f}}\left(-\Delta U_{\mathrm{E}}=W_{\text {donebyE-field }}\right)\). The definition of work does not change, except that now the work is done by the electric field: \(W_{\text {donebyE-field }}= F\left(x_{\mathrm{f}}-x_{\mathrm{i}}\right)\). For a charge that falls through a constant electric field \(E\), the force applied to the charge by the electric field is \(F=q E\). The change in electric potential energy of the charge is thus\\
\(-\Delta U_{\mathrm{E}}=W_{\text {donebyE-field }}=F d=q E\left(x_{\mathrm{f}}-x_{\mathrm{i}}\right)\)\\
18.21\\
or\\
\(\Delta U_{\mathrm{E}}=-q E\left(x_{\mathrm{f}}-x_{\mathrm{i}}\right)\).\\
18.22

This equation gives the change in electric potential energy of a charge \(q\) when it moves from position \(x_{\mathrm{i}}\) to position \(x_{\mathrm{f}}\) in a constant electric field \(E\).

Figure 18.22 shows how this analogy would work if we were close to Earth's surface, where gravity is constant. The top image shows a charge accelerating due to a constant electric field. Likewise, the round mass in the bottom image accelerates due to a constant gravitation field. In both cases, the potential energy of the particle decreases, and its kinetic energy increases.

\begin{figure}[h]
\begin{center}
  \includegraphics[max width=\textwidth]{b13713c9-2fd4-4196-9ad9-05e4dd292cb0-47}
\captionsetup{labelformat=empty}
\caption{Figure 18.22 In the top picture, a mass accelerates due to a constant electric field. In the bottom picture, the mass accelerates due to a constant gravitational field.}
\end{center}
\end{figure}

\section*{Teacher Support}
Teacher Support Point out the fringing fields at the ends of the plates. Ask what path the charge would take if it were released in a fringing field (note that the charge does not follow the field lines due to its inertia).

Ask students to draw the gravitational field lines in the bottom picture. Discuss why the mass does not follow the gravitational field lines.

\section*{Watch Physics}
Analogy between Gravity and Electricity This video discusses the analogy between gravitational potential energy and electric potential energy. It reviews the concepts of work and potential energy and shows the connection between a mass in a uniform gravitation field, such as on Earth's surface, and an electric charge in a uniform electric field.

Click to view content\\
If the electric field is not constant, then the equation \(\Delta U_{\mathrm{E}}=-q E\left(x_{\mathrm{f}}-x_{\mathrm{i}}\right)\) is not valid, and deriving the electric potential energy becomes more involved. For example, consider the electric potential energy of an assembly of two point\\
charges \(q_{1}\) and \(q_{2}\) of the same sign that are initially very far apart. We start by placing charge \(q_{1}\) at the origin of our coordinate system. This takes no electrical energy, because there is no electric field at the origin (because charge \(q_{2}\) is very far away). We then bring charge \(q_{2}\) in from very far away to a distance \(r\) from the center of charge \(q_{1}\). This requires some effort, because the electric field of charge \(q_{1}\) applies a repulsive force on charge \(q_{2}\). The energy it takes to assemble these two charges can be recuperated if we let them fly apart again. Thus, the charges have potential energy when they are a distance \(r\) apart. It turns out that the electric potential energy of a pair of point charges \(q_{1}\) and \(q_{2}\) a distance \(r\) apart is\\
\(U_{\mathrm{E}}=\frac{k q_{1} q_{2}}{r}\)\\
18.23

To recap, if charges \(q_{1}\) and \(q_{2}\) are free to move, they can accumulate kinetic energy by flying apart, and this kinetic energy can be used to do work. The maximum amount of work the two charges can do (if they fly infinitely far from each other) is given by the equation above.

Notice that if the two charges have opposite signs, then the potential energy is negative. This means that the charges have more potential to do work when they are far apart than when they are at a distance \(r\) apart. This makes sense: Opposite charges attract, so the charges can gain more kinetic energy if they attract each other from far away than if they start at only a short distance apart. Thus, they have more potential to do work when they are far apart. Figure 18.23 summarizes how the electric potential energy depends on charge and separation.

\begin{figure}[h]
\begin{center}
  \includegraphics[max width=\textwidth]{b13713c9-2fd4-4196-9ad9-05e4dd292cb0-48}
\captionsetup{labelformat=empty}
\caption{Figure 18.23 The potential energy depends on the sign of the charges and}
\end{center}
\end{figure}

their separation. The arrows on the charges indicate the direction in which the charges would move if released. When charges with the same sign are far apart, their potential energy is low, as shown in the top panel for two positive charges. The situation is the reverse for charges of opposite signs, as shown in the bottom panel.

\section*{Teacher Support}
Teacher Support Have students imagine that they have to push together or pull apart balls that carry these charges. Discuss which would require more work: pushing them together from far apart or from close together. Compare this with the analogous situations for the different charge combinations and for pushing together versus pulling apart.

\section*{Electric Potential}
Recall that to find the force applied by a fixed charge \(Q\) on any arbitrary test charge \(q\), it was convenient to define the electric field, which is the force per unit charge applied by \(Q\) on any test charge that we place in its electric field. The same Strategy is used here with electric potential energy: We now define the electric potential \(V\), which is the electric potential energy per unit charge.\\
\(V=\frac{U_{\mathrm{E}}}{q}\)\\
18.24

\section*{Teacher Support}
\section*{Teacher Support}
\section*{Misconception Alert}
Emphasize the difference between electric potential energy and electric potential. Although the latter seems to be shorthand for the former, the two terms have different meanings.

Normally, the electric potential is simply called the potential or voltage. The units for the potential are \(\mathrm{J} / \mathrm{C}\), which are given the name volt \((\mathrm{V})\) after the Italian physicist Alessandro Volta (1745-1827). From the equation \(U_{\mathrm{E}}=k q_{1} q_{2} / r\), the electric potential a distance \(r\) from a point charge \(q_{1}\) is\\
\(V=\frac{U_{\mathrm{E}}}{q_{2}}=\frac{k q_{1}}{r}\).\\
18.25

This equation gives the energy required per unit charge to bring a charge \(q_{2}\) from infinity to a distance \(r\) from a point charge \(q_{1}\). Mathematically, this is written as\\
\(V=\left.\frac{U_{E}}{q_{2}}\right|_{R=r}-\left.\frac{U_{E}}{q_{2}}\right|_{R=\infty}\).

Note that this equation actually represents a di erence in electric potential. However, because the second term is zero, it is normally not written, and we speak of the electric potential instead of the electric potential difference, or we just say the potential difference, or voltage). Below, when we consider the electric potential energy per unit charge between two points not infinitely far apart, we speak of electric potential di erence explicitly. Just remember that electric potential and electric potential difference are really the same thing; the former is used just when the electric potential energy is zero in either the initial or final charge configuration.

Coming back now to the electric potential a distance \(r\) from a point charge \(q_{1}\), note that \(q_{1}\) can be any arbitrary point charge, so we can drop the subscripts and simply write\\
\(V=\frac{k q}{r}\).

\subsection*{18.27}
Now consider the electric potential near a group of charges \(q_{1}, q_{2}\), and \(q_{3}\), as drawn in Figure 18.24. The electric potential is derived by considering the electric field. Electric fields follow the principle of superposition and can be simply added together, so the electric potential from different charges also add together. Thus, the electric potential of a point near a group of charges is\\
\(V=\frac{k q_{1}}{r_{1}}+\frac{k q_{2}}{r_{2}}+\frac{k q_{3}}{r_{3}}+\cdots\).\\
18.28\\
where \(r_{1}, r_{2}, r_{3}, \ldots\), are the distances from the center of charges \(q_{1}, q_{2}, q_{3}, \ldots\) to the point of interest, as shown in Figure 18.24.

\begin{figure}[h]
\begin{center}
  \includegraphics[max width=\textwidth]{b13713c9-2fd4-4196-9ad9-05e4dd292cb0-50}
\captionsetup{labelformat=empty}
\caption{Figure 18.24 The potential at the red point is simply the sum of the potentials due to each individual charge.}
\end{center}
\end{figure}

Now let's consider the electric potential in a uniform electric field. From the equation \(\Delta U_{\mathrm{E}}=-q E\left(x_{\mathrm{f}}-x_{\mathrm{i}}\right)\), we see that the potential difference in going from \(x_{\mathrm{i}}\) to \(x_{\mathrm{f}}\) in a uniform electric field \(E\) is\\
\(\Delta V=\frac{\Delta U_{E}}{q}=-E\left(x_{\mathrm{f}}-x_{\mathrm{i}}\right)\).\\
18.29

\section*{Tips For Success}
Notice from the equation \(\Delta V=-E\left(x_{\mathrm{f}}-x_{\mathrm{i}}\right)\) that the electric field can be written as\\
\(E=\frac{\Delta V}{x_{\mathrm{f}}-x_{\mathrm{i}}}\)\\
18.30\\
which means that the electric field has units of \(\mathrm{V} / \mathrm{m}\). Thus, if you know the potential difference between two points, calculating the electric field is very simple - you simply divide the potential difference by the distance!

Notice that a positive charge in a region with high potential will experience a force pushing it toward regions of lower potential. In this sense, potential is like pressure for fluids. Imagine a pipe containing fluid, with the fluid at one end of the pipe under high pressure and the fluid at the other end of the pipe under low pressure. If nothing prevents the fluid from flowing, it will flow from the high-pressure end to the low-pressure end. Likewise, a positive charge that is free to move will move from a region with high potential to a region with lower potential.

\section*{Watch Physics}
Voltage This video starts from electric potential energy and explains how this is related to electric potential (or voltage). The lecturer calculates the electric potential created by a uniform electric field.

Click to view content\\
Watch Physics: Voltage. This video explains the difference between electrical potential, or voltage, and electrical potential energy.

Click to view content\\
A point charge moves from \(\mathrm{x}=5.0 \mathrm{~m}\) to \(\mathrm{x}=11 \mathrm{~m}\) in a \(2.0 \mathrm{~N} / \mathrm{C}\) electric field aligned with the x -direction. What change in voltage does the charge experience?\\
a. 0.33 V\\
b. 6 V\\
c. 12 V\\
d. 32 V

\section*{Links To Physics}
Electric Animals Many animals generate and/or detect electric fields. This is useful for activities such as hunting, defense, navigation, communication, and mating. Because salt water is a relatively good conductor, electric fish have evolved in all the world's oceans. These fish have intrigued humans since the earliest times. In the nineteenth century, parties were even organized where the main attraction was getting a jolt from an electric fish! Scientists also studied electric fish to learn about electricity. Alessandro Volta based his research that led to batteries in 1799 on electric fish. He even referred to batteries as artificial electric organs, because he saw them as imitations of the electric organs of electric fish.

Animals that generate electricity are called electrogenic and those that detect electric fields are called electroreceptive. Most fish that are electrogenic are also electroreceptive. One of the most well-known electric fish is the electric eel (see Figure 18.25), which is both electrogenic and electroreceptive. These fish have three pairs of organs that produce the electric charge: the main organ, Hunter's organ, and Sach's organ. Together, these organs account for more than 80percent of the fish's body.

Electric eels can produce electric discharges of much greater voltage than what you would get from a standard wall socket. These discharges can stun or even kill their prey. They also use low-intensity discharges to navigate. The electric fields they generate reflect off nearby obstacles or animals and are then detected by electroreceptors in the eel's skin. The three organs that produce electricity contain electrolytes, which are substances that ionize when dissolved in water (or other liquids). An ionized atom or molecule is one that has lost or gained at least one electron, so it carries a net charge. Thus, a liquid solution containing an electrolyte conducts electricity, because the ions in the solution can move if an electric field is applied.

To produce large discharges, the main organ is used. It contains approximately 6,000 rows of electroplaques connected in a long chain. Connected this way, the voltage between electroplaques adds up, creating a large final voltage. Each electroplaque consists of a column of cells controlled by an excitor nerve. When triggered by the excitor nerve, the electroplaques allow ionized sodium to flow through them, creating a potential difference between electroplaques. These potentials add up, and a large current can flow through the electrolyte.

This geometry is reflected in batteries, which also use stacks of plates to produce larger potential differences.

\begin{figure}[h]
\begin{center}
  \includegraphics[max width=\textwidth]{b13713c9-2fd4-4196-9ad9-05e4dd292cb0-53}
\captionsetup{labelformat=empty}
\caption{Figure 18.25 An electric eel in its natural environment. (credit: Steven G. Johnson)}
\end{center}
\end{figure}

\section*{Grasp Check}
If an electric eel produces \(1,000 \mathrm{~V}\), which voltage is produced by each electroplaque in the main organ?\\
a. 0.17 mV\\
b. 1.7 mV\\
c. 17 mV\\
d. 170 mV

\section*{Worked Example}
X-ray Tube Dentists use X-rays to image their patients' teeth and bones. The X-ray tubes that generate X-rays contain an electron source separated by about 10 cm from a metallic target. The electrons are accelerated from the source to the target by a uniform electric field with a magnitude of about 100 \(\mathrm{kN} / \mathrm{C}\), as drawn in Figure 18.26. When the electrons hit the target, X-rays are produced. (a) What is the potential difference between the electron source and the metallic target? (b) What is the kinetic energy of the electrons when they reach the target, assuming that the electrons start at rest?

\begin{figure}[h]
\begin{center}
  \includegraphics[max width=\textwidth]{b13713c9-2fd4-4196-9ad9-05e4dd292cb0-54}
\captionsetup{labelformat=empty}
\caption{Figure 18.26 In an X-ray tube, a large current flows through the electron source, causing electrons to be ejected from the electron source. The ejected electrons are accelerated toward the target by the electric field. When they strike the target, X-rays are produced.}
\end{center}
\end{figure}

\section*{Strategy FOR (A)}
Use the equation \(\Delta V=-E\left(x_{\mathrm{f}}-x_{\mathrm{i}}\right)\) to find the potential difference given a constant electric field. Define the source position as \(x_{\mathrm{i}}=0\) and the target position as \(x_{\mathrm{f}}=10 \mathrm{~cm}\). To accelerate the electrons in the positive \(x\) direction, the electric field must point in the negative \(x\) direction. This way, the force \(F=q E\) on the electrons will point in the positive \(x\) direction, because both \(q\) and \(E\) are negative. Thus, \(E=-100 \times 10^{3} \mathrm{~N} / \mathrm{C}\).

Solution for (a)\\
Using \(x_{\mathrm{i}}=0\) and \(x_{\mathrm{f}}=10 \mathrm{~cm}=0.10 \mathrm{~m}\), the equation \(\Delta V=-E\left(x_{\mathrm{f}}-x_{\mathrm{i}}\right)\) tells us that the potential difference between the electron source and the target is\\
\(\Delta V=-E\left(x_{\mathrm{f}}-x_{\mathrm{i}}\right)=-\left(-100 \times 10^{3} \mathrm{~N} / \mathrm{C}\right)(0.10 \mathrm{~m}-0)=+10 \mathrm{kV}\).\\
18.31

Discussion for (a)\\
The potential difference is positive, so the energy per unit positive charge is higher at the target than at the source. This means that free positive charges would fall from the target to the source. However, electrons are negative charges,\\
so they accelerate from the source toward the target, gaining kinetic energy as they go.

\section*{Strategy FOR (B)}
Apply conservation of energy to find the final kinetic energy of the electrons. In going from the source to the target, the change in electric potential energy plus the change in kinetic energy of the electrons must be zero, so \(\Delta U_{E}+\Delta K=0\). The change in electric potential energy for moving through a constant electric field is given by the equation\\
\(\Delta U_{\mathrm{E}}=-q E\left(x_{\mathrm{f}}-x_{\mathrm{i}}\right)\),\\
where the electric field is \(E=-100 \times 10^{3} \mathrm{~N} / \mathrm{C}\). Because the electrons start at rest, their initial kinetic energy is zero. Thus, the change in kinetic energy is simply their final kinetic energy, so \(\Delta K=K_{\mathrm{f}}\).

Solution for (b)\\
Again \(x_{\mathrm{i}}=0\) and \(x_{\mathrm{f}}=10 \mathrm{~cm}=0.10 \mathrm{~m}\). The charge of an electron is \(q= -1.602 \times 10^{-19} \mathrm{C}\). Conservation of energy gives

\[
\begin{aligned}
\Delta U_{E}+\Delta K & =0 \\
-q E\left(x_{\mathrm{f}}-x_{\mathrm{i}}\right)+K_{\mathrm{f}} & =0 \\
K_{\mathrm{f}} & =q E\left(x_{\mathrm{f}}-x_{\mathrm{i} .}\right) .
\end{aligned}
\]

18.32

Inserting the known values into the right-hand side of this equation gives

\[
\begin{aligned}
K_{\mathrm{f}} & =\left(-1.60 \times 10^{-19} \mathrm{C}\right)\left(-100 \times 10^{3} \mathrm{~N} / \mathrm{C}\right)(0.10 \mathrm{~m}-0) \\
& =1.6 \times 10^{-15} \mathrm{~J}
\end{aligned}
\]

18.33

Discussion for (b)\\
This is a very small energy. However, electrons are very small, so they are easy to accelerate, and this energy is enough to make an electron go extremely fast. You can find their speed by using the definition of kinetic energy, \(K=\frac{1}{2} m v^{2}\). The result is that the electrons are moving at more than 100 million miles per hour!

\section*{Worked Example}
Electric Potential Energy of Doorknob and Dust Speck Consider again the doorknob from the example in the previous section. The doorknob is treated as a spherical conductor with a uniform static charge \(q_{1}=-1.5 \mathrm{nC}\) on its surface.

What is the electric potential energy between the doorknob and a speck of dust carrying a charge \(q_{2}=0.20 \mathrm{nC}\) at 1.0 cm from the front surface of the doorknob? The diameter of the doorknob is 5.0 cm .

\section*{Strategy}
As we did in the previous section, we treat the charge as if it were concentrated at the center of the doorknob. Again, as you will be able to validate in later physics classes, we can make this simplification, because the charge is uniformly distributed over the surface of the spherical object. Make a sketch of the situation and define a coordinate system, as shown in the image below. We use \(+x\) to indicate the outward direction perpendicular to the door, with \(x=0\) at the center of the doorknob. If the diameter of the doorknob is 5.0 cm , its radius is 2.5 cm . Thus, the speck of dust 1.0 cm from the surface of the doorknob is a distance \(r=2.5 \mathrm{~cm}+1.0 \mathrm{~cm}=3.5 \mathrm{~cm}\) from the center of the doorknob. To solve this problem, use the equation \(U_{\mathrm{E}}=k q_{1} q_{2} / r\).\\
\includegraphics[max width=\textwidth, center]{b13713c9-2fd4-4196-9ad9-05e4dd292cb0-56}

\section*{Solution}
The charge on the doorknob is \(q_{1}=-1.5 \mathrm{nC}=-1.5 \times 10^{-9} \mathrm{C}\), and the charge on the speck of dust is \(q_{2}=0.20 \mathrm{nC}=2.0 \times 10^{-10} \mathrm{C}\). The distance \(r=3.5 \mathrm{~cm}=0.035 \mathrm{~m}\). Inserting these values into the equation \(U_{\mathrm{E}}=k q_{1} q_{2} / r\) gives

\[
\begin{aligned}
U_{E} & =\frac{k q_{1} q_{2}}{r} \\
& =\frac{\left(8.99 \times 10^{9} \mathrm{~N} \cdot \mathrm{~m}^{2} / \mathrm{C}^{2}\right)\left(-1.5 \times 10^{-9} \mathrm{C}\right)\left(2.0 \times 10^{-10} \mathrm{C}\right)}{(0.035 \mathrm{~m})} \\
& =-7.7 \times 10^{-8} \mathrm{~J}
\end{aligned}
\]

18.34

Discussion\\
The energy is negative, which means that the energy will decrease that is, get even more negative as the speck of dust approaches the doorknob. This helps explain why dust accumulates on objects that carry a static charge. However, note\\
that insulators normally collect more static charge than conductors, because any charge that accumulates on insulators cannot move about on the insulator to find a way to escape. They must simply wait to be removed by some passing moist speck of dust or other host.

\section*{Practice Problems}
19.

What is the electric potential 10 cm from a -10 nC charge?\\
a. \(9.0 \times 10^{2} \mathrm{~V}\)\\
b. \(9.0 \times 10^{3} \mathrm{~V}\)\\
c. \(9.0 \times 10^{4} \mathrm{~V}\)\\
d. \(9.0 \times 10^{5} \mathrm{~V}\)\\
20.

An electron accelerates from 0 to \(10 \times 10^{4} \mathrm{~m} / \mathrm{s}\) in an electric field. Through what potential difference did the electron travel? The mass of an electron is \(9.11 \times 10^{-31} \mathrm{~kg}\), and its charge is \(-1.60 \times 10^{-19} \mathrm{C}\).\\
a. 29 mV\\
b. 290 mV\\
c. \(2,900 \mathrm{mV}\)\\
d. 29 V

\section*{Check Your Understanding}
21.

Gravitational potential energy is the potential for two masses to do work by virtue of their positions with respect to each other. What is the analogous definition of electric potential energy?\\
a. Electric potential energy is the potential for two charges to do work by virtue of their positions with respect to the origin point.\\
b. Electric potential energy is the potential for two charges to do work by virtue of their positions with respect to infinity.\\
c. Electric potential energy is the potential for two charges to do work by virtue of their positions with respect to each other.\\
d. Electric potential energy is the potential for single charges to do work by virtue of their positions with respect to their final positions.\\
22.

A negative charge is 10 m from a positive charge. Where would you have to move the negative charge to increase the potential energy of the system?\\
a. The negative charge should be moved closer to the positive charge.\\
b. The negative charge should be moved farther away from the positive charge.\\
c. The negative charge should be moved to infinity.\\
d. The negative charge should be placed just next to the positive charge.

\subsection*{18.5 Capacitors and Dielectrics}
\section*{Section Learning Objectives}
By the end of this section, you will be able to do the following:

\begin{itemize}
  \item Calculate the energy stored in a charged capacitor and the capacitance of a capacitor
  \item Explain the properties of capacitors and dielectrics
\end{itemize}

\section*{Teacher Support}
Teacher Support The learning objectives in this section will help your students master the following standards:

\begin{itemize}
  \item (5) The student knows the nature of forces in the physical world. The student is expected to:
  \item (F) design construct, and calculate in terms of current through, potential difference across, resistance of, and power used by electric circuit elements connected in both series and parallel combinations.
\end{itemize}

In addition, the High School Physics Laboratory Manual addresses content in this section in the lab titled: Electric Charge as well as the following standards:

\begin{itemize}
  \item (5) The student knows the nature of forces in the physical world. The student is expected to:
  \item (F) design construct, and calculate in terms of current through, potential difference across, resistance of, and power used by electric circuit elements connected in both series and parallel combinations.
\end{itemize}

\section*{Section Key Terms}
\section*{Teacher Support}
Teacher Support To present capacitors, this section emphasizes their capacity to store energy. Dielectrics are introduced as a way to increase the amount of energy that can be stored in a capacitor. To introduce the idea of energy storage, discuss with students other mechanisms of storing energy, such as dams or batteries. Ask which have greater capacity.

\section*{Capacitors}
\section*{Teacher Support}
Teacher Support Explain that electrical capacitors are vital parts of all electrical circuits. In fact, all electrical devices have a capacitance even if a capacitor\\
is not explicitly put into the device.\\[0pt]
[BL]Have students define how the word capacity is used in everyday life. Have them look up the definition in the dictionary. Compare and contrast the everyday meaning with the meaning of the term in physics.\\[0pt]
[OL]Ask students whether they have heard the word capacitor used in conjunction with electricity, such as in power stations or electric circuits. Have them describe how the word is used.\\[0pt]
[AL]Discuss how a spring has a capacity to store mechanical energy. Discuss which properties of the spring would increase its capacity to store energy. Point out that these properties are intrinsic to the spring.

Consider again the X-ray tube discussed in the previous sample problem. How can a uniform electric field be produced? A single positive charge produces an electric field that points away from it, as in Figure 18.17. This field is not uniform, because the space between the lines increases as you move away from the charge. However, if we combine a positive and a negative charge, we obtain the electric field shown in Figure 18.19(a). Notice that, between the charges, the electric field lines are more equally spaced.

What happens if we place, say, five positive charges in a line across from five negative charges, as in Figure 18.27? Now the region between the lines of charge contains a fairly uniform electric field.

\begin{figure}[h]
\begin{center}
  \includegraphics[max width=\textwidth]{b13713c9-2fd4-4196-9ad9-05e4dd292cb0-60}
\captionsetup{labelformat=empty}
\caption{Figure 18.27 The red dots are positive charges, and the blue dots are negative charges. The electric-field direction is shown by the red arrows. Notice that the electric field between the positive and negative dots is fairly uniform.}
\end{center}
\end{figure}

We can extend this idea even further and into two dimensions by placing two metallic plates face to face and charging one with positive charge and the other\\
with an equal magnitude of negative charge. This can be done by connecting one plate to the positive terminal of a battery and the other plate to the negative terminal, as shown in Figure 18.28. The electric field between these charged plates will be extremely uniform.

\begin{figure}[h]
\begin{center}
  \includegraphics[max width=\textwidth]{b13713c9-2fd4-4196-9ad9-05e4dd292cb0-61}
\captionsetup{labelformat=empty}
\caption{Figure 18.28 Two parallel metal plates are charged with opposite charge, by connecting the plates to the opposite terminals of a battery. The magnitude of the charge on each plate is the same.}
\end{center}
\end{figure}

Let's think about the work required to charge these plates. Before the plates are connected to the battery, they are neutral - that is, they have zero net charge. Placing the first positive charge on the left plate and the first negative charge on the right plate requires very little work, because the plates are neutral, so no opposing charges are present. Now consider placing a second positive charge on the left plate and a second negative charge on the right plate. Because the\\
first two charges repel the new arrivals, a force must be applied to the two new charges over a distance to put them on the plates. This is the definition of work, which means that, compared with the first pair, more work is required to put the second pair of charges on the plates. To place the third positive and negative charges on the plates requires yet more work, and so on. Where does this work come from? The battery! Its chemical potential energy is converted into the work required to separate the positive and negative charges.

Although the battery does work, this work remains within the battery-plate system. Therefore, conservation of energy tells us that, if the potential energy of the battery decreases to separate charges, the energy of another part of the system must increase by the same amount. In fact, the energy from the battery is stored in the electric field between the plates. This idea is analogous to considering that the potential energy of a raised hammer is stored in Earth's gravitational field. If the gravitational field were to disappear, the hammer would have no potential energy. Likewise, if no electric field existed between the plates, no energy would be stored between them.

If we now disconnect the plates from the battery, they will hold the energy. We could connect the plates to a lightbulb, for example, and the lightbulb would light up until this energy was used up. These plates thus have the capacity to store energy. For this reason, an arrangement such as this is called a capacitor. A capacitor is an arrangement of objects that, by virtue of their geometry, can store energy an electric field.

Various real capacitors are shown in Figure 18.29. They are usually made from conducting plates or sheets that are separated by an insulating material. They can be flat or rolled up or have other geometries.

\begin{figure}[h]
\begin{center}
  \includegraphics[max width=\textwidth]{b13713c9-2fd4-4196-9ad9-05e4dd292cb0-63}
\captionsetup{labelformat=empty}
\caption{Figure 18.29 Some typical capacitors. (credit: Windell Oskay)}
\end{center}
\end{figure}

The capacity of a capacitor is defined by its capacitance \(C\), which is given by \(C=\frac{Q}{V}\),

\subsection*{18.35}
where \(Q\) is the magnitude of the charge on each capacitor plate, and \(V\) is the potential difference in going from the negative plate to the positive plate. This means that both \(Q\) and \(V\) are always positive, so the capacitance is always positive. We can see from the equation for capacitance that the units of capacitance are C/V, which are called farads (F) after the nineteenth-century English physicist Michael Faraday.

The equation \(C=Q / V\) makes sense: A parallel-plate capacitor (like the one shown in Figure 18.28) the size of a football field could hold a lot of charge without requiring too much work per unit charge to push the charge into the\\
capacitor. Thus, \(Q\) would be large, and \(V\) would be small, so the capacitance \(C\) would be very large. Squeezing the same charge into a capacitor the size of a fingernail would require much more work, so \(V\) would be very large, and the capacitance would be much smaller.

Although the equation \(C=Q / V\) makes it seem that capacitance depends on voltage, in fact it does not. For a given capacitor, the ratio of the charge stored in the capacitor to the voltage difference between the plates of the capacitor always remains the same. Capacitance is determined by the geometry of the capacitor and the materials that it is made from. For a parallel-plate capacitor with nothing between its plates, the capacitance is given by\\
\(C_{0}=\varepsilon_{0} \frac{A}{d}\),\\
18.36\\
where \(A\) is the area of the plates of the capacitor and \(d\) is their separation. We use \(C_{0}\) instead of \(C\), because the capacitor has nothing between its plates (in the next section, we'll see what happens when this is not the case). The constant \(\varepsilon_{0}\), read epsilon zero is called the permittivity of free space, and its value is\\
\(\varepsilon_{0}=8.85 \times 10^{-12} \mathrm{~F} / \mathrm{m}\)\\
18.37

Coming back to the energy stored in a capacitor, we can ask exactly how much energy a capacitor stores. If a capacitor is charged by putting a voltage \(V\) across it for example, by connecting it to a battery with voltage \(V\)-the electrical potential energy stored in the capacitor is\\
\(U_{E}=\frac{1}{2} C V^{2}\).\\
18.38

Notice that the form of this equation is similar to that for kinetic energy, \(K= \frac{1}{2} m v^{2}\).

\section*{Watch Physics}
Where does Capacitance Come From? This video shows how capacitance is defined and why it depends only on the geometric properties of the capacitor, not on voltage or charge stored. In so doing, it provides a good review of the concepts of work and electric potential.

Click to view content

\section*{Grasp Check}
If you increase the distance between the plates of a capacitor, how does the capacitance change?\\
a. Doubling the distance between capacitor plates will reduce the capacitance four fold.\\
b. Doubling the distance between capacitor plates will reduce the capacitance two fold.\\
c. Doubling the distance between capacitor plates will increase the capacitance two times.\\
d. Doubling the distance between capacitor plates will increase the capacitance four times.

\section*{Virtual Physics}
Charge your Capacitor Click to view content\\
In this simulation, you are presented with a parallel-plate capacitor connected to a variable-voltage battery. The battery is initially at zero volts, so no charge is on the capacitor. Slide the battery slider up and down to change the battery voltage, and observe the charges that accumulate on the plates. Display the capacitance, top-plate charge, and stored energy as you vary the battery voltage. You can also display the electric-field lines in the capacitor. Finally, probe the voltage between different points in this circuit with the help of the voltmeter.

\section*{Grasp Check}
True or false - In a capacitor, the stored energy is always positive, regardless of whether the top plate is charged with negative or positive charge.\\
a. false\\
b. true

\section*{Worked Example}
Capacitance and Charge Stored in a Parallel Plate Capacitor (a) What is the capacitance of a parallel-plate capacitor with metal plates, each of area \(1.00 \mathrm{~m}^{2}\), separated by 0.0010 m ? (b) What charge is stored in this capacitor if a voltage of \(3.00 \times 10^{3} \mathrm{~V}\) is applied to it?

\section*{Strategy FOR (A)}
Use the equation \(C_{0}=\varepsilon_{0} \frac{A}{d}\).\\
Solution for (a)\\
Entering the given values into this equation for the capacitance of a parallelplate capacitor yields

\[
\begin{aligned}
C & =\varepsilon_{0} \frac{A}{d} \\
& =\left(8.85 \times 10^{-12} \mathrm{~F} / \mathrm{m}\right) \frac{1.00 \mathrm{~m}^{2}}{0.0010 \mathrm{~m}} \\
& =8.9 \times 10^{-9} \mathrm{~F} \\
& =8.9 \mathrm{nF} .
\end{aligned}
\]

18.39

Discussion for (a)\\
This small value for the capacitance indicates how difficult it is to make a device with a large capacitance. Special techniques help, such as using very-large-area thin foils placed close together or using a dielectric (to be discussed below).

\section*{Strategy FOR (B)}
Knowing \(C\), find the charge stored by solving the equation \(C=Q / V\), for the charge \(Q\).

Solution for (b)\\
The charge \(Q\) on the capacitor is

\[
\begin{aligned}
Q & =C V \\
& =\left(8.9 \times 10^{-9} \mathrm{~F}\right)\left(3.00 \times 10^{3} \mathrm{~V}\right) \\
& =2.7 \times 10^{-5} \mathrm{C}
\end{aligned}
\]

18.40

Discussion for (b)\\
This charge is only slightly greater than typical static electricity charges. More charge could be stored by using a dielectric between the capacitor plates.

\section*{Worked Example}
What battery is needed to charge a capacitor? Your friend provides you with a \(10 \mu \mathrm{~F}\) capacitor. To store \(120 \mu \mathrm{C}\) on this capacitor, what voltage battery should you buy?

\section*{Strategy}
Use the equation \(C=Q / V\) to find the voltage needed to charge the capacitor.\\
Solution\\
Solving \(C=Q / V\) for the voltage gives \(V=Q / C\). Inserting \(C=10 \mu \mathrm{~F}= 10 \times 10^{-6} \mathrm{~F}\) and \(Q=120 \mu \mathrm{C}=120 \times 10^{-6} \mathrm{C}\) gives\\
\(V=\frac{Q}{C}=\frac{120 \times 10^{-6} \mathrm{C}}{10 \times 10^{-6} \mathrm{~F}}=12 \mathrm{~V}\)\\
18.41

Discussion\\
Such a battery should be easy to procure. There is still a question of whether the battery contains enough energy to provide the desired charge. The equation \(U_{E}=\frac{1}{2} C V^{2}\) allows us to calculate the required energy.\\
\(U_{E}=\frac{1}{2} C V^{2}=\frac{1}{2}\left(10 \times 10^{-6} \mathrm{~F}\right)(12 \mathrm{~V})^{2}=72 \mathrm{~mJ}\)\\
18.42

A typical commercial battery can easily provide this much energy.

\section*{Practice Problems}
23.

What is the voltage on a 35 F with 25 nC of charge?\\
a. \(8.75 \times 10^{-13} \mathrm{~V}\)\\
b. \(0.71 \times 10^{-3} \mathrm{~V}\)\\
c. \(1.4 \times 10^{-3} \mathrm{~V}\)\\
d. \(1.4 \times 10^{3} \mathrm{~V}\)\\
24.

Which voltage is across a 100 F capacitor that stores 10 J of energy?\\
a. \(-4.5 \times 10^{2} \mathrm{~V}\)\\
b. \(4.5 \times 10^{2} \mathrm{~V}\)\\
c. \(\pm 4.5 \times 10^{2} \mathrm{~V}\)\\
d. \(\pm 9 \times 10^{2} \mathrm{~V}\)

\section*{Dielectrics}
\section*{Teacher Support}
Teacher Support Explain that dielectric is short for dielectric material, which has specific electrical properties to be discussed in this section. The word dielectric is used to indicate the energy-storage capacity of a material. Remind students that insulator is used to indicate the ability of a material to prevent the passage of electric charge.\\[0pt]
[BL][OL]Point out that the prefix \(d i\) means two or double. Combined with the word electric, this implies that a dielectric can have two electric charges.\\[0pt]
[AL]Ask students whether they know of other words that use the prefix \(d i\) in science (diatomic, carbon dioxide, dipole, ...).

Before working through some sample problems, let's look at what happens if we put an insulating material between the plates of a capacitor that has been charged and then disconnected from the charging battery, as illustrated in Figure 18.30. Because the material is insulating, the charge cannot move through it from one plate to the other, so the charge \(Q\) on the capacitor does not change. An electric field exists between the plates of a charged capacitor, so the insulating material becomes polarized, as shown in the lower part of the figure. An electrically insulating material that becomes polarized in an electric field is called a dielectric.

Figure 18.30 shows that the negative charge in the molecules in the material shifts to the left, toward the positive charge of the capacitor. This shift is due to the electric field, which applies a force to the left on the electrons in the molecules of the dielectric. The right sides of the molecules are now missing a bit of negative charge, so their net charge is positive.

\begin{figure}[h]
\begin{center}
  \includegraphics[max width=\textwidth]{b13713c9-2fd4-4196-9ad9-05e4dd292cb0-69}
\captionsetup{labelformat=empty}
\caption{Figure 18.30 The top and bottom capacitors carry the same charge \(Q\). The top capacitor has no dielectric between its plates. The bottom capacitor has a dielectric between its plates. The molecules in the dielectric are polarized by the electric field of the capacitor.}
\end{center}
\end{figure}

\section*{Teacher Support}
Teacher Support Point out the positive and negative surface charge on each side of the dielectric. Discuss with students that the electric-field lines are drawn so that they touch the surface charges, because electric-field lines always start or terminate on a charge. Thus, fewer electric-field lines will traverse the dielectric, meaning the electric field is weaker inside the dielectric.

All electrically insulating materials are dielectrics, but some are better dielectrics than others. A good dielectric is one whose molecules allow their electrons to shift strongly in an electric field. In other words, an electric field pulls their electrons a fair bit away from their atom, but they do not escape completely from their atom (which is why they are insulators).

Figure 18.31 shows a macroscopic view of a dielectric in a charged capacitor. Notice that the electric-field lines in the capacitor with the dielectric are spaced farther apart than the electric-field lines in the capacitor with no dielectric. This means that the electric field in the dielectric is weaker, so it stores less electrical potential energy than the electric field in the capacitor with no dielectric.

Where has this energy gone? In fact, the molecules in the dielectric act like tiny springs, and the energy in the electric field goes into stretching these springs. With the electric field thus weakened, the voltage difference between the two sides of the capacitor is smaller, so it becomes easier to put more charge on the capacitor. Placing a dielectric in a capacitor before charging it therefore allows more charge and potential energy to be stored in the capacitor. A parallel plate with a dielectric has a capacitance of\\
\(C=\kappa \varepsilon_{0} \frac{A}{d}=\kappa C_{0}\),\\
18.43\\
where \(\kappa\) (kappa) is a dimensionless constant called the dielectric constant. Because \(\kappa\) is greater than 1 for dielectrics, the capacitance increases when a dielectric is placed between the capacitor plates. The dielectric constant of several materials is shown in Table 18.1.

\begin{figure}[h]
\begin{center}
\captionsetup{labelformat=empty}
\caption{Table 18.1 Dielectric Constants for Various Materials at \(20{ }^{\circ} \mathrm{C}\)}
  \includegraphics[max width=\textwidth]{b13713c9-2fd4-4196-9ad9-05e4dd292cb0-71}
\end{center}
\end{figure}

Figure 18.31 The top and bottom capacitors carry the same charge Q. The top capacitor has no dielectric between its plates. The bottom capacitor has a dielectric between its plates. Because some electric-field lines terminate and\\
start on polarization charges in the dielectric, the electric field is less strong in the capacitor. Thus, for the same charge, a capacitor stores less energy when it contains a dielectric.

\section*{Teacher Support}
Teacher Support Emphasize that the electric-field lines in the dielectric are less dense than in the capacitor with no dielectric, which shows that the electric field is weaker in the dielectric.

\section*{Worked Example}
Capacitor for Camera Flash A typical flash for a point-and-shoot camera uses a capacitor of about 200 F . (a) If the potential difference between the capacitor plates is 100 V -that is, 100 V is placed "across the capacitor," how much energy is stored in the capacitor? (b) If the dielectric used in the capacitor were a \(0.010-\mathrm{mm}\)-thick sheet of nylon, what would be the surface area of the capacitor plates?

\section*{Strategy FOR (A)}
Given that \(V=100 \mathrm{~V}\) and \(C=200 \times 10^{-6} \mathrm{~F}\), we can use the equation \(U_{E}=\frac{1}{2} C V^{2}\), to find the electric potential energy stored in the capacitor.\\
Solution for (a)\\
Inserting the given quantities into \(U_{E}=\frac{1}{2} C V^{2}\) gives

\[
\begin{aligned}
U_{E} & =\frac{1}{2} C V^{2} \\
& =\frac{1}{2}\left(200 \times 10^{-6} \mathrm{~F}\right)(100 \mathrm{~V})^{2} \\
& =1.0 \mathrm{~J}
\end{aligned}
\]

18.44

Discussion for (a)\\
This is enough energy to lift a \(1-\mathrm{kg}\) ball about 1 m up from the ground. The flash lasts for about 0.001 s , so the power delivered by the capacitor during this brief time is \(P=\frac{U_{E}}{t}=\frac{1.0 \mathrm{~J}}{0.001 \mathrm{~s}}=1 \mathrm{~kW}\). Considering that a car engine delivers about 100 kW of power, this is not bad for a little capacitor!

\section*{Strategy FOR (B)}
Because the capacitor plates are in contact with the dielectric, we know that the spacing between the capacitor plates is \(d=0.010 \mathrm{~mm}=1.0 \times 10^{-5} \mathrm{~m}\). From the previous table, the dielectric constant of nylon is \(\kappa=3.4\). We can now use the equation \(C=\kappa \varepsilon_{0} \frac{A}{d}\) to find the area \(A\) of the capacitor.

Solution (b)\\
Solving the equation for the area \(A\) and inserting the known quantities gives

\[
\begin{aligned}
C & =\kappa \varepsilon_{0} \frac{A}{d} \\
A & =\frac{C d}{\kappa \varepsilon_{0}} \\
& =\frac{\left(200 \times 10^{-6} \mathrm{~F}\right)\left(1.0 \times 10^{-5} \mathrm{~m}\right)}{(3.4)\left(8.85 \times 10^{-12} \mathrm{~F} / \mathrm{m}\right)} \\
& =66 \mathrm{~m}^{2} .
\end{aligned}
\]

18.45

Discussion for (b)\\
This is much too large an area to roll into a capacitor small enough to fit in a handheld camera. This is why these capacitors don't use simple dielectrics but a more advanced technology to obtain a high capacitance.

\section*{Practice Problems}
25.

With 12 V across a capacitor, it accepts 10 mC of charge. What is its capacitance?\\
a. 0.83 F\\
b. 83 F\\
c. 120 F\\
d. 830 F\\
26.

A parallel-plate capacitor has an area of \(10 \mathrm{~cm}^{2}\) and the plates are separated by 100 m . If the capacitor contains paper between the plates, what is its capacitance?\\
a. \(3.3 \times 10^{-10} \mathrm{~F}\)\\
b. \(3.3 \times 10^{-8} \mathrm{~F}\)\\
c. \(3.3 \times 10^{-6} \mathrm{~F}\)\\
d. \(3.3 \times 10^{-4} \mathrm{~F}\)

\section*{Check Your Understanding}
27.

If the area of a parallel-plate capacitor doubles, how is the capacitance affected?\\
a. The capacitance will remain the same.\\
b. The capacitance will double.\\
c. The capacitance will increase four times.\\
d. The capacitance will increase eight times.\\
28.

If you double the area of a parallel-plate capacitor and reduce the distance between the plates by a factor of four, how is the capacitance affected?\\
a. It will increase by a factor of two.\\
b. It will increase by a factor of four.\\
c. It will increase by a factor of six.\\
d. It will increase by a factor of eight.

\section*{Ke Terms}
capacitor arrangement of objects that can store electrical energy by virtue of their geometry\\
conductor material through which electric charge can easily move, such as metals

Coulomb s law describes the electrostatic force between charged objects, which is proportional to the charge on each object and inversely proportional to the square of the distance between the objects\\
dielectric electrically insulating material that becomes polarized in an electric field\\
electric field defines the force per unit charge at all locations in space around a charge distribution\\
electric potential the electric potential energy per unit charge\\
electric potential energy the work that a charge can do by virtue of its position in an electric field\\
electron subatomic particle that carries one indivisible unit of negative electric charge\\
induction creating an unbalanced charge distribution in an object by moving a charged object toward it (but without touching)\\
insulator material through which a charge does not move, such as rubber\\
inverse-square law law that has the form of a ratio, with the denominator being the distance squared\\
law of conservation of charge states that total charge is constant in any process\\
polarization separation of charge induced by nearby excess charge\\
proton subatomic particle that carries the same magnitude charge as the electron, but its charge is positive\\
test charge positive electric charge whose with a charge magnitude so small that it does not significantly perturb any nearby charge distribution

Ke Equations\\
18.2 Coulomb's law\\
18.3 Electric Field\\
18.4 Electric Potential\\
18.5 Capacitors and Dielectrics

\section*{Section Summar}
\subsection*{18.1 Electrical Charges, Conservation of Charge, and Transfer of Charge}
\begin{itemize}
  \item Electric charge is a conserved quantity, which means it can be neither created nor destroyed.
  \item Electric charge comes in two varieties, which are called positive and negative.
  \item Charges with the same sign repel each other. Charges with opposite signs attract each other.
  \item Charges can move easily in conducting material. Charges cannot move easily in an insulating material.
  \item Objects can be charged in three ways: by contact, by conduction, and by induction.
  \item Although a polarized object may be neutral, its electrical charge is unbalanced, so one side of the object has excess negative charge and the other side has an equal magnitude of excess positive charge.
\end{itemize}

\subsection*{18.2 Coulomb's law}
\begin{itemize}
  \item Coulomb's law is an inverse square law and describes the electrostatic force between particles.
  \item The electrostatic force between charged objects is proportional to the charge on each object and inversely proportional to the distance squared between the objects.
  \item If Coulomb's law gives a negative result, the force is attractive; if the result is positive, the force is repulsive.
\end{itemize}

\subsection*{18.3 Electric Field}
\begin{itemize}
  \item The electric field defines the force per unit charge in the space around a charge distribution.
  \item For a point charge or a sphere of uniform charge, the electric field is inversely proportional to the distance from the point charge or from the center of the sphere.
  \item Electric-field lines never cross each other.
  \item More force is applied to a charge in a region with many electric field lines than in a region with few electric field lines.
  \item Electric field lines start at positive charges and point away from positive charges. They end at negative charges and point toward negative charges.
\end{itemize}

\subsection*{18.4 Electric Potential}
\begin{itemize}
  \item Electric potential energy is a concept similar to gravitational potential energy: It is the potential that charges have to do work by virtue of their positions relative to each other.
  \item Electric potential is the electric potential energy per unit charge.
  \item The potential is always measured between two points, where one point may be at infinity.
  \item Positive charges move from regions of high potential to regions of low potential.
  \item Negative charges move from regions of low potential to regions of high potential.
\end{itemize}

\subsection*{18.5 Capacitors and Dielectrics}
\begin{itemize}
  \item The capacitance of a capacitor depends only on the geometry of the capacitor and the materials from which it is made. It does not depend on the voltage across the capacitor.
  \item Capacitors store electrical energy in the electric field between their plates.
  \item A dielectric material is an insulator that is polarized in an electric field.
  \item Putting a dielectric between the plates of a capacitor increases the capacitance of the capacitor.
\end{itemize}

\end{document}