\documentclass[10pt]{article}
\usepackage[utf8]{inputenc}
\usepackage[T1]{fontenc}
\usepackage{graphicx}
\usepackage[export]{adjustbox}
\graphicspath{ {./images/} }
\usepackage{caption}
\usepackage{amsmath}
\usepackage{amsfonts}
\usepackage{amssymb}
\usepackage[version=4]{mhchem}
\usepackage{stmaryrd}

\begin{document}
\captionsetup{singlelinecheck=false}
\begin{figure}[h]
\begin{center}
  \includegraphics[max width=\textwidth]{08dea139-92b4-4d1d-9215-8cc499778a29-01}
\captionsetup{labelformat=empty}
\caption{Figure 23.1 Part of the Large Hadron Collider (LHC) at CERN, on the border of Switzerland and France. The LHC is a particle accelerator, designed to study fundamental particles. (credit: Image Editor, Flickr)}
\end{center}
\end{figure}

\section*{Chapter Outline}
\subsection*{23.1 The Four Fundamental Forces}
\subsection*{23.2 Quarks}
\subsection*{23.3 The Unification of Forces}
\section*{Introduction}
Following ideas remarkably similar to those of the ancient Greeks, we continue to look for smaller and smaller structures in nature, hoping ultimately to find and understand the most fundamental building blocks that exist. Atomic physics deals with the smallest units of elements and compounds. In its study, we have found a relatively small number of atoms with systematic properties, and these properties have explained a tremendous range of phenomena. Nuclear physics is concerned with the nuclei of atoms and their substructures. Here, a smaller number of components-the proton and neutron-make up all nuclei. Exploring the systematic behavior of their interactions has revealed even more about matter, forces, and energy. Particle physics deals with the substructures of atoms and nuclei and is particularly aimed at finding those truly fundamental particles that have no further substructure. Just as in atomic and nuclear physics, we have found a complex array of particles and properties with systematic characteristics analogous to the periodic table and the chart of nuclides. An underlying structure is apparent, and there is some reason to think that we are finding particles that have no substructure. Of course, we have been in similar situations before. For example, atoms were once thought to be the ultimate substructures. It is possible that we could continue to find deeper and deeper\\
structures without ever discovering the ultimate substructure-in science there is never complete certainty. See Figure 23.2.

The properties of matter are based on substructures called molecules and atoms. Each atom has the substructure of a nucleus surrounded by electrons, and their interactions explain atomic properties. Protons and neutrons - and the interactions between them-explain the stability and abundance of elements and form the substructure of nuclei. Protons and neutrons are not fundamental-they are composed of quarks. Like electrons and a few other particles, quarks may be the fundamental building blocks of all matter, lacking any further substructure. But the story is not complete because quarks and electrons may have substructures smaller than details that are presently observable.

\begin{figure}[h]
\begin{center}
  \includegraphics[max width=\textwidth]{08dea139-92b4-4d1d-9215-8cc499778a29-02}
\captionsetup{labelformat=empty}
\caption{Figure 23.2 A solid, a molecule, an atom, a nucleus, a nucleon (a particle that makes up the nucleus- either a proton or a neutron), and a quark.\\
This chapter covers the basics of particle physics as we know it today. An amazing convergence of topics is evolving in particle physics. We find that some particles are intimately related to forces and that nature on the smallest scale may have its greatest influence on the large scale character of the universe. It is an adventure exceeding the best science fiction because it is not only fantastic but also real.}
\end{center}
\end{figure}

\subsection*{23.1 The Four Fundamental Forces}
\section*{Section Learning Objectives}
By the end of the section, you will be able to do the following:

\begin{itemize}
  \item Define, describe, and differentiate the four fundamental forces
  \item Describe the carrier particles and explain how their exchange transmits force
  \item Explain how particle accelerators work to gather evidence about particle physics
\end{itemize}

\section*{Teacher Support}
Teacher Support The learning objectives in this section will help your students master the following standards:

\begin{itemize}
  \item (5) Science concepts. The student knows the nature of forces in the physical world. The student is expected to:
  \item (H) describe evidence for and effects of the strong and weak nuclear forces in nature.
\end{itemize}

\section*{Section Key Terms}
\section*{Teacher Support}
Teacher Support Prior to the section, have students create a list of different forces. Additionally, it may be valuable to review gravitational and electric fields, Rutherford's gold foil experiment, the Van de Graaff generator, particle decay, and the impulse - momentum theorem.

Despite the apparent complexity within the universe, there remain just four basic forces. These forces are responsible for all interactions known to science: from the very small to the very large to those that we experience in our day-today lives. These forces describe the movement of galaxies, the chemical reactions in our laboratories, the structure within atomic nuclei, and the cause of radioactive decay. They describe the true cause behind familiar terms like friction and the normal force. These four basic forces are known as fundamental because they alone are responsible for all observations of forces in nature. The four fundamental forces are gravity, electromagnetism, weak nuclear force, and strong nuclear force.

\section*{Teacher Support}
\section*{Teacher Support}
\begin{itemize}
  \item Some students may be disheartened by the idea that two of the four fundamental forces have barely been discussed in the course. Remind them that nearly all of their daily observations and interactions are due to the gravitational and electromagnetic forces and that this chapter will not be un-teaching their prior lessons. Discussions of the strong and weak nuclear forces are designed to help them understand particle organization and particle decay.
  \item Prior to beginning the lesson, have students predict the relative strengths of the four forces, from weakest to strongest. Refer to their rankings as you present the section. Later, this will be a useful tool to discuss why the weak force isn't the weakest force and should provide students a greater appreciation for the discovery of science.
\end{itemize}

\section*{Understanding the Four Forces}
The gravitational force is most familiar to us because it describes so many of our common observations. It explains why a dropped ball falls to the ground and why our planet orbits the Sun. It gives us the property of weight and determines much about the motion of objects in our daily lives. Because gravitational force acts between all objects of mass and has the ability to act over large distances, the gravitational force can be used to explain much of what we observe and can even describe the motion of objects on astronomical scales! That said, gravity is incredibly weak compared to the other fundamental forces and is the weakest of all of the fundamental forces. Consider this: The entire mass of Earth is needed to hold an iron nail to the ground. Yet with a simple magnet, the force of gravity can be overcome, allowing the nail to accelerate upward through space.

\section*{Teacher Support}
Teacher Support A brief nail-magnet demonstration as described above is a powerful way to help students recognize the weakness of gravity. Discussion can follow regarding why people tend to feel that gravity is a strong force, what would happen if a charged object the size of Earth existed, and so forth.

The electromagnetic force is responsible for both electrostatic interactions and the magnetic force seen between bar magnets. When focusing on the electrostatic relationship between two charged particles, the electromagnetic force is known as the coulomb force. The electromagnetic force is an important force in the chemical and biological sciences, as it is responsible for molecular connections like ionic bonding and hydrogen bonding. Additionally, the electromagnetic force is behind the common physics forces of friction and the normal force. Like the gravitational force, the electromagnetic force is an inverse square law. However, the electromagnetic force does not exist between any two objects of mass, only those that are charged.

\section*{Teacher Support}
Teacher Support Take a moment to review forces and chemistry concepts. Look at each from the perspective of the electromagnetic force. Consider terms like the normal force, friction, chemical bonding, surface tension, and capillary action. Allow students to consider the pervasive nature of the electromagnetic force in their daily observations.

When considering the structure of an atom, the electromagnetic force is somewhat apparent. After all, the electrons are held in place by an attractive force from the nucleus. But what causes the nucleus to remain intact? After all, if all protons are positive, it makes sense that the coulomb force between the protons would repel the nucleus apart immediately. Scientists theorized that another force must exist within the nucleus to keep it together. They further theorized that this nuclear force must be significantly stronger than gravity, which has been observed and measured for centuries, and also stronger than the electromagnetic force, which would cause the protons to want to accelerate away from each other.

The strong nuclear force is an attractive force that exists between all nucleons. This force, which acts equally between proton-proton connections, protonneutron connections, and neutron-neutron connections, is the strongest of all forces at short ranges. However, at a distance of \(10^{-13} \mathrm{~cm}\), or the diameter of a single proton, the force dissipates to zero. If the nucleus is large (it has many nucleons), then the distance between each nucleon could be much larger than the diameter of a single proton.

\section*{Teacher Support}
Teacher Support At this time, it may be useful to have students construct a table categorizing each for based on important characteristics: attractive or attractive/repulsive force, force strength, force range, and so on. While Table 23.1 will help in organization, students will make more sense of the material if they create a table on their own.\\[0pt]
[OL][AL]To make the strength-distance relationship of the strong force more visible, show students the graph below.\\
\includegraphics[max width=\textwidth, center]{08dea139-92b4-4d1d-9215-8cc499778a29-05}

Note that repulsion is considered a positive strong force, while attraction is considered a negative strong force.\\[0pt]
[AL]Ask students what distance aligns with the valley representing the maximum attractive force. Note-This is equivalent to the average distance between two nucleons.

The weak nuclear force is responsible for beta decay, as seen in the equation\\
\({ }_{Z}^{A} X_{N} \quad \rightarrow \quad{ }_{Z+1}^{A} Y_{N-1}+e+v\).\\
Recall that beta decay is when a beta particle is ejected from an atom. In order to accelerate away from the nucleus, the particle must be acted on by a force. Enrico Fermi was the first to envision this type of force. While this force is appropriately labeled, it remains stronger than the gravitational force. However, its range is even smaller than that of the strong force, as can be seen in Table 23.1. The weak nuclear force is more important than it may appear at this time, as will be addressed when we discuss quarks.

Table 23.1 Relative strength and range of the four fundamental forces

\section*{Transmitting the Four Fundamental Forces}
\section*{Teacher Support}
Teacher Support This subsection may be troubling to students who have a great appreciation for the gravitational field concept. You can reassure these students that the field concept is still used to explain and predict phenomena.

Just as it troubled Einstein prior to formulating the gravitational field theory, the concept of forces acting over a distance had greatly troubled particle physicists. That is, how does one proton know that another exists? Furthermore, what causes one proton to make a second proton repel? Or, for that matter, what is it about a proton that causes a neutron to attract? These mysterious interactions were first considered by Hideki Yukawa in 1935 and laid the foundation for much of what we now understand about particle physics.

Hideki Yukawa's focus was on the strong nuclear force and, in particular, its incredibly short range. His idea was a blend of particles, relativity, and quantum mechanics that was applicable to all four forces. Yukawa proposed that the nuclear force is actually transmitted by the exchange of particles, called carrier particles, and that what we commonly refer to as the force's field consists of these\\
carrier particles. Specifically for the strong nuclear force, Yukawa proposed that a previously unknown particle, called a pion, is exchanged between nucleons, transmitting the force between them. Figure 23.3 illustrates how a pion would carry a force between a proton and a neutron.

\begin{figure}[h]
\begin{center}
  \includegraphics[max width=\textwidth]{08dea139-92b4-4d1d-9215-8cc499778a29-07}
\captionsetup{labelformat=empty}
\caption{Figure 23.3 The strong nuclear force is transmitted between a proton and neutron by the creation and exchange of a pion. The pion, created through a temporary violation of conservation of mass-energy, travels from the proton to the neutron and is recaptured. It is not directly observable and is called a virtual particle. Note that the proton and neutron change identity in the process. The range of the force is limited by the fact that the pion can exist for only the short time allowed by the Heisenberg uncertainty principle. Yukawa used the finite range of the strong nuclear force to estimate the mass of the pion; the shorter the range, the larger the mass of the carrier particle.}
\end{center}
\end{figure}

In Yukawa's strong force, the carrier particle is assumed to be transmitted at the speed of light and is continually transferred between the two nucleons shown. The particle that Yukawa predicted was finally discovered within cosmic rays in 1947 . Its name, the pion, stands for pi meson, where meson means medium mass; it's a medium mass because it is smaller than a nucleon but larger than an electron. Yukawa launched the field that is now called quantum chromodynamics, and the carrier particles are now called gluons due to their strong binding power. The reason for the change in the particle name will be explained when quarks are discussed later in this section.

As you may assume, the strong force is not the only force with a carrier particle. Nuclear decay from the weak force also requires a particle transfer. In the weak force are the following three: the weak negative carrier, \(\mathrm{W}^{-}\); the weak positive carrier, \(\mathrm{W}^{+}\); and the zero charge carrier, \(\mathrm{Z}^{0}\). As we will see, Fermi inferred that these particles must carry mass, as the total mass of the products of nuclear decay is slightly larger than the total mass of all reactants after nuclear decay.

The carrier particle for the electromagnetic force is, not surprisingly, the photon. After all, just as a lightbulb can emit photons from a charged tungsten filament, the photon can be used to transfer information from one electrically charged particle to another. Finally, the graviton is the proposed carrier particle for gravity. While it has not yet been found, scientists are currently looking for evidence of its existence (see Boundless Physics: Searching for the Graviton).

\section*{Teacher Support}
Teacher Support [BL][OL]The concept of a carrier particle transmitting a force is similar to the concept of using energy to transmit information. How do you let your friend know that you are at school? Do you send a text message? Wave to them at their locker? Shout down the hallway? In each case, waves transfer energy from you to your friend. In the same manner, carrier particles are necessary to carry the weak nuclear force.

So how does a carrier particle transmit a fundamental force? Figure 23.4 shows a virtual photon transmitted from one positively charged particle to another. The transmitted photon is referred to as a virtual particle because it cannot be directly observed while transmitting the force. Figure 23.5 shows a way of graphing the exchange of a virtual photon between the two positively charged particles. This graph of time versus position is called a Feynman diagram, after the brilliant American physicist Richard Feynman (1918-1988), who developed it.

\begin{figure}[h]
\begin{center}
  \includegraphics[max width=\textwidth]{08dea139-92b4-4d1d-9215-8cc499778a29-08}
\captionsetup{labelformat=empty}
\caption{Figure 23.4 The image in part (a) shows the exchange of a virtual photon transmitting the electromagnetic force between charges, just as virtual pion exchange carries the strong nuclear force between nucleons. The image in part (b) shows that the photon cannot be directly observed in its passage because this would disrupt it and alter the force. In this case, the photon does not reach the other charge.}
\end{center}
\end{figure}

\section*{Teacher Support}
Teacher Support The term virtual particle actually refers to a disturbance in space created by the presence of the two nucleons. Use of the term virtual hints at the idea that the carrier particle should not be confused with a regular particle of mass. A full understanding of the true nature of virtual particles, however, relies on mathematics and theory beyond the scope of this text.

The Feynman diagram should be read from the bottom up to show the movement of particles over time. In it, you can see that the left proton is propelled leftward from the photon emission, while the right proton feels an impulse to the right when the photon is received. In addition to the Feynman diagram, Richard Feynman was one of the theorists who developed the field of quantum electrodynamics (QED), which further describes electromagnetic interactions on\\
the submicroscopic scale. For this work, he shared the 1965 Nobel Prize with Julian Schwinger and S.I. Tomonaga. A Feynman diagram explaining the strong force interaction hypothesized by Yukawa can be seen in Figure 23.6. Here, you can see the change in particle type due to the exchange of the pi meson.

\section*{Teacher Support}
Teacher Support [BL][OL][AL]To help students explain the Feynman diagram, ask a number of specific questions. In what direction is the left proton traveling? What happens to it when it releases a virtual photon? In what direction does the virtual photon travel? Is the photon received by the right proton at the same time as it is emitted? What effect does the photon have on the right proton's trajectory?\\
\includegraphics[max width=\textwidth, center]{08dea139-92b4-4d1d-9215-8cc499778a29-09(1)}

Figure 23.5 The Feynman diagram for the exchange of a virtual photon between two positively charged particles illustrates how electromagnetic force is transmitted on a quantum mechanical scale. Time is graphed vertically, while the distance is graphed horizontally. The two positively charged particles are seen to repel each other by the photon exchange.\\
\includegraphics[max width=\textwidth, center]{08dea139-92b4-4d1d-9215-8cc499778a29-09}

Figure 23.6 The image shows a Feynman diagram for the exchange of a + (pion) between a proton and a neutron, carrying the strong nuclear force between them. This diagram represents the situation shown more pictorially in Figure 23.3.

The relative masses of the listed carrier particles describe something valuable about the four fundamental forces, as can be seen in Table 23.2. W bosons (consisting of \(\mathrm{W}^{-}\)and \(\mathrm{W}^{+}\)bosons) and Z bosons ( \(\mathrm{Z}^{0}\) bosons), carriers of the weak nuclear force, are nearly 1,000 times more massive than pions, carriers of the strong nuclear force. Simultaneously, the distance that the weak nuclear force can be transmitted is approximately \(\frac{1}{1,000}\) times the strong force transmission distance. Unlike carrier particles, which have a limited range, the photon is a massless particle that has no limit to the transmission distance of the electromagnetic force. This relationship leads scientists to understand that the yet-unfound graviton is likely massless as well.

Table 23.2 Carrier particles and their relative masses compared to pions for the four fundamental forces

\section*{Boundless Physics}
Searching for the Graviton From Newton's Universal Law of Gravitation to Einstein's field equations, gravitation has held the focus of scientists for centuries. Given the discovery of carrier particles during the twentieth century, the importance of understanding gravitation has yet again gained the interest of prominent physicists everywhere.

With carrier particles discovered for three of the four fundamental forces, it is sensible to scientists that a similar particle, titled the graviton, must exist for the gravitational force. While evidence of this particle is yet to be uncovered, scientists are working diligently to discover its existence.

So what do scientists think about the unfound particle? For starters, the graviton (like the photon) should be a massless particle traveling at the speed of light. This is assumed because, like the electromagnetic force, gravity is an inverse square law, \(F \approx \frac{1}{r^{2}}\). Scientists also theorize that the graviton is an electrically neutral particle, as an empty space within the influence of gravity is chargeless.

\section*{Teacher Support}
Teacher Support Understanding why gravitons have a quantum mechanical spin of 2 is not necessary at this time. This information is provided so that students recognize that the graviton has different characteristics from the photon, another massless, chargeless carrier particle.

However, because gravity is such a weak force, searching for the graviton has resulted in some unique methods. LIGO, the Laser Interferometer GravitationalWave Observatory, is one tool currently being utilized (see Figure 23.7). While searching for a gravitational wave to find a carrier particle may seem counterintuitive, it is similar to the approach taken by Planck and Einstein to learn more about the photon. According to wave-particle duality, if a gravitational wave can be found, the graviton should be present along with it. Predicted by Einstein's theory of general relativity, scientists have been monitoring binary star systems for evidence of these gravitational waves.

\begin{figure}[h]
\begin{center}
  \includegraphics[max width=\textwidth]{08dea139-92b4-4d1d-9215-8cc499778a29-11}
\captionsetup{labelformat=empty}
\caption{Figure 23.7 In searching for gravitational waves, scientists are using the Laser Interferometer Gravitational-Wave Observatory (LIGO). Here we see the control room of LIGO in Hanford, Washington.}
\end{center}
\end{figure}

Particle accelerators like the Large Hadron Collider (LHC) are being used to search for the graviton through high-energy collisions. While scientists at the LHC speculate that the particle may not exist long enough to be seen, evidence of its prior existence, like footprints in the sand, can be found through gaps in projected energy and momentum.

Some scientists are even searching the remnants of the Big Bang in an attempt to find the graviton. By observing the cosmic background radiation, they are looking for anomalies in gravitational waves that would provide information about the gravity particles that existed at the start of our universe.

Regardless of the method used, scientists should know the graviton once they find it. A massless, chargeless particle with a spin of 2 and traveling at the speed of light-there is no other particle like it. Should it be found, its discovery would surely be considered by future generations to be on par with those of Newton and Einstein.

Why are binary star systems used by LIGO to find gravitational waves?\\
a. Binary star systems have high temperature.\\
b. Binary star systems have low density.\\
c. Binary star systems contain a large amount of mass, but because they are orbiting each other, the gravitational field between the two is much less.\\
d. Binary star systems contain a large amount of mass. As a result, the gravitational field between the two is great.

\section*{Accelerators Create Matter From Energy}
Before looking at all the particles that make up our universe, let us first examine some of the machines that create them. The fundamental process in creating unknown particles is to accelerate known particles, such as protons or electrons, and direct a beam of them toward a target. Collisions with target nuclei provide a wealth of information, such as information obtained by Rutherford in the gold foil experiment. If the energy of the incoming particles is large enough, new matter can even be created in the collision. The more energy input or \(\Delta E\), the more matter \(m\) can be created, according to mass energy equivalence \(m=\Delta E / c^{2}\). Limitations are placed on what can occur by known conservation laws, such as conservation of mass-energy, momentum, and charge. Even more interesting are the unknown limitations provided by nature. While some expected reactions do occur, others do not, and still other unexpected reactions may appear. New laws are revealed, and the vast majority of what we know about particle physics has come from accelerator laboratories. It is the particle physicist's favorite indoor sport.

Our earliest model of a particle accelerator comes from the Van de Graaff generator. The relatively simple device, which you have likely seen in physics demonstrations, can be manipulated to produce potentials as great as 50 million volts. While these machines do not have energies large enough to produce new particles, analysis of their accelerated ions was instrumental in exploring several aspects of the nucleus.

Another equally famous early accelerator is the cyclotron, invented in 1930 by the American physicist, E.O. Lawrence (1901-1958). Figure 23.8 is a visual representation with more detail. Cyclotrons use fixed-frequency alternating electric fields to accelerate particles. The particles spiral outward in a magnetic field, making increasingly larger radius orbits during acceleration. This clever arrangement allows the successive addition of electric potential energy with each loop. As a result, greater particle energies are possible than in a Van de Graaff generator.

\section*{Teacher Support}
Teacher Support [BL][OL][AL]There are two forces acting with the cyclotron in Figure 23.8. A centripetal force is created by the magnetic field, which can be verified to the students through use of the right-hand rule. Additionally, the voltage differential creates a linear force at the gap between\\
the two Dees. Note to the students that the voltage gap reverses direction at a constant rate so that the linear force is always in the direction of the particle's travel.

\begin{figure}[h]
\begin{center}
  \includegraphics[max width=\textwidth]{08dea139-92b4-4d1d-9215-8cc499778a29-13}
\captionsetup{labelformat=empty}
\caption{Figure 23.8 On the left is an artist's rendition of the popular physics demonstration tool, the Van de Graaff generator. A battery (A) supplies excess positive charge to a pointed conductor, the points of which spray the charge onto a moving insulating belt near the bottom. The pointed conductor (B) on top in the large sphere picks up the charge. (The induced electric field at the points is so large that it removes the charge from the belt.) This can be done because the charge does not remain inside the conducting sphere but moves to its outer surface. An ion source inside the sphere produces positive ions, which are accelerated away from the positive sphere to high velocities. On the right is a cyclotron. Cyclotrons use a magnetic field to cause particles to move in circular orbits. As the particles pass between the plates of the Dees, the voltage across the gap is oscillated to accelerate them twice in each orbit.}
\end{center}
\end{figure}

A synchrotron is a modification of the cyclotron in which particles continually travel in a fixed-radius orbit, increasing speed each time. Accelerating voltages are synchronized with the particles to accelerate them, hence the name. Additionally, magnetic field strength is increased to keep the orbital radius constant as energy increases. A ring of magnets and accelerating tubes, as shown in Figure 23.9, are the major components of synchrotrons. High-energy particles require strong magnetic fields to steer them, so superconducting magnets are commonly employed. Still limited by achievable magnetic field strengths, synchrotrons need to be very large at very high energies since the radius of a\\
high-energy particle's orbit is very large.

\section*{Teacher Support}
Teacher Support [BL][OL]Assist the students in constructing a chart or Venn diagram showing similarities and differences between the cyclotron and synchrotron. Consideration should include particle travel, rate of voltage oscillation, and device construction.

To further probe the nucleus, physicists need accelerators of greater energy and detectors of shorter wavelength. To do so requires not only greater funding but greater ingenuity as well. Colliding beams used at both the Fermi National Accelerator Laboratory (Fermilab; see Figure 23.11) near Chicago and the LHC in Switzerland are designed to reduce energy loss in particle collisions. Typical stationary particle detectors lose a large amount of energy to the recoiling target struck by the accelerating particle. By providing head-on collisions between particles moving in opposite directions, colliding beams make it possible to create particles with momenta and kinetic energies near zero. This allows for particles of greater energy and mass to be created. Figure 23.10 is a schematic representation of this effect. In addition to circular accelerators, linear accelerators can be used to reduce energy radiation losses. The Stanford Linear Accelerator Center (now called the SLAC National Accelerator Laboratory) in California is home to the largest such accelerator in the world.

\begin{figure}[h]
\begin{center}
  \includegraphics[max width=\textwidth]{08dea139-92b4-4d1d-9215-8cc499778a29-14}
\captionsetup{labelformat=empty}
\caption{Figure 23.9 (a) A synchrotron has a ring of magnets and accelerating tubes. The frequency of the accelerating voltages is increased to cause the beam particles to travel the same distance in a shorter time. The magnetic field should also be increased to keep each beam burst traveling in a fixed-radius path. Limits on magnetic field strength require these machines to be very large in order to accelerate particles to very high energies. (b) A positively charged particle is shown in the gap between accelerating tubes. (c) While the particle passes through the tube, the potentials are reversed so that there is another acceleration at the next gap. The frequency of the reversals needs to be varied as the particle is accelerated to achieve successive accelerations in each gap.}
\end{center}
\end{figure}

\begin{figure}[h]
\begin{center}
  \includegraphics[max width=\textwidth]{08dea139-92b4-4d1d-9215-8cc499778a29-15}
\captionsetup{labelformat=empty}
\caption{Figure 23.10 This schematic shows the two rings of Fermilab's accelerator and the scheme for colliding protons and antiprotons (not to scale).}
\end{center}
\end{figure}

\begin{figure}[h]
\begin{center}
  \includegraphics[max width=\textwidth]{08dea139-92b4-4d1d-9215-8cc499778a29-15(1)}
\captionsetup{labelformat=empty}
\caption{Figure 23.11 The Fermi National Accelerator Laboratory, near Batavia, Illinois, was a subatomic particle collider that accelerated protons and antiprotons to attain energies up to 1 Tev (a trillion electronvolts). The circular ponds near the rings were built to dissipate waste heat. This accelerator was shut down in September 2011. (credit: Fermilab, Reidar Hahn)}
\end{center}
\end{figure}

\section*{Check Your Understanding}
\section*{Teacher Support}
Teacher Support Use these questions to assess student achievement of the section's learning objectives. If students are struggling with a specific objective, these questions will help identify which and direct students to the relevant content.\\
1.

Which of the four forces is responsible for radioactive decay?\\
a. the electromagnetic force\\
b. the gravitational force\\
c. the strong nuclear force\\
d. the weak nuclear force\\
2.

What force or forces exist between an electron and a proton?\\
a. the strong nuclear force, the electromagnetic force, and gravity\\
b. the weak nuclear force, the strong nuclear force, and gravity\\
c. the weak nuclear force, the strong nuclear force, and the electromagnetic force\\
d. the weak nuclear force, the electromagnetic force, and gravity\\
3.

What is the proposed carrier particle for the gravitational force?\\
a. boson\\
b. graviton\\
c. gluon\\
d. photon\\
4.

What is the relationship between the mass and range of a carrier particle?\\
a. Range of a carrier particle is inversely proportional to its mass.\\
b. Range of a carrier particle is inversely proportional to square of its mass.\\
c. Range of a carrier particle is directly proportional to its mass.\\
d. Range of a carrier particle is directly proportional to square of its mass.\\
5.

What type of particle accelerator uses fixed-frequency oscillating electric fields to accelerate particles?\\
a. cyclotron\\
b. synchrotron\\
c. betatron\\
d. Van de Graaff accelerator\\
6.

How does the expanding radius of the cyclotron provide evidence of particle acceleration?\\
a. A constant magnetic force is exerted on particles at all radii. As the radius increases, the velocity of the particle must increase to maintain this constant force.\\
b. A constant centripetal force is exerted on particles at all radii. As the radius increases, the velocity of the particle must decrease to maintain this constant force.\\
c. A constant magnetic force is exerted on particles at all radii. As the radius increases, the velocity of the particle must decrease to maintain this constant force.\\
d. A constant centripetal force is exerted on particles at all radii. As the radius increases, the velocity of the particle must increase to maintain this constant force.\\
7.

Which of the four forces is responsible for the structure of galaxies?\\
a. electromagnetic force\\
b. gravity\\
c. strong nuclear force\\
d. weak nuclear force

\subsection*{23.2 Quarks}
\section*{Section Learning Objectives}
By the end of the section, you will be able to do the following:

\begin{itemize}
  \item Describe quarks and their relationship to other particles
  \item Distinguish hadrons from leptons
  \item Distinguish matter from antimatter
  \item Describe the standard model of the atom
  \item Define a Higgs boson and its importance to particle physics
\end{itemize}

\section*{Teacher Support}
Teacher Support The learning objectives in this section will help your students master the following standards:

\begin{itemize}
  \item (5) Science concepts. The student knows the nature of forces in the physical world. The student is expected to:
  \item (H) describe evidence for and effects of the strong and weak nuclear forces in nature.
\end{itemize}

\section*{Section Key Terms}
\section*{Teacher Support}
Teacher Support Prior to the section, it may be valuable to review the following terms and concepts: electron scattering, discrete quantities, color theory, and particle decay.

\section*{Quarks}
"The first principles of the universe are atoms and empty space. Everything else is merely thought to exist..."\\
"... Further, the atoms are unlimited in size and number, and they are borne along with the whole universe in a vortex, and thereby generate all composite things-fire, water, air, earth. For even these are conglomerations of given atoms. And it because of their solidity that these atoms are impassive and unalterable."\\
-Diogenes Laertius (summarizing the views of Democritus, circa 460-370 B.C.)\\
The search for fundamental particles is nothing new. Atomists of the Greek and Indian empires, like Democritus of fifth century B.C., openly wondered about the most finite components of our universe. Though dormant for centuries, curiosity about the atomic nature of matter was reinvigorated by Rutherford's gold foil experiment and the discovery of the nucleus. By the early 1930s, scientists believed they had fully determined the tiniest constituents of matter-in the form of the proton, neutron, and electron.

\section*{Teacher Support}
Teacher Support [AL]Have interested students read the texts of Democritus and Empedocles to see how atomism was viewed in the Greek era. Have the students compare and contrast these theories to our current scientific model.

This would be only partially true. At present, scientists know that there are hundreds of particles not unlike our electron and nucleons, all making up what some have termed the particle \(o o\) While we are confident that the electron remains fundamental, it is surrounded by a plethora of similar sounding terms, like leptons, hadrons, baryons, and mesons. Even though not every particle is considered fundamental, they all play a vital role in understanding the intricate structure of our universe.

A fundamental particle is defined as a particle with no substructure and no finite size. According to the Standard Model, there are three types of fundamental particles: leptons, quarks, and carrier particles. As you may recall, carrier particles are responsible for transmitting fundamental forces between their interacting masses. Leptons are a group of six particles not bound by the strong nuclear force, of which the electron is one. As for quarks, they are the fundamental building blocks of a group of particles called hadrons, a group that includes both the proton and the neutron.

\section*{Teacher Support}
Teacher Support [BL]It may be helpful to construct a graphical organizer, like a family tree, to show the relationship between various particles within the Standard Model. While this will eventually take place when the Standard Model is introduced, constructing a graphical organizer while moving through the unit will help provide structure to the students' knowledge.

Now for a brief history of quarks. Quarks were first proposed independently by American physicists Murray Gell-Mann and George Zweig in 1963. Originally, three quark types - or flavors - were proposed with the names up \((u)\), down \((d)\), and strange ( \(s\) ).

At first, physicists expected that, with sufficient energy, we should be able to free quarks and observe them directly. However, this has not proved possible,\\
as the current understanding is that the force holding quarks together is incredibly great and, much like a spring, increases in magnitude as the quarks are separated. As a result, when large energies are put into collisions, other particles are created-but no quarks emerge. With that in mind, there is compelling evidence for the existence of quarks. By 1967, experiments at the SLAC National Accelerator Laboratory scattering \(20-\mathrm{GeV}\) electrons from protons produced results like Rutherford had obtained for the nucleus nearly 60 years earlier. The SLAC scattering experiments showed unambiguously that there were three point-like (meaning they had sizes considerably smaller than the probe's wavelength) charges inside the proton as seen in Figure 23.12. This evidence made all but the most skeptical admit that there was validity to the quark substructure of hadrons.

\section*{Teacher Support}
Teacher Support Students should reflect on the idea that quarks have not been observed directly. Help them recognize that the lack of direct evidence does not mean that the evidence for quarks is weak. As will be seen in the discussion on fractional charge and color, the reason an individual quark has not been exposed may be more complex than expected.\\
\includegraphics[max width=\textwidth, center]{08dea139-92b4-4d1d-9215-8cc499778a29-20}

Figure 23.12 Scattering of high-energy electrons from protons at facilities like SLAC produces evidence of three point-like charges consistent with proposed quark properties. This experiment is analogous to Rutherford's discovery of the small size of the nucleus by scattering particles. High-energy electrons are used so that the probe wavelength is small enough to see details smaller than the proton.

The inclusion of the strange quark with Zweig and Gell-Mann's model concerned physicists. While the up and down quarks demonstrated fairly clear symmetry and were present in common fundamental particles like protons and neutrons, the strange quark did not have a counterpart of its own. This thought, coupled with the four known leptons at the time, caused scientists to predict that a fourth quark, yet to be found, also existed.

In 1974, two groups of physicists independently discovered a particle with this\\
new quark, labeled charmed. This completed the second \(e\) otic quark pair, strange (s) and charmed (c). A final pair of quarks was proposed when a third pair of leptons was discovered in 1975. The existence of the bottom (b) quark and the top ( t ) quark was verified through experimentation in 1976 and 1995, respectively. While it may seem odd that so much time would elapse between the original quark discovery in 1967 and the verification of the top quark in 1995, keep in mind that each quark discovered had a progressively larger mass. As a result, each new quark has required more energy to discover.

\section*{Tips For Success}
Note that a very important tenet of science occurred throughout the period of quark discovery. The charmed, bottom, and top quarks were all speculated on, and then were discovered some time later. Each of their discoveries helped to verify and strengthen the quark model. This process of speculation and verification continues to take place today and is part of what drives physicists to search for evidence of the graviton and Grand Unified Theory.

One of the most confounding traits of quarks is their electric charge. Long assumed to be discrete, and specifically a multiple of the elementary charge of the electron, the electric charge of an individual quark is fractional and thus seems to violate a presumed tenet of particle physics. The fractional charge of quarks, which are \(\pm\left(\frac{2}{3}\right) q_{e}\) and \(\pm\left(\frac{1}{3}\right) q_{e}\), are the only structures found in nature with a nonintegral number of charge \(q\). However, note that despite this odd construction, the fractional value of the quark does not violate the quantum nature of the charge. After all, free quarks cannot be found in nature, and all quarks are bound into arrangements in which an integer number of charge is constructed. Table 23.3 shows the six known quarks, in addition to their antiquark components, as will be discussed later in this section.

Table 23.3 Quarks and Antiquarks\\
While the term avor is used to differentiate between types of quarks, the concept of color is more analogous to the electric charge in that it is primarily responsible for the force interactions between quarks. Note-Take a moment to think about the electrostatic force. It is the electric charge that causes attraction and repulsion. It is the same case here but with a color charge. The three colors\\
available to a quark are red, green, and blue, with antiquarks having colors of anti-red (or cyan), anti-green (or magenta), and anti-blue (or yellow).

Why use colors when discussing quarks? After all, the quarks are not actually colored with visible light. The reason colors are used is because the properties of a quark are analogous to the three primary and secondary colors mentioned above. Just as different colors of light can be combined to create white, different colors of quark may be combined to construct a particle like a proton or neutron. In fact, for each hadron, the quarks must combine such that their color sums to white! Recall that two up quarks and one down quark construct a proton, as seen in Figure 23.12. The sum of the three quarks' colors-red, green, and blue-yields the color white. This theory of color interaction within particles is called quantum chromodynamics, or QCD. As part of QCD, the strong nuclear force can be explained using color. In fact, some scientists refer to the color force, not the strong force, as one of the four fundamental forces. Figure 23.13 is a Feynman diagram showing the interaction between two quarks by using the transmission of a colored gluon. Note that the gluon is also considered the charge carrier for the strong nuclear force.

\section*{Teacher Support}
Teacher Support [BL][OL]Review color theory with students to help them understand the usefulness of the color concept. When combined, what three colors create a white quark? When combined what three anticolors create a white antiquark? How does the combination of a color and anticolor create a white quark? Furthermore, move slowly through Figure 23.13. Why does the down quark turn from red to green? Why does this cause the strange quark to turn from green to red? What color is the gluon? Is color conserved in this interaction?\\
\includegraphics[max width=\textwidth, center]{08dea139-92b4-4d1d-9215-8cc499778a29-22}

Figure 23.13 The exchange of gluons between quarks carries the strong force and may change the color of the interacting quarks. While the colors of the individual quarks change, their flavors do not.

Note that quark flavor may have any color. For instance, in Figure 23.13, the\\
down quark has a red color and a green color. In other words, colors are not specific to a particle quark flavor.

\section*{Hadrons and Leptons}
Particles can be revealingly grouped according to what forces they feel between them. All particles (even those that are massless) are affected by gravity since gravity affects the space and time in which particles exist. All charged particles are affected by the electromagnetic force, as are neutral particles that have an internal distribution of charge (such as the neutron with its magnetic moment). Special names are given to particles that feel the strong and weak nuclear forces. Hadrons are particles that feel the strong nuclear force, whereas leptons are particles that do not. All particles feel the weak nuclear force. This means that hadrons are distinguished by being able to feel both the strong and weak nuclear forces. Leptons and hadrons are distinguished in other ways as well. Leptons are fundamental particles that have no measurable size, while hadrons are composed of quarks and have a diameter on the order of \(10^{-15} \mathrm{~m}\). Six particles, including the electron and neutrino, make up the list of known leptons. There are hundreds of complex particles in the hadron class, a few of which (including the proton and neutron) are listed in Table 23.4.

Table 23.4 List of Leptons and Hadrons.\\
There are many more leptons, mesons, and baryons yet to be discovered and measured. The purpose of trying to uncover the smallest indivisible things in existence is to explain the world around us through forces and the interactions\\
between particles, galaxies and objects. This is why a handful of scientists devote their life's work to smashing together small particles.

\section*{Teacher Support}
Teacher Support As a review of the four fundamental forces, place a particle on the board and have students identify which forces interact with it (electron: gravity, electromagnetic, weak; proton: all four). Then show that the electron is a type of lepton, while a proton is a type of hadron. Now have the students identify what forces act on the broad categories of particles of hadrons and leptons. Note that this concept will be reinforced during discussion of the Standard Model.

What internal structure makes a proton so different from an electron? The proton, like all hadrons, is made up of quarks. A few examples of hadron quark composition can be seen in Figure 23.14. As shown, each hadron is constructed of multiple quarks. As mentioned previously, the fractional quark charge in all four hadrons sums to the particle's integral value. Also, notice that the color composition for each of the four particles adds to white. Each of the particles shown is constructed of up, down, and their antiquarks. This is not surprising, as the quarks strange, charmed, top, and bottom are found in only our most exotic particles.

\begin{figure}[h]
\begin{center}
  \includegraphics[max width=\textwidth]{08dea139-92b4-4d1d-9215-8cc499778a29-24}
\captionsetup{labelformat=empty}
\caption{Figure 23.14 All baryons, such as the proton and neutron shown here, are composed of three quarks. All mesons, such as the pions shown here, are composed of a quark-antiquark pair. Arrows represent the spins of the quarks. The colors are such that they need to add to white for any possible combination of quarks.}
\end{center}
\end{figure}

You may have noticed that while the proton and neutron in Figure 23.14 are composed of three quarks, both pions are comprised of only two quarks. This refers to a final delineation in particle structure. Particles with three quarks are called baryons. These are heavy particles that can decay into another baryon. Particles with only two quarks-a-quark-anti-quark pair-are called mesons. These are particles of moderate mass that cannot decay into the more massive\\
baryons.\\
Before continuing, take a moment to view Figure 23.15. In this figure, you can see the strong force reimagined as a color force. The particles interacting in this figure are the proton and neutron, just as they were in Figure 23.6. This reenvisioning of the strong force as an interaction between colored quarks is the critical concept behind quantum chromodynamics.

\begin{figure}[h]
\begin{center}
  \includegraphics[max width=\textwidth]{08dea139-92b4-4d1d-9215-8cc499778a29-25}
\captionsetup{labelformat=empty}
\caption{Figure 23.15 This Feynman diagram shows the interaction between a proton and a neutron, corresponding to the interaction shown in Figure 23.6. This diagram, however, shows the quark and gluon details of the strong nuclear force interaction.}
\end{center}
\end{figure}

\section*{Matter and Antimatter}
Antimatter was first discovered in the form of the positron, the positively charged electron. In 1932, American physicist Carl Anderson discovered the positron in cosmic ray studies. Through a cloud chamber modified to curve the trajectories of cosmic rays, Anderson noticed that the curves of some particles followed that of a negative charge, while others curved like a positive charge. However, the positive curve showed not the mass of a proton but the mass of an\\
electron. This outcome is shown in Figure 23.16 and suggests the existence of a positively charged version of the electron, created by the destruction of solar photons.

\begin{figure}[h]
\begin{center}
  \includegraphics[max width=\textwidth]{08dea139-92b4-4d1d-9215-8cc499778a29-26}
\captionsetup{labelformat=empty}
\caption{Figure 23.16 The image above is from the Fermilab 15 foot bubble chamber and shows the production of an electron and positron (or antielectron) from an incident photon. This event is titled pair production and provides evidence of antimatter, as the two repel each other.}
\end{center}
\end{figure}

Antimatter is considered the opposite of matter. For most antiparticles, this means that they share the same properties as their original particles with the exception of their charge. This is why the positron can be considered a positive electron while the antiproton is considered a negative proton. The idea of an opposite charge for neutral particles (like the neutron) can be confusing, but it makes sense when considered from the quark perspective. Just as the neutron is composed of one up quark and two down quarks (of charge \(+\frac{2}{3}\) and \(-\frac{1}{3}\), respectively), the antineutron is composed of one anti-up quark and two antidown quarks (of charge \(-\frac{2}{3}\) and \(+\frac{1}{3}\), respectively). While the overall charge of the neutron remains the same, its constituent particles do not!

A word about antiparticles: Like regular particles, antiparticles could function just fine on their own. In fact, a universe made up of antimatter may operate just as our own matter-based universe does. However, we do not know fully whether this is the case. The reason for this is annihilation. Annihilation is the process of destruction that occurs when a particle and its antiparticle interact. As soon as two particles (like a positron and an electron) coincide,\\
they convert their masses to energy through the equation \(E=m c^{2}\). This mass-to-energy conversion, which typically results in photon release, happens instantaneously and makes it very difficult for scientists to study antimatter. That said, scientists have had success creating antimatter through high-energy particle collisions. Both antineutrons and antiprotons were created through accelerator experiments in 1956, and an anti-hydrogen atom was even created at CERN in 1995! As referenced in Figure 22.40, the annihilation of antiparticles is currently used in medical studies to determine the location of radioisotopes.

\section*{Teacher Support}
Teacher Support [AL]What would have occurred in the Big Bang if the numbers of matter and antimatter were the same? Students may likely be curious about why there is more matter than antimatter in the universe. Advanced students can be pointed toward research into charge parity violation.

\section*{Completing the Standard Model of the Atom}
The Standard Model of the atom refers to the current scientific view of the fundamental components and interacting forces of matter. The Standard Model (Figure 23.17) shows the six quarks that bind to form all hadrons, the six lepton particles already considered fundamental, the four carrier particles (or gauge bosons) that transmit forces between the leptons and quarks, and the recently added Higgs boson (which will be discussed shortly). This totals 17 fundamental particles, combinations of which are responsible for all known matter in our entire universe! When adding the antiquarks and antileptons, 31 components make up the Standard Model.

\begin{figure}[h]
\begin{center}
  \includegraphics[max width=\textwidth]{08dea139-92b4-4d1d-9215-8cc499778a29-28}
\captionsetup{labelformat=empty}
\caption{Figure 23.17 The Standard Model of elementary particles shows an organized view of all fundamental particles, as currently known: six quarks, six leptons, and four gauge bosons (or carrier particles). The Higgs boson, first observed in 2012, is a new addition to the Standard Model.}
\end{center}
\end{figure}

Figure 23.17 shows all particles within the Standard Model of the atom. Not only does this chart divide all known particles by color-coded group, but it also provides information on particle stability. Note that the color-coding system in this chart is separate from the red, green, and blue color labeling system of quarks. The first three columns represent the three families of matter. The first column, considered Family 1, represents particles that make up normal matter, constructing the protons, neutrons, and electrons that make up the common world. Family 2, represented from the charm quark to the muon neutrino, is comprised of particles that are more massive. The leptons in this group are less stable and more likely to decay. Family 3, represented by the third column, are more massive still and decay more quickly. The order of these families also conveniently represents the order in which these particles were discovered.

\section*{Tips For Success}
Look for trends that exist within the Standard Model. Compare the charge of each particle. Compare the spin. How does mass relate to the model structure? Recognizing each of these trends and asking questions will yield more insight into the organization of particles and the forces that dictate particle relationships. Our understanding of the Standard Model is still young, and the questions you may have in analyzing the Standard Model may be some of the same questions\\
that particle physicists are searching for answers to today!\\
The Standard Model also summarizes the fundamental forces that exist as particles interact. A closer look at the Standard Model, as shown in Figure 23.18, reveals that the arrangement of carrier particles describes these interactions.

\begin{figure}[h]
\begin{center}
  \includegraphics[max width=\textwidth]{08dea139-92b4-4d1d-9215-8cc499778a29-29}
\captionsetup{labelformat=empty}
\caption{Figure 23.18 The revised Standard Model shows the interaction between gauge bosons and other fundamental particles. These interactions are responsible for the fundamental forces, three of which are described through the chart's shaded areas.}
\end{center}
\end{figure}

Each of the shaded areas represents a fundamental force and its constituent particles. The red shaded area shows all particles involved in the strong nuclear force, which we now know is due to quantum chromodynamics. The blue shaded area corresponds to the electromagnetic force, while the green shaded area corresponds to the weak nuclear force, which affects all quarks and leptons. The electromagnetic force and weak nuclear force are considered united by the electroweak force within the Standard Model. Also, because definitive evidence of the graviton is yet to be found, it is not included in the Standard Model.

The Higgs Boson One interesting feature of the Standard Model shown in Figure 23.18 is that, while the gluon and photon have no mass, the Z and W bosons are very massive. What supplies these quickly moving particles with mass and not the gluons and photons? Furthermore, what causes some quarks to have more mass than others?

In the 1960s, British physicist Peter Higgs and others speculated that the W and Z bosons were actually just as massless as the gluon and photon. However, as the W and Z bosons traveled from one particle to another, they were slowed down by the presence of a Higgs field, much like a fish swimming through water. The thinking was that the existence of the Higgs field would slow down the bosons, causing them to decrease in energy and thereby transfer this energy to mass. Under this theory, all particles pass through the Higgs field, which exists throughout the universe. The gluon and photon travel through this field as well but are able to do so unaffected.

The presence of a force from the Higgs field suggests the existence of its own carrier particle, the Higgs boson. This theorized boson interacts with all particles but gluons and photons, transferring force from the Higgs field. Particles with large mass (like the top quark) are more likely to receive force from the Higgs boson.

While it is difficult to examine a field, it is somewhat simpler to find evidence of its carrier. On July 4, 2012, two groups of scientists at the LHC independently confirmed the existence of a Higgs-like particle. By examining trillions of protonproton collisions at energies of 7 to 8 TeV , LHC scientists were able to determine the constituent particles that created the protons. In this data, scientists found a particle with similar mass, spin, parity, and interactions with other particles that matched the Higgs boson predicted decades prior. On March 13, 2013, the existence of the Higgs boson was tentatively confirmed by CERN. Peter Higgs and Francois Englert received the Nobel Prize in 2013 for the "theoretical discovery of a mechanism that contributes to our understanding of the origin and mass of subatomic particles."

\section*{Teacher Support}
Teacher Support Reinforce to students that the identification of the Higgs particle was similar to the process used to find evidence of many fundamental particles. Contrast this process with the indirect process used to locate the individual quark.

\section*{Work In Physics}
Particle Physicist If you have an innate desire to unravel life's great mysteries and further understand the nature of the physical world, a career in particle physics may be for you!

Particle physicists have played a critical role in much of society's technological progress. From lasers to computers, televisions to space missions, splitting the atom to understanding the DNA molecule to MRIs and PET scans, much of our modern society is based on the work done by particle physicists.

While many particle physicists focus on specialized tasks in the fields of astronomy and medicine, the main goal of particle physics is to further scientists' understanding of the Standard Model. This may mean work in government, industry, or academics. Within the government, jobs in particle physics can be found within the National Institute for Standards and Technology, Department of Energy, NASA, and Department of Defense. Both the electronics and computer industries rely on the expertise of particle physicists. College teaching and research positions can also be potential career opportunities for particle physicists, though they often require some postgraduate work as a prerequisite. In addition, many particle physicists are employed to work on high-energy col-\\
liders. Domestic collider labs include the Brookhaven National Laboratory in New York, the Fermi National Accelerator Laboratory near Chicago, and the SLAC National Accelerator Laboratory operated by Stanford University. For those who like to travel, work at international collider labs can be found at the CERN facility in Switzerland in addition to institutes like the Budker Institute of Nuclear Physics in Russia, DESY in Germany, and KEK in Japan.

Shirley Jackson became the first African American woman to earn a Ph.D. from MIT back in 1973, and she went on to lead a highly successful career in the field of particle physics. Like Dr. Jackson, successful students of particle physics grow up with a strong curiosity in the world around them and a drive to continually learn more. If you are interested in exploring a career in particle physics, work to achieve good grades and SAT scores, and find time to read popular books on physics topics that interest you. While some math may be challenging, recognize that this is only a tool of physics and should not be considered prohibitive to the field. High-level work in particle physics often requires a Ph.D.; however, it is possible to find work with a master's degree. Additionally, jobs in industry and teaching can be achieved with solely an undergraduate degree.

What is the primary goal of all work in particle physics?\\
a. The primary goal is to further our understanding of the Standard Model.\\
b. The primary goal is to further our understanding of Rutherford's model.\\
c. The primary goal is to further our understanding of Bohr's model.\\
d. The primary goal is to further our understanding of Thomson's model.

\section*{Check Your Understanding}
\section*{Teacher Support}
Teacher Support Use these questions to assess student achievement of the section's learning objectives. If students are struggling with a specific objective, these questions will help identify which and direct students to the relevant content.\\
8.

In what particle were quarks originally discovered?\\
a. the electron\\
b. the neutron\\
c. the proton\\
d. the photon\\
9.

Why was the existence of the charm quark speculated, even though no direct evidence of it existed?\\
a. The existence of the charm quark was symmetrical with up and down quarks. Additionally, there were two known leptons at the time and only\\
two quarks.\\
b. The strange particle lacked the symmetry that existed with the up and down quarks. Additionally, there were four known leptons at the time and only three quarks.\\
c. The bottom particle lacked the symmetry that existed with the up and down quarks. Additionally, there were two known leptons at the time and only two quarks.\\
d. The existence of charm quarks was symmetrical with up and down quarks. Additionally, there were four known leptons at the time and only three quarks.\\
10.

What type of particle is the electron?\\
a. The electron is a lepton.\\
b. The electron is a hadron.\\
c. The electron is a baryon.\\
d. The electron is an antibaryon.\\
11.

How do the number of fundamental particles differ between hadrons and leptons?\\
a. Hadrons are constructed of at least three fundamental quark particles, while leptons are fundamental particles.\\
b. Hadrons are constructed of at least three fundamental quark particles, while leptons are constructed of two fundamental particles.\\
c. Hadrons are constructed of at least two fundamental quark particles, while leptons are constructed of three fundamental particles.\\
d. Hadrons are constructed of at least two fundamental quark particles, while leptons are fundamental particles.\\
12.

Does antimatter exist?\\
a. no\\
b. yes\\
13.

How does the deconstruction of a photon into an electron and a positron uphold the principles of mass and charge conservation?\\
a. The sum of the masses of an electron and a positron is equal to the mass of the photon before pair production. The sum of the charges on an electron and a positron is equal to the zero charge of the photon.\\
b. The sum of the masses of an electron and a positron is equal to the mass of the photon before pair production. The sum of the same charges on an electron and a positron is equal to the charge on a photon.\\
c. During the particle production the total energy of the photon is converted to the mass of an electron and a positron. The sum of the opposite charges on the electron and positron is equal to the zero charge of the photon.\\
d. During particle production, the total energy of the photon is converted to the mass of an electron and a positron. The sum of the same charges on an electron and a positron is equal to the charge on a photon.\\
14.

How many fundamental particles exist in the Standard Model, including the Higgs boson and the graviton (not yet observed)?\\
a. 12\\
b. 15\\
c. 13\\
d. 19\\
15.

Why do gluons interact only with particles in the first two rows of the Standard Model?\\
a. The leptons in the third and fourth rows do not have mass, but the gluons can interact between the quarks through gravity only.\\
b. The leptons in the third and fourth rows do not have color, but the gluons can interact between quarks through color interactions only.\\
c. The leptons in the third and fourth rows do not have spin, but the gluons can interact between quarks through spin interactions only.\\
d. The leptons in the third and fourth rows do not have charge, but the gluons can interact between quarks through charge interactions only.\\
16.

What fundamental property is provided by particle interaction with the Higgs boson?\\
a. charge\\
b. mass\\
c. spin\\
d. color\\
17.

Considering the Higgs field, what differentiates more massive particles from less massive particles?\\
a. More massive particles interact more with the Higgs field than the less massive particles.\\
b. More massive particles interact less with the Higgs field than the less massive particles.\\
18.

What particles were launched into the proton during the original discovery of the quark?\\
a. bosons\\
b. electrons\\
c. neutrons\\
d. photons

\subsection*{23.3 The Uni cation of Forces}
\section*{Section Learning Objectives}
By the end of the section, you will be able to do the following:

\begin{itemize}
  \item Define a grand unified theory and its importance
  \item Explain the evolution of the four fundamental forces from the Big Bang onward
  \item Explain how grand unification theories can be tested
\end{itemize}

\section*{Teacher Support}
Teacher Support The learning objectives in this section will help your students master the following standards:

\begin{itemize}
  \item (5) Science concepts. The student knows the nature of forces in the physical world. The student is expected to:
  \item (H) describe evidence for and effects of the strong and weak nuclear forces in nature.
\end{itemize}

\section*{Section Key Terms}
\section*{Understanding the Grand Unified Theory}
Present quests to show that the four basic forces are different manifestations of a single unified force that follow a long tradition. In the nineteenth century, the distinct electric and magnetic forces were shown to be intimately connected and are now collectively called the electromagnetic force. More recently, the weak nuclear force was united with the electromagnetic force. As shown in Figure 23.19, carrier particles transmit three of the four fundamental forces in very similar ways. With these considerations in mind, it is natural to suggest that a theory may be constructed in which the strong nuclear, weak nuclear, and electromagnetic forces are all unified. The search for a correct theory linking the forces, called the Grand Unified Theory (GUT), is explored in this section.

In the 1960s, the electroweak theory was developed by Steven Weinberg, Sheldon Glashow, and Abdus Salam. This theory proposed that the electromagnetic and weak nuclear forces are identical at sufficiently high energies. At lower energies, like those in our present-day universe, the two forces remain united but manifest themselves in different ways. One of the main consequences of the electroweak theory was the prediction of three short-range carrier particles, now known as the \(W^{+}, W^{-}\), and \(\mathrm{Z}^{0}\) bosons. Not only were three particles predicted, but the\\
mass of each \(\mathrm{W}^{+}\)and \(\mathrm{W}^{-}\)boson was predicted to be \(81 \mathrm{GeV} / c^{2}\), and that of the \(\mathrm{Z}^{0}\) boson was predicted to be \(90 \mathrm{GeV} / c^{2}\). In 1983, these carrier particles were observed at CERN with the predicted characteristics, including masses having those predicted values as given in Figure 23.17.

How can forces be unified? They are definitely distinct under most circumstances. For example, they are carried by different particles and have greatly different strengths. But experiments show that at extremely short distances and at extremely high energies, the strengths of the forces begin to become more similar, as seen in Figure 23.20.

\begin{figure}[h]
\begin{center}
  \includegraphics[max width=\textwidth]{08dea139-92b4-4d1d-9215-8cc499778a29-36}
\captionsetup{labelformat=empty}
\caption{Figure 23.19 The exchange of a virtual \(\mathrm{Z}^{0}\) particle (boson) carries the weak nuclear force between an electron and a neutrino in this Feynman diagram. This diagram is similar to the diagrams in Figure 23.6 and Figure 23.5 for the electromagnetic and strong nuclear forces.}
\end{center}
\end{figure}

As discussed earlier, the short ranges and large masses of the weak carrier bosons require correspondingly high energies to create them. Thus, the energy scale on the horizontal axis of Figure 23.20 also corresponds to shorter and shorter distances (going from left to right), with 100 GeV corresponding to approximately \(10^{-18} \mathrm{~m}\), for example. At that distance, the strengths of the electromagnetic and weak nuclear forces are the same. To test this, energies of about 100 GeV are put into the system. When this occurs, the \(\mathrm{W}^{+}, \mathrm{W}^{-}\), and \(\mathrm{Z}^{0}\) carrier particles are created and released. At those and higher energies, the masses of the carrier particles become less and less relevant, and the \(\mathrm{Z}^{0}\) boson in particular resembles the massless, chargeless photon. As further energy is added, the \(\mathrm{W}^{+}, \mathrm{W}^{-}\), and \(\mathrm{Z}^{0}\) particles are further transformed into massless carrier particles even more similar to photons and gluons.

\section*{Teacher Support}
Teacher Support [BL][OL][AL]Given the discussion in the prior paragraph, what should scientists expect to see as they probe the Grand Unified Theory? What characteristics should the gluon have at this energy?\\
\includegraphics[max width=\textwidth, center]{08dea139-92b4-4d1d-9215-8cc499778a29-37}

Figure 23.20 The relative strengths of the four basic forces vary with distance, and, hence, energy is needed to probe small distances. At ordinary energies (a few eV or less), the forces differ greatly. However, at energies available in accelerators, the weak nuclear and electromagnetic (EM) forces become unified. Unfortunately, the energies at which the strong nuclear and electroweak forces become the same are unreachable in any conceivable accelerator. The universe may provide a laboratory, and nature may show effects at ordinary energies that give us clues about the validity of this graph.

The extremely short distances and high energies at which the electroweak force becomes identical with the strong nuclear force are not reachable with any conceivable human-built accelerator. At energies of about \(10^{14} \mathrm{GeV}(16,000 \mathrm{~J}\) per particle), distances of about 10 to 30 m can be probed. Such energies are needed to test the theory directly, but these are about \(10^{10}\) times higher than the maximum energy associated with the LHC, and the distances are about 10 to 12 smaller than any structure we have direct knowledge of. This would be the realm of various GUTs, of which there are many, since there is no constraining evidence at these energies and distances. Past experience has shown that anytime you probe so many orders of magnitude further, you find the unexpected.

While direct evidence of a GUT is not presently possible, that does not rule out the ability to assess a GUT through an indirect process. Current GUTs require various other events as a consequence of their theory. Some GUTs require the existence of magnetic monopoles, very massive individual north- and south-pole particles, which have not yet been proven to exist, while others require the use of extra dimensions. However, not all theories result in the same consequences. For example, disproving the existence of magnetic monopoles will not disprove all GUTs. Much of the science we accept in our everyday lives is based on different models, each with their own strengths and limitations. Although a particular model may have drawbacks, that does not necessarily mean that it should be discounted completely.

One consequence of GUTs that can theoretically be assessed is proton decay. Multiple current GUTs hypothesize that the stable proton should actually decay at a lifetime of \(10^{31}\) years. While this time is incredibly large (keep in mind that the age of the universe is less than 14 billion years), scientists at the SuperKamiokande in Japan have used a 50,000 -ton tank of water to search for its existence. The decay of a single proton in the Super-Kamiokande tank would be observed by a detector, thereby providing support for the predicting GUT model. However, as of 2014, 17 years into the experiment, decay is yet to be found. This time span equates to a minimum limit on proton life of \(5.9 \times 10^{33}\) years. While this result certainly does not support many grand unifying theories, an acceptable model may still exist.

\section*{Tips For Success}
The Super-Kamiokande experiment is a clever use of proportional reasoning. Because it is not feasible to test for \(10^{31}\) years in order for a single proton to decay, scientists chose instead to manipulate the proton-time ratio. If one proton decays in \(10^{31}\) years, then in one year \(10^{-31}\) protons will decay. With this in mind, if scientists wanted to test the proton decay theory in one year, they would need \(10^{31}\) protons. While this is also unfeasible, the use of a 50,000 -ton tank of water helps to bring both the wait time and proton number to within reason.

\section*{The Standard Model and the Big Bang}
Nature is full of examples where the macroscopic and microscopic worlds intertwine. Newton realized that the nature of gravity on Earth that pulls an apple to the ground could explain the motion of the moon and planets so much farther away. Decays of tiny nuclei explain the hot interior of the Earth. Fusion of nuclei likewise explains the energy of stars. Today, the patterns in particle physics seem to be explaining the evolution and character of the universe. And the nature of the universe has implications for unexplored regions of particle physics.

In 1929, Edwin Hubble observed that all but the closest galaxies surrounding our own had a red shift in their hydrogen spectra that was proportional to their distance from us. Applying the Doppler Effect, Hubble recognized that this meant that all galaxies were receding from our own, with those farther away receding even faster. Knowing that our place in the universe was no more unique than any other, the implication was clear: The space within the universe itself was expanding. Just like pen marks on an expanding balloon, everything in the universe was accelerating away from everything else.

\section*{Teacher Support}
Teacher Support [BL][OL][AL]Modeling the Big Bang through an expanding balloon is a good demonstration to show that our perspective on the universe\\
is not unique, that all places are expanding equally from all others, and to provide basis for the Big Bang. This will also be useful to help students understand that while parts of the universe may be expanding from us at speeds greater than the speed of light, they are not violating relativity.

Figure 23.21 shows how the recession of galaxies looks like the remnants of a gigantic explosion, the famous Big Bang. Extrapolating backward in time, the Big Bang would have occurred between 13 and 15 billion years ago, when all matter would have been at a single point. From this, questions instantly arise. What caused the explosion? What happened before the Big Bang? Was there a before, or did time start then? For our purposes, the biggest question relating to the Big Bang is this: How does the Big Bang relate to the unification of the fundamental forces?\\
\includegraphics[max width=\textwidth, center]{08dea139-92b4-4d1d-9215-8cc499778a29-39}

Figure 23.21 Galaxies are flying apart from one another, with the more distant ones moving faster, as if a primordial explosion expelled the matter from which they formed. The most distant known galaxies move nearly at the speed of light relative to us.

To fully understand the conditions of the very early universe, recognize that as the universe contracts to the size of the Big Bang, changes will occur. The density and temperature of the universe will increase dramatically. As particles become closer together, they will become too close to exist as we know them. The high energies will create other, more unusual particles to exist in greater abundance. Knowing this, let's move forward from the start of the universe, beginning with the Big Bang, as illustrated in Figure 23.22.

Figure 23.22 The evolution of the universe from the Big Bang onward (from left to right) is intimately tied to the laws of physics, especially those of particle physics at the earliest stages. Theories of the unification of forces at high energies may be verified by their shaping of the universe and its evolution.

\section*{Teacher Support}
Teacher Support The discussion of the universe evolution in this text stops at the Quark Era. Have students consider why various other eras took place after the Quark Era. How did temperature, density, and pressure play a role in further evolution?

The Planck Epoch \(\left(0 \rightarrow 10^{-43} s\right)\)-Though scientists are unable to model the conditions of the Planck Epoch in the laboratory, speculation is that at this time compressed energy was great enough to reach the immense \(10^{19} \mathrm{GeV}\) necessary to unify gravity with all other forces. As a result, modern cosmology suggests\\
that all four forces would have existed as one force, a hypothetical superforce as suggested by the Theory of Everything.

The Grand Unification Epoch \(\left(10^{-43} \rightarrow 10^{-36} s\right)\)-As the universe expands, the temperatures necessary to maintain the superforce decrease. As a result, gravity separates, leaving the electroweak and strong nuclear forces together. At this time, the electromagnetic, weak, and strong forces are identical, matching the conditions requested in the Grand Unification Theory.

The Inflationary Epoch \(\left(10^{-36} \rightarrow 10^{-32} s\right)\)-The separation of the strong nuclear force from the electroweak force during this time is thought to have been responsible for the massive inflation of the universe. Corresponding to the steep diagonal line on the left side of Figure 23.22, the universe may have expanded by a factor of \(10^{50}\) or more in size. In fact, the expansion was so great during this time that it actually occurred faster than the speed of light! Unfortunately, there is little hope that we may be able to test the inflationary scenario directly since it occurs at energies near \(10^{14} \mathrm{GeV}\), vastly greater than the limits of modern accelerators.

The Electroweak Epoch \(\left(10^{-32} \rightarrow 10^{-11} s\right)\)-Now separated from both gravity and the strong nuclear force, the electroweak force exists as a singular force during this time period. As stated earlier, scientists are able to create the energies at this stage in the universe's expansion, needing only 100 GeV , as shown in Figure 23.20. W and Z bosons, as well as the Higgs boson, are released during this time.

The Quark Era \(\left(10^{-11} \rightarrow 10^{-6} s\right)\)-During the Quark Era, the universe has expanded and temperatures have decreased to the point at which all four fundamental forces have separated. Additionally, quarks began to take form as energies decreased.

As the universe expanded, further eras took place, allowing for the existence of hadrons, leptons, and photons, the fundamental particles of the standard model. Eventually, in nucleosynthesis, nuclei would be able to form, and the basic building blocks of atomic matter could take place. Using particle accelerators, we are very much working backwards in an attempt to understand the universe. It is encouraging to see that the macroscopic conditions of the Big Bang align nicely with our submicroscopic particle theory.

\section*{Check Your Understanding}
\section*{Teacher Support}
Teacher Support Use these questions to assess student achievement of the section's learning objectives. If students are struggling with a specific objective, these questions will help identify which and direct students to the relevant content.\\
19.

Is there one grand unified theory or multiple grand unifying theories?\\
a. one grand unifying theory\\
b. multiple grand unifying theories\\
20.

In what manner is \(\mathrm{E}=\mathrm{mc}^{\wedge}\{2\}\) considered a precursor to the Grand Unified Theory?\\
a. The grand unified theory seeks to relate the electroweak and strong nuclear forces to one another just as \(\mathrm{E}=\mathrm{mc}^{\wedge}\{2\}\) related energy and mass.\\
b. The grand unified theory seeks to relate the electroweak force and mass to one another just as \(\mathrm{E}=\mathrm{mc}^{\wedge}\{2\}\) related energy and mass.\\
c. The grand unified theory seeks to relate the mass and strong nuclear forces to one another just as \(\mathrm{E}=\mathrm{mc}^{\wedge}\{2\}\) related energy and mass.\\
d. The grand unified theory seeks to relate gravity and strong nuclear force to one another, just as \(\mathrm{E}=\mathrm{mc}^{\wedge}\{2\}\) related energy and mass.\\
21.

List the following eras in order of occurrence from the Big Bang: Electroweak Epoch, Grand Unification Epoch, Inflationary Epoch, Planck Epoch, Quark Era.\\
a. Quark Era, Grand Unification Epoch, Inflationary Epoch, Electroweak Epoch, Planck Epoch\\
b. Planck Epoch, Inflationary Epoch, Grand Unification Epoch, Electroweak Epoch, Quark Era\\
c. Planck Epoch, Electroweak Epoch, Grand Unification Epoch, Inflationary Epoch, Quark Era\\
d. Planck Epoch, Grand Unification Epoch, Inflationary Epoch, Electroweak Epoch, Quark Era\\
22.

How did the temperature of the universe change as it expanded?\\
a. The temperature of the universe increased.\\
b. The temperature of the universe decreased.\\
c. The temperature of the universe first decreased and then increased.\\
d. The temperature of the universe first increased and then decreased.\\
23.

Under current conditions, is it possible for scientists to use particle accelerators to verify the Grand Unified Theory?\\
a. No, there is not enough energy.\\
b. Yes, there is enough energy.\\
24.

Why are particles and antiparticles made to collide as shown in this image?\\
\includegraphics[max width=\textwidth, center]{08dea139-92b4-4d1d-9215-8cc499778a29-43}\\
a. Particles and antiparticles have the same mass.\\
b. Particles and antiparticles have different mass.\\
c. Particles and antiparticles have the same charge.\\
d. Particles and antiparticles have opposite charges.\\
25.

The existence of what particles were predicted as a consequence of the electroweak theory?\\
a. fermions\\
b. Higgs bosons\\
c. leptons\\
d. \(\mathrm{W}^{+}, \mathrm{W}^{-}\), and \(\mathrm{Z}^{0}\) bosons

\section*{Ke Terms}
\(\mathbf{W}^{+}\)boson positive carrier particle of the weak nuclear force\\
\(\mathbf{W}^{-}\)boson negative carrier particle of the weak nuclear force\\
\(\mathbf{Z}^{0}\) boson neutral carrier particle of the weak nuclear force\\
annihilation the process of destruction that occurs when a particle and antiparticle interact\\
antimatter matter constructed of antiparticles; antimatter shares most of the same properties of regular matter, with charge being the only difference between many particles and their antiparticle analogues\\
baryon hadrons that always decay to another baryon\\
Big Bang a gigantic explosion that threw out matter a few billion years ago\\
bottom quark a quark flavor\\
carrier particle a virtual particle exchanged in the transmission of a fundamental force\\
charmed quark a quark flavor, which is the counterpart of the strange quark\\
colliding beam head-on collisions between particles moving in opposite directions\\
color a property of quarks the relates to their interactions through the strong force\\
cyclotron accelerator that uses fixed-frequency alternating electric fields and fixed magnets to accelerate particles in a circular spiral path\\
down quark the second lightest of all quarks\\
Electroweak Epoch the stage before \(10^{-11}\) back to \(10^{-34}\) seconds after the Big Bang\\
electroweak theory theory showing connections between EM and weak forces\\
Feynman diagram a graph of time versus position that describes the exchange of virtual particles between subatomic particles\\
flavor quark type\\
gluons exchange particles of the nuclear strong force\\
Grand Unification Epoch the time period from \(10^{-43}\) to \(10^{-34}\) seconds after the Big Bang, when Grand Unification Theory, in which all forces except gravity are identical, governed the universe

Grand Unified Theory theory that shows unification of the strong and electroweak forces\\
graviton hypothesized particle exchanged between two particles of mass, transmitting the gravitational force between them\\
hadron particles composed of quarks that feel the strong and weak nuclear force

Higgs boson a massive particle that provides mass to the weak bosons and provides validity to the theory that carrier particles are identical under certain circumstances

Higgs field the field through which all fundamental particles travel that provides them varying mass through the transport of the Higgs boson

Inflationary Epoch the rapid expansion of the universe by an incredible factor of \(10^{-50}\) for the brief time from \(10^{-35}\) to about \(10^{-32}\) seconds\\
lepton fundamental particles that do not feel the nuclear strong force\\
meson hadrons that can decay to leptons and leave no hadrons\\
pair production the creation of a particle and antiparticle, commonly an electron and positron, due to the annihilation of a photon\\
particle physics the study of and the quest for those truly fundamental particles having no substructure\\
pion particle exchanged between nucleons, transmitting the strong nuclear force between them

Planck Epoch the earliest era of the universe, before \(10^{-43}\) seconds after the Big Bang\\
positron a particle of antimatter that has the properties of a positively charged electron\\
quantum chromodynamics the theory of color interaction between quarks that leads to understanding of the nuclear strong force\\
quantum electrodynamics the theory of electromagnetism on the particle scale\\
quark an elementary particle and fundamental constituent of matter that is a substructure of hadrons

Quark Era the time period from \(10^{-11}\) to \(10^{-6}\) seconds at which all four fundamental forces are separated and quarks begin to exit

Standard Model an organization of fundamental particles and forces that is a result of quantum chromodynamics and electroweak theory\\
strange quark the third lightest of all quarks\\
superforce the unification of all four fundamental forces into one force\\
synchrotron a version of a cyclotron in which the frequency of the alternating voltage and the magnetic field strength are increased as the beam particles are accelerated

Theory of Everything the theory that shows unification of all four fundamental forces\\
top quark a quark flavor\\
up quark the lightest of all quarks\\
weak nuclear force fundamental force responsible for particle decay

\section*{Section Summar}
\subsection*{23.1 The Four Fundamental Forces}
\begin{itemize}
  \item The four fundamental forces are gravity, the electromagnetic force, the weak nuclear force, and the strong nuclear force.
  \item A variety of particle accelerators have been used to explore the nature of subatomic particles and to test predictions of particle theories.
\end{itemize}

\subsection*{23.2 Quarks}
\begin{itemize}
  \item There are three types of fundamental particles-leptons, quarks, and carrier particles.
  \item Quarks come in six flavors and three colors and occur only in combinations that produce white.
  \item Hadrons are thought to be composed of quarks, with baryons having three quarks and mesons having a quark and an antiquark.
  \item Known particles can be divided into three major groups-leptons, hadrons, and carrier particles (gauge bosons).
  \item All particles of matter have an antimatter counterpart that has the opposite charge and certain other quantum numbers. These matter-antimatter pairs are otherwise very similar but will annihilate when brought together.
  \item The strong force is carried by eight proposed particles called gluons, which are intimately connected to a quantum number called color-their governing theory is thus called quantum chromodynamics (QCD). Taken together, QCD and the electroweak theory are widely accepted as the Standard Model of particle physics.
\end{itemize}

\subsection*{23.3 The Unification of Forces}
\begin{itemize}
  \item Attempts to show unification of the four forces are called Grand Unified Theories (GUTs) and have been partially successful, with connections proven between EM and weak forces in electroweak theory.
  \item Unification of the strong force is expected at such high energies that it cannot be directly tested, but it may have observable consequences in the as-yet-unobserved decay of the proton. Although unification of forces is generally anticipated, much remains to be done to prove its validity.
\end{itemize}

\end{document}