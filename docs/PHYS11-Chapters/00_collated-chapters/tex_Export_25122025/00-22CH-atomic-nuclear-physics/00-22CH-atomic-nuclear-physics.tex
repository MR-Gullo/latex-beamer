\documentclass[10pt]{article}
\usepackage[utf8]{inputenc}
\usepackage[T1]{fontenc}
\usepackage{graphicx}
\usepackage[export]{adjustbox}
\graphicspath{ {./images/} }
\usepackage{caption}
\usepackage{amsmath}
\usepackage{amsfonts}
\usepackage{amssymb}
\usepackage[version=4]{mhchem}
\usepackage{stmaryrd}

\DeclareUnicodeCharacter{2192}{\ifmmode\rightarrow\else{$\rightarrow$}\fi}

\begin{document}
\captionsetup{singlelinecheck=false}
\begin{figure}[h]
\begin{center}
  \includegraphics[max width=\textwidth]{6c6af082-56ef-4ff6-b696-4e9ed0d1965b-01}
\captionsetup{labelformat=empty}
\caption{Figure 22.1 Individual carbon atoms are visible in this image of a carbon nanotube made by a scanning tunneling electron microscope. (credit: Taner Yildirim, National Institute of Standards and Technology, Wikimedia Commons)}
\end{center}
\end{figure}

\section*{Chapter Outline}
22.1 The Structure of the Atom\\
22.2 Nuclear Forces and Radioactivity\\
22.3 Half Life and Radiometric Dating\\
22.4 Nuclear Fission and Fusion\\
22.5 Medical Applications of Radioactivity: Diagnostic Imaging and Radiation

\section*{Introduction}
From childhood on, we learn that atoms are a substructure of all things around us, from the air we breathe to the autumn leaves that blanket a forest trail. Invisible to the eye, the atoms have properties that are used to explain many phenomena-a theme found throughout this text. In this chapter, we discuss the discovery of atoms and their own substructures. We will then learn about the forces that keep them together and the tremendous energy they release when we break them apart. Finally, we will see how the knowledge and manipulation of atoms allows us to better understand geology, biology, and the world around us.

\section*{Teacher Support}
Teacher Support Have students briefly brainstorm a list of things they can recall about atoms. Encourage them to reflect upon any knowledge they may have from a prior chemistry of physical science class. In addition, see if students can describe the smallest structure that may relate to geology and biology. That\\
will help them to think microscopically in contexts beyond atomic structure and may assist with atomic applications.

\subsection*{22.1 The Structure of the Atom}
\section*{Section Learning Objectives}
By the end of this section, you will be able to do the following:

\begin{itemize}
  \item Describe Rutherford's experiment and his model of the atom
  \item Describe emission and absorption spectra of atoms
  \item Describe the Bohr model of the atom
  \item Calculate the energy of electrons when they change energy levels
  \item Calculate the frequency and wavelength of emitted photons when electrons change energy levels
  \item Describe the quantum model of the atom
\end{itemize}

\section*{Teacher Support}
Teacher Support The learning objectives in this section will help your students master the following standards:

\begin{itemize}
  \item (8) Science concepts. The student knows simple examples of atomic, nuclear, and quantum phenomena. The student is expected to:
  \item (B) compare and explain the emission spectra produced by various atoms.
\end{itemize}

In addition, the High School Physics Laboratory Manual addresses content in this section in the lab titled: The Atom, as well as the following standards:

\begin{itemize}
  \item (8) Science concepts. The student knows simple examples of atomic, nuclear, and quantum phenomena. The student is expected to:
  \item (B) compare and explain the emission spectra produced by various atoms.
\end{itemize}

\section*{Section Key Terms}
How do we know that atoms are really there if we cannot see them with our own eyes? While often taken for granted, our knowledge of the existence and structure of atoms is the result of centuries of contemplation and experimentation. The earliest known speculation on the atom dates back to the fifth century B.C., when Greek philosophers Leucippus and Democritus contemplated whether a substance could be divided without limit into ever smaller pieces. Since then, scientists such as John Dalton (1766-1844), Amadeo Avogadro (1776-1856), and

Dmitri Mendeleev (1834-1907) helped to discover the properties of that fundamental structure of matter. While much could be written about any number of important scientific philosophers, this section will focus on the role played by Ernest Rutherford (1871-1937). Though his understanding of our most elemental matter is rooted in the success of countless prior investigations, his surprising discovery about the interior of the atom is most fundamental in explaining so many well-known phenomena.

\section*{Teacher Support}
Teacher Support [AL]For students that are exceptionally curious, suggest studying the writings of Democritus and determining how much of his atomic theory holds true today.

\section*{Rutherford's Experiment}
In the early 1900's, the plum pudding model was the accepted model of the atom. Proposed in 1904 by J. J. Thomson, the model suggested that the atom was a spherical ball of positive charge, with negatively charged electrons scattered evenly throughout. In that model, the positive charges made up the pudding, while the electrons acted as isolated plums. During its short life, the model could be used to explain why most particles were neutral, although with an unbalanced number of plums, electrically charged atoms could exist.

\section*{Teacher Support}
Teacher Support The plum pudding model is an outdated term many students will not understand. A chocolate chip ice cream model may work better to describe Thomson's model to students. More advanced students may be able to describe other models on their own.

When Ernest Rutherford began his gold foil experiment in 1909, it is unlikely that anyone would have expected that the plum pudding model would be challenged. However, using a radioactive source, a thin sheet of gold foil, and a phosphorescent screen, Rutherford would uncover something so great that he would later call it "the most incredible event that has ever happened to me in my life"[James, L. K. (1993). Nobel Laureates in Chemistry, 1901-1992. Washington, DC: American Chemical Society.]

The experiment that Rutherford designed is shown in Figure 22.2. As you can see in, a radioactive source was placed in a lead container with a hole in one side to produce a beam of positively charged helium particles, called alpha particles. Then, a thin gold foil sheet was placed in the beam. When the high-energy alpha particles passed through the gold foil, they were scattered. The scattering was observed from the bright spots they produced when they struck the phosphor screen.

\section*{Teacher Support}
Teacher Support Take time with explaining this concept. Students may need to be reminded about phosphorescence. Additionally, radioactivity has not yet been covered in this course. At this point, it may make the most sense to state that source in the lead container "shoots small particles at high speeds."

\begin{figure}[h]
\begin{center}
  \includegraphics[max width=\textwidth]{6c6af082-56ef-4ff6-b696-4e9ed0d1965b-05}
\captionsetup{labelformat=empty}
\caption{Figure 22.2 Rutherford's experiment gave direct evidence for the size and mass of the nucleus by scattering alpha particles from a thin gold foil. The scattering of particles suggests that the gold nuclei are very small and contain nearly all of the gold atom's mass. Particularly significant in showing the size of the nucleus are alpha particles that scatter to very large angles, much like a soccer ball bouncing off a goalie's head.}
\end{center}
\end{figure}

The expectation of the plum pudding model was that the high-energy alpha particles would be scattered only slightly by the presence of the gold sheet. Because the energy of the alpha particles was much higher than those typically associated with atoms, the alpha particles should have passed through the thin foil much like a supersonic bowling ball would crash through a few dozen rows of bowling pins. Any deflection was expected to be minor, and due primarily to the electrostatic Coulomb force between the alpha particles and the foil's interior electric charges.

However, the true result was nothing of the sort. While the majority of alpha particles passed through the foil unobstructed, Rutherford and his collaborators Hans Geiger and Ernest Marsden found that alpha particles occasionally were scattered to large angles, and some even came back in the direction from which they came! The result, called Rutherford scattering, implied that the gold nuclei were actually very small when compared with the size of the gold atom. As shown in Figure 22.3, the dense nucleus is surrounded by mostly empty space of the atom, an idea verified by the fact that only 1 in 8,000 particles was scattered backward.

\section*{Teacher Support}
Teacher Support [BL][OL]Students should recognize two distinct results\\
from the gold foil experiment. (1) The gold nucleus is very dense. (2) The atom is mostly empty space. Have students identify what evidence from the gold foil experiment supports those two statements.\\[0pt]
[AL]Have the students determine whether backscattering would occur with the plum pudding model.

You may also want the students to try an investigation to explore the type of calculation required to determine the diameter of the nucleus from the number of collisions; it is available at this website.

\begin{figure}[h]
\begin{center}
  \includegraphics[max width=\textwidth]{6c6af082-56ef-4ff6-b696-4e9ed0d1965b-06}
\captionsetup{labelformat=empty}
\caption{Figure 22.3 An expanded view of the atoms in the gold foil in Rutherford's experiment. Circles represent the atoms that are about \(10^{-10} \mathrm{~m}\) in diameter, while the dots represent the nuclei that are about \(10^{-15} \mathrm{~m}\) in diameter. To be visible, the dots are much larger than scale-if the nuclei were actually the size of the dots, each atom would have a diameter of about five meters! Most alpha particles crash through but are relatively unaffected because of their high energy and the electron's small mass. Some, however, strike a nucleus and are scattered straight back. A detailed analysis of their interaction gives the size and mass of the nucleus.}
\end{center}
\end{figure}

Although the results of the experiment were published by his colleagues in 1909, it took Rutherford two years to convince himself of their meaning. Rutherford later wrote: "It was almost as incredible as if you fired a 15 -inch shell at a piece of tissue paper and it came back and hit you. On consideration, I realized that this scattering backwards ... [meant] ... the greatest part of the mass of the atom was concentrated in a tiny nucleus." In 1911, Rutherford published his analysis together with a proposed model of the atom, which was in part based on Geiger's work from the previous year. As a result of the paper, the size of the nucleus was determined to be about \(10^{-15} \mathrm{~m}\), or 100,000 times smaller than the atom. That implies a huge density, on the order of \(10^{15} \mathrm{~g} / \mathrm{cm}^{3}\), much greater than any macroscopic matter.

Based on the size and mass of the nucleus revealed by his experiment, as well as the mass of electrons, Rutherford proposed the planetary model of the atom. The planetary model of the atom pictures low-mass electrons orbiting a largemass nucleus. The sizes of the electron orbits are large compared with the size of the nucleus, and most of the atom is a vacuum. The model is analogous to how low-mass planets in our solar system orbit the large-mass Sun. In the atom, the attractive Coulomb force is analogous to gravitation in the planetary system (see Figure 22.4).

\begin{figure}[h]
\begin{center}
  \includegraphics[max width=\textwidth]{6c6af082-56ef-4ff6-b696-4e9ed0d1965b-07}
\captionsetup{labelformat=empty}
\caption{Figure 22.4 Rutherford's planetary model of the atom incorporates the characteristics of the nucleus, electrons, and the size of the atom. The model was the first to recognize the structure of atoms, in which low-mass electrons orbit a very small, massive nucleus in orbits much larger than the nucleus. The atom is mostly empty and is analogous to our planetary system.}
\end{center}
\end{figure}

\section*{Virtual Physics}
Rutherford Scattering Click to view content\\
How did Rutherford figure out the structure of the atom without being able to see it? Explore the answer through this simulation of the famous experiment in which he disproved the plum pudding model by observing alpha particles bouncing off atoms and determining that they must have a small core.

\section*{Tips For Success}
As you progress through the model of the atom, consider the effect that experimentation has on the scientific process. Ask yourself the following: What would our model of the atom be without Rutherford's gold foil experiment? What further understanding of the atom would not have been gained? How would that affect our current technologies? Though often confusing, experiments taking place today to further understand composition of the atom could perhaps have a similar effect.

\section*{Absorption and Emission Spectra}
\section*{Teacher Support}
Teacher Support Before beginning this section, it is important to review the concept of quantization. What does it mean to be quantized? Provide examples\\
of concepts that are quantum and concepts that are not. The review will make understanding emission spectra and orbital states much simpler.\\[0pt]
[EL]Spectrum refers to the range of electromagnetic radiation emitted by a form of matter or energy. The plural is spectra.

In 1900, Max Planck recognized that all energy radiated from a source is emitted by atoms in quantum states. How would that radical idea relate to the interior of an atom? The answer was first found by investigating the spectrum of light or emission spectrum produced when a gas is highly energized.

Figure 22.5 shows how to isolate the emission spectrum of one such gas. The gas is placed in the discharge tube at the left, where it is energized to the point at which it begins to radiate energy or emit light. The radiated light is channeled by a thin slit and then passed through a diffraction grating, which will separate the light into its constituent wavelengths. The separated light will then strike the photographic film on the right.

\section*{Teacher Support}
\section*{Teacher Support}
\begin{itemize}
  \item Be sure to explain how the discharge tube begins to radiate energy. When the gas is placed in an electric field, it can be supplied with enough energy that it begins to radiate. Focusing on an increase in energy via electric field will help students separate the concept from the continuous energy spectrum created by increasing the temperature of a radiating black body.
  \item If a diffraction grating is not handy, Figure 22.5 can be modeled using a prism in its place.
\end{itemize}

The line spectrum shown in part (b) of Figure 22.5 is the output shown on the film for excited iron. Note that this spectrum is not continuous but discrete. In other words, only particular wavelengths are emitted by the iron source. Why would that be the case?

\begin{figure}[h]
\begin{center}
  \includegraphics[max width=\textwidth]{6c6af082-56ef-4ff6-b696-4e9ed0d1965b-09}
\captionsetup{labelformat=empty}
\caption{Figure 22.5 Part (a) shows, from left to right, a discharge tube, slit, and diffraction grating producing a line spectrum. Part (b) shows the emission spectrum for iron. The discrete lines imply quantized energy states for the atoms that produce them. The line spectrum for each element is unique, providing a powerful and much-used analytical tool, and many line spectra were well known for many years before they could be explained with physics. (credit:(b) Yttrium91, Wikimedia Commons)}
\end{center}
\end{figure}

The spectrum of light created by excited iron shows a variety of discrete wavelengths emitted within the visible spectrum. Each element, when excited to the appropriate degree, will create a discrete emission spectrum as in part (b) of Figure 22.5. However, the wavelengths emitted will vary from element to element. The emission spectrum for iron was chosen for Figure 22.5 solely because a substantial portion of its emission spectrum is within the visible spectrum. Figure 22.6 shows the emission spectrum for hydrogen. Note that, while discrete, a large portion of hydrogen emission takes place in the ultraviolet and infrared regions.

\section*{Teacher Support}
Teacher Support Show pictures of glowing neon, argon, and helium gases. Have students speculate on the visible line spectra for those gases.

\begin{figure}[h]
\begin{center}
  \includegraphics[max width=\textwidth]{6c6af082-56ef-4ff6-b696-4e9ed0d1965b-10}
\captionsetup{labelformat=empty}
\caption{Figure 22.6 A schematic of the hydrogen spectrum shows several series named for those who contributed most to their determination. Part of the Balmer series is in the visible spectrum, while the Lyman series is entirely in the ultraviolet, and the Paschen series and others are in the infrared. Values of \(n_{\mathrm{f}}\) and \(n_{\mathrm{i}}\) are shown for some of the lines. Their importance will be described shortly.}
\end{center}
\end{figure}

Just as an emission spectrum shows all discrete wavelengths emitted by a gas, an absorption spectrum will show all light that is absorbed by a gas. Black lines exist where the wavelengths are absorbed, with the remainder of the spectrum lit by light is free to pass through. What relationship do you think exists between the black lines of a gas's absorption spectrum and the colored lines of its emission spectrum? Figure 22.7 shows the absorption spectrum of the Sun. The black lines are called Fraunhofer lines, and they correspond to the wavelengths absorbed by gases in the Sun's exterior.

\section*{Teacher Support}
Teacher Support [AL]Have students explore how Fraunhofer lines are used by astronomers to determine the composition of stars.\\[0pt]
[BL][OL]Ask the students to speculate on the absorption spectrum provided by a perfect emitter or a perfect absorber. What would the absorption spectrum look like for a black hole?

\begin{figure}[h]
\begin{center}
  \includegraphics[max width=\textwidth]{6c6af082-56ef-4ff6-b696-4e9ed0d1965b-10(1)}
\captionsetup{labelformat=empty}
\caption{Figure 22.7 The absorption spectrum of the Sun. The black lines appear at}
\end{center}
\end{figure}

wavelengths absorbed by the Sun's gas exterior. The energetic photons emitted from the Sun's interior are absorbed by gas in its exterior and reemitted in directions away from the observer. That results in dark lines within the absorption spectrum. The lines are called Fraunhofer lines, in honor of the German physicist who discovered them. Lines similar to those are used to determine the chemical composition of stars well outside our solar system.

\section*{Bohr's Explanation of the Hydrogen Spectrum}
To tie the unique signatures of emission spectra to the composition of the atom itself would require clever thinking. Niels Bohr (1885-1962), a Danish physicist, did just that, by making immediate use of Rutherford's planetary model of the atom. Bohr, shown in Figure 22.8, became convinced of its validity and spent part of 1912 at Rutherford's laboratory. In 1913, after returning to Copenhagen, he began publishing his theory of the simplest atom, hydrogen, based on Rutherford's planetary model.

\begin{figure}[h]
\begin{center}
  \includegraphics[max width=\textwidth]{6c6af082-56ef-4ff6-b696-4e9ed0d1965b-11}
\captionsetup{labelformat=empty}
\caption{Figure 22.8 Niels Bohr, Danish physicist, used the planetary model of the atom}
\end{center}
\end{figure}

to explain the atomic spectrum and size of the hydrogen atom. His many contributions to the development of atomic physics and quantum mechanics, his personal influence on many students and colleagues, and his personal integrity, especially in the face of Nazi oppression, earned him a prominent place in history. (credit: Unknown Author, Wikimedia Commons)

Bohr was able to derive the formula for the hydrogen spectrum using basic physics, the planetary model of the atom, and some very important new conjectures. His first conjecture was that only certain orbits are allowed: In other words, in an atom, the orbits of electrons are quantized. Each quantized orbit has a different distinct energy, and electrons can move to a higher orbit by absorbing energy or drop to a lower orbit by emitting energy. Because of the quantized orbits, the amount of energy emitted or absorbed must also be quantized, producing the discrete spectra seen in Figure 22.5 and Figure 22.7. In equation form, the amount of energy absorbed or emitted can be found as\\
\(\Delta E=E_{i}-E_{f}\),\\
22.1\\
where \(E_{i}\) refers to the energy of the initial quantized orbit, and \(E_{f}\) refers to the energy of the final orbits. Furthermore, the wavelength emitted can be found using the equation\\
\(h f=E_{i}-E_{f}\),\\
22.2\\
and relating the wavelength to the frequency found using the equation \(v=f \lambda\), where \(v\) corresponds to the speed of light.

It makes sense that energy is involved in changing orbits. For example, a burst of energy is required for a satellite to climb to a higher orbit. What is not expected is that atomic orbits should be quantized. Quantization is not observed for satellites or planets, which can have any orbit, given the proper energy (see Figure 22.9).\\
\includegraphics[max width=\textwidth, center]{6c6af082-56ef-4ff6-b696-4e9ed0d1965b-12}

Figure 22.9 The planetary model of the atom, as modified by Bohr, has the orbits of the electrons quantized. Only certain orbits are allowed, explaining why atomic spectra are discrete or quantized. The energy carried away from an atom by a photon comes from the electron dropping from one allowed orbit to another and is thus quantized. The same is true for atomic absorption of photons.

Figure 22.10 shows an energy-level diagram, a convenient way to display energy states. Each of the horizontal lines corresponds to the energy of an electron in a different orbital. Energy is plotted vertically with the lowest or ground state at the bottom and with excited states above. The vertical arrow downwards shows energy being emitted out of the atom due to an electron dropping from one excited state to another. That would correspond to a line shown on the atom's emission spectrum. The Lyman series shown in Figure 22.6 results from electrons dropping to the ground state, while the Balmer and Paschen series result to electrons dropping to the \(n=2\) and \(n=3\) states, respectively.\\
\includegraphics[max width=\textwidth, center]{6c6af082-56ef-4ff6-b696-4e9ed0d1965b-13}

Figure 22.10 An energy-level diagram plots energy vertically and is useful in visualizing the energy states of a system and the transitions between them. This diagram is for the hydrogen-atom electrons, showing a transition between two orbits having energies \(E_{4}\) and \(E_{2}\). The energy transition results in a Balmer series line in an emission spectrum.

\section*{Teacher Support}
Teacher Support [BL][OL]Quiz the students at this time. Show a large energy-level diagram on the board. Draw an arrow upward and ask whether energy is being emitted or absorbed. Have students draw an arrow showing more energy being absorbed. Draw a downward arrow and ask whether energy\\
is being emitted or absorbed. Have the students compare the amount of energy emitted for various energy-level drops.\\[0pt]
[AL]Have a student draw a planetary model to match the energy-level diagram shown above. Have students model the changes in energy states on their planetary model as they are shown on the energy-level diagram. Encourage students to use terminology such as Balmer series, Lyman series, and Paschen series in their explanations.

\section*{Energy and Wavelength of Emitted Hydrogen Spectra}
\section*{Teacher Support}
Teacher Support The negative value in the following equation can be difficult for students to understand. If you have already discussed potential energy in large-scale gravitational systems, you may remind students of that application prior to this discussion.

The energy associated with a particular orbital of a hydrogen atom can be found using the equation\\
\(E_{n}=-\frac{13.6 \mathrm{eV}}{n^{2}}(n=1,2,3, \ldots)\),\\
22.3\\
where \(n\) corresponds to the orbital value from the atom's nucleus. The negative value in the equation is based upon a baseline energy of zero when the electron is infinitely far from the atom. As a result, the negative value shows that energy is necessary to free the electron from its orbital state. The minimum energy to free the electron is also referred to as its binding energy. The equation is only valid for atoms with single electrons in their orbital shells (like hydrogen). For ionized atoms similar to hydrogen, the following formula may be used.\\
\(E_{n}=\frac{Z^{2}}{n^{2}} E_{o}(n=1,2,3, \ldots)\)\\
22.4

Please note that \(E_{o}\) corresponds to -13.6 eV , as mentioned earlier. Additionally, \(Z\) refers to the atomic number of the element studied. The atomic number is the number of protons in the nucleus-it is different for each element. The above equation is derived from some basic physics principles, namely conservation of energy, conservation of angular momentum, Coulomb's law, and centripetal force. There are three derivations that result in the orbital energy equations, and they are shown below. While you can use the energy equations without understanding the derivations, they will help to remind you of just how valuable those fundamental concepts are.

\section*{Teacher Support}
Teacher Support The following three derivations are meant for advanced students. If you have a student that is at or below level, you may skip them\\[0pt]
and progress directly to [link].\\
Derivation 1 (Finding the Radius of an Orbital) One primary difference between the planetary model of the solar system and the planetary model of the atom is the cause of the circular motion. While gravitation causes the motion of orbiting planets around an interior star, the Coulomb force is responsible for the circular shape of the electron's orbit. The magnitude of the centripetal force is \(\frac{m_{e} v^{2}}{r_{n}}\), while the magnitude of the Coulomb force is \(\frac{k\left(Z q_{e}\right)\left(q_{e}\right)}{r_{e}^{2}}\). The assumption here is that the nucleus is more massive than the stationary electron, and the electron orbits about it. That is consistent with the planetary model of the atom. Equating the Coulomb force and the centripetal force,\\
\(\frac{m_{e} v^{2}}{r_{n}}=\frac{k\left(Z q_{e}\right)\left(q_{e}\right)}{r_{e}^{2}}\),\\
22.5\\
which yields\\
\(r_{n}=\frac{k\left(Z q_{e}^{2}\right)}{m v^{2}}\).\\
22.6

Derivation 2 (Finding the Velocity of the Orbiting Electron) Bohr was clever enough to find a way to calculate the electron orbital energies in hydrogen. That was an important first step that has been improved upon, but it is well worth repeating here, because it does correctly describe many characteristics of hydrogen. Assuming circular orbits, Bohr proposed that the angular momentum \(L\) of an electron in its orbit is also quantized, that is, it has only specific, discrete values. The value for \(L\) is given by the formula\\
\(L=m_{e} v r_{n}=n \frac{h}{2 \pi}(n=1,2,3, \ldots)\),\\
22.7\\
where \(L\) is the angular momentum, \(m_{e}\) is the electron's mass, \(r_{n}\) is the radius of the \(n\)th orbit, and \(h\) is Planck's constant. Note that angular momentum is \(L=I \omega\). For a small object at a radius \(\mathrm{r}, I=m r^{2}\), and \(\omega=\frac{v}{r}\), so that \(L=I \omega=\left(m r^{2}\right)\left(\frac{v}{r}\right)=m v r\). Quantization says that the value of \(m v r\) can only be equal to \(h / 2,2 h / 2,3 h / 2\), etc. At the time, Bohr himself did not know why angular momentum should be quantized, but by using that assumption, he was able to calculate the energies in the hydrogen spectrum, something no one else had done at the time.

Derivation 3 (Finding the Energy of the Orbiting Electron) To get the electron orbital energies, we start by noting that the electron energy is the sum of its kinetic and potential energy.\\
\(E_{n}=K E+P E\)\\
22.8

Kinetic energy is the familiar \(K E=\frac{1}{2} m v^{2}\), assuming the electron is not moving at a relativistic speed. Potential energy for the electron is electrical, or \(P E= q_{e} V\), where \(V\) is the potential due to the nucleus, which looks like a point charge. The nucleus has a positive charge \(Z q_{e}\); thus, \(V=\frac{k Z q_{e}}{r_{n}}\), recalling an earlier equation for the potential due to a point charge from the chapter on Electricity and Magnetism. Since the electron's charge is negative, we see that \(P E=\frac{-k Z q_{e}^{2}}{r_{n}}\). Substituting the expressions for KE and PE,\\
\(E_{n}=\frac{1}{2} m_{e} v^{2}-\frac{k Z q_{e}^{2}}{r_{n}}\).\\
22.9

Now we solve for \(r_{n}\) and \(v\) using the equation for angular momentum \(L= m_{e} v r_{n}=n \frac{h}{2 \pi}(n=1,2,3, \ldots)\), giving\\
\(v=n \frac{h}{2 \pi m_{e} r_{n}}(n=1,2,3, \ldots)\)\\
22.10\\
and\\
\(r_{n}=n \frac{h}{2 \pi m_{e} v}(n=1,2,3, \ldots)\).\\
22.11

Substituting the expression for \(r_{n}\) and \(v\) into the above expressions for energy (KE and PE), and performing algebraic manipulation, yields\\
\(E_{n}=-\frac{Z^{2}}{n^{2}} E_{o}(n=1,2,3, \ldots)\)\\
22.12\\
for the orbital energies of hydrogen-like atoms. Here, \(E_{o}\) is the ground-state energy ( \(n=1\) ) for hydrogen ( \(Z=1\) ) and is given by\\
\(E_{o}=\frac{2 \pi^{2} q_{e}^{4} m_{e} k^{2}}{h^{2}}=13.6 \mathrm{eV}\).\\
22.13

Thus, for hydrogen,\\
\(E_{n}=-\frac{13.6 \mathrm{eV}}{n^{2}}(n=1,2,3, \ldots)\).\\
22.14

The relationship between orbital energies and orbital states for the hydrogen atom can be seen in Figure 22.11.

\begin{figure}[h]
\begin{center}
  \includegraphics[max width=\textwidth]{6c6af082-56ef-4ff6-b696-4e9ed0d1965b-17}
\captionsetup{labelformat=empty}
\caption{Figure 22.11 Energy-level diagram for hydrogen showing the Lyman, Balmer, and Paschen series of transitions. The orbital energies are calculated using the above equation, first derived by Bohr.}
\end{center}
\end{figure}

\section*{Worked Example}
A hydrogen atom is struck by a photon. How much energy must be absorbed from the photon to raise the electron of the hydrogen atom from its ground state to its second orbital?

\section*{Strategy}
The hydrogen atom has an atomic number of \(Z=1\). Raising the electron from the ground state to its second orbital will increase its orbital level from \(n=1\) to \(n=2\). The energy determined will be measured in electron-volts.

Solution

The amount of energy necessary to cause the change in electron state is the difference between the final and initial energies of the electron. The final energy state of the electron can be found using\\
\(E_{n}=\frac{Z^{2}}{n^{2}} E_{o}(n=1,2,3, \ldots)\).\\
22.15

Knowing the \(n\) and \(Z\) values for the hydrogen atom, and knowing that \(E_{o}=\) -13.6 eV , the result is\\
\(E_{f}=\frac{1^{2}}{2^{2}}(-13.6 \mathrm{eV})=-3.4 \mathrm{eV}\).\\
22.16

The original amount of energy associated with the electron is equivalent to the ground state orbital, or\\
\(E_{O}=\frac{1^{2}}{1^{2}}(-13.6 \mathrm{eV})=-13.6 \mathrm{eV}\).\\
22.17

The amount of energy necessary to change the orbital state of the electron can be found by determining the electron's change in energy.\\
\(\Delta E=E_{f}-E_{o}=(-3.4 \mathrm{eV})-(-13.6 \mathrm{eV})=+10.2 \mathrm{eV}\)\\
22.18

Discussion\\
The energy required to change the orbital state of the electron is positive. That means that for the electron to move to a state with greater energy, energy must be added to the atom. Should the electron drop back down to its original energy state, a change of -10.2 eV would take place, and 10.2 eV of energy would be emitted from the atom. Just as only quantum amounts of energy may be absorbed by the atom, only quantum amounts of energy can be emitted from the atom. That helps to explain many of the quantum light effects that you have learned about previously.

\section*{Worked Example}
Characteristic X-Ray Energy Calculate the approximate energy of an Xray emitted for an \(n=2\) to \(n=1\) transition in a tungsten anode in an X-ray tube.

\section*{Strategy}
How do we calculate energies in a multiple-electron atom? In the case of characteristic X-rays, the following approximate calculation is reasonable. Characteristic X-rays are produced when an inner-shell vacancy is filled. Inner-shell\\
electrons are nearer the nucleus than others in an atom and thus feel little net effect from the others. That is similar to what happens inside a charged conductor, where its excess charge is distributed over the surface so that it produces no electric field inside. It is reasonable to assume the inner-shell electrons have hydrogen-like energies, as given by\\
\(E_{n}=\frac{Z^{2}}{n^{2}} E_{o}(n=1,2,3, \ldots)\)\\
22.19

For tungsten, \(Z=74\), so that the effective charge is 73 .\\
Solution\\
The amount of energy given off as an X-ray is found using\\
\(\Delta E=h f=E_{i}-E_{f}\),\\
22.20\\
where\\
\(E_{f}=-\frac{Z^{2}}{1^{2}} E_{o}=-\frac{73^{2}}{1}(13.6 \mathrm{eV})=-72.5 \mathrm{keV}\)\\
22.21\\
and\\
\(E_{i}=-\frac{Z^{2}}{2^{2}} E_{o}=-\frac{73^{2}}{4}(13.6 \mathrm{eV})=-18.1 \mathrm{keV}\).\\
22.22

Thus,\\
\(\Delta E=E_{i}-E_{f}=(-18.1 \mathrm{keV})-(-72.5 \mathrm{keV})=54.4 \mathrm{keV}\).\\
22.23

Discussion\\
This large photon energy is typical of characteristic X-rays from heavy elements. It is large compared with other atomic emissions because it is produced when an inner-shell vacancy is filled, and inner-shell electrons are tightly bound. Characteristic X-ray energies become progressively larger for heavier elements because their energy increases approximately as \(Z^{2}\). Significant accelerating voltage is needed to create such inner-shell vacancies, because other shells are filled and you cannot simply bump one electron to a higher filled shell. You must remove it from the atom completely. In the case of tungsten, at least 72.5 kV is needed. Tungsten is a common anode material in X-ray tubes; so much of the energy of the impinging electrons is absorbed, raising its temperature, that a high-melting-point material like tungsten is required.

\section*{Teacher Support}
Teacher Support As with the previous section, the following derivation may be omitted for students below level. If it is omitted, students should pick up at -

The wavelength of light emitted by an atom can also be determined through basic derivations. Let us consider the energy of a photon emitted from a hydrogen atom in a downward transition, given by the equation\\
\(\Delta E=h f=E_{i}-E_{f}\)\\
22.24

Substituting \(E_{n}=\left(\frac{-13.6 \mathrm{eV}}{n^{2}}\right)\), we get\\
\(h f=(13.6 \mathrm{eV})\left(\frac{1}{n_{f}^{2}}-\frac{1}{n_{i}^{2}}\right)\).\\
22.25

Dividing both sides of the equation by \(h c\) gives us an expression for \(\frac{1}{\lambda}\),\\
\(\frac{h f}{h c}=\frac{f}{c}=\frac{1}{\lambda}=\frac{13.6 \mathrm{eV}}{h c}\left(\frac{1}{n_{f}^{2}}-\frac{1}{n_{i}^{2}}\right)\).\\
22.26

It can be shown that\\
\(\left(\frac{13.6 \mathrm{eV}}{h c}\right)=\frac{(13.6 \mathrm{eV})\left(1.602 \times 10^{-19} \mathrm{~J} / \mathrm{eV}\right)}{\left(6.602 \times 10^{-34} \mathrm{~J} \cdot \mathrm{~s}\right)\left(2.998 \times 10^{8} \mathrm{~m} / \mathrm{s}\right)}=1.097 \times 10^{7} \mathrm{~m}^{-1}=R\),\\
22.27\\
where \(R\) is the Rydberg constant.\\
Simplified, the formula for determining emitted wavelength can now be written as\\
\(\frac{1}{\lambda}=R\left(\frac{1}{n_{f}^{2}}-\frac{1}{n_{i}^{2}}\right)\).\\
22.28

\section*{Worked Example}
What wavelength of light is emitted by an electron dropping from the third orbital to the ground state of a hydrogen atom?

\section*{Strategy}
The ground state of a hydrogen atom is considered the first orbital of the atom. As a result, \(n_{f}=1\) and \(n_{i}=3\). The Rydberg constant has already been determined and will be constant regardless of atom chosen.

Solution\\
\(\frac{1}{\lambda}=R\left(\frac{1}{n_{f}^{2}}-\frac{1}{n_{i}^{2}}\right)\)\\
22.29

For the equation above, calculate wavelength based on the known energy states.\\
\(\frac{1}{\lambda}=1.097 \times 10^{7}\left(\frac{1}{1^{2}}-\frac{1}{3^{2}}\right)=9.751 \times 10^{6} \mathrm{~m}^{-1}\)\\
22.30

Rearranging the equation for wavelength yields\\
\(\lambda=1.026 \times 10^{-7} \mathrm{~m}=102.6 \mathrm{~nm}\).\\
22.31

Discussion\\
This wavelength corresponds to light in the ultraviolet spectrum. As a result, we would not be able to see the photon of light emitted when an electron drops from its third to first energy state. However, it is worth noting that by supplying light of wavelength precisely 102.6 nm , we can cause the electron in hydrogen to move from its first to its third orbital state.

\section*{Teacher Support}
Teacher Support Before progressing further, review Bohr's theory with the students.\\[0pt]
[BL][OL]Have students describe what is quantized and how quantum orbitals relate to the wavelengths of energy emitted.\\[0pt]
[AL]Have students explain how Bohr's theory is supported by blackbody radiation and the photoelectric effect. Explain that the photoelectric effect occurs when photons hitting the surface of a solar cell create an electric current in the cell, and that the photoelectric effect allows sunlight to be converted directly into electricity. See whether students can construct a deeper understanding behind the observations mentioned in the chapter on the Quantum Nature of Light.

\section*{Limits of Bohr's Theory and the Quantum Model of the Atom}
There are limits to Bohr's theory. It does not account for the interaction of bound electrons, so it cannot be fully applied to multielectron atoms, even one as simple as the two-electron helium atom. Bohr's model is what we call semiclassical. The orbits are quantized (nonclassical) but are assumed to be simple circular paths (classical). As quantum mechanics was developed, it became clear that there are no well-defined orbits; rather, there are clouds of probability. Additionally, Bohr's theory did not explain that some spectral lines are doublets or split into two when examined closely. While we shall examine a few of those aspects of quantum mechanics in more detail, it should be kept in mind\\
that Bohr did not fail. Rather, he made very important steps along the path to greater knowledge and laid the foundation for all of atomic physics that has since evolved.

\section*{DeBroglie's Waves}
\section*{Teacher Support}
Teacher Support If it has been a while since students have seen a standing wave, show them a standing wave using a slinky or electric oscillator. Show that a standing wave pattern can only emerge for specific frequencies. Similar to a guitar string only playing a particular frequency when plucked, the electron can only fit into a particular orbital when provided with the appropriate amount of energy.

Following Bohr's initial work on the hydrogen atom, a decade was to pass before Louis de Broglie proposed that matter has wave properties. The wave-like properties of matter were subsequently confirmed by observations of electron interference when scattered from crystals. Electrons can exist only in locations where they interfere constructively. How does that affect electrons in atomic orbits? When an electron is bound to an atom, its wavelength must fit into a small space, something like a standing wave on a string (see Figure 22.12). Orbits in which an electron can constructively interfere with itself are allowed. All orbits in which constructive interference cannot occur are not able to exist. Thus, only certain orbits are allowed. The wave nature of an electron, according to de Broglie, is why the orbits are quantized!

\begin{figure}[h]
\begin{center}
  \includegraphics[max width=\textwidth]{6c6af082-56ef-4ff6-b696-4e9ed0d1965b-22}
\captionsetup{labelformat=empty}
\caption{Figure 22.12 (a) Standing waves on a string have a wavelength related to the}
\end{center}
\end{figure}

length of the string, allowing them to interfere constructively. (b) If we imagine the string formed into a closed circle, we get a rough idea of how electrons in circular orbits can interfere constructively. (c) If the wavelength does not fit into the circumference, the electron interferes destructively; it cannot exist in such an orbit.

For a circular orbit, constructive interference occurs when the electron's wavelength fits neatly into the circumference, so that wave crests always align with crests and wave troughs align with troughs, as shown in Figure 22.12(b). More precisely, when an integral multiple of the electron's wavelength equals the circumference of the orbit, constructive interference is obtained. In equation form, the condition for constructive interference and an allowed electron orbit is\\
\(n \lambda_{n}=2 \pi r_{n}(n=1,2,3, \ldots)\),\\
22.32\\
where \(\lambda_{n}\) is the electron's wavelength and \(r_{n}\) is the radius of that circular orbit. Figure 22.13 shows the third and fourth orbitals of a hydrogen atom.

\begin{figure}[h]
\begin{center}
  \includegraphics[max width=\textwidth]{6c6af082-56ef-4ff6-b696-4e9ed0d1965b-23}
\captionsetup{labelformat=empty}
\caption{Figure 22.13 The third and fourth allowed circular orbits have three and four wavelengths, respectively, in their circumferences.}
\end{center}
\end{figure}

Heisenberg Uncertainty How does determining the location of an electron change its trajectory? The answer is fundamentally important-measurement affects the system being observed. It is impossible to measure a physical quantity exactly, and greater precision in measuring one quantity produces less precision in measuring a related quantity. It was Werner Heisenberg who first stated that limit to knowledge in 1929 as a result of his work on quantum mechanics and the wave characteristics of all particles (see Figure 22.14).

Figure 22.14 Werner Heisenberg was the physicist who developed the first version of true quantum mechanics. Not only did his work give a description of nature on the very small scale, it also changed our view of the availability of knowledge. Although he is universally recognized for the importance of his work by receiving the Nobel Prize in 1932, for example, Heisenberg remained in Germany during World War II and headed the German effort to build a nuclear bomb, permanently alienating himself from most of the scientific community. (credit: Unknown Author, Wikimedia Commons)

For example, you can measure the position of a moving electron by scattering light or other electrons from it. However, by doing so, you are giving the electron energy, and therefore imparting momentum to it. As a result, the momentum of the electron is affected and cannot be determined precisely. This change in momentum could be anywhere from close to zero up to the relative momentum of the electron \((p \approx h / \lambda)\). Note that, in this case, the particle is an electron, but the principle applies to any particle.

Viewing the electron through the model of wave-particle duality, Heisenberg recognized that, because a wave is not located at one fixed point in space, there is an uncertainty associated with any electron's position. That uncertainty in position, \(\Delta x\), is approximately equal to the wavelength of the particle. That is, \(\Delta x \approx \lambda\). There is an interesting trade-off between position and momentum. The uncertainty in an electron's position can be reduced by using a shorterwavelength electron, since \(\Delta x \approx \lambda\). But shortening the wavelength increases the uncertainty in momentum, since \(\Delta p \approx h / \lambda\). Conversely, the uncertainty in momentum can be reduced by using a longer-wavelength electron, but that increases the uncertainty in position. Mathematically, you can express the tradeoff by multiplying the uncertainties. The wavelength cancels, leaving\\
\(\Delta x \Delta p \approx h\).\\
Therefore, if one uncertainty is reduced, the other must increase so that their product is \(\approx h\). With the use of advanced mathematics, Heisenberg showed that the best that can be done in a simultaneous measurement of position and momentum is\\
\(\Delta x \Delta p \geq \frac{h}{4 \pi}\).\\
22.33

That relationship is known as the Heisenberg uncertainty principle.

\section*{Teacher Support}
Teacher Support [BL][OL]Another model for explaining the uncertainty principle for students struggling with the concept: Imagine searching for a floating balloon in a dark room. When your hand strikes the balloon, you provide an impulse and move it from its original spot. While you learned the\\
position of the object, you disturbed it during your search. As a result, it has a new momentum that is unknown!

The Quantum Model of the Atom Because of the wave characteristic of matter, the idea of well-defined orbits gives way to a model in which there is a cloud of probability, consistent with Heisenberg's uncertainty principle. Figure 22.15 shows how the principle applies to the ground state of hydrogen. If you try to follow the electron in some well-defined orbit using a probe that has a wavelength small enough to measure position accurately, you will instead knock the electron out of its orbit. Each measurement of the electron's position will find it to be in a definite location somewhere near the nucleus. Repeated measurements reveal a cloud of probability like that in the figure, with each speck the location determined by a single measurement. There is not a welldefined, circular-orbit type of distribution. Nature again proves to be different on a small scale than on a macroscopic scale.

\section*{Teacher Support}
Teacher Support [BL][OL]Upon viewing Figure 22.15, have students explain how the image is supported by Heisenberg's uncertainty principle.\\[0pt]
[AL]Have students explain how DeBroglie's waves tie to Figure 22.15. How would the electron cloud appear if the picture were of a carbon atom?\\
\includegraphics[max width=\textwidth, center]{6c6af082-56ef-4ff6-b696-4e9ed0d1965b-26}

Figure 22.15 The ground state of a hydrogen atom has a probability cloud describing the position of its electron. The probability of finding the electron is proportional to the darkness of the cloud. The electron can be closer or farther than the Bohr radius, but it is very unlikely to be a great distance from the nucleus.

\section*{Virtual Physics}
Models of the Hydrogen Atom Click to view content\\
How did scientists figure out the structure of atoms without looking at them? Try out different models by shooting light at the atom. Use this simulation to\\
see how the prediction of the model matches the experimental results.

\section*{Check Your Understanding}
1.

Alpha particles are positively charged. What influence did their charge have on the gold foil experiment?\\
a. The positively charged alpha particles were attracted by the attractive electrostatic force from the positive nuclei of the gold atoms.\\
b. The positively charged alpha particles were scattered by the attractive electrostatic force from the positive nuclei of the gold atoms.\\
c. The positively charged alpha particles were scattered by the repulsive electrostatic force from the positive nuclei of the gold atoms.\\
d. The positively charged alpha particles were attracted by the repulsive electrostatic force from the positive nuclei of the gold atoms.

\subsection*{22.2 Nuclear Forces and Radioacti it}
\section*{Section Learning Objectives}
By the end of this section, you will be able to do the following:

\begin{itemize}
  \item Describe the structure and forces present within the nucleus
  \item Explain the three types of radiation
  \item Write nuclear equations associated with the various types of radioactive decay
\end{itemize}

\section*{Teacher Support}
Teacher Support The learning objectives in this section will help your students master the following standards:

\begin{itemize}
  \item (5) Science concepts. The student knows the nature of forces in the physical world. The student is expected to:
  \item (H) describe evidence for and effects of the strong and weak nuclear forces in nature.
  \item (8) Science concepts. The student knows simple examples of atomic, nuclear, and quantum phenomena. The student is expected to:
  \item (B) compare and explain the emission spectra produced by various atoms; and
  \item (C) describe the significance of mass-energy equivalence and apply it in explanations of phenomena such as nuclear stability, fission, and fusion.
\end{itemize}

\section*{Section Key Terms}
\section*{Teacher Support}
Teacher Support [BL][OL][AL]As in the beginning of Section 1, have students create a list of facts they have learned about the atom. Have the students update their list throughout this section.

There is an ongoing quest to find the substructures of matter. At one time, it was thought that atoms would be the ultimate substructure. However, just when the first direct evidence of atoms was obtained, it became clear that they have a substructure and a tiny nucleus. The nucleus itself has spectacular characteristics. For example, certain nuclei are unstable, and their decay emits radiations with energies millions of times greater than atomic energies. Some of\\
the mysteries of nature, such as why the core of Earth remains molten and how the Sun produces its energy, are explained by nuclear phenomena. The exploration of radioactivity and the nucleus has revealed new fundamental particles, forces, and conservation laws. That exploration has evolved into a search for further underlying structures, such as quarks. In this section, we will explore the fundamentals of the nucleus and nuclear radioactivity.

\section*{The Structure of the Nucleus}
At this point, you are likely familiar with the neutron and proton, the two fundamental particles that make up the nucleus of an atom. Those two particles, collectively called nucleons, make up the small interior portion of the atom. Both particles have nearly the same mass, although the neutron is about two parts in 1,000 more massive. The mass of a proton is equivalent to 1,836 electrons, while the mass of a neutron is equivalent to that of 1,839 electrons. That said, each of the particles is significantly more massive than the electron.

When describing the mass of objects on the scale of nucleons and atoms, it is most reasonable to measure their mass in terms of atoms. The atomic mass unit (u) was originally defined so that a neutral carbon atom would have a mass of exactly 12 u . Given that protons and neutrons are approximately the same mass, that there are six protons and six neutrons in a carbon atom, and that the mass of an electron is minuscule in comparison, measuring this way allows for both protons and neutrons to have masses close to 1 u . Table 22.1 shows the mass of protons, neutrons, and electrons on the new scale.

\section*{Tips For Success}
For most conceptual situations, the difference in mass between the proton and neutron is insubstantial. In fact, for calculations that require fewer than four significant digits, both the proton and neutron masses may be considered equivalent to one atomic mass unit. However, when determining the amount of energy released in a nuclear reaction, as in Alpha Decay Energy Found from Nuclear Masses, the difference in mass cannot be ignored.

Another other useful mass unit on the atomic scale is the \(\mathrm{MeV} / c^{2}\). While rarely used in most contexts, it is convenient when one uses the equation \(E=m c^{2}\), as will be addressed later in this text.

Table 22.1 Atomic Masses for Multiple Units\\
To more completely characterize nuclei, let us also consider two other important quantities: the atomic number and the mass number. The atomic number, \(Z\), represents the number of protons within a nucleus. That value determines the\\
elemental quality of each atom. Every carbon atom, for instance, has a \(Z\) value of 6 , whereas every oxygen atom has a \(Z\) value of 8 . For clarification, only oxygen atoms may have a \(Z\) value of 8 . If the \(Z\) value is not 8 , the atom cannot be oxygen.

The mass number, \(A\), represents the total number of protons and neutrons, or nucleons, within an atom. For an ordinary carbon atom the mass number would be 12 , as there are typically six neutrons accompanying the six protons within the atom. In the case of carbon, the mass would be exactly 12 u . For oxygen, with a mass number of 16 , the atomic mass is 15.994915 u . Of course, the difference is minor and can be ignored for most scenarios. Again, because the mass of an electron is so small compared to the nucleons, the mass number and the atomic mass can be essentially equivalent. Figure 22.16 shows an example of Lithium-7, which has an atomic number of 3 and a mass number of 7 .

\section*{Teacher Support}
Teacher Support [BL]Students may confuse the terms mass number and atomic number. Remind them that the mass is based on both protons and neutrons and so mass number is a measure of both combined. The atomic number differentiates between two different atoms, which only protons can do. A few examples showing pictures of nuclei and having students identify the mass and atomic numbers of each should help.

How does the mass number help to differentiate one atom from another? If each atom of carbon has an atomic number of 6 , then what is the value of including the mass number at all? The intent of the mass number is to differentiate between various isotopes of an atom. The term isotope refers to the variation of atoms based upon the number of neutrons within their nucleus. While it is most common for there to be six neutrons accompanying the six protons within a carbon atom, it is possible to find carbon atoms with seven neutrons or eight neutrons. Those carbon atoms are respectively referred to as carbon- 13 and carbon- 14 atoms, with their mass numbers being their primary distinction. The isotope distinction is an important one to make, as the number of neutrons within an atom can affect a number of its properties, not the least of which is nuclear stability.

\begin{figure}[h]
\begin{center}
  \includegraphics[max width=\textwidth]{6c6af082-56ef-4ff6-b696-4e9ed0d1965b-31}
\captionsetup{labelformat=empty}
\caption{Figure 22.16 Lithium-7 has three protons and four neutrons within its nucleus. As a result, its mass number is 7 , while its atomic number is 3 . The actual mass of the atom is \(7.016 u\). Lithium 7 is an isotope of lithium.}
\end{center}
\end{figure}

\section*{Teacher Support}
Teacher Support Point out to students that the number of electrons is irrelevant to discussions of mass number, atomic number, and isotopes.

To more easily identify various atoms, their atomic number and mass number are typically written in a form of representation called the nuclide. The nuclide form appears as follows: \({ }_{A}^{Z} X_{N}\), where \(X\) is the atomic symbol and \(N\) represents the number of neutrons.

Let us look at a few examples of nuclides expressed in the \({ }_{A}^{Z} X_{N}\) notation. The nucleus of the simplest atom, hydrogen, is a single proton, or H11 (the zero for no neutrons is often omitted). To check the symbol, refer to the periodic table - you see that the atomic number \(Z\) of hydrogen is 1 . Since you are given that there are no neutrons, the mass number \(A\) is also 1 . There is a scarce form of hydrogen found in nature called deuterium; its nucleus has one proton and one neutron and, hence, twice the mass of common hydrogen. The symbol for deuterium is, thus, \({ }_{1}^{2} \mathrm{H}_{2}\). An even rarer-and radioactive-form of hydrogen is called tritium, since it has a single proton and two neutrons, and it is written \({ }_{1}^{3} \mathrm{H}_{2}\). The three varieties of hydrogen have nearly identical chemistries, but the nuclei differ greatly in mass, stability, and other characteristics. Again, the different nuclei are referred to as isotopes of the same element.

There is some redundancy in the symbols \(A, X, Z\), and \(N\). If the element \(X\) is known, then \(Z\) can be found in a periodic table. If both \(A\) and \(X\) are known, then \(N\) can also be determined by first finding \(Z\); then, \(N=A-Z\). Thus the simpler notation for nuclides is\\
\({ }^{A} X\),\\
which is sufficient and is most commonly used. For example, in this simpler notation, the three isotopes of hydrogen are \({ }^{1} \mathrm{H},{ }^{2} \mathrm{H}\), and \({ }^{3} \mathrm{H}\). For \({ }^{238} \mathrm{U}\), should we need to know, we can determine that \(Z=92\) for uranium from the periodic table, and thus, \(N=238-92=146\).

\section*{Teacher Support}
Teacher Support This explanation is provided to help students understand nomenclature later on in the text. However, it may be useful to have students practice writing nuclide notation to build confidence with the concept.

\section*{Radioactivity and Nuclear Forces}
In 1896, the French physicist Antoine Henri Becquerel (1852-1908) noticed something strange. When a uranium-rich mineral called pitchblende was placed on a completely opaque envelope containing a photographic plate, it darkened spots on the photographic plate.. Becquerel reasoned that the pitchblende must emit invisible rays capable of penetrating the opaque material. Stranger still was that no light was shining on the pitchblende, which means that the pitchblende was emitting the invisible rays continuously without having any energy input! There is an apparent violation of the law of conservation of energy, one that scientists can now explain using Einstein's famous equation \(E=m c^{2}\). It was soon evident that Becquerel's rays originate in the nuclei of the atoms and have other unique characteristics.

To this point, most reactions you have studied have been chemical reactions, which are reactions involving the electrons surrounding the atoms. However, two types of experimental evidence implied that Becquerel's rays did not originate with electrons, but instead within the nucleus of an atom.

First, the radiation is found to be only associated with certain elements, such as uranium. Whether uranium was in the form of an element or compound was irrelevant to its radiation. In addition, the presence of radiation does not vary with temperature, pressure, or ionization state of the uranium atom. Since all of those factors affect electrons in an atom, the radiation cannot come from electron transitions, as atomic spectra do.

The huge energy emitted during each event is the second piece of evidence that the radiation cannot be atomic. Nuclear radiation has energies on the order of \(10^{6} \mathrm{eV}\) per event, which is much greater than typical atomic energies that are a few eV, such as those observed in spectra and chemical reactions, and more than ten times as high as the most energetic X-rays.

\section*{Teacher Support}
Teacher Support To emphasize the point that most reactions are chemical, have students brainstorm a list of reactions from chemistry class. Have them\\
describe the changes that cause the reactions to take place. Are the reactions the result of nuclear interactions or electron interactions?

But why would reactions within the nucleus take place? And what would cause an apparently stable structure to begin emitting energy? Was there something special about Becquerel's uranium-rich pitchblende? To answer those questions, it is necessary to look into the structure of the nucleus. Though it is perhaps surprising, you will find that many of the same principles that we observe on a macroscopic level still apply to the nucleus.

Nuclear Stability A variety of experiments indicate that a nucleus behaves something like a tightly packed ball of nucleons, as illustrated in Figure 22.17. Those nucleons have large kinetic energies and, thus, move rapidly in very close contact. Nucleons can be separated by a large force, such as in a collision with another nucleus, but strongly resist being pushed closer together. The most compelling evidence that nucleons are closely packed in a nucleus is that the radius of a nucleus, \(r\), is found to be approximately\\
\(r=r_{o} A^{\frac{1}{3}}\),\\
22.35\\
where \(r_{o}=1.2\) femtometer ( fm ) and \(A\) is the mass number of the nucleus.\\
Note that \(r^{3} \propto A\). Since many nuclei are spherical, and the volume of a sphere is \(V=\left(\frac{4}{3}\right) \pi r^{3}\), we see that \(V \propto A\)-that is, the volume of a nucleus is proportional to the number of nucleons in it. That is what you expect if you pack nucleons so close that there is no empty space between them.\\
\includegraphics[max width=\textwidth, center]{6c6af082-56ef-4ff6-b696-4e9ed0d1965b-33}

\begin{itemize}
  \item Proton
  \item Neutron
\end{itemize}

Figure 22.17 Nucleons are held together by nuclear forces and resist both being pulled apart and pushed inside one another. The volume of the nucleus is the sum of the volumes of the nucleons in it, here shown in different colors to represent protons and neutrons.

So what forces hold a nucleus together? After all, the nucleus is very small and its protons, being positive, should exert tremendous repulsive forces on one another. Considering that, it seems that the nucleus would be forced apart, not together!

The answer is that a previously unknown force holds the nucleus together and makes it into a tightly packed ball of nucleons. This force is known as the strong nuclear force. The strong force has such a short range that it quickly fall to zero over a distance of only \(10^{-15}\) meters. However, like glue, it is very strong when the nucleons get close to one another.

\section*{Teacher Support}
Teacher Support The relationship between the repulsive Coulomb force and the attractive nuclear force can be modeled using a balloon. Squeeze the balloon, or have one student hold a balloon while the other pushes on it. The pressure of the air inside the balloon will model the Coulomb force while the push from the student models the nuclear force.

The balancing of the electromagnetic force with the nuclear forces is what allows the nucleus to maintain its spherical shape. If, for any reason, the electromagnetic force should overcome the nuclear force, components of the nucleus would be projected outward, creating the very radiation that Becquerel discovered!

Understanding why the nucleus would break apart can be partially explained using Table 22.2. The balance between the strong nuclear force and the electromagnetic force is a tenuous one. Recall that the attractive strong nuclear force exists between any two nucleons and acts over a very short range while the weaker repulsive electromagnetic force only acts between protons, although over a larger range. Considering the interactions, an imperfect balance between neutrons and protons can result in a nuclear reaction, with the result of regaining equilibrium.

Table 22.2 Comparing the Electromagnetic and Strong Forces\\
The radiation discovered by Becquerel was due to the large number of protons present in his uranium-rich pitchblende. In short, the large number of protons caused the electromagnetic force to be greater than the strong nuclear force. To regain stability, the nucleus needed to undergo a nuclear reaction called alpha () decay.

\section*{The Three Types of Radiation}
Radioactivity refers to the act of emitting particles or energy from the nucleus. When the uranium nucleus emits energetic nucleons in Becquerel's experiment, the radioactive process causes the nucleus to alter in structure. The alteration is called radioactive decay. Any substance that undergoes radioactive decay is said to be radioactive. That those terms share a root with the term radiation should not be too surprising, as they all relate to the transmission of energy.

\section*{Teacher Support}
Teacher Support Students have plenty of preconceptions about radioactivity. Discuss their preconceptions. Are their concerns related to the radioactive decay process or to the energy transmitted in the process?

Radioactivity can be understood as a tendency for a nucleus to reach equilibrium. Discuss with students other instances of objects desiring equilibrium. Such macroscopic interactions may help to make their understanding of radioactivity more tangible.

Alpha Decay Alpha decay refers to the type of decay that takes place when too many protons exist in the nucleus. It is the most common type of decay and causes the nucleus to regain equilibrium between its two competing internal forces. During alpha decay, the nucleus ejects two protons and two neutrons, allowing the strong nuclear force to regain balance with the repulsive electromagnetic force. The nuclear equation for an alpha decay process can be shown as follows.

XZAN → YZ-2A-4 N + H 24 e\\
22.36\\
\includegraphics[max width=\textwidth, center]{6c6af082-56ef-4ff6-b696-4e9ed0d1965b-35}

Figure 22.18 A nucleus undergoes alpha decay. The alpha particle can be seen as made up of two neutrons and two protons, which constitute a helium-4 atom.

Three things to note as a result of the above equation:

\begin{enumerate}
  \item By ejecting an alpha particle, the original nuclide decreases in atomic number. That means that Becquerel's uranium nucleus, upon decaying, is actually transformed into thorium, two atomic numbers lower on the periodic table! The process of changing elemental composition is called transmutation.
  \item Note that the two protons and two neutrons ejected from the nucleus combine to form a helium nucleus. Shortly after decay, the ejected helium ion typically acquires two electrons to become a stable helium atom.
  \item Finally, it is important to see that, despite the elemental change, physical conservation still takes place. The mass number of the new element and the alpha particle together equal the mass number of the original element. Also, the net charge of all particles involved remains the same before and after the transmutation.
\end{enumerate}

\section*{Teacher Support}
Teacher Support Differentiate between ionization and transmutation to reinforce that alpha radiation is an elemental change.

Beta Decay Like alpha decay, beta ( \(\beta\) ) decay also takes place when there is an imbalance between neutrons and protons within the nucleus. For beta decay, however, a neutron is transformed into a proton and electron or vice versa. The transformation allows for the total mass number of the atom to remain the same, although the atomic number will increase by one (or decrease by one). Once again, the transformation of the neutron allows for a rebalancing of the strong nuclear and electromagnetic forces. The nuclear equation for a beta decay process is shown below.\\
\({ }_{Z}^{A} X_{N} \rightarrow{ }_{Z+1}^{A} Y_{N-1}+e+v\)\\
The symbol \(v\) in the equation above stands for a high-energy particle called the neutrino. A nucleus may also emit a positron, and in that case \(Z\) decreases and \(N\) increases. It is beyond the scope of this section and will be discussed in further detail in the chapter on particles. It is worth noting, however, that the mass number and charge in all beta-decay reactions are conserved.

\begin{figure}[h]
\begin{center}
  \includegraphics[max width=\textwidth]{6c6af082-56ef-4ff6-b696-4e9ed0d1965b-36}
\captionsetup{labelformat=empty}
\caption{Figure 22.19 A nucleus undergoes beta decay. The neutron splits into a proton, electron, and neutrino. This particular decay is called \(\beta^{-}\)decay.}
\end{center}
\end{figure}

\section*{Teacher Support}
Teacher Support [AL]The mass number of the nucleus is conserved, but is the mass? Mention that that mass of a proton is slightly less than the mass of the neutron. Have students consider where that mass goes. Note - Not all of it goes to the electron.

Gamma Decay Gamma decay is a unique form of radiation that does not involve balancing forces within the nucleus. Gamma decay occurs when a nucleus drops from an excited state to the ground state. Recall that such a change in energy state will release energy from the nucleus in the form of a photon. The energy associated with the photon emitted is so great that its wavelength is shorter than that of an X-ray. Its nuclear equation is as follows.\\
\({ }_{Z}^{A} X_{N} \rightarrow X_{N}+\gamma\)\\
22.37

\begin{figure}[h]
\begin{center}
  \includegraphics[max width=\textwidth]{6c6af082-56ef-4ff6-b696-4e9ed0d1965b-37}
\captionsetup{labelformat=empty}
\caption{Figure 22.20 A nucleus undergoes gamma decay. The nucleus drops in energy state, releasing a gamma ray.}
\end{center}
\end{figure}

\section*{Teacher Support}
Teacher Support [OL][AL]To differentiate between the three types of decay, you can have the students construct a large Venn diagram of their radioactive properties. For students struggling with the idea, it may be easier to have them first construct a table of what they know about each type of decay.

\section*{Worked Example}
Creating a Decay Equation Write the complete decay equation in \({ }_{Z}^{A} X_{N}\) notation for beta decay producing \({ }^{137} \mathrm{Ba}\). Refer to the periodic table for values of \(Z\).

\section*{Strategy}
Beta decay results in an increase in atomic number. As a result, the original (or parent) nucleus, must have an atomic number of one fewer proton.\\
Solution\\
The equation for beta decay is as follows\\
\(\mathrm{XZAN} \rightarrow \mathrm{YZ}+1\) A \(\mathrm{N}-1+\mathrm{e}+\mathrm{v}\).\\
22.38

Considering that barium is the product (or daughter) nucleus and has an atomic number of 56, the original nucleus must be of an atomic number of 55. That corresponds to cesium, or Cs.

C 55137 s N → B 56137 a N-1 +e+v\\
22.39

The number of neutrons in the parent cesium and daughter barium can be determined by subtracting the atomic number from the mass number ( \(137-55\) for cesium, \(137-56\) for barium). Substitute those values for the \(N\) and \(N-1\) subscripts in the above equation.

C 55137 s \(82 \rightarrow\) B 56137 a \(81+\mathrm{e}+\mathrm{v}\)\\
22.40

Discussion\\
The terms parent and daughter nucleus refer to the reactants and products of a nuclear reaction. The terminology is not just used in this example, but in all nuclear reaction examples. The cesium-137 nuclear reaction poses a significant health risk, as its chemistry is similar to that of potassium and sodium, and so it can easily be concentrated in your cells if ingested.

\section*{Worked Example}
Alpha Decay Energy Found from Nuclear Masses Find the energy emitted in the \(\alpha\) decay of \({ }^{239} \mathrm{Pu}\).

\section*{Strategy}
Nuclear reaction energy, such as released in \(\alpha\) decay, can be found using the equation \(E=m c^{2}\). We must first find \(\Delta m\), the difference in mass between the parent nucleus and the products of the decay.

The mass of pertinent particles is as follows\\
\({ }^{239} \mathrm{Pu}: 239.052157 \mathrm{u}\)\\
\({ }^{235} \mathrm{U}: 235.043924 \mathrm{u}\)\\
\({ }^{4}\) He: 4.002602 u.\\
Solution\\
The decay equation for \({ }^{239} \mathrm{Pu}\) is\\
P \(239 \mathrm{u} \rightarrow \mathrm{U} 235+\mathrm{H} 4\) e .\\
22.41

Determine the amount of mass lost between the parent and daughter nuclei.\\
\(\Delta m=m\left({ }^{239} \mathrm{P} \mathrm{u}\right)-\left(m\left({ }^{239} \mathrm{U}\right)+m\left({ }^{4} \mathrm{He}\right)\right)\)\\
\(\Delta m=239.052157 \mathrm{u}-(235.043924 \mathrm{u}+4.002602 \mathrm{u})\)\\
\(\Delta m=0.0005631 \mathrm{u}\)\\
22.42

Now we can find \(E\) by entering \(\Delta m\) into the equation.\\
\(E=(\Delta m) c^{2}=(0.005631 \mathrm{u}) c^{2}\)\\
22.43

And knowing that \(1 \mathrm{u}=931.5 \mathrm{meV} / c^{2}\), we can find that\\
\(E=(0.005631)\left(931.5 \mathrm{MeV} / c^{2}\right)\left(c^{2}\right)=5.25 \mathrm{MeV}\).\\
22.44

Discussion\\
The energy released in this \(\alpha\) decay is in the MeV range, about \(10^{6}\) times as great as typical chemical reaction energies, consistent with previous discussions. Most of the energy becomes kinetic energy of the \(\alpha\) particle (or \({ }^{4} \mathrm{He}\) nucleus), which moves away at high speed.

The energy carried away by the recoil of the \({ }^{235} \mathrm{U}\) nucleus is much smaller, in order to conserve momentum. The \({ }^{235} \mathrm{U}\) nucleus can be left in an excited state to later emit photons ( \(\gamma\) rays). The decay is spontaneous and releases energy, because the products have less mass than the parent nucleus.

\section*{Properties of Radiation}
The charges of the three radiated particles differ. Alpha particles, with two protons, carry a net charge of +2 . Beta particles, with one electron, carry a net charge of -1 . Meanwhile, gamma rays are solely photons, or light, and carry no charge. The difference in charge plays an important role in how the three radiations affect surrounding substances.

\section*{Teacher Support}
Teacher Support [OL][AL]Show table 22.3 to students after they read the preceding paragraph. See if they can explain the penetration distances based on charge difference alone.\\[0pt]
[BL]-See if students can come up with a relationship between penetration distance and particle charge. If there were a radiation particle with a charge of -4 , what would you expect its penetration distance to be?

Alpha particles, being highly charged, will quickly interact with ions in the air and electrons within metals. As a result, they have a short range and short penetrating distance in most materials. Beta particles, being slightly less charged, have a larger range and larger penetrating distance. Gamma rays, on the other hand, have little electric interaction with particles and travel much farther. Two diagrams below show the importance of difference in penetration. Table 22.3 shows the distance of radiation penetration, and Figure 22.21 shows the influence various factors have on radiation penetration distance.

\begin{figure}[h]
\begin{center}
\captionsetup{labelformat=empty}
\caption{Table 22.3 Comparing Ranges of Radioactive Decay}
  \includegraphics[max width=\textwidth]{6c6af082-56ef-4ff6-b696-4e9ed0d1965b-40}
\end{center}
\end{figure}

Figure 22.21 The penetration or range of radiation depends on its energy, the material it encounters, and the type of radiation. (a) Greater energy means greater range. (b) Radiation has a smaller range in materials with high electron density. (c) Alphas have the smallest range, betas have a greater range, and gammas have the greatest range.

\section*{Links To Physics}
Radiation Detectors The first direct detection of radiation was Becquerel's darkened photographic plate. Photographic film is still the most common detector of ionizing radiation, being used routinely in medical and dental X-rays. Nuclear radiation can also be captured on film, as seen in Figure 22.22. The mechanism for film exposure by radiation is similar to that by photons. A quantum of energy from a radioactive particle interacts with the emulsion and alters it chemically, thus exposing the film. Provided the radiation has more than the few eV of energy needed to induce the chemical change, the chemical alteration will occur. The amount of film darkening is related to the type of radiation and amount of exposure. The process is not 100 percent efficient, since not all incident radiation interacts and not all interactions produce the chemical change.

\begin{figure}[h]
\begin{center}
  \includegraphics[max width=\textwidth]{6c6af082-56ef-4ff6-b696-4e9ed0d1965b-41}
\captionsetup{labelformat=empty}
\caption{Figure 22.22 Film badges contain film similar to that used in this dental X-ray film. It is sandwiched between various absorbers to determine the penetrating ability of the radiation as well as the amount. Film badges are worn to determine radiation exposure. (credit: Werneuchen, Wikimedia Commons)}
\end{center}
\end{figure}

Another very common radiation detector is the Geiger tube. The clicking and buzzing sound we hear in dramatizations and documentaries, as well as in our own physics labs, is usually an audio output of events detected by a Geiger counter. These relatively inexpensive radiation detectors are based on the simple and sturdy Geiger tube, shown schematically in Figure 22.23. A conducting cylinder with a wire along its axis is filled with an insulating gas so that a voltage applied between the cylinder and wire produces almost no current. Ionizing radiation passing through the tube produces free ion pairs that are attracted to the wire and cylinder, forming a current that is detected as a count. Not every particle is detected, since some radiation can pass through without producing enough ionization. However, Geiger counters are very useful in producing a\\
prompt output that reveals the existence and relative intensity of ionizing radiation.

\begin{figure}[h]
\begin{center}
  \includegraphics[max width=\textwidth]{6c6af082-56ef-4ff6-b696-4e9ed0d1965b-42}
\captionsetup{labelformat=empty}
\caption{Figure 22.23 (a) Geiger counters such as this one are used for prompt monitoring of radiation levels, generally giving only relative intensity and not identifying the type or energy of the radiation. (credit: Tim Vickers, Wikimedia Commons) (b) Voltage applied between the cylinder and wire in a Geiger tube affects ions and electrons produced by radiation passing through the gas-filled cylinder. Ions move toward the cylinder and electrons toward the wire. The resulting current is detected and registered as a count.}
\end{center}
\end{figure}

Another radiation detection method records light produced when radiation interacts with materials. The energy of the radiation is sufficient to excite atoms in a material that may fluoresce, such as the phosphor used by Rutherford's group. Materials called scintillators use a more complex process to convert radiation energy into light. Scintillators may be liquid or solid, and they can be very efficient. Their light output can provide information about the energy, charge, and type of radiation. Scintillator light flashes are very brief in duration, allowing the detection of a huge number of particles in short periods of time. Scintillation detectors are used in a variety of research and diagnostic applications. Among those are the detection of the radiation from distant galaxies using\\
satellite-mounted equipment and the detection of exotic particles in accelerator laboratories.

\section*{Virtual Physics}
Beta Decay Click to view content\\
Watch beta decay occur for a collection of nuclei or for an individual nucleus. With this applet, individuals or groups of students can compare half-lives!

\section*{Check Your Understanding}
2.

What leads scientists to infer that the nuclear strong force exists?\\
a. A strong force must hold all the electrons outside the nucleus of an atom.\\
b. A strong force must counteract the highly attractive Coulomb force in the nucleus.\\
c. A strong force must hold all the neutrons together inside the nucleus.\\
d. A strong force must counteract the highly repulsive Coulomb force between protons in the nucleus.

\subsection*{22.3 Half Life and Radiometric Dating}
\section*{Section Learning Objectives}
By the end of this section, you will be able to do the following:

\begin{itemize}
  \item Explain radioactive half-life and its role in radiometric dating
  \item Calculate radioactive half-life and solve problems associated with radiometric dating
\end{itemize}

\section*{Teacher Support}
Teacher Support The learning objectives in this section will help your students master the following standards:

\begin{itemize}
  \item (5) Science concepts. The student knows the nature of forces in the physical world. The student is expected to:
  \item (H) describe evidence for and effects of the strong and weak nuclear forces in nature.
  \item (8) Science concepts. The student knows simple examples of atomic, nuclear, and quantum phenomena. The student is expected to:
  \item (C) describe the significance of mass-energy equivalence and apply it in explanations of phenomena such as nuclear stability, fission, and fusion.
\end{itemize}

\section*{Section Key Terms}
\section*{Half-Life and the Rate of Radioactive Decay}
Unstable nuclei decay. However, some nuclides decay faster than others. For example, radium and polonium, discovered by Marie and Pierre Curie, decay faster than uranium. That means they have shorter lifetimes, producing a greater rate of decay. Here we will explore half-life and activity, the quantitative terms for lifetime and rate of decay.

Why do we use the term like half-life rather than lifetime? The answer can be found by examining Figure 22.24, which shows how the number of radioactive nuclei in a sample decreases with time. The time in which half of the original number of nuclei decay is defined as the half-life, \(t_{\frac{1}{2}}\). After one half-life passes, half of the remaining nuclei will decay in the next half-life. Then, half of that amount in turn decays in the following half-life. Therefore, the number of radioactive nuclei decreases from \(N\) to \(N / 2\) in one half-life, to \(N / 4\) in the next, to \(N / 8\) in the next, and so on. Nuclear decay is an example of a purely statistical process.

\section*{Tips For Success}
A more precise definition of half-life is that each nucleus has a 50 percent chance of surviving for a time equal to one half-life. If an individual nucleus survives through that time, it still has a 50 percent chance of surviving through another half-life. Even if it happens to survive hundreds of half-lives, it still has a 50 percent chance of surviving through one more. Therefore, the decay of a nucleus is like random coin flipping. The chance of heads is 50 percent, no matter what has happened before.

The probability concept aligns with the traditional definition of half-life. Provided the number of nuclei is reasonably large, half of the original nuclei should decay during one half-life period.

\section*{Teacher Support}
Teacher Support [BL]Prepare a few other examples of exponential decay so that students understand the concept of half-life. Atmospheric pressure above sea level or temperature difference between objects, for example, both show exponential decay. Show two different rates of decay for the same scenario so that students have another example of activity.

\begin{figure}[h]
\begin{center}
  \includegraphics[max width=\textwidth]{6c6af082-56ef-4ff6-b696-4e9ed0d1965b-45}
\captionsetup{labelformat=empty}
\caption{Figure 22.24 Radioactive decay reduces the number of radioactive nuclei over time. In one half-life ( \(t_{\frac{1}{2}}\) ), the number decreases to half of its original value. Half of what remains decays in the next half-life, and half of that in the next,}
\end{center}
\end{figure}

and so on. This is exponential decay, as seen in the graph of the number of nuclei present as a function of time.

The following equation gives the quantitative relationship between the original number of nuclei present at time zero ( \(N_{O}\) ) and the number ( \(N\) ) at a later time \(t\)\\
\(N=N_{O} e^{-\lambda t}\),\\
22.45\\
where \(e=2.71828 \ldots\) is the base of the natural logarithm, and \(\lambda\) is the decay constant for the nuclide. The shorter the half-life, the larger is the value of \(\lambda\), and the faster the exponential \(e^{-\lambda t}\) decreases with time. The decay constant can be found with the equation\\
\(\lambda=\frac{\ln (2)}{t_{1 / 2}} \approx \frac{0.693}{t_{1 / 2}}\).\\
22.46

\section*{Activity, the Rate of Decay}
What do we mean when we say a source is highly radioactive? Generally, it means the number of decays per unit time is very high. We define activity \(R\) to be the rate of decay expressed in decays per unit time. In equation form, this is\\
\(R=\frac{\Delta N}{\Delta t}\),\\
22.47\\
where \(\Delta N\) is the number of decays that occur in time \(\Delta t\).\\
Activity can also be determined through the equation\\
\(R=\lambda N\),\\
22.48\\
which shows that as the amount of radiative material ( \(N\) ) decreases, the rate of decay decreases as well.

The SI unit for activity is one decay per second and it is given the name becquerel \((\mathrm{Bq})\) in honor of the discoverer of radioactivity. That is,\\
\(1 \mathrm{~Bq}=1\) decay \(/\) second .\\
Activity \(R\) is often expressed in other units, such as decays per minute or decays per year. One of the most common units for activity is the curie ( Ci ), defined to be the activity of 1 g of \({ }^{226} \mathrm{Ra}\), in honor of Marie Curie's work with radium. The definition of the curie is\\
\(1 \mathrm{Ci}=3.70 \times 10^{10} \mathrm{~Bq}\),\\
22.49\\
or \(3.70 \times 10^{10}\) decays per second.

\section*{Radiometric Dating}
\section*{Teacher Support}
Teacher Support [AL]Show that carbon-14 can create nitrogen-14 when struck by neutrino in the atmosphere.

Radioactive dating or radiometric dating is a clever use of naturally occurring radioactivity. Its most familiar application is carbon-14 dating. Carbon-14 is an isotope of carbon that is produced when solar neutrinos strike \({ }^{14} \mathrm{~N}\) particles within the atmosphere. Radioactive carbon has the same chemistry as stable carbon, and so it mixes into the biosphere, where it is consumed and becomes part of every living organism. Carbon-14 has an abundance of 1.3 parts per trillion of normal carbon, so if you know the number of carbon nuclei in an object (perhaps determined by mass and Avogadro's number), you can multiply that number by \(1.3 \times 10^{-12}\) to find the number of \({ }^{14} \mathrm{C}\) nuclei within the object. Over time, carbon-14 will naturally decay back to \({ }^{14} \mathrm{~N}\) with a half-life of 5,730 years (note that this is an example of beta decay). When an organism dies, carbon exchange with the environment ceases, and \({ }^{14} \mathrm{C}\) is not replenished. By comparing the abundance of \({ }^{14} \mathrm{C}\) in an artifact, such as mummy wrappings, with the normal abundance in living tissue, it is possible to determine the artifact's age (or time since death). Carbon-14 dating can be used for biological tissues as old as 50 or 60 thousand years, but is most accurate for younger samples, since the abundance of \({ }^{14} \mathrm{C}\) nuclei in them is greater.

One of the most famous cases of carbon-14 dating involves the Shroud of Turin, a long piece of fabric purported to be the burial shroud of Jesus (see Figure 22.25 ). This relic was first displayed in Turin in 1354 and was denounced as a fraud at that time by a French bishop. Its remarkable negative imprint of an apparently crucified body resembles the then-accepted image of Jesus. As a result, the relic has been remained controversial throughout the centuries. Carbon-14 dating was not performed on the shroud until 1988, when the process had been refined to the point where only a small amount of material needed to be destroyed. Samples were tested at three independent laboratories, each being given four pieces of cloth, with only one unidentified piece from the shroud, to avoid prejudice. All three laboratories found samples of the shroud contain 92 percent of the \({ }^{14} \mathrm{C}\) found in living tissues, allowing the shroud to be dated (see How Old is the Shroud of Turin?).

\begin{figure}[h]
\begin{center}
  \includegraphics[max width=\textwidth]{6c6af082-56ef-4ff6-b696-4e9ed0d1965b-48}
\captionsetup{labelformat=empty}
\caption{Figure 22.25 Part of the Shroud of Turin, which shows a remarkable negative imprint likeness of Jesus complete with evidence of crucifixion wounds. The shroud first surfaced in the 14th century and was only recently carbon-14 dated. It has not been determined how the image was placed on the material. (credit: Butko, Wikimedia Commons)}
\end{center}
\end{figure}

\section*{Worked Example}
Carbon-11 Decay Carbon-11 has a half-life of 20.334 min . (a) What is the decay constant for carbon-11?

If 1 kg of carbon- 11 sample exists at the beginning of an hour, (b) how much material will remain at the end of the hour and (c) what will be the decay activity at that time?

\section*{Strategy}
Since \(N_{O}\) refers to the amount of carbon-11 at the start, then after one halflife, the amount of carbon- 11 remaining will be \(N_{O} / 2\). The decay constant is equivalent to the probability that a nucleus will decay each second. As a result,\\
the half-life will need to be converted to seconds.\\
Solution\\
(a)\\
\(N=N_{O} e^{-\lambda t}\)\\
22.50

Since half of the carbon-11 remains after one half-life, \(N / N_{O}=0.5\).\\
\(0.5=e^{-\lambda t}\)\\
22.51

Take the natural logarithm of each side to isolate the decay constant.\\
\(\ln (0.5)=-\lambda t\)\\
22.52

Convert the 20.334 min to seconds.

\[
\begin{aligned}
& -0.693=(-\lambda)(20.334 \mathrm{~min})\left(\frac{60 \mathrm{~s}}{1 \mathrm{~min}}\right) \\
& -0.693=(-\lambda)(1,220.04 \mathrm{~s}) \\
& \frac{-0.693}{1,220.04 \mathrm{~s}}=-\lambda \\
& \lambda=5.68 \times 10^{-4} \mathrm{~s}^{-1}
\end{aligned}
\]

22.53\\
(b) The amount of material after one hour can be found by using the equation\\
\(N=N_{O} e^{-\lambda t}\),\\
22.54\\
with \(t\) converted into seconds and \(N_{O}\) written as \(1,000 \mathrm{~g}\)\\
\(N=(1,000 \mathrm{~g}) e-(0.000568)(60.60)\)\\
\(N=129.4 g\)\\
22.55\\
(c) The decay activity after one hour can be found by using the equation\\
\(R=\lambda N\)\\
22.56\\
for the mass value after one hour.\\
\(R=\lambda N=\left(0.000568 \frac{\text { decays }}{\text { second }}\right)(129.4\) grams \()=0.0735 \mathrm{~Bq}\)\\
22.57

\section*{Discussion}
(a) The decay constant shows that 0.0568 percent of the nuclei in a carbon- 11 sample will decay each second. Another way of considering the decay constant is that a given carbon-11 nuclei has a 0.0568 percent probability of decaying each second. The decay of carbon-11 allows it to be used in positron emission topography (PET) scans; however, its 20.334 min half-life does pose challenges for its administration.\\
(b) One hour is nearly three full half-lives of the carbon-11 nucleus. As a result, one would expect the amount of sample remaining to be approximately one eighth of the original amount. The 129.4 g remaining is just a bit larger than one-eighth, which is sensible given a half-life of just over 20 min .\\
(c) Label analysis shows that the unit of Becquerel is sensible, as there are 0.0735 g of carbon-11 decaying each second. That is smaller amount than at the beginning of the hour, when \(R=\left(0.000568 \frac{\text { decay }}{\mathrm{s}}\right)(1,000 \mathrm{~g})=0.568 \mathrm{~g}\) of carbon-11 were decaying each second.

\section*{Worked Example}
How Old is the Shroud of Turin? Calculate the age of the Shroud of Turin given that the amount of \({ }^{14} \mathrm{C}\) found in it is 92 percent of that in living tissue.

\section*{Strategy}
Because 92 percent of the \({ }^{14} \mathrm{C}\) remains, \(N / N_{O}=0.92\). Therefore, the equation \(N=N_{O} e^{-\lambda t}\) can be used to find \(\lambda t\). We also know that the half-life of \({ }^{14} \mathrm{C}\) is 5,730 years, and so once \(\lambda t\) is known, we can find \(\lambda\) and then find \(t\) as requested. Here, we assume that the decrease in \({ }^{14} \mathrm{C}\) is solely due to nuclear decay.

Solution\\
Solving the equation \(N=N_{O} e^{-\lambda t}\) for \(N / N_{O}\) gives\\
\(\frac{N}{N_{O}}=e^{-\lambda t}\).\\
22.58

Thus,\\
\(0.92=e^{-\lambda t}\).\\
22.59

Taking the natural logarithm of both sides of the equation yields\\
\(\ln 0.92=-\lambda t\)\\
22.60\\
so that\\
\(-0.0834=-\lambda t\).\\
22.61

Rearranging to isolate \(t\) gives\\
\(t=\frac{0.0834}{\lambda}\).\\
22.62

Now, the equation \(\lambda=\frac{0.693}{t_{1 / 2}}\) can be used to find \(\lambda\) for \({ }^{14} \mathrm{C}\). Solving for \(\lambda\) and substituting the known half-life gives\\
\(\lambda=\frac{0.693}{t_{1 / 2}}=\frac{0.693}{5,730 \text { years }}=1.21 \times 10^{-4} \mathrm{y}^{-1}\).\\
22.63

We enter that value into the previous equation to find \(t\).\\
\(t=\frac{0.0834}{1.21 \times 10^{-4}}=690\) years.\\
22.64

Discussion\\
This dates the material in the shroud to 1988-690 = 1300. Our calculation is only accurate to two digits, so that the year is rounded to 1300 . The values obtained at the three independent laboratories gave a weighted average date of \(1320 \pm 60\). That uncertainty is typical of carbon-14 dating and is due to the small amount of 14 C in living tissues, the amount of material available, and experimental uncertainties (reduced by having three independent measurements). That said, is it notable that the carbon-14 date is consistent with the first record of the shroud's existence and certainly inconsistent with the period in which Jesus lived.

There are other noncarbon forms of radioactive dating. Rocks, for example, can sometimes be dated based on the decay of \({ }^{238} \mathrm{U}\). The decay series for \({ }^{238} \mathrm{U}\) ends with \({ }^{206} \mathrm{~Pb}\), so the ratio of those nuclides in a rock can be used an indication of how long it has been since the rock solidified. Knowledge of the \({ }^{238} \mathrm{U}\) half-life has shown, for example, that the oldest rocks on Earth solidified about \(3.5 \times 10^{9}\) years ago.

\section*{Virtual Physics}
Radioactive Dating Game Click to view content\\
Learn about different types of radiometric dating, such as carbon dating. Understand how decay and half-life work to enable radiometric dating to work. Play a game that tests your ability to match the percentage of the dating element that remains to the age of the object.

\subsection*{22.4 Nuclear Fission and Fusion}
\section*{Section Learning Objectives}
By the end of this section, you will be able to do the following:

\begin{itemize}
  \item Explain nuclear fission
  \item Explain nuclear fusion
  \item Describe how the processes of fission and fusion work in nuclear weapons and in generating nuclear power
\end{itemize}

\section*{Teacher Support}
Teacher Support The learning objectives in this section will help your students master the following standards:

\begin{itemize}
  \item (8) Science concepts. The student knows simple examples of atomic, nuclear, and quantum phenomena. The student is expected to:
  \item (C) describe the significance of mass-energy equivalence and apply it in explanations of phenomena such as nuclear stability, fission, and fusion.
\end{itemize}

\section*{Section Key Terms}
The previous section dealt with naturally occurring nuclear decay. Without human intervention, some nuclei will change composition in order to achieve a stable equilibrium. This section delves into a less-natural process. Knowing that energy can be emitted in various forms of nuclear change, is it possible to create a nuclear reaction through our own intervention? The answer to this question is yes. Through two distinct methods, humankind has discovered multiple ways of manipulating the atom to release its internal energy.

\section*{Nuclear Fission}
In simplest terms, nuclear fission is the splitting of an atomic bond. Given that it requires great energy separate two nucleons, it may come as a surprise to learn that splitting a nucleus can release vast potential energy. And although it is true that huge amounts of energy can be released, considerable effort is needed to do so in practice.

An unstable atom will naturally decay, but it may take millions of years to do so. As a result, a physical catalyst is necessary to produce useful energy through nuclear fission. The catalyst typically occurs in the form of a free neutron, projected directly at the nucleus of a high-mass atom.

\section*{Teacher Support}
Teacher Support [BL][OL] To ensure understanding, ask students why it is not likely that a stable atom would naturally decay.

As shown in Figure 22.26, a neutron strike can cause the nucleus to elongate, much like a drop of liquid water. This is why the model is known as the liquid drop model. As the nucleus elongates, nucleons are no longer so tightly packed, and the repulsive electromagnetic force can overcome the short-range strong nuclear force. The imbalance of forces can result in the two ends of the drop flying apart, with some of the nuclear binding energy released to the surroundings.\\
\includegraphics[max width=\textwidth, center]{6c6af082-56ef-4ff6-b696-4e9ed0d1965b-53}

Figure 22.26 Neutron-induced fission is shown. First, energy is put into a large nucleus when it absorbs a neutron. Acting like a struck liquid drop, the nucleus deforms and begins to narrow in the middle. Since fewer nucleons are in contact, the repulsive Coulomb force is able to break the nucleus into two parts with some neutrons also flying away.

As you can imagine, the consequences of the nuclei splitting are substantial. When a nucleus is split, it is not only energy that is released, but a small number of neutrons as well. Those neutrons have the potential to cause further fission in other nuclei, especially if they are directed back toward the other nuclei by a dense shield or neutron reflector (see part (d) of Figure 22.26).

\section*{Teacher Support}
Teacher Support [BL][OL][AL]At this point, it is a good idea to show a quick video of a chain reaction model. Good videos of a ping-pong ball dropped into a room full of ping-pong balls and mousetraps elucidate this idea very well and can be easily found online. One such video can be found here. Explain that if the plastic box were not there, the uncontrolled chain reaction would likely not occur. It is analogous to a dense shield or neutron reflector directing neutrons back to interact with more other nuclei and perpetuate the fission chain reaction.

However, not every neutron produced by fission induces further fission. Some neutrons escape the fissionable material, while others interact with a nucleus without making it split. We can enhance the number of fissions produced by neutrons by having a large amount of fissionable material as well as a neutron reflector. The minimum amount necessary for self-sustained fission of a given nuclide is called its critical mass. Some nuclides, such as \({ }^{239} \mathrm{Pu}\), produce more neutrons per fission than others, such as \({ }^{235}\) U. Additionally, some nuclides are easier to make fission than others. In particular, \({ }^{235} \mathrm{U}\) and \({ }^{239} \mathrm{Pu}\) are easier to fission than the much more abundant \({ }^{238} \mathrm{U}\). Both factors affect critical mass, which is smallest for \({ }^{239} \mathrm{Pu}\). The self-sustained fission of nuclei is commonly referred to as a chain reaction, as shown in Figure 22.27.\\
\includegraphics[max width=\textwidth, center]{6c6af082-56ef-4ff6-b696-4e9ed0d1965b-54}

Figure 22.27 A chain reaction can produce self-sustained fission if each fission produces enough neutrons to induce at least one more fission. This depends on several factors, including how many neutrons are produced in an average fission and how easy it is to make a particular type of nuclide fission.

A chain reaction can have runaway results. If each atomic split results in two nuclei producing a new fission, the number of nuclear reactions will increase exponentially. One fission will produce two atoms, the next round of fission will create four atoms, the third round eight atoms, and so on. Of course, each time fission occurs, more energy will be emitted, further increasing the power of the atomic reaction. And that is just if two neutrons create fission reactions each round. Perhaps you can now see why so many people consider atomic energy to be an exciting energy source!\\
To make a self-sustained nuclear fission reactor with \({ }^{235} \mathrm{U}\), it is necessary to slow down the neutrons. Water is very effective at this, since neutrons collide with protons in water molecules and lose energy. Figure 22.28 shows a schematic of a reactor design called the press ri ed ater reactor

\begin{figure}[h]
\begin{center}
  \includegraphics[max width=\textwidth]{6c6af082-56ef-4ff6-b696-4e9ed0d1965b-55}
\captionsetup{labelformat=empty}
\caption{Figure 22.28 A pressurized water reactor is cleverly designed to control the fission of large amounts of \({ }^{235} \mathrm{U}\), while using the heat produced in the fission reaction to create steam for generating electrical energy. Control rods adjust neutron flux so that it is self-sustaining. In case the reactor overheats and boils the water away, the chain reaction terminates, because water is needed to slow down the neutrons. This inherent safety feature can be overwhelmed in extreme circumstances.}
\end{center}
\end{figure}

Control rods containing nuclides that very strongly absorb neutrons are used to adjust neutron flux. To produce large amounts of power, reactors contain hundreds to thousands of critical masses, and the chain reaction easily becomes self-sustaining. Neutron flux must be carefully regulated to avoid an out-ofcontrol exponential increase in the rate of fission.

Control rods help prevent overheating, perhaps even a meltdown or explosive disassembly. The water that is used to slow down neutrons, necessary to get them to induce fission in \({ }^{235} \mathrm{U}\), and achieve criticality, provides a negative feedback for temperature increase. In case the reactor overheats and boils the water to steam or is breached, the absence of water kills the chain reaction. Considerable heat, however, can still be generated by the reactor's radioactive fission products. Other safety features, thus, need to be incorporated in the event of a loss of coolant accident, including auxiliary cooling water and pumps.

Energies in Nuclear Fission The following are two interesting facts to consider:

\begin{itemize}
  \item The average fission reaction produces 200 MeV of energy.
  \item If you were to measure the mass of the products of a nuclear reaction, you would find that their mass was slightly less than the mass of the original nucleus.
\end{itemize}

How are those things possible? Doesn't the fission reaction's production of energy violate the conservation of energy? Furthermore, doesn't the loss in mass in the reaction violate the conservation of mass? Those are important questions, and they can both be answered with one of the most famous equations in scientific history.\\
\(E=m c^{2}\)\\
22.65

Recall that, according to Einstein's theory, energy and mass are essentially the same thing. In the case of fission, the mass of the products is less than that of the reactants because the missing mass appears in the form of the energy released in the reaction, with a constant value of \(c^{2}\) Joules of energy converted for each kilogram of material. The value of \(c^{2}\) is substantial-from Einstein's equation, the amount of energy in just 1 gram of mass would be enough to support the average U.S. citizen for more than 270 years! The example below will show you how a mass-energy transformation of this type takes place.

\section*{Worked Example}
Calculating Energy from a Kilogram of Fissionable Fuel Calculate the amount of energy produced by the fission of 1.00 kg of \({ }^{235} \mathrm{U}\), given the average fission reaction of\\
\({ }^{235} \mathrm{U}\) produces 200 MeV .

\section*{Strategy}
The total energy produced is the number of \({ }^{235} \mathrm{U}\) atoms times the given energy per \({ }^{235} \mathrm{U}\) fission. We should therefore find the number of \({ }^{235} \mathrm{U}\) atoms in 1.00 kg .

Solution\\
The number of \({ }^{235} \mathrm{U}\) atoms in 1.00 kg is Avogadro's number times the number of moles. One mole of \({ }^{235} \mathrm{U}\) has a mass of 235.04 g ; thus, there are \((1,000 \mathrm{~g}) /(235.04 .00 \mathrm{~g} / \mathrm{mol})=4.25 \mathrm{~mol}\). The number of \({ }^{235} \mathrm{U}\) atoms is therefore\\
\((4.25 \mathrm{~mol})\left(6.02 \times 10^{23} \mathrm{U} / \mathrm{mol}\right)=2.56 \times 10^{24}\) atoms of \({ }^{235} \mathrm{U}\).\\
So the total energy released is\\
\(E=\left(2.56 \times 10^{24}\right)\left(\frac{200 \mathrm{MeV}}{235 \mathrm{U}}\right)\left(\frac{1.60 \times 10^{-13} \mathrm{~J}}{\mathrm{MeV}}\right)=8.21 \times 10^{13} \mathrm{~J}\).\\
22.67

Discussion\\
The result is another impressively large amount of energy, equivalent to about 14,000 barrels of crude oil or 600,000 gallons of gasoline. But, it is only one fourth the energy produced by the fusion of a kilogram of a mixture of deuterium and tritium. Even though each fission reaction yields about ten times the energy of a fusion reaction, the energy per kilogram of fission fuel is less, because there are far fewer moles per kilogram of the heavy nuclides. Fission fuel is also much scarcer than fusion fuel, and less than 1 percent of uranium (the 235 U ) is readily usable.

\section*{Virtual Physics}
Nuclear Fission Click to view content\\
Start a chain reaction, or introduce nonradioactive isotopes to prevent one. Use the applet to control energy production in a nuclear reactor!

\section*{Nuclear Fusion}
Nuclear fusion is defined as the combining, or fusing, of two nuclei and, the combining of nuclei also results in an emission of energy. For many, the concept is counterintuitive. After all, if energy is released when a nucleus is split, how can it also be released when nucleons are combined together? The difference between fission and fusion, which results from the size of the nuclei involved, will be addressed next.

\section*{Teacher Support}
Teacher Support The reading in this subsection is dense. Prepare your students to tread slowly through this material.

Remember that the structure of a nucleus is based on the interplay of the compressive nuclear strong force and the repulsive electromagnetic force. For nuclei that are less massive than iron, the nuclear force is actually stronger than that of the Coulomb force. As a result, when a low-mass nucleus absorbs nucleons, the added neutrons and protons bind the nucleus more tightly. The increased nuclear strong force does work on the nucleus, and energy is released.

Once the size of the created nucleus exceeds that of iron, the short-ranging nuclear force does not have the ability to bind a nucleus more tightly, and the emission of energy ceases. In fact, for fusion to occur for elements of greater mass than iron, energy must be added to the system! Figure 22.29 shows an energy-mass curve commonly used to describe nuclear reactions. Notice the location of iron (Fe) on the graph. All low-mass nuclei to the left of iron release energy through fusion, while all high-mass particles to the right of iron produce energy through fission.

\begin{figure}[h]
\begin{center}
  \includegraphics[max width=\textwidth]{6c6af082-56ef-4ff6-b696-4e9ed0d1965b-58}
\captionsetup{labelformat=empty}
\caption{Figure 22.29 Fusion of light nuclei to form medium-mass nuclei converts mass to energy, because binding energy per nucleon ( \(\mathrm{BE} / A\) ) is greater for the product nuclei. The larger \(\mathrm{BE} / A\) is, the less mass per nucleon, and so mass is converted to energy and released in such fusion reactions.}
\end{center}
\end{figure}

\section*{Tips For Success}
Just as it is not possible for the elements to the left of iron in the figure to naturally fission, it is not possible for elements to the right of iron to naturally\\
undergo fusion, as that process would require the addition of energy to occur. Furthermore, notice that elements commonly discussed in fission and fusion are elements that can provide the greatest change in binding energy, such as uranium and hydrogen.

Iron's location on the energy-mass curve is important, and explains a number of its characteristics, including its role as an elemental endpoint in fusion reactions in stars.

The major obstruction to fusion is the Coulomb repulsion force between nuclei. Since the attractive nuclear force that can fuse nuclei together is short ranged, the repulsion of like positive charges must be overcome in order to get nuclei close enough to induce fusion. Figure 22.30 shows an approximate graph of the potential energy between two nuclei as a function of the distance between their centers. The graph resembles a hill with a well in its center. A ball rolled to the left must have enough kinetic energy to get over the hump before it falls into the deeper well with a net gain in energy. So it is with fusion. If the nuclei are given enough kinetic energy to overcome the electric potential energy due to repulsion, then they can combine, release energy, and fall into a deep well. One way to accomplish that end is to heat fusion fuel to high temperatures so that the kinetic energy of thermal motion is sufficient to get the nuclei together.

\section*{Teacher Support}
Teacher Support For students struggling with the analogy, there is a popular carnival game with a bowling ball, called Roller Bowler, that models the potential energy well affectively. Some of them may find the bowling analogy useful.

\begin{figure}[h]
\begin{center}
  \includegraphics[max width=\textwidth]{6c6af082-56ef-4ff6-b696-4e9ed0d1965b-59}
\captionsetup{labelformat=empty}
\caption{Figure 22.30 Potential energy between two light nuclei graphed as a function of distance between them. If the nuclei have enough kinetic energy to get over the Coulomb repulsion hump, they combine, release energy, and drop into a deep attractive well.}
\end{center}
\end{figure}

You might think that, in our Sun, nuclei are constantly coming into contact and fusing. However, this is only partially true. Only at the Sun's core are the particles close enough and the temperature high enough for fusion to occur!

\section*{Teacher Support}
\section*{Teacher Support}
\begin{itemize}
  \item For the following proton-proton cycle, have students check to make sure that mass number and charge are conserved for each of the four equations. For each equation, ask how energy is emitted. It is important to move slowly through this section.
  \item Students will likely be unfamiliar with the terms positron and electron ne trina For now, express the positron as positi e electron (not a proton) and the electron neutrino as a massless electron.
\end{itemize}

In the series of reactions below, the Sun produces energy by fusing protons, or hydrogen nuclei ( \({ }^{1} \mathrm{H}\), by far the Sun's most abundant nuclide) into helium nuclei \({ }^{4} \mathrm{He}\). The principal sequence of fusion reactions forms what is called the proton-proton cycle\\
\({ }^{1} \mathrm{H}+{ }^{1} \mathrm{H} \rightarrow{ }^{2} \mathrm{H}+e^{+}+v_{e}\)\\
\({ }^{1} \mathrm{H}+{ }^{2} \mathrm{H} \rightarrow{ }^{3} \mathrm{He}+\gamma\)\\
\({ }^{3} \mathrm{He}+{ }^{3} \mathrm{He} \rightarrow{ }^{4} \mathrm{He}+{ }^{1} \mathrm{H}+{ }^{1} \mathrm{H}(12.86 \mathrm{MeV})\),\\
( 0.42 MeV )\\
( 5.49 MeV )\\
( 12.86 MeV ),\\
22.68\\
where \(e^{+}\)stands for a positron and \(v_{e}\) is an electron neutrino. The energy in parentheses is released by the reaction. Note that the first two reactions must occur twice for the third to be possible, so the cycle consumes six protons ( \({ }^{1} \mathrm{H}\) ) but gives back two. Furthermore, the two positrons produced will find two electrons and annihilate to form four more \(\gamma\) rays, for a total of six. The overall cycle is thus\\
\(2 e-+4{ }^{1} \mathrm{H} \rightarrow 4 \mathrm{He}+2 v_{e}+6 \gamma \quad(26.7 \mathrm{MeV})\),\\
where the 26.7 MeV includes the annihilation energy of the positrons and electrons and is distributed among all the reaction products. The solar interior is dense, and the reactions occur deep in the Sun where temperatures are highest. It takes about 32,000 years for the energy to diffuse to the surface and radiate away. However, the neutrinos can carry their energy out of the Sun in less than two seconds, because they interact so weakly with other matter. Negative feedback in the Sun acts as a thermostat to regulate the overall energy output. For instance, if the interior of the Sun becomes hotter than normal, the reaction rate increases, producing energy that expands the interior. The expansion cools it and lowers the reaction rate. Conversely, if the interior becomes too cool, it contracts, increasing the temperature and therefore the reaction rate (see Figure 22.31). Stars like the Sun are stable for billions of years, until a significant fraction of their hydrogen has been depleted.

\begin{figure}[h]
\begin{center}
  \includegraphics[max width=\textwidth]{6c6af082-56ef-4ff6-b696-4e9ed0d1965b-61}
\captionsetup{labelformat=empty}
\caption{Figure 22.31 Nuclear fusion in the Sun converts hydrogen nuclei into helium; fusion occurs primarily at the boundary of the helium core, where the temperature is highest and sufficient hydrogen remains. Energy released diffuses slowly to the surface, with the exception of neutrinos, which escape immediately. Energy production remains stable because of negative-feedback effects.}
\end{center}
\end{figure}

\section*{Nuclear Weapons and Nuclear Power}
The world was in political turmoil when fission was discovered in 1938. Compounding the troubles, the possibility of a self-sustained chain reaction was immediately recognized by leading scientists the world over. The enormous energy known to be in nuclei, but considered inaccessible, now seemed to be available on a large scale.

Within months after the announcement of the discovery of fission, Adolf Hitler banned the export of uranium from newly occupied Czechoslovakia. It seemed that the possible military value of uranium had been recognized in Nazi Germany, and that a serious effort to build a nuclear bomb had begun.

Alarmed scientists, many of whom fled Nazi Germany, decided to take action. None was more famous or revered than Einstein. It was felt that his help was needed to get the American government to make a serious effort at constructing nuclear weapons as a matter of survival. Leo Szilard, a Hungarian physicist who had emigrated to America, took a draft of a letter to Einstein, who, although a pacifist, signed the final version. The letter was for President Franklin Roosevelt, warning of the German potential to build extremely powerful bombs of a new type. It was sent in August of 1939, just before the German invasion of Poland that marked the start of World War II.

It was not until December 6, 1941, the day before the Japanese attack on Pearl Harbor, that the United States made a massive commitment to building a nuclear bomb. The top secret Manhattan Project was a crash program aimed at beating the Germans. It was carried out in remote locations, such as Los Alamos, New Mexico, whenever possible, and eventually came to cost billions of dollars and employ the efforts of more than 100,000 people. J. Robert Oppen-\\
heimer (1904-1967), a talented physicist, was chosen to head the project. The first major step was made by Enrico Fermi and his group in December 1942, when they completed the first self-sustaining nuclear reactor. This first atomic pile, built in a squash court at the University of Chicago, proved that a fission chain reaction was possible.

Plutonium was recognized as easier to fission with neutrons and, hence, a superior fission material very early in the Manhattan Project. Plutonium availability was uncertain, and so a uranium bomb was developed simultaneously. Figure 22.32 shows a gun-type bomb, which takes two subcritical uranium masses and shoots them together. To get an appreciable yield, the critical mass must be held together by the explosive charges inside the cannon barrel for a few microseconds. Since the buildup of the uranium chain reaction is relatively slow, the device to bring the critical mass together can be relatively simple. Owing to the fact that the rate of spontaneous fission is low, a neutron source is at the center the assembled critical mass.

\begin{figure}[h]
\begin{center}
  \includegraphics[max width=\textwidth]{6c6af082-56ef-4ff6-b696-4e9ed0d1965b-62}
\captionsetup{labelformat=empty}
\caption{Figure 22.32 A gun-type fission bomb for \({ }^{235} \mathrm{U}\) utilizes two subcritical masses forced together by explosive charges inside a cannon barrel. The energy yield depends on the amount of uranium and the time it can be held together before it disassembles itself.}
\end{center}
\end{figure}

Plutonium's special properties necessitated a more sophisticated critical mass assembly, shown schematically in Figure 22.33. A spherical mass of plutonium is surrounded by shaped charges (high explosives that focus their blast) that implode the plutonium, crushing it into a smaller volume to form a critical mass. The implosion technique is faster and more effective, because it compresses threedimensionally rather than one-dimensionally as in the gun-type bomb. Again, a neutron source is included to initiate the chain reaction.

\begin{figure}[h]
\begin{center}
  \includegraphics[max width=\textwidth]{6c6af082-56ef-4ff6-b696-4e9ed0d1965b-63}
\captionsetup{labelformat=empty}
\caption{Figure 22.33 An implosion created by high explosives compresses a sphere of \({ }^{239} \mathrm{Pu}\) into a critical mass. The superior fissionability of plutonium has made it the preferred bomb material.}
\end{center}
\end{figure}

Owing to its complexity, the plutonium bomb needed to be tested before there could be any attempt to use it. On July 16, 1945, the test named Trinity was conducted in the isolated Alamogordo Desert in New Mexico, about 200 miles south of Los Alamos (see Figure 22.34). A new age had begun. The yield of the Trinity device was about 10 kilotons (kT), the equivalent of 5,000 of the largest conventional bombs.

\begin{figure}[h]
\begin{center}
  \includegraphics[max width=\textwidth]{6c6af082-56ef-4ff6-b696-4e9ed0d1965b-64}
\captionsetup{labelformat=empty}
\caption{Figure 22.34 Trinity test (1945), the first nuclear bomb (credit: U.S. Department of Energy)}
\end{center}
\end{figure}

Although Germany surrendered on May 7, 1945, Japan had been steadfastly refusing to surrender for many months, resulting large numbers of civilian and military casualties. Invasion plans by the Allies estimated a million casualties of their own and untold losses of Japanese lives. The bomb was viewed as a way to end the war. The first bomb used was a gun-type uranium bomb dropped on Hiroshima on August 6 by the United States. Its yield of about 15 kT destroyed the city and killed an estimated 80,000 people, with 100,000 more being seriously injured. The second bomb was an implosion-type plutonium bomb dropped on Nagasaki only three days later. Its \(20-\mathrm{kT}\) yield killed at least 50,000 people, something less than Hiroshima because of the hilly terrain and the fact that it was a few kilometers off target. The Japanese were told that one bomb a week\\
would be dropped until they surrendered unconditionally, which they did on August 14. In actuality, the United States had only enough plutonium for one more bomb, as yet unassembled.

\section*{Teacher Support}
Teacher Support [BL][OL] [AL]Note that the hydrogen bomb is a fusion bomb, as its energy can only be released by combining multiple hydrogen nuclei together.

Knowing that fusion produces several times more energy per kilogram of fuel than fission, some scientists pursued the idea of constructing a fusion bomb. The first such bomb was detonated by the United States several years after the first fission bombs, on October 31, 1952, at Eniwetok Atoll in the Pacific Ocean. It had a yield of 10 megatons (MT), about 670 times that of the fission bomb that destroyed Hiroshima. The Soviet Union followed with a fusion device of its own in August 1953, and a weapons race, beyond the aim of this text to discuss, continued until the end of the Cold War.

Figure 22.35 shows a simple diagram of how a thermonuclear bomb is constructed. A fission bomb is exploded next to fusion fuel in the solid form of lithium deuteride. Before the shock wave blows it apart, \(\gamma\) rays heat and compress the fuel, and neutrons create tritium through the reaction \(n+{ }^{6} \mathrm{Li} \rightarrow{ }^{3} \mathrm{H}+{ }^{4} \mathrm{He}\). Additional fusion and fission fuels are enclosed in a dense shell of \({ }^{238} \mathrm{U}\). At the same time that the uranium shell reflects the neutrons back into the fuel to enhance its fusion, the fast-moving neutrons cause the plentiful and inexpensive \({ }^{238} \mathrm{U}\) to fission, part of what allows thermonuclear bombs to be so large.

\begin{figure}[h]
\begin{center}
  \includegraphics[max width=\textwidth]{6c6af082-56ef-4ff6-b696-4e9ed0d1965b-66}
\captionsetup{labelformat=empty}
\caption{Figure 22.35 This schematic of a fusion bomb (H-bomb) gives some idea of how the \({ }^{239} \mathrm{Pu}\) fission trigger is used to ignite fusion fuel. Neutrons and rays transmit energy to the fusion fuel, create tritium from deuterium, and heat and compress the fusion fuel. The outer shell of \({ }^{238} \mathrm{U}\) serves to reflect some neutrons back into the fuel, causing more fusion, and it boosts the energy output by fissioning itself when neutron energies become high enough.}
\end{center}
\end{figure}

Of course, not all applications of nuclear physics are as destructive as the weapons described above. Hundreds of nuclear fission power plants around the world attest to the fact that controlled fission is both practical and economical. Given growing concerns over global warming, nuclear power is often seen as a viable alternative to energy derived from fossil fuels.

\section*{Boundless Physics}
Fusion Reactors For decades, fusion reactors have been deemed the energ of the \(f t\) re A safer, cleaner, and more abundant potential source of energy than its fission counterpart, images of the fusion reactor have been conjured up each time the need for a renewable, environmentally friendly resource is discussed. Now, after more than half a century of speculating, some scientists believe that fusion reactors are nearly here.

In creating energy by combining atomic nuclei, the fusion reaction holds many\\
advantages over fission. First, fusion reactions are more efficient, releasing 3 to 4 times more energy than fission per gram of fuel. Furthermore, unlike fission reactions that require hea elements like uranium that are difficult to obtain, fusion requires light elements that are abundant in nature. The greatest advantage of the fusion reaction, however, is in its ability to be controlled. While traditional nuclear reactors create worries about meltdowns and radioactive waste, neither is a substantial concern with the fusion reaction. Consider that fusion reactions require a large amount of energy to overcome the repulsive Coulomb force and that the byproducts of a fusion reaction are largely limited to helium nuclei.

In order for fusion to occur, hydrogen isotopes of deuterium and tritium must be acquired. While deuterium can easily be gathered from ocean water, tritium is slightly more difficult to come by, though it can be manufactured from Earth's abundant lithium. Once acquired, the hydrogen isotopes are injected into an empty vessel and subjected to temperature and pressure great enough to mimic the conditions at the core of our Sun. Using carefully controlled high-frequency radio waves, the hydrogen isotopes are broken into plasma and further controlled through an electromagnetic field. As the electromagnetic field continues to exert pressure on the hydrogen plasma, enough energy is supplied to cause the hydrogen plasma to fuse into helium.

\begin{figure}[h]
\begin{center}
  \includegraphics[max width=\textwidth]{6c6af082-56ef-4ff6-b696-4e9ed0d1965b-67}
\captionsetup{labelformat=empty}
\caption{Figure 22.36 Tokamak confinement of nuclear fusion plasma. The magnetic field lines are used to confine the high-temperature plasma (purple). Research is currently being done to increase the efficiency of the tokamak confinement model.}
\end{center}
\end{figure}

Once the plasma fuses, high-velocity neutrons are ejected from the newly formed helium atoms. Those high velocity neutrons, carrying the excess energy stored\\
within bonds of the original hydrogen, are able to travel unaffected by the applied magnetic field. In doing so, they strike a barrier around the nuclear reactor, transforming their excess energy to heat. The heat is then harvested to make steam that drives turbines. Hydrogen's tremendous power is now usable!

The historical concern with nuclear fusion reactors is that the energy required to control the electromagnetic field is greater than the energy harvested from the hydrogen atoms. However, recent research by both Lockheed Martin engineers and scientists at the Lawrence Livermore National Laboratory has yielded exciting theoretical improvements in efficiency. At the time of this writing, a test facility called ITER (International Thermonuclear Experimental Reactor) is being constructed in southern France. A joint venture of the European Union, the United States, Japan, Russia, China, South Korea, and India, ITER is designed for further study into the future of nuclear fusion energy production.

\subsection*{22.5 Medical Applications of Radioacti it : Diagnostic Imaging and Radiation}
\section*{Section Learning Objectives}
By the end of this section, you will be able to do the following:

\begin{itemize}
  \item Describe how nuclear imaging works (e.g., radioisotope imaging, PET)
  \item Describe the ionizing effects of radiation and how they can be used for medical treatment
\end{itemize}

\section*{Teacher Support}
Teacher Support The learning objectives in this section will help your students master the following standards:

\begin{itemize}
  \item (8) Science concepts. The student knows simple examples of atomic, nuclear, and quantum phenomena. The student is expected to:
  \item (C) describe the significance of mass-energy equivalence and apply it in explanations of phenomena such as nuclear stability, fission, and fusion.
\end{itemize}

\section*{Section Key Terms}
\section*{Medical Applications of Nuclear Physics}
Applications of nuclear physics have become an integral part of modern life. From the bone scan that detects one cancer to the radioiodine treatment that cures another, nuclear radiation has diagnostic and therapeutic effects on medicine.

Medical Imaging A host of medical imaging techniques employ nuclear radiation. What makes nuclear radiation so useful? First, \(\gamma\) radiation can easily penetrate tissue; hence, it is a useful probe to monitor conditions inside the body. Second, nuclear radiation depends on the nuclide and not on the chemical compound it is in, so that a radioactive nuclide can be put into a compound designed for specific purposes. When that is done, the compound is said to be tagged. A tagged compound used for medical purposes is called a radiopharmaceutical. Radiation detectors external to the body can determine the location and concentration of a radiopharmaceutical to yield medically useful information. For example, certain drugs are concentrated in inflamed regions of\\
the body, and their locations can aid diagnosis and treatment as seen in Figure 22.37. Another application utilizes a radiopharmaceutical that the body sends to bone cells, particularly those that are most active, to detect cancerous tumors or healing points. Images can then be produced of such bone scans. Clever use of radioisotopes determines the functioning of body organs, such as blood flow, heart muscle activity, and iodine uptake in the thyroid gland. For instance, a radioactive form of iodine can be used to monitor the thyroid, a radioactive thallium salt can be used to follow the blood stream, and radioactive gallium can be used for cancer imaging.

\section*{Teacher Support}
Teacher Support [AL]Have students investigate why radioactive iodine, radioactive thallium salt, and radioactive gallium are specifically used for their purposes.

\begin{figure}[h]
\begin{center}
  \includegraphics[max width=\textwidth]{6c6af082-56ef-4ff6-b696-4e9ed0d1965b-71}
\captionsetup{labelformat=empty}
\caption{Figure 22.37 A radiopharmaceutical was used to produce this brain image of a patient with Alzheimer's disease. Certain features are computer enhanced. (credit: National Institutes of Health)}
\end{center}
\end{figure}

\section*{Teacher Support}
Teacher Support Refer students to the explanation of how a scintillator works from earlier in this chapter.

Once a radioactive compound has been ingested, a device like that shown in Figure 22.38 is used to monitor nuclear activity. The device, called an Anger camera or gamma camera uses a piece of lead with holes bored through it. The gamma rays are redirected through the collimator to narrow their beam, and are then interpreted using a device called a scintillator. The computer analysis of detector signals produces an image. One of the disadvantages of this detection method is that there is no depth information (i.e., it provides a two-dimensional view of the tumor as opposed to a three-dimensional view), because radiation from any location under that detector produces a signal.

\begin{figure}[h]
\begin{center}
  \includegraphics[max width=\textwidth]{6c6af082-56ef-4ff6-b696-4e9ed0d1965b-72}
\captionsetup{labelformat=empty}
\caption{Figure 22.38 An Anger or gamma camera consists of a lead collimator and an array of detectors. Gamma rays produce light flashes in the scintillators. The light output is converted to an electrical signal by the photomultipliers. A computer constructs an image from the detector output.}
\end{center}
\end{figure}

Single-photon-emission computer tomography (SPECT) used in conjunction with a CT scanner improves on the process carried out by the gamma camera. Figure 22.39 shows a patient in a circular array of SPECT detectors that may be stationary or rotated, with detector output used by a computer to construct a detailed image. The spatial resolution of this technique is poor, but the three-dimensional image created results in a marked improvement in contrast.\\
\includegraphics[max width=\textwidth, center]{6c6af082-56ef-4ff6-b696-4e9ed0d1965b-72(1)}

Figure 22.39 SPECT uses a rotating camera to form an image of the concentration of a radiopharmaceutical compound. (credit: Woldo, Wikimedia Commons)

Positron emission tomography (or PET) scans utilize images produced by \(\beta^{+}\) emitters. When the emitted positron \({ }^{+}\)encounters an electron, mutual annihilation occurs, producing two rays. Those \(\gamma\) rays have identical 0.511 MeV energies (the energy comes from the destruction of an electron or positron mass) and they move directly away from each other, allowing detectors to determine their point of origin accurately (as shown in Figure 22.40). It requires detectors on opposite sides to simultaneously (i.e., at the same time) detect photons of 0.511 MeV energy and utilizes computer imaging techniques similar to those in SPECT and CT scans. PET is used extensively for diagnosing brain disorders. It can note decreased metabolism in certain regions that accompany Alzheimer's disease. PET can also locate regions in the brain that become active when a person carries out specific activities, such as speaking, closing his or her eyes, and so on.

\begin{figure}[h]
\begin{center}
  \includegraphics[max width=\textwidth]{6c6af082-56ef-4ff6-b696-4e9ed0d1965b-73}
\captionsetup{labelformat=empty}
\caption{Figure 22.40 A PET system takes advantage of the two identical \(\gamma\)-ray photons produced by positron-electron annihilation. The \(\gamma\) rays are emitted in opposite directions, so that the line along which each pair is emitted is determined. Various events detected by several pairs of detectors are then analyzed by the computer to form an accurate image.}
\end{center}
\end{figure}

\section*{Ionizing Radiation on the Body}
We hear many seemingly contradictory things about the biological effects of ionizing radiation. It can cause cancer, burns, and hair loss, and yet it is used to treat and even cure cancer. How do we understand such effects? Once again, there is an underlying simplicity in nature, even in complicated biological organisms. All the effects of ionizing radiation on biological tissue can be understood by knowing that ionizing radiation affects molecules within cells, particularly

DNA molecules. Let us take a brief look at molecules within cells and how cells operate. Cells have long, double-helical DNA molecules containing chemical patterns called genetic codes that govern the function and processes undertaken by the cells. Damage to DNA consists of breaks in chemical bonds or other changes in the structural features of the DNA chain, leading to changes in the genetic code. In human cells, we can have as many as a million individual instances of damage to DNA per cell per day. The repair ability of DNA is vital for maintaining the integrity of the genetic code and for the normal functioning of the entire organism. A cell with a damaged ability to repair DNA, which could have been induced by ionizing radiation, can do one of the following:

\begin{itemize}
  \item The cell can go into an irreversible state of dormancy, known as senescence.
  \item The cell can commit suicide, known as programmed cell death.
  \item The cell can go into unregulated cell division, leading to tumors and cancers.
\end{itemize}

\section*{Teacher Support}
Teacher Support [BL][OL]Why does radiation cause damage to DNA? Have students view the disruption due to the energy that radiation carries. Emphasize that ionizing radiation is just radiation with enough energy to affect the structural makeup of a cell (i.e., to ionize means to effect electrons).

Since ionizing radiation damages the DNA, ionizing radiation has its greatest effect on cells that rapidly reproduce, including most types of cancer. Thus, cancer cells are more sensitive to radiation than normal cells and can be killed by it easily. Cancer is characterized by a malfunction of cell reproduction, and can also be caused by ionizing radiation. There is no contradiction to say that ionizing radiation can be both a cure and a cause.

Radiotherapy Radiotherapy is effective against cancer because cancer cells reproduce rapidly and, consequently, are more sensitive to radiation. The central problem in radiotherapy is to make the dose for cancer cells as high as possible while limiting the dose for normal cells. The ratio of abnormal cells killed to normal cells killed is called the therapeutic ratio, and all radiotherapy techniques are designed to enhance that ratio. Radiation can be concentrated in cancerous tissue by a number of techniques. One of the most prevalent techniques for well-defined tumors is a geometric technique shown in Figure 22.41. A narrow beam of radiation is passed through the patient from a variety of directions with a common crossing point in the tumor. The technique concentrates the dose in the tumor while spreading it out over a large volume of normal tissue.

\begin{figure}[h]
\begin{center}
  \includegraphics[max width=\textwidth]{6c6af082-56ef-4ff6-b696-4e9ed0d1965b-75}
\captionsetup{labelformat=empty}
\caption{Figure 22.41 The \({ }^{60} \mathrm{Co}\) source of \(\gamma\)-radiation is rotated around the patient so that the common crossing point is in the tumor, concentrating the dose there. This geometric technique works for well-defined tumors.}
\end{center}
\end{figure}

Another use of radiation therapy is through radiopharmaceuticals. Cleverly, radiopharmaceuticals are used in cancer therapy by tagging antibodies with radioisotopes. Those antibodies are extracted from the patient, cultured, loaded with a radioisotope, and then returned to the patient. The antibodies are then concentrated almost entirely in the tissue they developed to fight, thus localizing the radiation in abnormal tissue. This method is used with radioactive iodine to fight thyroid cancer. While the therapeutic ratio can be quite high for such short-range radiation, there can be a significant dose for organs that eliminate radiopharmaceuticals from the body, such as the liver, kidneys, and bladder. As with most radiotherapy, the technique is limited by the tolerable amount of damage to the normal tissue.

Radiation Dosage To quantitatively discuss the biological effects of ionizing radiation, we need a radiation dose unit that is directly related to those effects. To do define such a unit, it is important to consider both the biological organism and the radiation itself. Knowing that the amount of ionization is proportional to the amount of deposited energy, we define a radiation dose unit called the rad. It 1/100 of a joule of ionizing energy deposited per kilogram of tissue, which is

\[
1 \mathrm{rad}=0.01 \mathrm{~J} / \mathrm{kg} .
\]

\subsection*{22.69}
For example, if a \(50.0-\mathrm{kg}\) person is exposed to ionizing radiation over her entire body and she absorbs 1.00 J , then her whole-body radiation dose is\\
\((1.00 \mathrm{~J}) /(50.0 \mathrm{~kg})=0.0200 \mathrm{~J} / \mathrm{kg}=2.00 \mathrm{rad}\).

\subsection*{22.70}
If the same 1.00 J of ionizing energy were absorbed in her \(2.00-\mathrm{kg}\) forearm alone, then the dose to the forearm would be\\
\((1.00 \mathrm{~J}) /(2.00 \mathrm{~kg})=0.500 \mathrm{~J} / \mathrm{kg}=50.0 \mathrm{rad}\),\\
and the unaffected tissue would have a zero rad dose. When calculating radiation doses, you divide the energy absorbed by the mass of affected tissue. You must specify the affected region, such as the whole body or forearm in addition to giving the numerical dose in rads. Although the energy per kilogram in 1 rad is small, it can still have significant effects. Since only a few eV cause ionization, just 0.01 J of ionizing energy can create a huge number of ion pairs and have an effect at the cellular level.

The effects of ionizing radiation may be directly proportional to the dose in rads, but they also depend on the type of radiation and the type of tissue. That is, for a given dose in rads, the effects depend on whether the radiation is \(\alpha\), \(\beta, \gamma, \mathrm{X}\)-ray, or some other type of ionizing radiation. The relative biological effectiveness (RBE) relates to the amount of biological damage that can occur from a given type of radiation and is given in Table 22.4 for several types of ionizing radiation.

Table 22.4 Relative Biological Effectiveness

\section*{Tips For Success}
The RBEs given in Table 22.4 are approximate, but they yield certain valuable insights.

\begin{itemize}
  \item The eyes are more sensitive to radiation, because the cells of the lens do not repair themselves.
  \item Though both are neutral and have large ranges, neutrons cause more damage than \(\gamma\) rays because neutrons often cause secondary radiation when they are captured.
  \item Short-range particles such as \(\alpha\) rays have a severely damaging effect to internal anatomy, as their damage is concentrated and more difficult for the biological organism to repair. However, the skin can usually block alpha particles from entering the body.
\end{itemize}

Can you think of any other insights from the table?

A final dose unit more closely related to the effect of radiation on biological tissue is called the roentgen equivalent man, or rem. A combination of all factors mentioned previously, the roentgen equivalent man is defined to be the dose in rads multiplied by the relative biological effectiveness.\\
\(\mathrm{rem}=\mathrm{rad} \times \mathrm{RBE}\)\\
22.72

The large-scale effects of radiation on humans can be divided into two categories: immediate effects and long-term effects. Table 22.5 gives the immediate effects of whole-body exposures received in less than one day. If the radiation exposure is spread out over more time, greater doses are needed to cause the effects listed. Any dose less than 10 rem is called a low dose, a dose 10 to 100 rem is called a moderate dose, and anything greater than 100 rem is called a high dose.

Table 22.5 Immediate Effects of Radiation (Adults, Whole Body, Single Exposure)

\section*{Work In Physics}
Health Physicist Are you interested in learning more about radiation? Are you curious about studying radiation dosage levels and ensuring the safety of the environment and people that are most closely affected by it? If so, you may be interested in becoming a health physicist.

The field of health physics draws from a variety of science disciplines with the central aim of mitigating radiation concerns. Those that work as health physicists have a diverse array of potential jobs available to them, including those in research, industry, education, environmental protection, and governmental regulation. Furthermore, while the term health physicist may lead many to think of the medical field, there are plenty of applications within the military, industrial, and energy fields as well.

As a researcher, a health physicist can further environmental studies on the effects of radiation, design instruments for more accurate measurements, and assist in establishing valuable radiation standards. Within the energy field, a\\
health physicsist often acts as a manager, closely tied to all operations at all levels, from procuring appropirate equipment to monitoring health data. Within industry, the health physicist acts as a consultant, assisting industry management in important decisions, designing facilities, and choosing appropriate detection tools. The health physicist possesses a unique knowledge base that allows him or her to operate in a wide variety of interesting disciplines!

To become a health physicist, it is necessary to have a background in the physical sciences. Understanding the fields of biology, physiology, biochemistry, and genetics are all important as well. The ability to analyze and solve new problems is critical, and a natural aptitude for science and mathematics will assist in the continued necessary training. There are two possible certifications for health physicists: from the American Board of Health Physicists (ABHP) and the National Registry of Radiation Protection Technologists (NRRPT).

\section*{Ke Terms}
activity rate of decay for radioactive nuclides\\
alpha decay type of radioactive decay in which an atomic nucleus emits an alpha particle\\
anger camera common medical imaging device that uses a scintillator connected to a series of photomultipliers\\
atomic number number of protons in a nucleus\\
becquerel SI unit for rate of decay of a radioactive material\\
beta decay type of radioactive decay in which an atomic nucleus emits a beta particle\\
carbon-14 dating radioactive dating technique based on the radioactivity of carbon-14\\
chain reaction self-sustaining sequence of events, exemplified by the selfsustaining nature of a fission reaction at critical mass\\
critical mass minimum amount necessary for self-sustained fission of a given nuclide\\
decay constant quantity that is inversely proportional to the half-life and that is used in the equation for number of nuclei as a function of time\\
energy-level diagram a diagram used to analyze the energy levels of electrons in the orbits of an atom\\
excited state any state beyond the \(n=1\) orbital in which the electron stores energy

Fraunhofer lines black lines shown on an absorption spectrum that show the wavelengths absorbed by a gas\\
gamma decay type of radioactive decay in which an atomic nucleus emits a gamma ray

Geiger tube very common radiation detector that usually gives an audio output\\
ground state the \(n=1\) orbital of an electron\\
half-life time in which there is a 50 percent chance that a nucleus will decay\\
Heisenberg uncertainty principle fundamental limit to the precision with which pairs of quantities such as momentum and position can be measured\\
hydrogen-like atom any atom with only a single electron\\
isotope nuclei having the same and different \(N \mathrm{~s}\)\\
liquid drop model model of the atomic nucleus (useful only to understand some of its features) in which nucleons in a nucleus act like atoms in a drop\\
mass number number of nucleons in a nucleus\\
nuclear fission reaction in which a nucleus splits\\
nuclear fusion reaction in which two nuclei are combined, or fused, to form a larger nucleus\\
nucleons particles found inside nuclei\\
planetary model of the atom model of the atom that shows electrons orbiting like planets about a Sun-like nucleus\\
proton-proton cycle combined reactions\\
\({ }^{1} \mathrm{H}+{ }^{1} \mathrm{H} \quad \rightarrow \quad{ }^{2} \mathrm{H}+e^{-}+v_{e}\).\\
\({ }^{1} \mathrm{H}+{ }^{2} \mathrm{H} \rightarrow{ }^{3} \mathrm{He}+\gamma\)\\
and\\
\({ }^{3} \mathrm{He}+{ }^{3} \mathrm{He} \rightarrow{ }^{4} \mathrm{He}+{ }^{1} \mathrm{H}+{ }^{1} \mathrm{H}\)\\
that begins with hydrogen and ends with helium\\
rad amount of ionizing energy deposited per kilogram of tissue\\
radioactive substance or object that emits nuclear radiation\\
radioactive dating application of radioactive decay in which the age of a material is determined by the amount of radioactivity of a particular type that occurs\\
radioactive decay process by which an atomic nucleus of an unstable atom loses mass and energy by emitting ionizing particles\\
radioactivity emission of rays from the nuclei of atoms\\
radiopharmaceutical compound used for medical imaging\\
relative biological effectiveness (RBE) number that expresses the relative amount of damage that a fixed amount of ionizing radiation of a given type can inflict on biological tissues\\
roentgen equivalent man (rem) dose unit more closely related to effects in biological tissue

Rutherford scattering scattering of alpha particles by gold nuclei in the gold foil experiment

Rydberg constant a physical constant related to atomic spectra, with an established value of\\
\(1.097 \times 10^{7} \mathrm{~m}^{-1}\)\\
scintillator radiation detection method that records light produced when radiation interacts with materials\\
strong nuclear force attractive force that holds nucleons together within the nucleus\\
tagged having a radioactive substance attached (to a chemical compound) therapeutic ratio the ratio of abnormal cells killed to normal cells killed\\
transmutation process of changing elemental composition

\section*{Ke Equations}
22.1 The Structure of the Atom

\subsection*{22.2 Nuclear Forces and Radioactivity}
22.3 Half Life and Radiometric Dating

\subsection*{22.4 Nuclear Fission and Fusion}
22.5 Medical Applications of Radioactivity: Diagnostic Imaging and Radiation

\section*{Section Summar}
\subsection*{22.1 The Structure of the Atom}
\begin{itemize}
  \item Rutherford's gold foil experiment provided evidence that the atom is composed of a small, dense nucleus with electrons occupying the mostly empty space around it.
  \item Analysis of emission spectra shows that energy is emitted from energized gas in discrete quantities.
  \item The Bohr model of the atom describes electrons existing in discrete orbits, with discrete energies emitted and absorbed as the electrons decrease and increase in orbital energy.
  \item The energy emitted or absorbed by an electron as it changes energy state can be determined with the equation \(\Delta E=E_{i}-E_{f}\), where \(E_{n}=\frac{Z^{2}}{n^{2}} E_{o}(n=1,2,3, \ldots)\).
  \item The wavelength of energy absorbed or emitted by an electron as it changes energy state can be determined by the equation \(\underline{1}=R\left(\frac{1}{n_{f}^{2}}-\frac{1}{n_{i}^{2}}\right)\), where \(R=1.097 \times 10^{7} \mathrm{~m}^{-1}\).
  \item Described as an electron cloud, the quantum model of the atom is the result of de Broglie waves and Heisenberg's uncertainty principle.
\end{itemize}

\subsection*{22.2 Nuclear Forces and Radioactivity}
\begin{itemize}
  \item The structure of the nucleus is defined by its two nucleons, the neutron and proton.
  \item Atomic numbers and mass numbers are used to differentiate between various atoms and isotopes. Those numbers can be combined into an easily recognizable form called a nuclide.
  \item The size and stability of the nucleus is based upon two forces: the electromagnetic force and strong nuclear force.
  \item Radioactive decay is the alteration of the nucleus through the emission of particles or energy.
  \item Alpha decay occurs when too many protons exist in the nucleus. It results in the ejection of an alpha particle, as described in the equation X Z A N → Y Z \(-2 \mathrm{~A}-4 \mathrm{~N}+\mathrm{H} 24 \mathrm{e}\).
  \item Beta decay occurs when too many neutrons (or protons) exist in the nucleus. It results in the transmutation of a neutron into a proton, electron, and neutrino. The decay is expressed through the equation \(\mathrm{XZAN} \rightarrow \mathrm{Y} \mathrm{Z}+1 \mathrm{~A} \mathrm{~N}-1+\mathrm{e}+\mathrm{v}\). (Beta decay may also transform a proton into a neutron.)
  \item Gamma decay occurs when a nucleus in an excited state move to a more stable state, resulting in the release of a photon. Gamma decay is represented with the equation \(\mathrm{XZAN} \rightarrow \mathrm{XN}+\).
  \item The penetration distance of radiation depends on its energy, charge, and type of material it encounters.
\end{itemize}

\subsection*{22.3 Half Life and Radiometric Dating}
\begin{itemize}
  \item Radioactive half-life is the time it takes a sample of nuclei to decay to half of its original amount.
  \item The rate of radioactive decay is defined as the sample's activity, represented by the equation \(R=\frac{\Delta N}{\Delta t}\).
  \item Knowing the half-life of a radioactive isotope allows for the process of radioactive dating to determine the age of a material.
  \item If the half-life of a material is known, the age of the material can be found using the equation \(N=N_{O} e^{-t}\).
  \item The age of organic material can be determined using the decay of the carbon-14 isotope, while the age of rocks can be determined using the decay of uranium-238.
\end{itemize}

\subsection*{22.4 Nuclear Fission and Fusion}
\begin{itemize}
  \item Nuclear fission is the splitting of an atomic bond, releasing a large amount of potential energy previously holding the atom together. The amount of energy released can be determined through the equation \(E=m c^{2}\).
  \item Nuclear fusion is the combining, or fusing together, of two nuclei. Energy is also released in nuclear fusion as the combined nuclei are closer together, resulting in a decreased strong nuclear force.
  \item Fission was used in two nuclear weapons at the conclusion of World War II: the gun-type uranium bomb and the implosion-type plutonium bomb.
  \item While fission has been used in both nuclear weapons and nuclear reactors, fusion is capable of releasing more energy per reaction. As a result, fusion is a well-researched, if not yet well-controlled, energy source.
\end{itemize}

\subsection*{22.5 Medical Applications of Radioactivity: Diagnostic Imaging and Radiation}
\begin{itemize}
  \item Medical imaging occurs when a radiopharmaceutical placed in the body provides information to an array of radiation detectors outside the body.
  \item Devices utilizing medical imaging include the Anger camera, SPECT detector, and PET scan.
  \item Ionizing radiation can both cure and cause cancer through the manipulation of DNA molecules.
  \item Radiation dosage and its effect on the body can be measured using the quantities radiation dose unit (rad), relative biological effectiveness (RBE), and the roentgen equivalent man (rem).
\end{itemize}

\end{document}