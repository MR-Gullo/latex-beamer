\documentclass[10pt]{article}
\usepackage[utf8]{inputenc}
\usepackage[T1]{fontenc}
\usepackage{graphicx}
\usepackage[export]{adjustbox}
\graphicspath{ {./images/} }
\usepackage{caption}
\usepackage{amsmath}
\usepackage{amsfonts}
\usepackage{amssymb}
\usepackage[version=4]{mhchem}
\usepackage{stmaryrd}

\DeclareUnicodeCharacter{22C5}{\ifmmode\cdot\else{$\cdot$}\fi}

\begin{document}
\captionsetup{singlelinecheck=false}
\begin{figure}[h]
\begin{center}
  \includegraphics[max width=\textwidth]{f6bfaebc-615a-417d-a4a9-d242300d8de3-01}
\captionsetup{labelformat=empty}
\caption{Figure 8.1 NFC defensive backs Ronde Barber and Roy Williams along with linebacker Jeremiah Trotter gang tackle AFC running back LaDainian Tomlinson during the 2006 Pro Bowl in Hawaii. (United States Marine Corps)}
\end{center}
\end{figure}

\section*{Chapter Outline}
8.1 Linear Momentum, Force, and Impulse\\
8.2 Conservation of Momentum\\
8.3 Elastic and Inelastic Collisions

\section*{Introduction}
\section*{Teacher Support}
Teacher Support Point out to the students how players often collide with each other while playing American football. How do these collisions affect the players? Does colliding into someone change your velocity? Does it change your mass? What about the force of collision? What does it depend on? Would it hurt more if a heavier person collided into you or a faster person? Tell students that in this chapter they will learn about momentum, its relation to force and about collisions.

We know from everyday use of the word momentum that it is a tendency to continue on course in the same direction. Newscasters speak of sports teams or politicians gaining, losing, or maintaining the momentum to win. As we learned when studying about inertia, which is Newton's first law of motion, every object\\
or system has inertia-that is, a tendency for an object in motion to remain in motion or an object at rest to remain at rest. Mass is a useful variable that lets us quantify inertia. Momentum is mass in motion.

Momentum is important because it is conserved in isolated systems; this fact is convenient for solving problems where objects collide. The magnitude of momentum grows with greater mass and/or speed. For example, look at the football players in the photograph (Figure 8.1). They collide and fall to the ground. During their collisions, momentum will play a large part. In this chapter, we will learn about momentum, the different types of collisions, and how to use momentum equations to solve collision problems.

\section*{Teacher Support}
Teacher Support Before students begin this chapter, it would be useful to review these concepts: mass, inertia, Newton's laws of motion, angular motion, and moment of inertia.

\subsection*{8.1 Linear Momentum, Force, and Impulse}
\section*{Section Learning Objectives}
By the end of this section, you will be able to do the following:

\begin{itemize}
  \item Describe momentum, what can change momentum, impulse, and the impulse-momentum theorem
  \item Describe Newton's second law in terms of momentum
  \item Solve problems using the impulse-momentum theorem
\end{itemize}

\section*{Teacher Support}
Teacher Support The learning objectives in this section will help your students master the following standards:

\begin{itemize}
  \item (6) Science concepts. The student knows that changes occur within a physical system and applies the laws of conservation of energy and momentum. The student is expected to:
  \item (C) calculate the mechanical energy of, power generated within, impulse applied to, and momentum of a physical system.
\end{itemize}

\section*{Section Key Terms}
\section*{Teacher Support}
Teacher Support [BL][OL] Review inertia and Newton's laws of motion.\\[0pt]
[AL] Start a discussion about movement and collision. Using the example of football players, point out that both the mass and the velocity of an object are important considerations in determining the impact of collisions. The direction as well as the magnitude of velocity is very important.

\section*{Momentum, Impulse, and the Impulse-Momentum Theorem}
Linear momentum is the product of a system's mass and its velocity. In equation form, linear momentum \(\mathbf{p}\) is\\
\(\mathbf{p}=m \mathbf{v}\).\\
You can see from the equation that momentum is directly proportional to the object's mass \((m)\) and velocity \((\mathbf{v})\). Therefore, the greater an object's mass or the greater its velocity, the greater its momentum. A large, fast-moving object has greater momentum than a smaller, slower object.

Momentum is a vector and has the same direction as velocity \(\mathbf{v}\). Since mass is a scalar, when velocity is in a negative direction (i.e., opposite the direction of motion), the momentum will also be in a negative direction; and when velocity is in a positive direction, momentum will likewise be in a positive direction. The SI unit for momentum is \(\mathrm{kg} \mathrm{m} / \mathrm{s}\).

Momentum is so important for understanding motion that it was called the quantity of motion by physicists such as Newton. Force influences momentum, and we can rearrange Newton's second law of motion to show the relationship between force and momentum.

Recall our study of Newton's second law of motion \(\left(\mathbf{F}_{\text {net }}=m \mathbf{a}\right)\). Newton actually stated his second law of motion in terms of momentum: The net external force equals the change in momentum of a system divided by the time over which it changes. The change in momentum is the difference between the final and initial values of momentum.

In equation form, this law is\\
\(\mathbf{F}_{\text {net }}=\frac{\Delta \mathbf{p}}{\Delta t}\),\\
where \(\mathbf{F}_{\text {net }}\) is the net external force, \(\Delta \mathbf{p}\) is the change in momentum, and \(\Delta t\) is the change in time.

We can solve for \(\Delta \mathbf{p}\) by rearranging the equation\\
\(\mathbf{F}_{\text {net }}=\frac{\Delta \mathbf{p}}{\Delta t}\)\\
to be\\
\(\Delta \mathrm{p}=\mathrm{F}_{\text {net }} \Delta t\).\\
\(\mathrm{F}_{\text {net }} \Delta t\) is known as impulse and this equation is known as the impulsemomentum theorem. From the equation, we see that the impulse equals the average net external force multiplied by the time this force acts. It is equal to the change in momentum. The effect of a force on an object depends on how long it acts, as well as the strength of the force. Impulse is a useful concept because it quantifies the effect of a force. A very large force acting for a short time can have a great effect on the momentum of an object, such as the force of a racket hitting a tennis ball. A small force could cause the same change in momentum, but it would have to act for a much longer time.

\section*{Teacher Support}
Teacher Support [OL][AL] Explain that a large, fast-moving object has greater momentum than a smaller, slower object. This quality is called momentum.\\[0pt]
[BL][OL] Review the equation of Newton's second law of motion. Point out the two different equations for the law.

\section*{Newton's Second Law in Terms of Momentum}
When Newton's second law is expressed in terms of momentum, it can be used for solving problems where mass varies, since \(\Delta \mathrm{p}=\Delta(m \mathrm{v})\). In the more traditional form of the law that you are used to working with, mass is assumed to be constant. In fact, this traditional form is a special case of the law, where mass is constant. \(\mathrm{F}_{\text {net }}=m \mathrm{a}\) is actually derived from the equation:\\
\(\mathbf{F}_{\text {net }}=\frac{\Delta \mathbf{p}}{\Delta t}\)\\
For the sake of understanding the relationship between Newton's second law in its two forms, let's recreate the derivation of \(\mathrm{F}_{\text {net }}=m \mathrm{a}\) from\\
\(\mathbf{F}_{\text {net }}=\frac{\Delta \mathbf{p}}{\Delta t}\)\\
by substituting the definitions of acceleration and momentum.\\
The change in momentum \(\Delta \mathrm{p}\) is given by\\
\(\Delta \mathrm{p}=\Delta(m \mathrm{v})\).\\
If the mass of the system is constant, then\\
\(\Delta(m \mathrm{v})=m \Delta \mathrm{v}\).\\
By substituting \(m \Delta \mathrm{v}\) for \(\Delta \mathrm{p}\), Newton's second law of motion becomes\\
\(\mathbf{F}_{\text {net }}=\frac{\Delta \mathrm{p}}{\Delta t}=\frac{m \Delta \mathrm{v}}{\Delta t}\)\\
for a constant mass.\\
Because\\
\(\frac{\Delta \mathrm{v}}{\Delta t}=\mathrm{a}\),\\
we can substitute to get the familiar equation\\
\(\mathrm{F}_{\text {net }}=m \mathrm{a}\)\\
when the mass of the system is constant.

\section*{Teacher Support}
Teacher Support [BL][OL][AL] Show the two different forms of Newton's second law and how one can be derived from the other.

\section*{Tips For Success}
We just showed how \(\mathrm{F}_{\text {net }}=m\) applies only when the mass of the system is constant. An example of when this formula would not apply would be a moving rocket that burns enough fuel to significantly change the mass of the rocket. In this case, you would need to use Newton's second law expressed in terms of momentum to account for the changing mass.

\section*{Snap Lab}
Hand Movement and Impulse In this activity you will experiment with different types of hand motions to gain an intuitive understanding of the relationship between force, time, and impulse.

\begin{itemize}
  \item one ball
  \item one tub filled with water
\end{itemize}

Procedure:

\begin{enumerate}
  \item Try catching a ball while giving with the ball, pulling your hands toward your body.
  \item Next, try catching a ball while keeping your hands still.
  \item Hit water in a tub with your full palm. Your full palm represents a swimmer doing a belly flop.
  \item After the water has settled, hit the water again by diving your hand with your fingers first into the water. Your diving hand represents a swimmer doing a dive.
  \item Explain what happens in each case and why.
\end{enumerate}

What are some other examples of motions that impulse affects?\\
a. a football player colliding with another, or a car moving at a constant velocity\\
b. a car moving at a constant velocity, or an object moving in the projectile motion\\
c. a car moving at a constant velocity, or a racket hitting a ball\\
d. a football player colliding with another, or a racket hitting a ball

\section*{Teacher Support}
Teacher Support [OL][AL] Discuss the impact one feels when one falls or jumps. List the factors that affect this impact.

\section*{Links To Physics}
Engineering: Saving Lives Using the Concept of Impulse Cars during the past several decades have gotten much safer. Seat belts play a major role in automobile safety by preventing people from flying into the windshield in the event of a crash. Other safety features, such as airbags, are less visible or obvious, but are also effective at making auto crashes less deadly (see Figure 8.2). Many of these safety features make use of the concept of impulse from\\
physics. Recall that impulse is the net force multiplied by the duration of time of the impact. This was expressed mathematically as \(\Delta \mathrm{p}=\mathrm{F}_{\text {net }} \Delta t\).

\begin{figure}[h]
\begin{center}
  \includegraphics[max width=\textwidth]{f6bfaebc-615a-417d-a4a9-d242300d8de3-07}
\captionsetup{labelformat=empty}
\caption{Figure 8.2 Vehicles have safety features like airbags and seat belts installed.}
\end{center}
\end{figure}

Airbags allow the net force on the occupants in the car to act over a much longer time when there is a sudden stop. The momentum change is the same for an occupant whether an airbag is deployed or not. But the force that brings the occupant to a stop will be much less if it acts over a larger time. By rearranging the equation for impulse to solve for force \(\mathbf{F}_{\text {net }}=\frac{\Delta \mathbf{p}}{\Delta t}\), you can see how increasing \(\Delta t\) while \(\Delta \mathbf{p}\) stays the same will decrease \(\mathbf{F}_{\text {net }}\). This is another example of an inverse relationship. Similarly, a padded dashboard increases the time over which the force of impact acts, thereby reducing the force of impact.

Cars today have many plastic components. One advantage of plastics is their lighter weight, which results in better gas mileage. Another advantage is that a car will crumple in a collision, especially in the event of a head-on collision. A longer collision time means the force on the occupants of the car will be less. Deaths during car races decreased dramatically when the rigid frames of racing cars were replaced with parts that could crumple or collapse in the event of an accident.

\section*{Grasp Check}
You may have heard the advice to bend your knees when jumping. In this example, a friend dares you to jump off of a park bench onto the ground without bending your knees. You, of course, refuse. Explain to your friend why this would be a foolish thing. Show it using the impulse-momentum theorem.\\
a. Bending your knees increases the time of the impact, thus decreasing the force.\\
b. Bending your knees decreases the time of the impact, thus decreasing the force.\\
c. Bending your knees increases the time of the impact, thus increasing the force.\\
d. Bending your knees decreases the time of the impact, thus increasing the force.

\section*{Solving Problems Using the Impulse-Momentum Theorem}
\section*{Teacher Support}
Teacher Support Talk about the different strategies to be used while solving problems. Make sure that students know the assumptions made in each equation regarding certain quantities being constant or some quantities being negligible.

\section*{Worked Example}
Calculating Momentum: A Football Player and a Football (a) Calculate the momentum of a 110 kg football player running at \(8 \mathrm{~m} / \mathrm{s}\). (b) Compare the player's momentum with the momentum of a 0.410 kg football thrown hard at a speed of \(25 \mathrm{~m} / \mathrm{s}\).

\section*{Strategy}
No information is given about the direction of the football player or the football, so we can calculate only the magnitude of the momentum, \(p\). (A symbol in italics represents magnitude.) In both parts of this example, the magnitude of momentum can be calculated directly from the definition of momentum:\\
\(\mathrm{p}=m \mathrm{v}\)\\
Solution for (a)\\
To find the player's momentum, substitute the known values for the player's mass and speed into the equation.\\
\(\mathbf{p}_{\text {player }}=(110 \mathrm{~kg})(8 \mathrm{~m} / \mathrm{s})=880 \mathrm{~kg} \cdot \mathrm{~m} / \mathrm{s}\)\\
Solution for (b)\\
To find the ball's momentum, substitute the known values for the ball's mass and speed into the equation.\\
\(\mathbf{p}_{\text {ball }}=(0.410 \mathrm{~kg})(25 \mathrm{~m} / \mathrm{s})=10.25 \mathrm{~kg} \cdot \mathrm{~m} / \mathrm{s}\)\\
The ratio of the player's momentum to the ball's momentum is\\
\(\frac{\mathbf{p}_{\text {player }}}{\mathbf{p}_{\text {ball }}}=\frac{880}{10.3}=85.9\).\\
Discussion

Although the ball has greater velocity, the player has a much greater mass. Therefore, the momentum of the player is about 86 times greater than the momentum of the football.

\section*{Worked Example}
Calculating Force: Venus Williams' Racquet During the 2007 French Open, Venus Williams (Figure 8.3) hit the fastest recorded serve in a premier women's match, reaching a speed of \(58 \mathrm{~m} / \mathrm{s}(209 \mathrm{~km} / \mathrm{h})\). What was the average force exerted on the 0.057 kg tennis ball by Williams' racquet? Assume that the ball's speed just after impact was \(58 \mathrm{~m} / \mathrm{s}\), the horizontal velocity before impact is negligible, and that the ball remained in contact with the racquet for 5 ms (milliseconds).

\begin{figure}[h]
\begin{center}
  \includegraphics[max width=\textwidth]{f6bfaebc-615a-417d-a4a9-d242300d8de3-09}
\captionsetup{labelformat=empty}
\caption{Figure 8.3 Venus Williams playing in the 2013 US Open (Edwin Martinez, Flickr)}
\end{center}
\end{figure}

\section*{Strategy}
Recall that Newton's second law stated in terms of momentum is\\
\(\mathbf{F}_{\text {net }}=\frac{\Delta \mathbf{p}}{\Delta t}\).\\
As noted above, when mass is constant, the change in momentum is given by\\
\(\Delta \mathrm{p}=m \Delta \mathrm{v}=m\left(\mathrm{v}_{\mathrm{f}}-\mathrm{v}_{\mathrm{i}}\right)\),\\
where \(\mathbf{v}_{\mathrm{f}}\) is the final velocity and \(\mathbf{v}_{\mathrm{i}}\) is the initial velocity. In this example, the velocity just after impact and the change in time are given, so after we solve for \(\Delta \mathrm{p}\), we can use \(\mathbf{F}_{\text {net }}=\frac{\Delta \mathbf{p}}{\Delta t}\) to find the force.

Solution\\
To determine the change in momentum, substitute the values for mass and the initial and final velocities into the equation above.

\[
\begin{aligned}
\Delta \mathbf{p} & =m\left(\mathbf{v}_{f}-\mathbf{v}_{i}\right) \\
& =(0.057 \mathrm{~kg})(58 \mathrm{~m} / \mathrm{s}-0 \mathrm{~m} / \mathrm{s}) \\
& =3.306 \mathrm{~kg} \cdot \mathrm{~m} / \mathrm{s} \approx 3.3 \mathrm{~kg} \cdot \mathrm{~m} / \mathrm{s}
\end{aligned}
\]

8.1

Now we can find the magnitude of the net external force using \(\mathbf{F}_{\text {net }}=\frac{\Delta \mathbf{p}}{\Delta t}\)

\[
\begin{aligned}
\mathbf{F}_{\text {net }} & =\frac{\Delta \mathbf{p}}{\Delta t}=\frac{3.306}{5 \times 10^{-3}} \\
& =661 \mathrm{~N} \\
& \approx 660 \mathrm{~N}
\end{aligned}
\]

8.2

Discussion\\
This quantity was the average force exerted by Venus Williams' racquet on the tennis ball during its brief impact. This problem could also be solved by first finding the acceleration and then using \(\mathbf{F}_{\text {net }}=m \mathbf{a}\), but we would have had to do one more step. In this case, using momentum was a shortcut.

\section*{Practice Problems}
1.

A 5 kg bowling ball is rolled with a velocity of \(10 \mathrm{~m} / \mathrm{s}\). What is its momentum?\\
a. \(0.5 \backslash, \backslash \operatorname{text}\{\mathrm{~kg}\} \backslash \operatorname{cdot} \backslash \operatorname{text}\{\mathrm{m} / \mathrm{s}\}\)\\
b. \(2 \backslash, \backslash \operatorname{text}\{\mathrm{~kg}\} \backslash \operatorname{cdot} \backslash \operatorname{text}\{\mathrm{m} / \mathrm{s}\}\)\\
c. \(15 \backslash, \backslash \operatorname{text}\{\mathrm{~kg}\} \backslash \operatorname{cdot} \backslash \operatorname{text}\{\mathrm{m} / \mathrm{s}\}\)\\
d. \(50 \backslash, \backslash \operatorname{text}\{\mathrm{~kg}\} \backslash \operatorname{cdot} \backslash \operatorname{text}\{\mathrm{m} / \mathrm{s}\}\)\\
2.\\
\includegraphics[max width=\textwidth, center]{f6bfaebc-615a-417d-a4a9-d242300d8de3-11}\\
(credit: modification of work from Pinterest)\\
A \(155-\mathrm{g}\) baseball is incoming at a velocity of \(25 \mathrm{~m} / \mathrm{s}\). The batter hits the ball as shown in the image. The outgoing baseball has a velocity of \(20 \mathrm{~m} / \mathrm{s}\) at the angle shown.

What is the magnitudde of the impulse acting on the ball during the hit?\\
a. \(2.68 \mathrm{~kg} \mathrm{~m} / \mathrm{s}\).\\
b. \(5.42 \mathrm{~kg} \mathrm{~m} / \mathrm{s}\).\\
c. \(6.05 \mathrm{~kg} \mathrm{~m} / \mathrm{s}\).\\
d. \(8.11 \mathrm{~kg} \mathrm{~m} / \mathrm{s}\).

\section*{Check Your Understanding}
3.

What is linear momentum?\\
a. the sum of a system's mass and its velocity\\
b. the ratio of a system's mass to its velocity\\
c. the product of a system's mass and its velocity\\
d. the product of a system's moment of inertia and its velocity\\
4.

If an object's mass is constant, what is its momentum proportional to?\\
a. Its velocity\\
b. Its weight\\
c. Its displacement\\
d. Its moment of inertia\\
5.

What is the equation for Newton's second law of motion, in terms of mass, velocity, and time, when the mass of the system is constant?\\
a. \(\mathrm{F}_{\mathrm{net}}=\frac{\Delta \mathrm{v}}{\Delta m \Delta t}\)\\
b. \(\mathrm{F}_{\mathrm{net}}=\frac{m \Delta t}{\Delta \mathrm{v}}\)\\
c. \(\mathrm{F}_{\mathrm{net}}=\frac{m \Delta \mathrm{v}}{\Delta t}\)\\
d. \(\mathrm{F}_{\mathrm{net}}=\frac{\Delta m \Delta \mathrm{v}}{\Delta t}\)\\
6.

Give an example of a system whose mass is not constant.\\
a. A spinning top\\
b. A baseball flying through the air\\
c. A rocket launched from Earth\\
d. A block sliding on a frictionless inclined plane

\section*{Teacher Support}
Teacher Support Use the Check Your Understanding questions to assess whether students master the learning objectives of this section. If students are struggling with a specific objective, the assessment will help identify which objective is causing the problem and direct students to the relevant content.

\subsection*{8.2 Conser ation of Momentum}
\section*{Section Learning Objectives}
By the end of this section, you will be able to do the following:

\begin{itemize}
  \item Describe the law of conservation of momentum verbally and mathematically
\end{itemize}

\section*{Teacher Support}
Teacher Support The learning objectives in this section will help your students master the following standards:

\begin{itemize}
  \item (6) Science concepts. The student knows that changes occur within a physical system and applies the laws of conservation of energy and momentum. The student is expected to:
  \item (C) calculate the mechanical energy of, power generated within, impulse applied to, and momentum of a physical system
  \item (D) demonstrate and apply the laws of conservation of energy and conservation of momentum in one dimension
\end{itemize}

\section*{Section Key Terms}
\section*{Teacher Support}
Teacher Support In this section, students will apply what they have learned about momentum, impulse, and force.\\[0pt]
[BL][OL] Before students read the section, ask them what they understand by the word conservation. Have they come across it in any other law of physics?

\section*{Conservation of Momentum}
It is important we realize that momentum is conserved during collisions, explosions, and other events involving objects in motion. To say that a quantity is conserved means that it is constant throughout the event. In the case of conservation of momentum, the total momentum in the system remains the same before and after the collision.

You may have noticed that momentum was not conserved in some of the examples previously presented in this chapter. where forces acting on the objects produced large changes in momentum. Why is this? The systems of interest considered in those problems were not inclusive enough. If the systems were expanded to include more objects, then momentum would in fact be conserved\\
in those sample problems. It is always possible to find a larger system where momentum is conserved, even though momentum changes for individual objects within the system.

\section*{Teacher Support}
Teacher Support [OL][AL] Caution students that momentum is only conserved when the entire system affected is taken into account. Explain isolated system. Ask students to give examples of isolated systems. Ask them if these are perfectly isolated. Would it be possible to have perfectly isolated systems on Earth?

For example, if a football player runs into the goalpost in the end zone, a force will cause him to bounce backward. His momentum is obviously greatly changed, and considering only the football player, we would find that momentum is not conserved. However, the system can be expanded to contain the entire Earth. Surprisingly, Earth also recoils-conserving momentum-because of the force applied to it through the goalpost. The effect on Earth is not noticeable because it is so much more massive than the player, but the effect is real.

Next, consider what happens if the masses of two colliding objects are more similar than the masses of a football player and Earth-in the example shown in Figure 8.4 of one car bumping into another. Both cars are coasting in the same direction when the lead car, labeled \(m_{2}\), is bumped by the trailing car, labeled \(m_{1}\). The only unbalanced force on each car is the force of the collision, assuming that the effects due to friction are negligible. Car \(m 1\) slows down as a result of the collision, losing some momentum, while car \(m 2\) speeds up and gains some momentum. If we choose the system to include both cars and assume that friction is negligible, then the momentum of the two-car system should remain constant. Now we will prove that the total momentum of the two-car system does in fact remain constant, and is therefore conserved.

\begin{figure}[h]
\begin{center}
  \includegraphics[max width=\textwidth]{f6bfaebc-615a-417d-a4a9-d242300d8de3-15}
\captionsetup{labelformat=empty}
\caption{Figure 8.4 Car of mass \(m_{1}\) moving with a velocity of \(\mathbf{v}_{1}\) bumps into another car of mass \(m_{2}\) and velocity \(\mathbf{v}_{2}\). As a result, the first car slows down to a velocity of \(\mathbf{v}_{1}\) and the second speeds up to a velocity of \(\mathbf{v}_{2}\). The momentum of each car is changed, but the total momentum \(\mathbf{p}_{\text {tot }}\) of the two cars is the same before and after the collision if you assume friction is negligible.}
\end{center}
\end{figure}

Using the impulse-momentum theorem, the change in momentum of car 1 is given by\\
\(\Delta \mathbf{p}_{1}=\mathbf{F}_{1} \Delta t\),\\
where \(\mathbf{F}_{1}\) is the force on car 1 due to car 2 , and \(\Delta t\) is the time the force acts, or the duration of the collision.

Similarly, the change in momentum of car 2 is \(\Delta \mathbf{p}_{2}=\mathbf{F}_{2} \Delta t\) where \(\mathbf{F}_{2}\) is the force on car 2 due to car 1, and we assume the duration of the collision \(\Delta t\) is the same for both cars. We know from Newton's third law of motion that \(\mathbf{F}_{2}= -\mathbf{F}_{1}\), and so \(\Delta \mathbf{p}_{2}=-\mathbf{F}_{1} \Delta t=-\Delta \mathbf{p}_{1}\).

Therefore, the changes in momentum are equal and opposite, and \(\Delta \mathbf{p}_{1}+\Delta \mathbf{p}_{2}=0\) -

Because the changes in momentum add to zero, the total momentum of the two-car system is constant. That is,\\
\(\mathrm{p}_{1}+\mathrm{p}_{2}=\) constant\\
\(\mathbf{p}_{1}+\mathbf{p}_{2}=\mathbf{p}^{\prime}{ }_{1}+\mathbf{p}^{\prime}{ }_{2}\),\\
where \(\mathbf{p}_{1}\) and \(\mathbf{p}_{2}\) are the momenta of cars 1 and 2 after the collision.\\
This result that momentum is conserved is true not only for this example involving the two cars, but for any system where the net external force is zero, which is known as an isolated system. The law of conservation of momentum states that for an isolated system with any number of objects in it, the total momentum is conserved. In equation form, the law of conservation of momentum for an isolated system is written as\\
\(\mathrm{p}_{\text {tot }}=\) constant\\
or\\
\(\mathbf{p}_{\text {tot }}=\mathbf{p}_{\text {tot }}^{\prime}\),\\
where \(\mathbf{p}_{\text {tot }}\) is the total momentum, or the sum of the momenta of the individual objects in the system at a given time, and \(\mathbf{p}_{\text {tot }}\) is the total momentum some time later.

The conservation of momentum principle can be applied to systems as diverse as a comet striking the Earth or a gas containing huge numbers of atoms and molecules. Conservation of momentum appears to be violated only when the net external force is not zero. But another larger system can always be considered in which momentum is conserved by simply including the source of the external force. For example, in the collision of two cars considered above, the two-car system conserves momentum while each one-car system does not.

\section*{Tips For Success}
Momenta is the plural form of the word momentum. One object is said to have momentum, but two or more objects are said to have momenta.

\section*{Fun In Physics}
Angular Momentum in Figure Skating So far we have covered linear momentum, which describes the inertia of objects traveling in a straight line. But we know that many objects in nature have a curved or circular path. Just as linear motion has linear momentum to describe its tendency to move forward, circular motion has the equivalent angular momentum to describe how rotational motion is carried forward.

This is similar to how torque is analogous to force, angular acceleration is analogous to translational acceleration, and \(m r^{2}\) is analogous to mass or inertia. You may recall learning that the quantity \(m r^{2}\) is called the rotational inertia or moment of inertia of a point mass \(m\) at a distance \(r\) from the center of rotation.

We already know the equation for linear momentum, \(\mathbf{p}=m \mathbf{v}\). Since angular momentum is analogous to linear momentum, the moment of inertia ( \(I\) ) is analogous to mass, and angular velocity ( ) is analogous to linear velocity, it makes sense that angular momentum ( \(\mathbf{L}\) ) is defined as\\
\(\mathbf{L}=I\)\\
Angular momentum is conserved when the net external torque ( ) is zero, just as linear momentum is conserved when the net external force is zero.

Figure skaters take advantage of the conservation of angular momentum, likely without even realizing it. In Figure 8.5, a figure skater is executing a spin. The net torque on her is very close to zero, because there is relatively little friction between her skates and the ice, and because the friction is exerted very close to the pivot point. Both \(\mathbf{F}\) and \(r\) are small, and so is negligibly small.

\begin{figure}[h]
\begin{center}
  \includegraphics[max width=\textwidth]{f6bfaebc-615a-417d-a4a9-d242300d8de3-17}
\captionsetup{labelformat=empty}
\caption{Figure 8.5 (a) An ice skater is spinning on the tip of her skate with her arms extended. In the next image, (b), her rate of spin increases greatly when she pulls in her arms.}
\end{center}
\end{figure}

Consequently, she can spin for quite some time. She can do something else, too. She can increase her rate of spin by pulling her arms and legs in. Why does pulling her arms and legs in increase her rate of spin? The answer is that her angular momentum is constant, so that \(\mathbf{L}=\mathbf{L}\).

Expressing this equation in terms of the moment of inertia,\\
\(I=I^{\prime^{\prime}}\),\\
where the primed quantities refer to conditions after she has pulled in her arms and reduced her moment of inertia. Because \(I\) is smaller, the angular velocity ' must increase to keep the angular momentum constant. This allows her to spin much faster without exerting any extra torque.

A video is also available that shows a real figure skater executing a spin. It discusses the physics of spins in figure skating.

\section*{Teacher Support}
Teacher Support You can demonstrate a similar exercise in class using a revolving stool or chair. Ask a student to sit on the stool with outstretched arms, holding some weight in each hand. Rotate the stool and once a good speed is achieved, ask him to bring his hands in close to his body. He will start spinning faster.

\section*{Grasp Check}
Based on the equation \(\mathbf{L}=I\), how would you expect the moment of inertia of an object to affect angular momentum? How would angular velocity affect angular momentum?\\
a. Large moment of inertia implies large angular momentum, and large angular velocity implies large angular momentum.\\
b. Large moment of inertia implies small angular momentum, and large angular velocity implies small angular momentum.\\
c. Large moment of inertia implies large angular momentum, and large angular velocity implies small angular momentum.\\
d. Large moment of inertia implies small angular momentum, and large angular velocity implies large angular momentum.

\section*{Check Your Understanding}
7.

When is momentum said to be conserved?\\
a. When momentum is changing during an event\\
b. When momentum is increasing during an event\\
c. When momentum is decreasing during an event\\
d. When momentum is constant throughout an event\\
8.

A ball is hit by a racket and its momentum changes. Under what conditions is momentum conserved?\\
a. The momentum of the system can never be conserved in this case.\\
b. Tje momentum of the system is conserved if the momentum of the racket is not considered.\\
c. The momentum of the system is conserved if the momentum of the racket is also considered.\\
d. The momentum of the system is conserved if the momenta of the racket and the player are also considered.\\
9.

State the law of conservation of momentum.\\
a. Momentum is conserved for an isolated system with any number of objects in it.\\
b. Momentum is conserved for an isolated system with an even number of objects in it.\\
c. Momentum is conserved for an interacting system with any number of objects in it.\\
d. Momentum is conserved for an interacting system with an even number of objects in it.

\section*{Teacher Support}
Teacher Support Use the Check Your Understanding questions to assess whether students master the learning objectives of this section. If students are struggling with a specific objective, the assessment will help identify which objective is causing the problem and direct students to the relevant content.

\subsection*{8.3 Elastic and Inelastic Collisions}
\section*{Section Learning Objectives}
By the end of this section, you will be able to do the following:

\begin{itemize}
  \item Distinguish between elastic and inelastic collisions
  \item Solve collision problems by applying the law of conservation of momentum
\end{itemize}

\section*{Teacher Support}
Teacher Support The learning objectives in this section will help your students master the following standards:

\begin{itemize}
  \item (6) Science concepts. The student knows that changes occur within a physical system and applies the laws of conservation of energy and momentum. The student is expected to:
  \item (C) calculate the mechanical energy of, power generated within, impulse applied to, and momentum of a physical system;
  \item (D) demonstrate and apply the laws of conservation of energy and conservation of momentum in one dimension.
\end{itemize}

\section*{Section Key Terms}
\section*{Elastic and Inelastic Collisions}
When objects collide, they can either stick together or bounce off one another, remaining separate. In this section, we'll cover these two different types of collisions, first in one dimension and then in two dimensions.

In an elastic collision, the objects separate after impact and don't lose any of their kinetic energy. Kinetic energy is the energy of motion and is covered in detail elsewhere. The law of conservation of momentum is very useful here, and it can be used whenever the net external force on a system is zero. Figure 8.6 shows an elastic collision where momentum is conserved.

\begin{figure}[h]
\begin{center}
  \includegraphics[max width=\textwidth]{f6bfaebc-615a-417d-a4a9-d242300d8de3-21}
\captionsetup{labelformat=empty}
\caption{Figure 8.6 The diagram shows a one-dimensional elastic collision between two objects.}
\end{center}
\end{figure}

An animation of an elastic collision between balls can be seen by watching this video. It replicates the elastic collisions between balls of varying masses.

Perfectly elastic collisions can happen only with subatomic particles. Everyday observable examples of perfectly elastic collisions don't exist-some kinetic energy is always lost, as it is converted into heat transfer due to friction. However, collisions between everyday objects are almost perfectly elastic when they occur with objects and surfaces that are nearly frictionless, such as with two steel blocks on ice.

Now, to solve problems involving one-dimensional elastic collisions between two objects, we can use the equation for conservation of momentum. First, the equation for conservation of momentum for two objects in a one-dimensional collision is\\
\(\mathbf{p}_{1}+\mathbf{p}_{2}=\mathbf{p}_{1}^{\prime}+\mathbf{p}_{2}^{\prime}\left(\mathbf{F}_{\text {net }}=0\right)\).\\
Substituting the definition of momentum \(\mathbf{p}=m \mathbf{v}\) for each initial and final momentum, we get\\
\(m_{1} \mathrm{v}_{1}+m_{2} \mathrm{v}_{2}=m_{1} \mathrm{v}^{\prime}{ }_{1}+m_{2} \mathrm{v}^{\prime}{ }_{2}\),\\
where the primes ( ' ) indicate values after the collision; In some texts, you may see \(i\) for initial (before collision) and \(f\) for final (after collision). The equation assumes that the mass of each object does not change during the collision.

\section*{Watch Physics}
Momentum: Ice Skater Throws a Ball This video covers an elastic collision problem in which we find the recoil velocity of an ice skater who throws a ball straight forward. To clarify, Sal is using the equation\\
\(m_{\text {ball }} \mathbf{V}_{\text {ball }}+m_{\text {skater }} \mathbf{V}_{\text {skater }}=m_{\text {ball }} \mathbf{v}^{\prime}{ }_{\text {ball }}+m_{\text {skater }} \mathbf{v}_{\text {skater }}^{\prime}\).\\
Click to view content\\
The resultant vector of the addition of vectors \textbackslash overrightarrow\{ \(\backslash \operatorname{text}\{\mathrm{a}\}\}\) and \(\backslash\) overrightarrow \(\{\backslash \operatorname{text}\{\mathrm{b}\}\}\) is \(\backslash\) overrightarrow \(\{\backslash \operatorname{text}\{\mathrm{r}\}\}\). The magnitudes of \textbackslash overrightarrow\{ \textbackslash text\{a\}\}, \textbackslash overrightarrow \(\{\lfloor\operatorname{text}\{\mathrm{b}\}\}\), and ⋅ overrightarrow \(\{\backslash \operatorname{text}\{r\}\}\) are \(\mathrm{A}, \mathrm{B}\), and R , respectively. Which of the following is true?\\
a. \(\mathrm{R} \_\mathrm{x}+\mathrm{R} \_\mathrm{y}=0\)\\
b. A\_x + A\_y \(=\backslash\) overrightarrow \(\{\backslash \operatorname{text}\{\mathrm{A}\}\}\)\\
c. A\_x \(+\mathrm{B} \_\mathrm{y}=\mathrm{B} \_\mathrm{x}+\mathrm{A} \_\mathrm{y}\)\\
d. \(\mathrm{A} \_\mathrm{x}+\mathrm{B} \_\mathrm{x}=\mathrm{R} \_\mathrm{x}\)

Now, let us turn to the second type of collision. An inelastic collision is one in which objects stick together after impact, and kinetic energy is not conserved. This lack of conservation means that the forces between colliding objects may convert kinetic energy to other forms of energy, such as potential energy or thermal energy. The concepts of energy are discussed more thoroughly elsewhere. For inelastic collisions, kinetic energy may be lost in the form of heat. Figure 8.7 shows an example of an inelastic collision. Two objects that have equal masses head toward each other at equal speeds and then stick together. The two objects come to rest after sticking together, conserving momentum but not kinetic energy after they collide. Some of the energy of motion gets converted to thermal energy, or heat.

\begin{figure}[h]
\begin{center}
  \includegraphics[max width=\textwidth]{f6bfaebc-615a-417d-a4a9-d242300d8de3-22}
\captionsetup{labelformat=empty}
\caption{Figure 8.7 A one-dimensional inelastic collision between two objects. Momentum is conserved, but kinetic energy is not conserved. (a) Two objects of equal mass initially head directly toward each other at the same speed. (b) The objects stick together, creating a perfectly inelastic collision. In the case shown in}
\end{center}
\end{figure}

this figure, the combined objects stop; This is not true for all inelastic collisions.\\
Since the two objects stick together after colliding, they move together at the same speed. This lets us simplify the conservation of momentum equation from\\
\(m_{1} \mathrm{v}_{1}+m_{2} \mathrm{v}_{2}=m_{1} \mathrm{v}^{\prime}{ }_{1}+m_{2} \mathrm{v}^{\prime}{ }_{2}\)\\
to\\
\(m_{1} \mathrm{v}_{1}+m_{2} \mathrm{v}_{2}=\left(m_{1}+m_{2}\right) \mathrm{v}^{\prime}\)\\
for inelastic collisions, where \(\mathbf{v}\) is the final velocity for both objects as they are stuck together, either in motion or at rest.

\section*{Teacher Support}
Teacher Support [BL][OL] Review the concept of internal energy. Ask students what they understand by the words elastic and inelastic.\\[0pt]
[AL] Start a discussion about collisions. Ask students to give examples of elastic and inelastic collisions.

\section*{Watch Physics}
Introduction to Momentum This video reviews the definitions of momentum and impulse. It also covers an example of using conservation of momentum to solve a problem involving an inelastic collision between a car with constant velocity and a stationary truck. Note that Sal accidentally gives the unit for impulse as Joules; it is actually \(\mathrm{N} \cdot \mathrm{s}\) or \(\mathrm{k} \cdot \mathrm{gm} / \mathrm{s}\).

Click to view content

\section*{Grasp Check}
How would the final velocity of the car-plus-truck system change if the truck had some initial velocity moving in the same direction as the car? What if the truck were moving in the opposite direction of the car initially? Why?\\
a. If the truck was initially moving in the same direction as the car, the final velocity would be greater. If the truck was initially moving in the opposite direction of the car, the final velocity would be smaller.\\
b. If the truck was initially moving in the same direction as the car, the final velocity would be smaller. If the truck was initially moving in the opposite direction of the car, the final velocity would be greater.\\
c. The direction in which the truck was initially moving would not matter. If the truck was initially moving in either direction, the final velocity would be smaller.\\
d. The direction in which the truck was initially moving would not matter. If the truck was initially moving in either direction, the final velocity would be greater.

\section*{Snap Lab}
Ice Cubes and Elastic Collisions In this activity, you will observe an elastic collision by sliding an ice cube into another ice cube on a smooth surface, so that a negligible amount of energy is converted to heat.

\begin{itemize}
  \item Several ice cubes (The ice must be in the form of cubes.)
  \item A smooth surface
\end{itemize}

Procedure

\begin{enumerate}
  \item Find a few ice cubes that are about the same size and a smooth kitchen tabletop or a table with a glass top.
  \item Place the ice cubes on the surface several centimeters away from each other.
  \item Flick one ice cube toward a stationary ice cube and observe the path and velocities of the ice cubes after the collision. Try to avoid edge-on collisions and collisions with rotating ice cubes.
  \item Explain the speeds and directions of the ice cubes using momentum.
\end{enumerate}

Was the collision elastic or inelastic?\\
a. perfectly elastic\\
b. perfectly inelastic\\
c. Nearly perfect elastic\\
d. Nearly perfect inelastic

\section*{Tips For Success}
Here's a trick for remembering which collisions are elastic and which are inelastic: Elastic is a bouncy material, so when objects bounce off one another in the collision and separate, it is an elastic collision. When they don't, the collision is inelastic.

\section*{Solving Collision Problems}
The Khan Academy videos referenced in this section show examples of elastic and inelastic collisions in one dimension. In one-dimensional collisions, the incoming and outgoing velocities are all along the same line. But what about collisions, such as those between billiard balls, in which objects scatter to the side? These are two-dimensional collisions, and just as we did with two-dimensional\\
forces, we will solve these problems by first choosing a coordinate system and separating the motion into its \(x\) and \(y\) components.

One complication with two-dimensional collisions is that the objects might rotate before or after their collision. For example, if two ice skaters hook arms as they pass each other, they will spin in circles. We will not consider such rotation until later, and so for now, we arrange things so that no rotation is possible. To avoid rotation, we consider only the scattering of point masses-that is, structureless particles that cannot rotate or spin.

We start by assuming that \(\mathbf{F}_{\text {net }}=0\), so that momentum \(\mathbf{p}\) is conserved. The simplest collision is one in which one of the particles is initially at rest. The best choice for a coordinate system is one with an axis parallel to the velocity of the incoming particle, as shown in Figure 8.8. Because momentum is conserved, the components of momentum along the \(x\) - and \(y\)-axes, displayed as \(\mathbf{p}_{x}\) and \(\mathbf{p}_{y}\), will also be conserved. With the chosen coordinate system, \(\mathbf{p}_{y}\) is initially zero and \(\mathbf{p}_{x}\) is the momentum of the incoming particle.

\begin{figure}[h]
\begin{center}
  \includegraphics[max width=\textwidth]{f6bfaebc-615a-417d-a4a9-d242300d8de3-25}
\captionsetup{labelformat=empty}
\caption{Figure 8.8 A two-dimensional collision with the coordinate system chosen so that \(m_{2}\) is initially at rest and \(\mathbf{v}_{1}\) is parallel to the \(x\)-axis.}
\end{center}
\end{figure}

Now, we will take the conservation of momentum equation, \(\mathbf{p}_{1}+\mathbf{p}_{2}=\mathbf{p}_{1}+ \mathbf{p}_{2}\) and break it into its \(x\) and \(y\) components.

Along the \(x\)-axis, the equation for conservation of momentum is\\
\(\mathbf{p}_{1 \mathrm{x}}+\mathbf{p}_{2 \mathrm{x}}=\mathbf{p}_{1 \mathrm{x}}^{\prime}+\mathbf{p}_{2 \mathrm{x}}^{\prime}\).\\
In terms of masses and velocities, this equation is\\
\(m_{1} \mathbf{v}_{1 \mathrm{x}}+m_{2} \mathbf{v}_{2 \mathrm{x}}=m_{1} \mathbf{v}^{\prime}{ }_{1 \mathrm{x}}+m_{2} \mathbf{v}^{\prime}{ }_{2 \mathrm{x}}\).\\
8.3

But because particle 2 is initially at rest, this equation becomes\\
\(m_{1} \mathbf{v}_{1 \mathrm{x}}=m_{1} \mathbf{v}^{\prime}{ }_{1 \mathrm{x}}+m_{2} \mathbf{v}^{\prime}{ }_{2 \mathrm{x}}\).\\
8.4

The components of the velocities along the \(x\)-axis have the form \(\mathbf{v} \cos\). Because particle 1 initially moves along the \(x\)-axis, we find \(\mathbf{v}_{1 x}=\mathbf{v}_{1}\). Conservation of momentum along the \(x\)-axis gives the equation\\
\(m_{1} \mathrm{v}_{1}=m_{1} \mathrm{v}^{\prime}{ }_{1} \cos \theta_{1}+m_{2} \mathrm{v}^{\prime}{ }_{2} \cos \theta_{2}\),\\
where \(\theta_{1}\) and \(\theta_{2}\) are as shown in Figure 8.8.\\
Along the \(y\)-axis, the equation for conservation of momentum is\\
\(\mathbf{p}_{1 \mathrm{y}}+\mathbf{p}_{2 \mathrm{y}}=\mathbf{p}^{\prime}{ }_{1 \mathrm{y}}+\mathbf{p}^{\prime}{ }_{2 \mathrm{y}}\),\\
8.5\\
or\\
\(m_{1} \mathbf{v}_{1 \mathrm{y}}+m_{2} \mathbf{v}_{2 \mathrm{y}}=m_{1} \mathbf{v}^{\prime}{ }_{1 \mathrm{y}}+m_{2} \mathbf{v}^{\prime}{ }_{2 \mathrm{y}}\).\\
8.6

But \(\mathbf{v}_{1} y\) is zero, because particle 1 initially moves along the \(x\)-axis. Because particle 2 is initially at rest, \(\mathbf{v}_{2} y\) is also zero. The equation for conservation of momentum along the \(y\)-axis becomes\\
\(0=m_{1} \mathbf{v}^{\prime}{ }_{1} y+m_{2} \mathbf{v}^{\prime}{ }_{2} y\).\\
8.7

The components of the velocities along the \(y\)-axis have the form \(\mathbf{v} \sin \theta\). Therefore, conservation of momentum along the \(y\)-axis gives the following equation:\\
\(0=m_{1} \mathrm{v}^{\prime}{ }_{1} \sin \theta_{1}+m_{2} \mathrm{v}^{\prime}{ }_{2} \sin \theta_{2}\)

\section*{Teacher Support}
Teacher Support Review conservation of momentum and the equations derived in the previous sections of this chapter. Say that in the problems of this section, all objects are assumed to be point masses. Explain point masses.

\section*{Virtual Physics}
Collision Lab In this simulation, you will investigate collisions on an air hockey table. Place checkmarks next to the momentum vectors and momenta diagram options. Experiment with changing the masses of the balls and the initial speed of ball 1 . How does this affect the momentum of each ball? What\\
about the total momentum? Next, experiment with changing the elasticity of the collision. You will notice that collisions have varying degrees of elasticity, ranging from perfectly elastic to perfectly inelastic.

Click to view content

\section*{Grasp Check}
If you wanted to maximize the velocity of ball 2 after impact, how would you change the settings for the masses of the balls, the initial speed of ball 1 , and the elasticity setting? Why? Hint-Placing a checkmark next to the velocity vectors and removing the momentum vectors will help you visualize the velocity of ball 2, and pressing the More Data button will let you take readings.\\
a. Maximize the mass of ball 1 and initial speed of ball 1 ; minimize the mass of ball 2 ; and set elasticity to 50 percent.\\
b. Maximize the mass of ball 2 and initial speed of ball 1 ; minimize the mass of ball 1 ; and set elasticity to 100 percent.\\
c. Maximize the mass of ball 1 and initial speed of ball 1 ; minimize the mass of ball 2 ; and set elasticity to 100 percent.\\
d. Maximize the mass of ball 2 and initial speed of ball 1 ; minimize the mass of ball 1 ; and set elasticity to 50 percent.

\section*{Worked Example}
Calculating Velocity: Inelastic Collision of a Puck and a Goalie Find the recoil velocity of a 70 kg ice hockey goalie who catches a \(0.150-\mathrm{kg}\) hockey puck slapped at him at a velocity of \(35 \mathrm{~m} / \mathrm{s}\). Assume that the goalie is at rest before catching the puck, and friction between the ice and the puck-goalie system is negligible (see Figure 8.9).

\begin{figure}[h]
\begin{center}
  \includegraphics[max width=\textwidth]{f6bfaebc-615a-417d-a4a9-d242300d8de3-27}
\captionsetup{labelformat=empty}
\caption{Figure 8.9 An ice hockey goalie catches a hockey puck and recoils backward in an inelastic collision.}
\end{center}
\end{figure}

\section*{Strategy}
Momentum is conserved because the net external force on the puck-goalie system is zero. Therefore, we can use conservation of momentum to find the final velocity of the puck and goalie system. Note that the initial velocity of the goalie is zero and that the final velocity of the puck and goalie are the same.

Solution\\
For an inelastic collision, conservation of momentum is\\
\(m_{1} \mathbf{v}_{1}+m_{2} \mathbf{v}_{2}=\left(m_{1}+m_{2}\right) \mathbf{v}^{\prime}\),\\
8.8\\
where \(\mathbf{v}\) is the velocity of both the goalie and the puck after impact. Because the goalie is initially at rest, we know \(\mathbf{v}_{2}=0\). This simplifies the equation to \(m_{1} \mathbf{v}_{1}=\left(m_{1}+m_{2}\right) \mathbf{v}^{\prime}\).\\
8.9

Solving for \(\mathbf{v}\) yields\\
\(\mathrm{v}^{\prime}=\left(\frac{m_{1}}{m_{1}+m_{2}}\right) \mathrm{v}_{1}\).\\
8.10

Entering known values in this equation, we get

\[
\begin{aligned}
\mathbf{v}^{\prime} & =\left(\frac{0.150 \mathrm{~kg}}{70.0 \mathrm{~kg}+0.150 \mathrm{~kg}}\right)(35 \mathrm{~m} / \mathrm{s}) \\
& =7.48 \times 10^{-2} \mathrm{~m} / \mathrm{s}
\end{aligned}
\]

8.11

Discussion\\
This recoil velocity is small and in the same direction as the puck's original velocity.

\section*{Worked Example}
Calculating Final Velocity: Elastic Collision of Two Carts Two hard, steel carts collide head-on and then ricochet off each other in opposite directions on a frictionless surface (see Figure 8.10). Cart 1 has a mass of 0.350 kg and an initial velocity of \(2 \mathrm{~m} / \mathrm{s}\). Cart 2 has a mass of 0.500 kg and an initial velocity of \(-0.500 \mathrm{~m} / \mathrm{s}\). After the collision, cart 1 recoils with a velocity of \(-4 \mathrm{~m} / \mathrm{s}\). What is the final velocity of cart 2 ?

\begin{figure}[h]
\begin{center}
  \includegraphics[max width=\textwidth]{f6bfaebc-615a-417d-a4a9-d242300d8de3-29}
\captionsetup{labelformat=empty}
\caption{Figure 8.10 Two carts collide with each other in an elastic collision.}
\end{center}
\end{figure}

\section*{Strategy}
Since the track is frictionless, \(\mathbf{F}_{\text {net }}=0\) and we can use conservation of momentum to find the final velocity of cart 2 .

Solution\\
As before, the equation for conservation of momentum for a one-dimensional elastic collision in a two-object system is\\
\(m_{1} \mathbf{v}_{1}+m_{2} \mathbf{v}_{2}=m_{1} \mathbf{v}_{1}^{\prime}+m_{2} \mathbf{v}^{\prime}{ }_{2}\).\\
8.12

The only unknown in this equation is \(\mathbf{v}_{2}\). Solving for \(\mathbf{v}_{2}\) and substituting known values into the previous equation yields

\[
\begin{aligned}
\mathbf{v}_{2}^{\prime} & =\frac{m_{1} \mathbf{v}_{1}+m_{2} \mathbf{v}_{2}-m_{1} \mathbf{v}_{1}^{\prime}}{m_{2}} \\
& =\frac{(0.350 \mathrm{~kg})(2.00 \mathrm{~m} / \mathrm{s})+(0.500 \mathrm{~kg})(-0.500 \mathrm{~m} / \mathrm{s})-(0.350 \mathrm{~kg})(-4.00 \mathrm{~m} / \mathrm{s})}{0.500 \mathrm{~kg}} \\
& =3.70 \mathrm{~m} / \mathrm{s}
\end{aligned}
\]

\subsection*{8.13}
Discussion\\
The final velocity of cart 2 is large and positive, meaning that it is moving to the right after the collision.

\section*{Worked Example}
Calculating Final Velocity in a Two-Dimensional Collision Suppose the following experiment is performed (Figure 8.11). An object of mass 0.250 \(\mathrm{kg}\left(m_{1}\right)\) is slid on a frictionless surface into a dark room, where it strikes an initially stationary object of mass \(0.400 \mathrm{~kg}\left(m_{2}\right)\). The 0.250 kg object emerges from the room at an angle of \(45^{\mathrm{o}}\) with its incoming direction. The speed of the 0.250 kg object is originally \(2 \mathrm{~m} / \mathrm{s}\) and is \(1.50 \mathrm{~m} / \mathrm{s}\) after the collision. Calculate the magnitude and direction of the velocity ( \(v_{2}\) and \(\theta_{2}\) ) of the 0.400 kg object after the collision.

\begin{figure}[h]
\begin{center}
  \includegraphics[max width=\textwidth]{f6bfaebc-615a-417d-a4a9-d242300d8de3-30}
\captionsetup{labelformat=empty}
\caption{Figure 8.11 The incoming object of mass \(m_{1}\) is scattered by an initially stationary object. Only the stationary object's mass \(m_{2}\) is known. By measuring the angle and speed at which the object of mass \(m_{1}\) emerges from the room, it is possible to calculate the magnitude and direction of the initially stationary object's velocity after the collision.}
\end{center}
\end{figure}

\section*{Strategy}
Momentum is conserved because the surface is frictionless. We chose the coordinate system so that the initial velocity is parallel to the \(x\)-axis, and conservation of momentum along the \(x\) - and \(y\)-axes applies.

Everything is known in these equations except \(\mathbf{v}_{2}\) and 2 , which we need to find. We can find two unknowns because we have two independent equations-the equations describing the conservation of momentum in the \(x\) and \(y\) directions.

Solution

First, we'll solve both conservation of momentum equations ( \(m_{1} \mathrm{v}_{1}= m_{1} \mathrm{v}_{1}^{\prime} \cos \theta_{1}+m_{2} \mathrm{v}_{2}^{\prime} \cos \theta_{2}\) and \(\left.0=m_{1} \mathrm{v}_{1}^{\prime} \sin \theta_{1}+m_{2} \mathrm{v}_{2}^{\prime} \sin \theta_{2}\right)\) for \(\mathbf{v}_{2} \sin \theta_{2}\).\\
For conservation of momentum along x -axis, let's substitute \(\sin \theta_{2} / \tan \theta_{2}\) for \(\cos \theta_{2}\) so that terms may cancel out later on. This comes from rearranging the definition of the trigonometric identity \(\tan \theta=\sin \theta / \cos \theta\). This gives us\\
\(m_{1} \mathbf{v}_{1}=m_{1} \mathbf{v}_{1}^{\prime} \cos \theta_{1}+m_{2} \mathbf{v}_{2}^{\prime} \frac{\sin \theta_{2}}{\tan \theta_{2}}\).\\
8.14

Solving for \(\mathbf{v}_{2} \sin \theta_{2}\) yields\\
\(\mathbf{v}_{2}^{\prime} \sin \theta_{2}=\frac{\left(m_{1} \mathbf{v}_{1}-m_{1} \mathbf{v}_{1}^{\prime} \cos \theta_{1}\right)\left(\tan \theta_{2}\right)}{m_{2}}\).\\
8.15

For conservation of momentum along \(y\)-axis, solving for \(\mathbf{v}_{2} \sin \theta_{2}\) yields \(\mathbf{v}_{2}^{\prime} \sin \theta_{2}=\frac{-\left(m_{1} \mathbf{v}_{1}^{\prime} \sin \theta_{1}\right)}{m_{2}}\).\\
8.16

Since both equations equal \(\mathbf{v}_{2} \sin \theta_{2}\), we can set them equal to one another, yielding\\
\(\frac{\left(m_{1} \mathbf{v}_{1}-m_{1} \mathbf{v}_{1}^{\prime} \cos \theta_{1}\right)\left(\tan \theta_{2}\right)}{m_{2}}=\frac{-\left(m_{1} \mathbf{v}_{1}^{\prime} \sin \theta_{1}\right)}{m_{2}}\).\\
8.17

Solving this equation for \(\tan \theta_{2}\), we get\\
\(\tan \theta_{2}=\frac{\mathbf{v}_{1}^{\prime} \sin \theta_{1}}{\mathbf{v}_{1}^{\prime} \cos \theta_{1}-\mathbf{v}_{1}}\).\\
8.18

Entering known values into the previous equation gives\\
\(\tan \theta_{2}=\frac{(1.50)(0.707)}{(1.50)(0.707)-2.00}=-1.129\).\\
8.19

Therefore,\\
\(\theta_{2}=\tan ^{-1}(-1.129)=312^{0}\).\\
8.20

Since angles are defined as positive in the counterclockwise direction, \(m_{2}\) is scattered to the right.

We'll use the conservation of momentum along the \(y\)-axis equation to solve for \(\mathrm{v}_{2}\).\\
\(\mathbf{v}_{2}^{\prime}=-\frac{m_{1}}{m_{2}} \mathbf{v}_{1}^{\prime} \frac{\sin \theta_{1}}{\sin \theta_{2}}\)\\
8.21

Entering known values into this equation gives\\
\(\mathrm{v}^{\prime}{ }_{2}=-\frac{(0.250)}{(0.400)}(1.50)\left(\frac{0.7071}{-0.7485}\right)\).\\
8.22

Therefore,\\
\(\mathbf{v}^{\prime}{ }_{2}=0.886 \mathrm{~m} / \mathrm{s}\).\\
8.23

Discussion\\
Either equation for the \(x\) - or \(y\)-axis could have been used to solve for \(\mathbf{v}_{2}\), but the equation for the \(y\)-axis is easier because it has fewer terms.

\section*{Practice Problems}
10.

In an elastic collision, an object with momentum \(25 \backslash, \backslash \operatorname{text}\{\mathrm{~kg}\} \backslash\) cdot \textbackslash text\{m/s\} collides with another object moving to the right that has a momentum \(35 \backslash, \backslash \operatorname{text}\{\mathrm{~kg}\} \backslash \operatorname{cdot} \backslash \operatorname{text}\{\mathrm{m} / \mathrm{s}\}\). After the collision, both objects are still moving to the right, but the first object's momentum changes to \(10 \backslash, \backslash \operatorname{text}\{\mathrm{~kg}\} \backslash \operatorname{cdot} \backslash \operatorname{text}\{\mathrm{m} / \mathrm{s}\}\). What is the final momentum of the second object?\\
a. \(10 \backslash, \backslash \operatorname{text}\{\mathrm{~kg}\} \backslash \operatorname{cdot} \backslash \operatorname{text}\{\mathrm{m} / \mathrm{s}\}\)\\
b. \(20 \backslash, \backslash \operatorname{text}\{\mathrm{~kg}\} \backslash \operatorname{cdot} \backslash \operatorname{text}\{\mathrm{m} / \mathrm{s}\}\)\\
c. \(35 \backslash, \backslash \operatorname{text}\{\mathrm{~kg}\} \backslash \operatorname{cdot} \backslash \operatorname{text}\{\mathrm{m} / \mathrm{s}\}\)\\
d. \(50 \backslash, \backslash \operatorname{text}\{\mathrm{~kg}\} \backslash \operatorname{cdot} \backslash \operatorname{text}\{\mathrm{m} / \mathrm{s}\}\)\\
11.

In an elastic collision, an object with momentum \(25 \mathrm{~kg} \mathrm{~m} / \mathrm{s}\) collides with another that has a momentum \(35 \mathrm{~kg} \mathrm{~m} / \mathrm{s}\). The first object's momentum changes to \(10 \mathrm{~kg} \mathrm{~m} / \mathrm{s}\). What is the final momentum of the second object?\\
a. \(10 \mathrm{~kg} \mathrm{~m} / \mathrm{s}\)\\
b. \(20 \mathrm{~kg} \mathrm{~m} / \mathrm{s}\)\\
c. \(35 \mathrm{~kg} \mathrm{~m} / \mathrm{s}\)\\
d. \(50 \mathrm{~kg} \mathrm{~m} / \mathrm{s}\)

\section*{Check Your Understanding}
12.

What is an elastic collision?\\
a. An elastic collision is one in which the objects after impact are deformed permanently.\\
b. An elastic collision is one in which the objects after impact lose some of their internal kinetic energy.\\
c. An elastic collision is one in which the objects after impact do not lose any of their internal kinetic energy.\\
d. An elastic collision is one in which the objects after impact become stuck together and move with a common velocity.\\
13.

Are perfectly elastic collisions possible?\\
a. Perfectly elastic collisions are not possible.\\
b. Perfectly elastic collisions are possible only with subatomic particles.\\
c. Perfectly elastic collisions are possible only when the objects stick together after impact.\\
d. Perfectly elastic collisions are possible if the objects and surfaces are nearly frictionless.\\
14.

What is the equation for conservation of momentum for two objects in a onedimensional collision?\\
a. \(\mathbf{p}_{1}+\mathbf{p}_{1}=\mathbf{p}_{2}+\mathbf{p}_{2}\)\\
b. \(\mathbf{p}_{1}+\mathbf{p}_{2}=\mathbf{p}_{1}+\mathbf{p}_{2}\)\\
c. \(\mathbf{p}_{1}-\mathbf{p}_{2}=\mathbf{p}_{1}-\mathbf{p}_{2}\)\\
d. \(\mathbf{p}_{1}+\mathbf{p}_{2}+\mathbf{p}_{1}+\mathbf{p}_{2}=0\)

\section*{Teacher Support}
Teacher Support Use the Check Your Understanding questions to assess whether students master the learning objectives of this section. If students are struggling with a specific objective, the assessment will help identify which objective is causing the problem and direct students to the relevant content.

\section*{Ke Terms}
angular momentum the product of the moment of inertia and angular velocity\\
change in momentum the difference between the final and initial values of momentum; the mass times the change in velocity\\
elastic collision collision in which objects separate after impact and kinetic energy is conserved\\
impulse average net external force multiplied by the time the force acts; equal to the change in momentum\\
impulse-momentum theorem the impulse, or change in momentum, is the product of the net external force and the time over which the force acts\\
inelastic collision collision in which objects stick together after impact and kinetic energy is not conserved\\
isolated system system in which the net external force is zero\\
law of conservation of momentum when the net external force is zero, the total momentum of the system is conserved or constant\\
linear momentum the product of a system's mass and velocity\\
point masses structureless particles that cannot rotate or spin\\
recoil backward movement of an object caused by the transfer of momentum from another object in a collision

Ke Equations\\
8.1 Linear Momentum, Force, and Impulse

\subsection*{8.2 Conservation of Momentum}
8.3 Elastic and Inelastic Collisions

\section*{Section Summar}
\subsection*{8.1 Linear Momentum, Force, and Impulse}
\begin{itemize}
  \item Linear momentum, often referenced as momentum for short, is defined as the product of a system's mass multiplied by its velocity, \(\mathbf{p}=m \mathbf{v}\).
  \item The SI unit for momentum is \(\mathrm{kg} \mathrm{m} / \mathrm{s}\).
  \item Newton's second law of motion in terms of momentum states that the net external force equals the change in momentum of a system divided by the time over which it changes, \(\mathbf{F}_{\text {net }}=\frac{\Delta \mathbf{p}}{\Delta t}\).
  \item Impulse is the average net external force multiplied by the time this force acts, and impulse equals the change in momentum, \(\Delta \mathbf{p}=\mathbf{F}_{\text {net }} \Delta t\).
  \item Forces are usually not constant over a period of time, so we use the average of the force over the time it acts.
\end{itemize}

\subsection*{8.2 Conservation of Momentum}
\begin{itemize}
  \item The law of conservation of momentum is written \(\mathbf{p}_{\text {tot }}=\) constant or \(\mathbf{p}_{\text {tot }} =\mathbf{p}_{\text {tot }}\) (isolated system), where \(\mathbf{p}_{\text {tot }}\) is the initial total momentum and \(\mathbf{p}_{\text {tot }}\) is the total momentum some time later.
  \item In an isolated system, the net external force is zero.
  \item Conservation of momentum applies only when the net external force is zero, within the defined system.
\end{itemize}

\subsection*{8.3 Elastic and Inelastic Collisions}
\begin{itemize}
  \item If objects separate after impact, the collision is elastic; If they stick together, the collision is inelastic.
  \item Kinetic energy is conserved in an elastic collision, but not in an inelastic collision.
  \item The approach to two-dimensional collisions is to choose a convenient coordinate system and break the motion into components along perpendicular axes. Choose a coordinate system with the -axis parallel to the velocity of the incoming particle.
  \item Two-dimensional collisions of point masses, where mass 2 is initially at rest, conserve momentum along the initial direction of mass 1 , or the axis, and along the direction perpendicular to the initial direction, or the -axis.
  \item Point masses are structureless particles that cannot spin.
\end{itemize}

\end{document}