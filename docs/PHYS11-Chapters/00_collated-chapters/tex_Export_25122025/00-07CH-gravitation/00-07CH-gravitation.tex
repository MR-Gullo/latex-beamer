\documentclass[10pt]{article}
\usepackage[utf8]{inputenc}
\usepackage[T1]{fontenc}
\usepackage{graphicx}
\usepackage[export]{adjustbox}
\graphicspath{ {./images/} }
\usepackage{caption}
\usepackage{amsmath}
\usepackage{amsfonts}
\usepackage{amssymb}
\usepackage[version=4]{mhchem}
\usepackage{stmaryrd}

\begin{document}
\captionsetup{singlelinecheck=false}
\begin{figure}[h]
\begin{center}
  \includegraphics[max width=\textwidth]{c94e422a-deb3-4e1e-8f32-855dd556856c-01}
\captionsetup{labelformat=empty}
\caption{Figure 7.1 Johannes Kepler (left) showed how the planets move, and Isaac Newton (right) discovered that gravitational force caused them to move that way. ((left) unknown, Public Domain; (right) Sir Godfrey Kneller, Public Domain)}
\end{center}
\end{figure}

\section*{Chapter Outline}
7.1 Kepler's Laws of Planetary Motion\\
7.2 Newton's Law of Universal Gravitation and Einstein's Theory of General Relativity

\section*{Introduction}
\section*{Teacher Support}
Teacher Support Physics learning objectives come from 112.39 (c) Knowledge and Skills

\section*{Teacher Support}
Teacher Support Contrast the type of work that each scientist did. Both were important and tackled difficult problems. Kepler found patterns in a mountain of data. Newton found the underlying cause of those patterns.

What do a falling apple and the orbit of the moon have in common? You will learn in this chapter that each is caused by gravitational force. The motion\\
of all celestial objects, in fact, is determined by the gravitational force, which depends on their mass and separation.

Johannes Kepler discovered three laws of planetary motion that all orbiting planets and moons follow. Years later, Isaac Newton found these laws useful in developing his law of universal gravitation. This law relates gravitational force to the masses of objects and the distance between them. Many years later still, Albert Einstein showed there was a little more to the gravitation story when he published his theory of general relativity.

\section*{Teacher Support}
Teacher Support Before students begin this chapter, it is useful to review the following concepts:

\begin{itemize}
  \item Using significant figures in calculations-Demonstrate how to use the proper number of significant figures when adding and multiplying.
  \item Review scientific notation as related to significant figures.
  \item Converting units-Demonstrate how to convert from \(\mathrm{km} / \mathrm{h}\) to \(\mathrm{m} / \mathrm{s}\).
  \item Review dimensional analysis. For example, N is equivalent to \(\mathrm{kg} \mathrm{m} / \mathrm{s}^{2}\).
  \item Explain that metric units clearly distinguish between mass and weight, but that the commonly used English units do not.
  \item Calculating average-Demonstrate how to average two numbers.
  \item Review manipulation of formulas so that they may be expressed in terms of the unknown.
  \item Review Newton's laws of motion.
\end{itemize}

\subsection*{7.1 Kepler's Laws of Planetary Motion}
\section*{Section Learning Objectives}
By the end of this section, you will be able to do the following:

\begin{itemize}
  \item Explain Kepler's three laws of planetary motion
  \item Apply Kepler's laws to calculate characteristics of orbits
\end{itemize}

\section*{Teacher Support}
Teacher Support The learning objectives in this section will help your students master the following standards:

\begin{itemize}
  \item (4) Science concepts. The student knows and applies the laws governing motion in a variety of situations. The student is expected to:
  \item (C) analyze and describe accelerated motion in two dimensions using equations, including projectile and circular examples.
\end{itemize}

In this section students will apply Kepler's laws of planetary motion to objects in the solar system.\\[0pt]
[BL][OL] Discuss the historical setting in which Kepler worked. Most people still thought Earth was the center of the universe, and yet Kepler not only knew that the planets circled the sun, he found patterns in the paths they followed. What would it be like to be that far ahead of almost everyone? A fascinating description of this is given in the program Cosmos with Carl Sagan (Episode 3, Harmony of the Worlds).\\[0pt]
[AL] Explain that Kepler's laws were laws and not theories. Laws describe patterns in nature that always repeat themselves under the same set of conditions. Theories provide an explanation for the patterns. Kepler provided no explanation.

\section*{Section Key Terms}
\section*{Concepts Related to Kepler's Laws of Planetary Motion}
Examples of orbits abound. Hundreds of artificial satellites orbit Earth together with thousands of pieces of debris. The moon's orbit around Earth has intrigued humans from time immemorial. The orbits of planets, asteroids, meteors, and comets around the sun are no less interesting. If we look farther, we see almost unimaginable numbers of stars, galaxies, and other celestial objects orbiting one another and interacting through gravity.

All these motions are governed by gravitational force. The orbital motions of objects in our own solar system are simple enough to describe with a few fairly simple laws. The orbits of planets and moons satisfy the following two conditions:

\begin{itemize}
  \item The mass of the orbiting object, \(m\), is small compared to the mass of the object it orbits, \(M\).
  \item The system is isolated from other massive objects.
\end{itemize}

\section*{Teacher Support}
Teacher Support [OL] Ask the students to explain the criteria to see if they understand relative mass and isolated systems.

Based on the motion of the planets about the sun, Kepler devised a set of three classical laws, called Kepler's laws of planetary motion, that describe the orbits of all bodies satisfying these two conditions:

\begin{enumerate}
  \item The orbit of each planet around the sun is an ellipse with the sun at one focus.
  \item Each planet moves so that an imaginary line drawn from the sun to the planet sweeps out equal areas in equal times.
  \item The ratio of the squares of the periods of any two planets about the sun is equal to the ratio of the cubes of their average distances from the sun.
\end{enumerate}

These descriptive laws are named for the German astronomer Johannes Kepler (1571-1630). He devised them after careful study (over some 20 years) of a large amount of meticulously recorded observations of planetary motion done by Tycho Brahe (1546-1601). Such careful collection and detailed recording of methods and data are hallmarks of good science. Data constitute the evidence from which new interpretations and meanings can be constructed. Let's look closer at each of these laws.

\section*{Teacher Support}
Teacher Support [BL] Relate orbit to year and rotation to day. Be sure that students know that an object rotates on its axis and revolves around a parent body as it follows its orbit.\\[0pt]
[OL] See how many levels of orbital motion the students know and fill in the ones they don't. For example, moons orbit around planets; planets around stars; stars around the center of the galaxy, etc.\\[0pt]
[AL] From the point of view of Earth, which objects appear (incorrectly) to be orbiting Earth (stars, the sun, galaxies) and which can be seen to be orbiting parent bodies (the moon, moons of other planets, stars in other galaxies)?

Kepler's First Law The orbit of each planet about the sun is an ellipse with the sun at one focus, as shown in Figure 7.2. The planet's closest approach\\
to the sun is called perihelion and its farthest distance from the sun is called aphelion.

\begin{figure}[h]
\begin{center}
  \includegraphics[max width=\textwidth]{c94e422a-deb3-4e1e-8f32-855dd556856c-05}
\captionsetup{labelformat=empty}
\caption{Figure 7.2 (a) An ellipse is a closed curve such that the sum of the distances from a point on the curve to the two foci ( \(f_{1}\) and \(f_{2}\) ) is constant. (b) For any closed orbit, \(m\) follows an elliptical path with \(M\) at one focus. (c) The aphelion (ra) is the furthest distance between the planet and the sun, while the perihelion (rp) is the closest distance from the sun.}
\end{center}
\end{figure}

\section*{Teacher Support}
Teacher Support [AL] Ask for a definition of planet. Prepare to discuss Pluto's demotion if it comes up. Discuss the first criterion in terms of center of rotation of a moon-planet system. Explain that for all planet-moon systems in the solar system, the center of rotation is within the planet. This is not true for Pluto and its largest moon, Charon, because their masses are similar enough that they rotate around a point in space between them.

If you know the aphelion ( \(r_{\mathrm{a}}\) ) and perihelion ( \(r_{\mathrm{p}}\) ) distances, then you can calculate the semi-major axis ( \(a\) ) and semi-minor axis ( \(b\) ).\\
\(a=\frac{\left(r_{a}+r_{p}\right)}{2}\)\\
\(b=\sqrt{r_{a} r_{p}}\)

\section*{Teacher Support}
Teacher Support [AL] If any students are interested and proficient in algebra and geometry, ask them to derive a formula that relates the length of the string and the distance between pins to the major and minor axes of an ellipse. Explain that this is a real world problem for workers who design elliptical tabletops and mirrors.\\[0pt]
[BL][OL] Impress upon the students that Kepler had to crunch an enormous amount of data and that all his calculations had to be done by hand. Ask students to think of similar projects where scientists found order in a daunting amount of data (the periodic table, DNA structure, climate models, etc.).

\section*{Teacher Demonstration}
Demonstrate the pins and string method of drawing an ellipse, as shown in Figure 7.3, or have the students try it at home or in class.

Ask students: Why does the string and pin method create a shape that conforms to Kepler's second law? That is, why is the shape an ellipse?

\section*{Teacher Support}
Teacher Support Explain that the pins are the foci and explain what each of the three sections of string represents. Note that the pencil represents a planet and one of the pins represents the sun.

\begin{figure}[h]
\begin{center}
  \includegraphics[max width=\textwidth]{c94e422a-deb3-4e1e-8f32-855dd556856c-07}
\captionsetup{labelformat=empty}
\caption{Figure 7.3 You can draw an ellipse as shown by putting a pin at each focus, and then placing a loop of string around a pen and the pins and tracing a line on the paper.}
\end{center}
\end{figure}

Kepler's Second Law Each planet moves so that an imaginary line drawn from the sun to the planet sweeps out equal areas in equal times, as shown in Figure 7.4.

\begin{figure}[h]
\begin{center}
  \includegraphics[max width=\textwidth]{c94e422a-deb3-4e1e-8f32-855dd556856c-08}
\captionsetup{labelformat=empty}
\caption{Figure 7.4 The shaded regions have equal areas. The time for \(m\) to go from A to B is the same as the time to go from C to D and from E to F . The mass \(m\) moves fastest when it is closest to \(M\). Kepler's second law was originally devised for planets orbiting the sun, but it has broader validity.}
\end{center}
\end{figure}

\section*{Teacher Support}
Teacher Support Ask the students to imagine how complicated it would be to describe the motion of the planets mathematically, if it is assumed that Earth is stationary. And yet, people tried to do this for hundreds of years, while\\
overlooking the simple explanation that all planets circle the sun.\\[0pt]
[OL] Ask students to use this figure to understand why planets and comets travel faster when they are closer to the sun. Explain that time intervals and areas are constant, but both velocity and distance from the sun vary.

\section*{Tips For Success}
Note that while, for historical reasons, Kepler's laws are stated for planets orbiting the sun, they are actually valid for all bodies satisfying the two previously stated conditions.

Kepler's Third Law The ratio of the periods squared of any two planets around the sun is equal to the ratio of their average distances from the sun cubed. In equation form, this is\\
\(\frac{T_{1}^{2}}{T_{2}^{2}}=\frac{r_{1}^{3}}{r_{2}^{3}}\),\\
where \(T\) is the period (time for one orbit) and \(r\) is the average distance (also called orbital radius). This equation is valid only for comparing two small masses orbiting a single large mass. Most importantly, this is only a descriptive equation; it gives no information about the cause of the equality.

\section*{Teacher Support}
Teacher Support [BL] See if students can rearrange this equation to solve for any one of the variables when the other three are known.\\[0pt]
[AL] Show a solution for one of the periods \(T\) or radii \(r\) and ask students to interpret the fractional powers on the right hand side of the equation.\\[0pt]
[OL] Emphasize that this approach only works for two satellites orbiting the same parent body. The parent body must be the same because \(r^{2} / T^{2}= G M /\left(4 \pi^{2}\right)\) and \(M\) is the mass of the parent body. If \(M\) changes, the ratio \(r^{3} / T^{2}\) also changes.

\section*{Links To Physics}
History: Ptolemy vs. Copernicus Before the discoveries of Kepler, Copernicus, Galileo, Newton, and others, the solar system was thought to revolve around Earth as shown in Figure 7.5 (a). This is called the Ptolemaic model, named for the Greek philosopher Ptolemy who lived in the second century AD. The Ptolemaic model is characterized by a list of facts for the motions of planets, with no explanation of cause and effect. There tended to be a different rule for each heavenly body and a general lack of simplicity.

Figure 7.5 (b) represents the modern or Copernican model. In this model, a small set of rules and a single underlying force explain not only all planetary motion in the solar system, but also all other situations involving gravity. The breadth and simplicity of the laws of physics are compelling.

\begin{figure}[h]
\begin{center}
  \includegraphics[max width=\textwidth]{c94e422a-deb3-4e1e-8f32-855dd556856c-10}
\captionsetup{labelformat=empty}
\caption{Figure 7.5 (a) The Ptolemaic model of the universe has Earth at the center with the moon, the planets, the sun, and the stars revolving about it in complex circular paths. This geocentric (Earth-centered) model, which can be made progressively more accurate by adding more circles, is purely descriptive, containing no hints about the causes of these motions. (b) The Copernican heliocentric (sun-centered) model is a simpler and more accurate model.}
\end{center}
\end{figure}

Nicolaus Copernicus (1473-1543) first had the idea that the planets circle the sun, in about 1514 . It took him almost 20 years to work out the mathematical details for his model. He waited another 10 years or so to publish his work. It is thought he hesitated because he was afraid people would make fun of his theory. Actually, the reaction of many people was more one of fear and anger. Many people felt the Copernican model threatened their basic belief system. About 100 years later, the astronomer Galileo was put under house arrest for providing evidence that planets, including Earth, orbited the sun. In all, it took almost 300 years for everyone to admit that Copernicus had been right all along.

\section*{Grasp Check}
Explain why Earth does actually appear to be the center of the solar system.\\
a. Earth appears to be the center of the solar system because Earth is at the center of the universe, and everything revolves around it in a circular orbit.\\
b. Earth appears to be the center of the solar system because, in the reference frame of Earth, the sun, moon, and planets all appear to move across the sky as if they were circling Earth.\\
c. Earth appears to be at the center of the solar system because Earth is at the center of the solar system and all the heavenly bodies revolve around it.\\
d. Earth appears to be at the center of the solar system because Earth is located at one of the foci of the elliptical orbit of the sun, moon, and other planets.

\section*{Teacher Support}
Teacher Support Introduce the historical debate around the geocentric versus the heliocentric view of the universe. Stress how controversial this debate was at the time. Explain that this was important to people because their world view and cultural beliefs were at stake.

\section*{Virtual Physics}
Acceleration This simulation allows you to create your own solar system so that you can see how changing distances and masses determines the orbits of planets. Click Help for instructions.

Click to view content\\
When the central object is off center, how does the speed of the orbiting object vary?\\
a. The orbiting object moves fastest when it is closest to the central object and slowest when it is farthest away.\\
b. The orbiting object moves slowest when it is closest to the central object and fastest when it is farthest away.\\
c. The orbiting object moves with the same speed at every point on the circumference of the elliptical orbit.\\
d. There is no relationship between the speed of the object and the location of the planet on the circumference of the orbit.

\section*{Teacher Support}
Teacher Support Give the students ample time to manipulate this animation. It may take some time to get the parameters adjusted so that they can see how\\
mass and eccentricity affect the orbit. Initially, the planet is likely to disappear off the screen or crash into the sun.

\section*{Calculations Related to Kepler's Laws of Planetary Motion}
Kepler's First Law Refer back to Figure 7.2 (a). Notice which distances are constant. The foci are fixed, so distance \(f_{1} f_{2}\) is a constant. The definition of an ellipse states that the sum of the distances \(f_{1} m+m f_{2}\) is also constant. These two facts taken together mean that the perimeter of triangle \(\Delta f_{1} m f_{2}\) must also be constant. Knowledge of these constants will help you determine positions and distances of objects in a system that includes one object orbiting another.

Kepler's Second Law Refer back to Figure 7.4. The second law says that the segments have equal area and that it takes equal time to sweep through each segment. That is, the time it takes to travel from A to B equals the time it takes to travel from C to D , and so forth. Velocity \(\mathbf{v}\) equals distance \(d\) divided by time \(t: \mathbf{v}=d / t\). Then, \(t=d / \mathbf{v}\), so distance divided by velocity is also a constant. For example, if we know the average velocity of Earth on June 21 and December 21, we can compare the distance Earth travels on those days.

The degree of elongation of an elliptical orbit is called its eccentricity ( \(e\) ). Eccentricity is calculated by dividing the distance \(f\) from the center of an ellipse to one of the foci by half the long axis \(a\).\\
\((e)=f / a\)\\
7.1

When \(e=0\), the ellipse is a circle.\\
The area of an ellipse is given by \(A=\pi a b\), where \(b\) is half the short axis. If you know the axes of Earth's orbit and the area Earth sweeps out in a given period of time, you can calculate the fraction of the year that has elapsed.

\section*{Teacher Support}
Teacher Support [OL] Review the definitions of major and minor axes, semimajor and semi-minor axes, and distance \(f\). The major axis is the length of the ellipse and passes through both foci. The minor axis is the width of the ellipse and is perpendicular to the major axis. The semi-major and semi-minor axes are half of the major and minor axes, respectively.

\section*{Worked Example}
Kepler's First Law At its closest approach, a moon comes within 200,000 km of the planet it orbits. At that point, the moon is \(300,000 \mathrm{~km}\) from the\\
other focus of its orbit, \(f_{2}\). The planet is focus \(f_{1}\) of the moon's elliptical orbit. How far is the moon from the planet when it is \(260,000 \mathrm{~km}\) from \(f_{2}\) ?

\section*{Strategy}
Show and label the ellipse that is the orbit in your solution. Picture the triangle \(f_{1} \mathrm{~m} f_{2}\) collapsed along the major axis and add up the lengths of the three sides. Find the length of the unknown side of the triangle when the moon is 260,000 km from \(f_{2}\).

Solution\\
Perimeter of \(f_{1} m f_{2}=200,000 \mathrm{~km}+100,000 \mathrm{~km}+300,000 \mathrm{~km}=600,000 \mathrm{~km}\).\\
\(m f_{1}=600,000 \mathrm{~km}-(100,000 \mathrm{~km}+260,000 \mathrm{~km})=240,000 \mathrm{~km}\).\\
Discussion\\
The perimeter of triangle \(f_{1} m f_{2}\) must be constant because the distance between the foci does not change and Kepler's first law says the orbit is an ellipse. For any ellipse, the sum of the two sides of the triangle, which are \(f_{1} m\) and \(m f_{2}\), is constant.

\section*{Teacher Support}
Teacher Support Walk the students through the process of mentally collapsing the \(f_{1} m f_{2}\) at the end of the major axis to reveal what the three sides of the triangle \(f_{1} m f_{2}\) are equal to. Picture the sections of the string as the pencil approaches the major axis. This distance \(f_{1} f_{2}\) remains constant, \(f_{1} m\) is the distance from \(f_{1}\) to the end of the major axis, and \(m f_{2}\) is \(f_{1} m+f_{1} f_{2}\).\\[0pt]
[OL] Have students relate eccentricity, distance between foci, and shape of orbit.\\[0pt]
[AL] Ask for examples of orbits with high eccentricity (comets, Pluto) and low eccentricity (moon, Earth).

\section*{Worked Example}
Kepler's Second Law Figure 7.6 shows the major and minor axes of an ellipse. The semi-major and semi-minor axes are half of these, respectively.

\begin{figure}[h]
\begin{center}
  \includegraphics[max width=\textwidth]{c94e422a-deb3-4e1e-8f32-855dd556856c-14}
\captionsetup{labelformat=empty}
\caption{Figure 7.6 The major axis is the length of the ellipse, and the minor axis is the width of the ellipse. The semi-major axis is half the major axis, and the semi-minor axis is half the minor axis.}
\end{center}
\end{figure}

Earth's orbit is very slightly elliptical, with a semi-major axis of \(1.49598 \times 10^{8}\) km and a semi-minor axis of \(1.49577 \times 10^{8} \mathrm{~km}\). If Earth's period is 365.26 days, what area does an Earth-to-sun line sweep past in one day?

\section*{Strategy}
Each day, Earth sweeps past an equal-sized area, so we divide the total area by the number of days in a year to find the area swept past in one day. For total area use \(A=\pi a b\). Calculate \(A\), the area inside Earth's orbit and divide by the number of days in a year (i.e., its period).

Solution

\[
\begin{aligned}
\text { area per day } & =\frac{\text { total area }}{\text { total number of days }} \\
& =\frac{\pi a b}{365 \mathrm{~d}} \\
& =\frac{\pi\left(1.496 \times 10^{8} \mathrm{~km}\right)\left(1.496 \times 10^{3} \mathrm{~km}\right)}{365 \mathrm{~d}} \\
& =1.93 \times 10^{14} \mathrm{~km}^{2} / \mathrm{d}
\end{aligned}
\]

\section*{7.2}
The area swept out in one day is thus \(1.93 \times 10^{14} \mathrm{~km}^{2}\).\\
Discussion\\
The answer is based on Kepler's law, which states that a line from a planet to the sun sweeps out equal areas in equal times.

\section*{Teacher Support}
Teacher Support Explain that this formula is easy to remember because it is similar to \(A=\pi r^{2}\). Discuss Earth's eccentricity. Compare it with that of other planets, asteroids, or comets to further emphasize what defines a planet. Note that Earth has one of the least eccentric orbits and Mercury has the most eccentric orbit of the planets.\\[0pt]
[BL]Have the students memorized the value of \(\pi\) ?\\[0pt]
[OL][AL]What is the formula when \(a=b\) ? Is the formula familiar?\\[0pt]
[OL]Can the student verify this statement by rearranging the equation?

Kepler's Third Law Kepler's third law states that the ratio of the squares of the periods of any two planets ( \(T_{1}, T_{2}\) ) is equal to the ratio of the cubes of their average orbital distance from the sun \(\left(r_{1}, r_{2}\right)\). Mathematically, this is represented by\\
\(\frac{T_{1}^{2}}{T_{2}^{2}}=\frac{r_{1}^{3}}{r_{2}^{3}}\).\\
From this equation, it follows that the ratio \(r^{3} / T^{2}\) is the same for all planets in the solar system. Later we will see how the work of Newton leads to a value for this constant.

\section*{Worked Example}
Kepler's Third Law Given that the moon orbits Earth each 27.3 days and that it is an average distance of \(3.84 \times 10^{8} \mathrm{~m}\) from the center of Earth, calculate the period of an artificial satellite orbiting at an average altitude of \(1,500 \mathrm{~km}\) above Earth's surface.

\section*{Strategy}
The period, or time for one orbit, is related to the radius of the orbit by Kepler's third law, given in mathematical form by \(\frac{T_{1}^{2}}{T_{2}^{2}}=\frac{r_{1}^{3}}{r_{2}^{3}}\). Let us use the subscript 1 for the moon and the subscript 2 for the satellite. We are asked to find \(T_{2}\). The given information tells us that the orbital radius of the moon is \(r_{1}=3.84 \times 10^{8} \mathrm{~m}\), and that the period of the moon is \(T_{1}=27.3\) days . The height of the artificial satellite above Earth's surface is given, so to get the distance \(r_{2}\) from the center of Earth we must add the height to the radius of Earth ( 6380 km ). This gives \(r_{2}=1500 \mathrm{~km}+6380 \mathrm{~km}=7880 \mathrm{~km}\). Now all quantities are known, so \(T_{2}\) can be found.

Solution\\
To solve for \(T_{2}\), we cross-multiply and take the square root, yielding\\
\(T_{2}^{2}=T_{1}^{2}\left(\frac{r_{2}}{r_{1}}\right)^{3} ; T_{2}=T_{1}\left(\frac{r_{2}}{r_{1}}\right)^{\frac{3}{2}}\)\\
\(T_{2}=(27.3 \mathrm{~d})\left(\frac{24.0 \mathrm{~h}}{\mathrm{~d}}\right)\left(\frac{7880 \mathrm{~km}}{3.84 \times 10^{5} \mathrm{~km}}\right)^{\frac{3}{2}}=1.93 \mathrm{~h}\).

\section*{7.3}
Discussion\\
This is a reasonable period for a satellite in a fairly low orbit. It is interesting that any satellite at this altitude will complete one orbit in the same amount of time.

\section*{Teacher Support}
Teacher Support Remind the students that this only works when the satellites are small compared to the parent object and when both satellites orbit the same parent object.

\section*{Practice Problems}
1.

A planet with no axial tilt is located in another solar system. It circles its sun in a very elliptical orbit so that the temperature varies greatly throughout the year. If the year there has 612 days and the inhabitants celebrate the coldest day on day 1 of their calendar, when is the warmest day?\\
a. Day 1\\
b. Day 153\\
c. Day 306\\
d. Day 459\\
2.

A geosynchronous Earth satellite is one that has an orbital period of precisely 1 day. Such orbits are useful for communication and weather observation because the satellite remains above the same point on Earth (provided it orbits in the equatorial plane in the same direction as Earth's rotation). The ratio \(\backslash \operatorname{frac}\left\{\mathrm{r}^{\wedge} 3\right\}\left\{\mathrm{T}^{\wedge} 2\right\}\) for the moon is \(1.01 \backslash \operatorname{times} 10^{\wedge}\{18\} \backslash, \mid \operatorname{frac}\left\{\backslash \operatorname{text}\{\mathrm{km}\}^{\wedge} 3\right\}\left\{\mathrm{y}^{\wedge} 2\right\}\). Calculate the radius of the orbit of such a satellite.\\
a. \(2.75 \backslash\) times \(10^{\wedge} 3 \backslash, \backslash \operatorname{text}\{\mathrm{~km}\}\)\\
b. \(1.96 \backslash\) times \(10^{\wedge} 4 \backslash, \backslash \operatorname{text}\{\mathrm{~km}\}\)\\
c. \(1.40 \backslash\) times \(10^{\wedge} 5 \backslash, \backslash \operatorname{text}\{\mathrm{~km}\}\)\\
d. \(1.00 \backslash\) times \(10^{\wedge} 6 \backslash, \backslash \operatorname{text}\{\mathrm{~km}\}\)

\section*{Check Your Understanding}
3.

Are Kepler's laws purely descriptive, or do they contain causal information?\\
a. Kepler's laws are purely descriptive.\\
b. Kepler's laws are purely causal.\\
c. Kepler's laws are descriptive as well as causal.\\
d. Kepler's laws are neither descriptive nor causal.\\
4.

True or false-According to Kepler's laws of planetary motion, a satellite increases its speed as it approaches its parent body and decreases its speed as it moves away from the parent body.\\
a. True\\
b. False\\
5.

Identify the locations of the foci of an elliptical orbit.\\
a. One focus is the parent body, and the other is located at the opposite end of the ellipse, at the same distance from the center as the parent body.\\
b. One focus is the parent body, and the other is located at the opposite end of the ellipse, at half the distance from the center as the parent body.\\
c. One focus is the parent body and the other is located outside of the elliptical orbit, on the line on which is the semi-major axis of the ellipse.\\
d. One focus is on the line containing the semi-major axis of the ellipse, and the other is located anywhere on the elliptical orbit of the satellite.

\section*{Teacher Support}
Teacher Support Use the Check Your Answers questions to assess whether students master the learning objectives for this section. If students are struggling with a specific objective, the Check Your Answers will help identify which objective is causing the problem and direct students to the relevant content.

\subsection*{7.2 Newton's Law of Universal Gravitation and Einstein's Theory of General Relativity}
\section*{Section Learning Objectives}
By the end of this section, you will be able to do the following:

\begin{itemize}
  \item Explain Newton's law of universal gravitation and compare it to Einstein's theory of general relativity
  \item Perform calculations using Newton's law of universal gravitation
\end{itemize}

\section*{Teacher Support}
Teacher Support The learning objectives in this section will help your students master the following standards:

\begin{itemize}
  \item (4) Science concepts. The student knows and applies the laws governing motion in a variety of situations. The student is expected to:
  \item (D) calculate the effect of forces on objects, including the law of inertia, the relationship between force and acceleration, and the nature of force pairs between objects;
  \item (5) Science concepts. The student knows the nature of forces in the physical world. The student is expected to:
  \item (A) research and describe the historical development of the concepts of gravitational, electromagnetic, weak nuclear, and strong nuclear forces;
  \item (B) describe and calculate how the magnitude of the gravitational force between two objects depends on their masses and the distance between their centers.
\end{itemize}

\section*{Section Key Terms}
\section*{Teacher Support}
Teacher Support In this section, students will apply Newton's law of universal gravitation to objects close at hand and far off in the depths of the solar system.\\[0pt]
[BL][OL] Compare the contributions of Kepler, Newton, and Einstein. Place them historically with dates.\\[0pt]
[AL] Ask if anyone knows the difference between special relativity and general relativity. Special relativity is a theory of spacetime and applies to observers moving at constant velocity. General relativity is a theory of gravity and applies\\
to observers that are accelerating. General relativity is broader and includes special relativity, which was published first.

\section*{Concepts Related to Newton's Law of Universal Gravitation}
Sir Isaac Newton was the first scientist to precisely define the gravitational force, and to show that it could explain both falling bodies and astronomical motions. See Figure 7.7. But Newton was not the first to suspect that the same force caused both our weight and the motion of planets. His forerunner, Galileo Galilei, had contended that falling bodies and planetary motions had the same cause. Some of Newton's contemporaries, such as Robert Hooke, Christopher Wren, and Edmund Halley, had also made some progress toward understanding gravitation. But Newton was the first to propose an exact mathematical form and to use that form to show that the motion of heavenly bodies should be conic sections-circles, ellipses, parabolas, and hyperbolas. This theoretical prediction was a major triumph. It had been known for some time that moons, planets, and comets follow such paths, but no one had been able to propose an explanation of the mechanism that caused them to follow these paths and not others.\\
\includegraphics[max width=\textwidth, center]{c94e422a-deb3-4e1e-8f32-855dd556856c-20}

Figure 7.7 The popular legend that Newton suddenly discovered the law of universal gravitation when an apple fell from a tree and hit him on the head has an element of truth in it. A more probable account is that he was walking through an orchard and wondered why all the apples fell in the same direction with the same acceleration. Great importance is attached to it because Newton's universal law of gravitation and his laws of motion answered very old questions about nature and gave tremendous support to the notion of underlying simplicity and unity in nature. Scientists still expect underlying simplicity to emerge from their ongoing inquiries into nature.

\section*{Teacher Support}
Teacher Support [BL][OL]Ask students if it really is obvious why all things fall straight down. Ask them to back up their reasons. Ask if the name Halley rings a bell.\\[0pt]
[OL][AL]Ask if anyone thinks it is strange or even mysterious that a force can act at a distance across empty space. Ask the students to compare and contrast gravitational force with magnetic and electrostatic forces. Note how much force at a distance is like magic or having superpowers.

The gravitational force is relatively simple. It is always attractive, and it depends only on the masses involved and the distance between them. Expressed in modern language, Newton's universal law of gravitation states that every object in the universe attracts every other object with a force that is directed along a line joining them. The force is directly proportional to the product of their masses and inversely proportional to the square of the distance between them. This attraction is illustrated by Figure 7.8.

\begin{figure}[h]
\begin{center}
  \includegraphics[max width=\textwidth]{c94e422a-deb3-4e1e-8f32-855dd556856c-22}
\captionsetup{labelformat=empty}
\caption{Figure 7.8 Gravitational attraction is along a line joining the centers of mass (CM) of the two bodies. The magnitude of the force on each body is the same, consistent with Newton's third law (action-reaction).\\
For two bodies having masses \(m\) and \(M\) with a distance \(r\) between their centers of mass, the equation for Newton's universal law of gravitation is\\
\(\mathbf{F}=G \frac{m M}{r^{2}}\)\\
where \(\mathbf{F}\) is the magnitude of the gravitational force and \(G\) is a proportionality factor called the gravitational constant. \(G\) is a universal constant, meaning that}
\end{center}
\end{figure}

it is thought to be the same everywhere in the universe. It has been measured experimentally to be \(G=6.673 \times 10^{-11} \mathrm{~N} \cdot \mathrm{~m}^{2} / \mathrm{kg}^{2}\).

If a person has a mass of 60.0 kg , what would be the force of gravitational attraction on him at Earth's surface? \(G\) is given above, Earth's mass \(M\) is 5.97 \(\times 10^{24} \mathrm{~kg}\), and the radius \(r\) of Earth is \(6.38 \times 10^{6} \mathrm{~m}\). Putting these values into Newton's universal law of gravitation gives\\
\(\mathbf{F}=G \frac{m M}{r^{2}}=\left(6.673 \times 10^{-11} \frac{\mathrm{~N} \cdot \mathrm{~m}^{2}}{\mathrm{~kg}^{2}}\right)\left(\frac{(60.0 \mathrm{~kg})\left(5.97 \times 10^{24} \mathrm{~kg}\right)}{\left(6.38 \times 10^{6} \mathrm{~m}\right)^{2}}\right)=584 \mathrm{~N}\)\\
We can check this result with the relationship: \(\mathbf{F}=m \mathbf{g}=(60 \mathrm{~kg})\left(9.8 \mathrm{~m} / \mathrm{s}^{2}\right)=\) 588 N

You may remember that \(\mathbf{g}\), the acceleration due to gravity, is another important constant related to gravity. By substituting \(\mathbf{g}\) for \(\mathbf{a}\) in the equation for Newton's second law of motion we get \(\mathbf{F}=m \mathbf{g}\). Combining this with the equation for universal gravitation gives\\
\(m \mathbf{g}=G \frac{m M}{r^{2}}\)\\
Cancelling the mass \(m\) on both sides of the equation and filling in the values for the gravitational constant and mass and radius of the Earth, gives the value of \(g\), which may look familiar.\\
\(\mathbf{g}=G \frac{M}{r^{2}}=\left(6.67 \times 10^{-11} \frac{\mathrm{~N} \cdot \mathrm{~m}^{2}}{\mathrm{~kg}^{2}}\right)\left(\frac{5.98 \times 10^{24} \mathrm{~kg}}{\left(6.38 \times 10^{6} \mathrm{~m}\right)^{2}}\right)=9.80 \mathrm{~m} / \mathrm{s}^{2}\)\\
This is a good point to recall the difference between mass and weight. Mass is the amount of matter in an object; weight is the force of attraction between the mass within two objects. Weight can change because \(g\) is different on every moon and planet. An object's mass \(m\) does not change but its weight \(m \mathbf{g}\) can.

\section*{Teacher Support}
Teacher Support [BL][OL]Be sure no one is confusing \(G\) with \(\mathbf{g}\).\\[0pt]
[AL]Ask if anyone can explain why \(G\) is a universal constant that applies anywhere in the universe. Have them discuss the idea that the laws of physics are the same everywhere and that, at one time, people were not so sure about this. Emphasize that \(\mathbf{g}\) is not a universal constant.

\section*{Virtual Physics}
Gravity and Orbits Move the sun, Earth, moon and space station in this simulation to see how it affects their gravitational forces and orbital paths. Visualize the sizes and distances between different heavenly bodies. Turn off gravity to see what would happen without it!

Click to view content

Click to view content\\
Why doesn't the Moon travel in a smooth circle around the Sun?\\
a. The Moon is not affected by the gravitational field of the Sun.\\
b. The Moon is not affected by the gravitational field of the Earth.\\
c. The Moon is affected by the gravitational fields of both the Earth and the Sun, which are always additive.\\
d. The moon is affected by the gravitational fields of both the Earth and the Sun, which are sometimes additive and sometimes opposite.

\section*{Teacher Support}
Teacher Support This is a good animation of the Earth-Moon-Sun system. Have the students try all of the buttons. This will show the paths of the Earth and the moon separately and together. Explain the gravitational force and velocity vectors. Point out the interesting shape of the moon's path around the sun. Explain that the velocity vector of the moon changes because sometimes the moon is traveling in the direction of Earth's orbit and sometimes it is traveling in the opposite direction.

\section*{Snap Lab}
Take-Home Experiment: Falling Objects In this activity you will study the effects of mass and air resistance on the acceleration of falling objects. Make predictions (hypotheses) about the outcome of this experiment. Write them down to compare later with results.

\begin{itemize}
  \item Four sheets of \(8-1 / 2 \times 11\)-inch paper
\end{itemize}

Procedure

\begin{itemize}
  \item Take four identical pieces of paper.
  \item Crumple one up into a small ball.
  \item Leave one uncrumpled.
  \item Take the other two and crumple them up together, so that they make a ball of exactly twice the mass of the other crumpled ball.
  \item Now compare which ball of paper lands first when dropped simultaneously from the same height.
\end{itemize}

\begin{enumerate}
  \item Compare crumpled one-paper ball with crumpled two-paper ball.
  \item Compare crumpled one-paper ball with uncrumpled paper.
\end{enumerate}

Why do some objects fall faster than others near the surface of the earth if all mass is attracted equally by the force of gravity?\\
a. Some objects fall faster because of air resistance, which acts in the direction of the motion of the object and exerts more force on objects with less surface area.\\
b. Some objects fall faster because of air resistance, which acts in the direction opposite the motion of the object and exerts more force on objects with less surface area.\\
c. Some objects fall faster because of air resistance, which acts in the direction of motion of the object and exerts more force on objects with more surface area.\\
d. Some objects fall faster because of air resistance, which acts in the direction opposite the motion of the object and exerts more force on objects with more surface area.

\section*{Teacher Support}
Teacher Support Ask for predictions (hypotheses) about the outcome of this experiment. Have the students write them down to compare later with results.

It is possible to derive Kepler's third law from Newton's law of universal gravitation. Applying Newton's second law of motion to angular motion gives an expression for centripetal force, which can be equated to the expression for force in the universal gravitation equation. This expression can be manipulated to produce the equation for Kepler's third law. We saw earlier that the expression \(r^{3} / T^{2}\) is a constant for satellites orbiting the same massive object. The derivation of Kepler's third law from Newton's law of universal gravitation and Newton's second law of motion yields that constant:\\
\(\frac{r^{3}}{T^{2}}=\frac{G M}{4 \pi^{2}}\)\\
where \(M\) is the mass of the central body about which the satellites orbit (for example, the sun in our solar system). The usefulness of this equation will be seen later.

\section*{Teacher Support}
Teacher Support [OL]This equation illustrates the difference between Kepler's and Newton's work. Ask the students to explain why this is so.\\[0pt]
[AL]Ask the students what the attraction would be between two 10 kg balls separated by a distance of 1.0 m . Could they feel it? Later, ask them to calculate it after they have done some similar calculations. Solution:\\
\(\mathbf{F}=G \frac{m M}{r^{2}}=\left(6.67 \times 10^{-11} \frac{\mathrm{~N} \cdot \mathrm{~m}^{2}}{\mathrm{~kg}^{2}}\right)\left(\frac{10 \mathrm{~kg} \times 10 \mathrm{~kg}}{(1 \mathrm{~m})^{2}}\right)=6.67 \times 10^{-9} \mathrm{~N}\)\\
The universal gravitational constant \(G\) is determined experimentally. This definition was first done accurately in 1798 by English scientist Henry Cavendish (1731-1810), more than 100 years after Newton published his universal law of gravitation. The measurement of \(G\) is very basic and important because it determines the strength of one of the four forces in nature. Cavendish's experiment\\
was very difficult because he measured the tiny gravitational attraction between two ordinary-sized masses (tens of kilograms at most) by using an apparatus like that in Figure 7.9. Remarkably, his value for \(G\) differs by less than \(1 \%\) from the modern value.

\begin{figure}[h]
\begin{center}
  \includegraphics[max width=\textwidth]{c94e422a-deb3-4e1e-8f32-855dd556856c-26}
\captionsetup{labelformat=empty}
\caption{Figure 7.9 Cavendish used an apparatus like this to measure the gravitational attraction between two suspended spheres ( \(m\) ) and two spheres on a stand ( \(M\) ) by observing the amount of torsion (twisting) created in the fiber. The distance between the masses can be varied to check the dependence of the force on distance. Modern experiments of this type continue to explore gravity.}
\end{center}
\end{figure}

Einstein's Theory of General Relativity Einstein's theory of general relativity explained some interesting properties of gravity not covered by Newton's theory. Einstein based his theory on the postulate that acceleration and gravity have the same effect and cannot be distinguished from each other. He concluded that light must fall in both a gravitational field and in an accelerating reference frame. Figure 7.10 shows this effect (greatly exaggerated) in an accelerating elevator. In Figure 7.10 (a), the elevator accelerates upward in zero gravity. In Figure 7.10 (b), the room is not accelerating but is subject to gravity. The effect on light is the same: it "falls" downward in both situations. The person in the elevator cannot tell whether the elevator is accelerating in zero gravity or is stationary and subject to gravity. Thus, gravity affects the path of light, even though we think of gravity as acting between masses, while photons are massless.

\section*{Teacher Support}
Teacher Support [BL][OL]Ask the students to discuss the postulate. Can they relate the identity of gravity and acceleration to experience?

\begin{figure}[h]
\begin{center}
  \includegraphics[max width=\textwidth]{c94e422a-deb3-4e1e-8f32-855dd556856c-27}
\captionsetup{labelformat=empty}
\caption{Figure 7.10 (a) A beam of light emerges from a flashlight in an upwardaccelerating elevator. Since the elevator moves up during the time the light takes to reach the wall, the beam strikes lower than it would if the elevator were not accelerated. (b) Gravity must have the same effect on light, since it is not possible to tell whether the elevator is accelerating upward or is stationary and acted upon by gravity.}
\end{center}
\end{figure}

Einstein's theory of general relativity got its first verification in 1919 when starlight passing near the sun was observed during a solar eclipse. (See Figure 7.11.) During an eclipse, the sky is darkened and we can briefly see stars. Those on a line of sight nearest the sun should have a shift in their apparent positions. Not only was this shift observed, but it agreed with Einstein's predictions well within experimental uncertainties. This discovery created a scientific and public sensation. Einstein was now a folk hero as well as a very great scientist. The bending of light by matter is equivalent to a bending of space itself, with light following the curve. This is another radical change in our concept of space and time. It is also another connection that any particle with mass or energy (e.g., massless photons) is affected by gravity.

\begin{figure}[h]
\begin{center}
  \includegraphics[max width=\textwidth]{c94e422a-deb3-4e1e-8f32-855dd556856c-28}
\captionsetup{labelformat=empty}
\caption{Figure 7.11 This schematic shows how light passing near a massive body like the sun is curved toward it. The light that reaches the Earth then seems to be coming from different locations than the known positions of the originating stars. Not only was this effect observed, but the amount of bending was precisely what Einstein predicted in his general theory of relativity.\\
To summarize the two views of gravity, Newton envisioned gravity as a tug of war along the line connecting any two objects in the universe. In contrast, Einstein envisioned gravity as a bending of space-time by mass.}
\end{center}
\end{figure}

\section*{Boundless Physics}
NASA gravity probe B NASA's Gravity Probe B (GP-B) mission has confirmed two key predictions derived from Albert Einstein's general theory of relativity. The probe, shown in Figure 7.12 was launched in 2004. It carried four ultra-precise gyroscopes designed to measure two effects hypothesized by Einstein's theory:

\begin{itemize}
  \item The geodetic effect, which is the warping of space and time by the gravitational field of a massive body (in this case, Earth)
  \item The frame-dragging effect, which is the amount by which a spinning object pulls space and time with it as it rotates
\end{itemize}

\begin{figure}[h]
\begin{center}
  \includegraphics[max width=\textwidth]{c94e422a-deb3-4e1e-8f32-855dd556856c-29}
\captionsetup{labelformat=empty}
\caption{Figure 7.12 Artist concept of Gravity Probe B spacecraft in orbit around the Earth. (credit: NASA/MSFC)}
\end{center}
\end{figure}

Both effects were measured with unprecedented precision. This was done by pointing the gyroscopes at a single star while orbiting Earth in a polar orbit. As predicted by relativity theory, the gyroscopes experienced very small, but measureable, changes in the direction of their spin caused by the pull of Earth's gravity.

The principle investigator suggested imagining Earth spinning in honey. As Earth rotates it drags space and time with it as it would a surrounding sea of honey.

According to the general theory of relativity, a gravitational field bends light. What does this have to do with time and space?\\
a. Gravity has no effect on the space-time continuum, and gravity only affects\\
the motion of light.\\
b. The space-time continuum is distorted by gravity, and gravity has no effect on the motion of light.\\
c. Gravity has no effect on either the space-time continuum or on the motion of light.\\
d. The space-time continuum is distorted by gravity, and gravity affects the motion of light.

\section*{Teacher Support}
Teacher Support Explain that it is very exciting when a prediction of relativity theory is tested successfully. Some of the predictions were in doubt because they sounded so bizarre.

\section*{Calculations Based on Newton's Law of Universal Gravitation}
\section*{Tips For Success}
When performing calculations using the equations in this chapter, use units of kilograms for mass, meters for distances, newtons for force, and seconds for time.

The mass of an object is constant, but its weight varies with the strength of the gravitational field. This means the value of \(\mathbf{g}\) varies from place to place in the universe. The relationship between force, mass, and acceleration from the second law of motion can be written in terms of \(\mathbf{g}\).\\
\(\mathbf{F}=m \mathbf{a}=m \mathbf{g}\)\\
In this case, the force is the weight of the object, which is caused by the gravitational attraction of the planet or moon on which the object is located. We can use this expression to compare weights of an object on different moons and planets.

\section*{Teacher Support}
Teacher Support [BL] Check to make sure students are clear about the distinction between mass and weight.\\[0pt]
[OL] Recall the antics of astronauts of on the moon performed to illustrate the effect of a different value for \(\mathbf{g}\).

\section*{Watch Physics}
Mass and Weight Clarification This video shows the mathematical basis of the relationship between mass and weight. The distinction between mass and\\
weight are clearly explained. The mathematical relationship between mass and weight are shown mathematically in terms of the equation for Newton's law of universal gravitation and in terms of his second law of motion.

Click to view content

\section*{Grasp Check}
Would you have the same mass on the moon as you do on Earth? Would you have the same weight?\\
a. You would weigh more on the moon than on Earth because gravity on the moon is stronger than gravity on Earth.\\
b. You would weigh less on the moon than on Earth because gravity on the moon is weaker than gravity on Earth.\\
c. You would weigh less on the moon than on Earth because gravity on the moon is stronger than gravity on Earth.\\
d. You would weigh more on the moon than on Earth because gravity on the moon is weaker than gravity on Earth.

\section*{Teacher Support}
Teacher Support This may be a rather long-winded explanation of the massweight distinction, but it should drive home the point.

Two equations involving the gravitational constant, \(G\), are often useful. The first is Newton's equation, \(\mathbf{F}=G \frac{m M}{r^{2}}\). Several of the values in this equation are either constants or easily obtainable. \(\mathbf{F}\) is often the weight of an object on the surface of a large object with mass \(M\), which is usually known. The mass of the smaller object, \(m\), is often known, and \(G\) is a universal constant with the same value anywhere in the universe. This equation can be used to solve problems involving an object on or orbiting Earth or other massive celestial object. Sometimes it is helpful to equate the right-hand side of the equation to \(m \mathbf{g}\) and cancel the \(m\) on both sides.\\
The equation \(\frac{r^{3}}{T^{2}}=\frac{G M}{4 \pi^{2}}\) is also useful for problems involving objects in orbit. Note that there is no need to know the mass of the object. Often, we know the radius \(r\) or the period \(T\) and want to find the other. If these are both known, we can use the equation to calculate the mass of a planet or star.

\section*{Watch Physics}
Mass and Weight Clarification This video demonstrates calculations involving Newton's universal law of gravitation.

Click to view content

Watch Physics: Introduction to Newton's Law of Gravitation This video introduces Newton's law of gravitation.

Click to view content\\
What is the difference between \(g\) and \(G\) ?\\
a. \(g\) and \(G\) are both unchanging constants but have different units.\\
b. \(G\) is a universal constant that relates force with a pair of masses at a distance, while \(g\) relates force with mass and varies with location.\\
c. \(g\) describes acceleration while \(G\) describes gravitational force.\\
d. \(g\) describes gravitational force while \(G\) describes acceleration.

\section*{Teacher Support}
Teacher Support This video is a thorough demonstration of many of the calculations to be learned in this subsection.

\section*{Worked Example}
Change in \(\mathbf{g}\) The value of \(\mathbf{g}\) on the planet Mars is \(3.71 \mathrm{~m} / \mathrm{s}^{2}\). If you have a mass of 60.0 kg on Earth, what would be your mass on Mars? What would be your weight on Mars?

\section*{Strategy}
Weight equals acceleration due to gravity times mass: \(\mathbf{W}=m \mathbf{g}\). An object's mass is constant. Call acceleration due to gravity on Mars \(\mathbf{g}_{M}\) and weight on Mars \(\mathbf{W}_{M}\).

Solution\\
Mass on Mars would be the same, 60 kg .\\
\(\mathbf{W}_{M}=m \mathbf{g}_{M}=(60.0 \mathrm{~kg})\left(3.71 \mathrm{~m} / \mathrm{s}^{2}\right)=223 \mathrm{~N}\)\\
7.4

Discussion\\
The value of \(\mathbf{g}\) on any planet depends on the mass of the planet and the distance from its center. If the material below the surface varies from point to point, the value of \(\mathbf{g}\) will also vary slightly.

\section*{Teacher Support}
Teacher Support This is a typical mass-weight calculation.

\section*{Worked Example}
Earth's g at the Moon Find the acceleration due to Earth's gravity at the distance of the moon.\\
\(G=6.67 \times 10^{-11} \mathrm{~N} \cdot \mathrm{~m}^{2} / \mathrm{kg}^{2}\)\\
Earth-moon distance \(=3.84 \times 10^{8} \mathrm{~m}\)\\
Earth's mass \(=5.98 \times 10^{24} \mathrm{~kg}\)\\
7.5

Express the force of gravity in terms of \(g\).\\
\(\mathrm{F}=\mathrm{W}=m \mathrm{a}=m \mathrm{~g}\)\\
7.6

Combine with the equation for universal gravitation.\\
\(m \mathbf{g}=m G \frac{M}{r^{2}}\)\\
7.7

Solution\\
Cancel \(m\) and substitute.\\
\(\mathbf{g}=G \frac{M}{r^{2}}=\left(6.67 \times 10^{-11} \frac{\mathrm{~N} \cdot \mathrm{~m}^{2}}{\mathrm{~kg}^{2}}\right)\left(\frac{5.98 \times 10^{24} \mathrm{~kg}}{\left(3.84 \times 10^{8} \mathrm{~m}\right)^{2}}\right)=2.70 \times 10^{-3} \mathrm{~m} / \mathrm{s}^{2}\)\\
7.8

Discussion\\
The value of \(\mathbf{g}\) for the moon is \(1.62 \mathrm{~m} / \mathrm{s}^{2}\). Comparing this value to the answer, we see that Earth's gravitational influence on an object on the moon's surface would be insignificant.

\section*{Teacher Support}
Teacher Support [BL][OL] Review the meanings of all the symbols in these equations: \(\mathbf{F}, G, m, M, r, T\), and \(\pi\).\\[0pt]
[OL][AL] Have the students memorize the values of \(G, \mathbf{g}\), and to three significant figures.

\section*{Practice Problems}
6.

What is the mass of a person who weighs \(600 \backslash, \backslash \operatorname{text}\{\mathrm{~N}\}\) ?\\
a. \(6.00 \backslash, \backslash \operatorname{text}\{\mathrm{~kg}\}\)\\
b. \(61.2 \backslash, \backslash \operatorname{text}\{\mathrm{~kg}\}\)\\
c. \(600 \backslash, \backslash \operatorname{text}\{\mathrm{~kg}\}\)\\
d. \(610 \backslash, \backslash \operatorname{text}\{\mathrm{~kg}\}\)\\
7.

Calculate Earth's mass given that the acceleration due to gravity at the North Pole is \(9.830 \backslash \operatorname{text}\{\mathrm{~m} / \mathrm{s}\}^{\wedge} 2 \backslash!\) and the radius of the Earth is \(6371 \backslash, \backslash \operatorname{text}\{\mathrm{~km}\}\) from pole to center.\\
a. \(5.94 \backslash\) times \(10^{\wedge}\{17\} \backslash, \backslash \operatorname{text}\{\mathrm{kg}\}\)\\
b. \(5.94 \backslash\) times \(10^{\uparrow}\{24\} \backslash \operatorname{text}\{\mathrm{kg}\}\)\\
c. \(9.36 \backslash\) times \(10^{\wedge}\{17\} \backslash \operatorname{text}\{\mathrm{kg}\}\)\\
d. \(9.36 \backslash\) times \(10^{\wedge}\{24\} \backslash\) text \(\{\mathrm{kg}\}\)

\section*{Check Your Understanding}
8.

Some of Newton's predecessors and contemporaries also studied gravity and proposed theories. What important advance did Newton make in the study of gravity that the other scientists had failed to do?\\
a. He gave an exact mathematical form for the theory.\\
b. He added a correction term to a previously existing formula.\\
c. Newton found the value of the universal gravitational constant.\\
d. Newton showed that gravitational force is always attractive.\\
9.

State the law of universal gravitation in words only.\\
a. Gravitational force between two objects is directly proportional to the sum of the squares of their masses and inversely proportional to the square of the distance between them.\\
b. Gravitational force between two objects is directly proportional to the product of their masses and inversely proportional to the square of the distance between them.\\
c. Gravitational force between two objects is directly proportional to the sum of the squares of their masses and inversely proportional to the distance between them.\\
d. Gravitational force between two objects is directly proportional to the product of their masses and inversely proportional to the distance between them.\\
10.

Newton's law of universal gravitation explains the paths of what?\\
a. A charged particle\\
b. A ball rolling on a plane surface\\
c. A planet moving around the sun\\
d. A stone tied to a string and whirled at constant speed in a horizontal circle

\section*{Teacher Support}
Teacher Support Use the Check Your Answers questions to assess whether students master the learning objectives for this section. If students are struggling with a specific objective, the Check Your Answers will help identify which objective is causing the problem and direct students to the relevant content.

\section*{Key Terms}
aphelion furthest distance between a planet and the sun (called apoapsis for other celestial bodies)

Copernican model the model of the solar system where the sun is at the center of the solar system and all the planets orbit around it; this is also called the heliocentric model\\
eccentricity a measure of the separation of the foci of an ellipse\\
Einstein's theory of general relativity the theory that gravitational force results from the bending of spacetime by an object's mass\\
gravitational constant the proportionality constant in Newton's law of universal gravitation

Kepler's laws of planetary motion three laws derived by Johannes Kepler that describe the properties of all orbiting satellites

Newton's universal law of gravitation states that gravitational force between two objects is directly proportional to the product of their masses and inversely proportional to the square of the distance between them.\\
perihelion closest distance between a planet and the sun (called periapsis for other celestial bodies)

Ptolemaic model the model of the solar system where Earth is at the center of the solar system and the sun and all the planets orbit around it; this is also called the geocentric model

\section*{Key Equations}
7.1 Kepler's Laws of Planetary Motion\\
7.2 Newton's Law of Universal Gravitation and Einstein's Theory of General Relativity

\section*{Section Summary}
\subsection*{7.1 Kepler's Laws of Planetary Motion}
\begin{itemize}
  \item All satellites follow elliptical orbits.
  \item The line from the satellite to the parent body sweeps out equal areas in equal time.
  \item The radius cubed divided by the period squared is a constant for all satellites orbiting the same parent body.
\end{itemize}

\subsection*{7.2 Newton's Law of Universal Gravitation and Einstein's Theory of General Relativity}
\begin{itemize}
  \item Newton's law of universal gravitation provides a mathematical basis for gravitational force and Kepler's laws of planetary motion.
  \item Einstein's theory of general relativity shows that gravitational fields change the path of light and warp space and time.
  \item An object's mass is constant, but its weight changes when acceleration due to gravity, \(\mathbf{g}\), changes.
\end{itemize}

\end{document}