\documentclass[10pt]{article}
\usepackage[utf8]{inputenc}
\usepackage[T1]{fontenc}
\usepackage{graphicx}
\usepackage[export]{adjustbox}
\graphicspath{ {./images/} }
\usepackage{caption}
\usepackage{amsmath}
\usepackage{amsfonts}
\usepackage{amssymb}
\usepackage[version=4]{mhchem}
\usepackage{stmaryrd}
\usepackage{hyperref}
\hypersetup{colorlinks=true, linkcolor=blue, filecolor=magenta, urlcolor=cyan,}
\urlstyle{same}

\begin{document}
\captionsetup{singlelinecheck=false}
\begin{figure}[h]
\begin{center}
  \includegraphics[max width=\textwidth]{8b6c1c7c-84c8-44b6-9353-a83bfff7835e-01}
\captionsetup{labelformat=empty}
\caption{Figure 9.1 People on a roller coaster experience thrills caused by changes in types of energy. (Jonrev, Wikimedia Commons)}
\end{center}
\end{figure}

\section*{Chapter Outline}
9.1 Work, Power, and the Work-Energy Theorem\\
9.2 Mechanical Energy and Conservation of Energy\\
9.3 Simple Machines

\section*{Introduction}
\section*{Teacher Support}
Teacher Support Physics learning objectives come from 112.39 (c) Knowledge and Skills

\section*{Teacher Support}
Teacher Support Before students begin this chapter, it is useful to review the following concepts:

\begin{itemize}
  \item Using significant figures in calculations-Demonstrate how to use the proper number of significant figures when adding and multiplying.
  \item Converting units-Demonstrate how to convert from \(\mathrm{km} / \mathrm{h}\) to \(\mathrm{m} / \mathrm{s}\).
  \item Calculating average - Demonstrate how to average two numbers by dividing their sum by two.
  \item Commonly used terms-Explain that constant means unchanging, so constant speed refers to speed that is not changing. Explain that initial means starting, so initial time is the time at which the action of a problem begins. Explain that an object that is not moving is often described in physics as being at rest.
  \item Review the difference between mass and weight.
  \item Review the force of gravity and acceleration due to gravity.
\end{itemize}

Initiate a discussion about how speed changes at different points in a roller coaster ride. Also discuss acceleration and deceleration. Ask students to try to describe the physical experience of these changes.

Roller coasters have provided thrills for daring riders around the world since the nineteenth century. Inventors of roller coasters used simple physics to build the earliest examples using railroad tracks on mountainsides and old mines. Modern roller coaster designers use the same basic laws of physics to create the latest amusement park favorites. Physics principles are used to engineer the machines that do the work to lift a roller coaster car up its first big incline before it is set loose to roll. Engineers also have to understand the changes in the car's energy that keep it speeding over hills, through twists, turns, and even loops.

What exactly is energy? How can changes in force, energy, and simple machines move objects like roller coaster cars? How can machines help us do work? In this chapter, you will discover the answer to this question and many more, as you learn about work, energy, and simple machines.

\subsection*{9.1 Work, Power, and the Work-Energy Theorem}
\section*{Section Learning Objectives}
By the end of this section, you will be able to do the following:

\begin{itemize}
  \item Describe and apply the work-energy theorem
  \item Describe and calculate work and power
\end{itemize}

\section*{Teacher Support}
Teacher Support The learning objectives in this section will help your students master the following standards:

\begin{itemize}
  \item (6) Science concepts. The student knows that changes occur within a physical system and applies the laws of conservation of energy and momentum. The student is expected to:
  \item (A) describe and apply the work-energy theorem;
  \item (C) describe and calculate work and power.
\end{itemize}

In addition, the High School Physics Laboratory Manual addresses the following standards:

\begin{itemize}
  \item (6) Science concepts. The student knows that changes occur within a physical system and applies the laws of conservation of energy and momentum. The student is expected to:
  \item (C) calculate the mechanical energy of, power generated within, impulse applied to, and momentum of a physical system.
\end{itemize}

Use the lab titled Work and Energy as a supplement to address content in this section.

\section*{Section Key Terms}
\section*{Teacher Support}
Teacher Support In this section, students learn how work determines changes in kinetic energy and that power is the rate at which work is done.\\[0pt]
[BL][OL] Review understanding of mass, velocity, and acceleration due to gravity. Define the general definitions of the words potential and kinetic.\\[0pt]
[AL][AL] Remind students of the equation \(W=P E_{e}=\mathbf{f} m g\). Point out that acceleration due to gravity is a constant, therefore \(P E_{e}\) that results from work done by gravity will also be constant. Compare this to acceleration due to other forces, such as applying muscles to lift a rock, which may not be constant.

\section*{The Work-Energy Theorem}
In physics, the term work has a very specific definition. Work is application of force, \(\mathbf{f}\), to move an object over a distance, \(d\), in the direction that the force is applied. Work, \(W\), is described by the equation\\
\(W=\mathbf{f} d\).\\
Some things that we typically consider to be work are not work in the scientific sense of the term. Let's consider a few examples. Think about why each of the following statements is true.

\begin{itemize}
  \item Homework is not work.
  \item Lifting a rock upwards off the ground is work.
  \item Carrying a rock in a straight path across the lawn at a constant speed is not work.
\end{itemize}

The first two examples are fairly simple. Homework is not work because objects are not being moved over a distance. Lifting a rock up off the ground is work because the rock is moving in the direction that force is applied. The last example is less obvious. Recall from the laws of motion that force is not required to move an object at constant velocity. Therefore, while some force may be applied to keep the rock up off the ground, no net force is applied to keep the rock moving forward at constant velocity.

\section*{Teacher Support}
Teacher Support \([\mathrm{BL}][\mathrm{OL}]\) Explain that, when this theorem is applied to an object that is initially at rest and then accelerates, the \(\frac{1}{2} m \mathbf{v}_{1}^{2}\) term equals zero.\\[0pt]
[OL][AL] Work is measured in joules and \(W=\mathbf{f} d\). Force is measured in newtons and distance in meters, so joules are equivalent to newton-meters ( \(\mathrm{N} \cdot \mathrm{m}\) )

Work and energy are closely related. When you do work to move an object, you change the object's energy. You (or an object) also expend energy to do work. In fact, energy can be defined as the ability to do work. Energy can take a variety of different forms, and one form of energy can transform to another. In this chapter we will be concerned with mechanical energy, which comes in two forms: kinetic energy and potential energy.

\begin{itemize}
  \item Kinetic energy is also called energy of motion. A moving object has kinetic energy.
  \item Potential energy, sometimes called stored energy, comes in several forms. Gravitational potential energy is the stored energy an object has as a result of its position above Earth's surface (or another object in space). A roller coaster car at the top of a hill has gravitational potential energy.
\end{itemize}

Let's examine how doing work on an object changes the object's energy. If we apply force to lift a rock off the ground, we increase the rock's potential energy,\\
\(P E\). If we drop the rock, the force of gravity increases the rock's kinetic energy as the rock moves downward until it hits the ground.

The force we exert to lift the rock is equal to its weight, \(w\), which is equal to its mass, \(m\), multiplied by acceleration due to gravity, \(\mathbf{g}\).\\
\(\mathbf{f}=w=m \mathbf{g}\)\\
The work we do on the rock equals the force we exert multiplied by the distance, \(d\), that we lift the rock. The work we do on the rock also equals the rock's gain in gravitational potential energy, \(P E_{e}\).\\
\(W=P E_{e}=m \mathbf{g d}\)\\
Kinetic energy depends on the mass of an object and its velocity, \(\mathbf{v}\).\\
\(K E=\frac{1}{2} m \mathbf{v}^{2}\)\\
When we drop the rock the force of gravity causes the rock to fall, giving the rock kinetic energy. When work done on an object increases only its kinetic energy, then the net work equals the change in the value of the quantity \(\frac{1}{2} m \mathbf{v}^{2}\). This is a statement of the work-energy theorem, which is expressed mathematically as\\
\(W=\Delta K E=\frac{1}{2} m \mathbf{v}_{2}^{2}-\frac{1}{2} m \mathbf{v}_{1}^{2}\).\\
The subscripts \({ }_{2}\) and \({ }_{1}\) indicate the final and initial velocity, respectively. This theorem was proposed and successfully tested by James Joule, shown in Figure 9.2.

Does the name Joule sound familiar? The joule (J) is the metric unit of measurement for both work and energy. The measurement of work and energy with the same unit reinforces the idea that work and energy are related and can be converted into one another. \(1.0 \mathrm{~J}=1.0 \mathrm{~N} \mathrm{~m}\), the units of force multiplied by distance. \(1.0 \mathrm{~N}=1.0 \mathrm{~kg} \mathrm{~m} / \mathrm{s}^{2}\), so \(1.0 \mathrm{~J}=1.0 \mathrm{~kg} \mathrm{~m}^{2} / \mathrm{s}^{2}\). Analyzing the units of the term \((1 / 2) m \mathbf{v}^{2}\) will produce the same units for joules.

\begin{figure}[h]
\begin{center}
  \includegraphics[max width=\textwidth]{8b6c1c7c-84c8-44b6-9353-a83bfff7835e-06}
\captionsetup{labelformat=empty}
\caption{Figure 9.2 The joule is named after physicist James Joule (1818-1889). (C. H. Jeens, Wikimedia Commons)}
\end{center}
\end{figure}

\section*{Watch Physics}
Work and Energy This video explains the work energy theorem and discusses how work done on an object increases the object's KE.

Click to view content

\section*{Grasp Check}
True or false-The energy increase of an object acted on only by a gravitational force is equal to the product of the object's weight and the distance the object falls.\\
a. True\\
b. False

\section*{Teacher Support}
Teacher Support Repeat the information on kinetic and potential energy discussed earlier in the section. Have the students distinguish between and understand the two ways of increasing the energy of an object (1) applying a horizontal force to increase KE and (2) applying a vertical force to increase PE.

\section*{Calculations Involving Work and Power}
In applications that involve work, we are often interested in how fast the work is done. For example, in roller coaster design, the amount of time it takes to lift a roller coaster car to the top of the first hill is an important consideration. Taking a half hour on the ascent will surely irritate riders and decrease ticket sales. Let's take a look at how to calculate the time it takes to do work.

Recall that a rate can be used to describe a quantity, such as work, over a period of time. Power is the rate at which work is done. In this case, rate means per unit of time. Power is calculated by dividing the work done by the time it took to do the work.\\
\(P=\frac{W}{t}\)\\
Let's consider an example that can help illustrate the differences among work, force, and power. Suppose the woman in Figure 9.3 lifting the TV with a pulley gets the TV to the fourth floor in two minutes, and the man carrying the TV up the stairs takes five minutes to arrive at the same place. They have done the same amount of work ( \(\mathbf{f} d\) ) on the TV, because they have moved the same mass over the same vertical distance, which requires the same amount of upward force. However, the woman using the pulley has generated more power. This is because she did the work in a shorter amount of time, so the denominator of the power formula, \(t\), is smaller. (For simplicity's sake, we will leave aside for now the fact that the man climbing the stairs has also done work on himself.)

\begin{figure}[h]
\begin{center}
  \includegraphics[max width=\textwidth]{8b6c1c7c-84c8-44b6-9353-a83bfff7835e-08}
\captionsetup{labelformat=empty}
\caption{Figure 9.3 No matter how you move a TV to the fourth floor, the amount of work performed and the potential energy gain are the same.}
\end{center}
\end{figure}

Power can be expressed in units of watts \((\mathrm{W})\). This unit can be used to measure power related to any form of energy or work. You have most likely heard the term used in relation to electrical devices, especially light bulbs. Multiplying power by time gives the amount of energy. Electricity is sold in kilowatt-hours because that equals the amount of electrical energy consumed.

The watt unit was named after James Watt (1736-1819) (see Figure 9.4). He was a Scottish engineer and inventor who discovered how to coax more power out of steam engines.

E

Figure 9.4 Is James Watt thinking about watts? (Carl Frederik von Breda, Wikimedia Commons)

\section*{Teacher Support}
Teacher Support [BL][OL] Review the concept that work changes the energy of an object or system. Review the units of work, energy, force, and distance. Use the equations for mechanical energy and work to show what is work and what is not. Make it clear why holding something off the ground or carrying something over a level surface is not work in the scientific sense.\\[0pt]
[OL] Ask the students to use the mechanical energy equations to explain why each of these is or is not work. Ask them to provide more examples until they understand the difference between the scientific term work and a task that is simply difficult but not literally work (in the scientific sense).\\[0pt]
[BL][OL] Stress that power is a rate and that rate means "per unit of time." In the metric system this unit is usually seconds. End the section by clearing up any misconceptions about the distinctions between force, work, and power.\\[0pt]
[AL] Explain relationships between the units for force, work, and power. If \(W=\mathbf{f} d\) and work can be expressed in J , then \(P=\frac{W}{t}=\frac{\mathbf{f} d}{t}\) so power can be expressed in units of \(\frac{\mathrm{N} \cdot \mathrm{m}}{\mathrm{s}}\)\\
Also explain that we buy electricity in kilowatt-hours because, when power is multiplied by time, the time units cancel, which leaves work or energy.

\section*{Links To Physics}
Watt's Steam Engine James Watt did not invent the steam engine, but by the time he was finished tinkering with it, it was more useful. The first steam engines were not only inefficient, they only produced a back and forth, or reciprocal, motion. This was natural because pistons move in and out as the pressure in the chamber changes. This limitation was okay for simple tasks like pumping water or mashing potatoes, but did not work so well for moving a train. Watt was able build a steam engine that converted reciprocal motion to circular motion. With that one innovation, the industrial revolution was off and running. The world would never be the same. One of Watt's steam engines is shown in Figure 9.5. The video that follows the figure explains the importance of the steam engine in the industrial revolution.\\
\includegraphics[max width=\textwidth, center]{8b6c1c7c-84c8-44b6-9353-a83bfff7835e-11}

Figure 9.5 A late version of the Watt steam engine. (Nehemiah Hawkins, Wikimedia Commons)

\section*{Teacher Support}
Teacher Support Initiate a discussion on the historical significance of suddenly increasing the amount of power available to industries and transportation. Have students consider the fact that the speed of transportation increased roughly tenfold. Changes in how goods were manufactured were just as great. Ask students how they think the resulting changes in lifestyle compare to more recent changes brought about by innovations such as air travel and the Internet.

\section*{Watch Physics}
Watt's Role in the Industrial Revolution This video demonstrates how the watts that resulted from Watt's inventions helped make the industrial revolution possible and allowed England to enter a new historical era.

Click to view content

\section*{Grasp Check}
Which form of mechanical energy does the steam engine generate?\\
a. Potential energy\\
b. Kinetic energy\\
c. Nuclear energy\\
d. Solar energy

Before proceeding, be sure you understand the distinctions among force, work, energy, and power. Force exerted on an object over a distance does work. Work can increase energy, and energy can do work. Power is the rate at which work is done.

\section*{Worked Example}
Applying the Work-Energy Theorem An ice skater with a mass of 50 kg is gliding across the ice at a speed of \(8 \mathrm{~m} / \mathrm{s}\) when her friend comes up from behind and gives her a push, causing her speed to increase to \(12 \mathrm{~m} / \mathrm{s}\). How much work did the friend do on the skater?

\section*{Strategy}
The work-energy theorem can be applied to the problem. Write the equation for the theorem and simplify it if possible.\\
\(W=\Delta \mathrm{KE}=\frac{1}{2} m \mathbf{v}_{2}^{2}-\frac{1}{2} m \mathbf{v}_{1}^{2}\)

Simplify to \(W=\frac{1}{2} m\left(\mathbf{v}_{2}^{2}-\mathbf{v}_{1}^{2}\right)\)\\
Solution\\
Identify the variables. \(m=50 \mathrm{~kg}\),\\
\(\mathbf{v}_{2}=12 \frac{\mathrm{~m}}{\mathrm{~s}}\), and \(\mathbf{v}_{1}=8 \frac{\mathrm{~m}}{\mathrm{~s}}\)\\
9.1

Substitute.\\
\(W=\frac{1}{2} 50\left(12^{2}-8^{2}\right)=2,000 \quad \mathrm{~J}\)\\
9.2

Discussion\\
Work done on an object or system increases its energy. In this case, the increase is to the skater's kinetic energy. It follows that the increase in energy must be the difference in KE before and after the push.

\section*{Tips For Success}
This problem illustrates a general technique for approaching problems that require you to apply formulas: Identify the unknown and the known variables, express the unknown variables in terms of the known variables, and then enter all the known values.

\section*{Teacher Support}
Teacher Support Identify the three variables and choose the relevant equation. Distinguish between initial and final velocity and pay attention to the minus sign.

Identify the variables. \(m=50 \mathrm{~kg}\),\\
\(\mathbf{v}_{2}=12 \frac{\mathrm{~m}}{\mathrm{~s}}\), and \(\mathbf{v}_{1}=8 \frac{\mathrm{~m}}{\mathrm{~s}}\)\\
Substitute.\\
\(W=\frac{1}{2} 50\left(12^{2}-8^{2}\right)=2,000 \quad \mathrm{~J}\)

\section*{Practice Problems}
1.

\begin{figure}[h]
\begin{center}
  \includegraphics[max width=\textwidth]{8b6c1c7c-84c8-44b6-9353-a83bfff7835e-14}
\captionsetup{labelformat=empty}
\caption{Figure 9.6}
\end{center}
\end{figure}

A weightlifter lifts a 200 N barbell from the floor to a height of 2 m . How much work is done?\\
a. \(0 \backslash, \backslash \operatorname{text}\{\mathrm{~J}\}\)\\
b. \(100 \backslash, \backslash \operatorname{text}\{\mathrm{~J}\}\)\\
c. \(200 \backslash, \backslash \operatorname{text}\{\mathrm{~J}\}\)\\
d. \(400 \backslash, \backslash \operatorname{text}\{\mathrm{~J}\}\)\\
2.

Identify which of the following actions generates more power. Show your work.

\begin{itemize}
  \item carrying a \(100 \backslash, \backslash \operatorname{text}\{\mathrm{~N}\} \mathrm{TV}\) to the second floor in \(50 \backslash, \backslash \operatorname{text}\{\mathrm{~s}\}\) or
  \item carrying a \(24 \backslash, \backslash \operatorname{text}\{\mathrm{~N}\}\) watermelon to the second floor in \(10 \backslash, \backslash \operatorname{text}\{\mathrm{~s}\}\) ?\\
a. Carrying a \(100 \backslash, \backslash \operatorname{text}\{\mathrm{~N}\}\) TV generates more power than carrying a \(24 \backslash, \backslash \operatorname{text}\{\mathrm{~N}\}\) watermelon to the same height because power is defined as work done times the time interval.\\
b. Carrying a \(100 \backslash, \backslash \operatorname{text}\{\mathrm{~N}\}\) TV generates more power than carrying a \(24 \backslash, \backslash \operatorname{text}\{\mathrm{~N}\}\) watermelon to the same height because power is defined as the ratio of work done to the time interval.\\
c. Carrying a \(24 \backslash, \backslash \operatorname{text}\{\mathrm{~N}\}\) watermelon generates more power than carrying a \(100 \backslash, \backslash \operatorname{text}\{\mathrm{~N}\}\) TV to the same height because power is defined as work done times the time interval.\\
d. Carrying a \(24 \backslash, \backslash \operatorname{text}\{\mathrm{~N}\}\) watermelon generates more power than carrying a \(100 \backslash, \mid \operatorname{text}\{\mathrm{N}\}\) TV to the same height because power is defined as the ratio of work done and the time interval.
\end{itemize}

\section*{Check Your Understanding}
3.

Identify two properties that are expressed in units of joules.\\
a. work and force\\
b. energy and weight\\
c. work and energy\\
d. weight and force\\
4.

When a coconut falls from a tree, work \(W\) is done on it as it falls to the beach. This work is described by the equation\\
\(W=F d=\frac{1}{2} m v_{2}^{2}-\frac{1}{2} m v_{1}^{2}\).\\
9.3

Identify the quantities \(F, d, m, v_{1}\), and \(v_{2}\) in this event.\\
a. \(F\) is the force of gravity, which is equal to the weight of the coconut, \(d\) is the distance the nut falls, \(m\) is the mass of the earth, \(v_{1}\) is the initial velocity, and \(v_{2}\) is the velocity with which it hits the beach.\\
b. \(F\) is the force of gravity, which is equal to the weight of the coconut, \(d\) is the distance the nut falls, \(m\) is the mass of the coconut, \(v_{1}\) is the initial velocity, and \(v_{2}\) is the velocity with which it hits the beach.\\
c. \(F\) is the force of gravity, which is equal to the weight of the coconut, \(d\) is the distance the nut falls, \(m\) is the mass of the earth, \(v_{1}\) is the velocity with which it hits the beach, and \(v_{2}\) is the initial velocity.\\
d. \(F\) is the force of gravity, which is equal to the weight of the coconut, \(d\) is the distance the nut falls, \(m\) is the mass of the coconut, \(v_{1}\) is the velocity with which it hits the beach, and \(v_{2}\) is the initial velocity.

\section*{Teacher Support}
Teacher Support Use Check Your Understanding questions to assess students' achievement of the section's learning objectives. If students are struggling with a specific objective, the Check Your Understanding will help identify which one and direct students to the relevant content.

\subsection*{9.2 Mechanical Energy and Conservation of Energy}
\section*{Section Learning Objectives}
By the end of this section, you will be able to do the following:

\begin{itemize}
  \item Explain the law of conservation of energy in terms of kinetic and potential energy
  \item Perform calculations related to kinetic and potential energy. Apply the law of conservation of energy
\end{itemize}

\section*{Teacher Support}
Teacher Support The learning objectives in this section will help your students master the following standards:

\begin{itemize}
  \item (6) Science concepts. The student knows that changes occur within a physical system and applies the laws of conservation of energy and momentum. The student is expected to:
  \item (B) investigate examples of kinetic and potential energy and their transformations;
  \item (D) demonstrate and apply the laws of conservation of energy and conservation of momentum in one dimension.
\end{itemize}

In addition, the High School Physics Laboratory Manual addresses content in this section in the lab titled: Work and Energy, as well as the following standards:

\begin{itemize}
  \item (6) Science concepts. The student knows that changes occur within a physical system and applies the laws of conservation of energy and momentum.\\
The student is expected to:
  \item (B) investigate examples of kinetic and potential energy and their transformations;
  \item (D) demonstrate and apply the laws of conservation of energy and conservation of momentum in one dimension.
\end{itemize}

\section*{Section Key Terms}
\section*{Teacher Support}
Teacher Support [BL][OL] Begin by distinguishing mechanical energy from other forms of energy. Explain how the general definition of energy as the ability to do work makes perfect sense in terms of either form of mechanical energy. Discuss the law of conservation of energy and dispel any misconceptions related\\
to this law, such is the idea that moving objects just slow down naturally. Identify heat generated by friction as the usual explanation for apparent violations of the law.\\[0pt]
[AL] Start a discussion about how other useful forms of energy also end up as wasted heat, such as light, sound, and electricity. Try to get students to understand heat and temperature at a molecular level. Explain that energy lost to friction is really transforming kinetic energy at the macroscopic level to kinetic energy at the atomic level.

\section*{Mechanical Energy and Conservation of Energy}
We saw earlier that mechanical energy can be either potential or kinetic. In this section we will see how energy is transformed from one of these forms to the other. We will also see that, in a closed system, the sum of these forms of energy remains constant.

Quite a bit of potential energy is gained by a roller coaster car and its passengers when they are raised to the top of the first hill. Remember that the potential part of the term means that energy has been stored and can be used at another time. You will see that this stored energy can either be used to do work or can be transformed into kinetic energy. For example, when an object that has gravitational potential energy falls, its energy is converted to kinetic energy. Remember that both work and energy are expressed in joules.

Refer back to Figure 9.3. The amount of work required to raise the TV from point A to point B is equal to the amount of gravitational potential energy the TV gains from its height above the ground. This is generally true for any object raised above the ground. If all the work done on an object is used to raise the object above the ground, the amount work equals the object's gain in gravitational potential energy. However, note that because of the work done by friction, these energy-work transformations are never perfect. Friction causes the loss of some useful energy. In the discussions to follow, we will use the approximation that transformations are frictionless.

Now, let's look at the roller coaster in Figure 9.7. Work was done on the roller coaster to get it to the top of the first rise; at this point, the roller coaster has gravitational potential energy. It is moving slowly, so it also has a small amount of kinetic energy. As the car descends the first slope, its \(P E\) is converted to \(K E\). At the low point much of the original \(P E\) has been transformed to \(K E\), and speed is at a maximum. As the car moves up the next slope, some of the \(K E\) is transformed back into \(P E\) and the car slows down.

\begin{figure}[h]
\begin{center}
  \includegraphics[max width=\textwidth]{8b6c1c7c-84c8-44b6-9353-a83bfff7835e-19}
\captionsetup{labelformat=empty}
\caption{Figure 9.7 During this roller coaster ride, there are conversions between potential and kinetic energy.}
\end{center}
\end{figure}

\section*{Teacher Support}
Teacher Support [OL][AL] Ask if definitions of energy make sense to the class, and try to bring out any expressions of confusions or misconceptions. Help them make the logical leap that, if energy is the ability to do work, it makes sense that it is expressed by the same unit of measurement. Ask students to name all the forms of energy they can. Ask if this helps them get a feel for the nature of energy. Ask if they have a problem seeing how some forms of energy, such as sunlight, can do work.\\[0pt]
[BL][OL] You may want to introduce the concept of a reference point as the starting point of motion. Relate this to the origin of a coordinate grid.\\[0pt]
[BL] Make it clear that energy is a different property with different units than either force or power.\\[0pt]
[OL] Help students understand that the speed with which the TV is delivered is not part of the calculation of \(P E\). It is assumed that the speed is constant. Any \(K E\) due to increases in delivery speed will be lost when motion stops.\\[0pt]
[BL] Be sure there is a clear understanding of the distinction between kinetic and potential energy and between velocity and acceleration. Explain that the word potential means that the energy is available but it does not mean that it has to be used or will be used.

\section*{Virtual Physics}
Energy Skate Park Basics This simulation shows how kinetic and potential energy are related, in a scenario similar to the roller coaster. Observe the changes in \(K E\) and \(P E\) by clicking on the bar graph boxes. Also try the three differently shaped skate parks. Drag the skater to the track to start the animation.

Click to view content\\
This simulation (\href{http://phet.colorado.edu/en/simulation/energy-skate-parkbasics}{http://phet.colorado.edu/en/simulation/energy-skate-parkbasics}) shows how kinetic and potential energy are related, in a scenario similar to the roller coaster. Observe the changes in KE and PE by clicking on the bar graph boxes. Also try the three differently shaped skate parks. Drag the skater to the track to start the animation. The bar graphs show how KE and PE are transformed back and forth. Which statement best explains what happens to the mechanical energy of the system as speed is increasing?\\
a. The mechanical energy of the system increases, provided there is no loss of energy due to friction. The energy would transform to kinetic energy when the speed is increasing.\\
b. The mechanical energy of the system remains constant provided there is no loss of energy due to friction. The energy would transform to kinetic energy when the speed is increasing.\\
c. The mechanical energy of the system increases provided there is no loss of energy due to friction. The energy would transform to potential energy when the speed is increasing.\\
d. The mechanical energy of the system remains constant provided there is no loss of energy due to friction. The energy would transform to potential energy when the speed is increasing.

\section*{Teacher Support}
Teacher Support This animation shows the transformations between KE and \(P E\) and how speed varies in the process. Later we can refer back to the animation to see how friction converts some of the mechanical energy into heat and how total energy is conserved.

On an actual roller coaster, there are many ups and downs, and each of these is accompanied by transitions between kinetic and potential energy. Assume that no energy is lost to friction. At any point in the ride, the total mechanical energy is the same, and it is equal to the energy the car had at the top of the first rise. This is a result of the law of conservation of energy, which says that, in a closed system, total energy is conserved - that is, it is constant. Using subscripts 1 and 2 to represent initial and final energy, this law is expressed as \(K E_{1}+P E_{1}=K E_{2}+P E_{2}\).

Either side equals the total mechanical energy. The phrase in a closed system\\
means we are assuming no energy is lost to the surroundings due to friction and air resistance. If we are making calculations on dense falling objects, this is a good assumption. For the roller coaster, this assumption introduces some inaccuracy to the calculation.

\section*{Calculations involving Mechanical Energy and Conservation of Energy}
\section*{Tips For Success}
When calculating work or energy, use units of meters for distance, newtons for force, kilograms for mass, and seconds for time. This will assure that the result is expressed in joules.

\section*{Teacher Support}
Teacher Support [BL][OL] Impress upon the students the significant amount of work required to get a roller coaster car to the top of the first, highest point. Compare it to the amount of work it would take to walk to the top of the roller coaster. Ask students why they may feel tired if they had to walk or climb to the top of the roller coaster (they have to use energy to exert the force required to move their bodies upwards against the force of gravity). Check if students can correctly predict that the ratio of the mass of the car to a person's mass would be the ratio of work done and energy gained (for example, if the car's mass was 10 times a person's mass, the amount of work needed to move the car to the top of the hill would be 10 times the work needed to walk up the hill).

\section*{Watch Physics}
Conservation of Energy This video discusses conversion of \(P E\) to \(K E\) and conservation of energy. The scenario is very similar to the roller coaster and the skate park. It is also a good explanation of the energy changes studied in the snap lab.

Click to view content

\section*{Teacher Support}
Teacher Support Before showing the video, review all the equations involving kinetic and potential energy and conservation of energy. Also be sure the students have a qualitative understanding of the energy transformation taking place. Refer back to the snap lab and the simulation lab.

Watch Physics: Conservation of Energy. This video introduces the law of conservation of energy and explains how potential energy is converted to kinetic\\
energy.\\
Click to view content\\
Did you expect the speed at the bottom of the slope to be the same as when the object fell straight down? Which statement best explains why this is not exactly the case in real-life situations?\\
a. The speed was the same in the scenario in the animation because the object was sliding on the ice, where there is large amount of friction. In real life, much of the mechanical energy is lost as heat caused by friction.\\
b. The speed was the same in the scenario in the animation because the object was sliding on the ice, where there is small amount of friction. In real life, much of the mechanical energy is lost as heat caused by friction.\\
c. The speed was the same in the scenario in the animation because the object was sliding on the ice, where there is large amount of friction. In real life, no mechanical energy is lost due to conservation of the mechanical energy.\\
d. The speed was the same in the scenario in the animation because the object was sliding on the ice, where there is small amount of friction. In real life, no mechanical energy is lost due to conservation of the mechanical energy.

\section*{Worked Example}
Applying the Law of Conservation of Energy A 10 kg rock falls from a 20 m cliff. What is the kinetic and potential energy when the rock has fallen 10 m ?

\section*{Strategy}
Choose the equation.\\
\(K E_{1}+P E_{1}=K E_{2}+P E_{2}\)\\
9.4\\
\(K E=\frac{1}{2} m \mathbf{v}^{2} ; \quad P E=m \mathbf{g} h\)\\
9.5\\
\(\frac{1}{2} m \mathbf{v}_{1}^{2}+m \mathbf{g} h_{1}=\frac{1}{2} m \mathbf{v}_{2}^{2}+m \mathbf{g} h_{2}\)\\
9.6

List the knowns.\\
\(m=10 \mathrm{~kg}, \mathbf{v}_{1}=0, \mathbf{g}=9.80\)\\
\(\frac{\mathrm{m}}{\mathrm{s}^{2}}\),\\
9.7\\
\(h_{1}=20 \mathrm{~m}, h_{2}=10 \mathrm{~m}\)\\
Identify the unknowns.\\
\(K E_{2}\) and \(P E_{2}\)\\
Substitute the known values into the equation and solve for the unknown variables.

Solution\\
\(P E_{2}=m \mathbf{g} h_{2}=10(9.80) 10=980 \mathrm{~J}\)\\
9.8\\
\(K E_{2}=P E_{2}-\left(K E_{1}+P E_{1}\right)=980-\{[0-[10(9.80) 20]]\}=980 \mathrm{~J}\)\\
9.9

Discussion\\
Alternatively, conservation of energy equation could be solved for \(\mathbf{v}_{2}\) and \(K E_{2}\) could be calculated. Note that \(m\) could also be eliminated.

\section*{Tips For Success}
Note that we can solve many problems involving conversion between \(K E\) and \(P E\) without knowing the mass of the object in question. This is because kinetic and potential energy are both proportional to the mass of the object. In a situation where \(K E=P E\), we know that \(m \mathbf{g} h=(1 / 2) m \mathbf{v}^{2}\).

Dividing both sides by \(m\) and rearranging, we have the relationship\\
\(2 \mathbf{g} h=\mathbf{v}^{2}\).

\section*{Teacher Support}
Teacher Support Kinetic and potential energy are both proportional to the mass of the object. In a situation where \(K E=P E\), we know that \(m g h=\) (1/2) \(m \mathbf{v}^{2}\). Dividing both sides by \(m\) and rearranging, we get the relationship \(2 \mathbf{g} h=\mathbf{v}^{2}\).

\section*{Practice Problems}
5.

A child slides down a playground slide. If the slide is 3 m high and the child weighs 300 N , how much potential energy does the child have at the top of the slide? (Round \(g\) to\\
\(10 \mathrm{~m} / \mathrm{s}^{2}\).\\
)\\
a. 0 J\\
b. 100 J\\
c. 300 J\\
d. 900 J\\
6.

A 0.2 kg apple on an apple tree has a potential energy of 10 J . It falls to the ground, converting all of its PE to kinetic energy. What is the velocity of the apple just before it hits the ground?\\
a. \(0 \mathrm{~m} / \mathrm{s}\)\\
b. \(2 \mathrm{~m} / \mathrm{s}\)\\
c. \(10 \mathrm{~m} / \mathrm{s}\)\\
d. \(50 \mathrm{~m} / \mathrm{s}\)

\section*{Snap Lab}
Converting Potential Energy to Kinetic Energy In this activity, you will calculate the potential energy of an object and predict the object's speed when all that potential energy has been converted to kinetic energy. You will then check your prediction.

You will be dropping objects from a height. Be sure to stay a safe distance from the edge. Don't lean over the railing too far. Make sure that you do not drop objects into an area where people or vehicles pass by. Make sure that dropping objects will not cause damage.

You will need the following:\\
Materials for each pair of students:

\begin{itemize}
  \item Four marbles (or similar small, dense objects)
  \item Stopwatch
\end{itemize}

Materials for class:

\begin{itemize}
  \item Metric measuring tape long enough to measure the chosen height
  \item A scale
\end{itemize}

Instructions\\
Procedure

\begin{enumerate}
  \item Work with a partner. Find and record the mass of four small, dense objects per group.
  \item Choose a location where the objects can be safely dropped from a height of at least 15 meters. A bridge over water with a safe pedestrian walkway will work well.
  \item Measure the distance the object will fall.
  \item Calculate the potential energy of the object before you drop it using \(P E =m \mathbf{g} h=(9.80) m h\).
  \item Predict the kinetic energy and velocity of the object when it lands using \(P E=K E\) and so, \(m \mathbf{g} h=\frac{m \mathbf{v}^{2}}{2} ; \mathbf{v}=\sqrt{2(9.80) h}=4.43 \sqrt{h}\).
  \item One partner drops the object while the other measures the time it takes to fall.
  \item Take turns being the dropper and the timer until you have made four measurements.
  \item Average your drop multiplied by and calculate the velocity of the object when it landed using \(\mathbf{v}=\mathbf{a} t=\mathbf{g} t=(9.80) t\).
  \item Compare your results to your prediction.
\end{enumerate}

\section*{Teacher Support}
Teacher Support Before students begin the lab, find the nearest location where objects can be dropped safely from a height of at least 15 m .

As students work through the lab, encourage lab partners to discuss their observations. Encourage them to discuss differences in results between partners. Ask if there is any confusion about the equations they are using and whether they seem valid based on what they have already learned about mechanical energy. Ask them to discuss the effect of air resistance and how density is related to that effect.\\
7.

Galileo's experiments proved that, contrary to popular belief, heavy objects do not fall faster than light objects. How do the equations you used support this fact?\\
a. Heavy objects do not fall faster than the light objects because while conserving the mechanical energy of the system, the mass term gets cancelled and the velocity is independent of the mass. In real life, the variation in the velocity of the different objects is observed because of the non-zero air resistance.\\
b. Heavy objects do not fall faster than the light objects because while conserving the mechanical energy of the system, the mass term does not get cancelled and the velocity is dependent on the mass. In real life, the variation in the velocity of the different objects is observed because of the non-zero air resistance.\\
c. Heavy objects do not fall faster than the light objects because while conserving the mechanical energy the system, the mass term gets cancelled and the velocity is independent of the mass. In real life, the variation in the velocity of the different objects is observed because of zero air resistance.\\
d. Heavy objects do not fall faster than the light objects because while conserving the mechanical energy of the system, the mass term does not get cancelled and the velocity is dependent on the mass. In real life, the variation in the velocity of the different objects is observed because of zero air resistance.

\section*{Check Your Understanding}
8.

Describe the transformation between forms of mechanical energy that is happening to a falling skydiver before her parachute opens.\\
a. Kinetic energy is being transformed into potential energy.\\
b. Potential energy is being transformed into kinetic energy.\\
c. Work is being transformed into kinetic energy.\\
d. Kinetic energy is being transformed into work.\\
9.

True or false - If a rock is thrown into the air, the increase in the height would increase the rock's kinetic energy, and then the increase in the velocity as it falls to the ground would increase its potential energy.\\
a. True\\
b. False\\
10.

Identify equivalent terms for stored energy and energy of motion.\\
a. Stored energy is potential energy, and energy of motion is kinetic energy.\\
b. Energy of motion is potential energy, and stored energy is kinetic energy.\\
c. Stored energy is the potential as well as the kinetic energy of the system.\\
d. Energy of motion is the potential as well as the kinetic energy of the system.

\section*{Teacher Support}
Teacher Support Use the Check Your Understanding questions to assess students' achievement of the section's learning objectives. If students are struggling with a specific objective, the Check Your Understanding will help identify which one and direct students to the relevant content.

\subsection*{9.3 Simple Machines}
\section*{Section Learning Objectives}
By the end of this section, you will be able to do the following:

\begin{itemize}
  \item Describe simple and complex machines
  \item Calculate mechanical advantage and efficiency of simple and complex machines
\end{itemize}

\section*{Teacher Support}
Teacher Support The learning objectives in this section will help your students master the following standards:

\begin{itemize}
  \item (6) Science concepts. The student knows that changes occur within a physical system and applies the laws of conservation of energy and momentum. The student is expected to:
  \item (C) describe simple and complex machines and solve problems involving simple machines;
  \item (D) define input work, output work, mechanical advantage, and efficiency of machines.
\end{itemize}

In addition, the High School Physics Laboratory Manual addresses content in this section in the lab titled: Work and Energy, as well as the following standards:

\begin{itemize}
  \item (6) Science concepts. The student knows that changes occur within a physical system and applies the laws of conservation of energy and momentum. The student is expected to:
  \item (D) demonstrate and apply the laws of conservation of energy and conservation of momentum in one dimension.
\end{itemize}

\section*{Section Key Terms}
\section*{Teacher Support}
Teacher Support In this section you will apply what you have learned about work to find the mechanical advantage and efficiency of simple machines.\\[0pt]
[BL][OL] Ask the students what they know about machines and work. Dispel any misconceptions that machines reduce the amount of work. Be sure students do not equate machines and motors by asking for (and, if necessary, providing)\\
examples of machines that are not motorized. Explain that simple machines are often hand-held, and that they reduce force, not work.\\[0pt]
[AL] Ask for recall of the formula \(W=\mathbf{f} d\). Explain that the product of force and distance is critical to understanding simple machines. Because the amount of work is not changed, the term \(\mathbf{f} d\) does not change, but force can decrease if distance increases. This is the underlying principle of all simple machines.

\section*{Simple Machines}
Simple machines make work easier, but they do not decrease the amount of work you have to do. Why can't simple machines change the amount of work that you do? Recall that in closed systems the total amount of energy is conserved. A machine cannot increase the amount of energy you put into it. So, why is a simple machine useful? Although it cannot change the amount of work you do, a simple machine can change the amount of force you must apply to an object, and the distance over which you apply the force. In most cases, a simple machine is used to reduce the amount of force you must exert to do work. The down side is that you must exert the force over a greater distance, because the product of force and distance, \(\mathbf{f} d\), (which equals work) does not change.

Let's examine how this works in practice. In Figure 9.8(a), the worker uses a type of lever to exert a small force over a large distance, while the pry bar pulls up on the nail with a large force over a small distance. Figure 9.8(b) shows the how a lever works mathematically. The effort force, applied at \(\mathbf{F}_{\mathbf{e}}\), lifts the load (the resistance force) which is pushing down at \(\mathbf{F}_{r}\). The triangular pivot is called the fulcrum; the part of the lever between the fulcrum and \(\mathbf{F}_{e}\) is the effort arm, \(L_{e}\); and the part to the left is the resistance arm, \(L_{r}\). The mechanical advantage is a number that tells us how many times a simple machine multiplies the effort force. The ideal mechanical advantage, \(I M A\), is the mechanical advantage of a perfect machine with no loss of useful work caused by friction between moving parts. The equation for \(I M A\) is shown in Figure 9.8(b).

\begin{figure}[h]
\begin{center}
  \includegraphics[max width=\textwidth]{8b6c1c7c-84c8-44b6-9353-a83bfff7835e-28}
\captionsetup{labelformat=empty}
\caption{Figure 9.8 (a) A pry bar is a type of lever. (b) The ideal mechanical advantage equals the length of the effort arm divided by the length of the resistance arm of a lever.}
\end{center}
\end{figure}

In general, the \(I M A=\) the resistance force, \(\mathbf{F}_{r}\), divided by the effort force, \(\mathbf{F}_{e}\). IMA also equals the distance over which the effort is applied, \(d_{e}\), divided by the distance the load travels, \(d_{r}\).\\
\(I M A=\frac{\mathbf{F}_{r}}{\mathbf{F}_{e}}=\frac{d_{e}}{d_{r}}\)\\
Getting back to conservation of energy, for any simple machine, the work put into the machine, \(W_{i}\), equals the work the machine puts out, \(W_{o}\). Combining this with the information in the paragraphs above, we can write\\
\(W_{i}=W_{o}\)\\
\(\mathbf{F}_{e} d_{e}=\mathbf{F}_{r} d_{r}\)\\
If \(\mathbf{F}_{e}<\mathbf{F}_{r}\), then \(d_{e}>d_{r}\).\\
The equations show how a simple machine can output the same amount of work while reducing the amount of effort force by increasing the distance over which the effort force is applied.

\section*{Watch Physics}
Introduction to Mechanical Advantage This video shows how to calculate the IMA of a lever by three different methods: (1) from effort force and resistance force; (2) from the lengths of the lever arms, and; (3) from the distance over which the force is applied and the distance the load moves.

Click to view content

\section*{Teacher Support}
Teacher Support The beginning of this video may cause more confusion than illumination. It shows a derivation using trig functions that is beyond the scope of this chapter. Interested students may want to work their way through it. Most students should skip to the final two or three minutes which explain the basics of calculating IMA of a lever from different ratios. Review \(W=\mathbf{f} d\).

Watch Physics: Introduction to Mechanical Advantage. This video introduces simple machines, mechanical advantage and moments.

Click to view content\\
Two children of different weights are riding a seesaw. How do they position themselves with respect to the pivot point (the fulcrum) so that they are balanced?\\
a. The heavier child sits closer to the fulcrum.\\
b. The heavier child sits farther from the fulcrum.\\
c. Both children sit at equal distance from the fulcrum.\\
d. Since both have different weights, they will never be in balance.

Some levers exert a large force to a short effort arm. This results in a smaller force acting over a greater distance at the end of the resistance arm. Examples of this type of lever are baseball bats, hammers, and golf clubs. In another type of lever, the fulcrum is at the end of the lever and the load is in the middle, as in the design of a wheelbarrow.

\section*{Teacher Support}
Teacher Support [AL]Tell students there are two other classes of levers with different arrangements of load, fulcrum, and effort. Ask them first to try to sketch these. After they have discovered the three kinds, with or without your help, ask if they can think of examples of the types not shown in Figure 9.8.

The simple machine shown in Figure 9.9 is called a wheel and axle. It is actually a form of lever. The difference is that the effort arm can rotate in a complete circle around the fulcrum, which is the center of the axle. Force applied to the outside of the wheel causes a greater force to be applied to the rope that is wrapped around the axle. As shown in the figure, the ideal mechanical advantage is calculated by dividing the radius of the wheel by the radius of the axle. Any crank-operated device is an example of a wheel and axle.

\begin{figure}[h]
\begin{center}
  \includegraphics[max width=\textwidth]{8b6c1c7c-84c8-44b6-9353-a83bfff7835e-30}
\captionsetup{labelformat=empty}
\caption{Figure 9.9 Force applied to a wheel exerts a force on its axle.}
\end{center}
\end{figure}

\section*{Teacher Support}
Teacher Support [BL][OL] See if the students grasp the idea that a wheel and axle is really a type of lever. Show them that it looks more like a lever if the wheel is replaced by a crank. Give some examples: hand-powered windlass, steering wheel, door knob, and so on. Ask them why steering wheels had a greater diameter before power steering was invented.\\[0pt]
[AL] Explain that wheels on vehicles are not really simple machines in the same sense as the one in Figure 9.9. The axle on a vehicle does not do work on a load. Energy loss to friction is reduced, but nothing is lifted.

An inclined plane and a wedge are two forms of the same simple machine. A wedge is simply two inclined planes back to back. Figure 9.10 shows the simple formulas for calculating the IMAs of these machines. All sloping, paved surfaces for walking or driving are inclined planes. Knives and axe heads are examples of wedges.

\[
I M A=\frac{L}{h}
\]

\begin{figure}[h]
\begin{center}
  \includegraphics[max width=\textwidth]{8b6c1c7c-84c8-44b6-9353-a83bfff7835e-31}
\captionsetup{labelformat=empty}
\caption{Figure 9.10 An inclined plane is shown on the left, and a wedge is shown on the right.}
\end{center}
\end{figure}

\section*{Teacher Support}
Teacher Support [BL][OL] Talk about how inclined planes and wedges are similar and different. Note that, when using an inclined plane the load moves, but when using a wedge the load is stationary and the machine moves. Explain why more energy is usually lost to friction with these machines than with other simple machines.

The screw shown in Figure 9.11 is actually a lever attached to a circular inclined plane. Wood screws (of course) are also examples of screws. The lever part of these screws is a screw driver. In the formula for IMA, the distance between screw threads is called pitch and has the symbol \(P\).

\[
I M A=\frac{2 \pi L}{P}
\]

\begin{figure}[h]
\begin{center}
  \includegraphics[max width=\textwidth]{8b6c1c7c-84c8-44b6-9353-a83bfff7835e-32}
\captionsetup{labelformat=empty}
\caption{Figure 9.11 The screw shown here is used to lift very heavy objects, like the corner of a car or a house a short distance.}
\end{center}
\end{figure}

\section*{Teacher Support}
Teacher Support [BL][OL] Suggest that a screw is classified as a separate type of simple machine perhaps because it looks so different from what it really is-an inclined plane which sometimes is turned by a lever. Explain that the combined mechanical advantage can be great. Devices like the one shown in Figure 9.10 are used to lift cars and even houses. Have the students compare this screw to a wood screw and a circular stairway.\\[0pt]
[AL] Ask students how the forces exerted by a wood screw are different from those exerted by the screw in Figure 9.10. Ask for an explanation of the 2 in the equation for IMA.

Figure 9.12 shows three different pulley systems. Of all simple machines, mechanical advantage is easiest to calculate for pulleys. Simply count the number of ropes supporting the load. That is the IMA. Once again we have to exert force over a longer distance to multiply force. To raise a load 1 meter with a pulley system you have to pull \(N\) meters of rope. Pulley systems are often used to raise flags and window blinds and are part of the mechanism of construction cranes.

\begin{figure}[h]
\begin{center}
  \includegraphics[max width=\textwidth]{8b6c1c7c-84c8-44b6-9353-a83bfff7835e-33}
\captionsetup{labelformat=empty}
\caption{Figure 9.12 Three pulley systems are shown here.}
\end{center}
\end{figure}

\section*{Teacher Support}
Teacher Support [BL][OL] The calculation for IMA of a pulley seems too easy to be true, but it is. Ask students to try to understand why IMA is simply \(N\). Tell them that watching the video should make this point clear. Pulleys were once seen on sailing ships and farms, where they were used lift heavy loads. The overhang you may have seen on the end of old barn roofs is where a pulley was once attached. This way bales of hay could be lifted into the hay loft without getting wet. Pulleys can still be seen in use, most commonly on large building cranes.

\section*{Watch Physics}
Mechanical Advantage of Inclined Planes and Pulleys The first part of this video shows how to calculate the IMA of pulley systems. The last part shows how to calculate the IMA of an inclined plane.

Click to view content

\section*{Teacher Support}
Teacher Support Review what was learned about the \(I M A\) of inclined planes and pulley systems before watching the video. Remind the students that, for an ideal machine, work in = work out and that \(W=\mathbf{f} d\). The video shows how to find the \(\mathbf{f s}\) and the \(d\) s.

\section*{Grasp Check}
How could you use a pulley system to lift a light load to great height?\\
a. Reduce the radius of the pulley.\\
b. Increase the number of pulleys.\\
c. Decrease the number of ropes supporting the load.\\
d. Increase the number of ropes supporting the load.

A complex machine is a combination of two or more simple machines. The wire cutters in Figure 9.13 combine two levers and two wedges. Bicycles include wheel and axles, levers, screws, and pulleys. Cars and other vehicles are combinations of many machines.

\begin{figure}[h]
\begin{center}
  \includegraphics[max width=\textwidth]{8b6c1c7c-84c8-44b6-9353-a83bfff7835e-34}
\captionsetup{labelformat=empty}
\caption{Figure 9.13 Wire cutters are a common complex machine.}
\end{center}
\end{figure}

\section*{Teacher Support}
Teacher Support [BL][OL] Be sure students understand that a complex machine is just a combination of simple machines and is still fairly simple. Don't let them confuse the term with complicated machines such as computers. Note that the IMAs of the individual simple machines in a complex machine usually multiply because the output force of one machine becomes the input force of the other machine. For an additional fun activity, have the students search the Internet for Rube Goldberg machine.

\section*{Calculating Mechanical Advantage and Efficiency of Simple Machines}
In general, the \(I M A=\) the resistance force, \(\mathbf{F}_{\mathrm{r}}\), divided by the effort force, \(\mathbf{F}_{\mathrm{e}}\). IMA also equals the distance over which the effort is applied, \(d_{e}\), divided by the distance the load travels, \(d_{r}\).\\
IMA \(=\frac{\mathbf{F}_{r}}{\mathbf{F}_{e}}=\frac{d_{e}}{d_{r}}\)

Refer back to the discussions of each simple machine for the specific equations for the IMA for each type of machine.

No simple or complex machines have the actual mechanical advantages calculated by the IMA equations. In real life, some of the applied work always ends up as wasted heat due to friction between moving parts. Both the input work ( \(W_{i}\) ) and output work ( \(W_{o}\) ) are the result of a force, \(\mathbf{F}\), acting over a distance, \(d\).\\
\(W_{i}=\mathbf{F}_{i} d_{i} \operatorname{and} W_{o}=\mathbf{F}_{o} d_{o}\)\\
The efficiency output of a machine is simply the output work divided by the input work, and is usually multiplied by 100 so that it is expressed as a percent.\\
\(\%\) efficiency \(=\frac{W_{o}}{W_{i}} \times 100\)\\
Look back at the pictures of the simple machines and think about which would have the highest efficiency. Efficiency is related to friction, and friction depends on the smoothness of surfaces and on the area of the surfaces in contact. How would lubrication affect the efficiency of a simple machine?

\section*{Teacher Support}
Teacher Support [BL][OL] Review the material on loss of mechanical energy to heat and the law of conservation of energy. Explain how heat lost because of friction assures that \(W_{o}\) will always be less than \(W_{i}\) preventing efficiency from ever reaching \(100 \%\).

\section*{Worked Example}
Efficiency of a Lever The input force of 11 N acting on the effort arm of a lever moves 0.4 m , which lifts a 40 N weight resting on the resistance arm a distance of 0.1 m . What is the efficiency of the machine?

\section*{Strategy}
State the equation for efficiency of a simple machine, \(\%\) efficiency \(=\frac{W_{o}}{W_{i}} \times 100\), and calculate \(W_{o}\) and \(W_{i}\). Both work values are the product \(F d\).

Solution\\
\(W_{i}=\mathbf{F}_{i} d_{i}=(11)(0.4)=4.4 \mathrm{~J}\) and \(W_{o}=\mathbf{F}_{o} d_{o}=(40)(0.1)=4.0 \mathrm{~J}\), then \(\%\) efficiency \(=\frac{W_{o}}{W_{i}} \times 100=\frac{4.0}{4.4} \times 100=91 \%\)

Discussion\\
Efficiency in real machines will always be less than 100 percent because of work that is converted to unavailable heat by friction and air resistance. \(W_{o}\) and\\
\(W_{i}\) can always be calculated as a force multiplied by a distance, although these quantities are not always as obvious as they are in the case of a lever.

\section*{Teacher Support}
Teacher Support Teaching tip-When calculating efficiency, it is easy enough to understand what force in and force out are: the force you apply is force in and the weight of the object that is being lifted is force out. The input and output distances are easier to see for the lever, inclined plane and wedge. The other three are not as obvious. For a pulley system, the input distance is how far you pull the rope, and the output distance is the distance the load rises. For a wheel and axle, the input distance is the circumference of the wheel, and the output distance is the circumference of the axle. For a screw, the input distance is the circumference of the circle over which the force is applied, and the output distance is the distance between the screw threads.

\section*{Practice Problems}
11.

\begin{figure}[h]
\begin{center}
  \includegraphics[max width=\textwidth]{8b6c1c7c-84c8-44b6-9353-a83bfff7835e-36}
\captionsetup{labelformat=empty}
\caption{Figure 9.14}
\end{center}
\end{figure}

An inclined plane that is 5 m long and 2 m high is used to load a large crate into the back of a truck. What is the IMA of the inclined plane?\\
a. 0.4\\
b. 2.5\\
c. \(0.4 \backslash, \backslash \operatorname{text}\{\mathrm{~m}\}\)\\
d. \(2.5 \backslash, \backslash \operatorname{text}\{\mathrm{~m}\}\)\\
12.

If a pulley system can lift a 200 N load with an effort force of 52 N and has an efficiency of almost 100 percent, how many ropes are supporting the load?\\
a. 1 rope is required because the actual mechanical advantage is 0.26 .\\
b. 1 rope is required because the actual mechanical advantage is 3.80 .\\
c. 4 ropes are required because the actual mechanical advantage is 0.26 .\\
d. 4 ropes are required because the actual mechanical advantage is 3.80 .

\section*{Check Your Understanding}
13.

True or false - The efficiency of a simple machine is always less than 100 percent because some small fraction of the input work is always converted to heat energy due to friction.\\
a. True\\
b. False\\
14.

The circular handle of a faucet is attached to a rod that opens and closes a valve when the handle is turned. If the rod has a diameter of 1 cm and the IMA of the machine is 6 , what is the radius of the handle?\\
A. 0.08 cm\\
B. 0.17 cm\\
C. 3.0 cm\\
D. 6.0 cm

\section*{Teacher Support}
Teacher Support Use the Check Your Understanding questions to assess students' achievement of the section's learning objectives. If students are struggling with a specific objective, the Check Your Understanding will help identify which one and direct students to the relevant content.

\section*{Key Terms}
complex machine a machine that combines two or more simple machines\\
efficiency output work divided by input work\\
energy the ability to do work\\
gravitational potential energy energy acquired by doing work against gravity\\
ideal mechanical advantage the mechanical advantage of an idealized machine that loses no energy to friction\\
inclined plane a simple machine consisting of a slope\\
input work effort force multiplied by the distance over which it is applied\\
joule the metric unit for work and energy; equal to 1 newton meter ( Nm )\\
kinetic energy energy of motion\\
law of conservation of energy states that energy is neither created nor destroyed\\
lever a simple machine consisting of a rigid arm that pivots on a fulcrum\\
mechanical advantage the number of times the input force is multiplied\\
mechanical energy kinetic or potential energy\\
output work output force multiplied by the distance over which it acts\\
potential energy stored energy\\
power the rate at which work is done\\
pulley a simple machine consisting of a rope that passes over one or more grooved wheels\\
screw a simple machine consisting of a spiral inclined plane\\
simple machine a machine that makes work easier by changing the amount or direction of force required to move an object\\
watt the metric unit of power; equivalent to joules per second\\
wedge a simple machine consisting of two back-to-back inclined planes\\
wheel and axle a simple machine consisting of a rod fixed to the center of a wheel\\
work force multiplied by distance\\
work-energy theorem states that the net work done on a system equals the change in kinetic energy

\section*{Key Equations}
9.1 Work, Power, and the Work-Energy Theorem

\subsection*{9.2 Mechanical Energy and Conservation of Energy}
9.3 Simple Machines

\section*{Section Summary}
\subsection*{9.1 Work, Power, and the Work-Energy Theorem}
\begin{itemize}
  \item Doing work on a system or object changes its energy.
  \item The work-energy theorem states that an amount of work that changes the velocity of an object is equal to the change in kinetic energy of that object.The work-energy theorem states that an amount of work that changes the velocity of an object is equal to the change in kinetic energy of that object.
  \item Power is the rate at which work is done.
\end{itemize}

\subsection*{9.2 Mechanical Energy and Conservation of Energy}
\begin{itemize}
  \item Mechanical energy may be either kinetic (energy of motion) or potential (stored energy).
  \item Doing work on an object or system changes its energy.
  \item Total energy in a closed, isolated system is constant.
\end{itemize}

\subsection*{9.3 Simple Machines}
\begin{itemize}
  \item The six types of simple machines make work easier by changing the \(\mathbf{f} d\) term so that force is reduced at the expense of increased distance.
  \item The ratio of output force to input force is a machine's mechanical advantage.
  \item Combinations of two or more simple machines are called complex machines.
  \item The ratio of output work to input work is a machine's efficiency.
\end{itemize}

\end{document}