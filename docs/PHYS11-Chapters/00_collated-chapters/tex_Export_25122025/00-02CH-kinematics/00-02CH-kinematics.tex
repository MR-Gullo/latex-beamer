\documentclass[10pt]{article}
\usepackage[utf8]{inputenc}
\usepackage[T1]{fontenc}
\usepackage{graphicx}
\usepackage[export]{adjustbox}
\graphicspath{ {./images/} }
\usepackage{caption}
\usepackage{amsmath}
\usepackage{amsfonts}
\usepackage{amssymb}
\usepackage[version=4]{mhchem}
\usepackage{stmaryrd}

\DeclareUnicodeCharacter{00D7}{\ifmmode\times\else{$\times$}\fi}

\begin{document}
\captionsetup{singlelinecheck=false}
\begin{figure}[h]
\begin{center}
  \includegraphics[max width=\textwidth]{1bfcec96-a1a9-4e0f-9291-d0b7d1308d1c-01}
\captionsetup{labelformat=empty}
\caption{Figure 2.1 Shanghai Maglev. At this rate, a train traveling from Boston to Washington, DC, a distance of 439 miles, could make the trip in under an hour and a half. Presently, the fastest train on this route takes over six hours to cover this distance. (Alex Needham, Public Domain)}
\end{center}
\end{figure}

\section*{Chapter Outline}
2.1 Relative Motion, Distance, and Displacement\\
2.2 Speed and Velocity\\
2.3 Position vs. Time Graphs\\
2.4 Velocity vs. Time Graphs

\section*{Introduction}
\section*{Teacher Support}
Teacher Support Have the students describe the photo of the train and discuss its motion. Tell them they will learn about motion. Start the discussion with how a train moves, and guide them toward discussing concepts of displacement, velocity, and acceleration. Ask questions: How do we know something is moving? What defines motion? What direction does the train move? What adjectives describe its motion? If it was a moving ball instead of a train, how would its motion be different? How would the train's motion change if its wheels\\
were square instead of round or if it had studded tires? Try to uncover what ideas students already have about motion.

Outside of an airplane, have you ever traveled faster than 150 mph ? Can you imagine traveling in a train like the one shown in Figure 2.1 that goes close to 300 mph? Despite the high speed, the people riding in this train may not notice that they are moving at all unless they look out the window! This is because motion, even motion at 300 mph , is relative to the observer.

In this chapter, you will learn why it is important to identify a reference frame in order to clearly describe motion. For now, the motion you describe will be one-dimensional. Within this context, you will learn the difference between distance and displacement as well as the difference between speed and velocity. Then you will look at some graphing and problem-solving techniques.

\section*{Teacher Support}
Teacher Support Before students begin this chapter, it would be useful to review these concepts:

\begin{itemize}
  \item Using significant figures in calculations
  \item Converting units
  \item Calculating average
  \item Commonly used terms
\end{itemize}

Demonstrate how to use the proper number of significant figures when adding, subtracting, multiplying, and dividing. Demonstrate how to convert from \(\mathrm{km} / \mathrm{h}\) to \(\mathrm{m} / \mathrm{s}\). Demonstrate how to average two numbers by dividing their sum by 2 . Explain that constant means unchanging, so constant speed refers to speed that is not changing. Explain that initial means starting, so initial time is the time at which the action mentioned in a problem begins. Explain that an object that is not moving is often described in physics as being at rest.

\subsection*{2.1 Relative Motion, Distance, and Displacement}
\section*{Section Learning Objectives}
By the end of this section, you will be able to do the following:

\begin{itemize}
  \item Describe motion in different reference frames
  \item Define distance and displacement, and distinguish between the two
  \item Solve problems involving distance and displacement
\end{itemize}

\section*{Teacher Support}
Teacher Support The learning objectives in this section will help your students master the following standards:

\begin{itemize}
  \item (4) Science concepts. The student knows and applies the laws governing motion in a variety of situations. The student is expected to:
  \item (B) describe and analyze motion in one dimension using equations with the concepts of distance, displacement, speed, average velocity, instantaneous velocity, and acceleration;
  \item (F) identify and describe motion relative to different frames of reference.
\end{itemize}

\section*{Section Key Terms}
\section*{Teacher Support}
Teacher Support [BL][OL] Start by asking what position is and how it is defined. You can use a toy car or other object. Then ask how they know the object has moved. Lead them to the idea of a defined starting point. Then bring in the concept of a numbered line as a way of quantifying motion.\\[0pt]
[AL] Ask students to describe the path of movement and emphasize that direction is a necessary component of a definition of motion. Ask the students to form pairs, and ask each pair to come up with their own definition of motion. Then compare and discuss definitions as a class. What components are necessary for a definition of motion?

\section*{Defining Motion}
Our study of physics opens with kinematics-the study of motion without considering its causes. Objects are in motion everywhere you look. Everything\\
from a tennis game to a space-probe flyby of the planet Neptune involves motion. When you are resting, your heart moves blood through your veins. Even in inanimate objects, atoms are always moving.

How do you know something is moving? The location of an object at any particular time is its position. More precisely, you need to specify its position relative to a convenient reference frame. Earth is often used as a reference frame, and we often describe the position of an object as it relates to stationary objects in that reference frame. For example, a rocket launch would be described in terms of the position of the rocket with respect to Earth as a whole, while a professor's position could be described in terms of where she is in relation to the nearby white board. In other cases, we use reference frames that are not stationary but are in motion relative to Earth. To describe the position of a person in an airplane, for example, we use the airplane, not Earth, as the reference frame. (See Figure 2.2.) Thus, you can only know how fast and in what direction an object's position is changing against a background of something else that is either not moving or moving with a known speed and direction. The reference frame is the coordinate system from which the positions of objects are described.\\
\includegraphics[max width=\textwidth, center]{1bfcec96-a1a9-4e0f-9291-d0b7d1308d1c-04}

Figure 2.2 Are clouds a useful reference frame for airplane passengers? Why or why not? (Paul Brennan, Public Domain)

\section*{Teacher Support}
Teacher Support [OL][AL]Explain that the word kinematics comes from a Greek term meaning motion. It is related to other English words, such as cinema (movies, or moving pictures) and kinesiology (the study of human motion).

Your classroom can be used as a reference frame. In the classroom, the walls are not moving. Your motion as you walk to the door, can be measured against the stationary background of the classroom walls. You can also tell if other things in the classroom are moving, such as your classmates entering the classroom or a book falling off a desk. You can also tell in what direction something is moving in the classroom. You might say, "The teacher is moving toward the door." Your reference frame allows you to determine not only that something is moving but also the direction of motion.

You could also serve as a reference frame for others' movement. If you remained seated as your classmates left the room, you would measure their movement away from your stationary location. If you and your classmates left the room together, then your perspective of their motion would be change. You, as the reference frame, would be moving in the same direction as your other moving classmates. As you will learn in the Snap Lab, your description of motion can be quite different when viewed from different reference frames.

\section*{Teacher Support}
Teacher Support [BL][OL] You may want to introduce the concept of a reference point as the starting point of motion. Relate this to the origin of a coordinate grid.\\[0pt]
[AL] Explain that the reference frames considered in this chapter are inertial reference frames, which means they are not accelerating. Engage students in a discussion of how it is the difference in motion between the reference frame of the observer and the reference frame of the object that is important in describing motion. The reference frames used in this chapter might be moving at a constant speed relative to each other, but they are not accelerating relative to each other.\\[0pt]
[BL][OL][Visual] Misconception: Students may assume that a reference frame is a background of motion instead of the frame from which motion is viewed. Demonstrate the difference by having one student stand at the front of the class. Explain that this student represents the background. Walk once across the room between the student and the rest of the class. Ask the student and others in the class to describe the direction of your motion. The class might describe your motion as to the right, but the student who is standing as a background to your motion would describe the motion as to the left. Conclude by reminding students that the reference frame is the viewpoint of the observer, not the background.\\[0pt]
[BL] Have students practice describing simple examples of motion in the class from different reference frames. For example, slide a book across a desk. Ask students to describe its motion from their reference point, from the book's reference point, and from another student's reference point.

\section*{Snap Lab}
Looking at Motion from Two Reference Frames In this activity you will look at motion from two reference frames. Which reference frame is correct?

\begin{itemize}
  \item Choose an open location with lots of space to spread out so there is less chance of tripping or falling due to a collision and/or loose basketballs.
  \item 1 basketball
\end{itemize}

Procedure

\begin{enumerate}
  \item Work with a partner. Stand a couple of meters away from your partner. Have your partner turn to the side so that you are looking at your partner's profile. Have your partner begin bouncing the basketball while standing in place. Describe the motion of the ball.
  \item Next, have your partner again bounce the ball, but this time your partner should walk forward with the bouncing ball. You will remain stationary. Describe the ball's motion.
  \item Again have your partner walk forward with the bouncing ball. This time, you should move alongside your partner while continuing to view your partner's profile. Describe the ball's motion.
  \item Switch places with your partner, and repeat Steps 1-3.
\end{enumerate}

\section*{Grasp Check}
How do the different reference frames affect how you describe the motion of the ball?\\
a. The motion of the ball is independent of the reference frame and is same for different reference frames.\\
b. The motion of the ball is independent of the reference frame and is different for different reference frames.\\
c. The motion of the ball is dependent on the reference frame and is same for different reference frames.\\
d. The motion of the ball is dependent on the reference frames and is different for different reference frames.

\section*{Teacher Support}
Teacher Support Before students begin the lab, arrange a location where pairs of students can have ample room to walk forward at least several meters.

As students work through the lab, encourage lab partners to discuss their observations. In Steps 1 and 3, students should observe the ball move straight up and straight down. In Step 2, students should observe the ball in a zigzag path away from the stationary observer.

After the lab, lead students in discussing their observations. Ask them which reference frame is the correct one. Then emphasize that there is not a single correct reference frame. All reference frames are equally valid.

\section*{Links To Physics}
\begin{center}
\includegraphics[max width=\textwidth]{1bfcec96-a1a9-4e0f-9291-d0b7d1308d1c-08}
\end{center}

Figure 2.3 Galileo Galilei (1564-1642) studied motion and developed the concept of a reference frame. (Domenico Tintoretto)

The idea that a description of motion depends on the reference frame of the observer has been known for hundreds of years. The \(17^{\text {th }}\)-century astronomer Galileo Galilei (Figure 2.3) was one of the first scientists to explore this idea. Galileo suggested the following thought experiment: Imagine a windowless ship moving at a constant speed and direction along a perfectly calm sea. Is there a way that a person inside the ship can determine whether the ship is moving? You can extend this thought experiment by also imagining a person standing on the shore. How can a person on the shore determine whether the ship is moving?

Galileo came to an amazing conclusion. Only by looking at each other can a person in the ship or a person on shore describe the motion of one relative to the other. In addition, their descriptions of motion would be identical. A person inside the ship would describe the person on the land as moving past the ship. The person on shore would describe the ship and the person inside it as moving past. Galileo realized that observers moving at a constant speed and direction relative to each other describe motion in the same way. Galileo had discovered that a description of motion is only meaningful if you specify a reference frame.

Imagine standing on a platform watching a train pass by. According to Galileo's conclusions, how would your description of motion and the description of motion by a person riding on the train compare?\\
a. I would see the train as moving past me, and a person on the train would see me as stationary.\\
b. I would see the train as moving past me, and a person on the train would see me as moving past the train.\\
c. I would see the train as stationary, and a person on the train would see me as moving past the train.\\
d. I would see the train as stationary, and a person on the train would also see me as stationary.

Distance vs. Displacement As we study the motion of objects, we must first be able to describe the object's position. Before your parent drives you to school, the car is sitting in your driveway. Your driveway is the starting position for the car. When you reach your high school, the car has changed position. Its new position is your school.

\begin{figure}[h]
\begin{center}
  \includegraphics[max width=\textwidth]{1bfcec96-a1a9-4e0f-9291-d0b7d1308d1c-10}
\captionsetup{labelformat=empty}
\caption{Figure 2.4 Your total change in position is measured from your house to your school.}
\end{center}
\end{figure}

Physicists use variables to represent terms. We will use d to represent car's position. We will use a subscript to differentiate between the initial position, \(\mathbf{d}_{0}\), and the final position, \(\mathbf{d}_{\mathrm{f}}\). In addition, vectors, which we will discuss later, will be in bold or will have an arrow above the variable. Scalars will be italicized.

\section*{Tips For Success}
In some books, \(\mathbf{x}\) or \(\mathbf{s}\) is used instead of \(\mathbf{d}\) to describe position. In \(\mathbf{d}_{0}\), said \(d\) naught, the subscript 0 stands for initial. When we begin to talk about two-dimensional motion, sometimes other subscripts will be used to describe horizontal position, \(\mathbf{d}_{\mathrm{x}}\), or vertical position, \(\mathbf{d}_{\mathrm{y}}\). So, you might see references to \(\mathbf{d}_{0 \mathrm{x}}\) and \(\mathbf{d}_{\mathrm{fy}}\).\\
Now imagine driving from your house to a friend's house located several kilometers away. How far would you drive? The distance an object moves is the length of the path between its initial position and its final position. The distance you drive to your friend's house depends on your path. As shown in Figure 2.5, distance is different from the length of a straight line between two points. The distance you drive to your friend's house is probably longer than the straight line between the two houses.

\begin{figure}[h]
\begin{center}
  \includegraphics[max width=\textwidth]{1bfcec96-a1a9-4e0f-9291-d0b7d1308d1c-11(1)}
\captionsetup{labelformat=empty}
\caption{Figure 2.5 A short line separates the starting and ending points of this motion, but the distance along the path of motion is considerably longer.}
\end{center}
\end{figure}

We often want to be more precise when we talk about position. The description of an object's motion often includes more than just the distance it moves. For instance, if it is a five kilometer drive to school, the distance traveled is 5 kilometers. After dropping you off at school and driving back home, your parent will have traveled a total distance of 10 kilometers. The car and your parent will end up in the same starting position in space. The net change in position of an object is its displacement, or \(\Delta \mathbf{d}\). The Greek letter delta, \(\Delta\), means change in.

\begin{figure}[h]
\begin{center}
  \includegraphics[max width=\textwidth]{1bfcec96-a1a9-4e0f-9291-d0b7d1308d1c-11}
\captionsetup{labelformat=empty}
\caption{Figure 2.6 The total distance that your car travels is 10 km , but the total displacement is 0 .}
\end{center}
\end{figure}

\section*{Teacher Support}
\section*{Teacher Support}
\section*{Teacher Demonstration}
Help students learn the difference between distance and displacement by showing examples of motion.

\begin{enumerate}
  \item As students watch, walk straight across the room and have students estimate the length of your path.
  \item Then, at same starting point, walk along a winding path to the same ending point.
  \item Again, have students estimate the length of your path.
\end{enumerate}

Ask-Which motion showed displacement? Which showed distance? Point out that the first motion shows displacement, and the second shows distance along a path. In both cases, the starting and ending points were the same.\\[0pt]
[OL] Be careful that students do not assume that initial position is always zero. Emphasize that although initial position is often zero, motion can start from any position relative to a starting point.\\[0pt]
[Visual] Demonstrate positive and negative displacement by placing two meter sticks on the ground with their zero marks end-to-end. As students watch, place a small car at the zero mark. Slowly move the car to students' right a short distance and ask students what its displacement is. Then move the car to the left of the zero mark. Point out that the car now has a negative displacement.

Students will learn more about vectors and scalars later when they study twodimensional motion. For now, it is sufficient to introduce the terms and let students know that a vector includes information about direction.\\[0pt]
[BL] Ask students whether each of the following is a vector quantity or a scalar quantity: temperature (scalar), force (vector), mass (scalar).\\[0pt]
[OL] Ask students to provide examples of vector quantities and scalar quantities.\\[0pt]
[Kinesthetic] Provide students with large arrows cut from construction paper. Have them use the arrows to identify the magnitude (number or length of arrows) and direction of displacement. Emphasize that distance cannot be represented by arrows because distance does not include direction.

\section*{Snap Lab}
Distance vs. Displacement In this activity you will compare distance and displacement. Which term is more useful when making measurements?

\begin{itemize}
  \item 1 recorded song available on a portable device
  \item 1 tape measure
  \item 3 pieces of masking tape
  \item A room (like a gym) with a wall that is large and clear enough for all pairs of students to walk back and forth without running into each other.
\end{itemize}

\section*{Procedure}
\begin{enumerate}
  \item One student from each pair should stand with their back to the longest wall in the classroom. Students should stand at least 0.5 meters away from each other. Mark this starting point with a piece of masking tape.
  \item The second student from each pair should stand facing their partner, about two to three meters away. Mark this point with a second piece of masking tape.
  \item Student pairs line up at the starting point along the wall.
  \item The teacher turns on the music. Each pair walks back and forth from the wall to the second marked point until the music stops playing. Keep count of the number of times you walk across the floor.
  \item When the music stops, mark your ending position with the third piece of masking tape.
  \item Measure from your starting, initial position to your ending, final position.
  \item Measure the length of your path from the starting position to the second marked position. Multiply this measurement by the total number of times you walked across the floor. Then add this number to your measurement from step 6.
  \item Compare the two measurements from steps 6 and 7 .
  \item Which measurement is your total distance traveled?
  \item Which measurement is your displacement?
  \item When might you want to use one over the other?\\
a. Measurement of the total length of your path from the starting position to the final position gives the distance traveled, and the measurement from your initial position to your final position is the displacement. Use distance to describe the total path between starting and ending points, and use displacement to describe the shortest path between starting and ending points.\\
b. Measurement of the total length of your path from the starting position to the final position is distance traveled, and the measurement from your initial position to your final position is displacement. Use distance to describe the shortest path between starting and ending points, and use displacement to describe the total path between starting and ending points.\\
c. Measurement from your initial position to your final position is distance traveled, and the measurement of the total length of your path from the starting position to the final position is displacement. Use distance to describe the total path between starting and ending points, and use displacement to describe the shortest path between starting and ending points.\\
d. Measurement from your initial position to your final position is distance traveled, and the measurement of the total length of your path from the starting position to the final position is displacement. Use distance to de-\\
scribe the shortest path between starting and ending points, and use displacement to describe the total path between starting and ending points.
\end{enumerate}

\section*{Teacher Support}
Teacher Support Choose a room that is large enough for all students to walk unobstructed. Make sure the total path traveled is short enough that students can walk back and forth across it multiple times during the course of a song. Have them measure the distance between the two points and come to a consensus. When students measure their displacement, make sure that they measure forward from the direction they marked as the starting position. After they have completed the lab, have them discuss their results.

If you are describing only your drive to school, then the distance traveled and the displacement are the same-5 kilometers. When you are describing the entire round trip, distance and displacement are different. When you describe distance, you only include the magnitude, the size or amount, of the distance traveled. However, when you describe the displacement, you take into account both the magnitude of the change in position and the direction of movement.

In our previous example, the car travels a total of 10 kilometers, but it drives five of those kilometers forward toward school and five of those kilometers back in the opposite direction. If we ascribe the forward direction a positive ( + ) and the opposite direction a negative ( - ), then the two quantities will cancel each other out when added together.

A quantity, such as distance, that has magnitude (i.e., how big or how much) but does not take into account direction is called a scalar. A quantity, such as displacement, that has both magnitude and direction is called a vector.

\section*{Watch Physics}
Vectors \& Scalars This video introduces and differentiates between vectors and scalars. It also introduces quantities that we will be working with during the study of kinematics.

Click to view content\\
How does this video help you understand the difference between distance and displacement? Describe the differences between vectors and scalars using physical quantities as examples.\\
a. It explains that distance is a vector and direction is important, whereas displacement is a scalar and it has no direction attached to it.\\
b. It explains that distance is a scalar and direction is important, whereas displacement is a vector and it has no direction attached to it.\\
c. It explains that distance is a scalar and it has no direction attached to it, whereas displacement is a vector and direction is important.\\
d. It explains that both distance and displacement are scalar and no directions are attached to them.

\section*{Teacher Support}
Teacher Support Define the concepts of vectors and scalars before watching the video.\\[0pt]
[OL][BL] Come up with some examples of vectors and scalars and have the students classify each.\\[0pt]
[AL] Discuss how the concept of direction might be important for the study of motion.

Displacement Problems Hopefully you now understand the conceptual difference between distance and displacement. Understanding concepts is half the battle in physics. The other half is math. A stumbling block to new physics students is trying to wade through the math of physics while also trying to understand the associated concepts. This struggle may lead to misconceptions and answers that make no sense. Once the concept is mastered, the math is far less confusing.

So let's review and see if we can make sense of displacement in terms of numbers and equations. You can calculate an object's displacement by subtracting its original position, \(\mathbf{d}_{\mathbf{0}}\), from its final position \(\mathbf{d}_{\mathbf{f}}\). In math terms that means\\
\(\Delta \mathbf{d}=\mathbf{d}_{\mathrm{f}}-\mathbf{d}_{0}\).\\
If the final position is the same as the initial position, then \(\Delta \mathbf{d}=0\).\\
To assign numbers and/or direction to these quantities, we need to define an axis with a positive and a negative direction. We also need to define an origin, or \(O\). In Figure 2.6, the axis is in a straight line with home at zero and school in the positive direction. If we left home and drove the opposite way from school, motion would have been in the negative direction. We would have assigned it a negative value. In the round-trip drive, \(\mathbf{d}_{\mathrm{f}}\) and \(\mathbf{d}_{0}\) were both at zero kilometers. In the one way trip to school, \(\mathbf{d}_{\mathrm{f}}\) was at 5 kilometers and \(\mathbf{d}_{0}\) was at zero km . So, \(\Delta \mathbf{d}\) was 5 kilometers.

\section*{Tips For Success}
You may place your origin wherever you would like. You have to make sure that you calculate all distances consistently from your zero and you define one direction as positive and the other as negative. Therefore, it makes sense to choose the easiest axis, direction, and zero. In the example above, we took home to be zero because it allowed us to avoid having to interpret a solution with a negative sign.

\section*{Worked Example}
Calculating Distance and Displacement A cyclist rides 3 km west and then turns around and rides 2 km east. (a) What is her displacement? (b) What distance does she ride? (c) What is the magnitude of her displacement?\\
\includegraphics[max width=\textwidth, center]{1bfcec96-a1a9-4e0f-9291-d0b7d1308d1c-16}

\section*{Strategy}
To solve this problem, we need to find the difference between the final position and the initial position while taking care to note the direction on the axis. The final position is the sum of the two displacements, \(\Delta \mathbf{d}_{1}\) and \(\Delta \mathbf{d}_{2}\).

Solution\\
a. Displacement: The rider's displacement is \(\Delta \mathbf{d}=\mathbf{d}_{\mathrm{f}}-\mathbf{d}_{0}=-1 \mathrm{~km}\).\\
b. Distance: The distance traveled is \(3 \mathrm{~km}+2 \mathrm{~km}=5 \mathrm{~km}\).\\
c. The magnitude of the displacement is 1 km .

Discussion\\
The displacement is negative because we chose east to be positive and west to be negative. We could also have described the displacement as 1 km west. When calculating displacement, the direction mattered, but when calculating distance, the direction did not matter. The problem would work the same way if the problem were in the north-south or \(y\)-direction.

\section*{Tips For Success}
Physicists like to use standard units so it is easier to compare notes. The standard units for calculations are called SI units (International System of Units). SI units are based on the metric system. The SI unit for displacement is the meter (m), but sometimes you will see a problem with kilometers, miles, feet, or other units of length. If one unit in a problem is an SI unit and another is not, you will need to convert all of your quantities to the same system before you can carry out the calculation.

\section*{Teacher Support}
Teacher Support Point out to students that the distance for each segment is the absolute value of the displacement along a straight path.

\section*{Practice Problems}
1.

On an axis in which moving from right to left is positive, what is the displacement and distance of a student who walks 32 m to the right and then 17 m to the left?\\
a. Displacement is -15 m and distance is \(-49 m\).\\
b. Displacement is -15 m and distance is 49 m .\\
c. Displacement is 15 m and distance is -49 m .\\
d. Displacement is 15 m and distance is 49 m .\\
2.

Tiana jogs 1.5 km along a straight path and then turns and jogs 2.4 km in the opposite direction. She then turns back and jogs 0.7 km in the original direction. Let Tiana's original direction be the positive direction. What are the displacement and distance she jogged?\\
a. Displacement is 4.6 km , and distance is -0.2 km .\\
b. Displacement is -0.2 km , and distance is 4.6 km .\\
c. Displacement is 4.6 km , and distance is +0.2 km .\\
d. Displacement is +0.2 km , and distance is 4.6 km .

\section*{Work In Physics}
\section*{Mars Probe Explosion}
\begin{figure}[h]
\begin{center}
  \includegraphics[max width=\textwidth]{1bfcec96-a1a9-4e0f-9291-d0b7d1308d1c-18}
\captionsetup{labelformat=empty}
\caption{Figure 2.7 The Mars Climate Orbiter disaster illustrates the importance of using the correct calculations in physics. (NASA)}
\end{center}
\end{figure}

Physicists make calculations all the time, but they do not always get the right answers. In 1998, NASA, the National Aeronautics and Space Administration, launched the Mars Climate Orbiter, shown in Figure 2.7, a \(\$ 125\)-million-dollar satellite designed to monitor the Martian atmosphere. It was supposed to orbit the planet and take readings from a safe distance. The American scientists made\\
calculations in English units (feet, inches, pounds, etc.) and forgot to convert their answers to the standard metric SI units. This was a very costly mistake. Instead of orbiting the planet as planned, the Mars Climate Orbiter ended up flying into the Martian atmosphere. The probe disintegrated. It was one of the biggest embarrassments in NASA's history.\\
3.

In 1999 the Mars Climate Orbiter crashed because calculation were performed in English units instead of SI units. At one point the orbiter was just 187,000 feet above the surface, which was too close to stay in orbit. What was the height of the orbiter at this time in kilometers? (Assume 1 meter equals 3.281 feet.)\\
a. 16 km\\
b. 18 km\\
c. 57 km\\
d. 614 km

\section*{Teacher Support}
Teacher Support The text feature describes a real-life miscalculation made by astronomers at NASA. In this case, the Mars Climate Orbiter's orbit needed to be calculated precisely because its machinery was designed to withstand only a certain amount of atmospheric pressure. The orbiter had to be close enough to the planet to take measurements and far enough away that it could remain structurally sound. One way to teach this concept would be to pick an orbital distance from Mars and have the students calculate the distance of the path and the height from the surface both in SI units and in English units. Ask why failure to convert might be a problem.

\section*{Check Your Understanding}
4.

What does it mean when motion is described as relative?\\
a. It means that motion of any object is described relative to the motion of Earth.\\
b. It means that motion of any object is described relative to the motion of any other object.\\
c. It means that motion is independent of the frame of reference.\\
d. It means that motion depends on the frame of reference selected.\\
5.

If you and a friend are standing side-by-side watching a soccer game, would you both view the motion from the same reference frame?\\
a. Yes, we would both view the motion from the same reference point because both of us are at rest in Earth's frame of reference.\\
b. Yes, we would both view the motion from the same reference point because both of us are observing the motion from two points on the same straight line.\\
c. No, we would both view the motion from different reference points because motion is viewed from two different points; the reference frames are similar but not the same.\\
d. No, we would both view the motion from different reference points because response times may be different; so, the motion observed by both of us would be different.\\
6.

What is the difference between distance and displacement?\\
a. Distance has both magnitude and direction, while displacement has magnitude but no direction.\\
b. Distance has magnitude but no direction, while displacement has both magnitude and direction.\\
c. Distance has magnitude but no direction, while displacement has only direction.\\
d. There is no difference. Both distance and displacement have magnitude and direction.\\
7.

Which statement correctly describes a race car's distance traveled and magnitude of displacement during a one-lap car race around an oval track?\\
a. The perimeter of the race track is the distance; the shortest distance between the start line and the finish line is the magnitude of displacement.\\
b. The perimeter of the race track is the magnitude of displacement; the shortest distance between the start and finish line is the distance.\\
c. The perimeter of the race track is both the distance and magnitude of displacement.\\
d. The shortest distance between the start and the finish line is the magnitude of the displacement vector.\\
8.

Why is it important to specify a reference frame when describing motion?\\
a. Because Earth is continuously in motion; an object at rest on Earth will be in motion when viewed from outer space.\\
b. Because the position of a moving object can be defined only when there is a fixed reference frame.\\
c. Because motion is a relative term; it appears differently when viewed from different reference frames.\\
d. Because motion is always described in Earth's frame of reference; if another frame is used, it has to be specified with each situation.

\section*{Teacher Support}
Teacher Support Use the questions under Check Your Understanding to assess students' achievement of the section's learning objectives. If students are struggling with a specific objective, the formative assessment will help direct students to the relevant content.

\subsection*{2.2 Speed and Velocit}
\section*{Section Learning Objectives}
By the end of this section, you will be able to do the following:

\begin{itemize}
  \item Calculate the average speed of an object
  \item Relate displacement and average velocity
\end{itemize}

\section*{Teacher Support}
Teacher Support The learning objectives in this section will help your students master the following standards:

\begin{itemize}
  \item (4) Science concepts. The student knows and applies the laws governing motion in a variety of situations. The student is expected to:
  \item (B) describe and analyze motion in one dimension using equations with the concepts of distance, displacement, speed, average velocity, instantaneous velocity, and acceleration.
\end{itemize}

In addition, the High School Physics Laboratory Manual addresses content in this section in the lab titled: Position and Speed of an Object, as well as the following standards:

\begin{itemize}
  \item (4) Science concepts. The student knows and applies the laws governing motion in a variety of situations. The student is expected to:
  \item (B) describe and analyze motion in one dimension using equations with the concepts of distance, displacement, speed, average velocity, instantaneous velocity, and acceleration.
\end{itemize}

\section*{Section Key Terms}
\section*{Teacher Support}
Teacher Support In this section, students will apply what they have learned about distance and displacement to the concepts of speed and velocity.\\[0pt]
[BL][OL] Before students read the section, ask them to give examples of ways they have heard the word speed used. Then ask them if they have heard the word velocity used. Explain that these words are often used interchangeably in everyday life, but their scientific definitions are different. Tell students that they will learn about these differences as they read the section.\\[0pt]
[AL] Explain to students that velocity, like displacement, is a vector quantity. Ask them to speculate about ways that speed is different from velocity. After\\
they share their ideas, follow up with questions that deepen their thought process, such as: Why do you think that? What is an example? How might apply these terms to motion that you see every day?

\section*{Speed}
There is more to motion than distance and displacement. Questions such as, "How long does a foot race take?" and "What was the runner's speed?" cannot be answered without an understanding of other concepts. In this section we will look at time, speed, and velocity to expand our understanding of motion.

A description of how fast or slow an object moves is its speed. Speed is the rate at which an object changes its location. Like distance, speed is a scalar because it has a magnitude but not a direction. Because speed is a rate, it depends on the time interval of motion. You can calculate the elapsed time or the change in time, \(\Delta t\), of motion as the difference between the ending time and the beginning time\\
\(\Delta t=t_{\mathrm{f}}-t_{0}\).\\
The SI unit of time is the second ( s ), and the SI unit of speed is meters per second (m/s), but sometimes kilometers per hour (km/h), miles per hour (mph) or other units of speed are used.

When you describe an object's speed, you often describe the average over a time period. Average speed, \(v_{a v g}\), is the distance traveled divided by the time during which the motion occurs.\\
\(v_{\text {avg }}=\frac{\text { distance }}{\text { time }}\)\\
You can, of course, rearrange the equation to solve for either distance or time time \(=\frac{\text { distance }}{v_{\text {avg }}}\).\\
distance \(=v_{\text {avg }} \times\) time\\
Suppose, for example, a car travels 150 kilometers in 3.2 hours. Its average speed for the trip is

\[
\begin{aligned}
v_{a v g} & =\frac{\text { distance }}{\text { time }} \\
& =\frac{150 \mathrm{~km}}{3.2 \mathrm{~h}} \\
& =47 \mathrm{~km} / \mathrm{h}
\end{aligned}
\]

A car's speed would likely increase and decrease many times over a 3.2 hour trip. Its speed at a specific instant in time, however, is its instantaneous speed. A car's speedometer describes its instantaneous speed.

\section*{Teacher Support}
Teacher Support [OL][AL] Caution students that average speed is not al-\\
ways the average of an object's initial and final speeds. For example, suppose a car travels a distance of 100 km . The first 50 km it travels \(30 \mathrm{~km} / \mathrm{h}\) and the second 50 km it travels at \(60 \mathrm{~km} / \mathrm{h}\). Its average speed would be distance /(time interval \()=(100 \mathrm{~km}) /[(50 \mathrm{~km}) /(30 \mathrm{~km} / \mathrm{h})+(50 \mathrm{~km}) /(60 \mathrm{~km} / \mathrm{h})]=40 \mathrm{~km} / \mathrm{h}\). If the car had spent equal times at 30 km and 60 km rather than equal distances at these speeds, its average speed would have been \(45 \mathrm{~km} / \mathrm{h}\).\\[0pt]
[BL][OL] Caution students that the terms speed, average speed, and instantaneous speed are all often referred to simply as speed in everyday language. Emphasize the importance in science to use correct terminology to avoid confusion and to properly communicate ideas.

\begin{figure}[h]
\begin{center}
  \includegraphics[max width=\textwidth]{1bfcec96-a1a9-4e0f-9291-d0b7d1308d1c-24}
\captionsetup{labelformat=empty}
\caption{Figure 2.8 During a 30 -minute round trip to the store, the total distance traveled is 6 km . The average speed is \(12 \mathrm{~km} / \mathrm{h}\). The displacement for the round trip is zero, because there was no net change in position.}
\end{center}
\end{figure}

\section*{Worked Example}
Calculating Average Speed A marble rolls 5.2 m in 1.8 s . What was the marble's average speed?

\section*{Strategy}
We know the distance the marble travels, 5.2 m , and the time interval, 1.8 s . We can use these values in the average speed equation.

Solution\\
\(v_{\text {avg }}=\frac{\text { distance }}{\text { time }}=\frac{5.2 \mathrm{~m}}{1.8 \mathrm{~s}}=2.9 \mathrm{~m} / \mathrm{s}\)\\
Discussion\\
Average speed is a scalar, so we do not include direction in the answer. We can check the reasonableness of the answer by estimating: 5 meters divided by 2\\
seconds is \(2.5 \mathrm{~m} / \mathrm{s}\). Since \(2.5 \mathrm{~m} / \mathrm{s}\) is close to \(2.9 \mathrm{~m} / \mathrm{s}\), the answer is reasonable. This is about the speed of a brisk walk, so it also makes sense.

\section*{Practice Problems}
9.

A pitcher throws a baseball from the pitcher's mound to home plate in 0.46 s . The distance is 18.4 m . What was the average speed of the baseball?\\
a. \(40 \mathrm{~m} / \mathrm{s}\)\\
b. \(-40 \mathrm{~m} / \mathrm{s}\)\\
c. \(0.03 \mathrm{~m} / \mathrm{s}\)\\
d. \(8.5 \mathrm{~m} / \mathrm{s}\)\\
10.

Cassie walked to her friend's house with an average speed of \(1.40 \mathrm{~m} / \mathrm{s}\). The distance between the houses is 205 m . How long did the trip take her?\\
a. 146 s\\
b. 0.01 s\\
c. 2.50 min\\
d. 287 s

Velocity The vector version of speed is velocity. Velocity describes the speed and direction of an object. As with speed, it is useful to describe either the average velocity over a time period or the velocity at a specific moment. Average velocity is displacement divided by the time over which the displacement occurs.\\
\(\mathbf{v}_{\text {avg }}=\frac{\text { displacement }}{\text { time }}=\frac{\Delta \mathbf{d}}{\Delta t}=\frac{\mathbf{d}_{\mathrm{f}}-\mathbf{d}_{0}}{t_{\mathrm{f}}-t_{0}}\)\\
Velocity, like speed, has SI units of meters per second (m/s), but because it is a vector, you must also include a direction. Furthermore, the variable \(\mathbf{v}\) for velocity is bold because it is a vector, which is in contrast to the variable \(v\) for speed which is italicized because it is a scalar quantity.

\section*{Tips For Success}
It is important to keep in mind that the average speed is not the same thing as the average velocity without its direction. Like we saw with displacement and distance in the last section, changes in direction over a time interval have a bigger effect on speed and velocity.

Suppose a passenger moved toward the back of a plane with an average velocity of \(-4 \mathrm{~m} / \mathrm{s}\). We cannot tell from the average velocity whether the passenger stopped momentarily or backed up before he got to the back of the plane. To get more details, we must consider smaller segments of the trip over smaller time intervals such as those shown in Figure 2.9. If you consider infinitesimally small intervals, you can define instantaneous velocity, which is the velocity at\\
a specific instant in time. Instantaneous velocity and average velocity are the same if the velocity is constant.

\begin{figure}[h]
\begin{center}
  \includegraphics[max width=\textwidth]{1bfcec96-a1a9-4e0f-9291-d0b7d1308d1c-26}
\captionsetup{labelformat=empty}
\caption{Figure 2.9 The diagram shows a more detailed record of an airplane passenger heading toward the back of the plane, showing smaller segments of his trip.}
\end{center}
\end{figure}

Earlier, you have read that distance traveled can be different than the magnitude of displacement. In the same way, speed can be different than the magnitude of velocity. For example, you drive to a store and return home in half an hour. If your car's odometer shows the total distance traveled was 6 km , then your average speed was \(12 \mathrm{~km} / \mathrm{h}\). Your average velocity, however, was zero because your displacement for the round trip is zero.

\section*{Watch Physics}
Calculating Average Velocity or Speed This video reviews vectors and scalars and describes how to calculate average velocity and average speed when you know displacement and change in time. The video also reviews how to convert \(\mathrm{km} / \mathrm{h}\) to \(\mathrm{m} / \mathrm{s}\).

Click to view content\\
Which of the following fully describes a vector and a scalar quantity and correctly provides an example of each?\\
a. A scalar quantity is fully described by its magnitude, while a vector needs both magnitude and direction to fully describe it. Displacement is an example of a scalar quantity and time is an example of a vector quantity.\\
b. A scalar quantity is fully described by its magnitude, while a vector needs both magnitude and direction to fully describe it. Time is an example of\\
a scalar quantity and displacement is an example of a vector quantity.\\
c. A scalar quantity is fully described by its magnitude and direction, while a vector needs only magnitude to fully describe it. Displacement is an example of a scalar quantity and time is an example of a vector quantity.\\
d. A scalar quantity is fully described by its magnitude and direction, while a vector needs only magnitude to fully describe it. Time is an example of a scalar quantity and displacement is an example of a vector quantity.

\section*{Teacher Support}
Teacher Support This video does a good job of reinforcing the difference between vectors and scalars. The student is introduced to the idea of using 's' to denote displacement, which you may or may not wish to encourage. Before students watch the video, point out that the instructor uses \(\vec{s}\) for displacement instead of d, as used in this text. Explain the use of small arrows over variables is a common way to denote vectors in higher-level physics courses. Caution students that the customary abbreviations for hour and seconds are not used in this video. Remind students that in their own work they should use the abbreviations h for hour and s for seconds.

\section*{Worked Example}
Calculating Average Velocity A student has a displacement of 304 m north in 180 s . What was the student's average velocity?

\section*{Strategy}
We know that the displacement is 304 m north and the time is 180 s . We can use the formula for average velocity to solve the problem.

Solution\\
\(\mathbf{v}_{\text {avg }}=\frac{\Delta \mathbf{d}}{\Delta t}=\frac{304 \mathrm{~m}}{180 \mathrm{~s}}=1.7 \mathrm{~m} / \mathrm{s}\) north\\
2.1

Discussion\\
Since average velocity is a vector quantity, you must include direction as well as magnitude in the answer. Notice, however, that the direction can be omitted until the end to avoid cluttering the problem. Pay attention to the significant figures in the problem. The distance 304 m has three significant figures, but the time interval 180 s has only two, so the quotient should have only two significant figures.

\section*{Tips For Success}
Note the way scalars and vectors are represented. In this book d represents distance and displacement. Similarly, v represents speed, and v represents velocity. A variable that is not bold indicates a scalar quantity, and a bold variable indicates a vector quantity. Vectors are sometimes represented by small arrows above the variable.

\section*{Teacher Support}
Teacher Support Use this problem to emphasize the importance of using the correct number of significant figures in calculations. Some students have a tendency to include many digits in their final calculations. They incorrectly believe they are improving the accuracy of their answer by writing many of the digits shown on the calculator. Point out that doing this introduces errors into the calculations. In more complicated calculations, these errors can propagate and cause the final answer to be wrong. Instead, remind students to always carry one or two extra digits in intermediate calculations and to round the final answer to the correct number of significant figures.

\section*{Worked Example}
Solving for Displacement when Average Velocity and Time are Known Layla jogs with an average velocity of \(2.4 \mathrm{~m} / \mathrm{s}\) east. What is her displacement after 46 seconds?

\section*{Strategy}
We know that Layla's average velocity is \(2.4 \mathrm{~m} / \mathrm{s}\) east, and the time interval is 46 seconds. We can rearrange the average velocity formula to solve for the displacement.

Solution

\[
\begin{aligned}
\mathbf{v}_{\text {avg }} & =\frac{\Delta \mathbf{d}}{\Delta t} \\
\Delta \mathbf{d} & =\mathbf{v}_{\text {avg }} \Delta t \\
& =(2.4 \mathrm{~m} / \mathrm{s})(46 \mathrm{~s}) \\
& =1.1 \times 10^{2} \mathrm{~m} \text { east }
\end{aligned}
\]

\section*{2.2}
Discussion\\
The answer is about 110 m east, which is a reasonable displacement for slightly less than a minute of jogging. A calculator shows the answer as 110.4 m . We\\
chose to write the answer using scientific notation because we wanted to make it clear that we only used two significant figures.

\section*{Tips For Success}
Dimensional analysis is a good way to determine whether you solved a problem correctly. Write the calculation using only units to be sure they match on opposite sides of the equal mark. In the worked example, you have\\
\(\mathrm{m}=(\mathrm{m} / \mathrm{s})(\mathrm{s})\). Since seconds is in the denominator for the average velocity and in the numerator for the time, the unit cancels out leaving only m and, of course, \(\mathrm{m}=\mathrm{m}\).

\section*{Worked Example}
Solving for Time when Displacement and Average Velocity are Known Phillip walks along a straight path from his house to his school. How long will it take him to get to school if he walks 428 m west with an average velocity of \(1.7 \mathrm{~m} / \mathrm{s}\) west?

\section*{Strategy}
We know that Phillip's displacement is 428 m west, and his average velocity is \(1.7 \mathrm{~m} / \mathrm{s}\) west. We can calculate the time required for the trip by rearranging the average velocity equation.

Solution

\[
\begin{aligned}
\mathbf{v}_{\mathrm{avg}} & =\frac{\Delta \mathbf{d}}{\Delta t} \\
\Delta t & =\frac{\Delta \mathbf{d}}{\mathbf{v}_{\mathrm{avg}}} \\
& =\frac{428 \mathrm{~m}}{1.7 \mathrm{~m} / \mathrm{s}} \\
& =2.5 \times 10^{2} \mathrm{~s}
\end{aligned}
\]

2.3

Discussion\\
Here again we had to use scientific notation because the answer could only have two significant figures. Since time is a scalar, the answer includes only a magnitude and not a direction.

\section*{Practice Problems}
11.

A trucker drives along a straight highway for 0.25 h with a displacement of 16 km south. What is the trucker's average velocity?\\
a. \(4 \mathrm{~km} / \mathrm{h}\) north\\
b. \(4 \mathrm{~km} / \mathrm{h}\) south\\
c. \(64 \mathrm{~km} / \mathrm{h}\) north\\
d. \(64 \mathrm{~km} / \mathrm{h}\) south\\
12.

A bird flies with an average velocity of \(7.5 \mathrm{~m} / \mathrm{s}\) east from one branch to another in 2.4 s . It then pauses before flying with an average velocity of \(6.8 \mathrm{~m} / \mathrm{s}\) east for 3.5 s to another branch. What is the bird's total displacement from its starting point?\\
a. 42 m west\\
b. 6 m west\\
c. 6 m east\\
d. 42 m east

\section*{Virtual Physics}
The Walking Man In this simulation you will put your cursor on the man and move him first in one direction and then in the opposite direction. Keep the Introduction tab active. You can use the Charts tab after you learn about graphing motion later in this chapter. Carefully watch the sign of the numbers in the position and velocity boxes. Ignore the acceleration box for now. See if you can make the man's position positive while the velocity is negative. Then see if you can do the opposite.

Click to view content

\section*{Grasp Check}
Which situation correctly describes when the moving man's position was negative but his velocity was positive?\\
a. Man moving toward 0 from left of 0\\
b. Man moving toward 0 from right of 0\\
c. Man moving away from 0 from left of 0\\
d. Man moving away from 0 from right of 0

\section*{Teacher Support}
Teacher Support This is a powerful interactive animation, and it can be used for many lessons. At this point it can be used to show that displacement can be either positive or negative. It can also show that when displacement is negative, velocity can be either positive or negative. Later it can be used to show that velocity and acceleration can have different signs. It is strongly suggested that you keep students on the Introduction tab. The Charts tab can be used after students learn about graphing motion later in this chapter.

\section*{Check Your Understanding}
13.

Two runners traveling along the same straight path start and end their run at the same time. At the halfway mark, they have different instantaneous velocities. Is it possible for their average velocities for the entire trip to be the same?\\
a. Yes, because average velocity depends on the net or total displacement.\\
b. Yes, because average velocity depends on the total distance traveled.\\
c. No, because the velocities of both runners must remain exactly the same throughout the journey.\\
d. No, because the instantaneous velocities of the runners must remain the same at the midpoint but can vary at other points.\\
14.

If you divide the total distance traveled on a car trip (as determined by the odometer) by the time for the trip, are you calculating the average speed or the magnitude of the average velocity, and under what circumstances are these two quantities the same?\\
a. Average speed. Both are the same when the car is traveling at a constant speed and changing direction.\\
b. Average speed. Both are the same when the speed is constant and the car does not change its direction.\\
c. Magnitude of average velocity. Both are same when the car is traveling at a constant speed.\\
d. Magnitude of average velocity. Both are same when the car does not change its direction.\\
15.

Is it possible for average velocity to be negative?\\
a. Yes, if net displacement is negative.\\
b. Yes, if the object's direction changes during motion.\\
c. No, because average velocity describes only the magnitude and not the direction of motion.\\
d. No, because average velocity only describes the magnitude in the positive direction of motion.

\section*{Teacher Support}
Teacher Support Use the Check Your Understanding questions to assess students' achievement of the sections learning objectives. If students are struggling with a specific objective, the Check Your Understanding will help identify which and direct students to the relevant content. Assessment items in TUTOR will allow you to reassess.

\subsection*{2.3 Position vs. Time Graphs}
\section*{Section Learning Objectives}
By the end of this section, you will be able to do the following:

\begin{itemize}
  \item Explain the meaning of slope in position vs. time graphs
  \item Solve problems using position vs. time graphs
\end{itemize}

\section*{Teacher Support}
Teacher Support The learning objectives in this section will help your students master the following standards:

\begin{itemize}
  \item (4) Science concepts. The student knows and applies the laws governing motion in a variety of situations. The student is expected to:
  \item (A) generate and interpret graphs and charts describing different types of motion, including the use of real-time technology such as motion detectors or photogates.
\end{itemize}

\section*{Section Key Terms}
\section*{Teacher Support}
Teacher Support [BL][OL] Describe a scenario, for example, in which you launch a water rocket into the air. It goes up 150 ft , stops, and then falls back to the earth. Have the students assess the situation. Where would they put their zero? What is the positive direction, and what is the negative direction? Have a student draw a picture of the scenario on the board. Then draw a position vs. time graph describing the motion. Have students help you complete the graph. Is the line straight? Is it curved? Does it change direction? What can they tell by looking at the graph?\\[0pt]
[AL] Once the students have looked at and analyzed the graph, see if they can describe different scenarios in which the lines would be straight instead of curved? Where the lines would be discontinuous?

\section*{Graphing Position as a Function of Time}
A graph, like a picture, is worth a thousand words. Graphs not only contain numerical information, they also reveal relationships between physical quantities. In this section, we will investigate kinematics by analyzing graphs of position over time.

Graphs in this text have perpendicular axes, one horizontal and the other vertical. When two physical quantities are plotted against each other, the horizontal axis is usually considered the independent variable, and the vertical axis is the dependent variable. In algebra, you would have referred to the horizontal axis as the \(x\)-axis and the vertical axis as the \(y\)-axis. As in Figure 2.10, a straight-line graph has the general form \(y=m x+b\).

Here \(m\) is the slope, defined as the rise divided by the run (as seen in the figure) of the straight line. The letter \(b\) is the \(y\)-intercept which is the point at which the line crosses the vertical, \(y\)-axis. In terms of a physical situation in the real world, these quantities will take on a specific significance, as we will see below. (Figure 2.10.)

\begin{figure}[h]
\begin{center}
  \includegraphics[max width=\textwidth]{1bfcec96-a1a9-4e0f-9291-d0b7d1308d1c-33}
\captionsetup{labelformat=empty}
\caption{Figure 2.10 The diagram shows a straight-line graph. The equation for the straight line is \(y\) equals \(m x+b\).}
\end{center}
\end{figure}

In physics, time is usually the independent variable. Other quantities, such as displacement, are said to depend upon it. A graph of position versus time, therefore, would have position on the vertical axis (dependent variable) and time on the horizontal axis (independent variable). In this case, to what would the slope and \(y\)-intercept refer? Let's look back at our original example when studying distance and displacement.

The drive to school was 5 km from home. Let's assume it took 10 minutes to make the drive and that your parent was driving at a constant velocity the whole time. The position versus time graph for this section of the trip would look like that shown in Figure 2.11.

\begin{figure}[h]
\begin{center}
  \includegraphics[max width=\textwidth]{1bfcec96-a1a9-4e0f-9291-d0b7d1308d1c-34}
\captionsetup{labelformat=empty}
\caption{Figure 2.11 A graph of position versus time for the drive to school is shown. What would the graph look like if we added the return trip?}
\end{center}
\end{figure}

As we said before, \(\mathbf{d}_{0}=0\) because we call home our \(O\) and start calculating from there. In Figure 2.11, the line starts at \(\mathbf{d}=0\), as well. This is the \(b\) in our equation for a straight line. Our initial position in a position versus time graph is always the place where the graph crosses the \(x\)-axis at \(t=0\). What is the slope? The rise is the change in position, (i.e., displacement) and the run is the change in time. This relationship can also be written\\
\(\frac{\Delta \mathrm{d}}{\Delta t}\).\\
2.4

This relationship was how we defined average velocity. Therefore, the slope in a \(\mathbf{d}\) versus \(t\) graph, is the average velocity.

\section*{Tips For Success}
Sometimes, as is the case where we graph both the trip to school and the return trip, the behavior of the graph looks different during different time intervals. If the graph looks like a series of straight lines, then you can calculate the average velocity for each time interval by looking at the slope. If you then want to calculate the average velocity for the entire trip, you can do a weighted average.

Let's look at another example. Figure 2.12 shows a graph of position versus time for a jet-powered car on a very flat dry lake bed in Nevada.

\begin{figure}[h]
\begin{center}
  \includegraphics[max width=\textwidth]{1bfcec96-a1a9-4e0f-9291-d0b7d1308d1c-35}
\captionsetup{labelformat=empty}
\caption{Figure 2.12 The diagram shows a graph of position versus time for a jet-powered car on the Bonneville Salt Flats.}
\end{center}
\end{figure}

Using the relationship between dependent and independent variables, we see that the slope in the graph in Figure 2.12 is average velocity, \(\mathbf{v}_{\text {avg }}\) and the intercept is displacement at time zero-that is, \(\mathbf{d}_{0}\). Substituting these symbols into \(y=m x+b\) gives\\
\(\mathbf{d}=\mathbf{v} t+\mathbf{d}_{0}\)

\section*{2.5}
or\\
\(\mathbf{d}=\mathbf{d}_{0}+\mathbf{v} t\).

\section*{2.6}
Thus a graph of position versus time gives a general relationship among displacement, velocity, and time, as well as giving detailed numerical information about a specific situation. From the figure we can see that the car has a position of 400 m at \(t=0 \mathrm{~s}, 650 \mathrm{~m}\) at \(t=1.0 \mathrm{~s}\), and so on. And we can learn about the object's velocity, as well.

\section*{Teacher Support}
\section*{Teacher Support}
\section*{Teacher Demonstration}
Help students learn what different graphs of displacement vs. time look like.\\[0pt]
[Visual] Set up a meter stick.

\begin{enumerate}
  \item If you can find a remote control car, have one student record times as you send the car forward along the stick, then backwards, then forward again with a constant velocity.
  \item Take the recorded times and the change in position and put them together.
  \item Get the students to coach you to draw a position vs. time graph.
\end{enumerate}

Each leg of the journey should be a straight line with a different slope. The parts where the car was going forward should have a positive slope. The part where it is going backwards would have a negative slope.\\[0pt]
[OL] Ask if the place that they take as zero affects the graph.\\[0pt]
[AL] Is it realistic to draw any position graph that starts at rest without some curve in it? Why might we be able to neglect the curve in some scenarios?\\[0pt]
[All] Discuss what can be uncovered from this graph. Students should be able to read the net displacement, but they can also use the graph to determine the total distance traveled. Then ask how the speed or velocity is reflected in this graph. Direct students in seeing that the steepness of the line (slope) is a measure of the speed and that the direction of the slope is the direction of the motion.\\[0pt]
[AL] Some students might recognize that a curve in the line represents a sort of slope of the slope, a preview of acceleration which they will learn about in the next chapter.

\section*{Snap Lab}
Graphing Motion In this activity, you will release a ball down a ramp and graph the ball's displacement vs. time.

\begin{itemize}
  \item Choose an open location with lots of space to spread out so there is less chance for tripping or falling due to rolling balls.
  \item 1 ball
  \item 1 board
  \item 2 or 3 books
  \item 1 stopwatch
  \item 1 tape measure
  \item 6 pieces of masking tape
  \item 1 piece of graph paper
  \item 1 pencil
\end{itemize}

Procedure

\begin{enumerate}
  \item Build a ramp by placing one end of the board on top of the stack of books. Adjust location, as necessary, until there is no obstacle along the straight line path from the bottom of the ramp until at least the next 3 m .
  \item Mark distances of \(0.5 \mathrm{~m}, 1.0 \mathrm{~m}, 1.5 \mathrm{~m}, 2.0 \mathrm{~m}, 2.5 \mathrm{~m}\), and 3.0 m from the bottom of the ramp. Write the distances on the tape.
  \item Have one person take the role of the experimenter. This person will release the ball from the top of the ramp. If the ball does not reach the 3.0 m mark, then increase the incline of the ramp by adding another book. Repeat this Step as necessary.
  \item Have the experimenter release the ball. Have a second person, the timer, begin timing the trial once the ball reaches the bottom of the ramp and stop the timing once the ball reaches 0.5 m . Have a third person, the recorder, record the time in a data table.
  \item Repeat Step 4, stopping the times at the distances of \(1.0 \mathrm{~m}, 1.5 \mathrm{~m}, 2.0 \mathrm{~m}\), 2.5 m , and 3.0 m from the bottom of the ramp.
  \item Use your measurements of time and the displacement to make a position vs. time graph of the ball's motion.
  \item Repeat Steps 4 through 6, with different people taking on the roles of experimenter, timer, and recorder. Do you get the same measurement values regardless of who releases the ball, measures the time, or records the result? Discuss possible causes of discrepancies, if any.
\end{enumerate}

True or False: The average speed of the ball will be less than the average velocity of the ball.\\
a. True\\
b. False

\section*{Teacher Support}
Teacher Support [BL][OL] Emphasize that the motion in this lab is the motion of the ball as it rolls along the floor. Ask students where there zero should be.\\[0pt]
[AL] Ask students what the graph would look like if they began timing at the top versus the bottom of the ramp. Why would the graph look different? What might account for the difference?\\[0pt]
[BL][OL] Have the students compare the graphs made with different individuals taking on different roles. Ask them to determine and compare average speeds for each interval. What were the absolute differences in speeds, and what were the percent differences? Do the differences appear to be random, or are there systematic differences? Why might there be systematic differences between the two sets of measurements with different individuals in each role?\\[0pt]
[BL][OL] Have the students compare the graphs made with different individuals taking on different roles. Ask them to determine and compare average speeds for each interval. What were the absolute differences in speeds, and what were the percent differences? Do the differences appear to be random, or are there\\
systematic differences? Why might there be systematic differences between the two sets of measurements with different individuals in each role?

Solving Problems Using Position vs. Time Graphs So how do we use graphs to solve for things we want to know like velocity?

\section*{Worked Example}
Using Position-Time Graph to Calculate Average Velocity: Jet Car Find the average velocity of the car whose position is graphed in Figure 1.13.

\section*{Strategy}
The slope of a graph of \(d\) vs. \(t\) is average velocity, since slope equals rise over run.\\
slope \(=\frac{\Delta \mathbf{d}}{\Delta t}=\mathbf{v}\)\\
2.7

Since the slope is constant here, any two points on the graph can be used to find the slope.

Solution

\begin{enumerate}
  \item Choose two points on the line. In this case, we choose the points labeled on the graph: \((6.4 \mathrm{~s}, 2000 \mathrm{~m})\) and \((0.50 \mathrm{~s}, 525 \mathrm{~m})\). (Note, however, that you could choose any two points.)
  \item Substitute the \(\mathbf{d}\) and \(t\) values of the chosen points into the equation. Remember in calculating change ( \(\Delta\) ) we always use final value minus initial value.
\end{enumerate}

\[
\begin{aligned}
\mathbf{v} & =\frac{\Delta \mathbf{d}}{\Delta t} \\
& =\frac{2000 \mathrm{~m}-525 \mathrm{~m}}{6.4 \mathrm{~s}-0.50 \mathrm{~s}}, \\
& =250 \mathrm{~m} / \mathrm{s} \\
2.8 &
\end{aligned}
\]

Discussion\\
This is an impressively high land speed ( \(900 \mathrm{~km} / \mathrm{h}\), or about \(560 \mathrm{mi} / \mathrm{h}\) ): much greater than the typical highway speed limit of \(27 \mathrm{~m} / \mathrm{s}\) or \(96 \mathrm{~km} / \mathrm{h}\), but considerably shy of the record of \(343 \mathrm{~m} / \mathrm{s}\) or \(1,234 \mathrm{~km} / \mathrm{h}\), set in 1997.

\section*{Teacher Support}
Teacher Support If the graph of position is a straight line, then the only thing students need to know to calculate the average velocity is the slope of the line, rise/run. They can use whichever points on the line are most convenient.

But what if the graph of the position is more complicated than a straight line? What if the object speeds up or turns around and goes backward? Can we figure out anything about its velocity from a graph of that kind of motion? Let's take another look at the jet-powered car. The graph in Figure 2.13 shows its motion as it is getting up to speed after starting at rest. Time starts at zero for this motion (as if measured with a stopwatch), and the displacement and velocity are initially 200 m and \(15 \mathrm{~m} / \mathrm{s}\), respectively.

\begin{figure}[h]
\begin{center}
  \includegraphics[max width=\textwidth]{1bfcec96-a1a9-4e0f-9291-d0b7d1308d1c-39}
\captionsetup{labelformat=empty}
\caption{Figure 2.13 The diagram shows a graph of the position of a jet-powered car during the time span when it is speeding up. The slope of a distance versus time graph is velocity. This is shown at two points. Instantaneous velocity at any point is the slope of the tangent at that point.}
\end{center}
\end{figure}

\begin{figure}[h]
\begin{center}
  \includegraphics[max width=\textwidth]{1bfcec96-a1a9-4e0f-9291-d0b7d1308d1c-39(1)}
\captionsetup{labelformat=empty}
\caption{Figure 2.14 A U.S. Air Force jet car speeds down a track. (Matt Trostle, Flickr)}
\end{center}
\end{figure}

The graph of position versus time in Figure 2.13 is a curve rather than a straight\\
line. The slope of the curve becomes steeper as time progresses, showing that the velocity is increasing over time. The slope at any point on a position-versustime graph is the instantaneous velocity at that point. It is found by drawing a straight line tangent to the curve at the point of interest and taking the slope of this straight line. Tangent lines are shown for two points in Figure 2.13. The average velocity is the net displacement divided by the time traveled.

\section*{Worked Example}
Using Position-Time Graph to Calculate Average Velocity: Jet Car, Take Two Calculate the instantaneous velocity of the jet car at a time of 25 s by finding the slope of the tangent line at point Q in Figure 2.13.

\section*{Strategy}
The slope of a curve at a point is equal to the slope of a straight line tangent to the curve at that point.

Solution

\begin{enumerate}
  \item Find the tangent line to the curve at \(t=25 \mathrm{~s}\).
  \item Determine the endpoints of the tangent. These correspond to a position of \(1,300 \mathrm{~m}\) at time 19 s and a position of 3120 m at time 32 s .
  \item Plug these endpoints into the equation to solve for the slope, \(\mathbf{v}\).
\end{enumerate}

\[
\begin{aligned}
\text { slope } & =v_{Q}=\frac{\Delta d_{Q}}{\Delta t_{Q}} \\
& =\frac{(3120-1300) \mathrm{m}}{(32-19) \mathrm{s}} \\
& =\frac{1820 \mathrm{~m}}{13 \mathrm{~s}} \\
& =140 \mathrm{~m} / \mathrm{s} \\
2.9 &
\end{aligned}
\]

Discussion\\
The entire graph of \(\mathbf{v}\) versus \(t\) can be obtained in this fashion.

\section*{Teacher Support}
Teacher Support A curved line is a more complicated example. Define tangent as a line that touches a curve at only one point. Show that as a straight line changes its angle next to a curve, it actually hits the curve multiple times at the base, but only one line will never touch at all. This line forms a right angle to the radius of curvature, but at this level, they can just kind of eyeball it. The slope of this line gives the instantaneous velocity. The most useful part of this line is that students can tell when the velocity is increasing, decreasing, positive, negative, and zero.\\[0pt]
[AL] You could find the instantaneous velocity at each point along the graph and if you graphed each of those points, you would have a graph of the velocity.

\section*{Practice Problems}
16.

Calculate the average velocity of the object shown in the graph below over the whole time interval.\\
\includegraphics[max width=\textwidth, center]{1bfcec96-a1a9-4e0f-9291-d0b7d1308d1c-41}\\
a. \(0.25 \mathrm{~m} / \mathrm{s}\)\\
b. \(0.31 \mathrm{~m} / \mathrm{s}\)\\
c. \(3.2 \mathrm{~m} / \mathrm{s}\)\\
d. \(4.00 \mathrm{~m} / \mathrm{s}\)\\
17.

True or False: By taking the slope of the curve in the graph you can verify that the velocity of the jet car is \(125 \backslash, \backslash \operatorname{text}\{\mathrm{~m} / \mathrm{s}\}\) at \(\mathrm{t}=20 \backslash, \backslash \operatorname{text}\{\mathrm{~s}\}\).\\
\includegraphics[max width=\textwidth, center]{1bfcec96-a1a9-4e0f-9291-d0b7d1308d1c-41(1)}\\
a. True\\
b. False

\section*{Check Your Understanding}
18.

Which of the following information about motion can be determined by looking at a position vs. time graph that is a straight line?\\
a. frame of reference\\
b. average acceleration\\
c. velocity\\
d. direction of force applied\\
19.

True or False: The position vs time graph of an object that is speeding up is a straight line.\\
a. True\\
b. False

\section*{Teacher Support}
Teacher Support Use the Check Your Understanding questions to assess students' achievement of the section's learning objectives. If students are struggling with a specific objective, the Check Your Understanding will help identify direct students to the relevant content.

\subsection*{2.4 Velocit vs. Time Graphs}
\section*{Section Learning Objectives}
By the end of this section, you will be able to do the following:

\begin{itemize}
  \item Explain the meaning of slope and area in velocity vs. time graphs
  \item Solve problems using velocity vs. time graphs
\end{itemize}

\section*{Teacher Support}
Teacher Support The learning objectives in this section will help your students master the following standards:

\begin{itemize}
  \item (4) Science concepts. The student knows and applies the laws governing motion in a variety of situations. The student is expected to:
  \item (A) generate and interpret graphs and charts describing different types of motion, including the use of real-time technology such as motion detectors or photogates.
\end{itemize}

\section*{Section Key Terms}
\section*{Teacher Support}
Teacher Support Ask students to use their knowledge of position graphs to construct velocity vs. time graphs. Alternatively, provide an example of a velocity vs. time graph and ask students what information can be derived from the graph. Ask-Is it the same information as in a position vs. time graph? How is the information portrayed differently? Is there any new information in a velocity vs. time graph?

\section*{Graphing Velocity as a Function of Time}
Earlier, we examined graphs of position versus time. Now, we are going to build on that information as we look at graphs of velocity vs. time. Velocity is the rate of change of displacement. Acceleration is the rate of change of velocity; we will discuss acceleration more in another chapter. These concepts are all very interrelated.

\section*{Virtual Physics}
Maze Game In this simulation you will use a vector diagram to manipulate a ball into a certain location without hitting a wall. You can manipulate the ball directly with position or by changing its velocity. Explore how these factors\\
change the motion. If you would like, you can put it on the \(a\) setting, as well. This is acceleration, which measures the rate of change of velocity. We will explore acceleration in more detail later, but it might be interesting to take a look at it here.

Click to view content\\
If a person takes 3 steps and ends up in the exact same place as their starting point, what must be true?\\
a. The three steps must have equal displacement\\
b. The displacement of the third step is larger than the displacement of the first two.\\
c. The average velocity must add up to zero.\\
d. The distance and average velocity must add up to zero.

What can we learn about motion by looking at velocity vs. time graphs? Let's return to our drive to school, and look at a graph of position versus time as shown in Figure 2.15.

\begin{figure}[h]
\begin{center}
  \includegraphics[max width=\textwidth]{1bfcec96-a1a9-4e0f-9291-d0b7d1308d1c-44}
\captionsetup{labelformat=empty}
\caption{Figure 2.15 A graph of position versus time for the drive to and from school is shown.}
\end{center}
\end{figure}

We assumed for our original calculation that your parent drove with a constant velocity to and from school. We now know that the car could not have gone from rest to a constant velocity without speeding up. So the actual graph would be curved on either end, but let's make the same approximation as we did then, anyway.

\section*{Tips For Success}
It is common in physics, especially at the early learning stages, for certain things to be neglected, as we see here. This is because it makes the concept clearer or the calculation easier. Practicing physicists use these kinds of short-cuts, as well.

It works out because usually the thing being neglected is small enough that it does not significantly affect the answer. In the earlier example, the amount of time it takes the car to speed up and reach its cruising velocity is very small compared to the total time traveled.

Looking at this graph, and given what we learned, we can see that there are two distinct periods to the car's motion-the way to school and the way back. The average velocity for the drive to school is \(0.5 \mathrm{~km} /\) minute . We can see that the average velocity for the drive back is \(-0.5 \mathrm{~km} /\) minute . If we plot the data showing velocity versus time, we get another graph (Figure 2.16):

\begin{figure}[h]
\begin{center}
  \includegraphics[max width=\textwidth]{1bfcec96-a1a9-4e0f-9291-d0b7d1308d1c-45}
\captionsetup{labelformat=empty}
\caption{Figure 2.16 Graph of velocity versus time for the drive to and from school.}
\end{center}
\end{figure}

We can learn a few things. First, we can derive a \(\mathbf{v}\) versus \(t\) graph from a \(\mathbf{d}\) versus \(t\) graph. Second, if we have a straight-line position-time graph that is positively or negatively sloped, it will yield a horizontal velocity graph. There are a few other interesting things to note. Just as we could use a position vs. time graph to determine velocity, we can use a velocity vs. time graph to determine position. We know that \(\mathbf{v}=\mathbf{d} / t\). If we use a little algebra to re-arrange the equation, we see that \(\mathbf{d}=\mathbf{v} \times t\). In Figure 2.16, we have velocity on the \(y\)-axis and time along the \(x\)-axis. Let's take just the first half of the motion. We get \(0.5 \mathrm{~km} /\) minute \(\times 10\) minutes. The units for minutes cancel each other, and we get 5 km , which is the displacement for the trip to school. If we calculate the same for the return trip, we get -5 km . If we add them together, we see that the net displacement for the whole trip is 0 km , which it should be because we started and ended at the same place.

\section*{Tips For Success}
You can treat units just like you treat numbers, so a \(\mathrm{km} / \mathrm{km}=1\) (or, we say, it cancels out). This is good because it can tell us whether or not we have\\
calculated everything with the correct units. For instance, if we end up with m \(\times \mathrm{s}\) for velocity instead of \(\mathrm{m} / \mathrm{s}\), we know that something has gone wrong, and we need to check our math. This process is called dimensional analysis, and it is one of the best ways to check if your math makes sense in physics.

The area under a velocity curve represents the displacement. The velocity curve also tells us whether the car is speeding up. In our earlier example, we stated that the velocity was constant. So, the car is not speeding up. Graphically, you can see that the slope of these two lines is 0 . This slope tells us that the car is not speeding up, or accelerating. We will do more with this information in a later chapter. For now, just remember that the area under the graph and the slope are the two important parts of the graph. Just like we could define a linear equation for the motion in a position vs. time graph, we can also define one for a velocity vs. time graph. As we said, the slope equals the acceleration, a. And in this graph, the \(y\)-intercept is \(\mathbf{v}_{0}\). Thus, \(\mathbf{v}=\mathbf{v}_{0}+\mathbf{a} t\).

But what if the velocity is not constant? Let's look back at our jet-car example. At the beginning of the motion, as the car is speeding up, we saw that its position is a curve, as shown in Figure 2.17.

\begin{figure}[h]
\begin{center}
  \includegraphics[max width=\textwidth]{1bfcec96-a1a9-4e0f-9291-d0b7d1308d1c-46}
\captionsetup{labelformat=empty}
\caption{Figure 2.17 A graph is shown of the position of a jet-powered car during the time span when it is speeding up. The slope of a d vs. t graph is velocity. This is shown at two points. Instantaneous velocity at any point is the slope of the tangent at that point.}
\end{center}
\end{figure}

You do not have to do this, but you could, theoretically, take the instantaneous velocity at each point on this graph. If you did, you would get Figure 2.18, which is just a straight line with a positive slope.

\begin{figure}[h]
\begin{center}
  \includegraphics[max width=\textwidth]{1bfcec96-a1a9-4e0f-9291-d0b7d1308d1c-47}
\captionsetup{labelformat=empty}
\caption{Figure 2.18 The graph shows the velocity of a jet-powered car during the time span when it is speeding up.}
\end{center}
\end{figure}

Again, if we take the slope of the velocity vs. time graph, we get the acceleration, the rate of change of the velocity. And, if we take the area under the slope, we get back to the displacement.

\section*{Teacher Support}
\section*{Teacher Support}
\section*{Teacher Demonstration}
Return to the scenario of the drive to and from school. Re-draw the V-shaped position graph. Ask the students what the velocity is at different times on that graph. Students should then be able to see that the corresponding velocity graph is a horizontal line at \(0.5 \mathrm{~km} /\) minute and then a horizontal line at \(-0.5 \mathrm{~km} /\) minute. Then draw a few velocity graphs and see if they can get the corresponding position graph.\\[0pt]
[OL][AL] Have students describe the relationship between the velocity and the position on these graphs. Ask-Can a velocity graph be used to find the position? Can a velocity graph be used to find anything else?\\[0pt]
[AL] What is wrong with this graph? Ask students whether the velocity could actually be constant from rest or shift to negative so quickly. What would more realistic graphs look like? How inaccurate is it to ignore the non-constant portion of the motion?\\[0pt]
[OL] Students should be able to see that if a position graph is a straight line, then the velocity graph will be a horizontal line. Also, the instantaneous velocity can be read off the velocity graph at any moment, but more steps are needed to calculate the average velocity.\\[0pt]
[AL] Guide students in seeing that the area under the velocity curve is actually the position and the slope represents the rate of change of the velocity, just as the slope of the position line represents the rate of change of the position.

\section*{Solving Problems using Velocity-Time Graphs}
Most velocity vs. time graphs will be straight lines. When this is the case, our calculations are fairly simple.

\section*{Worked Example}
Using Velocity Graph to Calculate Some Stuff: Jet Car Use this figure to (a) find the displacement of the jet car over the time shown (b) calculate the rate of change (acceleration) of the velocity. (c) give the instantaneous velocity at 5 s , and (d) calculate the average velocity over the interval shown.

\section*{Strategy}
a. The displacement is given by finding the area under the line in the velocity vs. time graph.\\
b. The acceleration is given by finding the slope of the velocity graph.\\
c. The instantaneous velocity can just be read off of the graph.\\
d. To find the average velocity, recall that \(\mathbf{v}_{\text {avg }}=\frac{\Delta \mathbf{d}}{\Delta t}=\frac{\mathbf{d}_{\mathrm{f}}-\mathbf{d}_{0}}{t_{\mathrm{f}}-t_{0}}\)

Solution\\
a. 1. Analyze the shape of the area to be calculated. In this case, the area is made up of a rectangle between 0 and \(20 \mathrm{~m} / \mathrm{s}\) stretching to 30 s . The area of a rectangle is length × width. Therefore, the area of this piece is 600 m .\\
2. Above that is a triangle whose base is 30 s and height is \(140 \mathrm{~m} / \mathrm{s}\). The area of a triangle is \(0.5 \times\) length × width. The area of this piece, therefore, is \(2,100 \mathrm{~m}\).\\
3. Add them together to get a net displacement of \(2,700 \mathrm{~m}\).\\
b. 1. Take two points on the velocity line. Say, \(t=5 \mathrm{~s}\) and \(t=25 \mathrm{~s}\). At \(t =5 \mathrm{~s}\), the value of \(\mathbf{v}=40 \mathrm{~m} / \mathrm{s}\).\\
At \(t=25 \mathrm{~s}, \mathbf{v}=140 \mathrm{~m} / \mathrm{s}\).

\[
\begin{aligned}
\mathbf{a} & =\frac{\Delta \mathbf{v}}{\Delta t} \\
& =\frac{100 \mathrm{~m} / \mathrm{s}}{20 \mathrm{~s}} \\
& =5 \mathrm{~m} / \mathrm{s}^{2}
\end{aligned}
\]

c. The instantaneous velocity at \(t=5 \mathrm{~s}\), as we found in part (b) is just 40 \(\mathrm{m} / \mathrm{s}\).\\
d. 1. Find the net displacement, which we found in part (a) was \(2,700 \mathrm{~m}\).\\
2. Find the total time which for this case is 30 s .\\
3. Divide \(2,700 \mathrm{~m} / 30 \mathrm{~s}=90 \mathrm{~m} / \mathrm{s}\).

Discussion\\
The average velocity we calculated here makes sense if we look at the graph. \(100 \mathrm{~m} / \mathrm{s}\) falls about halfway across the graph and since it is a straight line, we would expect about half the velocity to be above and half below.

\section*{Teacher Support}
Teacher Support The quantities solved for are slightly different in the different kinds of graphs, but students should begin to see that the process of analyzing or breaking down any of these graphs is similar. Ask-Where are the turning points in the motion? When is the object moving forward? What does a curve in the graph mean? Also, students should start to have an intuitive understanding of the relationship between position and velocity graphs.

\section*{Tips For Success}
You can have negative position, velocity, and acceleration on a graph that describes the way the object is moving. You should never see a graph with negative time on an axis. Why?

Most of the velocity vs. time graphs we will look at will be simple to interpret. Occasionally, we will look at curved graphs of velocity vs. time. More often, these curved graphs occur when something is speeding up, often from rest. Let's look back at a more realistic velocity vs. time graph of the jet car's motion that takes this speeding up stage into account.

\begin{figure}[h]
\begin{center}
  \includegraphics[max width=\textwidth]{1bfcec96-a1a9-4e0f-9291-d0b7d1308d1c-50}
\captionsetup{labelformat=empty}
\caption{Figure 2.19 The graph shows a more accurate graph of the velocity of a jetpowered car during the time span when it is speeding up.}
\end{center}
\end{figure}

\section*{Worked Example}
Using Curvy Velocity Graph to Calculate Some Stuff: Jet Car, Take Two Use Figure 2.19 to (a) find the approximate displacement of the jet car over the time shown, (b) calculate the instantaneous acceleration at \(t=30 \mathrm{~s}\), (c) find the instantaneous velocity at 30 s , and (d) calculate the approximate average velocity over the interval shown.

\section*{Strategy}
a. Because this graph is an undefined curve, we have to estimate shapes over smaller intervals in order to find the areas.\\
b. Like when we were working with a curved displacement graph, we will need to take a tangent line at the instant we are interested and use that to calculate the instantaneous acceleration.\\
c. The instantaneous velocity can still be read off of the graph.\\
d. We will find the average velocity the same way we did in the previous example.

Solution\\
a. 1. This problem is more complicated than the last example. To get a good estimate, we should probably break the curve into four sections. \(0 \rightarrow 10 \mathrm{~s}, 10 \rightarrow 20 \mathrm{~s}, 20 \rightarrow 40 \mathrm{~s}\), and \(40 \rightarrow 70 \mathrm{~s}\).\\
2. Calculate the bottom rectangle (common to all pieces). \(165 \mathrm{~m} / \mathrm{s} \times 70 \mathrm{~s}=11,550 \mathrm{~m}\).\\
3. Estimate a triangle at the top, and calculate the area for each section. Section \(1=225 \mathrm{~m}\); section \(2=100 \mathrm{~m}+450 \mathrm{~m}=550 \mathrm{~m}\); section 3 \(=150 \mathrm{~m}+1,300 \mathrm{~m}=1,450 \mathrm{~m}\); section \(4=2,550 \mathrm{~m}\).\\
4. Add them together to get a net displacement of \(16,325 \mathrm{~m}\).\\
b. Using the tangent line given, we find that the slope is \(1 \mathrm{~m} / \mathrm{s}^{2}\).\\
c. The instantaneous velocity at \(t=30 \mathrm{~s}\), is \(240 \mathrm{~m} / \mathrm{s}\).\\
d. 1. Find the net displacement, which we found in part (a), was 16,325 m.\\
2. Find the total time, which for this case is 70 s .\\
3. Divide \(\frac{16,325 \mathrm{~m}}{70 \mathrm{~s}} \sim 233 \mathrm{~m} / \mathrm{s}\)

Discussion\\
This is a much more complicated process than the first problem. If we were to use these estimates to come up with the average velocity over just the first 30 s we would get about \(191 \mathrm{~m} / \mathrm{s}\). By approximating that curve with a line, we get an average velocity of \(202.5 \mathrm{~m} / \mathrm{s}\). Depending on our purposes and how precise an answer we need, sometimes calling a curve a straight line is a worthwhile approximation.

\section*{Teacher Support}
Teacher Support Finding the tangent line can be a challenging concept for high school students, and they need to understand it theoretically. If you drew a regular curve inside of the curve at the point you are interested in, you could draw a radius of that curve. The tangent line would be the line perpendicular to that radius.\\[0pt]
[OL] Have the students compare this problem and the last one. Ask-What is the difference? When would you care about the more accurate picture of the motion? And when would it really not matter? Why would you ever want to look at a less accurate depiction of motion?

\section*{Practice Problems}
20.

\begin{figure}[h]
\begin{center}
  \includegraphics[max width=\textwidth]{1bfcec96-a1a9-4e0f-9291-d0b7d1308d1c-52}
\captionsetup{labelformat=empty}
\caption{Figure 2.20}
\end{center}
\end{figure}

Consider the velocity vs. time graph shown below of a person in an elevator. Suppose the elevator is initially at rest. It then speeds up for 3 seconds, maintains that velocity for 15 seconds, then slows down for 5 seconds until it stops. Find the instantaneous velocity at \(t=10 \mathrm{~s}\) and \(t=23 \mathrm{~s}\).\\
a. Instantaneous velocity at \(t=10 \mathrm{~s}\) and \(t=23 \mathrm{~s}\) are \(0 \mathrm{~m} / \mathrm{s}\) and \(0 \mathrm{~m} / \mathrm{s}\).\\
b. Instantaneous velocity at \(t=10 \mathrm{~s}\) and \(t=23 \mathrm{~s}\) are \(0 \mathrm{~m} / \mathrm{s}\) and \(3 \mathrm{~m} / \mathrm{s}\).\\
c. Instantaneous velocity at \(t=10 \mathrm{~s}\) and \(t=23 \mathrm{~s}\) are \(3 \mathrm{~m} / \mathrm{s}\) and \(0 \mathrm{~m} / \mathrm{s}\).\\
d. Instantaneous velocity at \(t=10 \mathrm{~s}\) and \(t=23 \mathrm{~s}\) are \(3 \mathrm{~m} / \mathrm{s}\) and \(1.5 \mathrm{~m} / \mathrm{s}\).

\begin{figure}[h]
\begin{center}
  \includegraphics[max width=\textwidth]{1bfcec96-a1a9-4e0f-9291-d0b7d1308d1c-53}
\captionsetup{labelformat=empty}
\caption{Figure 2.21}
\end{center}
\end{figure}

Calculate the net displacement and the average velocity of the elevator over the time interval shown.\\
a. Net displacement is 45 m and average velocity is \(2.10 \mathrm{~m} / \mathrm{s}\).\\
b. Net displacement is 45 m and average velocity is \(2.28 \mathrm{~m} / \mathrm{s}\).\\
c. Net displacement is 57 m and average velocity is \(2.66 \mathrm{~m} / \mathrm{s}\).\\
d. Net displacement is 57 m and average velocity is \(2.48 \mathrm{~m} / \mathrm{s}\).

\section*{Snap Lab}
Graphing Motion, Take Two In this activity, you will graph a moving ball's velocity vs. time.

\begin{itemize}
  \item your graph from the earlier Graphing Motion Snap Lab!
  \item 1 piece of graph paper
  \item 1 pencil
\end{itemize}

Procedure

\begin{enumerate}
  \item Take your graph from the earlier Graphing Motion Snap Lab! and use it to create a graph of velocity vs. time.
  \item Use your graph to calculate the displacement.
  \item 
\end{enumerate}

Describe the graph and explain what it means in terms of velocity and acceleration.\\
a. The graph shows a horizontal line indicating that the ball moved with a constant velocity, that is, it was not accelerating.\\
b. The graph shows a horizontal line indicating that the ball moved with a constant velocity, that is, it was accelerating.\\
c. The graph shows a horizontal line indicating that the ball moved with a variable velocity, that is, it was not accelerating.\\
d. The graph shows a horizontal line indicating that the ball moved with a variable velocity, that is, it was accelerating.

\section*{Teacher Support}
Teacher Support In this lab, students will use the displacement graph they drew in the last snap lab to create a velocity graph. If the rolling ball slowed down in the last snap lab, perhaps due to the ramp being too low, then the graph may not show constant velocity.

\section*{Check Your Understanding}
23.

What information could you obtain by looking at a velocity vs. time graph?\\
a. acceleration\\
b. direction of motion\\
c. reference frame of the motion\\
d. shortest path\\
24.

How would you use a position vs. time graph to construct a velocity vs. time graph and vice versa?\\
a. The slope of a position vs. time curve is used to construct a velocity vs. time curve, and the slope of a velocity vs. time curve is used to construct a position vs. time curve.\\
b. The slope of a position vs. time curve is used to construct a velocity vs. time curve, and the area of a velocity vs. time curve is used to construct a position vs. time curve.\\
c. The area of a position vs. time curve is used to construct a velocity vs. time curve, and the slope of a velocity vs. time curve is used to construct a position vs. time curve.\\
d. The area of a position vs. time curve is used to construct a velocity vs. time curve, and the area of a velocity vs. time curve is used to construct a position vs. time curve.

\section*{Teacher Support}
Teacher Support Use the Check Your Understanding questions to assess students' achievement of the section's learning objectives. If students are struggling with a specific objective, the Check Your Understanding will help direct students to the relevant content.

\section*{Ke Terms}
acceleration the rate at which velocity changes\\
average speed distance traveled divided by time during which motion occurs\\
average velocity displacement divided by time over which displacement occurs\\
dependent variable the variable that changes as the independent variable changes\\
displacement the change in position of an object against a fixed axis\\
distance the length of the path actually traveled between an initial and a final position\\
independent variable the variable, usually along the horizontal axis of a graph, that does not change based on human or experimental action; in physics this is usually time\\
instantaneous speed speed at a specific instant in time\\
instantaneous velocity velocity at a specific instant in time\\
kinematics the study of motion without considering its causes\\
magnitude size or amount\\
position the location of an object at any particular time\\
reference frame a coordinate system from which the positions of objects are described\\
scalar a quantity that has magnitude but no direction\\
speed rate at which an object changes its location\\
tangent a line that touches another at exactly one point\\
vector a quantity that has both magnitude and direction\\
velocity the speed and direction of an object

\section*{Ke Equations}
2.1 Relative Motion, Distance, and Displacement\\
2.2 Speed and Velocity\\
2.3 Position vs. Time Graphs\\
2.4 Velocity vs. Time Graphs

\section*{Section Summar}
\subsection*{2.1 Relative Motion, Distance, and Displacement}
\begin{itemize}
  \item A description of motion depends on the reference frame from which it is described.
  \item The distance an object moves is the length of the path along which it moves.
  \item Displacement is the difference in the initial and final positions of an object.
\end{itemize}

\subsection*{2.2 Speed and Velocity}
\begin{itemize}
  \item Average speed is a scalar quantity that describes distance traveled divided by the time during which the motion occurs.
  \item Velocity is a vector quantity that describes the speed and direction of an object.
  \item Average velocity is displacement over the time period during which the displacement occurs. If the velocity is constant, then average velocity and instantaneous velocity are the same.
\end{itemize}

\subsection*{2.3 Position vs. Time Graphs}
\begin{itemize}
  \item Graphs can be used to analyze motion.
  \item The slope of a position vs. time graph is the velocity.
  \item For a straight line graph of position, the slope is the average velocity.
  \item To obtain the instantaneous velocity at a given moment for a curved graph, find the tangent line at that point and take its slope.
\end{itemize}

\subsection*{2.4 Velocity vs. Time Graphs}
\begin{itemize}
  \item The slope of a velocity vs. time graph is the acceleration.
  \item The area under a velocity vs. time curve is the displacement.
  \item Average velocity can be found in a velocity vs. time graph by taking the weighted average of all the velocities.
\end{itemize}

\end{document}