\documentclass[10pt]{article}
\usepackage[utf8]{inputenc}
\usepackage[T1]{fontenc}
\usepackage{graphicx}
\usepackage[export]{adjustbox}
\graphicspath{ {./images/} }
\usepackage{caption}
\usepackage{amsmath}
\usepackage{amsfonts}
\usepackage{amssymb}
\usepackage[version=4]{mhchem}
\usepackage{stmaryrd}
\usepackage{underscore}

\title{Key Equations }

\author{}
\date{}


\begin{document}
\maketitle
\captionsetup{singlelinecheck=false}
\begin{figure}[h]
\begin{center}
  \includegraphics[max width=\textwidth]{64c96986-1ad4-421f-aa2e-1bd997fa6081-01}
\captionsetup{labelformat=empty}
\caption{Figure 16.1 Flat, smooth surfaces reflect light to form mirror images. (credit: NASA Goddard Photo and Video, via Flickr)}
\end{center}
\end{figure}

\section*{Chapter Outline}
\subsection*{16.1 Reflection}
\subsection*{16.2 Refraction}
\subsection*{16.3 Lenses}
\section*{Introduction}
\section*{Teacher Support}
Teacher Support Explain that light sometimes behaves mysteriously when it interacts with matter. Briefly mention wave-particle duality and let students know this puzzle is studied later. Tell them that, for the interactions in this chapter, light is well behaved and predictable. They will begin to understand that light travels across space in straight lines and changes direction when it bounces off or passes through matter. The angles of these changes of direction can be calculated from equations that depend on the nature of the matter involved. These are the equations that are applied when engineers design microscopes and telescopes. They are also the equations that explain how the human eye works.\\[0pt]
[BL][OL] Review the idea that electromagnetic radiation (unlike sound waves) can travel across empty space as well as through some media.\\[0pt]
[AL] Throughout this chapter, light is represented as traveling in straight lines as long as it is traveling through a single medium or empty space. You may have heard that a prediction of Einstein's relativity theory is that a strong gravitational field can bend the path of light. This hypothesis has been verified, but the effect is only significant in the case of light passing through a very large gravitational field, such as that close to our Sun. In the case of light paths studied in this chapter, this effect is insignificant.

\section*{Misconception Alert}
You have heard that light travels as waves. Do not think the path of light is wavy. The path is a straight line. The wave aspect refers to changes in the electric and magnetic fields that make up electromagnetic radiation.\\
"In another moment Alice was through the glass, and had jumped lightly down into the Looking-glass room."

\begin{itemize}
  \item Through the Looking Glass by Lewis Carol
\end{itemize}

Through the Looking Glass tells of the adventures of Alice after she steps from the real world, through a mirror, and into the virtual world. In this chapter we examine the optical meanings of real and virtual, as well as other concepts that make up the field of optics.

The light from this page or screen is formed into an image by the lens of your eyes, much as the lens of the camera that made the photograph at the beginning of this chapter. Mirrors, like lenses, can also form images, which in turn are captured by your eyes.

Optics is the branch of physics that deals with the behavior of visible light and other electromagnetic waves. For now, we concentrate on the propagation of light and its interaction with matter.

It is convenient to divide optics into two major parts based on the size of objects that light encounters. When light interacts with an object that is several times as large as the light's wavelength, its observable behavior is similar to a ray; it does not display its wave characteristics prominently. We call this part of optics geometric optics. This chapter focuses on situations for which geometric optics is suited.

\section*{Teacher Support}
Teacher Support Before students begin this chapter, it is useful to review the following concepts:

\begin{itemize}
  \item Geometry of angles, including adding and subtracting angles
  \item Trigonometric functions
  \item The electromagnetic spectrum
  \item Characteristics of electromagnetic radiation, including the speed of light
\end{itemize}

\subsection*{16.1 Reflection}
\section*{Section Learning Objectives}
By the end of this section, you will be able to do the following:

\begin{itemize}
  \item Explain reflection from mirrors, describe image formation as a consequence of reflection from mirrors, apply ray diagrams to predict and interpret image and object locations, and describe applications of mirrors
  \item Perform calculations based on the law of reflection and the equations for curved mirrors
\end{itemize}

\section*{Teacher Support}
Teacher Support The learning objectives in this section help your students master the following standards:

\begin{itemize}
  \item (7) Science concepts. The student knows the characteristics and behavior of waves. The student is expected to:
  \item (D) investigate behaviors of waves, including reflection, refraction, diffraction, interference, resonance, and the Doppler effect;
  \item (E) describe and predict image formation as a consequence of reflection from a plane mirror and refraction through a thin convex lens; and
  \item (F) describe the role of wave characteristics and behaviors in medical and industrial applications.
\end{itemize}

In addition, the High School Physics Laboratory Manual addresses content in this section in the lab titled: Mirrors and Lenses, as well as the following standards:

\begin{itemize}
  \item (7) Science concepts. The student knows the characteristics and behavior of waves. The student is expected to:
  \item (D) investigate behaviors of waves, including reflection, refraction, diffraction, interference, resonance, and the Doppler effect.
\end{itemize}

\section*{Section Key Terms}
\section*{Characteristics of Mirrors}
\section*{Teacher Support}
Teacher Support [BL]Recall that, in geometry, angles are numbers that tell\\
how far two straight lines are spread apart. The lines must be straight lines for the number to have meaning.\\[0pt]
[OL]Geometry is the study of relationships involving points, lines, angles, and shapes. In this chapter, we are focused on the first three ideas.\\[0pt]
[AL]In this chapter, we apply equations that use trigonometric functions that describe the properties of angles. Trigonometric functions are ratios of the lengths of two sides of a right triangle. There are six possible ratios; therefore, there are six such functions.

There are three ways, as shown in Figure 16.2, in which light can travel from a source to another location. It can come directly from the source through empty space, such as from the Sun to Earth. Light can travel to an object through various media, such as air and glass. Light can also arrive at an object after being reflected, such as by a mirror. In all these cases, light is modeled as traveling in a straight line, called a ray. Light may change direction when it encounters the surface of a different material (such as a mirror) or when it passes from one material to another (such as when passing from air into glass). It then continues in a straight line - that is, as a ray. The word ray comes from mathematics. Here it means a straight line that originates from some point. It is acceptable to visualize light rays as laser rays (or even science fiction depictions of ray guns).

\begin{figure}[h]
\begin{center}
  \includegraphics[max width=\textwidth]{64c96986-1ad4-421f-aa2e-1bd997fa6081-04}
\captionsetup{labelformat=empty}
\caption{Figure 16.2 Three methods for light to travel from a source to another location are shown. (a) Light reaches the upper atmosphere of Earth by traveling through empty space directly from the source (the Sun). (b) This light can reach a person in one of two ways. It can travel through a medium, such as air or glass, and typically travels from one medium to another. It can also reflect from an object, such as a mirror.}
\end{center}
\end{figure}

Because light moves in straight lines, that is, as rays, and changes directions when it interacts with matter, it can be described through geometry and\\
trigonometry. This part of optics, described by straight lines and angles, is therefore called geometric optics. There are two laws that govern how light changes direction when it interacts with matter: the law of reflection, for situations in which light bounces off matter; and the law of refraction, for situations in which light passes through matter. In this section, we consider the geometric optics of reflection.

\section*{Teacher Support}
Teacher Support [BL]Explain that light bounces is a simplification. The geometry of the path of a bouncing ball is similar to that of light, but what happens at the point of impact is different at the molecular level.\\[0pt]
[OL]Indicate that the terms right angle, perpendicular, and normal line all mean the same thing: a vertical line at a \(90^{\circ}\) angle to a flat surface.\\[0pt]
[AL]Recount and explain all the possible interactions of light with matter. Light can be absorbed at the surface of an opaque object. Some colors of light may be absorbed and others reflected. Light often is partially absorbed and partially reflected. It may also be transmitted through a transparent material, such as water or glass. Typically, if the surface of a transparent material is smooth, such as that of a window pane, light is transmitted partially and reflected partially.

Whenever we look into a mirror or squint at sunlight glinting from a lake, we are seeing a reflection. How does the reflected light travel from the object to your eyes? The law of reflection states: The angle of reflection, \(\theta_{\mathrm{r}}\), equals the angle of incidence, \(\theta_{\mathrm{i}}\). This law governs the behavior of all waves when they interact with a smooth surface, and therefore describe the behavior of light waves as well. The reflection of light is simplified when light is treated as a ray. This concept is illustrated in Figure 16.3, which also shows how the angles are measured relative to the line perpendicular to the surface at the point where the light ray strikes it. This perpendicular line is also called the normal line, or just the normal. Light reflected in this way is referred to as specular (from the Latin word for mirror: speculum).

We expect to see reflections from smooth surfaces, but Figure 16.4, illustrates how a rough surface reflects light. Because the light is reflected from different parts of the surface at different angles, the rays go in many different directions, so the reflected light is diffused. Diffused light allows you to read a printed page from almost any angle because some of the rays go in different directions. Many objects, such as people, clothing, leaves, and walls, have rough surfaces and can be seen from many angles. A mirror, on the other hand, has a smooth surface and reflects light at specific angles.

\begin{figure}[h]
\begin{center}
  \includegraphics[max width=\textwidth]{64c96986-1ad4-421f-aa2e-1bd997fa6081-06}
\captionsetup{labelformat=empty}
\caption{Figure 16.3 The law of reflection states that the angle of reflection, \(\mathbf{r}\), equals the angle of incidence, \(\mathbf{i}\). The angles are measured relative to the line perpendicular to the surface at the point where the ray strikes the surface. The incident and reflected rays, along with the normal, lie in the same plane.}
\end{center}
\end{figure}

\begin{figure}[h]
\begin{center}
  \includegraphics[max width=\textwidth]{64c96986-1ad4-421f-aa2e-1bd997fa6081-06(1)}
\captionsetup{labelformat=empty}
\caption{Figure 16.4 Light is diffused when it reflects from a rough surface. Here, many parallel rays are incident, but they are reflected at many different angles because the surface is rough.}
\end{center}
\end{figure}

When we see ourselves in a mirror, it appears that our image is actually behind the mirror. We see the light coming from a direction determined by the law of reflection. The angles are such that our image is exactly the same distance behind the mirror, \(d_{\mathrm{i}}\), as the distance we stand away from the mirror, \(d_{\mathrm{o}}\). Although these mirror images make objects appear to be where they cannot be (such as behind a solid wall), the images are not figments of our imagination. Mirror images can be photographed and videotaped by instruments and look just as they do to our eyes, which are themselves optical instruments. An image in a mirror is said to be a virtual image, as opposed to a real image. A virtual image is formed when light rays appear to diverge from a point without actually doing so.

Figure 16.5 helps illustrate how a flat mirror forms an image. Two rays are shown emerging from the same point, striking the mirror, and reflecting into the observer's eye. The rays can diverge slightly, and both still enter the eye. If the rays are extrapolated backward, they seem to originate from a common point behind the mirror, allowing us to locate the image. The paths of the reflected rays into the eye are the same as if they had come directly from that\\
point behind the mirror. Using the law of reflection-the angle of reflection equals the angle of incidence - we can see that the image and object are the same distance from the mirror. This is a virtual image, as defined earlier.

\begin{figure}[h]
\begin{center}
  \includegraphics[max width=\textwidth]{64c96986-1ad4-421f-aa2e-1bd997fa6081-07}
\captionsetup{labelformat=empty}
\caption{Figure 16.5 When two sets of rays from common points on an object are reflected by a flat mirror into the eye of an observer, the reflected rays seem to originate from behind the mirror, which determines the position of the virtual image.}
\end{center}
\end{figure}

\section*{Fun In Physics}
Mirror Mazes Figure 16.6 is a chase scene from an old silent film called The Circus, starring Charlie Chaplin. The chase scene takes place in a mirror maze. You may have seen such a maze at an amusement park or carnival. Finding your way through the maze can be very difficult. Keep in mind that only one image in the picture is real-the others are virtual.

\begin{figure}[h]
\begin{center}
  \includegraphics[max width=\textwidth]{64c96986-1ad4-421f-aa2e-1bd997fa6081-08(1)}
\captionsetup{labelformat=empty}
\caption{Figure 16.6 Charlie Chaplin is in a mirror maze. Which image is real?}
\end{center}
\end{figure}

One of the earliest uses of mirrors for creating the illusion of space is seen in the Palace of Versailles, the former home of French royalty. Construction of the Hall of Mirrors (Figure 16.7) began in 1678. It is still one of the most popular tourist attractions at Versailles.

\begin{figure}[h]
\begin{center}
  \includegraphics[max width=\textwidth]{64c96986-1ad4-421f-aa2e-1bd997fa6081-08}
\captionsetup{labelformat=empty}
\caption{Figure 16.7 Tourists love to wander in the Hall of Mirrors at the Palace of Versailles. (credit: Michal Osmenda, Flickr)}
\end{center}
\end{figure}

\section*{Grasp Check}
Only one Charlie in this image (Figure 16.8) is real. The others are all virtual images of him. Can you tell which is real? Hint-His hat is tilted to one side.

\begin{figure}[h]
\begin{center}
  \includegraphics[max width=\textwidth]{64c96986-1ad4-421f-aa2e-1bd997fa6081-09}
\captionsetup{labelformat=empty}
\caption{Figure 16.8}
\end{center}
\end{figure}

a. The virtual images have their hats tilted to the right.\\
b. The virtual images have their hats tilted to the left.\\
c. The real images have their hats tilted to the right.\\
d. The real images have their hats tilted to the left.

\section*{Watch Physics}
Virtual Image This video explains the creation of virtual images in a mirror. It shows the location and orientation of the images using ray diagrams, and\\
relates the perception to the human eye.\\
Click to view content\\
Watch Physics: Virtual Image. This video explains the concept of virtual images.

Click to view content\\
Compare the distance of an object from a mirror to the apparent distance of its virtual image behind the mirror.\\
a. The distances of the image and the object from the mirror are the same.\\
b. The distances of the image and the object from the mirror are always different.\\
c. The image is formed at infinity if the object is placed near the mirror.\\
d. The image is formed near the mirror if the object is placed at infinity.

\section*{Teacher Support}
Teacher Support Have students construct a ray diagram for an object reflected in a plane mirror. Point out to them that all information can be represented in the diagram by using just paper, a pencil, a ruler, and a protractor. Students may use the preceding video and Figure 16.5 to help them to draw the necessary rays for the diagram. Have them compare the position and orientation of the virtual image with that of the object, paying particular attention to the identical distances that the object and image have with respect to the mirror surface.

\section*{Teacher Support}
\section*{Teacher Support}
\section*{Misconception Alert}
[BL]Ask students to define virtual and dispel any misconceptions. Explain the term in relation to geometric optics.\\[0pt]
[OL]Explain that a real focal point is a point at which there is a concentration of light energy that can be transformed into other useful forms. At a virtual focal point, on the other hand, light energy cannot be concentrated because no light actually goes to that point.\\[0pt]
[AL]Explain the difference between a parabolic shape and a spherical shape. Use drawings of a cross-section of each. Point out that, for a short section of a curved mirror with very little curvature, a spherical mirror approximates a parabolic one.

Some mirrors are curved instead of flat. A mirror that curves inward is called a concave mirror, whereas one that curves outward is called a convex mirror. Pick up a well-polished metal spoon and you can see an example of each type of curvature. The side of the spoon that holds the food is a concave mirror; the back of the spoon is a convex mirror. Observe your image on both sides of the spoon.

\section*{Tips For Success}
You can remember the difference between concave and convex by thinking, Concave means caved in.

Ray diagrams can be used to find the point where reflected rays converge or appear to converge, or the point from which rays appear to diverge. This is called the focal point, F . The distance from F to the mirror along the central axis (the line perpendicular to the center of the mirror's surface) is called the focal length, \(f\). Figure 16.9 shows the focal points of concave and convex mirrors.

\begin{figure}[h]
\begin{center}
  \includegraphics[max width=\textwidth]{64c96986-1ad4-421f-aa2e-1bd997fa6081-11}
\captionsetup{labelformat=empty}
\caption{Figure 16.9 (a, b) The focal length for the concave mirror in (a), formed by converging rays, is in front of the mirror, and has a positive value. The focal length for the convex mirror in (b), formed by diverging rays, appears to be behind the mirror, and has a negative value.}
\end{center}
\end{figure}

Images formed by a concave mirror vary, depending on which side of the focal point the object is placed. For any object placed on the far side of the focal point with respect to the mirror, the rays converge in front of the mirror to form a real image, which can be projected onto a surface, such as a screen or sheet of paper However, for an object located inside the focal point with respect to the concave mirror, the image is virtual. For a convex mirror the image is always virtual - that is, it appears to be behind the mirror. The ray diagrams in Figure 16.10 show how to determine the nature of the image formed by concave and convex mirrors.

\begin{figure}[h]
\begin{center}
  \includegraphics[max width=\textwidth]{64c96986-1ad4-421f-aa2e-1bd997fa6081-12}
\captionsetup{labelformat=empty}
\caption{Figure 16.10 (a) The image of an object placed outside the focal point of a concave mirror is inverted and real. (b) The image of an object placed inside the focal point of a concave mirror is erect and virtual. (c) The image of an object formed by a convex mirror is erect and virtual.}
\end{center}
\end{figure}

The information in Figure 16.10 is summarized in Table 16.1.

Table 16.1 Curved Mirror Images This table details the type and orientation of\\
images formed by concave and convex mirrors.

\section*{Snap Lab}
\section*{Concave and Convex Mirrors}
\begin{itemize}
  \item Silver spoon and silver polish, or a new spoon made of any shiny metal
\end{itemize}

Instructions\\
Procedure

\begin{enumerate}
  \item Choose any small object with a top and a bottom, such as a short nail or tack, or a coin, such as a quarter. Observe the object's reflection on the back of the spoon.
  \item Observe the reflection of the object on the front (bowl side) of the spoon when held away from the spoon at a distance of several inches.
  \item Observe the image while slowly moving the small object toward the bowl of the spoon. Continue until the object is all the way inside the bowl of the spoon.
  \item You should see one point where the object disappears and then reappears. This is the focal point.
\end{enumerate}

\section*{Teacher Support}
Teacher Support Describe the differences in the image of the object on the two sides of the focal point. Explain the change. Identify which of the images you saw were real and which were virtual.

Using a concave mirror, you look at the reflection of a faraway object. The image size changes if you move the object closer to the mirror. Why does the image disappear entirely when the object is at the mirror's focal point?\\
a. The height of the image became infinite.\\
b. The height of the object became zero.\\
c. The intensity of intersecting light rays became zero.\\
d. The intensity of intersecting light rays increased.

\section*{Teacher Support}
Teacher Support [BL][OL]Ask students to identify as many examples as they can of curved mirrors that are used in everyday applications. Supply any they miss: security mirrors, mirrors for entering and exiting a driveway with poor visibility, rear-view mirrors, mirrors for application of cosmetics, and so on.

\section*{Watch Physics}
Parabolic Mirrors and Real Images This video uses ray diagrams to show the special feature of parabolic mirrors that makes them ideal for either projecting light energy in parallel rays, with the source being at the focal point of the parabola, or for collecting at the focal point light energy from a distant source.

Click to view content\\
Watch Physics: Parabolic Mirrors and Real Images. This video explains parabolic mirrors and real images.

Click to view content\\
Explain why using a parabolic mirror for a car headlight throws much more light on the highway than a flat mirror.\\
a. The rays do not polarize after reflection.\\
b. The rays are dispersed after reflection.\\
c. The rays are polarized after reflection.\\
d. The rays become parallel after reflection.

\section*{Teacher Support}
\section*{Teacher Support}
\section*{Teacher Demonstration}
Have students use the demonstration in the video to construct a ray diagram that shows that rays from an object (upright arrow) placed at the focal point of a concave mirror emerge parallel to the central axis.

You should be able to notice everyday applications of curved mirrors. One common example is the use of security mirrors in stores, as shown in Figure 16.11.

\begin{figure}[h]
\begin{center}
  \includegraphics[max width=\textwidth]{64c96986-1ad4-421f-aa2e-1bd997fa6081-15}
\captionsetup{labelformat=empty}
\caption{Figure 16.11 Security mirrors are convex, producing a smaller, upright image. Because the image is smaller, a larger area is imaged compared with what would be observed for a flat mirror; hence, security is improved. (credit: Laura D'Alessandro, Flickr)}
\end{center}
\end{figure}

Some telescopes also use curved mirrors and no lenses (except in the eyepieces) both to magnify images and to change the path of light. Figure 16.12 shows a Schmidt-Cassegrain telescope. This design uses a spherical primary concave mirror and a convex secondary mirror. The image is projected onto the focal plane by light passing through the perforated primary mirror. The effective focal length of such a telescope is the focal length of the primary mirror multiplied by the magnification of the secondary mirror. The result is a telescope with a focal length much greater than the length of the telescope itself.

\begin{figure}[h]
\begin{center}
  \includegraphics[max width=\textwidth]{64c96986-1ad4-421f-aa2e-1bd997fa6081-16}
\captionsetup{labelformat=empty}
\caption{Figure 16.12 This diagram shows the design of a Schmidt-Cassegrain telescope.}
\end{center}
\end{figure}

A parabolic concave mirror has the very useful property that all light from a distant source, on reflection by the mirror surface, is directed to the focal point. Likewise, a light source placed at the focal point directs all the light it emits in parallel lines away from the mirror. This case is illustrated by the ray diagram in Figure 16.13. The light source in a car headlight, for example, is located at the focal point of a parabolic mirror.

\section*{Car headlight / Spotlight}
\texttt{https://cdn.mathpix.com/cropped/64c96986-1ad4-421f-aa2e-1bd997fa6081-16.jpg?height=472&width=521&top_left_y=1142&top_left_x=453}

Figure 16.13 The bulb in this ray diagram of a car headlight is located at the focal point of a parabolic mirror.

Parabolic mirrors are also used to collect sunlight and direct it to a focal point, where it is transformed into heat, which in turn can be used to generate electricity. This application is shown in Figure 16.14.

\begin{figure}[h]
\begin{center}
  \includegraphics[max width=\textwidth]{64c96986-1ad4-421f-aa2e-1bd997fa6081-16(1)}
\captionsetup{labelformat=empty}
\caption{Figure 16.14 Parabolic trough collectors are used to generate electricity in southern California. (credit: kjkolb, Wikimedia Commons)}
\end{center}
\end{figure}

\section*{The Application of the Curved Mirror Equations}
\section*{Teacher Support}
Teacher Support [BL][OL]Review operations for manipulating fractions and for rearranging equations involving fractional values of variables.\\[0pt]
[AL]Demonstrate how to solve equations of the type \(\frac{1}{a}=\frac{1}{b}+\frac{1}{c}\) for any of the variables in terms of the other two. Rearrange so that the variable solved for is not a reciprocal.

Curved mirrors and the images they create involve a fairly small number of variables: the mirror's radius of curvature, \(R\); the focal length, \(f\); the distances of the object and image from the mirror, \(d_{o}\) and \(d_{i}\), respectively; and the heights of the object and image, \(h_{o}\) and \(h_{i}\), respectively. The signs of these values indicate whether the image is inverted, erect (upright), real, or virtual. We now look at the equations that relate these variables and apply them to everyday problems.

Figure 16.15 shows the meanings of most of the variables we will use for calculations involving curved mirrors.

\begin{figure}[h]
\begin{center}
  \includegraphics[max width=\textwidth]{64c96986-1ad4-421f-aa2e-1bd997fa6081-17}
\captionsetup{labelformat=empty}
\caption{Figure 16.15 Look for the variables, \(d_{o}, d_{i}, h_{o}, h_{i}\), and \(f\) in this figure.}
\end{center}
\end{figure}

The basic equation that describes both lenses and mirrors is the lens/mirror equation\\
\(\frac{1}{f}=\frac{1}{d_{i}}+\frac{1}{d_{o}}\).\\
This equation can be rearranged several ways. For example, it may be written to solve for focal length.\\
\(f=\frac{d_{i} d_{o}}{d_{o}+d_{i}}\)\\
Magnification, \(m\), is the ratio of the size of the image, \(h_{i}\), to the size of the object, \(h_{o}\). The value of \(m\) can be calculated in two ways.\\
\(m=\frac{h_{i}}{h_{o}}=\frac{-d_{i}}{d_{o}}\)\\
This relationship can be written to solve for any of the variables involved. For example, the height of the image is given by\\
\(h_{i}=-h_{o}\left(\frac{d_{i}}{d_{o}}\right)\).\\
We saved the simplest equation for last. The radius of curvature of a curved mirror, \(R\), is simply twice the focal length.\\
\(R=2 f\)\\
We can learn important information from the algebraic sign of the result of a calculation using the previous equations:

\begin{itemize}
  \item A negative \(d_{i}\) indicates a virtual image; a positive value indicates a real image
  \item A negative \(h_{i}\) indicates an inverted image; a positive value indicates an erect image
  \item For concave mirrors, \(f\) is positive; for convex mirrors, \(f\) is negative
\end{itemize}

Now let's apply these equations to solve some problems.

\section*{Worked Example}
Calculating Focal Length A person standing 6.0 m from a convex security mirror forms a virtual image that appears to be 1.0 m behind the mirror. What is the focal length of the mirror?

\section*{Strategy}
The person is the object, so \(d_{o}=6.0 \mathrm{~m}\). We know that, for this situation, \(d_{o}\) is positive. The image is virtual, so the value for the image distance is negative, so \(d_{i}=-1.0 \mathrm{~m}\).\\
Now, use the appropriate version of the lens/mirror equation to solve for focal length by substituting the known values.

Solution\\
\(f=\frac{d_{i} d_{o}}{d_{o}+d_{i}}=\frac{(-1.0)(6.0)}{6.0+(-1.0)}=\frac{-6.0}{5.0}=-1.2 \mathrm{~m}\)\\
Discussion\\
The negative result is expected for a convex mirror. This indicates the focal point is behind the mirror.

\section*{Worked Example}
Calculating Object Distance Electric room heaters use a concave mirror to reflect infrared (IR) radiation from hot coils. Note that IR radiation follows the same law of reflection as visible light. Given that the mirror has a radius of curvature of 50.0 cm and produces an image of the coils 3.00 m in front of the mirror, where are the coils with respect to the mirror?

\section*{Strategy}
We are told that the concave mirror projects a real image of the coils at an image distance \(d_{i}=3.00 \mathrm{~m}\). The coils are the object, and we are asked to find their location - that is, to find the object distance \(d_{\mathrm{o}}\). We are also given the radius of curvature of the mirror, so that its focal length is \(f=R / 2=25.0 \mathrm{~cm}\) (a positive value, because the mirror is concave, or converging). We can use the lens/mirror equation to solve this problem.

Solution\\
Because \(d_{i}\) and \(f\) are known, the lens/mirror equation can be used to find \(d_{\mathrm{o}}\).\\
\(\frac{1}{f}=\frac{1}{d_{i}}+\frac{1}{d_{o}}\)\\
16.1

Rearranging to solve for \(d_{o}\), we have\\
\(d_{o}=\frac{d_{i} f}{d_{i}-f}\).\\
16.2

Entering the known quantities gives us\\
\(d_{o}=\frac{\left(3.00 \times 10^{2}\right)(25.0)}{\left(3.00 \times 10^{2}\right)-25.0}=27.3 \mathrm{~cm}\).\\
16.3

Discussion\\
Note that the object (the coil filament) is farther from the mirror than the mirror's focal length. This is a case 1 image ( \(d_{\mathrm{o}}>f\) and \(f\) positive), consistent with the fact that a real image is formed. You get the most concentrated thermal energy directly in front of the mirror and 3.00 m away from it. In general, this is not desirable because it could cause burns. Usually, you want the rays to emerge parallel, and this is accomplished by having the filament at the focal point of the mirror.

Note that the filament here is not much farther from the mirror than the focal length, and that the image produced is considerably farther away.

\section*{Practice Problems}
1.

A concave mirror has a radius of curvature of \(0.8 \backslash, \backslash \operatorname{text}\{\mathrm{~m}\}\). What is the focal length of the mirror?\\
a. \(-0.8 \backslash, \backslash \operatorname{text}\{\mathrm{~m}\}\)\\
b. \(-0.4 \backslash, \backslash \operatorname{text}\{\mathrm{~m}\}\)\\
c. \(0.4 \backslash, \backslash \operatorname{text}\{\mathrm{~m}\}\)\\
d. \(0.8 \backslash, \backslash \operatorname{text}\{\mathrm{~m}\}\)\\
2.

What is the focal length of a makeup mirror that produces a magnification of 1.50 when a person's face is 12.0 cm away? Construct a ray diagram using paper, a pencil and a ruler to confirm your calculation.\\
a. -36.0 cm\\
b. -7.20 cm\\
c. 7.20 cm\\
d. 36.0 cm

\section*{Check Your Understanding}
\section*{Teacher Support}
Teacher Support Use these questions to assess student achievement of the section's learning objectives. If students are struggling with a specific objective, these questions will help identify which one, and then you can direct students to the relevant content.\\
3.

How does the object distance, \(\mathrm{d}_{\mathrm{o}}\), compare with the focal length, f , for a concave mirror that produces an image that is real and inverted?\\
a. \(d_{o}>f\), where \(d_{o}\) and \(f\) are object distance and focal length, respectively.\\
b. \(d_{o}<f\), where \(d_{o}\) and \(f\) are object distance and focal length, respectively.\\
c. \(d_{o}=f\), where do and \(f\) are object distance and focal length, respectively.\\
d. \(\mathrm{d}_{\mathrm{o}}=0\), where do is the object distance.\\
4.

Use the law of reflection to explain why it is not a good idea to polish a mirror with coarse sandpaper.\\
a. The surface becomes smooth. A smooth surface produces a sharp image.\\
b. The surface becomes irregular. An irregular surface produces a sharp image.\\
c. The surface becomes smooth. A smooth surface transmits but does not reflect light.\\
d. The surface becomes irregular. An irregular surface produces a blurred image.\\
5.

An object is placed in front of a concave mirror at a distance that is greater than the focal length of the mirror. Will the image produced by the mirror be real or virtual? Will it be erect or inverted?\\
a. It is real and erect.\\
b. It is real and inverted.\\
c. It is virtual and inverted.\\
d. It is virtual and erect.

\subsection*{16.2 Refraction}
\section*{Section Learning Objectives}
By the end of this section, you will be able to do the following:

\begin{itemize}
  \item Explain refraction at media boundaries, predict the path of light after passing through a boundary (Snell's law), describe the index of refraction of materials, explain total internal reflection, and describe applications of refraction and total internal reflection
  \item Perform calculations based on the law of refraction, Snell's law, and the conditions for total internal reflection
\end{itemize}

\section*{Teacher Support}
Teacher Support The learning objectives in this section help your students master the following standards:

\begin{itemize}
  \item (7) Science concepts. The student knows the characteristics and behavior of waves. The student is expected to:
  \item (D) investigate behaviors of waves, including reflection, refraction, diffraction, interference, resonance, and the Doppler effect; and
  \item (F) describe the role of wave characteristics and behaviors in medical and industrial applications.
\end{itemize}

In addition, the High School Physics Laboratory Manual addresses content in this section in the lab titled: Mirrors and Lenses, as well as the following standards:

\begin{itemize}
  \item (7) Science concepts. The student knows the characteristics and behavior of waves. The student is expected to:
  \item (D) investigate behaviors of waves, including reflection, refraction, diffraction, interference, resonance, and the Doppler effect.
\end{itemize}

\section*{Section Key Terms}
\section*{The Law of Refraction}
\section*{Teacher Support}
Teacher Support [BL][OL]Remind students that the maximum speed of light is its speed in a vacuum. This is a fundamental constant of physics. The maximum speed of light is equal to \(3.00 \times 10^{8} \mathrm{~m} / \mathrm{s}\). Have your students memorize this value.

You may have noticed some odd optical phenomena when looking into a fish tank. For example, you may see the same fish appear to be in two different places (Figure 16.16). This is because light coming to you from the fish changes direction when it leaves the tank and, in this case, light rays traveling along two different paths both reach our eyes. The changing of a light ray's direction (loosely called bending) when it passes a boundary between materials of different composition, or between layers in single material where there are changes in temperature and density, is called refraction. Refraction is responsible for a tremendous range of optical phenomena, from the action of lenses to voice transmission through optical fibers.

\begin{figure}[h]
\begin{center}
  \includegraphics[max width=\textwidth]{64c96986-1ad4-421f-aa2e-1bd997fa6081-23}
\captionsetup{labelformat=empty}
\caption{Figure 16.16 Looking at the fish tank as shown, we can see the same fish in two different locations, because light changes directions when it passes from water to air. In this case, light rays traveling on two different paths change direction as they travel from water to air, and so reach the observer. Consequently, the fish appears to be in two different places. This bending of light is called refraction and is responsible for many optical phenomena.}
\end{center}
\end{figure}

\section*{Teacher Support}
Teacher Support [BL]An angle is the measure of the separation of two lines or rays originating from a single point. The length of the lines is not relevant.\\[0pt]
[OL][AL]The trigonometric function sine (sin) for a given angle is the ratio of the side of a right triangle opposite that angle to the hypotenuse of that triangle.

Why does light change direction when passing from one material (medium) to another? It is because light changes speed when going from one material to another. This behavior is typical of all waves and is especially easy to apply to light because light waves have very small wavelengths, and so they can be treated as rays. Before we study the law of refraction, it is useful to discuss the speed of light and how it varies between different media.

The speed of light is now known to great precision. In fact, the speed of light in a vacuum, \(c\), is so important, and is so precisely known, that it is accepted as one of the basic physical quantities, and has the fixed value\\
\(c=2.9972458 \times 10^{8} \mathrm{~m} / \mathrm{s} \approx 3.00 \times 10^{8} \mathrm{~m} / \mathrm{s}\)\\
16.4\\
where the approximate value of \(3.00 \times 10^{8} \mathrm{~m} / \mathrm{s}\) is used whenever three-digit precision is sufficient. The speed of light through matter is less than it is in a vacuum, because light interacts with atoms in a material. The speed of light depends strongly on the type of material, given that its interaction with different atoms, crystal lattices, and other substructures varies. We define the index of refraction, \(n\), of a material to be\\
\(n=\frac{c}{v}\),\\
where \(v\) is the observed speed of light in the material. Because the speed of light is always less than \(c\) in matter and equals \(c\) only in a vacuum, the index of refraction (plural: indices of refraction) is always greater than or equal to one.

Table 16.2 lists the indices of refraction in various common materials.

Table 16.2 Indices of Refraction The table lists the indices of refraction for various materials that are transparent to light. Note, that light travels the slowest in the materials with the greatest indices of refraction.

Figure 16.17 provides an analogy for and a description of how a ray of light changes direction when it passes from one medium to another. As in the previous section, the angles are measured relative to a perpendicular to the surface at the point where the light ray crosses it. The change in direction of the light ray depends on how the speed of light changes. The change in the speed of light is related to the indices of refraction of the media involved. In the situations shown in Figure 16.17, medium 2 has a greater index of refraction than medium 1. This difference in index of refraction means that the speed of light is less in medium 2 than in medium 1. Note that, in Figure 16.17(a), the path of the ray moves closer to the perpendicular when the ray slows down. Conversely, in Figure 16.17(b), the path of the ray moves away from the perpendicular when the ray speeds up. The path is exactly reversible. In both cases, you can imagine what happens by thinking about pushing a lawn mower from a footpath onto grass, and vice versa. Going from the footpath to grass, the right front wheel is slowed and pulled to the side as shown. This is the same change in direction for light when it goes from a fast medium to a slow one. When going from the grass to the footpath, the left front wheel moves faster than the others, and the mower changes direction as shown. This, too, is the same change in direction as light going from slow to fast.\\
\includegraphics[max width=\textwidth, center]{64c96986-1ad4-421f-aa2e-1bd997fa6081-25}

Figure 16.17 The change in direction of a light ray depends on how the speed of light changes when it crosses from one medium to another. For the situations shown here, the speed of light is greater in medium 1 than in medium 2. (a) A ray of light moves closer to the perpendicular when it slows down. This is analogous to what happens when a lawnmower goes from a footpath (medium 1) to grass (medium 2). (b) A ray of light moves away from the perpendicular when it speeds up. This is analogous to what happens when a lawnmower goes from grass (medium 2) to the footpath (medium 1). The paths are exactly reversible.

\section*{Snap Lab}
Bent Pencil A classic observation of refraction occurs when a pencil is placed in a glass filled halfway with water. Do this and observe the shape of the pencil when you look at it sideways through air, glass, and water.

\begin{itemize}
  \item A full-length pencil
  \item A glass half full of water
\end{itemize}

Instructions\\
Procedure

\begin{enumerate}
  \item Place the pencil in the glass of water.
  \item Observe the pencil from the side.
  \item Explain your observations.
\end{enumerate}

\section*{Teacher Support}
Teacher Support Look up the refractive indices of air, glass, and water in Table 16.2. Think about how a ray of light changes direction for these transitions: air to glass and glass to water.

\section*{Virtual Physics}
Bending Light Click to view content\\
The Bending Light simulation allows you to show light refracting as it crosses the boundaries between various media (download animation first to view). It also shows the reflected ray. You can move the protractor to the point where the light meets the boundary and measure the angle of incidence, the angle of refraction, and the angle of reflection. You can also insert a prism into the beam to view the spreading, or dispersion, of white light into colors, as discussed later in this section. Use the ray option at the upper left.

PhET Explorations: Bending Light. Explore bending of light between two media with different indices of refraction. See how changing from air to water to glass changes the bending angle. Play with prisms of different shapes and make rainbows.

Click to view content\\
A light ray moving upward strikes a horizontal boundary at an acute angle relative to the perpendicular and enters the medium above the boundary. What must be true for the light to bend away from the perpendicular?\\
a. The medium below the boundary must have a greater index of refraction than the medium above.\\
b. The medium below the boundary must have a lower index of refraction than the medium above.\\
c. The medium below the boundary must have an index of refraction of zero. d. The medium above the boundary must have an infinite index of refraction.

\section*{Teacher Support}
Teacher Support Have students try all the different tabs at the top of the simulation. Point out to students that, although the tools work in both Ray and Wave mode, some may be easier to use in Wave mode because the region where the tool is able to read is larger.

\section*{Teacher Support}
Teacher Support [BL]Be sure students understand that if \(c\) is always greater than \(v, n\) must always be greater than one. Demonstrating division using numbers that can be divided easily can reinforce student understanding.\\[0pt]
[OL]Explain that, unlike the law of reflection, the law of refraction is most easily expressed as an equation, rather than in words. Walk students through the lawnmower analogy in Figure 16.17. Suggest other wheeled vehicles with which they may be more familiar, and other surfaces, such as sand.\\[0pt]
[AL]Ask students to try to explain why a prism separates white light into a rainbow of colors, but a window pane does not. If they cannot explain it, show them a ray diagram of light transmitted through a flat sheet of glass.

The amount that a light ray changes direction depends both on the incident angle and the amount that the speed changes. For a ray at a given incident angle, a large change in speed causes a large change in direction, and thus a large change in the angle of refraction. The exact mathematical relationship is the law of refraction, or Snell s law, which is stated in equation form as\\
\(n_{1} \sin \theta_{1}=n_{2} \sin \theta_{2}\) or \(\frac{n_{1}}{n_{2}}=\frac{\sin \theta_{2}}{\sin \theta_{1}}\).\\
In terms of speeds, Snell's law becomes\\
\(\frac{\sin \theta_{1}}{\sin \theta_{2}}=\frac{v_{1}}{v_{2}}\).\\
16.5

Here, \(n_{1}\) and \(n_{2}\) are the indices of refraction for media 1 and 2, respectively, and \({ }_{1}\) and \({ }_{2}\) are the angles between the rays and the perpendicular in the respective media 1 and 2, as shown in Figure 16.17. The incoming ray is called the incident ray and the outgoing ray is called the refracted ray. The associated angles are called the angle of incidence and the angle of refraction. Later, we apply Snell's law to some practical situations.

Dispersion is defined as the spreading of white light into the wavelengths of which it is composed. This happens because the index of refraction varies slightly\\
with wavelength. Figure 16.18 shows how a prism disperses white light into the colors of the rainbow.

\begin{figure}[h]
\begin{center}
  \includegraphics[max width=\textwidth]{64c96986-1ad4-421f-aa2e-1bd997fa6081-28}
\captionsetup{labelformat=empty}
\caption{Figure 16.18 (a) A pure wavelength of light ( \(\lambda\) ) falls onto a prism and is refracted at both surfaces. (b) White light is dispersed by the prism (spread of light exaggerated). Because the index of refraction varies with wavelength, the angles of refraction vary with wavelength. A sequence of red to violet is produced, because the index of refraction increases steadily with decreasing wavelength.}
\end{center}
\end{figure}

Rainbows are produced by a combination of refraction and reflection. You may have noticed that you see a rainbow only when you turn your back to the Sun. Light enters a drop of water and is reflected from the back of the drop, as shown in Figure 16.19. The light is refracted both as it enters and as it leaves the drop. Because the index of refraction of water varies with wavelength, the light is dispersed and a rainbow is observed.

\begin{figure}[h]
\begin{center}
  \includegraphics[max width=\textwidth]{64c96986-1ad4-421f-aa2e-1bd997fa6081-29}
\captionsetup{labelformat=empty}
\caption{Figure 16.19 Part of the light falling on this water drop enters and is reflected from the back of the drop. This light is refracted and dispersed both as it enters and as it leaves the drop.}
\end{center}
\end{figure}

\section*{Watch Physics}
Dispersion This video explains how refraction disperses white light into its composite colors.

Click to view content\\
Watch Physics: Dispersion. This video introduces and explains how prisms split white light into so many colors through a process called dispersion.

Click to view content\\
Which colors of the rainbow bend most when refracted?\\
a. Colors with a longer wavelength and higher frequency bend most when refracted.\\
b. Colors with a shorter wavelength and higher frequency bend most when refracted.\\
c. Colors with a shorter wavelength and lower frequency bend most when refracted.\\
d. Colors with a longer wavelength and a lower frequency bend most when refracted.

\section*{Teacher Support}
Teacher Support Have students note that the dependence of the index of refraction on the speed of light implies a dependence on wavelength, as discussed in the video. This is because \(v=f \lambda\) for light in a medium. Different wavelengths of light travel at different speeds, and so refract differently at media boundaries. Also, have students note that the frequency of light is not affected by refraction; it remains constant.

A good-quality mirror reflects more than 90 percent of the light that falls on it; the mirror absorbs the rest. But, it would be useful to have a mirror that reflects\\
all the light that falls on it. Interestingly, we can produce total reflection using an aspect of refraction. Consider what happens when a ray of light strikes the surface between two materials, such as is shown in Figure 16.20(a). Part of the light crosses the boundary and is refracted; the rest is reflected. If, as shown in the figure, the index of refraction for the second medium is less than the first, the ray bends away from the perpendicular. Because \(n_{1}>n_{2}\), the angle of refraction is greater than the angle of incidence - that is, \(\theta_{2}>\theta_{1}\). Now, imagine what happens as the incident angle is increased. This causes \(\theta_{2}\) to increase as well. The largest the angle of refraction, \(\theta_{2}\), can be is \(90^{\circ}\), as shown in Figure 16.20(b). The critical angle, \(\theta_{\mathrm{c}}\), for a combination of two materials is defined to be the incident angle, \(\theta_{1}\), which produces an angle of refraction of \(90^{\circ}\). That is, \(\theta_{\mathrm{c}}\) is the incident angle for which \(\theta_{2}=90^{\circ}\). If the incident angle, \(\theta_{1}\), is greater than the critical angle, as shown in Figure 16.20(c), then all the light is reflected back into medium 1, a condition called total internal re ection.

\begin{figure}[h]
\begin{center}
  \includegraphics[max width=\textwidth]{64c96986-1ad4-421f-aa2e-1bd997fa6081-31}
\captionsetup{labelformat=empty}
\caption{Figure 16.20 (a) A ray of light crosses a boundary where the speed of light increases and the index of refraction decreases-that is, \(\mathrm{n}_{2}<\mathrm{n}_{1}\). The refracted ray bends away from the perpendicular. (b) The critical angle, \(\theta_{\mathrm{c}}\), is the one for which the angle of refraction is \(90^{\circ}\). (c) Total internal reflection occurs when the incident angle is greater than the critical angle.}
\end{center}
\end{figure}

Recall that Snell's law states the relationship between angles and indices of refraction. It is given by\\
\(n_{1} \sin \theta_{1}=n_{2} \sin \theta_{2}\).

When the incident angle equals the critical angle ( \(\theta_{1}=\theta_{\mathrm{c}}\) ), the angle of refraction is \(90^{\circ}\left(\theta_{2}=90^{\circ}\right)\). Noting that \(\sin 90^{\circ}=1\), Snell's law in this case becomes\\
\(n_{1} \sin \theta_{1}=n_{2}\).\\
The critical angle, \(\theta_{\mathrm{c}}\), for a given combination of materials is thus\\
\(\theta_{c}=\sin ^{-1}\left(\frac{n_{2}}{n_{1}}\right)\),\\
for \(n_{1}>n_{2}\).

\section*{Teacher Support}
Teacher Support [OL]The superscript in \(\sin ^{-1}\) is not a power. It indicates arcsine, which is an inverse trigonometric function. It means, "that angle whose sine equals (in this case) \(\left(n_{2} / n_{1}\right)\)."

Total internal reflection occurs for any incident angle greater than the critical angle, \(\theta_{\mathrm{c}}\), and it can only occur when the second medium has an index of refraction less than the first. Note that the previous equation is written for a light ray that travels in medium 1 and reflects from medium 2, as shown in Figure 16.20.

There are several important applications of total internal reflection. Total internal reflection, coupled with a large index of refraction, explains why diamonds sparkle more than other materials. The critical angle for a diamond-to-air surface is only \(24.4^{\circ}\); so, when light enters a diamond, it has trouble getting back out (Figure 16.21). Although light freely enters the diamond at different angles, it can exit only if it makes an angle less than \(24.4^{\circ}\) with the normal to a given surface. Facets on diamonds are specifically intended to make this unlikely, so that the light can exit only in certain places. Diamonds with very few impurities are very clear, so the light makes many internal reflections and is concentrated at the few places it can exit-hence the sparkle.\\
\includegraphics[max width=\textwidth, center]{64c96986-1ad4-421f-aa2e-1bd997fa6081-32}

Figure 16.21 Light cannot escape a diamond easily because its critical angle with air is so small. Most reflections are total and the facets are placed so that light can exit only in particular ways, thus concentrating the light and making the diamond sparkle.

A light ray that strikes an object that consists of two mutually perpendicular reflecting surfaces is reflected back exactly parallel to the direction from which it came. This parallel reflection is true whenever the reflecting surfaces are perpendicular, and it is independent of the angle of incidence. Such an object is called a corner re ector because the light bounces from its inside corner. Many inexpensive reflector buttons on bicycles, cars, and warning signs have corner reflectors designed to return light in the direction from which it originates. Corner reflectors are perfectly efficient when the conditions for total internal reflection are satisfied. With common materials, it is easy to obtain a critical angle that is less than \(45^{\circ}\). One use of these perfect mirrors is in binoculars, as shown in Figure 16.22. Another application is for periscopes used in submarines.

\begin{figure}[h]
\begin{center}
  \includegraphics[max width=\textwidth]{64c96986-1ad4-421f-aa2e-1bd997fa6081-33}
\captionsetup{labelformat=empty}
\caption{Figure 16.22 These binoculars use corner reflectors with total internal reflection to get light to the observer's eyes.}
\end{center}
\end{figure}

Fiber optics are one common application of total internal reflection. In communications, fiber optics are used to transmit telephone, internet, and cable TV signals, and they use the transmission of light down fibers of plastic or glass. Because the fibers are thin, light entering one is likely to strike the inside surface at an angle greater than the critical angle and, thus, be totally reflected (Figure 16.23). The index of refraction outside the fiber must be smaller than inside, a condition that is satisfied easily by coating the outside of the fiber with a material that has an appropriate refractive index. In fact, most fibers have a varying refractive index to allow more light to be guided along the fiber through total internal reflection. Rays are reflected around corners as shown in the figure, making the fibers into tiny light pipes.

\begin{figure}[h]
\begin{center}
  \includegraphics[max width=\textwidth]{64c96986-1ad4-421f-aa2e-1bd997fa6081-34(1)}
\captionsetup{labelformat=empty}
\caption{Figure 16.23 (a) Fibers in bundles are clad by a material that has a lower index of refraction than the core to ensure total internal reflection, even when fibers are in contact with one another. A single fiber with its cladding is shown. (b) Light entering a thin fiber may strike the inside surface at large, or grazing, angles, and is completely reflected if these angles exceed the critical angle. Such rays continue down the fiber, even following it around corners, because the angles of reflection and incidence remain large.}
\end{center}
\end{figure}

\section*{Links To Physics}
Medicine: Endoscopes A medical device called an endoscope is shown in Figure 16.24.\\
\includegraphics[max width=\textwidth, center]{64c96986-1ad4-421f-aa2e-1bd997fa6081-34}

Figure 16.24 Endoscopes, such as the one drawn here, send light down a flexible fiber optic tube, which sends images back to a doctor in charge of performing a medical procedure.

The word endoscope means looking inside. Doctors use endoscopes to look inside hollow organs in the human body and inside body cavities. These devices are used to diagnose internal physical problems. Images may be transmitted to an eyepiece or sent to a video screen. Another channel is sometimes included to allow the use of small surgical instruments. Such surgical procedures include collecting biopsies for later testing, and removing polyps and other growths.

Identify the process that allows light and images to travel through a tube that is not straight.\\
a. The process is refraction of light.\\
b. The process is dispersion of light.\\
c. The process is total internal reflection of light.\\
d. The process is polarization of light.

\section*{Calculations with the Law of Refraction}
\section*{Teacher Support}
Teacher Support [BL]If an equation has two variables and a constant, such as \(n=c / v\), the value of only one variable is needed to find the other.\\[0pt]
[OL]Explain the difference between sine and arcsine. Explain why \(\sin 90^{\circ}=1\). Note that the index of refraction is a dimensionless number.\\[0pt]
[AL]Show why the values \(\sin 0^{\circ}, \sin 30^{\circ}, \sin 45^{\circ}, \sin 60^{\circ}\), and \(\sin 90^{\circ}\) can be expressed in the form\\
\(\frac{1}{2} \sqrt{0}, \frac{1}{2} \sqrt{1}, \frac{1}{2} \sqrt{2}, \frac{1}{2} \sqrt{3}, \frac{1}{2} \sqrt{4}\).\\
16.7

Show that the numerical values of these expressions are \(0,0.5,0.707,0.866\), and 1.00 , respectively.

The calculation problems that follow require application of the following equations:\\
\(n=\frac{c}{v}\),\\
16.8\\
\(n_{1} \sin \theta_{1}=n_{2} \sin \theta_{2}\) or \(\frac{n_{1}}{n_{2}}=\frac{\sin \theta_{2}}{\sin \theta_{1}}\),\\
and\\
\(\theta_{c}=\sin ^{-1}\left(\frac{n_{2}}{n_{1}}\right)\), for \(n_{1}>n_{2}\).\\
These are the equations for refractive index, the mathematical statement of the law of refraction (Snell's law), and the equation for the critical angle.

\section*{Watch Physics}
Snell s Law Example 1 This video leads you through calculations based on the application of the equation that represents Snell's law.

Click to view content\\
Watch Physics: Snell's Law Example 1. This video is a worked example of Snell's law.

Click to view content\\
Which two types of variables are included in Snell's law?\\
a. The two types of variables are density of a material and the angle made by the light ray with the normal.\\
b. The two types of variables are density of a material and the thickness of a material.\\
c. The two types of variables are refractive index and thickness of each material.\\
d. The two types of variables are refractive index of a material and the angle made by a light ray with the normal.

\section*{Worked Example}
Calculating Index of Refraction from Speed Calculate the index of refraction for a solid medium in which the speed of light is \(2.012 \times 10^{8} \mathrm{~m} / \mathrm{s}\), and identify the most likely substance, based on the previous table of indicies of refraction.

\section*{Strategy}
We know the speed of light, \(c\), is \(3.00 \times 10^{8} \mathrm{~m} / \mathrm{s}\), and we are given \(v\). We can simply plug these values into the equation for index of refraction, \(n\).

Solution\\
\(n=\frac{c}{v}=\frac{3.00 \times 10^{8} \mathrm{~m} / \mathrm{s}}{2.012 \times 10^{8} \mathrm{~m} / \mathrm{s}}=1.49\)\\
16.9

This value matches that of polystyrene exactly, according to the table of indices of refraction (Table 16.2).

Discussion\\
The three-digit approximation for \(c\) is used, which in this case is all that is needed. Many values in the table are only given to three significant figures. Note that the units for speed cancel to yield a dimensionless answer, which is correct.

\section*{Worked Example}
Calculating Index of Refraction from Angles Suppose you have an unknown, clear solid substance immersed in water and you wish to identify it by finding its index of refraction. You arrange to have a beam of light enter it at an angle of \(45.00^{\circ}\), and you observe the angle of refraction to be \(40.30^{\circ}\). What are the index of refraction of the substance and its likely identity?

\section*{Strategy}
We must use the mathematical expression for the law of refraction to solve this problem because we are given angle data, not speed data.\\
\(\frac{n_{1}}{n_{2}}=\frac{\sin \theta_{2}}{\sin \theta_{1}}\)\\
16.10

The subscripts 1 and 2 refer to values for water and the unknown, respectively, where 1 represents the medium from which the light is coming and 2 is the new medium it is entering. We are given the angle values, and the table of indicies of refraction gives us \(n\) for water as 1.333. All we have to do before solving the problem is rearrange the equation\\
\(n_{2}=\frac{n_{1} \sin \theta_{1}}{\sin \theta_{2}}\).\\
16.11

Solution\\
\(n_{2}=\frac{(1.333)(0.7071)}{0.6468}=1.457\)\\
16.12

The best match from Table 16.2 is fused quartz, with \(n=1.458\).\\
Discussion\\
Note the relative sizes of the variables involved. For example, a larger angle has a larger sine value. This checks out for the two angles involved. Note that the smaller value of \(\theta_{2}\) compared with \(\theta_{1}\) indicates the ray has bent toward normal. This result is to be expected if the unknown substance has a greater \(n\) value than that of water. The result shows that this is the case.

\section*{Worked Example}
Calculating Critical Angle Verify that the critical angle for light going from water to air is \(48.6^{\circ}\). (See Table 16.2, the table of indices of refraction.)

\section*{Strategy}
First, choose the equation for critical angle\\
\(\theta_{c}=\sin ^{-1}\left(\frac{n_{2}}{n_{1}}\right)\), for \(n_{1}>n_{2}\).\\
16.13

Then, look up the \(n\) values for water, \(n_{1}\), and air, \(n_{2}\). Find the value of \(\frac{n_{2}}{n_{1}}\). Last, find the angle that has a sine equal to this value and it compare with the given angle of \(48.6^{\circ}\).

Solution\\
For water, \(n_{1}=1.333\); for air, \(n_{2}=1.0003\). So,\\
\(\frac{n_{2}}{n_{1}}=\frac{1.0003}{1.333}=0.7504\)\\
\(\sin ^{-1}(0.7504)=48.63^{\circ}\).\\
16.14

Discussion\\
Remember, when we try to find a critical angle, we look for the angle at which light can no longer escape past a medium boundary by refraction. It is logical, then, to think of subscript 1 as referring to the medium the light is trying to leave, and subscript 2 as where it is trying (unsuccessfully) to go. So water is 1 and air is 2 .

\section*{Practice Problems}
6.

The refractive index of ethanol is 1.36 . What is the speed of light in ethanol?\\
a. \(2.25 \times 108 \mathrm{~m} / \mathrm{s}\)\\
b. \(2.21 \times 107 \mathrm{~m} / \mathrm{s}\)\\
c. \(2.25 \times 109 \mathrm{~m} / \mathrm{s}\)\\
d. \(2.21 \times 108 \mathrm{~m} / \mathrm{s}\)

\section*{7.}
The refractive index of air is 1.0003 and the refractive index of crystalline quartz is 1.544 . What is the critical angle for a ray of light going from crystalline quartz into air?\\
a. \(49.61^{\wedge}\{\backslash\) circ \(\}\)\\
b. \(20.19^{\wedge}\{\backslash\) circ \(\}\)\\
c. \(0.6479 \backslash, \backslash \operatorname{text}\{\operatorname{rad}\}\)\\
d. \(0.7048 \backslash, \backslash \operatorname{text}\{\mathrm{rad}\}\)

\section*{Check Your Understanding}
\section*{Teacher Support}
Teacher Support Use these questions to assess student achievement of the section's learning objectives. If students are struggling with a specific objective, these questions help identify which one, and then you can direct students to the relevant content.\\
8.

Which law is expressed by the equation \(\mathrm{n} \_1 ~ \backslash \sin \backslash\) vartheta\_ \(1=\mathrm{n} \_2 \backslash \sin \backslash\) vartheta\_2?\\
a. This is Ohm's law.\\
b. This is Wien's displacement law.\\
c. This is Snell's law.\\
d. This is Newton's law.\\
9.

Explain why the index of refraction is always greater than or equal to one.\\
a. The formula for index of refraction, \(\backslash \operatorname{text}\{\mathrm{n}\}\), of a material is \(\mathrm{n}=\) \textbackslash frac\{ \(\backslash\) text\{speed of light in a material\}\}\{|text\{speed of light in a vacuum \(\}\}=\backslash \operatorname{frac}\{\mathrm{v}\}\{\mathrm{c}\}\), where \(\backslash \operatorname{text}\{\mathrm{v}\}>\backslash \operatorname{text}\{\mathrm{c}\}\), so \(\backslash \operatorname{text}\{\mathrm{n}\}\) is always greater than one.\\
b. The formula for index of refraction, \(\backslash \operatorname{text}\{n\}\), of a material is \(n=\) \textbackslash frac\{ \(\backslash\) text\{speed of light in a vacuum\}\}\{\textbackslash text\{speed of light in a material \(\}\}=\backslash \operatorname{frac}\{\mathrm{c}\}\{\mathrm{v}\}\), where \(\backslash \operatorname{text}\{\mathrm{c}\}>\backslash \operatorname{text}\{\mathrm{v}\}\), so \(\backslash \operatorname{text}\{\mathrm{n}\}\) is always greater than one.\\
c. The formula for index of refraction, \(\backslash \operatorname{text}\{\mathrm{n}\}\), of a material is \(\mathrm{n}=\) \textbackslash text\{speed of light in a vacuum\} \textbackslash times \textbackslash text\{speed of light in a materaial \(\}=\mathrm{c} \backslash\) times v , where \(\backslash \operatorname{text}\{\mathrm{c}\}\), \(\backslash \operatorname{text}\{\mathrm{v}\}>1\), so \(\backslash \operatorname{text}\{\mathrm{n}\}\) is always greater than one.\\
d. The formula for refractive index, \(\backslash \operatorname{text}\{\mathrm{n}\}\), of a material is \(\mathrm{n}= \backslash\) frac\{ 1\(\}\{\backslash\) text\{speed of light in a vacuum\} \textbackslash times \textbackslash text\{speed of light in a material \(\}\}=\backslash \operatorname{frac}\{1\}\{\mathrm{c} \backslash\) times v\(\}\), where \% , so \(\backslash \operatorname{text}\{\mathrm{n}\}\) is always greater than one.\\
10.

Write an equation that expresses the law of refraction.\\
a. \(\backslash\) frac \(\{\) n\_1 \(\}\{\) n\_ 2\(\}=\backslash\) frac \(\{\backslash \sin \backslash\) theta\_1 \(\}\{\backslash \sin \backslash\) theta\_2 \(\}\)\\
b. \(\backslash \operatorname{frac}\{\) n\_2 \(\}\{\) n\_1 \(\}=\backslash \operatorname{left}(\backslash \operatorname{frac}\{\backslash \sin \backslash \text { theta_2 }\}\{\backslash \sin \backslash \text { theta_1 }\} \backslash \text { right })^{\wedge} 2\)\\
c. \(\backslash \operatorname{frac}\{\) n\_\_1 \(\}\{\) n\_2 \(\}=\backslash \operatorname{left}(\backslash \operatorname{frac}\{\backslash \sin \backslash \text { theta_2 }\}\{\backslash \sin \backslash \text { theta_1 }\} \backslash \text { right })^{\wedge} 2\)\\
d. \(\backslash\) frac \(\{\) n\_1 1\(\}\{\) n\_ 2\(\}=\backslash\) frac \(\{\backslash \sin \backslash\) theta\_ 2\(\}\{\backslash \sin \backslash\) theta\_1 \(\}\)

\subsection*{16.3 Lenses}
\section*{Section Learning Objectives}
By the end of this section, you will be able to do the following:

\begin{itemize}
  \item Describe and predict image formation and magnification as a consequence of refraction through convex and concave lenses, use ray diagrams to confirm image formation, and discuss how these properties of lenses determine their applications
  \item Explain how the human eye works in terms of geometric optics
  \item Perform calculations, based on the thin-lens equation, to determine image and object distances, focal length, and image magnification, and use these calculations to confirm values determined from ray diagrams
\end{itemize}

\section*{Section Key Terms}
\section*{Characteristics of Lenses}
\section*{Teacher Support}
Teacher Support [BL][OL][AL]Review the lens/mirror equation from the Reflection section. Review the terms focal point, focal length, object distance, image distance, concave, convex, converging, and diverging from the Reflection section.

Lenses are found in a huge array of optical instruments, ranging from a simple magnifying glass to the eye to a camera's zoom lens. In this section, we use the law of refraction to explore the properties of lenses and how they form images.

Some of what we learned in the earlier discussion of curved mirrors also applies to the study of lenses. Concave, convex, focal point F , and focal length \(f\) have the same meanings as before, except each measurement is made from the center of the lens instead of the surface of the mirror. The convex lens shown in Figure 16.25 has been shaped so that all light rays that enter it parallel to its central axis cross one another at a single point on the opposite side of the lens. The central axis, or axis, is defined to be a line normal to the lens at its center. Such a lens is called a converging lens because of the converging effect it has on light rays. An expanded view of the path of one ray through the lens is shown in Figure 16.25 to illustrate how the ray changes direction both as it enters and as it leaves the lens. Because the index of refraction of the lens is greater than that of air, the ray moves toward the perpendicular as it enters and away from the perpendicular as it leaves. (This is in accordance with the law of refraction.) As a result of the shape of the lens, light is thus bent toward the axis at both surfaces.

\begin{figure}[h]
\begin{center}
  \includegraphics[max width=\textwidth]{64c96986-1ad4-421f-aa2e-1bd997fa6081-41}
\captionsetup{labelformat=empty}
\caption{Figure 16.25 Rays of light entering a convex, or converging, lens parallel to its axis converge at its focal point, F . Ray 2 lies on the axis of the lens. The distance from the center of the lens to the focal point is the focal length, \(f\), of the lens. An expanded view of the path taken by ray 1 shows the perpendiculars and the angles of incidence and refraction at both surfaces.}
\end{center}
\end{figure}

Note that rays from a light source placed at the focal point of a converging lens emerge parallel from the other side of the lens. You may have heard of the trick of using a converging lens to focus rays of sunlight to a point. Such a concentration of light energy can produce enough heat to ignite paper.

Figure 16.26 shows a concave lens and the effect it has on rays of light that enter it parallel to its axis (the path taken by ray 2 in the figure is the axis of the lens). The concave lens is a diverging lens because it causes the light rays to bend away (diverge) from its axis. In this case, the lens has been shaped so all light rays entering it parallel to its axis appear to originate from the same point, F , defined to be the focal point of a diverging lens. The distance from the center of the lens to the focal point is again called the focal length, or " \(f\) " of the lens. Note that the focal length of a diverging lens is defined to be negative. An expanded view of the path of one ray through the lens is shown in Figure 16.26 to illustrate how the shape of the lens, together with the law of refraction, causes the ray to follow its particular path and diverge.

\begin{figure}[h]
\begin{center}
  \includegraphics[max width=\textwidth]{64c96986-1ad4-421f-aa2e-1bd997fa6081-41(1)}
\captionsetup{labelformat=empty}
\caption{Figure 16.26 Rays of light enter a concave, or diverging, lens parallel to its axis}
\end{center}
\end{figure}

diverge and thus appear to originate from its focal point, F . The dashed lines are not rays; they indicate the directions from which the rays appear to come. The focal length, \(f\), of a diverging lens is negative. An expanded view of the path taken by ray 1 shows the perpendiculars and the angles of incidence and refraction at both surfaces.

The power, \(P\), of a lens is very easy to calculate. It is simply the reciprocal of the focal length, expressed in meters\\
\(P=\frac{1}{f}\).\\
16.15

The units of power are diopters, D , which are expressed in reciprocal meters. If the focal length is negative, as it is for the diverging lens in Figure 16.26, then the power is also negative.

\section*{Teacher Support}
Teacher Support [BL][OL]Explain that for ray tracing, the focal point is needed. It is possible to calculate the location of the focal point using the law of refraction (Snell's law) and the refractive index of the lens material, but this process is time-consuming and difficult to do accurately. Repeat the definitions of real, virtual, upright, and inverted, as they apply to images.

\section*{Misconception Alert}
It is an unfortunate fact that the word power is used for two completely different concepts. If you examine a prescription for eyeglasses, the lens powers are given in diopters. If you examine the label on a motor, the energy consumption rate is given as power in watts.

In some circumstances, a lens forms an image at an obvious location, such as when a movie projector casts an image onto a screen. In other cases, the image location is less obvious. Where, for example, is the image formed by eyeglasses? We use ray tracing for thin lenses to illustrate how they form images, and we develop equations to describe the image-formation quantitatively. These are the rules for ray tracing:

\begin{enumerate}
  \item A ray entering a converging lens parallel to its axis passes through the focal point, F , of the lens on the other side
  \item A ray entering a diverging lens parallel to its axis seems to come from the focal point, F , on the side of the entering ray
  \item A ray passing through the center of either a converging or a diverging lens does not change direction
  \item A ray entering a converging lens through its focal point exits parallel to its axis
  \item A ray that enters a diverging lens by heading toward the focal point on the opposite side exits parallel to the axis
\end{enumerate}

Consider an object some distance away from a converging lens, as shown in Figure 16.27. To find the location and size of the image formed, we trace the paths of select light rays originating from one point on the object. In this example, the originating point is the top of a woman's head. Figure 16.27 shows three rays from the top of the object that can be traced using the ray-tracing rules just listed. Rays leave this point traveling in many directions, but we concentrate on only a few, which have paths that are easy to trace. The first ray is one that enters the lens parallel to its axis and passes through the focal point on the other side (rule 1). The second ray passes through the center of the lens without changing direction (rule 3). The third ray passes through the nearer focal point on its way into the lens and leaves the lens parallel to its axis (rule 4). All rays that come from the same point on the top of the person's head are refracted in such a way as to cross at the same point on the other side of the lens. The image of the top of the person's head is located at this point. Rays from another point on the object, such as the belt buckle, also cross at another common point, forming a complete image, as shown. Although three rays are traced in Figure 16.27, only two are necessary to locate the image. It is best to trace rays for which there are simple ray-tracing rules. Before applying ray tracing to other situations, let us consider the example shown in Figure 16.27 in more detail.

\begin{figure}[h]
\begin{center}
  \includegraphics[max width=\textwidth]{64c96986-1ad4-421f-aa2e-1bd997fa6081-44}
\captionsetup{labelformat=empty}
\caption{Figure 16.27 Ray tracing is used to locate the image formed by a lens. Rays originating from the same point on the object are traced. The three chosen rays}
\end{center}
\end{figure}

each follow one of the rules for ray tracing, so their paths are easy to determine. The image is located at the point where the rays cross. In this case, a real image - one that can be projected on a screen-is formed.

The image formed in Figure 16.27 is a real image - meaning, it can be projected. That is, light rays from one point on the object actually cross at the location of the image and can be projected onto a screen, a piece of film, or the retina of an eye.

In Figure 16.27, the object distance, \(d_{o}\), is greater than \(f\). Now we consider a ray diagram for a convex lens where \(d_{o}<f\), and another diagram for a concave lens.

\section*{Virtual Physics}
Geometric Optics Click to view content\\
This animation shows you how the image formed by a convex lens changes as you change object distance, curvature radius, refractive index, and diameter of the lens. To begin, choose Principal Rays in the upper left menu and then try varying some of the parameters indicated at the upper center. Show Help supplies a few helpful labels.

PhET Explorations: Geometric Optics How does a lens form an image? See how light rays are refracted by a lens. Watch how the image changes when you adjust the focal length of the lens, move the object, move the lens, or move the screen.

How does the focal length, f , change with an increasing radius of curvature? How doesf change with an increasing refractive index?\\
a. The focal length increases in both cases: when the radius of curvature and the refractive index increase.\\
b. The focal length decreases in both cases: when the radius of curvature and the refractive index increase.\\
c. The focal length increases when the radius of curvature increases; it decreases when the refractive index increases.\\
d. The focal length decreases when the radius of curvature increases; it increases in when the refractive index increases.

\section*{Teacher Support}
Teacher Support After students become familiar with the operation of the animation, have them select the 2nd Point option at the lower left of the menu. This option provides the rays from a second point on the object. This point can be adjusted so that rays from both the top and, for example, the center of the object may be studied.

Table 16.3 Three Types of Images Formed by Lenses

The examples in Figure 16.27 and Figure 16.28 represent the three possible cases - case 1, case 2, and case 3-summarized in Table 16.3. In the table, \(m\) is magnification; the other symbols have the same meaning as they did for curved mirrors.

\begin{figure}[h]
\begin{center}
  \includegraphics[max width=\textwidth]{64c96986-1ad4-421f-aa2e-1bd997fa6081-46}
\captionsetup{labelformat=empty}
\caption{Figure 16.28 (a) The image is virtual and larger than the object. (b) The image is virtual and smaller than the object.}
\end{center}
\end{figure}

\section*{Snap Lab}
\section*{Focal Length}
\begin{itemize}
  \item Temperature extremes-Very hot or very cold temperatures are encountered in this lab that can cause burns. Use protective mitts, eyewear, and clothing when handling very hot or very cold objects. Notify your teacher immediately of any burns.
  \item EYE SAFETY-Looking at the Sun directly can cause permanent eye damage. Do not look at the Sun through any lens.
  \item Several lenses
  \item A sheet of white paper
  \item A ruler or tape measure
\end{itemize}

Instructions\\
Procedure

\begin{enumerate}
  \item Find several lenses and determine whether they are converging or diverging. In general, those that are thicker near the edges are diverging and those that are thicker near the center are converging.
  \item On a bright, sunny day take the converging lenses outside and try focusing the sunlight onto a sheet of white paper.
  \item Determine the focal lengths of the lenses. Have one partner slowly move the lens toward and away from the paper until you find the distance at which the light spot is at its brightest. Have the other partner measure the distance from the lens to the bright spot. Be careful, because the paper may start to burn, depending on the type of lens.
\end{enumerate}

\section*{Teacher Support}
Teacher Support Be sure there are no flammable materials near the place where you do the experiment. For example, bare concrete pavement is acceptable, but dry, brown grass or leaves is not. Do not look at the Sun through any lens! This could cause permanent eye damage!

True or false - The bright spot that appears in focus on the paper is an image of the Sun.\\
a. True\\
b. False

\section*{Teacher Support}
Teacher Support [BL]Ask students to name as many tools and instruments as they can that incorporate one or more lenses. Fill in the ones they miss with magnifying glass, camera, eye, telescope, microscope, and movie and slide projectors.\\[0pt]
[OL]Explain how the human eye is analogous to a camera. Discuss where the lens, aperture, and focal point is in each. Discuss how a camera focuses objects at different distances by moving the lens whereas the eye does this by changing the shape of the lens.\\[0pt]
[AL]Ask students to define index of refraction and explain how it affects the path of light rays passing through a lens. How would the path vary with changes in the index of refraction of the lens material? With changes in the wavelengths that make up the light ray?

Image formation by lenses can also be calculated from simple equations. We learn how these calculations are carried out near the end of this section.

Some common applications of lenses with which we are all familiar are magnifying glasses, eyeglasses, cameras, microscopes, and telescopes. We take a look at the latter two examples, which are the most complex. We have already seen the design of a telescope that uses only mirrors in Figure 16.12. Figure 16.29 shows the design of a telescope that uses two lenses. Part (a) of the figure shows the design of the telescope used by Galileo. It produces an upright image, which is more convenient for many applications. Part (b) shows an arrangement of lenses used in many astronomical telescopes. This design produces an inverted image, which is less of a problem when viewing celestial objects.

\section*{Teacher Support}
Teacher Support [OL][AL]Explain why it does not matter whether an image of a celestial object is inverted. Ask your students whether they agree with the statement: There is no up or down in space.

\section*{Misconception Alert}
We do not realize that light rays come from every part of an object and pass through every part of the lens; all are used to form the final image. In general, we feel the entire lens, or mirror, is needed to form an image. Actually, half a lens forms the same, although fainter, image.

\begin{figure}[h]
\begin{center}
  \includegraphics[max width=\textwidth]{64c96986-1ad4-421f-aa2e-1bd997fa6081-49}
\captionsetup{labelformat=empty}
\caption{Figure 16.29 (a) Galileo made telescopes with a convex objective and a concave eyepiece. They produce an upright image and are used in spyglasses. (b) Most simple telescopes have two convex lenses. The objective forms a case 1 image, which is the object for the eyepiece. The eyepiece forms a case 2 final image that is magnified.}
\end{center}
\end{figure}

Figure 16.30 shows the path of light through a typical microscope. Microscopes were first developed during the early 1600s by eyeglass makers in the Netherlands and Denmark. The simplest compound microscope is constructed from two convex lenses, as shown schematically in Figure 16.30. The first lens is called the objective lens; it has typical magnification values from \(5 \times\) to \(100 \times\). In standard microscopes, the objectives are mounted such that when you switch between them, the sample remains in focus. Objectives arranged in this way are described as parfocal. The second lens, the eyepiece, also referred to as the ocular, has several lenses that slide inside a cylindrical barrel. The focusing ability is provided by the movement of both the objective lens and the eyepiece. The\\
purpose of a microscope is to magnify small objects, and both lenses contribute to the final magnification. In addition, the final enlarged image is produced in a location far enough from the observer to be viewed easily because the eye cannot focus on objects or images that are too close.

\begin{figure}[h]
\begin{center}
  \includegraphics[max width=\textwidth]{64c96986-1ad4-421f-aa2e-1bd997fa6081-50}
\captionsetup{labelformat=empty}
\caption{Figure 16.30 A compound microscope composed of two lenses, an objective and an eyepiece. The objective forms a case 1 image that is larger than the object. This first image is the object for the eyepiece. The eyepiece forms a case 2 final image that is magnified even further.}
\end{center}
\end{figure}

Real lenses behave somewhat differently from how they are modeled using rays diagrams or the thin-lens equations. Real lenses produce aberrations. An aberration is a distortion in an image. There are a variety of aberrations that result from lens size, material, thickness, and the position of the object. One common type of aberration is chromatic aberration, which is related to color. Because the index of refraction of lenses depends on color, or wavelength, images are produced at different places and with different magnifications for different colors. The law of reflection is independent of wavelength, so mirrors do not have this problem. This result is another advantage for the use of mirrors in optical systems such as telescopes.

Figure 16.31(a) shows chromatic aberration for a single convex lens, and its partial correction with a two-lens system. The index of refraction of the lens increases with decreasing wavelength, so violet rays are refracted more than red rays, and are thus focused closer to the lens. The diverging lens corrects this in part, although it is usually not possible to do so completely. Lenses made of different materials and with different dispersions may be used. For example, an achromatic doublet consisting of a converging lens made of crown glass in contact with a diverging lens made of flint glass can reduce chromatic aberration\\
dramatically (Figure 16.31(b)).

\begin{figure}[h]
\begin{center}
  \includegraphics[max width=\textwidth]{64c96986-1ad4-421f-aa2e-1bd997fa6081-51}
\captionsetup{labelformat=empty}
\caption{Figure 16.31 (a) Chromatic aberration is caused by the dependence of a lens's index of refraction on color (wavelength). The lens is more powerful for violet (V) than for red (R), producing images with different colors, locations, and magnifications. (b) Multiple-lens systems can correct chromatic aberrations in part, but they may require lenses of different materials and add to the expense of optical systems such as cameras.}
\end{center}
\end{figure}

\section*{Physics of the Eye}
\section*{Teacher Support}
Teacher Support [BL]Review bending of light by refraction at a boundary between media of differing refracting indices. Note that the greater the difference in refraction indices, the more the light is bent.\\[0pt]
[OL]Show students a camera and point out the parts of the camera that have the same function as analogous parts of the eye: aperture (iris), lens (lens), film or light-sensitive screen (retina), and memory (brain).\\[0pt]
[AL]Have students with glasses compare them. Ask if they know whether their glasses correct for nearsightedness or farsightedness. Explain why glasses that correct nearsightedness are diverging (concave) and why glasses that correct farsightedness are converging (convex).

The eye is perhaps the most interesting of all optical instruments. It is remarkable in how it forms images and in the richness of detail and color they eye can detect. However, our eyes commonly need some correction to reach what is called normal vision, but should be called ideal vision instead. Image formation\\
by our eyes and common vision correction are easy to analyze using geometric optics. Figure 16.32 shows the basic anatomy of the eye. The cornea and lens form a system that, to a good approximation, acts as a single thin lens. For clear vision, a real image must be projected onto the light-sensitive retina, which lies at a fixed distance from the lens. The lens of the eye adjusts its power to produce an image on the retina for objects at different distances. The center of the image falls on the fovea, which has the greatest density of light receptors and the greatest acuity (sharpness) in the visual field. There are no receptors at the place where the optic nerve meets the eye, which is called the blind spot. An image falling on this spot cannot be seen. The variable opening (or pupil) of the eye along with chemical adaptation allows the eye to detect light intensities from the lowest observable to \(10^{10}\) times greater (without damage). Ten orders of magnitude is an incredible range of detection. Our eyes perform a vast number of functions, such as sense direction, movement, sophisticated colors, and distance. Processing of visual nerve impulses begins with interconnections in the retina and continues in the brain. The optic nerve conveys signals received by the eye to the brain.\\
\includegraphics[max width=\textwidth, center]{64c96986-1ad4-421f-aa2e-1bd997fa6081-52}

Figure 16.32 The cornea and lens of an eye act together to form a real image on the light-sensing retina, which has its densest concentration of receptors in the fovea, and a blind spot over the optic nerve. The power of the lens of an eye is adjustable to provide an image on the retina for varying object distances.

Refractive indices are crucial to image formation using lenses. Table 16.4 shows refractive indices relevant to the eye. The biggest change in the refractive index-and the one that causes the greatest bending of rays-occurs at the cornea rather than the lens. The ray diagram in Figure 16.33 shows image formation by the cornea and lens of the eye. The rays bend according to the refractive indices provided in Table 16.4. The cornea provides about two-thirds of the magnification of the eye because the speed of light changes considerably while traveling from air into the cornea. The lens provides the remaining magnification needed to produce an image on the retina. The cornea and lens can be treated as a single thin lens, although the light rays pass through several layers of material (such as the cornea, aqueous humor, several layers in the lens, and vitreous humor), changing direction at each interface. The image formed\\
is much like the one produced by a single convex lens. This result is a case 1 image. Images formed in the eye are inverted, but the brain inverts them once more to make them seem upright.

\begin{figure}[h]
\begin{center}
\captionsetup{labelformat=empty}
\caption{Table 16.4 Refractive Indices Relevant to the Eye}
  \includegraphics[max width=\textwidth]{64c96986-1ad4-421f-aa2e-1bd997fa6081-53}
\end{center}
\end{figure}

Figure 16.33 An image is formed on the retina, with light rays converging most at the cornea and on entering and exiting the lens. Rays from the top and bottom of the object are traced and produce an inverted real image on the retina. The distance to the object is drawn smaller than scale.

As noted, the image must fall precisely on the retina to produce clear visionthat is, the image distance, \(d_{\mathrm{i}}\), must equal the lens-to-retina distance. Because the lens-to-retina distance does not change, \(d_{\mathrm{i}}\) must be the same for objects at all distances. The eye manages to vary the distance by varying the power (and focal length) of the lens to accommodate for objects at various distances. In Figure 16.33, you can see the small ciliary muscles above and below the lens that change the shape of the lens and, thus, the focal length.

The need for some type of vision correction is very common. Common vision defects are easy to understand, and some are simple to correct. Figure 16.34 illustrates two common vision defects. Nearsightedness, or myopia, is the inability to see distant objects clearly while close objects are in focus. The nearsighted eye overconverges the nearly parallel rays from a distant object, and the rays cross in front of the retina. More divergent rays from a close object are converged on the retina, producing a clear image. Farsightedness, or hyperopia, is the inability to see close objects clearly whereas distant objects may be in focus. A farsighted eye does not converge rays from a close object sufficiently to make\\
the rays meet on the retina. Less divergent rays from a distant object can be converged for a clear image.

\begin{figure}[h]
\begin{center}
  \includegraphics[max width=\textwidth]{64c96986-1ad4-421f-aa2e-1bd997fa6081-54(1)}
\captionsetup{labelformat=empty}
\caption{Figure 16.34 (a) The nearsighted (myopic) eye converges rays from a distant object in front of the retina; thus, they are diverging when they strike the retina, and produce a blurry image. This divergence can be caused by the lens of the eye being too powerful (in other words, too short a focal length) or the length of the eye being too great. (b) The farsighted (hyperopic) eye is unable to converge the rays from a close object by the time they strike the retina and produce ... blurry close vision. This poor convergence can be caused by insufficient power (in other words, too long a focal length) in the lens or by the eye being too short.}
\end{center}
\end{figure}

Because the nearsighted eye overconverges light rays, the correction for nearsightedness involves placing a diverging spectacle lens in front of the eye. This lens reduces the power of an eye that has too short a focal length (Figure 16.35(a)). Because the farsighted eye underconverges light rays, the correction for farsightedness is to place a converging spectacle lens in front of the eye. This lens increases the power of an eye that has too long a focal length (Figure 16.35(b)).\\
\includegraphics[max width=\textwidth, center]{64c96986-1ad4-421f-aa2e-1bd997fa6081-54}

Figure 16.35 (a) Correction of nearsightedness requires a diverging lens that compensates for the overconvergence by the eye. The diverging lens produces an image closer to the eye than the object so that the nearsighted person can see it clearly. (b) Correction of farsightedness uses a converging lens that compensates for the underconvergence by the eye. The converging lens produces an image farther from the eye than the object so that the farsighted person can see it clearly. In both (a) and (b), the rays that meet at the retina represent corrected vision, and the other rays represent blurred vision without corrective lenses.

\section*{Calculations Using Lens Equations}
As promised, there are no new equations to memorize. We can use equations already presented for solving problems involving curved mirrors. Careful analysis allows you to apply these equations to lenses. Here are the equations you need\\
\(P=\frac{1}{f}\),\\
where \(P\) is power, expressed in reciprocal meters \(\left(\mathrm{m}^{-1}\right)\) rather than diopters (D), and \(f\) is focal length, expressed in meters (m). You also need\\
\(\frac{1}{f}=\frac{1}{d_{i}}+\frac{1}{d_{o}}\),\\
where, as before, \(d_{\mathrm{o}}\) and \(d_{\mathrm{i}}\) are object distance and image distance, respectively. Remember, this equation is usually more useful if rearranged to solve for one of the variables. For example,\\
\(d_{i}=\frac{f d_{o}}{d_{o}-f}\).\\
The equations for magnification, \(m\), are also the same as for mirrors\\
\(m=\frac{h_{i}}{h_{o}}=-\frac{d_{i}}{d_{o}}\),\\
where \(h_{\mathrm{i}}\) and \(h_{\mathrm{o}}\) are the image height and object height, respectively. Remember, also, that a negative \(d_{\mathrm{i}}\) value indicates a virtual image and a negative \(h_{\mathrm{i}}\) value indicates an inverted image.

These are the steps to follow when solving a lens problem:

\begin{itemize}
  \item Step 1. Examine the situation to determine that image formation by a lens is involved.
  \item Step 2. Determine whether ray tracing, the thin-lens equations, or both should be used. A sketch is very helpful even if ray tracing is not specifically required by the problem. Write useful symbols and values on the sketch.
  \item Step 3. Identify exactly what needs to be determined in the problem (identify the unknowns).
  \item Step 4. Make a list of what is given or can be inferred from the problem as stated (identify the knowns). It is helpful to determine whether the situation involves a case 1,2 , or 3 image. Although these are just names\\
for types of images, they have certain characteristics (given in Table 16.3) that can be of great use in solving problems.
  \item Step 5. If ray tracing is required, use the ray-tracing rules listed earlier in this section.
  \item Step 6. Most quantitative problems require the use of the thin-lens equations. These equations are solved in the usual manner by substituting knowns and solving for unknowns. Several worked examples were included earlier and can serve as guides.
  \item Step 7. Check whether the answer is reasonable. Does it make sense? If you identified the type of image (case 1, 2, or 3) correctly, you should assess whether your answer is consistent with the type of image, magnification, and so on.
\end{itemize}

All problems will be solved by one or more of the equations just presented, with ray tracing used only for general analysis of the problem. The steps then simplify to the following:

\begin{enumerate}
  \item Identify the unknown.
  \item Identify the knowns.
  \item Choose an equation, plug in the knowns, and solve for the unknown.
\end{enumerate}

Here are some worked examples:

\section*{Worked Example}
\section*{The Power of a Magnifying Glass}
\section*{Strategy}
The Sun is so far away that its rays are nearly parallel when they reach Earth. The magnifying glass is a convex (or converging) lens, focusing the nearly parallel rays of sunlight. Thus, the focal length of the lens is the distance from the lens to the spot, and its power, in diopters (D), is the inverse of this distance (in reciprocal meters).\\
Solution\\
The focal length of the lens is the distance from the center of the lens to the spot, which we know to be 8.00 cm . Thus,\\
\(f=8.00 \mathrm{~cm}\).\\
16.16

To find the power of the lens, we must first convert the focal length to meters; then, we substitute this value into the equation for power.\\
\(P=\frac{1}{f}=\frac{1}{0.0800 \mathrm{~m}}=12.5 \mathrm{D}\)\\
16.17

Discussion

This result demonstrates a relatively powerful lens. Remember that the power of a lens in diopters should not be confused with the familiar concept of power in watts.

\section*{Worked Example}
Image Formation by a Convex Lens A clear glass light bulb is placed 0.75 m from a convex lens with a 0.50 m focal length, as shown in Figure 16.36. Use ray tracing to get an approximate location for the image. Then, use the mirror/lens equations to calculate (a) the location of the image and (b) its magnification. Verify that ray tracing and the thin-lens and magnification equations produce consistent results.

\begin{figure}[h]
\begin{center}
  \includegraphics[max width=\textwidth]{64c96986-1ad4-421f-aa2e-1bd997fa6081-57}
\captionsetup{labelformat=empty}
\caption{Figure 16.36 A light bulb placed 0.75 m from a lens with a 0.50 m focal length produces a real image on a poster board, as discussed in the previous example. Ray tracing predicts the image location and size.}
\end{center}
\end{figure}

\section*{Strategy}
Because the object is placed farther away from a converging lens than the focal length of the lens, this situation is analogous to the one illustrated in the previous figure of a series of drawings showing a woman standing to the left of a lens. Ray tracing to scale should produce similar results for \(d_{\mathrm{i}}\). Numerical solutions for \(d_{\mathrm{i}}\) and \(m\) can be obtained using the thin-lens and magnification equations, noting that \(d_{\mathrm{o}}=0.75 \mathrm{~m}\) and \(f=0.50 \mathrm{~m}\).

Solution\\
The ray tracing to scale in Figure 16.36 shows two rays from a point on the bulb's filament crossing about 1.50 m on the far side of the lens. Thus, the image distance, \(d_{\mathrm{i}}\), is about 1.50 m . Similarly, the image height based on ray tracing is greater than the object height by about a factor of two, and the image is inverted. Thus, \(m\) is about -2 . The minus sign indicates the image is inverted. The lens equation can be rearranged to solve for \(d_{\mathrm{i}}\) from the given information.\\
\(d_{i}=\frac{f d_{o}}{d_{o}-f}=\frac{(0.50)(0.75)}{0.75-0.50}=1.5 \mathrm{~m}\)\\
16.18

Now, we use \(\frac{d_{i}}{d_{o}}\) to find \(m\).\\
\(m=-\frac{d_{i}}{d_{o}}=-\frac{1.5}{0.75}=-2.0\)\\
16.19

Discussion\\
Note that the minus sign causes the magnification to be negative when the image is inverted. Ray tracing and the use of the lens equation produce consistent results. The thin-lens equation gives the most precise results, and is limited only by the accuracy of the given information. Ray tracing is limited by the accuracy with which you draw, but it is highly useful both conceptually and visually.

\section*{Worked Example}
Image Formation by a Concave Lens Suppose an object, such as a book page, is held 6.50 cm from a concave lens with a focal length of -10.0 cm . Such a lens could be used in eyeglasses to correct pronounced nearsightedness. What magnification is produced?

\section*{Strategy}
This example is identical to the preceding one, except that the focal length is negative for a concave or diverging lens. The method of solution is therefore the same, but the results are different in important ways.

Solution\\
\(d_{i}=\frac{f d_{o}}{d_{o}-f}=\frac{(-10.0)(6.50)}{6.50-(-10.0)}=-3.94 \mathrm{~cm}\)\\
16.20

Now the magnification equation can be used to find the magnification, \(m\), because both \(d_{\mathrm{i}}\) and \(d_{\mathrm{o}}\) are known. Entering their values gives\\
\(m=-\frac{d_{i}}{d_{o}}=-\frac{-3.94}{6.50}=0.606\).\\
16.21

Discussion\\
A number of results in this example are true of all case 3 images. Magnification is positive (as calculated), meaning the image is upright. The magnification is also less than one, meaning the image is smaller than the object-in this case, a little more than half its size. The image distance is negative, meaning the image is on the same side of the lens as the object. The image is virtual. The image is closer to the lens than the object, because the image distance is smaller in magnitude than the object distance. The location of the image is not obvious\\
when you look through a concave lens. In fact, because the image is smaller than the object, you may think it is farther away; however, the image is closer than the object-a fact that is useful in correcting nearsightedness.

\section*{Watch Physics}
The Lens Equation and Problem Solving The video shows calculations for both concave and convex lenses. It also explains real versus virtual images, erect versus inverted images, and the significance of negative and positive signs for the involved variables.

Click to view content\\
Watch Physics: Thin Lens Equation and Problem Solving This video gives examples using the thin lens equation.

Click to view content\\
If a lens has a magnification of \(-\backslash \operatorname{frac}\{1\}\{2\}\), how does the image compare with the object in height and orientation?\\
a. The image is erect and is half as tall as the object.\\
b. The image is erect and twice as tall as the object.\\
c. The image is inverted and is half as tall as the object.\\
d. The image is inverted and is twice as tall as the object.

\section*{Practice Problems}
11.

A lens has a focal length of \(12.5 \backslash, \backslash \operatorname{text}\{\mathrm{~cm}\}\). What is the power of the lens?\\
a. The power of the lens is \(0.0400 \backslash, \backslash \operatorname{text}\{\mathrm{D}\}\).\\
b. The power of the lens is \(0.0800 \backslash, \backslash \operatorname{text}\{D\}\).\\
c. The power of the lens is \(4.00 \backslash, \backslash \operatorname{text}\{\mathrm{D}\}\).\\
d. The power of the lens is \(8.00 \backslash, \backslash \operatorname{text}\{\mathrm{D}\}\).\\
12.

If a lens produces a \(5.00-\mathrm{cm}\) tall image of an \(8.00-\mathrm{cm}\)-high object when placed 10.0 cm from the lens, what is the apparent image distance? Construct a ray diagram using paper, a pencil, and a ruler to confirm your calculation.\\
a. -3.12 cm\\
b. -6.25 cm\\
c. 3.12 cm\\
d. 6.25 cm

\section*{Check Your Understanding}
\section*{Teacher Support}
Teacher Support Use these questions to assess student achievement of the section's learning objectives. If students are struggling with a specific objective, these questions help identify which one, and then you can direct students to the relevant content.\\
13.

A lens has a magnification that is negative. What is the orientation of the image?\\
a. Negative magnification means the image is erect and real.\\
b. Negative magnification means the image is erect and virtual.\\
c. Negative magnification means the image is inverted and virtual.\\
d. Negative magnification means the image is inverted and real.\\
14.

Which part of the eye controls the amount of light that enters?\\
a. the pupil\\
b. the iris\\
c. the cornea\\
d. the retina\\
15.

An object is placed between the focal point and a convex lens. Describe the image that is formed in terms of its orientation and whether the image is real or virtual.\\
a. The image is real and erect.\\
b. The image is real and inverted.\\
c. The image is virtual and erect.\\
d. The image is virtual and inverted.\\
16.

A farsighted person buys a pair of glasses to correct her farsightedness. Describe the main symptom of farsightedness and the type of lens that corrects it.\\
a. Farsighted people cannot focus on objects that are far away, but they can see nearby objects easily. A convex lens is used to correct this.\\
b. Farsighted people cannot focus on objects that are close up, but they can see far-off objects easily. A concave lens is used to correct this.\\
c. Farsighted people cannot focus on objects that are close up, but they can see distant objects easily. A convex lens is used to correct this.\\
d. Farsighted people cannot focus on objects that are either close up or far away. A concave lens is used to correct this.

\section*{Teacher Support}
Teacher Support This performance task supports the following:\\
NGSS HS-PS4-1: Use mathematical representations to support a claim regarding relationships among the frequency, wavelength, and speed of waves traveling in various media; and

NGSS HS-PS4-5: Communicate technical information about how some technological devices use the principles of wave behavior and wave interactions with matter to transmit and capture information and energy.

\begin{itemize}
  \item The empty bottle distorts vision, but it does not have much of a lens effect.
  \item 4:\\
a. Shining the light through the bottle onto a wall reveals the focal length. When the image on the wall is smallest and brightest, the distance from the center of the bottle to the wall is the focal length.\\
b. Look through the bottle at the bright object. When the image of the object is most vaguely defined and large, the distance from the center of the bottle to the object is the focal length.\\
c. The water-filled bottle is a better lens because water has a greater index of refraction than air.
\end{itemize}

\section*{Key Terms}
aberration a distortion in an image produced by a lens\\
angle of incidence the angle, with respect to the normal, at which a ray meets a boundary between media or a reflective surface\\
angle of reflection the angle, with respect to the normal, at which a ray leaves a reflective surface\\
angle of refraction the angle between the normal and the refracted ray\\
central axis a line perpendicular to the center of a lens or mirror extending in both directions\\
chromatic aberration an aberration related to color\\
concave lens a lens that causes light rays to diverge from the central axis\\
concave mirror a mirror with a reflective side that is curved inward\\
converging lens a convex lens\\
convex lens a lens that causes light rays to converge toward the central axis\\
convex mirror a mirror with a reflective side that is curved outward\\
critical angle an incident angle that produces an angle of refraction of \(90^{\circ}\)\\
dispersion separation of white light into its component wavelengths\\
diverging lens a concave lens\\
focal length the distance from the focal point to the mirror\\
focal point the point at which rays converge or appear to converge\\
incident ray the incoming ray toward a medium boundary or a reflective surface\\
index of refraction the speed of light in a vacuum divided by the speed of light in a given material\\
law of reflection the law that indicates the angle of reflection equals the angle of incidence\\
law of refraction the law that describes the relationship between refractive indices of materials on both sides of a boundary and the change in the path of light crossing the boundary, as given by the equation \(n_{1} \sin \theta_{1}= n_{2} \sin \theta_{2}\)\\
ray light traveling in a straight line\\
real image an optical image formed when light rays converge and pass through the image, producing an image that can be projected onto a screen\\
refracted ray the light ray after it has been refracted

Snell s law the law of refraction expressed mathematically as \(n_{1} \sin \theta_{1}= n_{2} \sin \theta_{2}\)\\
total internal reflection reflection of light traveling through a medium with a large refractive index at a boundary of a medium with a low refractive index under conditions such that refraction cannot occur\\
virtual image the point from which light rays appear to diverge without actually doing so

16.1 Reflection\\
16.2 Refraction\\
16.3 Lenses

\section*{Section Summary}
\subsection*{16.1 Reflection}
\begin{itemize}
  \item The angle of reflection equals the angle of incidence.
  \item Plane mirrors and convex mirrors reflect virtual, erect images. Concave mirrors reflect light to form real, inverted images or virtual, erect images, depending on the location of the object.
  \item Image distance, height, and other characteristics can be calculated using the lens/mirror equation and the magnification equation.
\end{itemize}

\subsection*{16.2 Refraction}
\begin{itemize}
  \item The index of refraction for a material is given by the speed of light in a vacuum divided by the speed of light in that material.
  \item Snell's law states the relationship between indices of refraction, the incident angle, and the angle of refraction.
  \item The critical angle, \(\theta_{\mathrm{c}}\), determines whether total internal refraction can take place, and can be calculated according to \(\theta_{c}=\sin ^{-1}\left(\frac{n_{2}}{n_{1}}\right)\).
\end{itemize}

\subsection*{16.3 Lenses}
\begin{itemize}
  \item The characteristics of images formed by concave and convex lenses can be predicted using ray tracing. Characteristics include real versus virtual, inverted versus upright, and size.
  \item The human eye and corrective lenses can be explained using geometric optics.
  \item Characteristics of images formed by lenses can be calculated using the mirror/lens equation.
\end{itemize}

\end{document}