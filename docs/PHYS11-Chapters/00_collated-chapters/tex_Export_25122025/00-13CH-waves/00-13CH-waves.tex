\documentclass[10pt]{article}
\usepackage[utf8]{inputenc}
\usepackage[T1]{fontenc}
\usepackage{graphicx}
\usepackage[export]{adjustbox}
\graphicspath{ {./images/} }
\usepackage{caption}
\usepackage{amsmath}
\usepackage{amsfonts}
\usepackage{amssymb}
\usepackage[version=4]{mhchem}
\usepackage{stmaryrd}

\begin{document}
\captionsetup{singlelinecheck=false}
\begin{figure}[h]
\begin{center}
  \includegraphics[max width=\textwidth]{e5042e07-d38c-4a98-ad7f-01ff1992bfa2-01}
\captionsetup{labelformat=empty}
\caption{Figure 13.1 Waves in the ocean behave similarly to all other types of waves. (Steve Jurveston, Flickr)}
\end{center}
\end{figure}

\section*{Chapter Outline}
\subsection*{13.1 Types of Waves}
13.2 Wave Properties: Speed, Amplitude, Frequency, and Period\\
13.3 Wave Interaction: Superposition and Interference

\section*{Introduction}
\section*{Teacher Support}
Tell students that, in this chapter, they will learn about different waves, their properties, and their interactions. Ask students to name and describe the types of waves they have learned about previously. Students may be familiar with water waves as well as light waves, sound waves, electromagnetic waves, etc. Ask students how these waves are different from and similar to each other.

Recall from the chapter on Motion in Two Dimensions that oscillations-the back-and-forth movement between two points-involve force and energy. Some oscillations create waves, such as the sound waves created by plucking a guitar string. Other examples of waves include earthquakes and visible light. Even subatomic particles, such as electrons, can behave like waves. You can make water waves in a swimming pool by slapping the water with your hand. Some\\
of these waves, such as water waves, are visible; others, such as sound waves, are not. But every wave is a disturbance that moves from its source and carries energy. In this chapter, we will learn about the different types of waves, their properties, and how they interact with one another.

\section*{Teacher Support}
Teacher Support Before starting the chapter, it would help to review the concepts of force, oscillations, and simple harmonic motion.

\subsection*{13.1 Types of Waves}
\section*{Section Learning Objectives}
By the end of this section, you will be able to do the following:

\begin{itemize}
  \item Define mechanical waves and medium, and relate the two
  \item Distinguish a pulse wave from a periodic wave
  \item Distinguish a longitudinal wave from a transverse wave and give examples of such waves
\end{itemize}

\section*{Teacher Support}
Teacher Support The learning objectives in this section will help your students master the following standards:

\begin{itemize}
  \item (7) Science concepts. The student knows the characteristics and behavior of waves. The student is expected to:
  \item (A) examine and describe oscillatory motion and wave propagation in various types of media.
\end{itemize}

\section*{Section Key Terms}
\section*{Mechanical Waves}
What do we mean when we say something is a wave? A wave is a disturbance that travels or propagates from the place where it was created. Waves transfer energy from one place to another, but they do not necessarily transfer any mass. Light, sound, and waves in the ocean are common examples of waves. Sound and water waves are mechanical waves; meaning, they require a medium to travel through. The medium may be a solid, a liquid, or a gas, and the speed of the wave depends on the material properties of the medium through which it is traveling. However, light is not a mechanical wave; it can travel through a vacuum such as the empty parts of outer space.

A familiar wave that you can easily imagine is the water wave. For water waves, the disturbance is in the surface of the water, an example of which is the disturbance created by a rock thrown into a pond or by a swimmer splashing the water surface repeatedly. For sound waves, the disturbance is caused by a change in air pressure, an example of which is when the oscillating cone inside a speaker creates a disturbance. For earthquakes, there are several types of disturbances, which include the disturbance of Earth's surface itself and the pressure disturbances under the surface. Even radio waves are most\\
easily understood using an analogy with water waves. Because water waves are common and visible, visualizing water waves may help you in studying other types of waves, especially those that are not visible.

Water waves have characteristics common to all waves, such as amplitude, period, frequency, and energy, which we will discuss in the next section.

\section*{Teacher Support}
\section*{Teacher Support}
\section*{Misconception Alert}
Many people think that water waves push water from one direction to another. In reality, however, the particles of water tend to stay in one location only, except for moving up and down due to the energy in the wave. The energy moves forward through the water, but the water particles stay in one place. If you feel yourself being pushed in an ocean, what you feel is the energy of the wave, not the rush of water. If you put a cork in water that has waves, you will see that the water mostly moves it up and down.\\[0pt]
[BL][OL][AL] Ask students to give examples of mechanical and nonmechanical waves.

\section*{Pulse Waves and Periodic Waves}
If you drop a pebble into the water, only a few waves may be generated before the disturbance dies down, whereas in a wave pool, the waves are continuous. A pulse wave is a sudden disturbance in which only one wave or a few waves are generated, such as in the example of the pebble. Thunder and explosions also create pulse waves. A periodic wave repeats the same oscillation for several cycles, such as in the case of the wave pool, and is associated with simple harmonic motion. Each particle in the medium experiences simple harmonic motion in periodic waves by moving back and forth periodically through the same positions.

\section*{Teacher Support}
Teacher Support [BL] Any kind of wave, whether mechanical or nonmechanical, or transverse or longitudinal, can be in the form of a pulse wave or a periodic wave.

Consider the simplified water wave in Figure 13.2. This wave is an up-anddown disturbance of the water surface, characterized by a sine wave pattern. The uppermost position is called the crest and the lowest is the trough. It causes a seagull to move up and down in simple harmonic motion as the wave crests and troughs pass under the bird.

\begin{figure}[h]
\begin{center}
\texttt{https://cdn.mathpix.com/cropped/e5042e07-d38c-4a98-ad7f-01ff1992bfa2-05.jpg?height=362&width=832&top_left_y=442&top_left_x=455}
\captionsetup{labelformat=empty}
\caption{Figure 13.2 An idealized ocean wave passes under a seagull that bobs up and down in simple harmonic motion.}
\end{center}
\end{figure}

\section*{Longitudinal Waves and Transverse Waves}
Mechanical waves are categorized by their type of motion and fall into any of two categories: transverse or longitudinal. Note that both transverse and longitudinal waves can be periodic. A transverse wave propagates so that the disturbance is perpendicular to the direction of propagation. An example of a transverse wave is shown in Figure 13.3, where a woman moves a toy spring up and down, generating waves that propagate away from herself in the horizontal direction while disturbing the toy spring in the vertical direction.

\begin{figure}[h]
\begin{center}
  \includegraphics[max width=\textwidth]{e5042e07-d38c-4a98-ad7f-01ff1992bfa2-05(1)}
\captionsetup{labelformat=empty}
\caption{Figure 13.3 In this example of a transverse wave, the wave propagates horizontally and the disturbance in the toy spring is in the vertical direction.}
\end{center}
\end{figure}

In contrast, in a longitudinal wave, the disturbance is parallel to the direction of propagation. Figure 13.4 shows an example of a longitudinal wave, where the woman now creates a disturbance in the horizontal direction - which is the same direction as the wave propagation-by stretching and then compressing the toy spring.\\
\includegraphics[max width=\textwidth, center]{e5042e07-d38c-4a98-ad7f-01ff1992bfa2-05}

Figure 13.4 In this example of a longitudinal wave, the wave propagates horizontally and the disturbance in the toy spring is also in the horizontal direction.

\section*{Tips For Success}
Longitudinal waves are sometimes called compression waves or compressional waves, and transverse waves are sometimes called shear waves.

\section*{Teacher Support}
\section*{Teacher Support}
\section*{Teacher Demonstration}
Transverse and longitudinal waves may be demonstrated in the class using a spring or a toy spring, as shown in the figures.

Waves may be transverse, longitudinal, or a combination of the two. The waves on the strings of musical instruments are transverse (as shown in Figure 13.5), and so are electromagnetic waves, such as visible light. Sound waves in air and water are longitudinal. Their disturbances are periodic variations in pressure that are transmitted in fluids.

\begin{figure}[h]
\begin{center}
  \includegraphics[max width=\textwidth]{e5042e07-d38c-4a98-ad7f-01ff1992bfa2-06}
\captionsetup{labelformat=empty}
\caption{Figure 13.5 The wave on a guitar string is transverse. However, the sound wave coming out of a speaker rattles a sheet of paper in a direction that shows that such sound wave is longitudinal.}
\end{center}
\end{figure}

Sound in solids can be both longitudinal and transverse. Essentially, water waves are also a combination of transverse and longitudinal components, although the simplified water wave illustrated in Figure 13.2 does not show the longitudinal motion of the bird.

Earthquake waves under Earth's surface have both longitudinal and transverse components as well. The longitudinal waves in an earthquake are called pressure or P -waves, and the transverse waves are called shear or S -waves. These components have important individual characteristics; for example, they propagate at different speeds. Earthquakes also have surface waves that are similar to surface waves on water.

\section*{Teacher Support}
Teacher Support Energy propagates differently in transverse and longitudinal waves. It is important to know the type of the wave in which energy is propagating to understand how it may affect the materials around it.

\section*{Watch Physics}
Introduction to Waves This video explains wave propagation in terms of momentum using an example of a wave moving along a rope. It also covers the differences between transverse and longitudinal waves, and between pulse and periodic waves.

Click to view content\\
Watch Physics: Introduction to Waves. This video is an introduction to transverse and longitudinal waves.

Click to view content\\
In a longitudinal sound wave, after a compression wave moves through a region, the density of molecules briefly decreases. Why is this?\\
a. After a compression wave, some molecules move forward temporarily.\\
b. After a compression wave, some molecules move backward temporarily.\\
c. After a compression wave, some molecules move upward temporarily.\\
d. After a compression wave, some molecules move downward temporarily.

\section*{Fun In Physics}
The Physics of Surfing Many people enjoy surfing in the ocean. For some surfers, the bigger the wave, the better. In one area off the coast of central California, waves can reach heights of up to 50 feet in certain times of the year (Figure 13.6).

\begin{figure}[h]
\begin{center}
  \includegraphics[max width=\textwidth]{e5042e07-d38c-4a98-ad7f-01ff1992bfa2-08}
\captionsetup{labelformat=empty}
\caption{Figure 13.6 A surfer negotiates a steep take-off on a winter day in California while his friend watches. (Ljsurf, Wikimedia Commons)}
\end{center}
\end{figure}

How do waves reach such extreme heights? Other than unusual causes, such as when earthquakes produce tsunami waves, most huge waves are caused simply by interactions between the wind and the surface of the water. The wind pushes up against the surface of the water and transfers energy to the water in the process. The stronger the wind, the more energy transferred. As waves start to form, a larger surface area becomes in contact with the wind, and even more energy is transferred from the wind to the water, thus creating higher waves. Intense storms create the fastest winds, kicking up massive waves that travel out from the origin of the storm. Longer-lasting storms and those storms that affect a larger area of the ocean create the biggest waves since they transfer more energy. The cycle of the tides from the Moon's gravitational pull also plays a small role in creating waves.

Actual ocean waves are more complicated than the idealized model of the simple transverse wave with a perfect sinusoidal shape. Ocean waves are examples of orbital progressive waves, where water particles at the surface follow a circular path from the crest to the trough of the passing wave, then cycle back again to their original position. This cycle repeats with each passing wave.

As waves reach shore, the water depth decreases and the energy of the wave is compressed into a smaller volume. This creates higher waves - an effect known as shoaling.

Since the water particles along the surface move from the crest to the trough, surfers hitch a ride on the cascading water, gliding along the surface. If ocean waves work exactly like the idealized transverse waves, surfing would be much less exciting as it would simply involve standing on a board that bobs up and down in place, just like the seagull in the previous figure.

Additional information and illustrations about the scientific principles behind surfing can be found in the "Using Science to Surf Better!" video.

If we lived in a parallel universe where ocean waves were longitudinal, what would a surfer's motion look like?\\
a. The surfer would move side-to-side/back-and-forth vertically with no horizontal motion.\\
b. The surfer would forward and backward horizontally with no vertical motion.

\section*{Check Your Understanding}
\section*{Teacher Support}
Teacher Support Use these questions to assess students' achievement of the section's Learning Objectives. If students are struggling with a specific objective, these questions will help identify such objective and direct them to the relevant content.\\
1.

What is a wave?\\
a. A wave is a force that propagates from the place where it was created.\\
b. A wave is a disturbance that propagates from the place where it was created.\\
c. A wave is matter that provides volume to an object.\\
d. A wave is matter that provides mass to an object.\\
2.

Do all waves require a medium to travel? Explain.\\
a. No, electromagnetic waves do not require any medium to propagate.\\
b. No, mechanical waves do not require any medium to propagate.\\
c. Yes, both mechanical and electromagnetic waves require a medium to propagate.\\
d. Yes, all transverse waves require a medium to travel.\\
3.

What is a pulse wave?\\
a. A pulse wave is a sudden disturbance with only one wave generated.\\
b. A pulse wave is a sudden disturbance with only one or a few waves generated.\\
c. A pulse wave is a gradual disturbance with only one or a few waves generated.\\
d. A pulse wave is a gradual disturbance with only one wave generated.\\
4.

Is the following statement true or false? A pebble dropped in water is an example of a pulse wave.\\
a. False\\
b. True\\
5.

What are the categories of mechanical waves based on the type of motion?\\
a. Both transverse and longitudinal waves\\
b. Only longitudinal waves\\
c. Only transverse waves\\
d. Only surface waves

\section*{6.}
In which direction do the particles of the medium oscillate in a transverse wave?\\
a. Perpendicular to the direction of propagation of the transverse wave\\
b. Parallel to the direction of propagation of the transverse wave

\subsection*{13.2 Wave Properties: Speed, Amplitude, Frequency, and Period}
\section*{Section Learning Objectives}
By the end of this section, you will be able to do the following:

\begin{itemize}
  \item Define amplitude, frequency, period, wavelength, and velocity of a wave
  \item Relate wave frequency, period, wavelength, and velocity
  \item Solve problems involving wave properties
\end{itemize}

\section*{Teacher Support}
Teacher Support The learning objectives in this section will help your students master the following standards:

\begin{itemize}
  \item (7) Science concepts. The student knows the characteristics and behavior of waves. The student is expected to:
  \item (B) investigate and analyze the characteristics of waves, including velocity, frequency, amplitude, and wavelength, and calculate using the relationship between wave speed, frequency, and wavelength;
  \item (D) investigate the behaviors of waves, including reflection, refraction, diffraction, interference, resonance, and the Doppler effect.
\end{itemize}

\section*{Section Key Terms}
\section*{Teacher Support}
Teacher Support [BL][OL][AL] Review amplitude, period, and frequency for simple harmonic motion.

\section*{Wave Variables}
In the chapter on motion in two dimensions, we defined the following variables to describe harmonic motion:

\begin{itemize}
  \item Amplitude-maximum displacement from the equilibrium position of an object oscillating around such equilibrium position
  \item Frequency-number of events per unit of time
  \item Period-time it takes to complete one oscillation
\end{itemize}

For waves, these variables have the same basic meaning. However, it is helpful to word the definitions in a more specific way that applies directly to waves:

\begin{itemize}
  \item Amplitude - distance between the resting position and the maximum displacement of the wave
  \item Frequency-number of waves passing by a specific point per second
  \item Period-time it takes for one wave cycle to complete
\end{itemize}

In addition to amplitude, frequency, and period, their wavelength and wave velocity also characterize waves. The wavelength \(\lambda\) is the distance between adjacent identical parts of a wave, parallel to the direction of propagation. The wave velocity \(v_{w}\) is the speed at which the disturbance moves.

\section*{Tips For Success}
Wave velocity is sometimes also called the propagation velocity or propagation speed because the disturbance propagates from one location to another.

Consider the periodic water wave in Figure 13.7. Its wavelength is the distance from crest to crest or from trough to trough. The wavelength can also be thought of as the distance a wave has traveled after one complete cycle - or one period. The time for one complete up-and-down motion is the simple water wave's period \(T\). In the figure, the wave itself moves to the right with a wave velocity \(v_{\mathrm{w}}\). Its amplitude \(X\) is the distance between the resting position and the maximum displacement-either the crest or the trough-of the wave. It is important to note that this movement of the wave is actually the disturbance moving to the right, not the water itself; otherwise, the bird would move to the right. Instead, the seagull bobs up and down in place as waves pass underneath, traveling a total distance of \(2 X\) in one cycle. However, as mentioned in the text feature on surfing, actual ocean waves are more complex than this simplified example.

\begin{figure}[h]
\begin{center}
  \includegraphics[max width=\textwidth]{e5042e07-d38c-4a98-ad7f-01ff1992bfa2-12}
\captionsetup{labelformat=empty}
\caption{Figure 13.7 The wave has a wavelength , which is the distance between adjacent identical parts of the wave. The up-and-down disturbance of the surface propagates parallel to the surface at a speed \(\mathrm{v}_{\mathrm{w}}\).}
\end{center}
\end{figure}

\section*{Watch Physics}
Amplitude, Period, Frequency, and Wavelength of Periodic Waves This video is a continuation of the video "Introduction to Waves" from the\\
"Types of Waves" section. It discusses the properties of a periodic wave: amplitude, period, frequency, wavelength, and wave velocity.

Click to view content

\section*{Tips For Success}
The crest of a wave is sometimes also called the peak.\\
Watch Physics: Amplitude, Period, Frequency and Wavelength of Periodic Waves. This video introduces several concepts of sound; amplitude, period, frequency, and wavelength of periodic waves.

Click to view content\\
If you are on a boat in the trough of a wave on the ocean, and the wave amplitude is \(1 \backslash, \backslash \operatorname{text}\{\mathrm{~m}\}\), what is the wave height from your position?\\
a. \(1 \backslash, \backslash \operatorname{text}\{\mathrm{~m}\}\)\\
b. \(2 \backslash, \backslash \operatorname{text}\{\mathrm{~m}\}\)\\
c. \(4 \backslash, \backslash \operatorname{text}\{\mathrm{~m}\}\)\\
d. \(8 \backslash, \backslash \operatorname{text}\{\mathrm{~m}\}\)

\section*{The Relationship between Wave Frequency, Period, Wavelength, and Velocity}
Since wave frequency is the number of waves per second, and the period is essentially the number of seconds per wave, the relationship between frequency and period is\\
\(f=\frac{1}{T}\)\\
13.1\\
or\\
\(T=\frac{1}{f}\),\\
13.2\\
just as in the case of harmonic motion of an object. We can see from this relationship that a higher frequency means a shorter period. Recall that the unit for frequency is hertz (Hz), and that 1 Hz is one cycle-or one wave-per second.

The speed of propagation \(v_{\mathrm{w}}\) is the distance the wave travels in a given time, which is one wavelength in a time of one period. In equation form, it is written as\\
\(v_{w}=\frac{\lambda}{T}\)\\
13.3\\
or\\
\(v_{w}=f \lambda\).\\
13.4

From this relationship, we see that in a medium where \(v_{\mathrm{w}}\) is constant, the higher the frequency, the smaller the wavelength. See Figure 13.8.

\begin{figure}[h]
\begin{center}
  \includegraphics[max width=\textwidth]{e5042e07-d38c-4a98-ad7f-01ff1992bfa2-14}
\captionsetup{labelformat=empty}
\caption{Figure 13.8 Because they travel at the same speed in a given medium, lowfrequency sounds must have a greater wavelength than high-frequency sounds. Here, the lower-frequency sounds are emitted by the large speaker, called a woofer, while the higher-frequency sounds are emitted by the small speaker, called a tweeter.}
\end{center}
\end{figure}

\section*{Teacher Support}
Teacher Support [BL] For sound, a higher frequency corresponds to a higher pitch while a lower frequency corresponds to a lower pitch. Amplitude corresponds to the loudness of the sound.\\[0pt]
[BL][OL] Since sound at all frequencies has the same speed in air, a change in frequency means a change in wavelength.\\[0pt]
[Figure Support] The same speaker is capable of reproducing both high- and lowfrequency sounds. However, high frequencies have shorter wavelengths and are hence best reproduced by a speaker with a small, hard, and tight cone (tweeter), whereas lower frequencies are best reproduced by a large and soft cone (woofer).

These fundamental relationships hold true for all types of waves. As an example, for water waves, \(v_{\mathrm{w}}\) is the speed of a surface wave; for sound, \(v_{\mathrm{w}}\) is the speed of sound; and for visible light, \(v_{\mathrm{w}}\) is the speed of light. The amplitude \(X\) is completely independent of the speed of propagation \(v_{\mathrm{w}}\) and depends only on the amount of energy in the wave.

\section*{Snap Lab}
Waves in a Bowl In this lab, you will take measurements to determine how the amplitude and the period of waves are affected by the transfer of energy from a cork dropped into the water. The cork initially has some potential\\
energy when it is held above the water-the greater the height, the higher the potential energy. When it is dropped, such potential energy is converted to kinetic energy as the cork falls. When the cork hits the water, that energy travels through the water in waves.

\begin{itemize}
  \item Large bowl or basin
  \item Water
  \item Cork (or ping pong ball)
  \item Stopwatch
  \item Measuring tape
\end{itemize}

Instructions\\
Procedure

\begin{enumerate}
  \item Fill a large bowl or basin with water and wait for the water to settle so there are no ripples.
  \item Gently drop a cork into the middle of the bowl.
  \item Estimate the wavelength and the period of oscillation of the water wave that propagates away from the cork. You can estimate the period by counting the number of ripples from the center to the edge of the bowl while your partner times it. This information, combined with the bowl measurement, will give you the wavelength when the correct formula is used.
  \item Remove the cork from the bowl and wait for the water to settle again.
  \item Gently drop the cork at a height that is different from the first drop.
  \item Repeat Steps 3 to 5 to collect a second and third set of data, dropping the cork from different heights and recording the resulting wavelengths and periods.
  \item Interpret your results.
\end{enumerate}

A cork is dropped into a pool of water creating waves. Does the wavelength depend upon the height above the water from which the cork is dropped?\\
a. No, only the amplitude is affected.\\
b. Yes, the wavelength is affected.

\section*{Teacher Support}
Teacher Support Students can measure the bowl beforehand to help them make a better estimation of the wavelength.

\section*{Links To Physics}
\section*{Geology: Physics of Seismic Waves}
\begin{center}
\includegraphics[max width=\textwidth]{e5042e07-d38c-4a98-ad7f-01ff1992bfa2-16}
\end{center}

Figure 13.9 The destructive effect of an earthquake is a palpable evidence of the energy carried in the earthquake waves. The Richter scale rating of earthquakes is related to both their amplitude and the energy they carry. (Petty Officer 2nd Class Candice Villarreal, U.S. Navy)

Geologists rely heavily on physics to study earthquakes since earthquakes involve several types of wave disturbances, including disturbance of Earth's surface and pressure disturbances under the surface. Surface earthquake waves are similar to surface waves on water. The waves under Earth's surface have both longitudinal and transverse components. The longitudinal waves in an earthquake are called pressure waves ( P -waves) and the transverse waves are called shear waves ( S waves). These two types of waves propagate at different speeds, and the speed at which they travel depends on the rigidity of the medium through which they are traveling. During earthquakes, the speed of P -waves in granite is significantly higher than the speed of S-waves. Both components of earthquakes travel more slowly in less rigid materials, such as sediments. P -waves have speeds of 4 to 7\\
\(\mathrm{km} / \mathrm{s}\), and S -waves have speeds of 2 to \(5 \mathrm{~km} / \mathrm{s}\), but both are faster in more rigid materials. The P -wave gets progressively farther ahead of the S -wave as they travel through Earth's crust. For that reason, the time difference between the P - and S -waves is used to determine the distance to their source, the epicenter of the earthquake.

We know from seismic waves produced by earthquakes that parts of the interior of Earth are liquid. Shear or transverse waves cannot travel through a liquid and are not transmitted through Earth's core. In contrast, compression or longitudinal waves can pass through a liquid and they do go through the core.

All waves carry energy, and the energy of earthquake waves is easy to observe based on the amount of damage left behind after the ground has stopped moving. Earthquakes can shake whole cities to the ground, performing the work of thousands of wrecking balls. The amount of energy in a wave is related to its amplitude. Large-amplitude earthquakes produce large ground displacements and greater damage. As earthquake waves spread out, their amplitude decreases, so there is less damage the farther they get from the source.

\section*{Grasp Check}
What is the relationship between the propagation speed, frequency, and wavelength of the S -waves in an earthquake?\\
a. The relationship between the propagation speed, frequency, and wavelength is \(v_{\mathrm{w}}=f \lambda\).\\
b. The relationship between the propagation speed, frequency, and wavelength is \(v_{\mathrm{w}}=\frac{f}{\lambda}\).\\
c. The relationship between the propagation speed, frequency, and wavelength is \(v_{\mathrm{w}}=\frac{\lambda}{f}\).\\
d. The relationship between the propagation speed, frequency, and wavelength is \(v_{\mathrm{w}}=\sqrt{f \lambda}\).

\section*{Virtual Physics}
Wave on a String Click to view content\\
In this animation, watch how a string vibrates in slow motion by choosing the Slow Motion setting. Select the No End and Manual options, and wiggle the end of the string to make waves yourself. Then switch to the Oscillate setting to generate waves automatically. Adjust the frequency and the amplitude of the oscillations to see what happens. Then experiment with adjusting the damping and the tension.

\section*{Grasp Check}
Which of the settings-amplitude, frequency, damping, or tension-changes the amplitude of the wave as it propagates? What does it do to the amplitude?\\
a. Frequency; it decreases the amplitude of the wave as it propagates.\\
b. Frequency; it increases the amplitude of the wave as it propagates.\\
c. Damping; it decreases the amplitude of the wave as it propagates.\\
d. Damping; it increases the amplitude of the wave as it propagates.

\section*{Solving Wave Problems}
\section*{Worked Example}
Calculate the Velocity of Wave Propagation: Gull in the Ocean Calculate the wave velocity of the ocean wave in the previous figure if the distance between wave crests is 10.0 m and the time for a seagull to bob up and down is 5.00 s .

\section*{Strategy}
The values for the wavelength\\
\((\lambda=10.0 \mathrm{~m})\)\\
and the period ( \(T=5.00 \mathrm{~s}\) ) are given and we are asked to find \(v_{w}\) Therefore, we can use \(v_{w}=\frac{\lambda}{T}\) to find the wave velocity.\\
Solution\\
Enter the known values into \(v_{w}=\frac{\lambda}{T}\)\\
\(v_{w}=\frac{10.0 \mathrm{~m}}{5.00 \mathrm{~s}}=2.00 \mathrm{~m} / \mathrm{s}\).\\
13.5

Discussion\\
This slow speed seems reasonable for an ocean wave. Note that in the figure, the wave moves to the right at this speed, which is different from the varying speed at which the seagull bobs up and down.

\section*{Worked Example}
Calculate the Period and the Wave Velocity of a Toy Spring The woman in Figure 13.3 creates two waves every second by shaking the toy spring up and down. (a)What is the period of each wave? (b) If each wave travels 0.9 meters after one complete wave cycle, what is the velocity of wave propagation?

\section*{Strategy FOR (A)}
To find the period, we solve for \(T=\frac{1}{f}\), given the value of the frequency ( \(f= 2 \mathrm{~s}^{-1}\) ).

Solution for (a)\\
Enter the known value into \(T=\frac{1}{f}\)\\
\(T=\frac{1}{2 \mathrm{~s}^{-1}}=0.5 \mathrm{~s}\).\\
13.6

\section*{Strategy FOR (B)}
Since one definition of wavelength is the distance a wave has traveled after one complete cycle - or one period-the values for the wavelength\\
\((\lambda=0.9 \mathrm{~m})\)\\
as well as the frequency are given. Therefore, we can use \(v_{\mathrm{w}}=f \lambda\) to find the wave velocity.

Solution for (b)\\
Enter the known values into \(v_{\mathrm{w}}=f \lambda\)\\
\(v_{\mathrm{w}}=f \lambda=\left(2 \mathrm{~s}^{-1}\right)(0.9 \mathrm{~m})=1.8 \mathrm{~m} / \mathrm{s}\).\\
Discussion\\
We could have also used the equation \(v_{\mathrm{w}}=\frac{\lambda}{T}\) to solve for the wave velocity since we already know the value of the period ( \(T=0.5 \mathrm{~s}\) ) from our calculation in part (a), and we would come up with the same answer.

\section*{Practice Problems}
7.

The frequency of a wave is 10 Hz . What is its period?\\
a. The period of the wave is 100 s .\\
b. The period of the wave is 10 s .\\
c. The period of the wave is 0.01 s .\\
d. The period of the wave is 0.1 s .\\
8.

What is the velocity of a wave whose wavelength is 2 m and whose frequency is 5 Hz ?\\
a. \(20 \mathrm{~m} / \mathrm{s}\)\\
b. \(2.5 \mathrm{~m} / \mathrm{s}\)\\
c. \(0.4 \mathrm{~m} / \mathrm{s}\)\\
d. \(10 \mathrm{~m} / \mathrm{s}\)

\section*{Check Your Understanding}
\section*{Teacher Support}
Teacher Support Use these questions to assess students' achievement of the section's Learning Objectives. If students are struggling with a specific objective, these questions will help identify such objective and direct them to the relevant content.\\
9.

What is the amplitude of a wave?\\
a. A quarter of the total height of the wave\\
b. Half of the total height of the wave\\
c. Two times the total height of the wave\\
d. Four times the total height of the wave\\
10.

What is meant by the wavelength of a wave?\\
a. The wavelength is the distance between adjacent identical parts of a wave, parallel to the direction of propagation.\\
b. The wavelength is the distance between adjacent identical parts of a wave, perpendicular to the direction of propagation.\\
c. The wavelength is the distance between a crest and the adjacent trough of a wave, parallel to the direction of propagation.\\
d. The wavelength is the distance between a crest and the adjacent trough of a wave, perpendicular to the direction of propagation.\\
11.

How can you mathematically express wave frequency in terms of wave period?\\
a. \(\mathrm{f}=\backslash \operatorname{frac}\{1\}\{\mathrm{T}\}\)\\
b. \(\mathrm{f}=\backslash \operatorname{left}(\backslash \operatorname{frac}\{1\}\{\mathrm{T}\} \backslash \text { right })^{\wedge} 2\)\\
c. \(\backslash \operatorname{text}\{\mathrm{f}\}=\backslash \operatorname{text}\{\mathrm{T}\}\)\\
d. \(\mathrm{f}=\backslash \operatorname{left}(\mathrm{T} \backslash \text { right })^{\wedge} 2\)\\
12.

When is the wavelength directly proportional to the period of a wave?\\
a. When the velocity of the wave is halved\\
b. When the velocity of the wave is constant\\
c. When the velocity of the wave is doubled\\
d. When the velocity of the wave is tripled

\subsection*{13.3 Wave Interaction: Superposition and Interference}
\section*{Section Learning Objectives}
By the end of this section, you will be able to do the following:

\begin{itemize}
  \item Describe superposition of waves
  \item Describe interference of waves and distinguish between constructive and destructive interference of waves
  \item Describe the characteristics of standing waves
  \item Distinguish reflection from refraction of waves
\end{itemize}

\section*{Teacher Support}
Teacher Support The learning objectives in this section will help your students master the following standards:

\begin{itemize}
  \item (7) Science concepts. The student knows the characteristics and behavior of waves. The student is expected to:
  \item (D) investigate the behaviors of waves, including reflection, refraction, diffraction, interference, resonance, and the Doppler effect.
\end{itemize}

In addition, the High School Physics Laboratory Manual addresses content in this section in the lab titled: Waves, as well as the following standards:

\begin{itemize}
  \item (7) Science concepts. The student knows the characteristics and behavior of waves. The student is expected to:
  \item (D) investigate behaviors of waves, including reflection, refraction, diffraction, interference, resonance, and the Doppler effect.
\end{itemize}

\section*{Section Key Terms}
\section*{Teacher Support}
Teacher Support [BL][OL] Review waves, their types, and their properties, as covered in the previous sections.

\section*{Superposition of Waves}
Most waves do not look very simple. They look more like the waves in Figure 13.10, rather than the simple water wave considered in the previous sections, which has a perfect sinusoidal shape.

\begin{figure}[h]
\begin{center}
  \includegraphics[max width=\textwidth]{e5042e07-d38c-4a98-ad7f-01ff1992bfa2-22}
\captionsetup{labelformat=empty}
\caption{Figure 13.10 These waves result from the superposition of several waves from different sources, producing a complex pattern. (Waterborough, Wikimedia Commons)}
\end{center}
\end{figure}

\section*{Teacher Support}
Teacher Support The horizontal waves in the picture bounce off the wall of the lake seen in the front part of the picture. These superimpose or combine with waves moving in a different direction. When they combine, their energies get added, forming higher peaks and lower crests in specific places. This is why the water has a crisscross pattern.

Most waves appear complex because they result from two or more simple waves that combine as they come together at the same place at the same time-a phenomenon called superposition.

Waves superimpose by adding their disturbances; each disturbance corresponds to a force, and all the forces add. If the disturbances are along the same line, then the resulting wave is a simple addition of the disturbances of the individual waves, that is, their amplitudes add.

\section*{Wave Interference}
The two special cases of superposition that produce the simplest results are pure constructive interference and pure destructive interference.

Pure constructive interference occurs when two identical waves arrive at the same point exactly in phase. When waves are exactly in phase, the crests of the two waves are precisely aligned, as are the troughs. Refer to Figure 13.11. Because the disturbances add, the pure constructive interference of two waves with the same amplitude produces a wave that has twice the amplitude of the two individual waves, but has the same wavelength.

\begin{figure}[h]
\begin{center}
  \includegraphics[max width=\textwidth]{e5042e07-d38c-4a98-ad7f-01ff1992bfa2-23(1)}
\captionsetup{labelformat=empty}
\caption{Figure 13.11 The pure constructive interference of two identical waves produces a wave with twice the amplitude but the same wavelength.}
\end{center}
\end{figure}

Figure 13.12 shows two identical waves that arrive exactly out of phase-that is, precisely aligned crest to trough-producing pure destructive interference. Because the disturbances are in opposite directions for this superposition, the resulting amplitude is zero for pure destructive interference; that is, the waves completely cancel out each other.

\begin{figure}[h]
\begin{center}
  \includegraphics[max width=\textwidth]{e5042e07-d38c-4a98-ad7f-01ff1992bfa2-23}
\captionsetup{labelformat=empty}
\caption{Figure 13.12 The pure destructive interference of two identical waves produces zero amplitude, or complete cancellation.}
\end{center}
\end{figure}

While pure constructive interference and pure destructive interference can occur, they are not very common because they require precisely aligned identical waves. The superposition of most waves that we see in nature produces a combination of constructive and destructive interferences.

Waves that are not results of pure constructive or destructive interference can vary from place to place and time to time. The sound from a stereo, for example, can be loud in one spot and soft in another. The varying loudness means that the sound waves add partially constructively and partially destructively at different locations. A stereo has at least two speakers that create sound waves, and waves can reflect from walls. All these waves superimpose.

An example of sounds that vary over time from constructive to destructive is found in the combined whine of jet engines heard by a stationary passenger. The volume of the combined sound can fluctuate up and down as the sound from the two engines varies in time from constructive to destructive.

The two previous examples considered waves that are similar-both stereo speakers generate sound waves with the same amplitude and wavelength, as do the jet engines. But what happens when two waves that are not similar, that is, having different amplitudes and wavelengths, are superimposed? An example of the superposition of two dissimilar waves is shown in Figure 13.13. Here again, the disturbances add and subtract, but they produce an even more complicatedlooking wave. The resultant wave from the combined disturbances of two dissimilar waves looks much different than the idealized sinusoidal shape of a periodic wave.\\
\texttt{https://cdn.mathpix.com/cropped/e5042e07-d38c-4a98-ad7f-01ff1992bfa2-24.jpg?height=125&width=512&top_left_y=1003&top_left_x=457}\\
\includegraphics[max width=\textwidth, center]{e5042e07-d38c-4a98-ad7f-01ff1992bfa2-24}

Wave 2\\
\includegraphics[max width=\textwidth, center]{e5042e07-d38c-4a98-ad7f-01ff1992bfa2-24(1)}

Figure 13.13 The superposition of nonidentical waves exhibits both constructive and destructive interferences.

\section*{Virtual Physics}
Wave Interference Click to view content\\
In this simulation, make waves with a dripping faucet, an audio speaker, or a laser by switching between the water, sound, and light tabs. Contrast and compare how the different types of waves behave. Try rotating the view from top to side to make observations. Then experiment with adding a second source or a pair of slits to create an interference pattern.

PhET Explorations: Wave Interference. Make waves with a dripping faucet, audio speaker, or laser! Add a second source or a pair of slits to create an interference pattern.

Click to view content\\
In the water tab, compare the waves generated by one drip versus two drips. What happens to the amplitude of the waves when there are two drips? Is this constructive or destructive interference? Why would this be the case?\\
a. The amplitude of the water waves remains same because of the destructive interference as the drips of water hit the surface at the same time.\\
b. The amplitude of the water waves is canceled because of the destructive interference as the drips of water hit the surface at the same time.\\
c. The amplitude of water waves remains same because of the constructive interference as the drips of water hit the surface at the same time.\\
d. The amplitude of water waves doubles because of the constructive interference as the drips of water hit the surface at the same time.

\section*{Standing Waves}
Sometimes waves do not seem to move and they appear to just stand in place, vibrating. Such waves are called standing waves and are formed by the superposition of two or more waves moving in opposite directions. The waves move through each other with their disturbances adding as they go by. If the two waves have the same amplitude and wavelength, then they alternate between constructive and destructive interference. Standing waves created by the superposition of two identical waves moving in opposite directions are illustrated in Figure 13.14.

\begin{figure}[h]
\begin{center}
  \includegraphics[max width=\textwidth]{e5042e07-d38c-4a98-ad7f-01ff1992bfa2-25}
\captionsetup{labelformat=empty}
\caption{Figure 13.14 A standing wave is created by the superposition of two identical waves moving in opposite directions. The oscillations are at fixed locations in space and result from alternating constructive and destructive interferences.}
\end{center}
\end{figure}

As an example, standing waves can be seen on the surface of a glass of milk in a refrigerator. The vibrations from the refrigerator motor create waves on the milk that oscillate up and down but do not seem to move across the surface. The two waves that produce standing waves may be due to the reflections from the side of the glass.

Earthquakes can create standing waves and cause constructive and destructive interferences. As the earthquake waves travel along the surface of Earth and reflect off denser rocks, constructive interference occurs at certain points. As a result, areas closer to the epicenter are not damaged while areas farther from the epicenter are damaged.

Standing waves are also found on the strings of musical instruments and are due to reflections of waves from the ends of the string. Figure 13.15 and Figure 13.16 show three standing waves that can be created on a string that is fixed at both ends. When the wave reaches the fixed end, it has nowhere else to go but back where it came from, causing the reflection. The nodes are the points\\
where the string does not move; more generally, the nodes are the points where the wave disturbance is zero in a standing wave. The fixed ends of strings must be nodes, too, because the string cannot move there.

The antinode is the location of maximum amplitude in standing waves. The standing waves on a string have a frequency that is related to the propagation speed \(v_{w}\) of the disturbance on the string. The wavelength \(\lambda\) is determined by the distance between the points where the string is fixed in place.\\
\includegraphics[max width=\textwidth, center]{e5042e07-d38c-4a98-ad7f-01ff1992bfa2-26(1)}

Figure 13.15 The figure shows a string oscillating with its maximum disturbance as the antinode.

\begin{figure}[h]
\begin{center}
  \includegraphics[max width=\textwidth]{e5042e07-d38c-4a98-ad7f-01ff1992bfa2-26}
\captionsetup{labelformat=empty}
\caption{Figure 13.16 The figure shows a string oscillating with multiple nodes.}
\end{center}
\end{figure}

\section*{Reflection and Refraction of Waves}
As we saw in the case of standing waves on the strings of a musical instrument, reflection is the change in direction of a wave when it bounces off a barrier, such as a fixed end. When the wave hits the fixed end, it changes direction, returning to its source. As it is reflected, the wave experiences an inversion, which means that it flips vertically. If a wave hits the fixed end with a crest, it will return as a trough, and vice versa (Henderson 2015). Refer to Figure 13.17.

\begin{figure}[h]
\begin{center}
  \includegraphics[max width=\textwidth]{e5042e07-d38c-4a98-ad7f-01ff1992bfa2-27}
\captionsetup{labelformat=empty}
\caption{Figure 13.17 A wave is inverted after reflection from a fixed end.}
\end{center}
\end{figure}

\section*{Tips For Success}
If the end is not fixed, it is said to be a free end, and no inversion occurs. When the end is loosely attached, it reflects without inversion, and when the end is not attached to anything, it does not reflect at all. You may have noticed this while changing the settings from Fixed End to Loose End to No End in the Waves on a String PhET simulation.

Rather than encountering a fixed end or barrier, waves sometimes pass from one medium into another, for instance, from air into water. Different types of media have different properties, such as density or depth, that affect how a wave travels through them. At the boundary between media, waves experience refraction - they change their path of propagation. As the wave bends, it also changes its speed and wavelength upon entering the new medium. Refer to Figure 13.18.

\begin{figure}[h]
\begin{center}
  \includegraphics[max width=\textwidth]{e5042e07-d38c-4a98-ad7f-01ff1992bfa2-27(1)}
\captionsetup{labelformat=empty}
\caption{Figure 13.18 A wave refracts as it enters a different medium.}
\end{center}
\end{figure}

For example, water waves traveling from the deep end to the shallow end of a swimming pool experience refraction. They bend in a path closer to perpendicular to the surface of the water, propagate slower, and decrease in wavelength as they enter shallower water.

\section*{Check Your Understanding}
\section*{Teacher Support}
Teacher Support Use these questions to assess students' achievement of the section's learning objectives. If students are struggling with a specific objective, these questions will help identify such objective and direct them to the relevant content.\\
13.

What is the superposition of waves?\\
a. When a single wave splits into two different waves at a point\\
b. When two waves combine at the same place at the same time\\
14.

How do waves superimpose on one another?\\
a. By adding their frequencies\\
b. By adding their wavelengths\\
c. By adding their disturbances\\
d. By adding their speeds\\
15.

What is interference of waves?\\
a. Interference is a superposition of two waves to form a resultant wave with higher or lower frequency.\\
b. Interference is a superposition of two waves to form a wave of larger or smaller amplitude.\\
c. Interference is a superposition of two waves to form a resultant wave with higher or lower velocity.\\
d. Interference is a superposition of two waves to form a resultant wave with longer or shorter wavelength.\\
16.

Is the following statement true or false? The two types of interference are constructive and destructive interferences.\\
a. True\\
b. False\\
17.

What are standing waves?\\
a. Waves that appear to remain in one place and do not seem to move\\
b. Waves that seem to move along a trajectory\\
18.

How are standing waves formed?\\
a. Standing waves are formed by the superposition of two or more waves moving in opposite directions.\\
b. Standing waves are formed by the superposition of two or more waves moving in the same direction.\\
c. Standing waves are formed by the superposition of two or more waves moving in perpendicular directions.\\
d. Standing waves are formed by the superposition of two or more waves moving in arbitrary directions.\\
19.

What is the reflection of a wave?\\
a. The reflection of a wave is the change in amplitude of a wave when it bounces off a barrier.\\
b. The reflection of a wave is the change in frequency of a wave when it bounces off a barrier.\\
c. The reflection of a wave is the change in velocity of a wave when it bounces off a barrier.\\
d. The reflection of a wave is the change in direction of a wave when it bounces off a barrier.\\
20.

What is inversion of a wave?\\
a. Inversion occurs when a wave reflects off a fixed end and the wave amplitude changes sign.\\
b. Inversion occurs when a wave reflects off a loose end and the wave amplitude changes sign.\\
c. Inversion occurs when a wave reflects off a fixed end without the wave amplitude changing sign.\\
d. Inversion occurs when a wave reflects off a loose end without the wave amplitude changing sign.

\section*{Key Terms}
antinode location of maximum amplitude in standing waves\\
constructive interference when two waves arrive at the same point exactly in phase; that is, the crests of the two waves are precisely aligned, as are the troughs\\
destructive interference when two identical waves arrive at the same point exactly out of phase that is precisely aligned crest to trough\\
inversion vertical flipping of a wave after reflection from a fixed end\\
longitudinal wave wave in which the disturbance is parallel to the direction of propagation\\
mechanical wave wave that requires a medium through which it can travel\\
medium solid, liquid, or gas material through which a wave propagates\\
nodes points where the string does not move; more generally, points where the wave disturbance is zero in a standing wave\\
periodic wave wave that repeats the same oscillation for several cycles and is associated with simple harmonic motion\\
pulse wave sudden disturbance with only one wave or a few waves generated\\
reflection change in direction of a wave at a boundary or fixed end\\
refraction bending of a wave as it passes from one medium to another medium with a different density\\
standing wave wave made by the superposition of two waves of the same amplitude and wavelength moving in opposite directions and which appears to vibrate in place\\
superposition phenomenon that occurs when two or more waves arrive at the same point\\
transverse wave wave in which the disturbance is perpendicular to the direction of propagation\\
wave disturbance that moves from its source and carries energy\\
wave velocity speed at which the disturbance moves; also called the propagation velocity or propagation speed\\
wavelength distance between adjacent identical parts of a wave

\section*{Key Equations}
13.2 Wave Properties: Speed, Amplitude, Frequency, and Period

\section*{Section Summary}
\subsection*{13.1 Types of Waves}
\begin{itemize}
  \item A wave is a disturbance that moves from the point of creation and carries energy but not mass.
  \item Mechanical waves must travel through a medium.
  \item Sound waves, water waves, and earthquake waves are all examples of mechanical waves.
  \item Light is not a mechanical wave since it can travel through a vacuum.
  \item A periodic wave is a wave that repeats for several cycles, whereas a pulse wave has only one crest or a few crests and is associated with a sudden disturbance.
  \item Periodic waves are associated with simple harmonic motion.
  \item A transverse wave has a disturbance perpendicular to its direction of propagation, whereas a longitudinal wave has a disturbance parallel to its direction of propagation.
\end{itemize}

\subsection*{13.2 Wave Properties: Speed, Amplitude, Frequency, and Period}
\begin{itemize}
  \item A wave is a disturbance that moves from the point of creation at a wave velocity \(v_{\mathrm{w}}\).
  \item A wave has a wavelength \(\lambda\), which is the distance between adjacent identical parts of the wave.
  \item The wave velocity and the wavelength are related to the wave's frequency and period by \(v_{\mathrm{w}}=\frac{\lambda}{T}\) or \(v_{\mathrm{w}}=f \lambda\).
  \item The time for one complete wave cycle is the period \(T\).
  \item The number of waves per unit time is the frequency \(f\).
  \item The wave frequency and the period are inversely related to one another.
\end{itemize}

\subsection*{13.3 Wave Interaction: Superposition and Interference}
\begin{itemize}
  \item Superposition is the combination of two waves at the same location.
  \item Constructive interference occurs when two identical waves are superimposed exactly in phase.
  \item Destructive interference occurs when two identical waves are superimposed exactly out of phase.
  \item A standing wave is a wave produced by the superposition of two waves. It varies in amplitude but does not propagate.
  \item The nodes are the points where there is no motion in standing waves.
  \item An antinode is the location of maximum amplitude of a standing wave.
  \item Reflection causes a wave to change direction.
  \item Inversion occurs when a wave reflects from a fixed end.
  \item Refraction causes a wave's path to bend and occurs when a wave passes from one medium into another medium with a different density.
\end{itemize}

\end{document}