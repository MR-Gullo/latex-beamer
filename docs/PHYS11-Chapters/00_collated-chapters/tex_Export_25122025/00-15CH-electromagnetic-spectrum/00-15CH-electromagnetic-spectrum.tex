\documentclass[10pt]{article}
\usepackage[utf8]{inputenc}
\usepackage[T1]{fontenc}
\usepackage{graphicx}
\usepackage[export]{adjustbox}
\graphicspath{ {./images/} }
\usepackage{caption}
\usepackage{amsmath}
\usepackage{amsfonts}
\usepackage{amssymb}
\usepackage[version=4]{mhchem}
\usepackage{stmaryrd}

\begin{document}
\captionsetup{singlelinecheck=false}
\begin{figure}[h]
\begin{center}
  \includegraphics[max width=\textwidth]{5703778b-3a35-40a5-96b5-75c6410c9ea9-01}
\captionsetup{labelformat=empty}
\caption{Figure 15.1 Human eyes detect these orange sea goldie fish swimming over a coral reef in the blue waters of the Gulf of Eilat, in the Red Sea, using visible light. (credit: David Darom, Wikimedia Commons)}
\end{center}
\end{figure}

\section*{Chapter Outline}
15.1 The Electromagnetic Spectrum\\
15.2 The Behavior of Electromagnetic Radiation

\section*{Introduction}
\section*{Teacher Support}
Teacher Support Review the concepts about waves learned in earlier chapters. Ask students to compare and contrast light waves and sound waves. Dispel any misconceptions about relative speed of light and sound. Ask students to describe the path of light that allows them to see the colors of the fish and the coral. Ask if anyone can define light. Explain that it is just one form of electromagnetic radiation and that the other forms are invisible. Ask them to name other forms of electromagnetic radiation and fill in the ones they miss.

The beauty of a coral reef, the warm radiance of sunshine, the sting of sunburn, the X-ray revealing a broken bone, even microwave popcorn - all are brought to us by electromagnetic waves. The list of the various types of electromagnetic waves, ranging from radio transmission waves to nuclear gamma-ray ( -ray) emissions, is interesting in itself.

Even more intriguing is that all of these different phenomena are manifestations of the same thing-electromagnetic waves (see Figure 15.1). What are electromagnetic waves? How are they created, and how do they travel? How can we\\
understand their widely varying properties? What is the relationship between electric and magnetic effects? These and other questions will be explored.

\section*{Teacher Support}
Teacher Support Before students begin this chapter, it would be useful to review the following concepts:

\begin{itemize}
  \item Significant figures-demonstrate how to obtain the proper number of significant figures when adding and multiplying
  \item Scientific notation and how it expresses significant figures
  \item Converting units-demonstrate how to convert from \(\mathrm{km} / \mathrm{h}\) to \(\mathrm{m} / \mathrm{s}\); review metric length units, including nanometers, meters, and kilometers; and, show how units cancel in calculations
  \item Wave propagation, including wavelength, frequency, and amplitude
\end{itemize}

\subsection*{15.1 The Electromagnetic Spectrum}
\section*{Section Learning Objectives}
By the end of this section, you will be able to do the following:

\begin{itemize}
  \item Define the electromagnetic spectrum, and describe it in terms of frequencies and wavelengths
  \item Describe and explain the differences and similarities of each section of the electromagnetic spectrum and the applications of radiation from those sections
\end{itemize}

\section*{Teacher Support}
Teacher Support The learning objectives in this section will help your students master the following standards

\begin{itemize}
  \item (7) Science concepts. The student knows the characteristics and behavior of waves. The student is expected to:
  \item (A) examine and describe oscillatory motion and wave propagation in various types of media;
  \item (B) investigate and analyze characteristics of waves, including velocity, frequency, amplitude, and wavelength, and calculate using the relationship between wave speed, frequency, and wavelength;
  \item (C) compare characteristics and behaviors of transverse waves, including electromagnetic waves and the electromagnetic spectrum, and characteristics and behaviors of longitudinal waves, including sound waves; and
  \item (F) describe the role of wave characteristics and behaviors in medical and industrial applications.
\end{itemize}

In addition, the High School Physics Laboratory Manual addresses content in this section in the lab titled: Light and Color, as well as the following standards:

\begin{itemize}
  \item (7) Science concepts. The student knows the characteristics and behavior of waves. The student is expected to
  \item (C) compare characteristics and behaviors of transverse waves, including electromagnetic waves and the electromagnetic spectrum, and characteristics and behaviors of longitudinal waves, including sound waves.
  \item (8) Science concepts. The student knows simple examples of atomic, nuclear, and quantum phenomena. The student is expected to
  \item (B) compare and explain the emission spectra produced by various atoms.
\end{itemize}

\section*{Section Key Terms}
\section*{Teacher Support}
Teacher Support [BL]Explain that the term spectrum refers to a physical property that has a broad range with values that are continuous in some cases and, in other cases, discrete. Ask for other examples of spectra, for example, sound, people's heights, etc.\\[0pt]
[OL]Ask students to name ways that sunlight affects Earth. Provide examples that students don't name: photosynthesis, weather, climate, seasons, warming, etc. Discuss energy transformations that take place after light enters the atmosphere, such as transformations in food chains and ecosystems. Ask students if they can explain how the energy in fossil fuels was originally light energy.

\section*{Misconception Alert}
The light we can see is called visible light. Dispel any misconceptions that visible light is somehow different from radiation we cannot see, except for frequency and wavelength. The fact that some radiation is visible has to do with how the eye functions, not with the radiation itself.

\section*{The Electromagnetic Spectrum}
We generally take light for granted, but it is a truly amazing and mysterious form of energy. Think about it: Light travels to Earth across millions of kilometers of empty space. When it reaches us, it interacts with matter in various ways to generate almost all the energy needed to support life, provide heat, and cause weather patterns. Light is a form of electromagnetic radiation (EMR). The term light usually refers to visible light, but this is not the only form of EMR. As we will see, visible light occupies a narrow band in a broad range of types of electromagnetic radiation.

\section*{Teacher Support}
Teacher Support [OL]Discuss electric, magnetic, and gravitational fields. Point out how these three fields are similar, and how they differ.\\[0pt]
[AL]Describe vectors as having magnitude and direction, and explain that fields are vector quantities. In these cases, the fields are made up of forces acting in a direction.

Electromagnetic radiation is generated by a moving electric charge, that is, by an electric current. As you will see when you study electricity, an electric current generates both an electric field, E, and a magnetic field, B. These fields are perpendicular to each other. When the moving charge oscillates, as in an\\
alternating current, an EM wave is propagated. Figure 15.2 shows how an electromagnetic wave moves away from the source-indicated by the \(\sim\) symbol.

\section*{Teacher Support}
Teacher Support [BL]Review wave properties: frequency, wavelength, and amplitude. Ask students to recall sound and water waves, and explain how they relate to these properties.\\[0pt]
[OL]Explain that an important difference between EM waves and other waves is that they can travel across empty space.\\[0pt]
[AL]Ask if students remember the differences between longitudinal and transverse waves. Give examples. Explain that waves carry energy, not matter.

\section*{Watch Physics}
Electromagnetic Waves and the Electromagnetic Spectrum This video, link below, is closely related to the following figure. If you have questions about EM wave properties, the EM spectrum, how waves propagate, or definitions of any of the related terms, the answers can be found in this video.

Click to view content

\section*{Grasp Check}
In an electromagnetic wave, how are the magnetic field, the electric field, and the direction of propagation oriented to each other?\\
a. All three are parallel to each other and are along the \(x\)-axis.\\
b. All three are mutually perpendicular to each other.\\
c. The electric field and magnetic fields are parallel to each other and perpendicular to the direction of propagation.\\
d. The magnetic field and direction of propagation are parallel to each other along the \(y\)-axis and perpendicular to the electric field.

\section*{Teacher Support}
Teacher Support Direct students to use this video as a way of connecting to the information in the following two figures, as well as to the following table.

\section*{Virtual Physics}
Radio Waves and Electromagnetic Fields Click to view content\\
This simulation demonstrates wave propagation. The EM wave is propagated from the broadcast tower on the left, just as in Figure 15.2. You can make the\\
wave yourself or allow the animation to send it. When the wave reaches the antenna on the right, it causes an oscillating current. This is how radio and television signals are transmitted and received.

\section*{Grasp Check}
Where do radio waves fall on the electromagnetic spectrum?\\
a. Radio waves have the same wavelengths as visible light.\\
b. Radio waves fall on the high-frequency side of visible light.\\
c. Radio waves fall on the short-wavelength side of visible light.\\
d. Radio waves fall on the low-frequency side of visible light.

\section*{Teacher Support}
Teacher Support Connect the discussion from the previous video, in which the generation of an electromagnetic wave is described, to this application of transmission and reception of electromagnetic waves. In particular, point out how the reception of the radio wave is essentially identical to the method by which the wave is generated. Explain also that these electromagnetic waves are the carrier waves on which audio or visual signals - either analog or digital-are placed, so that they can be transmitted to receivers within a certain range of the broadcast antenna.

\begin{figure}[h]
\begin{center}
  \includegraphics[max width=\textwidth]{5703778b-3a35-40a5-96b5-75c6410c9ea9-06}
\captionsetup{labelformat=empty}
\caption{Figure 15.2 A part of the electromagnetic wave sent out from an oscillating charge at one instant in time. The electric and magnetic fields ( \(\mathbf{E}\) and \(\mathbf{B}\) ) are in phase, and they are perpendicular to each other and to the direction of propagation. For clarity, the waves are shown only along one direction, but they propagate out in other directions too.}
\end{center}
\end{figure}

From your study of sound waves, recall these features that apply to all types of waves:

\begin{itemize}
  \item Wavelength-The distance between two wave crests or two wave troughs, expressed in various metric measures of distance
  \item Frequency-The number of wave crests that pass a point per second, expressed in hertz ( Hz or \(\mathrm{s}^{-1}\) )
  \item Amplitude: The height of the crest above the null point
\end{itemize}

As mentioned, electromagnetic radiation takes several forms. These forms are characterized by a range of frequencies. Because frequency is inversely proportional to wavelength, any form of EMR can also be represented by its range of wavelengths. Figure 15.3 shows the frequency and wavelength ranges of various types of EMR. With how many of these types are you familiar?

\begin{figure}[h]
\begin{center}
  \includegraphics[max width=\textwidth]{5703778b-3a35-40a5-96b5-75c6410c9ea9-07}
\captionsetup{labelformat=empty}
\caption{Figure 15.3 The electromagnetic spectrum, showing the major categories of electromagnetic waves. The range of frequencies and wavelengths is remarkable. The dividing line between some categories is distinct, whereas other categories overlap.}
\end{center}
\end{figure}

Take a few minutes to study the positions of the various types of radiation on the EM spectrum, above. Sometimes all radiation with frequencies lower than those of visible light are referred to as infrared (IR) radiation. This includes radio waves, which overlap with the frequencies used for media broadcasts of TV and radio signals. The microwave radiation that you see on the diagram is the same radiation that is used in a microwave oven. What we feel as radiant heat is also a form of low-frequency EMR.

\section*{Teacher Support}
Teacher Support [BL]Notice that most harmful forms of EM radiation are on the high-frequency end of the spectrum.\\[0pt]
[OL]Ask which forms of EM radiation students have heard about. Ask them to describe the types of radiation they remember, and correct any misconceptions. Discuss the difference between ionizing radiation and nonionizing radiation, and the difference between electromagnetic radiation and other types of radiationalpha, beta, etc.

\section*{Misconception Alert}
Heat waves, a type of infrared radiation, are basically no different from other EM waves. We feel them as heat because they have a frequency that interacts with our bodies in a way that transforms EM energy into thermal energy.

All the high-frequency radiation to the right of visible light is sometimes referred to as ultraviolet (UV) radiation. This includes X-rays and gamma ( ) rays. The narrow band that is visible light extends from lower-frequency red light to higherfrequency violet light, thus the terms are infrared (below red) and ultraviolet (beyond violet).

\section*{Boundless Physics}
Maxwell's Equations The Scottish physicist James Clerk Maxwell (18311879) is regarded widely to have been the greatest theoretical physicist of the nineteenth century. Although he died young, Maxwell not only formulated a complete electromagnetic theory, represented by Maxwell's equations, he also developed the kinetic theory of gases, and made significant contributions to the understanding of color vision and the nature of Saturn's rings.

Maxwell brought together all the work that had been done by brilliant physicists, such as Ørsted, Coulomb, Ampere, Gauss, and Faraday, and added his own insights to develop the overarching theory of electromagnetism. Maxwell's equations are paraphrased here in words because their mathematical content is beyond the level of this text. However, the equations illustrate how apparently simple mathematical statements can elegantly unite and express a multitude of concepts - why mathematics is the language of science.

Maxwell's Equations

\begin{enumerate}
  \item Electric field lines originate on positive charges and terminate on negative charges. The electric field is defined as the force per unit charge on a test charge, and the strength of the force is related to the electric constant, \(0 \cdot\)
  \item Magnetic field lines are continuous, having no beginning or end. No magnetic monopoles are known to exist. The strength of the magnetic force is related to the magnetic constant, \({ }_{0}\).
  \item A changing magnetic field induces an electromotive force (emf) and, hence, an electric field. The direction of the emf opposes the change, changing direction of the magnetic field.
  \item Magnetic fields are generated by moving charges or by changing electric fields.
\end{enumerate}

Maxwell's complete theory shows that electric and magnetic forces are not separate, but different manifestations of the same thing-the electromagnetic force. This classical unification of forces is one motivation for current attempts to unify the four basic forces in nature - the gravitational, electromagnetic, strong nuclear, and weak nuclear forces. The weak nuclear and electromagnetic forces\\
have been unified, and further unification with the strong nuclear force is expected; but, the unification of the gravitational force with the other three has proven to be a real head-scratcher.

One final accomplishment of Maxwell was his development in 1855 of a process that could produce color photographic images. In 1861, he and photographer Thomas Sutton worked together on this process. The color image was achieved by projecting red, blue, and green light through black-and-white photographs of a tartan ribbon, each photo itself exposed in different-colored light. The final image was projected onto a screen (see Figure 15.4).

\begin{figure}[h]
\begin{center}
  \includegraphics[max width=\textwidth]{5703778b-3a35-40a5-96b5-75c6410c9ea9-09}
\captionsetup{labelformat=empty}
\caption{Figure 15.4 Maxwell and Sutton's photograph of a colored ribbon. This was the first durable color photograph. The plaid tartan of the Scots made a colorful photographic subject.}
\end{center}
\end{figure}

\section*{Teacher Support}
Teacher Support Features that encouraged mathematicians and physicists to accept Maxwell's equations is that they are seen as being both elegant andconsidering the difference between an electric charge and a magnetic dipole, which give rise to the respective fields-essentially symmetrical. When scientists are looking for an approach to developing a new theory, they usually begin with the simplest and most symmetrical explanations. An example of such symmetry is the fact that electrons and protons have equal and opposite charges. You can see the symmetry in the four statements, given above, that describe the equations. The first two statements show a similar treatment of electric and magnetic fields, and the last two describe how a magnetic field can generate an electric field, and vice versa.

From our present-day perspective, we can now see the significance of Maxwell's equations. This was the first step in the quest to unify all natural forces under one theory. After Maxwell unified the electric and magnetic forces as the electromagnetic force, others unified this force with the weak nuclear force, and there is evidence that the strong nuclear force can be unified with the electroweak force. The only force that has resisted unification with the others is the gravitational force. A theory that would unify all forces is often referred as a grand unified theory or a theory of everything. The quest for such a theory is still underway.

Describe electromagnetic force as explained by Maxwell's equations.\\
a. According to Maxwell's equations, electromagnetic force gives rise to electric force and magnetic force.\\
b. According to Maxwell's equations, electric force and magnetic force are different manifestations of electromagnetic force.\\
c. According to Maxwell's equations, electric force is the cause of electromagnetic force.\\
d. According to Maxwell's equations, magnetic force is the cause of electromagnetic force.

\section*{Characteristics of Electromagnetic Radiation}
All the EM waves mentioned above are basically the same form of radiation. They can all travel across empty space, and they all travel at the speed of light in a vacuum. The basic difference between types of radiation is their differing frequencies. Each frequency has an associated wavelength. As frequency increases across the spectrum, wavelength decreases. Energy also increases with frequency. Because of this, higher frequencies penetrate matter more readily. Some of the properties and uses of the various EM spectrum bands are listed in Table 15.1.

\section*{Teacher Support}
Teacher Support [BL]Explain transparency and opacity. Discuss how some materials are transparent to certain frequencies but opaque to others. Ask students for examples of materials that can be penetrated by some EM frequencies but not by others. Ask for examples of materials that are transparent to visible light and materials that are opaque to visible light.\\[0pt]
[OL]Ask students why a lead apron is laid across dental patients during dental X-rays. Explain that X-rays are at the high-energy end of the spectrum and that they are very penetrating. They are only stopped by very dense materials, such as lead.\\[0pt]
[AL]Ask if students can explain Earth's greenhouse effect in terms of the penetrating power of various frequencies of EM radiation. Explain that the atmosphere is more transparent to visible light than to heat waves. Visible light penetrates the atmosphere and warms Earth's surface. The heated surface radiates heat waves, which are trapped partially by certain gases in the atmosphere.

Table 15.1 Electromagnetic Waves This table shows how each type of EM radiation is produced, how it is applied, as well as environmental and health issues associated with it.

The narrow band of visible light is a combination of the colors of the rainbow. Figure 15.5 shows the section of the EM spectrum that includes visible light. The frequencies corresponding to these wavelengths are \(4.0 \times 10^{14} \mathrm{~s}^{-1}\) at the red end to \(7.9 \times 10^{14} \mathrm{~s}^{-1}\) at the violet end. This is a very narrow range, considering that the EM spectrum spans about 20 orders of magnitude.

\begin{figure}[h]
\begin{center}
  \includegraphics[max width=\textwidth]{5703778b-3a35-40a5-96b5-75c6410c9ea9-11}
\captionsetup{labelformat=empty}
\caption{Figure 15.5 A small part of the electromagnetic spectrum that includes its visible}
\end{center}
\end{figure}

components. The divisions between infrared, visible, and ultraviolet are not perfectly distinct, nor are the divisions between the seven rainbow colors

\section*{Teacher Support}
Teacher Support [BL]Review the primary and secondary colors of pigments. Note that this is subtractive color mixing.\\[0pt]
[OL]Explain the difference between subtractive and additive color mixing. The colors on the subtractive color wheel are made by pigments that absorb all colors but one. Therefore, when these colors all overlap, all light is absorbed and the result is black. White light is a combination of all colors, so when all colors are added together on the additive color wheel, the result is white. Explain that cyan is a shade of blue and that magenta is a shade of red.

\section*{Tips For Success}
Wavelengths of visible light are often given in nanometers, nm . One nm equals \(10^{-9} \mathrm{~m}\). For example, yellow light has a wavelength of about 600 nm , or \(6 \times 10^{-7}\) m.

As a child, you probably learned the color wheel, shown on the left in Figure 15.6. It helps if you know what color results when you mix different colors of paint together. Mixing two of the primary pigment colors-magenta, yellow, or cyantogether results in a secondary color. For example, mixing cyan and yellow makes green. This is called subtractive color mixing. Mixing different colors of light together is quite different. The diagram on the right shows additive color mixing. In this case, the primary colors are red, green, and blue, and the secondary colors are cyan, magenta, and yellow. Mixing pigments and mixing light are different because materials absorb light by a different set of rules than does the perception of light by the eye. Notice that, when all colors are subtracted, the result is no color, or black. When all colors are added, the result is white light. We see the reverse of this when white sunlight is separated into the visible spectrum by a prism or by raindrops when a rainbow appears in the sky.

\begin{figure}[h]
\begin{center}
  \includegraphics[max width=\textwidth]{5703778b-3a35-40a5-96b5-75c6410c9ea9-13}
\captionsetup{labelformat=empty}
\caption{Figure 15.6 Mixing colored pigments follows the subtractive color wheel, and mixing colored light follows the additive color wheel.}
\end{center}
\end{figure}

\section*{Virtual Physics}
Color Vision Click to view content\\
This video demonstrates additive color and color filters. Try all the settings except Photons.

PhET Explorations: Color Vision. Make a whole rainbow by mixing red, green, and blue light. Change the wavelength of a monochromatic beam or filter white light. View the light as a solid beam, or see the individual photons.

Click to view content\\
Ordinary white light is a combination of all colors of visible light. How would a blue absorption filter placed in front of a white light source affect the light you observe?\\
a. A blue filter absorbs blue light, causing the observed light to be a combination of the other colors.\\
b. A blue filter absorbs the opposite color of light-orange, causing the observed light to be blue.\\
c. A blue filter permits only blue light to pass though, absorbing the other colors and leaving blue light for the observer.\\
d. A blue filter permits only the opposite color light-orange-to pass through, leaving orange light for the observer.

\section*{Teacher Support}
Teacher Support Have students adjust the different colored lights for the RGB bulb simulation, first with individual settings, then with combinations of two and three colors to see what colors result and are perceived. Similarly, with the Single Bulb simulation, have students note how different filter settings affect what colors are seen for light with different color components.

\section*{Links To Physics}
Animal Color Perception The physics of color perception has interesting links to zoology. Other animals have very different views of the world than humans, especially with respect to which colors can be seen. Color is detected by cells in the eye called cones. Humans have three cones that are sensitive to three different ranges of electromagnetic wavelengths. They are called red, blue, and green cones, although these colors do not correspond exactly to the centers of the three ranges. The ranges of wavelengths that each cone detects are red, 500 to 700 nm ; green, 450 to 630 nm ; and blue, 400 to 500 nm .

Most primates also have three kinds of cones and see the world much as we do. Most mammals other than primates only have two cones and have a less colorful view of things. Dogs, for example see blue and yellow, but are color blind to red and green. You might think that simpler species, such as fish and insects, would have less sophisticated vision, but this is not the case. Many birds, reptiles, amphibians, and insects have four or five different cones in their eyes. These species don't have a wider range of perceived colors, but they see more hues, or combinations of colors. Also, some animals, such as bees or rattlesnakes, see a range of colors that is as broad as ours, but shifted into the ultraviolet or infrared.

These differences in color perception are generally adaptations that help the animals survive. Colorful tropical birds and fish display some colors that are too subtle for us to see. These colors are believed to play a role in the species mating rituals. Figure 15.7 shows the colors visible and the color range of vision in humans, bees, and dogs.\\
\includegraphics[max width=\textwidth, center]{5703778b-3a35-40a5-96b5-75c6410c9ea9-14}

\begin{figure}[h]
\begin{center}
  \includegraphics[max width=\textwidth]{5703778b-3a35-40a5-96b5-75c6410c9ea9-14(1)}
\captionsetup{labelformat=empty}
\caption{Figure 15.7 Humans, bees, and dogs see colors differently. Dogs see fewer colors than humans, and bees see a different range of colors.}
\end{center}
\end{figure}

\section*{Teacher Support}
Teacher Support The symbiotic relationship between plants and their pollinators-bees, birds, etc.-is related to color perception. Plants have evolved to have flowers with colors that bees can see easily, and bees can find those flowers easily to collect the nectar they need for survival.

\section*{Grasp Check}
The belief that bulls are enraged by seeing the color red is a misconception. What did you read in this Links to Physics that shows why this belief is incorrect?\\
a. Bulls are color-blind to every color in the spectrum of colors.\\
b. Bulls are color-blind to the blue colors in the spectrum of colors.\\
c. Bulls are color-blind to the red colors in the spectrum of colors.\\
d. Bulls are color-blind to the green colors in the spectrum of colors.

Humans have found uses for every part of the electromagnetic spectrum. We will take a look at the uses of each range of frequencies, beginning with visible light. Most of our uses of visible light are obvious; without it our interaction with our surroundings would be much different. We might forget that nearly all of our food depends on the photosynthesis process in plants, and that the energy for this process comes from the visible part of the spectrum. Without photosynthesis, we would also have almost no oxygen in the atmosphere.

\section*{Teacher Support}
Teacher Support [BL]Ask how different frequencies of EM radiation are applied. Name each frequency range, and ask the students to supply the application, for example, X-rays used in medical imaging.\\[0pt]
[OL]Ask students if they know why low-frequency radiation generally has different uses than high-frequency radiation. Explain that it has to do with penetrating power, which is related to health hazards.\\[0pt]
[AL]Mention the ranges of TV signals designated very high frequency (VHF) and ultrahigh frequency (UHF). Explain that these frequencies are only relatively high compared to radio broadcast frequencies. Their place in the whole EM spectrum is at the low end.

The low-frequency, infrared region of the spectrum has many applications in media broadcasting. Television, radio, cell phone, and remote-control devices all broadcast and/or receive signals with these wavelengths. AM and FM radio signals are both low-frequency radiation. They are in different regions of the spectrum, but that is not their basic difference. AM and FM are abbreviations for amplitude modulation and frequency modulation. Information in AM signals has the form of changes in amplitude of the radio waves; information in FM signals has the form of changes in wave frequency.

Another application of long-wavelength radiation is found in microwave ovens. These appliances cook or warm food by irradiating it with EM radiation in the microwave frequency range. Most kitchen microwaves use a frequency of \(2.45 \times 10^{9} \mathrm{~Hz}\). These waves have the right amount of energy to cause polar molecules, such as water, to rotate faster. Polar molecules are those that have a partial charge separation. The rotational energy of these molecules is given up to surrounding matter as heat. The first microwave ovens were called Radaranges because they were based on radar technology developed during World War II.

Radar uses radiation with wavelengths similar to those of microwaves to detect the location and speed of distant objects, such as airplanes, weather formations, and motor vehicles. Radar information is obtained by receiving and analyzing the echoes of microwaves reflected by an object. The speed of the object can be measured using the Doppler shift of the returning waves. This is the same effect you learned about when you studied sound waves. Like sound waves, EM waves are shifted to higher frequencies by an object moving toward an observer, and to lower frequencies by an object moving away from the observer. Astronomers use this same Doppler effect to measure the speed at which distant galaxies are moving away from us. In this case, the shift in frequency is called the red shift, because visible frequencies are shifted toward the lower-frequency, red end of the spectrum.

Exposure to any radiation with frequencies greater than those of visible light carries some health hazards. All types of radiation in this range are known to cause cell damage. The danger is related to the high energy and penetrating ability of these EM waves. The likelihood of being harmed by any of this radiation depends largely on the amount of exposure. Most people try to reduce exposure to UV radiation from sunlight by using sunscreen and protective clothing. Physicians still use X-rays to diagnose medical problems, but the intensity of the radiation used is extremely low. Figure 15.8 shows an X-ray image of a patient's chest cavity.

One medical-imaging technique that involves no danger of exposure is magnetic resonance imaging (MRI). MRI is an important imaging and research tool in medicine, producing highly detailed two- and three-dimensional images. Radio waves are broadcast, absorbed, and reemitted in a resonance process that is sensitive to the density of nuclei, usually hydrogen nuclei-protons.

\begin{figure}[h]
\begin{center}
  \includegraphics[max width=\textwidth]{5703778b-3a35-40a5-96b5-75c6410c9ea9-17}
\captionsetup{labelformat=empty}
\caption{Figure 15.8 This shadow X-ray image shows many interesting features, such as artificial heart valves, a pacemaker, and wires used to close the sternum. (credit: P.P. Urone)}
\end{center}
\end{figure}

\section*{Check Your Understanding}
\section*{Teacher Support}
Teacher Support Use these questions to assess student achievement of the section's Learning Objectives. If students are struggling with a specific objective, these questions will help identify any gaps and direct students to the relevant content.\\
1.

Identify the fields produced by a moving charged particle.\\
a. Both an electric field and a magnetic field will be produced.\\
b. Neither a magnetic field nor an electric field will be produced.\\
c. A magnetic field, but no electric field will be produced.\\
d. Only the electric field, but no magnetic field will be produced.\\
2.

X-rays carry more energy than visible light. Compare the frequencies and wavelengths of these two types of EM radiation.\\
a. Visible light has higher frequencies and shorter wavelengths than X-rays.\\
b. Visible light has lower frequencies and shorter wavelengths than X-rays.\\
c. Visible light has higher frequencies and longer wavelengths than X-rays.\\
d. Visible light has lower frequencies and longer wavelengths than X-rays.\\
3.

How does wavelength change as frequency increases across the EM spectrum?\\
a. The wavelength increases.\\
b. The wavelength first increases and then decreases.\\
c. The wavelength first decreases and then increases.\\
d. The wavelength decreases.\\
4.

Why are X-rays used in imaging of broken bones, rather than radio waves?\\
a. X-rays have higher penetrating energy than radio waves.\\
b. X-rays have lower penetrating energy than radio waves.\\
c. X-rays have a lower frequency range than radio waves.\\
d. X-rays have longer wavelengths than radio waves.\\
5.

Identify the fields that make up an electromagnetic wave.\\
a. both an electric field and a magnetic field\\
b. neither a magnetic field nor an electric field\\
c. only a magnetic field, but no electric field\\
d. only an electric field, but no magnetic field

\subsection*{15.2 The Beha ior of Electromagnetic Radiation}
\section*{Section Learning Objectives}
By the end of this section, you will be able to do the following:

\begin{itemize}
  \item Describe the behavior of electromagnetic radiation
  \item Solve quantitative problems involving the behavior of electromagnetic radiation
\end{itemize}

\section*{Teacher Support}
Teacher Support The learning objectives in this section will help your students master the following standards:

\begin{itemize}
  \item (7) Science concepts. The student knows the characteristics and behavior of waves. The student is expected to
  \item (B) investigate and analyze characteristics of waves, including velocity, frequency, amplitude, and wavelength, and calculate using the relationship between wave speed, frequency, and wavelength;
  \item (C) compare characteristics and behaviors of transverse waves, including electromagnetic waves and the electromagnetic spectrum, and characteristics and behaviors of longitudinal waves, including sound waves; and
  \item (D) investigate behaviors of waves, including reflection, refraction, diffraction, interference, resonance, and the Doppler Effect.
\end{itemize}

\section*{Section Key Terms}
\section*{Teacher Support}
Teacher Support [BL]Impress on students how incredibly fast light travels. Start with \(3.00 \times 10^{8} \mathrm{~m} / \mathrm{s}\) and express it as \(300,000 \mathrm{~km} / \mathrm{s}\), or roughly the distance to the moon or the odometer reading on an old, very durable car.\\[0pt]
[OL]Discuss fundamental physical constants. Ask students to name some others in addition to the speed of light. Supply the gravitational constant and Planck's constant. Note that these are different from mathematical constants, such as pi.\\[0pt]
[AL]Explain that physical constants have the same value everywhere in the universe. This realization was an important event in the history of science.

\section*{Misconception Alert}
Both light and sound seem to travel very rapidly from a human perspective. Impress on students that the speed of sound is many orders of magnitude slower than the speed of light.

\section*{Types of Electromagnetic Wave Behavior}
In a vacuum, all electromagnetic radiation travels at the same incredible speed of \(3.00 \times 10^{8} \mathrm{~m} / \mathrm{s}\), which is equal to 671 million miles per hour. This is one of the fundamental physical constants. It is referred to as the speed of light and is given the symbol \(c\). The space between celestial bodies is a near vacuum, so the light we see from the Sun, stars, and other planets has traveled here at the speed of light. Keep in mind that all EM radiation travels at this speed. All the different wavelengths of radiation that leave the Sun make the trip to Earth in the same amount of time. That trip takes 8.3 minutes. Light from the nearest star, besides the Sun, takes 4.2 years to reach Earth, and light from the nearest galaxy-a dwarf galaxy that orbits the Milky Way-travels 25,000 years on its way to Earth. You can see why we call very long distances astronomical.

When light travels through a physical medium, its speed is always less than the speed of light. For example, light travels in water at three-fourths the value of \(c\). In air, light has a speed that is just slightly slower than in empty space: 99.97 percent of \(c\). Diamond slows light down to just 41 percent of c. When light changes speeds at a boundary between media, it also changes direction. The greater the difference in speeds, the more the path of light bends. In other chapters, we look at this bending, called refraction, in greater detail. We introduce refraction here to help explain a phenomenon called thin-film interference.

\section*{Teacher Support}
Teacher Support [BL]Describe the rainbow colors that result from thin-film interference and ask students for examples. Fill in the ones they miss with soap bubbles, oil slicks, and compact discs.\\[0pt]
[OL]Describe refraction qualitatively. Base the discussion on change of speed; there is no need to introduce refractive index yet. Give some examples from everyday experience. For example, explain why objects underwater are not exactly where they appear to be.\\[0pt]
[AL]Explain refraction with the analogy of a four-wheeled vehicle veering from solid pavement into sand. When wheels on one side enter the sand-the slower medium-the vehicle swerves toward the sand. This is because the wheels on that side move more slowly than those on the pavement side.

Have you ever wondered about the rainbow colors you often see on soap bubbles, oil slicks, and compact discs? This occurs when light is both refracted by and reflected from a very thin film. The diagram shows the path of light through\\
such a thin film. The symbols \(n_{1}, n_{2}\), and \(n_{3}\) indicate that light travels at different speeds in each of the three materials. Learn more about this topic in the chapter on diffraction and interference.

Figure 15.9 shows the result of thin film interference on the surface of soap bubbles. Because ray 2 travels a greater distance, the two rays become out of phase. That is, the crests of the two emerging waves are no longer moving together. This causes interference, which reinforces the intensity of the wavelengths of light that create the bands of color. The color bands are separated because each color has a different wavelength. Also, the thickness of the film is not uniform, and different thicknesses cause colors of different wavelengths to interfere in different places. Note that the film must be very, very thin-somewhere in the vicinity of the wavelengths of visible light.

\begin{figure}[h]
\begin{center}
  \includegraphics[max width=\textwidth]{5703778b-3a35-40a5-96b5-75c6410c9ea9-21}
\captionsetup{labelformat=empty}
\caption{Figure 15.9 Light striking a thin film is partially reflected (ray 1) and partially refracted at the top surface. The refracted ray is partially reflected at the bottom surface and emerges as ray 2 . These rays will interfere in a way that depends on the thickness of the film and the indices of refraction of the various media.}
\end{center}
\end{figure}

\section*{Teacher Support}
\section*{Teacher Support}
\section*{Misconception Alert}
Do not confuse polar molecules with polarized light. If a molecule is polar, it refers to a separation of negative and positive electric charges. Polarized light\\
is light whose electric field component vibrates in a specific plane.\\
You have probably experienced how polarized sunglasses reduce glare from the surface of water or snow. The effect is caused by the wave nature of light. Looking back at Figure 15.2, we see that the electric field moves in only one direction perpendicular to the direction of propagation. Light from most sources vibrates in all directions perpendicular to propagation. Light with an electric field that vibrates in only one direction is called polarized. A diagram of polarized light would look like Figure 15.2.

Polarized glasses are an example of a polarizing filter. These glasses absorb most of the horizontal light waves and transmit the vertical waves. This cuts down glare, which is caused by horizontal waves. Figure 15.10 shows how waves traveling along a rope can be used as a model of how a polarizing filter works. The oscillations in one rope are in a vertical plane and are said to be vertically polarized. Those in the other rope are in a horizontal plane and are horizontally polarized. If a vertical slit is placed on the first rope, the waves pass through. However, a vertical slit blocks the horizontally polarized waves. For EM waves, the direction of the electric field oscillation is analogous to the disturbances on the ropes.

\begin{figure}[h]
\begin{center}
  \includegraphics[max width=\textwidth]{5703778b-3a35-40a5-96b5-75c6410c9ea9-22}
\captionsetup{labelformat=empty}
\caption{Figure 15.10 The transverse oscillations in one rope are in a vertical plane, and those in the other rope are in a horizontal plane. The first is said to be vertically polarized, and the other is said to be horizontally polarized. Vertical slits pass vertically polarized waves and block horizontally polarized waves.}
\end{center}
\end{figure}

Light can also be polarized by reflection. Most of the light reflected from water, glass, or any highly reflective surface is polarized horizontally. Figure 15.11 shows the effect of a polarizing lens on light reflected from the surface of water.

\begin{figure}[h]
\begin{center}
  \includegraphics[max width=\textwidth]{5703778b-3a35-40a5-96b5-75c6410c9ea9-23}
\captionsetup{labelformat=empty}
\caption{Figure 15.11 These two photographs of a river show the effect of a polarizing filter in reducing glare in light reflected from the surface of water. Part (b) of this figure was taken with a polarizing filter and part (a) was taken without. As a result, the reflection of clouds and sky observed in part (a) is not observed in part (b). Polarizing sunglasses are particularly useful on snow and water.}
\end{center}
\end{figure}

\section*{Watch Physics}
Polarization of Light, Linear and Circular This video explains the polarization of light in great detail. Before viewing the video, look back at the drawing of an electromagnetic wave from the previous section. Try to visualize the two-dimensional drawing in three dimensions.

Click to view content

\section*{Grasp Check}
How do polarized glasses reduce glare reflected from the ocean?\\
a. They block horizontally polarized and vertically polarized light.\\
b. They are transparent to horizontally polarized and vertically polarized light.\\
c. They block horizontally polarized rays and are transparent to vertically polarized rays.\\
d. They are transparent to horizontally polarized light and block vertically polarized light.

\section*{Teacher Support}
Teacher Support The video explains how to generate linear polarized light, and how, by uniformly changing the phase of linear polarized light, circularly polarized light can be produced.

\section*{Snap Lab}
\section*{Polarized Glasses}
\begin{itemize}
  \item EYE SAFETY-Looking at the Sun directly can cause permanent eye damage. Avoid looking directly at the Sun.
  \item two pairs of polarized sunglasses OR
  \item two lenses from one pair of polarized sunglasses
\end{itemize}

Procedure

\begin{enumerate}
  \item Look through both or either polarized lens and record your observations.
  \item Hold the lenses, one in front of the other. Hold one lens stationary while you slowly rotate the other lens. Record your observations, including the relative angles of the lenses when you make each observation.
  \item Find a reflective surface on which the Sun is shining. It could be water, glass, a mirror, or any other similar smooth surface. The results will be more dramatic if the sunlight strikes the surface at a sharp angle.
  \item Observe the appearance of the surface with your naked eye and through one of the polarized lenses.
  \item Observe any changes as you slowly rotate the lens, and note the angles at which you see changes.
\end{enumerate}

\section*{Teacher Support}
Teacher Support Explain your observations made through the polarized lenses. Describe what happened when you held the lenses together and rotated one of them. Explain the changes during rotation in terms of planes of polarization. Explain your observation of the reflective surface in terms of planes of polarization.

\section*{Grasp Check}
If you buy sunglasses in a store, how can you be sure that they are polarized?\\
a. When one pair of sunglasses is placed in front of another and rotated in the plane of the body, the light passing through the sunglasses will be blocked at two positions due to refraction of light.\\
b. When one pair of sunglasses is placed in front of another and rotated in the plane of the body, the light passing through the sunglasses will be blocked at two positions due to reflection of light.\\
c. When one pair of sunglasses is placed in front of another and rotated in the plane of the body, the light passing through the sunglasses will be blocked at two positions due to the polarization of light.\\
d. When one pair of sunglasses is placed in front of another and rotated in the plane of the body, the light passing through the sunglasses will be blocked at two positions due to the bending of light waves.

\section*{Quantitative Treatment of Electromagnetic Waves}
\section*{Teacher Support}
Teacher Support [BL]Remember that the equation for speed is \(v=d / t\), where \(v\) is velocity or speed, \(d\) is distance, and \(t\) is time. Rearrange the equation so that it can be used to solve for either distance or time if the other two values are known.\\[0pt]
[OL]Ask students to recall ****the metric units of frequency and wavelength. The answer is \(\mathrm{s}^{-1}\), or Hz for frequency and meters or any other metric distance unit for wavelength.

\section*{Misconception Alert}
A light year is a measure of distance, not time.\\
We can use the speed of light, \(c\), to carry out several simple but interesting calculations. If we know the distance to a celestial object, we can calculate how long it takes its light to reach us. Of course, we can also make the reverse calculation if we know the time it takes for the light to travel to us. For an object at a very great distance from Earth, it takes many years for its light to reach us. This means that we are looking at the object as it existed in the distant past. The object may, in fact, no longer exist. Very large distances in the universe are measured in light years. One light year is the distance that light travels in one year, which is \(9.46 \times 10^{12}\) kilometers or \(5.88 \times 10^{12}\) miles (...and \(10^{12}\) is a trillion!).

A useful equation involving \(c\) is\\
\(c=f \lambda\)\\
15.1\\
where \(f\) is frequency in Hz , and \(\lambda\) is wavelength in meters.

\section*{Worked Example}
Frequency and Wavelength Calculation For example, you can calculate the frequency of yellow light with a wavelength of \(6.00 \times 10^{-7} \mathrm{~m}\).

\section*{Strategy}
Rearrange the equation to solve for frequency.\\
\(f=\frac{c}{\lambda}\)\\
15.2

Solution\\
Substitute the values for the speed of light and wavelength into the equation.\\
\(f=\frac{3.00 \times 10^{8} \mathrm{~m} / \mathrm{s}}{6.00 \times 10^{-7} \mathrm{~m}}=5.00 \times 10^{14} \mathrm{~s}^{-1}=5.00 \times 10^{14} \mathrm{~Hz}\)\\
15.3

Discussion\\
Manipulating exponents of 10 in a fraction can be tricky. Be sure you keep track of the + and - exponents correctly. Checking back to the diagram of the electromagnetic spectrum in the previous section shows that \(10^{14}\) is a reasonable order of magnitude for the frequency of yellow light.

The frequency of a wave is proportional to the energy the wave carries. The actual proportionality constant will be discussed in a later chapter. Since frequency is inversely proportional to wavelength, we also know that wavelength is inversely proportional to energy. Keep these relationships in mind as general rules.

\section*{Teacher Support}
Teacher Support [BL]Good lighting for reading is important. We will see that moving twice as far away from a light source does not give you half as much light. It decreases much more than that.\\[0pt]
[AL]Recall the inverse-square law for gravitational force. Any force or radiation that spreads out in all directions will decrease with the square of the distance.

The rate at which light is radiated from a source is called luminous flux, \(P\), and it is measured in lumens (lm). Energy-saving light bulbs, which provide more luminous flux for a given use of electricity, are now available. One of these bulbs is called a compact fluorescent lamp; another is an LED (light-emitting diode) bulb. If you wanted to replace an old incandescent bulb with an energy saving bulb, you would want the new bulb to have the same brightness as the old one. To compare bulbs accurately, you would need to compare the lumens each one puts out. Comparing wattage - that is, the electric power used - would be misleading. Both wattage and lumens are stated on the packaging.

The luminous flux of a bulb might be \(2,000 \mathrm{~lm}\). That accounts for all the light radiated in all directions. However, what we really need to know is how much light falls on an object, such as a book, at a specific distance. The number of lumens per square meter is called illuminance, and is given in units of lux (lx). Picture a light bulb in the middle of a sphere with a \(1-\mathrm{m}\) radius. The total surface of the sphere equals \(4 r^{2} \mathrm{~m}^{2}\). The illuminance then is given by\\
illuminance \(=\frac{P}{4 \pi r^{2}}\).\\
15.4

What happens if the radius of the sphere is increased 2 m ? The illuminance is now only one-fourth as great, because the \(r^{2}\) term in the denominator is 4\\
instead of 1. Figure 15.12 shows how illuminance decreases with the inverse square of the distance.\\
\includegraphics[max width=\textwidth, center]{5703778b-3a35-40a5-96b5-75c6410c9ea9-27}

Figure 15.12 The diagram shows why the illuminance varies inversely with the square of the distance from a source of light.

\section*{Worked Example}
Calculating Illuminance A woman puts a new bulb in a floor lamp beside an easy chair. If the luminous flux of the bulb is rated at \(2,000 \mathrm{~lm}\), what is the illuminance on a book held 2.00 m from the bulb?

\section*{Strategy}
Choose the equation and list the knowns.\\
Equation: illuminance \(=\frac{P}{4 \pi r^{2}}\)\\
\(P=2,000 \mathrm{~lm}\)\\
\(=3.14\)\\
\(r=2.00 \mathrm{~m}\)\\
Solution\\
Substitute the known values into the equation.

\[
\begin{aligned}
\text { illuminance } & =\frac{P}{4 \pi r^{2}} \\
& =\frac{2,000 \mathrm{~lm}}{4(3.14)(2.00)^{2} \mathrm{~m}^{2}} \\
& =39.8 \mathrm{~lm} / \mathrm{m}^{2} \\
& =39.8 \mathrm{~lx}
\end{aligned}
\]

Discussion

Try some other distances to illustrate how greatly light fades with distance from its source. For example, at 3 m the illuminance is only 17.7 lux. Parents often scold children for reading in light that is too dim. Instead of shouting, "You'll ruin your eyes!" it might be better to explain the inverse square law of illuminance to the child.

\section*{Practice Problems}
6.

Red light has a wavelength of \(7.0 \times 10-7 \mathrm{~m}\) and a frequency of \(4.3 \times 1014 \mathrm{~Hz}\). Use these values to calculate the speed of light in a vacuum.\\
a. \(3 \times 1020 \mathrm{~m} / \mathrm{s}\)\\
b. \(3 \times 1015 \mathrm{~m} / \mathrm{s}\)\\
c. \(3 \times 1014 \mathrm{~m} / \mathrm{s}\)\\
d. \(3 \times 108 \mathrm{~m} / \mathrm{s}\)\\
7.

A light bulb has a luminous flux of 942 lumens. What is the illuminance on a surface \(3.00 \backslash, \backslash \operatorname{text}\{\mathrm{~m}\}\) from the bulb when it is lit?\\
a. \(33.32 \backslash, \backslash \operatorname{text}\{\mathrm{~lx}\}\)\\
b. \(26.15 \backslash, \backslash \operatorname{text}\{\mathrm{~lx}\}\)\\
c. \(2.77 \backslash, \backslash \operatorname{text}\{\mathrm{~lx}\}\)\\
d. \(8.33 \backslash, \backslash \operatorname{text}\{\mathrm{~lx}\}\)

\section*{Check Your Understanding}
\section*{Teacher Support}
Teacher Support Use these questions to assess student achievement of the section's Learning Objectives. If students are struggling with a specific objective, these questions will help identify any gaps and direct students to the relevant content.\\
8.

Give an example of a place where light travels at the speed of \(3.00 \times 10^{8} \mathrm{~m} / \mathrm{s}\).\\
a. outer space\\
b. water\\
c. Earth's atmosphere\\
d. quartz glass\\
9.

Explain in terms of distances and the speed of light why it is currently very unlikely that humans will visit planets that circle stars other than our Sun.\\
a. The spacecrafts used for travel are very heavy and thus very slow.\\
b. Spacecrafts do not have a constant source of energy to run them.\\
c. If a spacecraft could attain a maximum speed equal to that of light, it would still be too slow to cover astronomical distances.\\
d. Spacecrafts can attain a maximum speed equal to that of light, but it is difficult to locate planets around stars.

\section*{Ke Terms}
electric field a field that tells us the force per unit charge at all locations in space around a charge distribution\\
electromagnetic radiation (EMR) radiant energy that consists of oscillating electric and magnetic fields\\
illuminance number of lumens per square meter, given in units of lux (lx)\\
interference increased or decreased light intensity caused by the phase differences between waves\\
lumens unit of measure for luminous flux\\
luminous flux rate at which light is radiated from a source\\
lux unit of measure for illuminance\\
magnetic field the directional lines around a magnetic material that indicates the direction and magnitude of the magnetic force

Maxwell's equations equations that describe the interrelationship between electric and magnetic fields, and how these fields combine to form electromagnetic radiation\\
polarized light light whose electric field component vibrates in a specific plane

\section*{Ke Equations}
15.2 The Behavior of Electromagnetic Radiation

\section*{Section Summar}
\subsection*{15.1 The Electromagnetic Spectrum}
\begin{itemize}
  \item The electromagnetic spectrum is made up of a broad range of frequencies of electromagnetic radiation.
  \item All frequencies of EM radiation travel at the same speed in a vacuum and consist of an electric field and a magnetic field. The types of EM radiation have different frequencies and wavelengths, and different energies and penetrating ability.
\end{itemize}

\subsection*{15.2 The Behavior of Electromagnetic Radiation}
\begin{itemize}
  \item EM radiation travels at different speeds in different media, produces colors on thin films, and can be polarized to oscillate in only one direction.
  \item Calculations can be based on the relationship among the speed, frequency, and wavelength of light, and on the relationship among luminous flux, illuminance, and distance.
\end{itemize}

\end{document}