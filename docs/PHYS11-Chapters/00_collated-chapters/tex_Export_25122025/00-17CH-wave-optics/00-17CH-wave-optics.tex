\documentclass[10pt]{article}
\usepackage[utf8]{inputenc}
\usepackage[T1]{fontenc}
\usepackage{graphicx}
\usepackage[export]{adjustbox}
\graphicspath{ {./images/} }
\usepackage{caption}
\usepackage{amsmath}
\usepackage{amsfonts}
\usepackage{amssymb}
\usepackage[version=4]{mhchem}
\usepackage{stmaryrd}

\begin{document}
\captionsetup{singlelinecheck=false}
\begin{figure}[h]
\begin{center}
  \includegraphics[max width=\textwidth]{2964a1e9-ab41-4e16-802f-4d831a0a3895-01}
\captionsetup{labelformat=empty}
\caption{Figure 17.1 The colors reflected by this compact disc vary with angle and are not caused by pigments. Colors such as these are direct evidence of the wave character of light. (credit: Reggie Mathalone)}
\end{center}
\end{figure}

\section*{Chapter Outline}
\subsection*{17.1 Understanding Diffraction and Interference}
17.2 Applications of Diffraction, Interference, and Coherence

\section*{Introduction}
\section*{Teacher Support}
Teacher Support Review the concepts related to waves and wave propagation. Remind students that, in the previous chapter, they learned about the ray aspect of light and the phenomena that can be explained in terms of rays. Those included reflection and refraction and their related applications, such as curved mirrors and lenses. Explain that, in this chapter, they will be looking at light behavior that is explained by light as waves. That will include diffraction and interference, as demonstrated by light when it passes through narrow slits. Finally, they will look at applications related to waves, including lasers, spectroscopes, and the resolution of optical instruments.\\[0pt]
[BL]Recall that the speed of light is a fundamental constant. It is also the fastest possible speed.\\[0pt]
[OL]Review the similarities and differences of light waves, sound waves, and water waves. Discuss relative speeds, wavelengths, methods of propagation, longitudinal vs. transverse waves, and media.\\[0pt]
[AL]Ask students to recall any wave behavior they have witnessed on the surfaces of bodies of water. Do they understand how those waves propagate? Have\\
they seen two wavefronts interfering? Can they describe patterns of crests and troughs?

\section*{Misconception Alert}
Ask students to compare the speeds of light, sound, and water waves. Correct any large misconceptions about relative speeds. Light travels about one million times faster than sound, which travels about one hundred times faster than water waves.

Examine a compact disc under white light, noting the colors observed and their locations on the disc. Using the CD, explore the spectra of a few light sources, such as a candle flame, an incandescent bulb, and fluorescent light. If you have ever looked at the reds, blues, and greens in a sunlit soap bubble and wondered how straw-colored soapy water could produce them, you have hit upon one of the many phenomena that can only be explained by the wave character of light. That and other interesting phenomena, such as the dispersion of white light into a rainbow of colors when passed through a narrow slit, cannot be explained fully by geometric optics. In such cases, light interacts with small objects and exhibits its wave characteristics. The topic of this chapter is the branch of optics that considers the behavior of light when it exhibits wave characteristics.

\section*{Teacher Support}
Teacher Support Before students begin this chapter, it is useful to review the following concepts:

\begin{itemize}
  \item Significant figures-demonstrate how to obtain the proper number of significant figures when adding and multiplying.
  \item Scientific notation and how it expresses significant figures
  \item Converting units-demonstrate how to convert from \(\mathrm{km} / \mathrm{h}\) to \(\mathrm{m} / \mathrm{s}\). Review metric length units, including nanometers, meters, and kilometers. Show how units cancel in calculations.
  \item Wave propagation, including wavelength, frequency, and amplitude
  \item Refractive index
  \item Trigonometric functions and inverse functions
\end{itemize}

\subsection*{17.1 Understanding Diffraction and Interference}
\section*{Section Learning Objectives}
By the end of this section, you will be able to do the following:

\begin{itemize}
  \item Explain wave behavior of light, including diffraction and interference, including the role of constructive and destructive interference in Young's single-slit and double-slit experiments
  \item Perform calculations involving diffraction and interference, in particular the wavelength of light using data from a two-slit interference pattern
\end{itemize}

\section*{Teacher Support}
Teacher Support The learning objectives in this section will help your students master the following standards:

\begin{itemize}
  \item (7) Science concepts. The student knows the characteristics and behavior of waves. The student is expected to:
  \item (D) investigate behaviors of waves, including reflection, refraction, diffraction, interference, resonance, and the Doppler effect
\end{itemize}

\section*{Section Key Terms}
\section*{Diffraction and Interference}
\section*{Teacher Support}
Teacher Support [BL]Explain constructive and destructive interference graphically on the board.\\[0pt]
[OL]Ask students to look closely at a shadow. Ask why the edges are not sharp lines. Explain that this is caused by diffraction, one of the wave properties of electromagnetic radiation. Define the nanometer in relation to other metric length measurements.\\[0pt]
[AL]Ask students which, among speed, frequency, and wavelength, stay the same, and which change, when a ray of light travels from one medium to another. Discuss those quantities in terms of colors (wavelengths) of visible light.

We know that visible light is the type of electromagnetic wave to which our eyes responds. As we have seen previously, light obeys the equation\\
\(c=f \lambda\),\\
where \(c=3.00 \times 10^{8} \mathrm{~m} / \mathrm{s}\) is the speed of light in vacuum, \(f\) is the frequency of the electromagnetic wave in Hz (or \(\mathrm{s}^{-1}\) ), and \(\lambda\) is its wavelength in m . The\\
range of visible wavelengths is approximately 380 to 750 nm . As is true for all waves, light travels in straight lines and acts like a ray when it interacts with objects several times as large as its wavelength. However, when it interacts with smaller objects, it displays its wave characteristics prominently. Interference is the identifying behavior of a wave.

In Figure 17.2, both the ray and wave characteristics of light can be seen. The laser beam emitted by the observatory represents ray behavior, as it travels in a straight line. Passing a pure, one-wavelength beam through vertical slits with a width close to the wavelength of the beam reveals the wave character of light. Here we see the beam spreading out horizontally into a pattern of bright and dark regions that are caused by systematic constructive and destructive interference. As it is characteristic of wave behavior, interference is observed for water waves, sound waves, and light waves.

\begin{figure}[h]
\begin{center}
  \includegraphics[max width=\textwidth]{2964a1e9-ab41-4e16-802f-4d831a0a3895-04}
\captionsetup{labelformat=empty}
\caption{Figure 17.2 (a) The light beam emitted by a laser at the Paranal Observatory (part of the European Southern Observatory in Chile) acts like a ray, traveling in a straight line. (credit: Yuri Beletsky, European Southern Observatory) (b) A laser beam passing through a grid of vertical slits produces an interference pattern-characteristic of a wave. (credit: Shim'on and Slava Rybka, Wikimedia Commons)}
\end{center}
\end{figure}

That interference is a characteristic of energy propagation by waves is demonstrated more convincingly by water waves. Figure 17.3 shows water waves passing through gaps between some rocks. You can easily see that the gaps are similar in width to the wavelength of the waves and that this causes an interference pattern as the waves pass beyond the gaps. A cross-section across the waves in the foreground would show the crests and troughs characteristic of an interference pattern.

\begin{figure}[h]
\begin{center}
  \includegraphics[max width=\textwidth]{2964a1e9-ab41-4e16-802f-4d831a0a3895-05}
\captionsetup{labelformat=empty}
\caption{Figure 17.3 Incoming waves (at the top of the picture) pass through the gaps in the rocks and create an interference pattern (in the foreground).}
\end{center}
\end{figure}

Light has wave characteristics in various media as well as in a vacuum. When light goes from a vacuum to some medium, such as water, its speed and wavelength change, but its frequency, \(f\), remains the same. The speed of light in a medium is \(v=c / n\), where \(n\) is its index of refraction. If you divide both sides of the equation \(c=f \lambda\) by \(n\), you get \(c / n=v=f \lambda / n\). Therefore, \(v=f \lambda_{n}\), where \(\lambda_{n}\) is the wavelength in a medium, and

\[
\lambda_{n}=\frac{\lambda}{n},
\]

where \(\lambda\) is the wavelength in vacuum and \(n\) is the medium's index of refraction. It follows that the wavelength of light is smaller in any medium than it is in vacuum. In water, for example, which has \(n=1.333\), the range of visible wavelengths is \((380 \mathrm{~nm}) / 1.333\) to \((760 \mathrm{~nm}) / 1.333\), or \(\lambda_{n}=285-570 \mathrm{~nm}\). Although wavelengths change while traveling from one medium to another, colors do not, since colors are associated with frequency.

The Dutch scientist Christiaan Huygens (1629-1695) developed a useful technique for determining in detail how and where waves propagate. He used wavefronts, which are the points on a wave's surface that share the same, constant phase (such as all the points that make up the crest of a water wave). Huygens's principle states, "Every point on a wavefront is a source of wavelets that spread out in the forward direction at the same speed as the wave itself. The new wavefront is a line tangent to all of the wavelets."

Figure 17.4 shows how Huygens's principle is applied. A wavefront is the long edge that moves; for example, the crest or the trough. Each point on the wavefront emits a semicircular wave that moves at the propagation speed \(v\). These are drawn later at a time, \(t\), so that they have moved a distance \(s= v t\). The new wavefront is a line tangent to the wavelets and is where the wave is located at time \(t\). Huygens's principle works for all types of waves, including water waves, sound waves, and light waves. It will be useful not only in describing how light waves propagate, but also in how they interfere.

\begin{figure}[h]
\begin{center}
  \includegraphics[max width=\textwidth]{2964a1e9-ab41-4e16-802f-4d831a0a3895-06}
\captionsetup{labelformat=empty}
\caption{Figure 17.4 Huygens's principle applied to a straight wavefront. Each point on the wavefront emits a semicircular wavelet that moves a distance \(s=v t\). The new wavefront is a line tangent to the wavelets.}
\end{center}
\end{figure}

What happens when a wave passes through an opening, such as light shining through an open door into a dark room? For light, you expect to see a sharp shadow of the doorway on the floor of the room, and you expect no light to bend around corners into other parts of the room. When sound passes through a door, you hear it everywhere in the room and, thus, you understand that sound spreads out when passing through such an opening. What is the difference between the behavior of sound waves and light waves in this case? The answer is that the wavelengths that make up the light are very short, so that the light acts like a ray. Sound has wavelengths on the order of the size of the door, and so it bends around corners.

\section*{Teacher Support}
Teacher Support [OL]Discuss the fact that, for a diffraction pattern to be visible, the width of a slit must be roughly the wavelength of the light. Try to give students an idea of the size of visible light wavelengths by noting that a human hair is roughly 100 times wider.

If light passes through smaller openings, often called slits, you can use Huygens's principle to show that light bends as sound does (see Figure 17.5). The bending of a wave around the edges of an opening or an obstacle is called diffraction. Diffraction is a wave characteristic that occurs for all types of waves. If diffraction is observed for a phenomenon, it is evidence that the phenomenon is produced by waves. Thus, the horizontal diffraction of the laser beam after it passes through slits in Figure 17.2 is evidence that light has the properties of a wave.

\begin{figure}[h]
\begin{center}
  \includegraphics[max width=\textwidth]{2964a1e9-ab41-4e16-802f-4d831a0a3895-07(1)}
\captionsetup{labelformat=empty}
\caption{Figure 17.5 Huygens's principle applied to a straight wavefront striking an opening. The edges of the wavefront bend after passing through the opening, a process called diffraction. The amount of bending is more extreme for a small opening, consistent with the fact that wave characteristics are most noticeable for interactions with objects about the same size as the wavelength.}
\end{center}
\end{figure}

Once again, water waves present a familiar example of a wave phenomenon that is easy to observe and understand, as shown in Figure 17.6.

\begin{figure}[h]
\begin{center}
  \includegraphics[max width=\textwidth]{2964a1e9-ab41-4e16-802f-4d831a0a3895-07}
\captionsetup{labelformat=empty}
\caption{Figure 17.6 Ocean waves pass through an opening in a reef, resulting in a diffraction pattern. Diffraction occurs because the opening is similar in width to the wavelength of the waves.}
\end{center}
\end{figure}

\section*{Watch Physics}
Single-Slit Interference This video works through the math needed to predict diffraction patterns that are caused by single-slit interference.

Click to view content\\
Which values of \(m\) denote the location of destructive interference in a single-slit\\
diffraction pattern?\\
a. whole integers, excluding zero\\
b. whole integers\\
c. real numbers excluding zero\\
d. real numbers

The fact that Huygens's principle worked was not considered enough evidence to prove that light is a wave. People were also reluctant to accept light's wave nature because it contradicted the ideas of Isaac Newton, who was still held in high esteem. The acceptance of the wave character of light came after 1801, when the English physicist and physician Thomas Young (1773-1829) did his now-classic double-slit experiment (see Figure 17.7).

\begin{figure}[h]
\begin{center}
  \includegraphics[max width=\textwidth]{2964a1e9-ab41-4e16-802f-4d831a0a3895-08}
\captionsetup{labelformat=empty}
\caption{Figure 17.7 Young's double-slit experiment. Here, light of a single wavelength passes through a pair of vertical slits and produces a diffraction pattern on the screen-numerous vertical light and dark lines that are spread out horizontally. Without diffraction and interference, the light would simply make two lines on the screen.}
\end{center}
\end{figure}

When light passes through narrow slits, it is diffracted into semicircular waves, as shown in Figure 17.8 (a). Pure constructive interference occurs where the waves line up crest to crest or trough to trough. Pure destructive interference occurs where they line up crest to trough. The light must fall on a screen and be scattered into our eyes for the pattern to be visible. An analogous pattern for water waves is shown in Figure 17.8 (b). Note that regions of constructive and destructive interference move out from the slits at well-defined angles to the original beam. Those angles depend on wavelength and the distance between the slits, as you will see below.

\begin{figure}[h]
\begin{center}
  \includegraphics[max width=\textwidth]{2964a1e9-ab41-4e16-802f-4d831a0a3895-09}
\captionsetup{labelformat=empty}
\caption{Figure 17.8 Double slits produce two sources of waves that interfere. (a) Light spreads out (diffracts) from each slit, because the slits are narrow. The waves overlap and interfere constructively (bright lines) and destructively (dark regions). You can only see the effect if the light falls onto a screen and is scattered into your eyes. (b) The double-slit interference pattern for water waves is nearly identical to that for light. Wave action is greatest in regions of constructive interference and least in regions of destructive interference. (c) When light that has passed through double slits falls on a screen, we see a pattern such as this.}
\end{center}
\end{figure}

\section*{Virtual Physics}
Wave Interference Click to view content\\
This simulation demonstrates most of the wave phenomena discussed in this section. First, observe interference between two sources of electromagnetic radiation without adding slits. See how water waves, sound, and light all show interference patterns. Stay with light waves and use only one source. Create diffraction patterns with one slit and then with two. You may have to adjust slit width to see the pattern.

Visually compare the slit width to the wavelength. When do you get the bestdefined diffraction pattern?\\
a. when the slit width is larger than the wavelength\\
b. when the slit width is smaller than the wavelength\\
c. when the slit width is comparable to the wavelength\\
d. when the slit width is infinite

\section*{Calculations Involving Diffraction and Interference}
\section*{Teacher Support}
Teacher Support [BL]The Greek letter \(\theta\) is spelled theta. The Greek letter \(\lambda\) is spelled lamda. Both are pronounced the way you would expect from the spelling. The plurals of maximum and minimum are maxima and minima, respectively.\\[0pt]
[OL]Explain that monochromatic means one color. Monochromatic also means one frequency. The sine of an angle is the opposite side of a right triangle divided by the hypotenuse. Opposite means opposite the given acute angle. Note that the sign of an angle is always 1 .

The fact that the wavelength of light of one color, or monochromatic light, can be calculated from its two-slit diffraction pattern in Young's experiments supports the conclusion that light has wave properties. To understand the basis of such calculations, consider how two waves travel from the slits to the screen. Each slit is a different distance from a given point on the screen. Thus different numbers of wavelengths fit into each path. Waves start out from the slits in phase (crest to crest), but they will end up out of phase (crest to trough) at the screen if the paths differ in length by half a wavelength, interfering destructively. If the paths differ by a whole wavelength, then the waves arrive in phase (crest to crest) at the screen, interfering constructively. More generally, if the paths taken by the two waves differ by any half-integral number of wavelengths \(\left(\frac{1}{2} \lambda, \frac{3}{2} \lambda, \frac{5}{2} \lambda\right.\), etc.), then destructive interference occurs. Similarly, if the paths taken by the two waves differ by any integral number of wavelengths ( \(\lambda, 2 \lambda, 3 \lambda\), etc.), then constructive interference occurs.

Figure 17.9 shows how to determine the path-length difference for waves traveling from two slits to a common point on a screen. If the screen is a large distance away compared with the distance between the slits, then the angle \(\theta\) between the path and a line from the slits perpendicular to the screen (see the figure) is nearly the same for each path. That approximation and simple trigonometry show the length difference, \(\Delta L\), to be \(d \sin \theta\), where \(d\) is the distance between the slits,\\
\(\Delta L=d \sin \theta\).\\
To obtain constructive interference for a double slit, the path-length difference must be an integral multiple of the wavelength, or\\
\(d \sin \theta=m \lambda\), for \(m=0,1,-1,2,-2, \ldots\) (constructive).\\
Similarly, to obtain destructive interference for a double slit, the path-length difference must be a half-integral multiple of the wavelength, or\\
\(d \sin \theta=\left(m+\frac{1}{2}\right) \lambda\), for \(m=0,1,-1,2,-2, \ldots\) (destructive).\\
The number \(m\) is the order of the interference. For example, \(m=4\) is fourthorder interference.

\begin{figure}[h]
\begin{center}
  \includegraphics[max width=\textwidth]{2964a1e9-ab41-4e16-802f-4d831a0a3895-11}
\captionsetup{labelformat=empty}
\caption{Figure 17.9 The paths from each slit to a common point on the screen differ by an amount \(d \sin \theta\), assuming the distance to the screen is much greater than the distance between the slits (not to scale here).}
\end{center}
\end{figure}

Figure 17.10 shows how the intensity of the bands of constructive interference decreases with increasing angle.

\begin{figure}[h]
\begin{center}
  \includegraphics[max width=\textwidth]{2964a1e9-ab41-4e16-802f-4d831a0a3895-11(1)}
\captionsetup{labelformat=empty}
\caption{Figure 17.10 The interference pattern for a double slit has an intensity that falls off with angle. The photograph shows multiple bright and dark lines, or fringes, formed by light passing through a double slit.}
\end{center}
\end{figure}

Light passing through a single slit forms a diffraction pattern somewhat different from that formed by double slits. Figure 17.11 shows a single-slit diffraction pattern. Note that the central maximum is larger than those on either side, and that the intensity decreases rapidly on either side.

\begin{figure}[h]
\begin{center}
  \includegraphics[max width=\textwidth]{2964a1e9-ab41-4e16-802f-4d831a0a3895-12}
\captionsetup{labelformat=empty}
\caption{Figure 17.11 (a) Single-slit diffraction pattern. Monochromatic light passing through a single slit produces a central maximum and many smaller and dimmer maxima on either side. The central maximum is six times higher than shown. (b) The drawing shows the bright central maximum and dimmer and thinner maxima on either side. (c) The location of the minima are shown in terms of \(\lambda\) and \(D\).}
\end{center}
\end{figure}

The analysis of single-slit diffraction is illustrated in Figure 17.12. Assuming the screen is very far away compared with the size of the slit, rays heading toward a common destination are nearly parallel. That approximation allows a series of trigonometric operations that result in the equations for the minima produced by destructive interference.\\
\(D \sin \theta=m \lambda\)\\
or\\
\(\frac{D y}{L}=m \lambda\)\\
When rays travel straight ahead, they remain in phase and a central maximum is obtained. However, when rays travel at an angle \(\theta\) relative to the original direction of the beam, each ray travels a different distance to the screen, and they can arrive in or out of phase. Thus, a ray from the center travels a distance \(\lambda / 2\) farther than the ray from the top edge of the slit, they arrive out of phase, and they interfere destructively. Similarly, for every ray between the top and the center of the slit, there is a ray between the center and the bottom of the slit that travels a distance \(\lambda / 2\) farther to the common point on the screen, and so interferes destructively. Symmetrically, there will be another minimum at the same angle below the direct ray.

\begin{figure}[h]
\begin{center}
  \includegraphics[max width=\textwidth]{2964a1e9-ab41-4e16-802f-4d831a0a3895-13}
\captionsetup{labelformat=empty}
\caption{Figure 17.12 Equations for a single-slit diffraction pattern, where is the wavelength of light, \(D\) is the slit width, \(\theta\) is the angle between a line from the slit to a minimum and a line perpendicular to the screen, \(L\) is the distance from the slit to the screen, \(y\) is the distance from the center of the pattern to the minimum, and \(m\) is a nonzero integer indicating the order of the minimum.}
\end{center}
\end{figure}

Below we summarize the equations needed for the calculations to follow.\\
The speed of light in a vacuum, \(c\), the wavelength of the light, \(\lambda\), and its frequency, \(f\), are related as follows.\\
\(c=f \lambda\)\\
The wavelength of light in a medium, \(\lambda_{n}\), compared to its wavelength in a vacuum, \(\lambda\), is given by\\
\(\lambda_{n}=\frac{\lambda}{n}\).\\
17.1

To calculate the positions of constructive interference for a double slit, the pathlength difference must be an integral multiple, \(m\), of the wavelength. \(\lambda\)\\
\(d \sin \theta=m \lambda\), for \(m=0,1,-1,2,-2, \ldots\) (constructive),\\
where \(d\) is the distance between the slits and \(\theta\) is the angle between a line from the slits to the maximum and a line perpendicular to the barrier in which the slits are located. To calculate the positions of destructive interference for a double slit, the path-length difference must be a half-integral multiple of the wavelength:\\
\(d \sin \theta=\left(m+\frac{1}{2}\right) \lambda\), for \(m=0,1,-1,2,-2, \ldots\) (destructive).\\
For a single-slit diffraction pattern, the width of the slit, \(D\), the distance of the first \((m=1)\) destructive interference minimum, \(y\), the distance from the slit to the screen, \(L\), and the wavelength, \(\lambda\), are given by\\
\(\frac{D y}{L}=\lambda\).\\
Also, for single-slit diffraction,\\
\(D \sin \theta=m \lambda\),\\
where \(\theta\) is the angle between a line from the slit to the minimum and a line perpendicular to the screen, and \(m\) is the order of the minimum.

\section*{Worked Example}
Two-Slit Interference Suppose you pass light from a He-Ne laser through two slits separated by 0.0100 mm , and you find that the third bright line on a screen is formed at an angle of \(10.95^{\circ}\) relative to the incident beam. What is the wavelength of the light?

\section*{Strategy}
The third bright line is due to third-order constructive interference, which means that \(m=3\). You are given \(d=0.0100 \mathrm{~mm}\) and \(\theta=10.95^{\mathrm{o}}\). The wavelength can thus be found using the equation \(d \sin \theta=m \lambda\) for constructive interference.

Solution\\
The equation is \(d \sin \theta=m \lambda\). Solving for the wavelength, \(\lambda\), gives\\
\(\lambda=\frac{d \sin \theta}{m}\).\\
17.2

Substituting known values yields\\
\(\lambda=\frac{(0.0100 \mathrm{~mm})\left(\sin 10.95^{\circ}\right)}{3}=6.33 \times 10^{-4} \mathrm{~mm}=633 \mathrm{~nm}\).\\
17.3

Discussion\\
To three digits, 633 nm is the wavelength of light emitted by the common He-Ne laser. Not by coincidence, this red color is similar to that emitted by neon lights. More important, however, is the fact that interference patterns can be used to measure wavelength. Young did that for visible wavelengths. His analytical technique is still widely used to measure electromagnetic spectra. For a given order, the angle for constructive interference increases with \(\lambda\), so spectra (measurements of intensity versus wavelength) can be obtained.

\section*{Worked Example}
Single-Slit Diffraction Visible light of wavelength 550 nm falls on a single slit and produces its second diffraction minimum at an angle of \(45.0^{\circ}\) relative to the incident direction of the light. What is the width of the slit?

\section*{Strategy}
From the given information, and assuming the screen is far away from the slit, you can use the equation \(D \sin \theta=m \lambda\) to find \(D\).

Solution\\
Quantities given are \(\lambda=550 \mathrm{~nm}, m=2\), and \(\theta_{2}=45.0^{\circ}\). Solving the equation \(D \sin \theta=m \lambda\) for \(D\) and substituting known values gives\\
\(D=\frac{m \lambda}{\sin \theta}=\frac{2(550 \mathrm{~nm})}{\sin 45.0^{\circ}}=1.56 \times 10^{-6} \mathrm{~m}\).\\
17.4

Discussion\\
You see that the slit is narrow (it is only a few times greater than the wavelength of light). That is consistent with the fact that light must interact with an object comparable in size to its wavelength in order to exhibit significant wave effects, such as this single-slit diffraction pattern.

\section*{Practice Problems}
1.

Monochromatic light from a laser passes through two slits separated by \(0.00500 \backslash, \backslash\) text \(\{\mathrm{mm}\}\). The third bright line on a screen is formed at an angle of \(18.0^{\wedge}\{\backslash\) circ \(\}\) relative to the incident beam. What is the wavelength of the light?\\
a. \(51.5 \backslash, \backslash \operatorname{text}\{\mathrm{~nm}\}\)\\
b. \(77.3 \backslash, \backslash \operatorname{text}\{\mathrm{~nm}\}\)\\
c. \(515 \backslash, \backslash \operatorname{text}\{\mathrm{~nm}\}\)\\
d. \(773 \backslash, \backslash \operatorname{text}\{\mathrm{~nm}\}\)\\
2.

What is the width of a single slit through which \(610-\mathrm{nm}\) orange light passes to form a first diffraction minimum at an angle of \(30.0^{\circ}\) ?\\
a. \(0.863 \mu \mathrm{~m}\)\\
b. \(0.704 \mu \mathrm{~m}\)\\
c. \(0.610 \mu \mathrm{~m}\)\\
d. \(1.22 \mu \mathrm{~m}\)

\section*{Check Your Understanding}
\section*{Teacher Support}
Teacher Support Use these problems to assess student achievement of the section's learning objectives. If students are struggling with a specific objective, these problems will help identify which and direct students to the relevant topics.\\
3.

Which aspect of a beam of monochromatic light changes when it passes from a vacuum into water, and how does it change?\\
a. The wavelength first decreases and then increases.\\
b. The wavelength first increases and then decreases.\\
c. The wavelength increases.\\
d. The wavelength decreases.\\
4.

Go outside in the sunlight and observe your shadow. It has fuzzy edges, even if you do not. Is this a diffraction effect? Explain.\\
a. This is a diffraction effect. Your whole body acts as the origin for a new wavefront.\\
b. This is a diffraction effect. Every point on the edge of your shadow acts as the origin for a new wavefront.\\
c. This is a refraction effect. Your whole body acts as the origin for a new wavefront.\\
d. This is a refraction effect. Every point on the edge of your shadow acts as the origin for a new wavefront.\\
5.

Which aspect of monochromatic green light changes when it passes from a vacuum into diamond, and how does it change?\\
a. The wavelength first decreases and then increases.\\
b. The wavelength first increases and then decreases.\\
c. The wavelength increases.\\
d. The wavelength decreases.

\subsection*{17.2 Applications of Diffraction, Interference, and Coherence}
\section*{Section Learning Objectives}
By the end of this section, you will be able to do the following:

\begin{itemize}
  \item Explain behaviors of waves, including reflection, refraction, diffraction, interference, and coherence, and describe applications based on these behaviors
  \item Perform calculations related to applications based on wave properties of light
\end{itemize}

\section*{Teacher Support}
Teacher Support The learning objectives in this section will help your students master the following standards:

\begin{itemize}
  \item (7) Science concepts. The student knows the characteristics and behavior of waves. The student is expected to:
  \item (D) investigate behaviors of waves, including reflection, refraction, diffraction, interference, resonance, and the Doppler effect and
  \item (F) describe the role of wave characteristics and behaviors in medical and industrial applications.
\end{itemize}

\section*{Section Key Terms}
\section*{Wave-Based Applications of Light}
\section*{Teacher Support}
Teacher Support [BL]Define the terms laser, diffraction grating, and resolution.\\[0pt]
[OL]Remind students of the meaning of coherent light in terms of wave properties. Ask students to name examples of lasers and diffraction gratings. Suggest common ones they miss, such as lasers used in surgery, as pointers, for reading CDs, and diffraction gratings on the surface of CDs, iridescent minerals, backs of beetles, and in spectroscopes.\\[0pt]
[AL]Explain how, for very short wavelengths \(\left(\lambda<10^{-12} \mathrm{~m}\right)\), the limit of resolution is related to Heisenberg's uncertainty principle.

\section*{Misconception Alert}
Perfect resolution is impossible. There will always be some blurring of images, no matter what the size of the aperture or the wavelength of light used to make an image.

In 1917, Albert Einstein was thinking about photons and excited atoms. He considered an atom excited by a certain amount of energy and what would happen if that atom were hit by a photon with the same amount of energy. He suggested that the atom would emit a photon with that amount of energy, and it would be accompanied by the original photon. The exciting part is that you would have two photons with the same energy and they would be in phase. Those photons could go on to hit other excited atoms, and soon you would have a stream of in-phase photons. Such a light stream is said to be coherent. Some four decades later, Einstein's idea found application in a process called, light amplification by stimulated emission of radiation. Take the first letters of all the words (except by and "of") and write them in order. You get the word laser (see Figure 17.2 (a)), which is the name of the device that produces such a beam of light.

Laser beams are directional, very intense, and narrow (only about 0.5 mm in diameter). These properties lead to a number of applications in industry and medicine. The following are just a few examples:

\begin{itemize}
  \item This chapter began with a picture of a compact disc (see Figure 17.1). Those audio and data-storage devices began replacing cassette tapes during the 1990s. CDs are read by interpreting variations in reflections of a laser beam from the surface.
  \item Some barcode scanners use a laser beam.
  \item Lasers are used in industry to cut steel and other metals.
  \item Lasers are bounced off reflectors that astronauts left on the Moon. The time it takes for the light to make the round trip can be used to make precise calculations of the Earth-Moon distance.
  \item Laser beams are used to produce holograms. The name hologram means entire picture (from the Greek holo-, as in holistic), because the image is three-dimensional. A viewer can move around the image and see it from different perspectives. Holograms take advantage of the wave properties of light, as opposed to traditional photography which is based on geometric optics. A holographic image is produced by constructive and destructive interference of a split laser beam.
  \item One of the advantages of using a laser as a surgical tool is that it is accompanied by very little bleeding.
  \item Laser eye surgery has improved the vision of many people, without the need for corrective lenses. A laser beam is used to change the shape of the lens of the eye, thus changing its focal length.
\end{itemize}

\section*{Virtual Physics}
Lasers Click to view content\\
This animation allows you to examine the workings of a laser. First view the picture of a real laser. Change the energy of the incoming photons, and see if you can match it to an excitation level that will produce pairs of coherent photons. Change the excitation level and try to match it to the incoming photon energy.

In the animation there is only one excited atom. Is that the case for a real laser? Explain.\\
a. No, a laser would have two excited atoms.\\
b. No, a laser would have several million excited atoms.\\
c. Yes, a laser would have only one excited atom.\\
d. No, a laser would have on the order of \(10^{23}\) excited atoms.

An interesting thing happens if you pass light through a large number of evenlyspaced parallel slits. Such an arrangement of slits is called a diffraction grating. An interference pattern is created that is very similar to the one formed by double-slit diffraction (see Figure 17.8 and Figure 17.9). A diffraction grating can be manufactured by scratching glass with a sharp tool to form a number of precisely positioned parallel lines, which act like slits. Diffraction gratings work both for transmission of light, as in Figure 17.13, and for reflection of light, as on the butterfly wings or the Australian opal shown in Figure 17.14, or the CD pictured in the opening illustration of this chapter. In addition to their use as novelty items, diffraction gratings are commonly used for spectroscopic dispersion and analysis of light. What makes them particularly useful is the fact that they form a sharper pattern than do double slits. That is, their bright regions are narrower and brighter, while their dark regions are darker. Figure 17.15 shows idealized graphs demonstrating the sharper pattern. Natural diffraction gratings occur in the feathers of certain birds. Tiny, fingerlike structures in regular patterns act as reflection gratings, producing constructive interference that gives the feathers colors not solely due to their pigmentation. The effect is called iridescence.

\begin{figure}[h]
\begin{center}
  \includegraphics[max width=\textwidth]{2964a1e9-ab41-4e16-802f-4d831a0a3895-20}
\captionsetup{labelformat=empty}
\caption{Figure 17.13 A diffraction grating consists of a large number of evenly-spaced parallel slits. (a) Light passing through the grating is diffracted in a pattern similar to a double slit, with bright regions at various angles. (b) The pattern obtained for white light incident on a grating. The central maximum is white, and the higher-order maxima disperse white light into a rainbow of colors.}
\end{center}
\end{figure}

\begin{figure}[h]
\begin{center}
  \includegraphics[max width=\textwidth]{2964a1e9-ab41-4e16-802f-4d831a0a3895-20(1)}
\captionsetup{labelformat=empty}
\caption{Figure 17.14 (a) This Australian opal and (b) the butterfly wings have rows of reflectors that act like reflection gratings, reflecting different colors at different angles. (credit: (a) Opals-On-Black.com, via Flickr (b) whologwhy, Flickr)}
\end{center}
\end{figure}

\begin{figure}[h]
\begin{center}
  \includegraphics[max width=\textwidth]{2964a1e9-ab41-4e16-802f-4d831a0a3895-21}
\captionsetup{labelformat=empty}
\caption{Figure 17.15 Idealized graphs of the intensity of light passing through a double slit (a) and a diffraction grating (b) for monochromatic light. Maxima can be produced at the same angles, but those for the diffraction grating are narrower, and hence sharper. The maxima become narrower and the regions between become darker as the number of slits is increased.}
\end{center}
\end{figure}

\section*{Snap Lab}
\section*{Diffraction Grating}
\begin{itemize}
  \item A CD (compact disc) or DVD
  \item A measuring tape
  \item Sunlight near a white wall
\end{itemize}

Instructions\\
Procedure

\begin{enumerate}
  \item Hold the CD in direct sunlight near the wall, and move it around until a circular rainbow pattern appears on the wall.
  \item Measure the distance from the CD to the wall and the distance from the center of the circular pattern to a color in the rainbow. Use those two distances to calculate \(\tan \theta\). Find \(\sin \theta\).
  \item Look up the wavelength of the color you chose. That is \(\lambda\).
  \item Solve \(d \sin \theta=m \lambda\) for \(d\).
  \item Compare your answer to the usual spacing between CD tracks, which is \(1,600 \mathrm{~nm}(1.6 \mathrm{~m})\).
\end{enumerate}

\begin{itemize}
  \item How do you know what number to use for \(m\) ?\\
a. Count the rainbow rings preceding the chosen color.\\
b. Calculate mfrom the frequency of the light of the chosen color.\\
c. Calculate \(m\) from the wavelength of the light of the chosen color.\\
d. The value of \(m\) is fixed for every color.
\end{itemize}

\section*{Fun In Physics}
CD Players Can you see the grooves on a CD or DVD (see Figure 17.16)? You may think you can because you know they are there, but they are extremely narrow- 1,600 in a millimeter. Because the width of the grooves is similar to wavelengths of visible light, they form a diffraction grating. That is why you see rainbows on a CD. The colors are attractive, but they are incidental to the functions of storing and retrieving audio and other data.

\begin{figure}[h]
\begin{center}
  \includegraphics[max width=\textwidth]{2964a1e9-ab41-4e16-802f-4d831a0a3895-22}
\captionsetup{labelformat=empty}
\caption{Figure 17.16 For its size, this CD holds a surprising amount of information. Likewise, the CD player it is in houses a surprising number of electronic devices.}
\end{center}
\end{figure}

The grooves are actually one continuous groove that spirals outward from the center. Data are recorded in the grooves as binary code (zeroes and ones) in small pits. Information in the pits is detected by a laser that tracks along the groove. It gets even more complicated: The speed of rotation must be varied as the laser tracks toward the circumference so that the linear speed along the groove remains constant. There is also an error correction mechanism to prevent the laser beam from getting off track. A diffraction grating is used to create the first two maxima on either side of the track. If those maxima are not the same distance from the track, an error is indicated and then corrected.

The pits are reflective because they have been coated with a thin layer of aluminum. That allows the laser beam to be reflected back and directed toward\\
a photodiode detector. The signal can then be processed and converted to the audio we hear.

The longest wavelength of visible light is about 780 nm . How does that compare to the distance between CD grooves?\\
a. The grooves are about 3 times the longest wavelength of visible light.\\
b. The grooves are about 2 times the longest wavelength of visible light.\\
c. The grooves are about 2 times the shortest wavelength of visible light.\\
d. The grooves are about 3 times the shortest wavelength of visible light.

\section*{Links To Physics}
Biology: DIC Microscopy If you were completely transparent, it would be hard to recognize you from your photograph. The same problem arises when using a traditional microscope to view or photograph small transparent objects such as cells and microbes. Microscopes using differential interference contrast (DIC) solve the problem by making it possible to view microscopic objects with enhanced contrast, as shown in Figure 17.17.\\
\includegraphics[max width=\textwidth, center]{2964a1e9-ab41-4e16-802f-4d831a0a3895-24}

Figure 17.17 This aquatic organism was photographed with a DIC microscope. (credit: Public Library of Science)

A DIC microscope separates a polarized light source into two beams polarized at right angles to each other and coherent with each other, that is, in phase. After passing through the sample, the beams are recombined and realigned so they have the same plane of polarization. They then create an interference pattern caused by the differences in their optical path and the refractive indices of the parts of the sample they passed through. The result is an image with contrast and shadowing that could not be observed with traditional optics.

Where are diffraction gratings used? Diffraction gratings are key components of monochromators-devices that separate the various wavelengths of incoming light and allow a beam with only a specific wavelength to pass through. Monochromators are used, for example, in optical imaging of particular wavelengths from biological or medical samples. A diffraction grating can be chosen to specifically analyze a wavelength of light emitted by molecules in diseased cells in a biopsy sample, or to help excite strategic molecules in the sample with a selected frequency of light. Another important use is in optical fiber technologies where fibers are designed to provide optimum performance at specific wavelengths. A range of diffraction gratings is available for selecting specific wavelengths for such use.

Diffraction gratings are used in spectroscopes to separate a light source into its component wavelengths. When a material is heated to incandescence, it gives off wavelengths of light characteristic of the chemical makeup of the material. A pure substance will produce a spectrum that is unique, thus allowing identification of the substance. Spectroscopes are also used to measure wavelengths both shorter and longer than visible light. Such instruments have become especially useful to astronomers and chemists. Figure 17.18 shows a diagram of a spectroscope.

\begin{figure}[h]
\begin{center}
  \includegraphics[max width=\textwidth]{2964a1e9-ab41-4e16-802f-4d831a0a3895-25}
\captionsetup{labelformat=empty}
\caption{Figure 17.18 The diagram shows the function of a diffraction grating in a spectroscope.}
\end{center}
\end{figure}

Light diffracts as it moves through space, bending around obstacles and interfering constructively and destructively. While diffraction allows light to be used as a spectroscopic tool, it also limits the detail we can obtain in images.

Why are diffraction gratings used in spectroscopes rather than just two slits?\\
a. The bands produced by diffraction gratings are dimmer but sharper than the bands produced by two slits.\\
b. The bands produced by diffraction gratings are brighter, though less sharp, than the bands produced by two slits.\\
c. The bands produced by diffraction gratings are brighter and sharper than the bands produced by two slits.\\
d. The bands produced by diffraction gratings are dimmer and less sharp, but more widely dispersed, than the bands produced by two slits.

Figure 17.19 (a) shows the effect of passing light through a small circular aperture. Instead of a bright spot with sharp edges, a spot with a fuzzy edge surrounded by circles of light is obtained. This pattern is caused by diffraction similar to that produced by a single slit. Light from different parts of the circular aperture interferes constructively and destructively. The effect is most noticeable when the aperture is small, but the effect is there for large apertures, too.

\begin{figure}[h]
\begin{center}
  \includegraphics[max width=\textwidth]{2964a1e9-ab41-4e16-802f-4d831a0a3895-26}
\captionsetup{labelformat=empty}
\caption{Figure 17.19 (a) Monochromatic light passed through a small circular aperture produces this diffraction pattern. (b) Two point light sources that are close to one another produce overlapping images because of diffraction. (c) If they are closer together, they cannot be resolved, that is, distinguished.}
\end{center}
\end{figure}

How does diffraction affect the detail that can be observed when light passes through an aperture? Figure 17.19 (b) shows the diffraction pattern produced by two point light sources that are close to one another. The pattern is similar to that for a single point source, and it is just barely possible to tell that there are two light sources rather than one. If they are closer together, as in Figure 17.19 (c), you cannot distinguish them, thus limiting the detail, or resolution, you can obtain. That limit is an inescapable consequence of the wave nature of light.

There are many situations in which diffraction limits the resolution. The acuity of vision is limited because light passes through the pupil, the circular aperture of the eye. Be aware that the diffraction-like spreading of light is due to the limited diameter of a light beam, not the interaction with an aperture. Thus light passing through a lens with a diameter of \(D\) shows the diffraction effect and spreads, blurring the image, just as light passing through an aperture of diameter \(D\) does. Diffraction limits the resolution of any system having a lens or mirror. Telescopes are also limited by diffraction, because of the finite diameter, \(D\), of their primary mirror.

\section*{Calculations Involving Diffraction Gratings and Resolution}
\section*{Teacher Support}
Teacher Support [BL]Review problem-solving techniques: Identify the knowns and unknowns. Convert measurements of the same property to the same units of measurements. Choose the equation and rearrange it, if necessary, to solve for the unknown.\\[0pt]
[OL]Review the meaning of arcsine in particular and of inverse trigonometric functions in general. Explain the radian as a unit of measure for angles, and relate it to degrees.

Early in the chapter, it was mentioned that when light passes from one medium to another, its speed and wavelength change, but its frequency remains constant. The equation\\
\(\lambda_{n}=\frac{\lambda}{n}\)\\
shows how to the wavelength in a given medium, \(\lambda_{n}\), is related to the wavelength in a vacuum, \(\lambda\), and the refractive index, \(n\), of the medium. The equation is useful for calculating the change in wavelength of a monochromatic laser beam in various media. The analysis of a diffraction grating is very similar to that for a double slit. As you know from the discussion of double slits in Young's doubleslit experiment, light is diffracted by, and spreads out after passing through, each slit. Rays travel at an angle \(\theta\) relative to the incident direction. Each ray travels a different distance to a common point on a screen far away. The rays start in phase, and they can be in or out of phase when they reach a screen, depending on the difference in the path lengths traveled. Each ray travels a distance that differs by \(d \sin \theta\) from that of its neighbor, where \(d\) is the distance between slits. If \(d \sin \theta\) equals an integral number of wavelengths, the rays all arrive in phase, and constructive interference (a maximum) is obtained. Thus, the condition necessary to obtain constructive interference for a diffraction grating is\\
\(d \sin \theta=m \lambda\), for \(m=0,1,-1,2,-2, \ldots\),\\
where \(d\) is the distance between slits in the grating, \(\lambda\) is the wavelength of the light, and \(m\) is the order of the maximum. Note that this is exactly the same equation as for two slits separated by \(d\). However, the slits are usually closer in diffraction gratings than in double slits, producing fewer maxima at larger angles.

\section*{Watch Physics}
Diffraction Grating This video explains the geometry behind the diffraction pattern produced by a diffraction grating.

Click to view content\\
Watch Physics: Diffraction Grating. This video explains diffraction of light through several holes.

Click to view content\\
The equation that gives the points of constructive interference produced by a diffraction grating is \(\mathrm{d} \backslash \sin \backslash\) theta \(=\mathrm{m} \backslash\) lambda. Why does that equation look familiar?\\
a. It is the same as the equation for destructive interference for a double-slit diffraction pattern.\\
b. It is the same as the equation for constructive interference for a double-slit diffraction pattern.\\
c. It is the same as the equation for constructive interference for a single-slit diffraction pattern.\\
d. It is the same as the equation for destructive interference for a single-slit diffraction pattern.

Just what is the resolution limit of an aperture or lens? To answer that question, consider the diffraction pattern for a circular aperture, which, similar to the diffraction pattern of light passing through a slit, has a central maximum that is wider and brighter than the maxima surrounding it (see Figure 17.19 (a)). It can be shown that, for a circular aperture of diameter \(D\), the first minimum in the diffraction pattern occurs at \(\theta=1.22 \lambda / D\), provided that the aperture is large compared with the wavelength of light, which is the case for most optical instruments. The accepted criterion for determining the diffraction limit to resolution based on diffraction was developed by Lord Rayleigh in the 19th century. The Rayleigh criterion for the diffraction limit to resolution states that two images are just resolvable when the center of the diffraction pattern of one is directly over the first minimum of the diffraction pattern of the other. See Figure 17.20 (b). The first minimum is at an angle of \(\theta=1.22 \lambda / D\), so that two point objects are just resolvable if they are separated by the angle\\
\(\theta=1.22 \frac{\lambda}{D}\),\\
where \(\lambda\) is the wavelength of the light (or other electromagnetic radiation) and \(D\) is the diameter of the aperture, lens, mirror, etc., with which the two objects are observed. In the expression above, \(\theta\) has units of radians.

\begin{figure}[h]
\begin{center}
  \includegraphics[max width=\textwidth]{2964a1e9-ab41-4e16-802f-4d831a0a3895-29}
\captionsetup{labelformat=empty}
\caption{Figure 17.20 (a) Graph of intensity of the diffraction pattern for a circular aperture. Note that, similar to a single slit, the central maximum is wider and brighter than those to the sides. (b) Two point objects produce overlapping diffraction patterns. Shown here is the Rayleigh criterion for their being just resolvable. The central maximum of one pattern lies on the first minimum of the other.}
\end{center}
\end{figure}

\section*{Snap Lab}
\section*{Resolution}
\begin{itemize}
  \item A sheet of white paper
  \item A black pen or pencil
  \item A measuring tape
\end{itemize}

Instructions\\
Procedure

\begin{enumerate}
  \item Draw two lines several mm apart on a white sheet of paper.
  \item Move away from the sheet as it is held upright, and measure the distance at which you can just distinguish (resolve) the lines as separate.
  \item Use \(\theta=1.22 \frac{\lambda}{D}\) to calculate \(D\) the diameter of your pupil. Use the distance between the lines and the maximum distance at which they were resolved to calculate \(\theta\). Use the average wavelength for visible light as the value for \(\lambda\).
  \item Compare your answer to the average pupil diameter of 3 mm .
\end{enumerate}

Describe resolution in terms of minima and maxima of diffraction patterns.\\
a. The limit for resolution is when the minimum of the pattern for one of the lines is directly over the first minimum of the pattern for the other line.\\
b. The limit for resolution is when the maximum of the pattern for one of the lines is directly over the first minimum of the pattern for the other line.\\
c. The limit for resolution is when the maximum of the pattern for one of the lines is directly over the second minimum of the pattern for the other line.\\
d. The limit for resolution is when the minimum of the pattern for one of the lines is directly over the second maximum of the pattern for the other line.

\section*{Worked Example}
Change of Wavelength A monochromatic laser beam of green light with a wavelength of 550 nm passes from air to water. The refractive index of water is 1.33. What will be the wavelength of the light after it enters the water?

\section*{Strategy}
You can assume that the refractive index of air is the same as that of light in a vacuum because they are so close. You then have all the information you need to solve for \(\lambda_{n}\).

Solution\\
\(\lambda_{n}=\frac{\lambda}{n}=\frac{550 \mathrm{~nm}}{1.33}=414 \mathrm{~nm}\)\\
17.5

Discussion\\
The refractive index of air is 1.0003 , so the approximation holds for three significant figures. You would not see the light change color, however. Color is determined by frequency, not wavelength.

\section*{Worked Example}
Diffraction Grating A diffraction grating has 2000 lines per centimeter. At what angle will the first-order maximum form for green light with a wavelength of 520 nm ?

\section*{Strategy}
You are given enough information to calculate \(d\), and you are given the values of \(\lambda\) and \(m\). You will have to find the \(\arcsin\) of a number to find \(\theta\).

Solution

First find \(d\).\\
\(d=\frac{1 \mathrm{~cm}}{2,000}=5.00 \times 10^{-4} \mathrm{~cm}=5,000 \mathrm{~nm}\)\\
17.6

Rearrange the equation for constructive interference conditions for a diffraction grating, and substitute the known values.

\[
\begin{aligned}
d \sin \theta & =m \lambda \\
\theta & =\sin ^{-1} \frac{m \lambda}{d} \\
& =\sin ^{-1}\left(\frac{(1)(520)}{5,000}\right) \\
& =5.97
\end{aligned}
\]

Discussion\\
This angle seems reasonable for the first maximum. Recall that the meaning of \(\sin { }^{1}(\) or \(\arcsin )\) is the angle with a sine that is (the unknown). Remember that the value of \(\sin \theta\) will not be greater than 1 for any value of \(\theta\).

\section*{Worked Example}
Resolution What is the minimum angular spread of a \(633-\mathrm{nm}\)-wavelength He-Ne laser beam that is originally 1.00 mm in diameter?

\section*{Strategy}
The diameter of the beam is the same as if it were coming through an aperture of that size, so \(D=1.00 \mathrm{~mm}\). You are given \(\lambda\), and you must solve for \(\theta\).

Solution\\
\(\theta=\frac{(1.22) \lambda}{D}=\frac{(1.22)(633 \mathrm{~nm})}{1.00 \times 10^{6} \mathrm{~nm}}=7.72 \times 10^{-4} \mathrm{rad}=0.0442^{\circ}\)\\
17.7

Discussion\\
The conversion factor for radians to degrees is 1.000 radian \(=57.3^{\circ}\). The spread is very small and would not be noticeable over short distances. The angle represents the angular separation of the central maximum and the first minimum.

\section*{Practice Problems}
6.

A beam of yellow light has a wavelength of 600 nm in a vacuum and a wavelength of 397 nm in Plexiglas. What is the refractive index of Plexiglas?\\
a. 1.51\\
b. 2.61\\
c. 3.02\\
d. 3.77

\section*{7.}
What is the angle between two just-resolved points of light for a 3.00 mm diameter pupil, assuming an average wavelength of 550 nm ?\\
a. 224 rad\\
b. 183 rad\\
c. \(1.83 \times 10^{-4} \mathrm{rad}\)\\
d. \(2.24 \times 10^{-4} \mathrm{rad}\)

\section*{Check Your Understanding}
\section*{Teacher Support}
Teacher Support Use these questions to assess student achievement of the section's learning objectives. If students are struggling with a specific objective, these questions will help identify which and direct students to the relevant content.\\
8.

How is an interference pattern formed by a diffraction grating different from the pattern formed by a double slit?\\
a. The pattern is colorful.\\
b. The pattern is faded.\\
c. The pattern is sharper.\\
d. The pattern is curved.\\
9.

A beam of light always spreads out. Why can a beam not be produced with parallel rays to prevent spreading?\\
a. Light is always polarized.\\
b. Light is always reflected.\\
c. Light is always refracted.\\
d. Light is always diffracted.\\
10.

Compare interference patterns formed by a double slit and by a diffraction grating in terms of brightness and narrowness of bands.\\
a. The pattern formed has broader and brighter bands.\\
b. The pattern formed has broader and duller bands.\\
c. The pattern formed has narrower and duller bands.\\
d. The pattern formed has narrower and brighter bands.\\
11.

Describe the slits in a diffraction grating in terms of number and spacing, as compared to a two-slit diffraction setup.\\
a. The slits in a diffraction grating are broader, with space between them that is greater than the separation of the two slits in two-slit diffraction.\\
b. The slits in a diffraction grating are broader, with space between them that is the same as the separation of the two slits in two-slit diffraction.\\
c. The slits in a diffraction grating are narrower, with space between them that is the same as the separation of the two slits in two-slit diffraction.\\
d. The slits in a diffraction grating are narrower, with space between them that is greater than the separation of the two slits in two-slit diffraction.

\section*{Key Terms}
differential interference contrast (DIC) separating a polarized light source into two beams polarized at right angles to each other and coherent with each other then, after passing through the sample, recombining and realigning the beams so they have the same plane of polarization, and then creating an interference pattern caused by the differences in their optical path and the refractive indices of the parts of the sample they passed through; the result is an image with contrast and shadowing that could not be observed with traditional optics\\
diffraction bending of a wave around the edges of an opening or an obstacle\\
diffraction grating many of evenly spaced slits having dimensions such that they produce an interference pattern

Huygens s principle Every point on a wavefront is a source of wavelets that spread out in the forward direction at the same speed as the wave itself; the new wavefront is a line tangent to all of the wavelets.\\
iridescence the effect that occurs when tiny, fingerlike structures in regular patterns act as reflection gratings, producing constructive interference that gives feathers colors not solely due to their pigmentation\\
laser acronym for a device that produces light amplification by stimulated emission of radiation\\
monochromatic one color\\
monochromator device that separates the various wavelengths of incoming light and allows a beam with only a specific wavelength to pass through

Rayleigh criterion two images are just resolvable when the center of the diffraction pattern of one is directly over the first minimum of the diffraction pattern of the other\\
resolution degree to which two images can be distinguished from one another, which is limited by diffraction\\
wavefront points on a wave surface that all share an identical, constant phase

\section*{Key Equations}
17.1 Understanding Diffraction and Interference\\
17.2 Applications of Diffraction, Interference, and Coherence

\section*{Section Summary}
\subsection*{17.1 Understanding Diffraction and Interference}
\begin{itemize}
  \item The wavelength of light varies with the refractive index of the medium.
  \item Slits produce a diffraction pattern if their width and separation are similar to the wavelength of light passing through them.
  \item Interference bands of a single-slit diffraction pattern can be predicted.
  \item Interference bands of a double-slit diffraction pattern can be predicted.
\end{itemize}

\subsection*{17.2 Applications of Diffraction, Interference, and Coherence}
\begin{itemize}
  \item The focused, coherent radiation emitted by lasers has many uses in medicine and industry.
  \item Characteristics of diffraction patterns produced with diffraction gratings can be determined.
  \item Diffraction gratings have been incorporated in many instruments, including microscopes and spectrometers.
  \item Resolution has a limit that can be predicted.
\end{itemize}

\end{document}