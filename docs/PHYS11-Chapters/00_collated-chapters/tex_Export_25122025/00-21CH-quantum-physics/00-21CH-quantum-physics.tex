\documentclass[10pt]{article}
\usepackage[utf8]{inputenc}
\usepackage[T1]{fontenc}
\usepackage{graphicx}
\usepackage[export]{adjustbox}
\graphicspath{ {./images/} }
\usepackage{caption}
\usepackage{amsmath}
\usepackage{amsfonts}
\usepackage{amssymb}
\usepackage[version=4]{mhchem}
\usepackage{stmaryrd}

\begin{document}
\captionsetup{singlelinecheck=false}
\begin{figure}[h]
\begin{center}
  \includegraphics[max width=\textwidth]{6470bb83-82b9-4463-a603-8051d77f8e0e-01}
\captionsetup{labelformat=empty}
\caption{Figure 21.1 In Lewis Carroll's classic text Alice's Adventures in Wonderland, Alice follows a rabbit down a hole into a land of curiosity. While many of her interactions in Wonderland are of surprising consequence, they follow a certain inherent logic. (credit: modification of work by John Tenniel, Wikimedia Commons)}
\end{center}
\end{figure}

\section*{Chapter Outline}
21.1 Planck and Quantum Nature of Light

\subsection*{21.2 Einstein and the Photoelectric Effect}
\subsection*{21.3 The Dual Nature of Light}
\section*{Introduction}
\section*{Teacher Support}
\section*{Teacher Support}
\begin{itemize}
  \item Discuss with students what features of Alice in Wonderland were unusual compared to life as Alice typically knew it. Question whether an inherent logic still existed in Wonderland. These discussions should prepare students not only to anticipate the bizarre nature of quantum mechanics but also to look for its underlying structure.
  \item Ask students to review what concepts of physics they consider normal. Look for big ideas: the structure of matter, conservation of momentum and energy, and the like. Have the students consider earlier instances in the course of which these concepts have been challenged or expanded. A reminder of relativity of gravitational fields could remind them that simple concepts are often more than what they seem.
\end{itemize}

At first glance, the quantum nature of light can be a strange and bewildering concept. Between light acting as discrete chunks, massless particles providing momenta, and fundamental particles behaving like waves, it may often seem like something out of Alice in Wonderland.

For many, the study of this branch of physics can be as enthralling as Lewis Carroll's classic novel. Recalling the works of legendary characters and brilliant scientists such as Einstein, Planck, and Compton, the study of light's quantum nature will provide you an interesting tale of how a clever interpretation of some small details led to the most important discoveries of the past 150 years. From the electronics revolution of the twentieth century to our future progress in solar energy and space exploration, the quantum nature of light should yield a rabbit hole of curious consequence, within which lie some of the most fascinating truths of our time.

\section*{Teacher Support}
Teacher Support Before students begin this chapter, it would be useful to review the following concepts:

\begin{itemize}
  \item The wave nature of light-particularly the relationship between wavelength, frequency, and speed.
  \item The electromagnetic (EM) spectrum-review what separates ultraviolet from visible light and other portions of the EM spectrum. Review the various EM divisions in order of increasing frequency.
  \item The reflection and absorption of light upon reaching a new boundary. No need to discuss refraction or Snell's Law, just a reminder that the energy will be reflected or transmitted.
  \item Conservation of momentum and energy within a macroscopic collision. This will be useful when discussing photon momentum in The Dual Nature of Light.
  \item The double-slit experiment and other evidence that light acts as a wave.
\end{itemize}

\subsection*{21.1 Planck and Quantum Nature of Light}
\section*{Section Learning Objectives}
By the end of this section, you will be able to do the following:

\begin{itemize}
  \item Describe blackbody radiation
  \item Define quantum states and their relationship to modern physics
  \item Calculate the quantum energy of lights
  \item Explain how photon energies vary across divisions of the electromagnetic spectrum
\end{itemize}

\section*{Teacher Support}
Teacher Support The learning objectives in this section will help your students master the following standards:

\begin{itemize}
  \item (3) Scientific processes. The student uses critical thinking, scientific reasoning, and problem solving to make informed decisions within and outside the classroom. The student is expected to:
  \item (D): explain the impacts of the scientific contributions of a variety of historical and contemporary scientists on scientific thought and society.
  \item (8) Science concepts. The student knows simple examples of atomic, nuclear, and quantum phenomena. The student is expected to:
  \item (B): compare and explain the emission spectra produced by various atoms; and
  \item (D): give examples of applications of atomic and nuclear phenomena such as radiation therapy, diagnostic imaging, and nuclear power, and examples of quantum phenomena such as digital cameras.
\end{itemize}

\section*{Section Key Terms}
\section*{Blackbodies}
\section*{Teacher Support}
\section*{Teacher Support}
\begin{itemize}
  \item Prior to beginning this section, it would be a good idea to review wave concepts including frequency, wavelength, and amplitude. Have students write down a list of equations or statements that relate to the three concepts.
  \item \hspace{0pt} [BL][OL]Discuss what could be meant by the term blackbody. Why do some objects appear black? Furthermore, why do we see objects that are red as red? It is said that black is the absence of color, but what does that mean in terms of the light reflected into our eyes?
  \item \hspace{0pt} [AL]Discuss what can happen to energy when it strikes a surface. Discuss how it can be reflected or transmitted. If a blackbody is perfectly black, what must be happening to all of the energy incident upon it?
  \item \hspace{0pt} [EL]Reinforce that the term blackbody is nothing more than its name suggests - that is, a body that is perfectly black. Discuss what perfectly black means. Is a black piece of paper perfectly black?
\end{itemize}

Our first story of curious significance begins with a T-shirt. You are likely aware that wearing a tight black T-shirt outside on a hot day provides a significantly less comfortable experience than wearing a white shirt. Black shirts, as well as all other black objects, will absorb and re-emit a significantly greater amount of radiation from the sun. This shirt is a good approximation of what is called a blackbody.

\section*{Teacher Support}
Teacher Support Occasionally, texts refer to blackbody and perfect blackbody as two different concepts. It is likely best to refer to anything that is not a perfect blackbody as an approximation of a blackbody in order to avoid confusion.

A perfect blackbody is one that absorbs and re-emits all radiated energy that is incident upon it. Imagine wearing a tight shirt that did this! This phenomenon is often modeled with quite a different scenario. Imagine carving a small hole in an oven that can be heated to very high temperatures. As the temperature of this container gets hotter and hotter, the radiation out of this dark hole would increase as well, re-emitting all energy provided it by the increased temperature. The hole may even begin to glow in different colors as the temperature is increased. Like a burner on your stove, the hole would glow red, then orange, then blue, as the temperature is increased. In time, the hole would continue to glow but the light would be invisible to our eyes. This container is a good model of a perfect blackbody.

It is the analysis of blackbodies that led to one of the most consequential discoveries of the twentieth century. Take a moment to carefully examine Figure 21.2. What relationships exist? What trends can you see? The more time you spend interpreting this figure, the closer you will be to understanding quantum physics!

\begin{figure}[h]
\begin{center}
  \includegraphics[max width=\textwidth]{6470bb83-82b9-4463-a603-8051d77f8e0e-05}
\captionsetup{labelformat=empty}
\caption{Figure 21.2 Graphs of blackbody radiation (from an ideal radiator) at three different radiator temperatures. The intensity or rate of radiation emission increases dramatically with temperature, and the peak of the spectrum shifts toward the visible and ultraviolet parts of the spectrum. The shape of the spectrum cannot be described with classical physics.}
\end{center}
\end{figure}

\section*{Teacher Support}
Teacher Support It is important for students to make sense of Figure 21.2 before progressing further. Have students independently create a list of observations from the graph. When presenting their observations, press the students on the specifics of their observations.\\[0pt]
[BL]Discuss what variables are being graphed. Have them complete the statement: \(\_\_\_\_\) is dependent upon \(\_\_\_\_\) . Discuss what is meant by intensity. What is the difference between being mad and intensely mad?\\[0pt]
[OL]Discuss what the peak of each graph refers to. Ask if the radiation intensity depends upon the wavelength of the radiation. How do they know this? What do the peaks on each graph mean?\\[0pt]
[AL]Discuss why there are three lines on the graph. Does it make sense that an increase in temperature would cause the line of the graph to be raised? Why does this make sense? A good challenging exercise would be to have the students re-graph the information in order to represent EM radiation intensity against frequency.

\section*{Tips For Success}
When encountering a new graph, it is best to try to interpret the graph before you read about it. Doing this will make the following text more meaningful and\\
will help to remind yourself of some of the key concepts within the section.

\section*{Understanding Blackbody Graphs}
Figure 21.2 is a plot of radiation intensity against radiated wavelength. In other words, it shows how the intensity of radiated light changes when a blackbody is heated to a particular temperature.

It may help to just follow the bottom-most red line labeled \(3,000 \mathrm{~K}\), red hot. The graph shows that when a blackbody acquires a temperature of \(3,000 \mathrm{~K}\), it radiates energy across the electromagnetic spectrum. However, the energy is most intensely emitted at a wavelength of approximately 1000 nm . This is in the infrared portion of the electromagnetic spectrum. While a body at this temperature would appear red-hot to our eyes, it would truly appear 'infraredhot' if we were able to see the entire spectrum.

A few other important notes regarding Figure 21.2:

\begin{itemize}
  \item As temperature increases, the total amount of energy radiated increases. This is shown by examining the area underneath each line.
  \item Regardless of temperature, all red lines on the graph undergo a consistent pattern. While electromagnetic radiation is emitted throughout the spectrum, the intensity of this radiation peaks at one particular wavelength.
  \item As the temperature changes, the wavelength of greatest radiation intensity changes. At \(4,000 \mathrm{~K}\), the radiation is most intense in the yellow-green portion of the spectrum. At \(6,000 \mathrm{~K}\), the blackbody would radiate white hot, due to intense radiation throughout the visible portion of the electromagnetic spectrum. Remember that white light is the emission of all visible colors simultaneously.
  \item As the temperature increases, the frequency of light providing the greatest intensity increases as well. Recall the equation \(v=f \lambda\). Because the speed of light is constant, frequency and wavelength are inversely related. This is verified by the leftward movement of the three red lines as temperature is increased.
\end{itemize}

\section*{Teacher Support}
Teacher Support Discuss the bullet points above. Why does an increase in temperature result in an increase in the total amount of energy radiated? Do you have personal experience with the relationship described in bullet point \#3? Students may not have answers as to the causal factors for some of the observations in the above bullet points. Remind them that this is okay as these why questions were the big questions being asked by physicists at the turn of the twentieth century!\\[0pt]
[BL][OL]Do you have personal evidence to show that as temperature increases the energy radiated increases as well?\\[0pt]
[AL]Remind students that temperature is just a measure of the average kinetic energy of particles in a gas. Does this definition support bullet point \(\# 1\) ?

While in science it is important to categorize observations, theorizing as to why the observations exist is crucial to scientific advancement. Why doesn't a blackbody emit radiation evenly across all wavelengths? Why does the temperature of the body change the peak wavelength that is radiated? Why does an increase in temperature cause the peak wavelength emitted to decrease? It is questions like these that drove significant research at the turn of the twentieth century. And within the context of these questions, Max Planck discovered something of tremendous importance.

\section*{Planck's Revolution}
\section*{Teacher Support}
Teacher Support Planck's revolution is very much the story of the scientific method-reconciling disconnects between theory and experimental results. Encourage the students to think of other events - either historical or within their own lives - in which a predominant theory was shown to be incorrect when confronted with overwhelming evidence to the contrary. Possible examples include the geocentric model, the ether, or the four elements.

The prevailing theory at the time of Max Planck's discovery was that intensity and frequency were related by the equation \(I=\frac{2 k T}{\lambda^{2}}\). This equation, derived from classical physics and using wave phenomena, infers that as wavelength increases, the intensity of energy provided will decrease with an inverse-squared relationship. This relationship is graphed in Figure 21.3 and shows a troubling trend. For starters, it should be apparent that the graph from this equation does not match the blackbody graphs found experimentally. Additionally, it shows that for an object of any temperature, there should be an infinite amount of energy quickly emitted in the shortest wavelengths. When theory and experimental results clash, it is important to re-evaluate both models. The disconnect between theory and reality was termed the ultraviolet catastrophe.

\begin{figure}[h]
\begin{center}
  \includegraphics[max width=\textwidth]{6470bb83-82b9-4463-a603-8051d77f8e0e-08}
\captionsetup{labelformat=empty}
\caption{Figure 21.3 The graph above shows the true spectral measurements by a blackbody against those predicted by the classical theory at the time. The discord between the predicted classical theory line and the actual results is known as the ultraviolet catastrophe.}
\end{center}
\end{figure}

Due to concerns over the ultraviolet catastrophe, Max Planck began to question whether another factor impacted the relationship between intensity and wavelength. This factor, he posited, should affect the probability that short wavelength light would be emitted. Should this factor reduce the probability of short wavelength light, it would cause the radiance curve to not progress infinitely as in the classical theory, but would instead cause the curve to precipitate back downward as is shown in the \(5,000 \mathrm{~K}, 4,000 \mathrm{~K}\), and \(3,000 \mathrm{~K}\) temperature lines of the graph in Figure 21.3. Planck noted that this factor, whatever it may be, must also be dependent on temperature, as the intensity decreases at lower and lower wavelengths as the temperature increases.

The determination of this probability factor was a groundbreaking discovery in physics, yielding insight not just into light but also into energy and matter itself. It would be the basis for Planck's 1918 Nobel Prize in Physics and would result in the transition of physics from classical to modern understanding. In an attempt to determine the cause of the probability factor, Max Planck constructed a new theory. This theory, which created the branch of physics called quantum mechanics, speculated that the energy radiated by the blackbody could exist only in specific numerical, or quantum, states. This theory is described by the\\
equation \(E=n h f\), where \(n\) is any nonnegative integer ( \(0,1,2,3, \ldots\) ) and \(h\) is Planck's constant, given by \(h=6.626 \times 10^{-34} \mathrm{~J} \cdot \mathrm{~s}\), and \(f\) is frequency.

Through this equation, Planck's probability factor can be more clearly understood. Each frequency of light provides a specific quantized amount of energy. Low frequency light, associated with longer wavelengths would provide a smaller amount of energy, while high frequency light, associated with shorter wavelengths, would provide a larger amount of energy. For specified temperatures with specific total energies, it makes sense that more low frequency light would be radiated than high frequency light. To a degree, the relationship is like pouring coins through a funnel. More of the smaller pennies would be able to pass through the funnel than the larger quarters. In other words, because the value of the coin is somewhat related to the size of the coin, the probability of a quarter passing through the funnel is reduced!

Furthermore, an increase in temperature would signify the presence of higher energy. As a result, the greater amount of total blackbody energy would allow for more of the high frequency, short wavelength, energies to be radiated. This permits the peak of the blackbody curve to drift leftward as the temperature increases, as it does from the \(3,000 \mathrm{~K}\) to \(4,000 \mathrm{~K}\) to \(5,000 \mathrm{~K}\) values. Furthering our coin analogy, consider a wider funnel. This funnel would permit more quarters to pass through and allow for a reduction in concern about the probability factor.

In summary, it is the interplay between the predicted classical model and the quantum probability that creates the curve depicted in Figure 21.3. Just as quarters have a higher currency denomination than pennies, higher frequencies come with larger amounts of energy. However, just as the probability of a quarter passing through a fixed diameter funnel is reduced, so is the probability of a high frequency light existing in a fixed temperature object. As is often the case in physics, it is the balancing of multiple incredible ideas that finally allows for better understanding.

\section*{Quantization}
\section*{Teacher Support}
Teacher Support [EL]Quantum is related to the word quantity, a measure of the amount of something. Discuss why the term quantum would be useful in this context.\\[0pt]
[BL, OL, AL]Quantum vs. continuous states is well described when considering clocks. A digital clock represents quantum states-it reads 11:14 a.m., then 11:15 a.m. An analog clock with a continually gliding second hand is a good representation of continuous states-it does not appear to pause at any one instant. What would you consider an analog clock that ticks each second? What would you consider a grandfather clock?

It may be helpful at this point to further consider the idea of quantum states.

Atoms, molecules, and fundamental electron and proton charges are all examples of physical entities that are quantized - that is, they appear only in certain discrete values and do not have every conceivable value. On the macroscopic scale, this is not a revolutionary concept. A standing wave on a string allows only particular harmonics described by integers. Going up and down a hill using discrete stair steps causes your potential energy to take on discrete values as you move from step to step. Furthermore, we cannot have a fraction of an atom, or part of an electron's charge, or 14.33 cents. Rather, everything is built of integral multiples of these substructures.

That said, to discover quantum states within a phenomenon that science had always considered continuous would certainly be surprising. When Max Planck was able to use quantization to correctly describe the experimentally known shape of the blackbody spectrum, it was the first indication that energy was quantized on a small scale as well. This discovery earned Planck the Nobel Prize in Physics in 1918 and was such a revolutionary departure from classical physics that Planck himself was reluctant to accept his own idea. The general acceptance of Planck's energy quantization was greatly enhanced by Einstein's explanation of the photoelectric effect (discussed in the next section), which took energy quantization a step further.

\begin{figure}[h]
\begin{center}
  \includegraphics[max width=\textwidth]{6470bb83-82b9-4463-a603-8051d77f8e0e-11}
\captionsetup{labelformat=empty}
\caption{Figure 21.4 The German physicist Max Planck had a major influence on the early development of quantum mechanics, being the first to recognize that energy is sometimes quantized. Planck also made important contributions to special relativity and classical physics. (credit: Library of Congress, Prints and Photographs Division, Wikimedia Commons)}
\end{center}
\end{figure}

\section*{Worked Example}
\section*{How Many Photons per Second Does a Typical Light Bulb Produce?}
Assuming that 10 percent of a 100-W light bulb's energy output is in the visible range (typical for incandescent bulbs) with an average wavelength of 580 nm , calculate the number of visible photons emitted per second.

\section*{Strategy}
The number of visible photons per second is directly related to the amount of energy emitted each second, also known as the bulb's power. By determining the bulb's power, the energy emitted each second can be found. Since the\\
power is given in watts, which is joules per second, the energy will be in joules. By comparing this to the amount of energy associated with each photon, the number of photons emitted each second can be determined.

Solution\\
The power in visible light production is 10.0 percent of 100 W , or \(10.0 \mathrm{~J} / \mathrm{s}\). The energy of the average visible photon is found by substituting the given average wavelength into the formula\\
\(E=n h f=\frac{n h c}{\lambda}\).\\
By rearranging the above formula to determine energy per photon, this produces\\
\(E / n=\frac{\left(6.63 \times 10^{-34} \mathrm{~J} \cdot \mathrm{~s}\right)\left(3.00 \times 10^{8} \mathrm{~m} / \mathrm{s}\right)}{580 \times 10^{-9} \mathrm{~m}}=3.43 \times 10^{-19} \mathrm{~J} /\) photon .\\
21.1

The number of visible photons per second is thus\\
\(\frac{\text { photons }}{\text { sec }}=\frac{10.0 \mathrm{~J} / \mathrm{s}}{3.43 \times 10^{-19} \mathrm{~J} / \text { photon }}=2.92 \times 10^{19}\) photons \(/ \mathrm{s}\).\\
Discussion\\
This incredible number of photons per second is verification that individual photons are insignificant in ordinary human experience. However, it is also a verification of our everyday experience - on the macroscopic scale, photons are so small that quantization becomes essentially continuous.

\section*{Worked Example}
How does Photon Energy Change with Various Portions of the EM Spectrum? Refer to the Graphs of Blackbody Radiation shown in the first figure in this section. Compare the energy necessary to radiate one photon of infrared light and one photon of visible light.

\section*{Strategy}
To determine the energy radiated, it is necessary to use the equation \(E=n h f\). It is also necessary to find a representative frequency for infrared light and visible light.

Solution\\
According to the first figure in this section, one representative wavelength for infrared light is \(2000 \mathrm{~nm}\left(2.000 \times 10^{-6} \mathrm{~m}\right)\). The associated frequency of an infrared light is\\
\(f=\frac{c}{\lambda}=\frac{3.00 \times 10^{8} \mathrm{~m} / \mathrm{s}}{2.000 \times 10^{-6} \mathrm{~m}}=1.50 \times 10^{14} \mathrm{~Hz}\).\\
21.2

Using the equation \(E=n h f\), the energy associated with one photon of representative infrared light is\\
\(\frac{E}{n}=h \cdot f=\left(6.63 \times 10^{-34} \mathrm{~J} \cdot \mathrm{~s}\right)\left(1.50 \times 10^{14} \mathrm{~Hz}\right)=9.95 \times 10^{-20} \frac{\mathrm{~J}}{\text { photon }}\).\\
21.3

The same process above can be used to determine the energy associated with one photon of representative visible light. According to the first figure in this section, one representative wavelength for visible light is 500 nm .\\
\(f=\frac{c}{\lambda}=\frac{3.00 \times 10^{8} \mathrm{~m} / \mathrm{s}}{5.00 \times 10^{-7} \mathrm{~m}}=6.00 \times 10^{14} \mathrm{~Hz}\).\\
21.4\\
\(\frac{E}{n}=h \cdot f=\left(6.63 \times 10^{-34} \mathrm{~J} \cdot \mathrm{~s}\right)\left(6.00 \times 10^{14} \mathrm{~Hz}\right)=3.98 \times 10^{-19} \frac{\mathrm{~J}}{\text { photon }}\).\\
21.5

Discussion\\
This example verifies that as the wavelength of light decreases, the quantum energy increases. This explains why a fire burning with a blue flame is considered more dangerous than a fire with a red flame. Each photon of short-wavelength blue light emitted carries a greater amount of energy than a long-wavelength red light. This example also helps explain the differences in the \(3,000 \mathrm{~K}, 4,000\) K , and \(6,000 \mathrm{~K}\) lines shown in the first figure in this section. As the temperature is increased, more energy is available for a greater number of short-wavelength photons to be emitted.

\section*{Practice Problems}
1.

An AM radio station broadcasts at a frequency of \(1,530 \mathrm{kHz}\). What is the energy in Joules of a photon emitted from this station?\\
a. \(10.1 \times 10^{-26} \mathrm{~J}\)\\
b. \(1.01 \times 10^{-28} \mathrm{~J}\)\\
c. \(1.01 \times 10^{-29} \mathrm{~J}\)\\
d. \(1.01 \times 10^{-27} \mathrm{~J}\)\\
2.

A photon travels with energy of 1.0 eV . What type of EM radiation is this photon?\\
a. visible radiation\\
b. microwave radiation\\
c. infrared radiation\\
d. ultraviolet radiation

\section*{Check Your Understanding}
3.

Do reflective or absorptive surfaces more closely model a perfect blackbody?\\
a. reflective surfaces\\
b. absorptive surfaces\\
4.

A black T-shirt is a good model of a blackbody. However, it is not perfect. What prevents a black T-shirt from being considered a perfect blackbody?\\
a. The T-shirt reflects some light.\\
b. The T-shirt absorbs all incident light.\\
c. The T-shirt re-emits all the incident light.\\
d. The T-shirt does not reflect light.\\
5.

What is the mathematical relationship linking the energy of a photon to its frequency?\\
a. \(E=h()\)\\
b. \(E=h /\)\\
c. \(E=h / f\)\\
d. \(E=h f\)\\
6.

Why do we not notice quantization of photons in everyday experience?\\
a. because the size of each photon is very large\\
b. because the mass of each photon is so small\\
c. because the energy provided by photons is very large\\
d. because the energy provided by photons is very small\\
7.

Two flames are observed on a stove. One is red while the other is blue. Which flame is hotter?\\
a. The red flame is hotter because red light has lower frequency.\\
b. The red flame is hotter because red light has higher frequency.\\
c. The blue flame is hotter because blue light has lower frequency.\\
d. The blue flame is hotter because blue light has higher frequency.\\
8.

Your pupils dilate when visible light intensity is reduced. Does wearing sunglasses that lack UV blockers increase or decrease the UV hazard to your eyes? Explain.\\
a. Increase, because more high-energy UV photons can enter the eye.\\
b. Increase, because less high-energy UV photons can enter the eye.\\
c. Decrease, because more high-energy UV photons can enter the eye.\\
d. Decrease, because less high-energy UV photons can enter the eye.

\section*{9.}
The temperature of a blackbody radiator is increased. What will happen to the most intense wavelength of light emitted as this increase occurs?\\
a. The wavelength of the most intense radiation will vary randomly.\\
b. The wavelength of the most intense radiation will increase.\\
c. The wavelength of the most intense radiation will remain unchanged.\\
d. The wavelength of the most intense radiation will decrease.

\subsection*{21.2 Einstein and the Photoelectric E ect}
\section*{Section Learning Objectives}
By the end of this section, you will be able to do the following:

\begin{itemize}
  \item Describe Einstein's explanation of the photoelectric effect
  \item Describe how the photoelectric effect could not be explained by classical physics
  \item Calculate the energy of a photoelectron under given conditions
  \item Describe use of the photoelectric effect in biological applications, photoelectric devices and movie soundtracks
\end{itemize}

\section*{Teacher Support}
Teacher Support The learning objectives in this section will help your students master the following standards:

\begin{itemize}
  \item (3) Scientific processes. The student uses critical thinking, scientific reasoning, and problem solving to make informed decisions within and outside the classroom. The student is expected to:
  \item (D): explain the impacts of the scientific contributions of a variety of historical and contemporary scientists on scientific thought and society.
  \item (8) Science concepts. The student knows simple examples of atomic, nuclear, and quantum phenomena. The student is expected to:
  \item (A): describe the photoelectric effect and the dual nature of light.
\end{itemize}

\section*{Section Key Terms}
\section*{The Photoelectric Effect}
\section*{Teacher Support}
[EL]Ask the students what they think the term photoelectric means. How does the term relate to its definition?

When light strikes certain materials, it can eject electrons from them. This is called the photoelectric effect, meaning that light (photo) produces electricity. One common use of the photoelectric effect is in light meters, such as those that adjust the automatic iris in various types of cameras. Another use is in solar cells, as you probably have in your calculator or have seen on a rooftop or a roadside sign. These make use of the photoelectric effect to convert light into electricity for running different devices.

\section*{Teacher Support}
Teacher Support [BL][OL]Discuss with students what may cause light to eject electrons from a material. Are there certain materials that are more susceptible to having electrons ejected?\\[0pt]
[AL]Ask students why a light meter would be useful in a camera. How could the number of electrons emitted from the light meter control the camera's iris? Have students draw a diagram of the camera that may demonstrate this effect.

\begin{figure}[h]
\begin{center}
  \includegraphics[max width=\textwidth]{6470bb83-82b9-4463-a603-8051d77f8e0e-17}
\captionsetup{labelformat=empty}
\caption{Figure 21.5 The photoelectric effect can be observed by allowing light to fall on the metal plate in this evacuated tube. Electrons ejected by the light are}
\end{center}
\end{figure}

collected on the collector wire and measured as a current. A retarding voltage between the collector wire and plate can then be adjusted so as to determine the energy of the ejected electrons. (credit: P. P. Urone)

\section*{Revolutionary Properties of the Photoelectric Effect}
When Max Planck theorized that energy was quantized in a blackbody radiator, it is unlikely that he would have recognized just how revolutionary his idea was. Using tools similar to the light meter in Figure 21.5, it would take a scientist of Albert Einstein's stature to fully discover the implications of Max Planck's radical concept.

Through careful observations of the photoelectric effect, Albert Einstein realized that there were several characteristics that could be explained only if \(E M\) radiation is itself quantized. While these characteristics will be explained a bit later in this section, you can already begin to appreciate why Einstein's idea is very important. It means that the apparently continuous stream of energy in an EM wave is actually not a continuous stream at all. In fact, the EM wave itself is actually composed of tiny quantum packets of energy called photons.

In equation form, Einstein found the energy of a photon or photoelectron to be\\
\(E=h f\),\\
where \(E\) is the energy of a photon of frequency \(f\) and \(h\) is Planck's constant. A beam from a flashlight, which to this point had been considered a wave, instead could now be viewed as a series of photons, each providing a specific amount of energy see Figure 21.6. Furthermore, the amount of energy within each individual photon is based upon its individual frequency, as dictated by \(E=h f\). As a result, the total amount of energy provided by the beam could now be viewed as the sum of all frequency-dependent photon energies added together.\\
\includegraphics[max width=\textwidth, center]{6470bb83-82b9-4463-a603-8051d77f8e0e-18}

Figure 21.6 An EM wave of frequency \(f\) is composed of photons, or individual quanta of EM radiation. The energy of each photon is \(E=h f\), where \(h\) is Planck's constant and \(f\) is the frequency of the EM radiation. Higher intensity means more photons per unit area per second. The flashlight emits large numbers of photons of many different frequencies, hence others have energy \(E^{\prime}=h f^{\prime}\), and so on.

\section*{Teacher Support}
Teacher Support It is important for students to be comfortable with the material to this point before moving forward. To ensure that they are, one task that you may have them do is to draw a few pictures similar to Figure 21.6. Have the students draw photons leaving a low intensity flashlight vs. a high intensity flashlight, a high frequency flashlight vs. a low frequency flashlight, and a high wavelength flashlight vs. a low wavelength flashlight. These diagrams will help ensure the students understand fundamental concepts before moving to the difficult proofs that follow.

Just as with Planck's blackbody radiation, Einstein's concept of the photon could take hold in the scientific community only if it could succeed where classical physics failed. The photoelectric effect would be a key to demonstrating Einstein's brilliance.

Consider the following five properties of the photoelectric effect. All of these properties are consistent with the idea that individual photons of EM radiation are absorbed by individual electrons in a material, with the electron gaining the photon's energy. Some of these properties are inconsistent with the idea that EM radiation is a simple wave. For simplicity, let us consider what happens with monochromatic EM radiation in which all photons have the same energy \(h f\).\\
\texttt{https://cdn.mathpix.com/cropped/6470bb83-82b9-4463-a603-8051d77f8e0e-19.jpg?height=255&width=1054&top_left_y=1342&top_left_x=455}\\
\includegraphics[max width=\textwidth, center]{6470bb83-82b9-4463-a603-8051d77f8e0e-19}

Electrons knocked out\\
\includegraphics[max width=\textwidth, center]{6470bb83-82b9-4463-a603-8051d77f8e0e-19(1)}

Figure 21.7 Incident radiation strikes a clean metal surface, ejecting multiple electrons from it. The manner in which the frequency and intensity of the incoming radiation affect the ejected electrons strongly suggests that electromagnetic radiation is quantized. This event, called the photoelectric effect, is strong evidence for the existence of photons.

\begin{enumerate}
  \item If we vary the frequency of the EM radiation falling on a clean metal surface, we find the following: For a given material, there is a threshold frequency \(f_{0}\) for the EM radiation below which no electrons are ejected, regardless of intensity. Using the photon model, the explanation for this is clear. Individual photons interact with individual electrons. Thus if the energy of an individual photon is too low to break an electron away, no electrons will be ejected. However, if EM radiation were a simple wave, sufficient energy could be obtained simply by increasing the intensity.
  \item Once EM radiation falls on a material, electrons are ejected without delay. As soon as an individual photon of sufficiently high frequency is absorbed by an individual electron, the electron is ejected. If the EM radiation were a simple wave, several minutes would be required for sufficient energy to be deposited at the metal surface in order to eject an electron.
  \item The number of electrons ejected per unit time is proportional to the intensity of the EM radiation and to no other characteristic. High-intensity EM radiation consists of large numbers of photons per unit area, with all photons having the same characteristic energy, \(h f\). The increased number of photons per unit area results in an increased number of electrons per unit area ejected.
  \item If we vary the intensity of the EM radiation and measure the energy of ejected electrons, we find the following: The maximum kinetic energy of ejected electrons is independent of the intensity of the EM radiation. Instead, as noted in point 3 above, increased intensity results in more electrons of the same energy being ejected. If EM radiation were a simple wave, a higher intensity could transfer more energy, and higher-energy electrons would be ejected.
  \item The kinetic energy KE of an ejected electron equals the photon energy minus the binding energy BE of the electron in the specific material. An individual photon can give all of its energy to an electron. The photon's energy is partly used to break the electron away from the material. The remainder goes into the ejected electron's kinetic energy. In equation form, this is given by\\
\(K E_{e}=h f-B E\),\\
21.6\\
where \(K E_{e}\) is the maximum kinetic energy of the ejected electron, \(h f\) is the photon's energy, and BE is the binding energy of the electron to the particular material. This equation explains the properties of the photoelectric effect quantitatively and demonstrates that BE is the minimum amount of energy necessary to eject an electron. If the energy supplied is less than BE , the electron\\
cannot be ejected. The binding energy can also be written as \(B E=h f_{0}\), where \(f_{0}\) is the threshold frequency for the particular material. Figure 21.8 shows a graph of maximum \(K E_{e}\) versus the frequency of incident EM radiation falling on a particular material.\\
\includegraphics[max width=\textwidth, center]{6470bb83-82b9-4463-a603-8051d77f8e0e-21}
\end{enumerate}

Figure 21.8 A graph of the kinetic energy of an ejected electron, \(\mathrm{KE}_{e}\), versus the frequency of EM radiation impinging on a certain material. There is a threshold frequency below which no electrons are ejected, because the individual photon interacting with an individual electron has insufficient energy to break it away. Above the threshold energy, \(\mathrm{KE}_{e}\) increases linearly with \(f\), consistent with \(\mathrm{KE}_{e} =h f-\mathrm{BE}\). The slope of this line is \(h\), so the data can be used to determine Planck's constant experimentally.

\section*{Teacher Support}
Teacher Support Show students Figure 21.8. What would be the kinetic energy of an electron if \(f\) is less than \(f_{0}\) ? What does this mean? Why would this be the case? These questions aim to help students internalize the concept of binding energy.

\section*{Tips For Success}
The following five pieces of information can be difficult to follow without some organization. It may be useful to create a table of expected results of each of the five properties, with one column showing the classical wave model result and one column showing the modern photon model result.

The table may look something like Table 21.1

Table 21.1 Table of Expected Results

\section*{Teacher Support}
Teacher Support It may be useful to complete the table above as a class. This material takes some time to interpret, so encourage students to move slowly. Once completed, your table may look like Table 21.2.

Table 21.2 Completed Table

\section*{Virtual Physics}
Photoelectric Effect Click to view content\\
In this demonstration, see how light knocks electrons off a metal target, and recreate the experiment that spawned the field of quantum mechanics.

\section*{Grasp Check}
In the circuit provided, what are the three ways to increase the current?\\
a. increase the intensity, increase the wavelength, alter the target\\
b. decrease the intensity, increase the wavelength, alter the target\\
c. decrease the intensity, decrease the wavelength, alter the target\\
d. increase the intensity, decrease the wavelength, alter the target

\section*{Worked Example}
Photon Energy and the Photoelectric Effect: A Violet Light (a) What is the energy in joules and electron volts of a photon of \(420-\mathrm{nm}\) violet light? (b) What is the maximum kinetic energy of electrons ejected from calcium by 420 nm violet light, given that the binding energy of electrons for calcium metal is 2.71 eV ?

\section*{Strategy}
To solve part (a), note that the energy of a photon is given by \(E=h f\). For part (b), once the energy of the photon is calculated, it is a straightforward application of \(K E_{e}=h f-B E\) to find the ejected electron's maximum kinetic energy, since BE is given.

Solution for (a)\\
Photon energy is given by\\
\(E=h f\).\\
Since we are given the wavelength rather than the frequency, we solve the familiar relationship \(c=f \lambda\) for the frequency, yielding\\
\(f=\frac{c}{\lambda}\)\\
21.7

Combining these two equations gives the useful relationship\\
\(E=\frac{h c}{\lambda}\).\\
21.8

Now substituting known values yields\\
\(E=\frac{\left(6.63 \times 10^{-34} \mathrm{~J} \cdot \mathrm{~s}\right)\left(3.00 \times 10^{8} \mathrm{~m} / \mathrm{s}\right)}{4.20 \times 10^{-7} \mathrm{~m}}=4.74 \times 10^{-19} \mathrm{~J}\).\\
21.9

Converting to eV , the energy of the photon is\\
\(E=\left(4.74 \times 10^{-19} \mathrm{~J} \cdot \mathrm{~s}\right) \frac{1 \mathrm{eV}}{1.60 \times 10^{-19} \mathrm{~J}}=2.96 \mathrm{eV}\).\\
21.10

Solution for (b)\\
Finding the kinetic energy of the ejected electron is now a simple application of the equation \(K E_{e}=h f-B E\). Substituting the photon energy and binding energy yields\\
\(K E_{e}=h f-B E=2.96 \mathrm{eV}-2.71 \mathrm{eV}=0.25 \mathrm{eV}\).\\
21.11

Discussion\\
The energy of this 420 nm photon of violet light is a tiny fraction of a joule, and so it is no wonder that a single photon would be difficult for us to sense directly-humans are more attuned to energies on the order of joules. But looking at the energy in electron volts, we can see that this photon has enough energy to affect atoms and molecules. A DNA molecule can be broken with about 1 eV of energy, for example, and typical atomic and molecular energies\\
are on the order of eV , so that the photon in this example could have biological effects, such as sunburn. The ejected electron has rather low energy, and it would not travel far, except in a vacuum. The electron would be stopped by a retarding potential of only 0.26 eV , a slightly larger KE than calculated above. In fact, if the photon wavelength were longer and its energy less than 2.71 eV , then the formula would give a negative kinetic energy, an impossibility. This simply means that the 420 nm photons with their 2.96 eV energy are not much above the frequency threshold. You can see for yourself that the threshold wavelength is 458 nm (blue light). This means that if calcium metal were used in a light meter, the meter would be insensitive to wavelengths longer than those of blue light. Such a light meter would be completely insensitive to red light, for example.

\section*{Practice Problems}
10.

What is the longest-wavelength EM radiation that can eject a photoelectron from silver, given that the bonding energy is 4.73 eV ? Is this radiation in the visible range?\\
a. \(2.63 \times 10^{-7} \mathrm{~m} ; \mathrm{No}\), the radiation is in microwave region.\\
b. \(2.63 \times 10^{-7} \mathrm{~m}\); No, the radiation is in visible region.\\
c. \(2.63 \times 10^{-7} \mathrm{~m} ; \mathrm{No}\), the radiation is in infrared region.\\
d. \(2.63 \times 10^{-7} \mathrm{~m}\); No, the radiation is in ultraviolet region.\\
11.

What is the maximum kinetic energy in eV of electrons ejected from sodium metal by \(450-\mathrm{nm}\) EM radiation, given that the binding energy is 2.28 eV ?\\
a. 0.48 V\\
b. 0.82 eV\\
c. 1.21 eV\\
d. 0.48 eV

\section*{Technological Applications of the Photoelectric Effect}
While Einstein's understanding of the photoelectric effect was a transformative discovery in the early 1900s, its presence is ubiquitous today. If you have watched streetlights turn on automatically in response to the setting sun, stopped elevator doors from closing simply by putting your hands between them, or turned on a water faucet by sliding your hands near it, you are familiar with the electric eye, a name given to a group of devices that use the photoelectric effect for detection.

All these devices rely on photoconductive cells. These cells are activated when light is absorbed by a semi-conductive material, knocking off a free electron. When this happens, an electron void is left behind, which attracts a nearby\\
electron. The movement of this electron, and the resultant chain of electron movements, produces a current. If electron ejection continues, further holes are created, thereby increasing the electrical conductivity of the cell. This current can turn switches on and off and activate various familiar mechanisms.

One such mechanism takes place where you may not expect it. Next time you are at the movie theater, pay close attention to the sound coming out of the speakers. This sound is actually created using the photoelectric effect! The audiotape in the projector booth is a transparent piece of film of varying width. This film is fed between a photocell and a bright light produced by an exciter lamp. As the transparent portion of the film varies in width, the amount of light that strikes the photocell varies as well. As a result, the current in the photoconductive circuit changes with the width of the filmstrip. This changing current is converted to a changing frequency, which creates the soundtrack commonly heard in the theater.

\section*{Work In Physics}
Solar Energy Physicist According to the U.S. Department of Energy, Earth receives enough sunlight each hour to power the entire globe for a year. While converting all of this energy is impossible, the job of the solar energy physicist is to explore and improve upon solar energy conversion technologies so that we may harness more of this abundant resource.

The field of solar energy is not a new one. For over half a century, satellites and spacecraft have utilized photovoltaic cells to create current and power their operations. As time has gone on, scientists have worked to adapt this process so that it may be used in homes, businesses, and full-scale power stations using solar cells like the one shown in Figure 21.9.

\begin{figure}[h]
\begin{center}
  \includegraphics[max width=\textwidth]{6470bb83-82b9-4463-a603-8051d77f8e0e-26}
\captionsetup{labelformat=empty}
\caption{Figure 21.9 A solar cell is an example of a photovoltaic cell. As light strikes the cell, the cell absorbs the energy of the photons. If this energy exceeds the binding energy of the electrons, then electrons will be forced to move in the cell, thereby producing a current. This current may be used for a variety of purposes. (credit: U.S. Department of Energy)}
\end{center}
\end{figure}

Solar energy is converted to electrical energy in one of two manners: direct transfer through photovoltaic cells or thermal conversion through the use of a CSP, concentrating solar power, system. Unlike electric eyes, which trip a mechanism when current is lost, photovoltaic cells utilize semiconductors to\\
directly transfer the electrons released through the photoelectric effect into a directed current. The energy from this current can then be converted for storage, or immediately used in an electric process. A CSP system is an indirect method of energy conversion. In this process, light from the Sun is channeled using parabolic mirrors. The light from these mirrors strikes a thermally conductive material, which then heats a pool of water. This water, in turn, is converted to steam, which turns a turbine and creates electricity. While indirect, this method has long been the traditional means of large-scale power generation.

There are, of course, limitations to the efficacy of solar power. Cloud cover, nightfall, and incident angle strike at high altitudes are all factors that directly influence the amount of light energy available. Additionally, the creation of photovoltaic cells requires rare-earth minerals that can be difficult to obtain. However, the major role of a solar energy physicist is to find ways to improve the efficiency of the solar energy conversion process. Currently, this is done by experimenting with new semi conductive materials, by refining current energy transfer methods, and by determining new ways of incorporating solar structures into the current power grid.

Additionally, many solar physicists are looking into ways to allow for increased solar use in impoverished, more remote locations. Because solar energy conversion does not require a connection to a large-scale power grid, research into thinner, more mobile materials will permit remote cultures to use solar cells to convert sunlight collected during the day into stored energy that can then be used at night.

Regardless of the application, solar energy physicists are an important part of the future in responsible energy growth. While a doctoral degree is often necessary for advanced research applications, a bachelor's or master's degree in a related science or engineering field is typically enough to gain access into the industry. Computer skills are very important for energy modeling, including knowledge of CAD software for design purposes. In addition, the ability to collaborate and communicate with others is critical to becoming a solar energy physicist.

\section*{Grasp Check}
What role does the photoelectric effect play in the research of a solar energy physicist?\\
a. The understanding of photoelectric effect allows the physicist to understand the generation of light energy when using photovoltaic cells.\\
b. The understanding of photoelectric effect allows the physicist to understand the generation of electrical energy when using photovoltaic cells.\\
c. The understanding of photoelectric effect allows the physicist to understand the generation of electromagnetic energy when using photovoltaic cells.\\
d. The understanding of photoelectric effect allows the physicist to understand the generation of magnetic energy when using photovoltaic cells.

\section*{Check Your Understanding}
12.

How did Einstein's model of photons change the view of a beam of energy leaving a flashlight?\\
a. A beam of light energy is now considered a continual stream of wave energy, not photons.\\
b. A beam of light energy is now considered a collection of photons, each carrying its own individual energy.\\
13.

True or false-Visible light is the only type of electromagnetic radiation that can cause the photoelectric effect.\\
a. false\\
b. true\\
14.

Is the photoelectric effect a direct consequence of the wave character of EM radiation, or the particle character of EM radiation?\\
a. The photoelectric effect is a direct consequence of the particle nature of EM radiation.\\
b. The photoelectric effect is a direct consequence of the wave nature of EM radiation.\\
c. The photoelectric effect is a direct consequence of both the wave and particle nature of EM radiation.\\
d. The photoelectric effect is a direct consequence of neither the wave nor the particle nature of EM radiation.\\
15.

Which aspects of the photoelectric effect can only be explained using photons?\\
a. aspects 1,2, and 3\\
b. aspects 1,2, and 4\\
c. aspects 1, 2, 4 and 5\\
d. aspects 1, 2, 3, 4 and 5\\
16.

In a photovoltaic cell, what energy transformation takes place?\\
a. Solar energy transforms into electric energy.\\
b. Solar energy transforms into mechanical energy.\\
c. Solar energy transforms into thermal energy.\\
d. In a photovoltaic cell, thermal energy transforms into electric energy.\\
17.

True or false-A current is created in a photoconductive cell, even if only one electron is expelled from a photon strike.\\
a. false\\
b. true\\
18.

What is a photon and how is it different from other fundamental particles?\\
a. A photon is a quantum packet of energy; it has infinite mass.\\
b. A photon is a quantum packet of energy; it is massless.\\
c. A photon is a fundamental particle of an atom; it has infinite mass.\\
d. A photon is a fundamental particle of an atom; it is massless.

\section*{Ke Terms}
blackbody object that absorbs all radiated energy that strikes it and also emits energy across all wavelengths of the electromagnetic spectrum

Compton effect phenomenon whereby X-rays scattered from materials have decreased energy\\
electric eye group of devices that use the photoelectric effect for detection\\
particle-wave duality property of behaving like either a particle or a wave; the term for the phenomenon that all particles have wave-like characteristics and waves have particle-like characteristics\\
photoelectric effect phenomenon whereby some materials eject electrons when exposed to light\\
photoelectron electron that has been ejected from a material by a photon of light\\
photon a quantum, or particle, of electromagnetic radiation\\
photon momentum amount of momentum of a photon, calculated by \(\mathbf{p}=\frac{h}{\lambda}\)\\
quantized the fact that certain physical entities exist only with particular discrete values and not every conceivable value\\
quantum discrete packet or bundle of a physical entity such as energy\\
ultraviolet catastrophe misconception that blackbodies would radiate high frequency energy at a much higher rate than energy radiated at lower frequencies

\section*{Ke Equations}
21.1 Planck and Quantum Nature of Light\\
21.2 Einstein and the Photoelectric Effect\\
21.3 The Dual Nature of Light

\section*{Section Summar}
\subsection*{21.1 Planck and Quantum Nature of Light}
\begin{itemize}
  \item A blackbody will radiate energy across all wavelengths of the electromagnetic spectrum.
  \item Radiation of a blackbody will peak at a particular wavelength, dependent on the temperature of the blackbody.
  \item Analysis of blackbody radiation led to the field of quantum mechanics, which states that radiated energy can only exist in discrete quantum states.
\end{itemize}

\subsection*{21.2 Einstein and the Photoelectric Effect}
\begin{itemize}
  \item The photoelectric effect is the process in which EM radiation ejects electrons from a material.
  \item Einstein proposed photons to be quanta of EM radiation having energy \(E=h f\), where \(f\) is the frequency of the radiation.
  \item All EM radiation is composed of photons. As Einstein explained, all characteristics of the photoelectric effect are due to the interaction of individual photons with individual electrons.
  \item The maximum kinetic energy \(\mathrm{KE}_{e}\) of ejected electrons (photoelectrons) is given by \(K E_{e}=h f-B E\), where \(h f\) is the photon energy and BE is the binding energy (or work function) of the electron in the particular material.
\end{itemize}

\subsection*{21.3 The Dual Nature of Light}
\begin{itemize}
  \item Compton scattering provided evidence that photon-electron interactions abide by the principles of conservation of momentum and conservation of energy.
  \item The momentum of individual photons, quantified by \(\mathbf{p}=\frac{h}{\lambda}\), can be used to explain observations of comets and may lead to future space technologies.
  \item Electromagnetic waves and matter have both wave-like and particle-like properties. This phenomenon is defined as particle-wave duality.
\end{itemize}

\end{document}