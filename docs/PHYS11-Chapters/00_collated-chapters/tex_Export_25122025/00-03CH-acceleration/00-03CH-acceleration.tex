\documentclass[10pt]{article}
\usepackage[utf8]{inputenc}
\usepackage[T1]{fontenc}
\usepackage{graphicx}
\usepackage[export]{adjustbox}
\graphicspath{ {./images/} }
\usepackage{caption}
\usepackage{amsmath}
\usepackage{amsfonts}
\usepackage{amssymb}
\usepackage[version=4]{mhchem}
\usepackage{stmaryrd}

\begin{document}
\captionsetup{singlelinecheck=false}
\begin{figure}[h]
\begin{center}
  \includegraphics[max width=\textwidth]{b9ef3717-e0c7-4b61-bf9f-f8eb1ba3da83-01}
\captionsetup{labelformat=empty}
\caption{Figure 3.1 A plane slows down as it comes in for landing in St. Maarten. Its acceleration is in the opposite direction of its velocity. (Steve Conry, Flickr)}
\end{center}
\end{figure}

\section*{Chapter Outline}
\subsection*{3.1 Acceleration}
\subsection*{3.2 Representing Acceleration with Equations and Graphs}
\section*{Introduction}
\section*{Teacher Support}
Teacher Support Ask the students to give definitions of acceleration. Dispel any misconceptions such as, acceleration means very high speed or going faster. Emphasize that acceleration does not just indicate speeding up; acceleration also includes slowing down or changing direction. Explain that acceleration is the change in either the magnitude or direction of velocity, or both. Have students list the objects in the opening image that are moving. Then ask which are definitely accelerating and which might be accelerating. Review the use of + and - signs as they relate to acceleration and velocity. Explain that, when studying motion, these symbols are often used to indicate the direction of motion. The + symbol typically represents motion that is to the right or upward, whereas - typically represents motion that is to the left or downward.

You may have heard the term accelerator, referring to the gas pedal in a car. When the gas pedal is pushed down, the flow of gasoline to the engine increases, which increases the car's velocity. Pushing on the gas pedal results in acceleration because the velocity of the car increases, and acceleration is defined as\\
a change in velocity. You need two quantities to define velocity: a speed and a direction. Changing either of these quantities, or both together, changes the velocity. You may be surprised to learn that pushing on the brake pedal or turning the steering wheel also causes acceleration. The first reduces the speed and so changes the velocity, and the second changes the direction and also changes the velocity.

In fact, any change in velocity-whether positive, negative, directional, or any combination of these - is called an acceleration in physics. The plane in the picture is said to be accelerating because its velocity is decreasing as it prepares to land. To begin our study of acceleration, we need to have a clear understanding of what acceleration means.

\section*{Teacher Support}
Teacher Support Before students begin this chapter, it is useful to review the following concepts:

\begin{itemize}
  \item Significant figures-demonstrate how to obtain the proper number of significant figures when adding and multiplying
  \item Scientific notation and how it expresses significant figures
  \item Converting units-demonstrate how to convert from \(\mathrm{km} / \mathrm{h}\) to \(\mathrm{m} / \mathrm{s}\); show how units cancel in calculations
  \item Calculating average -demonstrate how to calculate the average of two numbers
  \item Commonly used terms-explain that constant means unchanging, so constant acceleration refers to acceleration that is not changing in time
  \item Explain that initial means starting or beginning, so the initial time is the time at which the action of interest begins
  \item Explain that an object that is not moving is often described in physics as being at rest
  \item Review conventions of coordinate systems
  \item Review kinematics concepts introduced earlier: vectors, displacement, velocity, and speed
\end{itemize}

\subsection*{3.1 Acceleration}
\section*{Section Learning Objectives}
By the end of this section, you will be able to do the following:

\begin{itemize}
  \item Explain acceleration and determine the direction and magnitude of acceleration in one dimension
  \item Analyze motion in one dimension using kinematic equations and graphic representations
\end{itemize}

\section*{Teacher Support}
Teacher Support The Learning Objectives in this section will help your students master the following standards:

\begin{itemize}
  \item (4) Science concepts. The student knows and applies the laws governing motion in a variety of situations. The student is expected to:
  \item (A) generate and interpret graphs and charts describing different types of motion, including the use of real-time technology such as motion detectors or photogates;
  \item (B) describe and analyze motion in one dimension using equations with the concepts of distance, displacement, speed, average velocity, instantaneous velocity, and acceleration.
\end{itemize}

In addition, the High School Physics Laboratory Manual addresses content in this section in the lab titled: Position and Speed of an Object, as well as the following standards:

\begin{itemize}
  \item (4) Science concepts. The student knows and applies the laws governing motion in a variety of situations. The student is expected to:
  \item (B) describe and analyze motion in one dimension using equations with the concepts of distance, displacement, speed, average velocity, instantaneous velocity, and acceleration.
\end{itemize}

\section*{Section Key Terms}
\section*{Teacher Support}
Teacher Support [BL][OL] Begin a general discussion about acceleration and deceleration. Ask for examples of both. Explain that deceleration is not used in physics because acceleration is either positive or negative. Lead students to their topics of interest, such as motor vehicles or sports. Explain that the capital Greek letter delta always means final minus initial and that the net change may be zero, positive, or negative.\\[0pt]
[AL] See how much students remember about vectors. What does a vector arrow represent? Ask them to name some quantities that are vectors and some that are scalars.

\section*{Defining Acceleration}
Throughout this chapter we will use the following terms: time, displacement, velocity, and acceleration. Recall that each of these terms has a designated variable and SI unit of measurement as follows:

\begin{itemize}
  \item Time: \(t\), measured in seconds (s)
  \item Displacement: \(\Delta d\), measured in meters (m)
  \item Velocity: \(v\), measured in meters per second (m/s)
  \item Acceleration: \(a\), measured in meters per second per second ( \(\mathrm{m} / \mathrm{s}^{2}\), also called meters per second squared)
  \item Also note the following:\\
\(-\Delta\) means change in
  \item The subscript 0 refers to an initial value; sometimes subscript \(i\) is instead used to refer to initial value.
  \item The subscript f refers to final value
  \item A bar over a symbol, such as \(\bar{a}\), means average
\end{itemize}

\section*{Teacher Support}
Teacher Support [BL] Review definitions of the terms: time, displacement, velocity, and acceleration. Point out that the variables commonly used to represent these quantities are the first letters of the corresponding term.\\[0pt]
[OL] Verify that the students know the SI units in which time, displacement, velocity, and acceleration are expressed. Note that these are some of the seven base units of the metric system. Explain that converting to base units is a good first step when calculating these quantities. Explain the meaning of seconds squared in the denominator of the units of acceleration.\\[0pt]
[AL] Review all the base units of the metric system. Explain how these units are interrelated. For example, show how length is defined by time.\\[0pt]
[BL][OL] Use the equation \(\bar{a}=\frac{\Delta v}{\Delta t}=\frac{v_{\mathrm{f}}-v_{0}}{t_{\mathrm{f}}-t_{0}}\) to emphasize the relationship between \(\Delta\) and the subscripts f and 0 . Distinguish between constant and variable acceleration. There could be confusion here, especially in the case of increasing acceleration. Be sure students understand that the word deceleration is not used in physics and that acceleration may be either positive or negative.\\[0pt]
[AL] See if students can use the concept of acceleration to understand confusing statements such as "a decrease in the rate of increase." For example, use the concept of acceleration to analyze the statement "the rate of increase in the cost of health care is decreasing." If the increase in the cost is defined as positive, then the acceleration in the cost of health care would be negative.\\[0pt]
[OL] The arrow for acceleration that points opposite to the arrow for velocity may be confusing. Explain that the acceleration arrow points in the direction opposite the velocity because the velocity is getting smaller, i.e., the velocity arrow is getting shorter.

Acceleration is the change in velocity divided by a period of time during which the change occurs. The SI units of velocity are \(\mathrm{m} / \mathrm{s}\) and the SI units for time are s , so the SI units for acceleration are \(\mathrm{m} / \mathrm{s}^{2}\). Average acceleration is given by\\
\(\bar{a}=\frac{\Delta v}{\Delta t}=\frac{v_{\mathrm{f}}-v_{0}}{t_{\mathrm{f}}-t_{0}}\).\\
Average acceleration is distinguished from instantaneous acceleration, which is acceleration at a specific instant in time. The magnitude of acceleration is often not constant over time. For example, runners in a race accelerate at a greater rate in the first second of a race than during the following seconds. You do not need to know all the instantaneous accelerations at all times to calculate average acceleration. All you need to know is the change in velocity (i.e., the final velocity minus the initial velocity) and the change in time (i.e., the final time minus the initial time), as shown in the formula. Note that the average acceleration can be positive, negative, or zero. A negative acceleration is simply an acceleration in the negative direction.

Keep in mind that although acceleration points in the same direction as the change in velocity, it is not always in the direction of the velocity itself. When an object slows down, its acceleration is opposite to the direction of its velocity. In everyday language, this is called deceleration; but in physics, it is accelerationwhose direction happens to be opposite that of the velocity. For now, let us assume that motion to the right along the \(x\)-axis is positive and motion to the left is negative.

Figure 3.2 shows a car with positive acceleration in (a) and negative acceleration in (b). The arrows represent vectors showing both direction and magnitude of velocity and acceleration.

\begin{figure}[h]
\begin{center}
  \includegraphics[max width=\textwidth]{b9ef3717-e0c7-4b61-bf9f-f8eb1ba3da83-06(1)}
\captionsetup{labelformat=empty}
\caption{Figure 3.2 The car is speeding up in (a) and slowing down in (b).}
\end{center}
\end{figure}

\begin{center}
\includegraphics[max width=\textwidth]{b9ef3717-e0c7-4b61-bf9f-f8eb1ba3da83-06}
\end{center}

Velocity and acceleration are both vector quantities. Recall that vectors have both magnitude and direction. An object traveling at a constant velocitytherefore having no acceleration-does accelerate if it changes direction. So, turning the steering wheel of a moving car makes the car accelerate because the velocity changes direction.

\section*{Virtual Physics}
The Moving Man With this animation, you can produce both variations of acceleration and velocity shown in Figure 3.2, plus a few more variations. Vary the velocity and acceleration by sliding the red and green markers along the scales. Keeping the velocity marker near zero will make the effect of acceleration\\
more obvious. Try changing acceleration from positive to negative while the man is moving. We will come back to this animation and look at the Charts view when we study graphical representation of motion.

Click to view content

\section*{Teacher Support}
Teacher Support Have students use a very low setting for velocity and acceleration because it is easier to see how the motion changes. Show students how setting velocity as positive and acceleration as negative creates the motion that resembles that of an object thrown into the air.

\section*{Grasp Check}
(a)\\
\includegraphics[max width=\textwidth, center]{b9ef3717-e0c7-4b61-bf9f-f8eb1ba3da83-07}

\[
\begin{aligned}
& y \\
& \uparrow \\
& \longrightarrow x
\end{aligned}
\]

(b)\\
\includegraphics[max width=\textwidth, center]{b9ef3717-e0c7-4b61-bf9f-f8eb1ba3da83-07(1)}

\section*{Figure 3.3}
Which part, (a) or (b), is represented when the velocity vector is on the positive side of the scale and the acceleration vector is set on the negative side of the scale? What does the car's motion look like for the given scenario?\\
a. Part (a). The car is slowing down because the acceleration and the velocity vectors are acting in the opposite direction.\\
b. Part (a). The car is speeding up because the acceleration and the velocity vectors are acting in the same direction.\\
c. Part (b). The car is slowing down because the acceleration and velocity vectors are acting in the opposite directions.\\
d. Part (b). The car is speeding up because the acceleration and the velocity vectors are acting in the same direction.

\section*{Calculating Average Acceleration}
Look back at the equation for average acceleration. You can see that the calculation of average acceleration involves three values: change in time, \((\Delta t)\); change in velocity, \((\Delta v)\); and acceleration ( \(a\) ).

Change in time is often stated as a time interval, and change in velocity can often be calculated by subtracting the initial velocity from the final velocity. Average acceleration is then simply change in velocity divided by change in time. Before you begin calculating, be sure that all distances and times have been converted to meters and seconds. Look at these examples of acceleration of a subway train.

\section*{Teacher Support}
Teacher Support [BL][OL] Before beginning the calculations, verify that students understand the equation for acceleration. Do they understand what it means when quantities have a plus or minus sign? Do they understand the units for each variable?

\section*{Worked Example}
An Accelerating Subway Train A subway train accelerates from rest to \(30.0 \mathrm{~km} / \mathrm{h}\) in 20.0 s . What is the average acceleration during that time interval?

\section*{Strategy}
Start by making a simple sketch.

\begin{figure}[h]
\begin{center}
  \includegraphics[max width=\textwidth]{b9ef3717-e0c7-4b61-bf9f-f8eb1ba3da83-09}
\captionsetup{labelformat=empty}
\caption{Figure 3.4}
\end{center}
\end{figure}

This problem involves four steps:

\begin{enumerate}
  \item Convert to units of meters and seconds.
  \item Determine the change in velocity.
  \item Determine the change in time.
  \item Use these values to calculate the average acceleration.
\end{enumerate}

Solution

\begin{enumerate}
  \item Identify the knowns. Be sure to read the problem for given information, which may not look like numbers. When the problem states that the train starts from rest, you can write down that the initial velocity is \(0 \mathrm{~m} / \mathrm{s}\). Therefore, \(v_{0}=0 ; v_{\mathrm{f}}=30.0 \mathrm{~km} / \mathrm{h}\); and \(\Delta t=20.0 \mathrm{~s}\).
  \item Convert the units.
\end{enumerate}

\begin{itemize}
  \item \(\frac{30.0 \mathrm{~km}}{\mathrm{~h}} \times \frac{10^{3} \mathrm{~m}}{1 \mathrm{~km}} \times \frac{1 \mathrm{~h}}{3600 \mathrm{~s}}=8.333 \frac{\mathrm{~m}}{\mathrm{~s}}\)\\
3.1
\end{itemize}

\begin{enumerate}
  \setcounter{enumi}{2}
  \item Calculate change in velocity, \(\Delta v=v_{\mathrm{f}}-v_{0}=8.333 \mathrm{~m} / \mathrm{s}-0=+8.333 \mathrm{~m} / \mathrm{s}\), where the plus sign means the change in velocity is to the right.
  \item We know \(\Delta t\), so all we have to do is insert the known values into the formula for average acceleration.
\end{enumerate}

\begin{itemize}
  \item \(\bar{a}=\frac{\Delta v}{\Delta t}=\frac{8.333 \mathrm{~m} / \mathrm{s}}{20.00 \mathrm{~s}}=+0.417 \frac{\mathrm{~m}}{\mathrm{~s}^{2}}\)\\
3.2
\end{itemize}

Discussion

The plus sign in the answer means that acceleration is to the right. This is a reasonable conclusion because the train starts from rest and ends up with a velocity directed to the right (i.e., positive). So, acceleration is in the same direction as the change in velocity, as it should be.

\section*{Teacher Support}
Teacher Support Note that extra digits were carried along and rounding off to the correct number of significant figures, 3 , was not done until the final answer was calculated.

\section*{Worked Example}
An Accelerating Subway Train Now, suppose that at the end of its trip, the train slows to a stop in 8.00 s from a speed of \(30.0 \mathrm{~km} / \mathrm{h}\). What is its average acceleration during this time?

\section*{Strategy}
Again, make a simple sketch.

\begin{figure}[h]
\begin{center}
  \includegraphics[max width=\textwidth]{b9ef3717-e0c7-4b61-bf9f-f8eb1ba3da83-10}
\captionsetup{labelformat=empty}
\caption{Figure 3.5}
\end{center}
\end{figure}

In this case, the train is decelerating and its acceleration is negative because it is pointing to the left. As in the previous example, we must find the change in velocity and change in time, then solve for acceleration.

Solution

\begin{enumerate}
  \item Identify the knowns: \(v_{0}=30.0 \mathrm{~km} / \mathrm{h} ; v_{\mathrm{f}}=0 ;\) and \(\Delta t=8.00 \mathrm{~s}\).
  \item Convert the units. From the first problem, we know that \(30.0 \mathrm{~km} / \mathrm{h}= 8.333 \mathrm{~m} / \mathrm{s}\).
  \item Calculate change in velocity, \(\Delta v=v_{\mathrm{f}}-v_{0}=0-8.333 \mathrm{~m} / \mathrm{s}=-8.333 \mathrm{~m} / \mathrm{s}\), where the minus sign means that the change in velocity points to the left.
  \item We know \(\Delta t=8.00 \mathrm{~s}\), so all we have to do is insert the known values into the equation for average acceleration.
\end{enumerate}

\begin{itemize}
  \item \(\bar{a}=\frac{\Delta v}{\Delta t}=\frac{-8.333 \mathrm{~m} / \mathrm{s}}{8.00 \mathrm{~s}}=-1.04 \frac{\mathrm{~m}}{\mathrm{~s}^{2}}\)\\
3.3
\end{itemize}

\section*{Discussion}
The minus sign indicates that acceleration is to the left. This is reasonable because the train initially has a positive velocity in this problem, and a negative acceleration would reduce the velocity. Again, acceleration is in the same direction as the change in velocity, which is negative in this case. This acceleration can be called a deceleration because it has a direction opposite to the velocity.

\section*{Teacher Support}
Teacher Support Help students see the relationship between the direction of the vector arrows and the plus and minus signs. Explain that one indication of the sign for acceleration is that it is in the direction opposite that of the velocity. Also point out that correctly identifying the initial and final speeds will result in the correct sign for acceleration.

\section*{Tips For Success}
\begin{itemize}
  \item It is easier to get plus and minus signs correct if you always assume that motion is away from zero and toward positive values on the \(x\)-axis. This way \(v\) always starts off being positive and points to the right. If speed is increasing, then acceleration is positive and also points to the right. If speed is decreasing, then acceleration is negative and points to the left.
  \item It is a good idea to carry two extra significant figures from step-to-step when making calculations. Do not round off with each step. When you arrive at the final answer, apply the rules of significant figures for the operations you carried out and round to the correct number of digits. Sometimes this will make your answer slightly more accurate.
\end{itemize}

\section*{Practice Problems}
1.

A cheetah can accelerate from rest to a speed of \(30.0 \backslash, \backslash \operatorname{text}\{\mathrm{~m} / \mathrm{s}\}\) in \(7.00 \backslash, \backslash \operatorname{text}\{\mathrm{~s}\}\). What is its acceleration?\\
a. \(\{-0.23\} \backslash, \backslash \operatorname{text}\{\mathrm{m} / \mathrm{s}\}^{\wedge} 2\)\\
b. \(\{-4.29\} \backslash, \backslash \operatorname{text}\{\mathrm{m} / \mathrm{s}\}^{\wedge} 2\)\\
c. \(0.23 \backslash, \backslash \operatorname{text}\{\mathrm{~m} / \mathrm{s}\}^{\wedge} 2\)\\
d. \(4.29 \backslash, \backslash \operatorname{text}\{\mathrm{~m} / \mathrm{s}\}^{\wedge} 2\)\\
2.

A women backs her car out of her garage with an acceleration of \(1.40 \backslash, \backslash \operatorname{text}\{\mathrm{~m} / \mathrm{s}\}^{\wedge} 2 \backslash!\). How long does it take her to reach a speed of \(2.00 \backslash, \backslash \operatorname{text}\{\mathrm{~m} / \mathrm{s}\}\) ?\\
a. \(0.70 \backslash, \backslash \operatorname{text}\{\mathrm{~s}\}\)\\
b. \(1.43 \backslash, \backslash \operatorname{text}\{\mathrm{~s}\}\)\\
c. \(2.80 \backslash, \backslash \operatorname{text}\{\mathrm{~s}\}\)\\
d. \(3.40 \backslash, \backslash \operatorname{text}\{\mathrm{~s}\}\)

\section*{Watch Physics}
Acceleration This video shows the basic calculation of acceleration and some useful unit conversions.

Click to view content

\section*{Teacher Support}
Teacher Support Ask students to note the explanation of units and the identification of the vector quantities. Tell them the calculations demonstrated in the video are fairly straightforward and that the definitions given for displacement, elapsed time, velocity, and acceleration should be clear.

Why is acceleration a vector quantity?\\
a. It is a vector quantity because it has magnitude as well as direction.\\
b. It is a vector quantity because it has magnitude but no direction.\\
c. It is a vector quantity because it is calculated from distance and time.\\
d. It is a vector quantity because it is calculated from speed and time.

What will be the change in velocity each second if acceleration is \(10 \mathrm{~m} / \mathrm{s} / \mathrm{s}\) ?\\
a. An acceleration of \(10 \backslash, \backslash \operatorname{text}\{\mathrm{~m} / \mathrm{s} / \mathrm{s}\}\) means that every second, the velocity increases by \(10 \backslash, \backslash \operatorname{text}\{\mathrm{~m} / \mathrm{s}\}\).\\
b. An acceleration of \(10 \backslash, \backslash \operatorname{text}\{\mathrm{~m} / \mathrm{s} / \mathrm{s}\}\) means that every second, the velocity decreases by \(10 \backslash, \backslash \operatorname{text}\{\mathrm{~m} / \mathrm{s}\}\).\\
c. An acceleration of \(10 \backslash, \backslash \operatorname{text}\{\mathrm{~m} / \mathrm{s} / \mathrm{s}\}\) means that every \(10 \backslash, \backslash \operatorname{text}\{\) seconds \(\}\), the velocity increases by \(10 \backslash, \backslash \operatorname{text}\{\mathrm{~m} / \mathrm{s}\}\).\\
d. An acceleration of \(10 \backslash, \backslash \operatorname{text}\{\mathrm{~m} / \mathrm{s} / \mathrm{s}\}\) means that every \(10 \backslash, \backslash \operatorname{text}\{\) seconds \(\}\), the velocity decreases by \(10 \backslash, \backslash \operatorname{text}\{\mathrm{~m} / \mathrm{s}\}\).

\section*{Snap Lab}
Measure the Acceleration of a Bicycle on a Slope In this lab you will take measurements to determine if the acceleration of a moving bicycle is constant. If the acceleration is constant, then the following relationships hold: \(\bar{v}=\frac{\Delta d}{\Delta t}=\frac{v_{0}+v_{\mathrm{f}}}{2}\) If \(v_{0}=0\), then \(v_{\mathrm{f}}=2 \bar{v}\) and \(\bar{a}=\frac{v_{\mathrm{f}}}{\Delta t}\)\\
You will work in pairs to measure and record data for a bicycle coasting down an incline on a smooth, gentle slope. The data will consist of distances traveled and elapsed times.

\begin{itemize}
  \item Find an open area to minimize the risk of injury during this lab.
  \item stopwatch
  \item measuring tape
  \item bicycle
\end{itemize}

\begin{enumerate}
  \item Find a gentle, paved slope, such as an incline on a bike path. The more gentle the slope, the more accurate your data will likely be.
  \item Mark uniform distances along the slope, such as \(5 \mathrm{~m}, 10 \mathrm{~m}\), etc.
  \item Determine the following roles: the bike rider, the timer, and the recorder. The recorder should create a data table to collect the distance and time data.
  \item Have the rider at the starting point at rest on the bike. When the timer calls Start, the timer starts the stopwatch and the rider begins coasting down the slope on the bike without pedaling.
  \item Have the timer call out the elapsed times as the bike passes each marked point. The recorder should record the times in the data table. It may be necessary to repeat the process to practice roles and make necessary adjustments.
  \item Once acceptable data has been recorded, switch roles. Repeat Steps 3-5 to collect a second set of data.
  \item Switch roles again to collect a third set of data.
  \item Calculate average acceleration for each set of distance-time data. If your result for \(a\) is not the same for different pairs of \(\Delta v\) and \(\Delta t\), then acceleration is not constant.
  \item Interpret your results.
\end{enumerate}

\section*{Teacher Support}
Teacher Support Explain that two factors that could prevent uniform acceleration are (i) friction between the tires and the pavement and friction in the bicycle axles, and (ii) air resistance. Discuss methods for minimizing these factors-e.g., selecting a smoother surface for the bike to coast, greasing the axles, etc. Explain that friction will only decrease acceleration, but air resistance to a tail wind would increase acceleration. Discuss why it would be difficult to study constant acceleration if students were to pedal the bicycle. Note that the given kinematic equation that is valid for constant acceleration, which\\
is presented at the start of the Snap Lab, will be presented in further detail in the following section.

Prior to the lab, investigate appropriate areas around the school that have gentle, uniform slopes. Should the number of bicycles be limited, consider conducting the lab as a whole class or in larger clusters. Ensure that the planned paths of student groups do not cross and that there is adequate space for riders to stop without risk of injury.

\section*{Grasp Check}
If you graph the average velocity ( \(y\)-axis) vs. the elapsed time ( \(x\)-axis), what would the graph look like if acceleration is uniform?\\
a. a horizontal line on the graph\\
b. a diagonal line on the graph\\
c. an upward-facing parabola on the graph\\
d. a downward-facing parabola on the graph

\section*{Check Your Understanding}
\section*{Teacher Support}
Teacher Support Use these questions to assess student achievement of the section's Learning Objectives. If students are struggling with a specific objective, these questions will help identify any gaps and direct students to the relevant content.\\
3.

What are three ways an object can accelerate?\\
a. By speeding up, maintaining constant velocity, or changing direction\\
b. By speeding up, slowing down, or changing direction\\
c. By maintaining constant velocity, slowing down, or changing direction\\
d. By speeding up, slowing down, or maintaining constant velocity\\
4.

What is the difference between average acceleration and instantaneous acceleration?\\
a. Average acceleration is the change in displacement divided by the elapsed time; instantaneous acceleration is the acceleration at a given point in time.\\
b. Average acceleration is acceleration at a given point in time; instantaneous acceleration is the change in displacement divided by the elapsed time.\\
c. Average acceleration is the change in velocity divided by the elapsed time; instantaneous acceleration is acceleration at a given point in time.\\
d. Average acceleration is acceleration at a given point in time; instantaneous acceleration is the change in velocity divided by the elapsed time.\\
5.

What is the rate of change of velocity called?\\
a. time\\
b. displacement\\
c. velocity\\
d. acceleration

\section*{3. Representing Acceleration ith Eq ations and Graphs}
\section*{Section Learning Objectives}
By the end of this section, you will be able to do the following:

\begin{itemize}
  \item Explain the kinematic equations related to acceleration and illustrate them with graphs
  \item Apply the kinematic equations and related graphs to problems involving acceleration
\end{itemize}

\section*{Teacher Support}
Teacher Support The Learning Objectives in this section will help your students master the following standards:

\begin{itemize}
  \item (4) Science concepts. The student knows and applies the laws governing motion in a variety of situations. The student is expected to:
  \item (A) generate and interpret graphs and charts describing different types of motion, including the use of real-time technology such as motion detectors or photogates;
  \item (B) describe and analyze motion in one dimension using equations with the concepts of distance, displacement, speed, average velocity, instantaneous velocity, and acceleration.
\end{itemize}

In addition, the High School Physics Laboratory Manual addresses content in this section in the lab titled: Acceleration, as well as the following standards:

\begin{itemize}
  \item (4) Science concepts. The student knows and applies the laws governing motion in a variety of situations. The student is expected to:
  \item (A) generate and interpret graphs and charts describing different types of motion, including the use of real-time technology such as motion detectors or photogates.
\end{itemize}

\section*{Section Key Terms}
\section*{Teacher Support}
Teacher Support [BL] Briefly review displacement, time, velocity, and acceleration; their variables, and their units.\\[0pt]
[OL][AL] Explain that this section introduces five equations that allow us to solve a wider range of problems than just finding acceleration from time and velocity. Review graphical analysis, including axes, algebraic signs, how to designate points on a coordinate plane, i.e., \((x, y)\), slopes, and intercepts. Explain that these equations can also be represented graphically.

\section*{How the Kinematic Equations are Related to Acceleration}
We are studying concepts related to motion: time, displacement, velocity, and especially acceleration. We are only concerned with motion in one dimension. The kinematic equations apply to conditions of constant acceleration and show how these concepts are related. Constant acceleration is acceleration that does not change over time. The first kinematic equation relates displacement \(d\), average velocity \(v\), and time \(t\).\\
\(d=d_{0}+\bar{v} t\)

\section*{3.4}
The initial displacement \(d_{0}\) is often 0 , in which case the equation can be written as \(v=\frac{d}{t}\)

This equation says that average velocity is displacement per unit time. We will express velocity in meters per second. If we graph displacement versus time, as in Figure 3.6, the slope will be the velocity. Whenever a rate, such as velocity, is represented graphically, time is usually taken to be the independent variable and is plotted along the \(x\) axis.

\begin{figure}[h]
\begin{center}
  \includegraphics[max width=\textwidth]{b9ef3717-e0c7-4b61-bf9f-f8eb1ba3da83-17}
\captionsetup{labelformat=empty}
\caption{Figure 3.6 The slope of displacement versus time is velocity.}
\end{center}
\end{figure}

The second kinematic equation, another expression for average velocity \(\bar{v}\), is simply the initial velocity plus the final velocity divided by two.\\
\(\bar{v}=\frac{v_{0}+v_{\mathrm{f}}}{2}\)

\section*{3.5}
Now we come to our main focus of this chapter; namely, the kinematic equations that describe motion with constant acceleration. In the third kinematic equation, acceleration is the rate at which velocity increases, so velocity at any point equals initial velocity plus acceleration multiplied by time\\
\(v=v_{0}+a t\) Also, if we start from rest \(\left(v_{0}=0\right)\), we can write \(a=\frac{v}{t}\)

\section*{3.6}
Note that this third kinematic equation does not have displacement in it. Therefore, if you do not know the displacement and are not trying to solve for a displacement, this equation might be a good one to use.

The third kinematic equation is also represented by the graph in Figure 3.7.

\begin{figure}[h]
\begin{center}
  \includegraphics[max width=\textwidth]{b9ef3717-e0c7-4b61-bf9f-f8eb1ba3da83-18(1)}
\captionsetup{labelformat=empty}
\caption{Figure 3.7 The slope of velocity versus time is acceleration.}
\end{center}
\end{figure}

The fourth kinematic equation shows how displacement is related to acceleration \(d=d_{0}+v_{0} t+\frac{1}{2} a t^{2}\).

\section*{3.7}
When starting at the origin, \(d_{0}=0\) and, when starting from rest, \(v_{0}=0\), in which case the equation can be written as\\
\(a=\frac{2 d}{t^{2}}\).\\
This equation tells us that, for constant acceleration, the slope of a plot of \(2 d\) versus \(t^{2}\) is acceleration, as shown in Figure 3.8.\\
\includegraphics[max width=\textwidth, center]{b9ef3717-e0c7-4b61-bf9f-f8eb1ba3da83-18}

Figure 3.8 When acceleration is constant, the slope of \(2 d\) versus \(t^{2}\) gives the acceleration.

The fifth kinematic equation relates velocity, acceleration, and displacement\\
\(v^{2}=v_{0}^{2}+2 a\left(d-d_{0}\right)\).\\
3.8

This equation is useful for when we do not know, or do not need to know, the time.

When starting from rest, the fifth equation simplifies to\\
\(a=\frac{v^{2}}{2 d}\).\\
According to this equation, a graph of velocity squared versus twice the displacement will have a slope equal to acceleration.

\begin{figure}[h]
\begin{center}
  \includegraphics[max width=\textwidth]{b9ef3717-e0c7-4b61-bf9f-f8eb1ba3da83-19}
\captionsetup{labelformat=empty}
\caption{Figure 3.9}
\end{center}
\end{figure}

Note that, in reality, knowns and unknowns will vary. Sometimes you will want to rearrange a kinematic equation so that the knowns are the values on the axes and the unknown is the slope. Sometimes the intercept will not be at the origin \((0,0)\). This will happen when \(v_{0}\) or \(d_{0}\) is not zero. This will be the case when the object of interest is already in motion, or the motion begins at some point other than at the origin of the coordinate system.

\section*{Teacher Support}
Teacher Support \([\mathrm{BL}]\) Be sure everyone is completely comfortable with the idea that velocity is displacement divided by the time during which the displacement occurs. Use everyday examples.\\[0pt]
[OL] Remind students that they studied velocity in earlier chapters. Relate the simplified equation\\
\(\bar{v}=\frac{d}{t}\)\\
to a graph of displacement versus time. Point out that the plot starts at \((0,0)\) because the initial velocity is zero. Pick a segment on the graph and explain how to find the slope at this segment. Explain that Figure 3.6 would show a straight line if velocity were changing; that is, if the object were accelerating. Show how the graph would change for both negative and positive acceleration.\\[0pt]
[OL] Build on the understanding of velocity to introduce acceleration. Contrast constant velocity with changing velocity. Use everyday examples from transportation, sports, or amusement park rides. Explain that Figure 3.7 represents constant acceleration because it is a straight line. Give examples of increasing, decreasing, and constant acceleration, and explain how each affects the shape of plots of velocity versus time.\\[0pt]
[OL][AL] Ask students why all three of the plots are straight lines. Refer to the graph of \(2 d\) vs \(t^{2}\) and explain why this is a straight line, whereas \(d\) vs \(t\) would be nonlinear.

\section*{Virtual Physics}
The Moving Man (Part 2) Look at the Moving Man simulation again and this time use the Charts view. Again, vary the velocity and acceleration by sliding the red and green markers along the scales. Keeping the velocity marker near zero will make the effect of acceleration more obvious. Observe how the graphs of position, velocity, and acceleration vary with time. Note which are linear plots and which are not.

Click to view content

\section*{Teacher Support}
Teacher Support Ask the students to click the Charts option and adjust the settings for the animation to reproduce the plots found in Figure 3.6 and Figure 3.7.

\section*{Grasp Check}
On a velocity versus time plot, what does the slope represent?\\
a. Acceleration\\
b. Displacement\\
c. Distance covered\\
d. Instantaneous velocity

\section*{Grasp Check}
On a position versus time plot, what does the slope represent?\\
a. Acceleration\\
b. Displacement\\
c. Distance covered\\
d. Instantaneous velocity

The kinematic equations are applicable when you have constant acceleration.

\begin{enumerate}
  \item \(d=d_{0}+\bar{v} t\), or \(\bar{v}=\frac{d}{t}\) when \(d_{0}=0\)
  \item \(\bar{v}=\frac{v_{0}+v_{\mathrm{f}}}{2}\)
  \item \(v=v_{0}+a t\), or \(a=\frac{v}{t}\) when \(v_{0}=0\)
  \item \(d=d_{0}+v_{0} t+\frac{1}{2} a t^{2}\), or \(a=\frac{2 d}{t^{2}}\) when \(d_{0}=0\) and \(v_{0}=0\)
  \item \(v^{2}=v_{0}^{2}+2 a\left(d-d_{0}\right)\), or \(a=\frac{2 d}{t^{2}}\) when \(d_{0}=0\) and \(v_{0}=0\)
\end{enumerate}

\section*{Teacher Support}
Teacher Support [OL] Go through the algebra to show how the kinematic equations can be simplified when some of the values are zero. Note that any motion starting or ending in a motionless state will simplify the equations.

\section*{Applying Kinematic Equations to Situations of Constant Acceleration}
Problem-solving skills are essential to success in a science and life in general. The ability to apply broad physical principles, which are often represented by equations, to specific situations is a very powerful form of knowledge. It is much more powerful than memorizing a list of facts. Analytical skills and problemsolving abilities can be applied to new situations, whereas a list of facts cannot be made long enough to contain every possible circumstance. Essential analytical skills will be developed by solving problems in this text and will be useful for understanding physics and science in general throughout your life.

Problem-Solving Steps While no single step-by-step method works for every problem, the following general procedures facilitate problem solving and make the answers more meaningful. A certain amount of creativity and insight are required as well.

\begin{enumerate}
  \item Examine the situation to determine which physical principles are involved. It is vital to draw a simple sketch at the outset. Decide which direction is positive and note that on your sketch.
  \item Identify the knowns: Make a list of what information is given or can be inferred from the problem statement. Remember, not all given information will be in the form of a number with units in the problem. If something starts from rest, we know the initial velocity is zero. If something stops, we know the final velocity is zero.
  \item Identify the unknowns: Decide exactly what needs to be determined in the problem.
  \item Find an equation or set of equations that can help you solve the problem. Your list of knowns and unknowns can help here. For example, if time is not needed or not given, then the fifth kinematic equation, which does not include time, could be useful.
  \item Insert the knowns along with their units into the appropriate equation and obtain numerical solutions complete with units. This step produces the numerical answer; it also provides a check on units that can help you find\\
errors. If the units of the answer are incorrect, then an error has been made.
  \item Check the answer to see if it is reasonable: Does it make sense? This final step is extremely important because the goal of physics is to accurately describe nature. To see if the answer is reasonable, check its magnitude, its sign, and its units. Are the significant figures correct?
\end{enumerate}

\section*{Summary of Problem Solving}
\begin{itemize}
  \item Determine the knowns and unknowns.
  \item Find an equation that expresses the unknown in terms of the knowns. More than one unknown means more than one equation is needed.
  \item Solve the equation or equations.
  \item Be sure units and significant figures are correct.
  \item Check whether the answer is reasonable.
\end{itemize}

\section*{Teacher Support}
Teacher Support [BL][OL][AL] Consider starting with the simplified summary of problem solving and then expand it to the more wordy description above. Stress that this is not a strict recipe that can solve all problems. Applying analytical skills is still required. If the preliminary analysis is correct and the knowns and unknowns are correctly sorted, the rest should be correct. Try applying the method to some non-mathematical everyday problems. Ask for suggestions.

\section*{Fun In Physics}
\section*{Drag Racing}
\begin{figure}[h]
\begin{center}
  \includegraphics[max width=\textwidth]{b9ef3717-e0c7-4b61-bf9f-f8eb1ba3da83-23}
\captionsetup{labelformat=empty}
\caption{Figure 3.10 Smoke rises from the tires of a dragster at the beginning of a drag race. (Lt. Col. William Thurmond. Photo courtesy of U.S. Army.)}
\end{center}
\end{figure}

The object of the sport of drag racing is acceleration. Period! The races take place from a standing start on a straight one-quarter-mile ( 402 m ) track. Usually two cars race side by side, and the winner is the driver who gets the car past the quarter-mile point first. At the finish line, the cars may be going more than 300 miles per hour ( \(134 \mathrm{~m} / \mathrm{s}\) ). The driver then deploys a parachute to bring the car to a stop because it is unsafe to brake at such high speeds. The cars, called dragsters, are capable of accelerating at \(26 \mathrm{~m} / \mathrm{s}^{2}\). By comparison, a typical sports car that is available to the general public can accelerate at about \(5 \mathrm{~m} / \mathrm{s}^{2}\).

Several measurements are taken during each drag race:

\begin{itemize}
  \item Reaction time is the time between the starting signal and when the front of the car crosses the starting line.
  \item Elapsed time is the time from when the vehicle crosses the starting line to when it crosses the finish line. The record is a little over 3 s .
  \item Speed is the average speed during the last 20 m before the finish line. The record is a little under 400 mph .
\end{itemize}

The video shows a race between two dragsters powered by jet engines. The actual race lasts about four seconds and is near the end of the video.

\section*{Teacher Support}
Teacher Support Explain the direction and magnitude of acceleration and velocity vectors before and after the braking chute is deployed. Explain that \(\frac{1}{4}\) mile is 440 yards. If any students are on the track team, you might ask them to describe this distance. Compare record times for this track event to 4 seconds.

A dragster crosses the finish line with a velocity of \(140 \backslash, \backslash \operatorname{text}\{\mathrm{~m} / \mathrm{s}\}\). Assuming the vehicle maintained a constant acceleration from start to finish, what was its average velocity for the race?\\
a. \(0 \backslash, \backslash \operatorname{text}\{\mathrm{~m} / \mathrm{s}\}\)\\
b. \(35 \backslash, \backslash \operatorname{text}\{\mathrm{~m} / \mathrm{s}\}\)\\
c. \(70 \backslash, \backslash \operatorname{text}\{\mathrm{~m} / \mathrm{s}\}\)\\
d. \(140 \backslash, \backslash \operatorname{text}\{\mathrm{~m} / \mathrm{s}\}\)

\section*{Worked Example}
Acceleration of a Dragster The time it takes for a dragster to cross the finish line is unknown. The dragster accelerates from rest at \(26 \mathrm{~m} / \mathrm{s}^{2}\) for a quarter mile \((0.250 \mathrm{mi})\). What is the final velocity of the dragster?

\section*{Strategy}
The equation \(v^{2}=v_{0}^{2}+2 a\left(d-d_{0}\right)\) is ideally suited to this task because it gives the velocity from acceleration and displacement, without involving the time.

Solution

\begin{enumerate}
  \item Convert miles to meters.
\end{enumerate}

\begin{itemize}
  \item \((0.250 \mathrm{mi}) \times \frac{1609 \mathrm{~m}}{1 \mathrm{mi}}=402 \mathrm{~m}\)\\
3.9
\end{itemize}

\begin{enumerate}
  \setcounter{enumi}{1}
  \item Identify the known values. We know that \(v_{0}=0\) since the dragster starts from rest, and we know that the distance traveled, \(d-d_{0}\) is 402 m . Finally, the acceleration is constant at \(a=26.0 \mathrm{~m} / \mathrm{s}^{2}\).
  \item Insert the knowns into the equation \(v^{2}=v_{0}^{2}+2 a\left(d-d_{0}\right)\) and solve for \(v\).
\end{enumerate}

\begin{itemize}
  \item \(v^{2}=0+2\left(26.0 \frac{\mathrm{~m}}{\mathrm{~s}^{2}}\right)(402 \mathrm{~m})=2.09 \times 10^{4} \frac{\mathrm{~m}^{2}}{\mathrm{~s}^{2}}\)\\
3.10
\end{itemize}

Taking the square root gives us \(v=\sqrt{2.09 \times 10^{4} \frac{\mathrm{~m}^{2}}{\mathrm{~s}^{2}}}=145 \frac{\mathrm{~m}}{\mathrm{~s}}\).\\
Discussion\\
\(145 \mathrm{~m} / \mathrm{s}\) is about \(522 \mathrm{~km} /\) hour or about \(324 \mathrm{mi} / \mathrm{h}\), but even this breakneck speed is short of the record for the quarter mile. Also, note that a square root has two values. We took the positive value because we know that the velocity must be in the same direction as the acceleration for the answer to make physical sense.

An examination of the equation \(v^{2}=v_{0}^{2}+2 a\left(d-d_{0}\right)\) can produce further insights into the general relationships among physical quantities:

\begin{itemize}
  \item The final velocity depends on the magnitude of the acceleration and the distance over which it applies.
  \item For a given acceleration, a car that is going twice as fast does not stop in twice the distance - it goes much further before it stops. This is why, for example, we have reduced speed zones near schools.
\end{itemize}

\section*{Teacher Support}
Teacher Support [OL] Work through the problem-solving steps with the student.

\begin{enumerate}
  \item What is the goal?
  \item What is known and unknown?
  \item Which equation expresses the unknown in terms of the knowns?
  \item Solve for the unknown.
  \item Insert known values.
  \item Calculate.
  \item Checking that answer is reasonable and has the correct units, sign, and significant figures.
\end{enumerate}

Repeat for the second sample problem.\\[0pt]
[AL] Initiate a discussion on variation in gravity. Compare gravity on Earth to gravity on the moon. Explain the difference between \(g\) and constants that are the same everywhere in the universe, such as the speed of light.

\section*{Practice Problems}
6.

Dragsters can reach a top speed of \(145 \mathrm{~m} / \mathrm{s}\) in only 4.45 s . Calculate the average acceleration for such a dragster.\\
a. \(-32.6 \mathrm{~m} / \mathrm{s}^{2}\)\\
b. \(0 \mathrm{~m} / \mathrm{s}^{2}\)\\
c. \(32.6 \mathrm{~m} / \mathrm{s}^{2}\)\\
d. \(145 \mathrm{~m} / \mathrm{s}^{2}\)

\section*{7.}
An Olympic-class sprinter starts a race with an acceleration of \(4.50 \mathrm{~m} / \mathrm{s}^{2}\). Assuming she can maintain that acceleration, what is her speed 2.40 s later?\\
a. \(4.50 \mathrm{~m} / \mathrm{s}\)\\
b. \(10.8 \mathrm{~m} / \mathrm{s}\)\\
c. \(19.6 \mathrm{~m} / \mathrm{s}\)\\
d. \(44.1 \mathrm{~m} / \mathrm{s}\)

\section*{Constant Acceleration}
In many cases, acceleration is not uniform because the force acting on the accelerating object is not constant over time. A situation that gives constant acceleration is the acceleration of falling objects. When air resistance is not a factor, all objects near Earth's surface fall with an acceleration of about 9.80 \(\mathrm{m} / \mathrm{s}^{2}\). Although this value decreases slightly with increasing altitude, it may be assumed to be essentially constant. The value of \(9.80 \mathrm{~m} / \mathrm{s}^{2}\) is labeled \(g\) and is referred to as acceleration due to gravity. Gravity is the force that causes nonsupported objects to accelerate downward-or, more precisely, toward the center of Earth. The magnitude of this force is called the weight of the object and is given by \(m g\) where \(m\) is the mass of the object (in kg ). In places other than on Earth, such as the Moon or on other planets, \(g\) is not \(9.80 \mathrm{~m} / \mathrm{s}^{2}\), but takes on other values. When using \(g\) for the acceleration \(a\) in a kinematic equation, it is usually given a negative sign because the acceleration due to gravity is downward.

\section*{Work In Physics}
\section*{Effects of Rapid Acceleration}
\begin{figure}[h]
\begin{center}
  \includegraphics[max width=\textwidth]{b9ef3717-e0c7-4b61-bf9f-f8eb1ba3da83-27}
\captionsetup{labelformat=empty}
\caption{Figure 3.11 Astronauts train using G Force Simulators. (NASA)}
\end{center}
\end{figure}

When in a vehicle that accelerates rapidly, you experience a force on your entire body that accelerates your body. You feel this force in automobiles and slightly more on amusement park rides. For example, when you ride in a car that turns, the car applies a force on your body to make you accelerate in the direction in which the car is turning. If enough force is applied, you will accelerate at 9.80 \(\mathrm{m} / \mathrm{s}^{2}\). This is the same as the acceleration due to gravity, so this force is called one G .

One G is the force required to accelerate an object at the acceleration due to gravity at Earth's surface. Thus, one G for a paper cup is much less than one G for an elephant, because the elephant is much more massive and requires a\\
greater force to make it accelerate at \(9.80 \mathrm{~m} / \mathrm{s}^{2}\). For a person, a G of about 4 is so strong that his or her face will distort as the bones accelerate forward through the loose flesh. Other symptoms at extremely high Gs include changes in vision, loss of consciousness, and even death. The space shuttle produces about 3 Gs during takeoff and reentry. Some roller coasters and dragsters produce forces of around 4 Gs for their occupants. A fighter jet can produce up to 12 Gs during a sharp turn.

Astronauts and fighter pilots must undergo G-force training in simulators. The video shows the experience of several people undergoing this training.

People, such as astronauts, who work with G forces must also be trained to experience zero G-also called free fall or weightlessness-which can cause queasiness. NASA has an aircraft that allows it occupants to experience about 25 s of free fall. The aircraft is nicknamed the Vomit Comet.

\section*{Teacher Support}
Teacher Support Mention that, later in this course, students will encounter some interesting concepts related to gravity and acceleration when studying the general theory of relativity. In part, the theory is based on the idea that gravity cannot be distinguished from acceleration, either experientially or mathematically. When in an upward-bound elevator, can you really tell whether you are accelerating or whether gravity has suddenly become stronger?

\section*{Grasp Check}
A common way to describe acceleration is to express it in multiples of \(g\), Earth's gravitational acceleration. If a dragster accelerates at a rate of \(39.2 \mathrm{~m} / \mathrm{s}^{2}\), how many \(g\) 's does the driver experience?\\
a. \(1.5 g\)\\
b. \(4.0 g\)\\
c. \(10.5 g\)\\
d. \(24.5 g\)

\section*{Worked Example}
Falling Objects A person standing on the edge of a high cliff throws a rock straight up with an initial velocity \(v_{0}\) of \(13 \mathrm{~m} / \mathrm{s}\).\\
(a) Calculate the position and velocity of the rock at \(1.00,2.00\), and 3.00 seconds after it is thrown. Ignore the effect of air resistance.

\section*{Strategy}
Sketch the initial velocity and acceleration vectors and the axes.

\begin{figure}[h]
\begin{center}
  \includegraphics[max width=\textwidth]{b9ef3717-e0c7-4b61-bf9f-f8eb1ba3da83-29}
\captionsetup{labelformat=empty}
\caption{Figure 3.12 Initial conditions for rock thrown straight up.}
\end{center}
\end{figure}

List the knowns: time \(t=1.00 \mathrm{~s}, 2.00 \mathrm{~s}\), and 3.00 s ; initial velocity \(v_{0}=13 \mathrm{~m} / \mathrm{s}\); acceleration \(a=g=-9.80 \mathrm{~m} / \mathrm{s}^{2}\); and position \(y_{0}=0 \mathrm{~m}\)

List the unknowns: \(y_{1}, y_{2}\), and \(y_{3} ; v_{1}, v_{2}\), and \(v_{3}\)-where \(1,2,3\) refer to times \(1.00 \mathrm{~s}, 2.00 \mathrm{~s}\), and 3.00 s

Choose the equations.\\
\(d=d_{0}+v_{0} t+\frac{1}{2} a t^{2}\) becomes \(y=y_{0}+v_{0} t-\frac{1}{2} g t^{2}\)\\
3.11\\
\(v=v_{0}+a t\) becomes \(v=v_{0}+-g t\)\\
3.12

These equations describe the unknowns in terms of knowns only.\\
Solution\\
\(y_{1}=0+(13.0 \mathrm{~m} / \mathrm{s})(1.00 \mathrm{~s})+\frac{\left(-9.80 \mathrm{~m} / \mathrm{s}^{2}\right)(1.00 \mathrm{~s})^{2}}{2}=8.10 \mathrm{~m}\)\\
\(y_{2}=0+(13.0 \mathrm{~m} / \mathrm{s})(2.00 \mathrm{~s})+\frac{\left(-9.80 \mathrm{~m} / \mathrm{s}^{2}\right)(2.00 \mathrm{~s})^{2}}{2}=6.40 \mathrm{~m}\)\\
\(y_{3}=0+(13.0 \mathrm{~m} / \mathrm{s})(3.00 \mathrm{~s})+\frac{\left(-9.80 \mathrm{~m} / \mathrm{s}^{2}\right)(3.00 \mathrm{~s})^{2}}{2}=-5.10 \mathrm{~m}\)\\
\(v_{1}=13.0 \mathrm{~m} / \mathrm{s}+\left(-9.80 \mathrm{~m} / \mathrm{s}^{2}\right)(1.00 \mathrm{~s})=3.20 \mathrm{~m} / \mathrm{s}\)\\
\(v_{2}=13.0 \mathrm{~m} / \mathrm{s}+\left(-9.80 \mathrm{~m} / \mathrm{s}^{2}\right)(2.00 \mathrm{~s})=-6.60 \mathrm{~m} / \mathrm{s}\)\\
\(v_{3}=13.0 \mathrm{~m} / \mathrm{s}+\left(-9.80 \mathrm{~m} / \mathrm{s}^{2}\right)(3.00 \mathrm{~s})=-16.4 \mathrm{~m} / \mathrm{s}\)\\
Discussion

The first two positive values for \(y\) show that the rock is still above the edge of the cliff, and the third negative \(y\) value shows that it has passed the starting point and is below the cliff. Remember that we set up to be positive. Any position with a positive value is above the cliff, and any velocity with a positive value is an upward velocity. The first value for v is positive, so the rock is still on the way up. The second and third values for v are negative, so the rock is on its way down.\\
(b) Make graphs of position versus time, velocity versus time, and acceleration versus time. Use increments of 0.5 s in your graphs.

\section*{Strategy}
Time is customarily plotted on the \(x\)-axis because it is the independent variable. Position versus time will not be linear, so calculate points for \(0.50 \mathrm{~s}, 1.50 \mathrm{~s}\), and 2.50 s . This will give a curve closer to the true, smooth shape.

Solution\\
The three graphs are shown in Figure 3.13.\\
\includegraphics[max width=\textwidth, center]{b9ef3717-e0c7-4b61-bf9f-f8eb1ba3da83-31(1)}\\
\includegraphics[max width=\textwidth, center]{b9ef3717-e0c7-4b61-bf9f-f8eb1ba3da83-31}

Figure 3.13

\section*{Discussion}
\begin{itemize}
  \item \(y\) vs. \(t\) does not represent the two-dimensional parabolic path of a trajectory. The path of the rock is straight up and straight down. The slope of a line tangent to the curve at any point on the curve equals the velocity at that point-i.e., the instantaneous velocity.
  \item Note that the \(v\) vs. \(t\) line crosses the vertical axis at the initial velocity and crosses the horizontal axis at the time when the rock changes direction and begins to fall back to Earth. This plot is linear because acceleration is constant.
  \item The \(a\) vs. \(t\) plot also shows that acceleration is constant; that is, it does not change with time.
\end{itemize}

\section*{Teacher Support}
Teacher Support Prior to solving the problem, have students consider the following questions:

\begin{enumerate}
  \item What is the goal?
  \item What is known and unknown?
  \item Which equation expresses the unknown in terms of the knowns?
\end{enumerate}

After solving the problem, have students check that the answer is reasonable and has the correct units, sign, and significant figures.

\section*{Practice Problems}
8.

A cliff diver pushes off horizontally from a cliff and lands in the ocean \(2.00 \backslash, \backslash\) text \(\{\mathrm{s}\}\) later. How fast was he going when he entered the water?\\
a. \(0 \backslash, \backslash \operatorname{text}\{\mathrm{~m} / \mathrm{s}\}\)\\
b. \(19.0 \backslash, \backslash \operatorname{text}\{\mathrm{~m} / \mathrm{s}\}\)\\
c. \(19.6 \backslash, \backslash \operatorname{text}\{\mathrm{~m} / \mathrm{s}\}\)\\
d. \(20.0 \backslash, \backslash \operatorname{text}\{\mathrm{~m} / \mathrm{s}\}\)\\
9.

A girl drops a pebble from a high cliff into a lake far below. She sees the splash of the pebble hitting the water \(2.00 \backslash, \backslash \operatorname{text}\{\mathrm{~s}\}\) later. How fast was the pebble going when it hit the water?\\
a. \(9.80 \backslash, \backslash \operatorname{text}\{\mathrm{~m} / \mathrm{s}\}\)\\
b. \(10.0 \backslash, \backslash \operatorname{text}\{\mathrm{~m} / \mathrm{s}\}\)\\
c. \(19.6 \backslash, \backslash \operatorname{text}\{\mathrm{~m} / \mathrm{s}\}\)\\
d. \(20.0 \backslash, \backslash \operatorname{text}\{\mathrm{~m} / \mathrm{s}\}\)

\section*{Check Your Understanding}
\section*{Teacher Support}
Teacher Support Use these questions to assess students' achievement of the sections Learning Objectives. If students are struggling with a specific objective, the formative assessment will help identify which objective is the problem so that you can direct students to the relevant content.\\
10.

Identify the four variables found in the kinematic equations.\\
a. Displacement, Force, Mass, and Time\\
b. Acceleration, Displacement, Time, and Velocity\\
c. Final Velocity, Force, Initial Velocity, and Mass\\
d. Acceleration, Final Velocity, Force, and Initial Velocity\\
11.

Which of the following steps is always required to solve a kinematics problem?\\
a. Find the force acting on the body.\\
b. Find the acceleration of a body.\\
c. Find the initial velocity of a body.\\
d. Find a suitable kinematic equation and then solve for the unknown quantity.\\
12.

Which of the following provides a correct answer for a problem that can be solved using the kinematic equations?\\
a. A body starts from rest and accelerates at \(4 \backslash, \backslash \operatorname{text}\{\mathrm{~m} / \mathrm{s}\}^{\wedge} 2 \backslash!\) for \(2 \backslash, \backslash \operatorname{text}\{\mathrm{~s}\}\). The body's final velocity is \(8 \backslash, \backslash \operatorname{text}\{\mathrm{~m} / \mathrm{s}\}\).\\
b. A body starts from rest and accelerates at \(4 \backslash, \backslash \operatorname{text}\{\mathrm{~m} / \mathrm{s}\}^{\wedge} 2 \backslash\) ! for \(2 \backslash, \backslash \operatorname{text}\{\mathrm{~s}\}\). The body's final velocity is \(16 \backslash, \backslash \operatorname{text}\{\mathrm{~m} / \mathrm{s}\}\).\\
c. A body with a mass of \(2 \backslash, \backslash \operatorname{text}\{\mathrm{~kg}\}\) is acted upon by a force of \(4 \backslash, \backslash \operatorname{text}\{\mathrm{~N}\}\). The acceleration of the body is \(2 \backslash, \backslash \operatorname{text}\{\mathrm{~m} / \mathrm{s}\}^{\wedge} 2 \backslash!\).\\
d. A body with a mass of \(2 \backslash, \backslash \operatorname{text}\{\mathrm{~kg}\}\) is acted upon by a force of \(4 \backslash, \backslash \operatorname{text}\{\mathrm{~N}\}\). The acceleration of the body is \(0.5 \backslash, \backslash \operatorname{text}\{\mathrm{~m} / \mathrm{s}\}^{\wedge} 2 \backslash!\).

\section*{Ke Terms}
acceleration due to gravity acceleration of an object that is subject only to the force of gravity; near Earth's surface this acceleration is \(9.80 \mathrm{~m} / \mathrm{s}^{2}\)\\
average acceleration change in velocity divided by the time interval over which it changed\\
constant acceleration acceleration that does not change with respect to time\\
instantaneous acceleration rate of change of velocity at a specific instant in time\\
kinematic equations the five equations that describe motion in terms of time, displacement, velocity, and acceleration\\
negative acceleration acceleration in the negative direction

\section*{Ke Eq ations}
\subsection*{3.1 Acceleration}
\subsection*{3.2 Representing Acceleration with Equations and Graphs}
\section*{Section S mmar}
\subsection*{3.1 Acceleration}
\begin{itemize}
  \item Acceleration is the rate of change of velocity and may be negative or positive.
  \item Average acceleration is expressed in \(\mathrm{m} / \mathrm{s}^{2}\) and, in one dimension, can be calculated using \(\bar{a}=\frac{\Delta v}{\Delta t}=\frac{v_{\mathrm{f}}-v_{0}}{t_{\mathrm{f}}-t_{o}}\).
\end{itemize}

\subsection*{3.2 Representing Acceleration with Equations and Graphs}
\begin{itemize}
  \item The kinematic equations show how time, displacement, velocity, and acceleration are related for objects in motion.
  \item In general, kinematic problems can be solved by identifying the kinematic equation that expresses the unknown in terms of the knowns.
  \item Displacement, velocity, and acceleration may be displayed graphically versus time.
\end{itemize}

\end{document}