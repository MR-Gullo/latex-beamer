\documentclass[10pt]{article}
\usepackage[utf8]{inputenc}
\usepackage[T1]{fontenc}
\usepackage{graphicx}
\usepackage[export]{adjustbox}
\graphicspath{ {./images/} }
\usepackage{caption}
\usepackage{amsmath}
\usepackage{amsfonts}
\usepackage{amssymb}
\usepackage[version=4]{mhchem}
\usepackage{stmaryrd}

\begin{document}
\captionsetup{singlelinecheck=false}
\begin{figure}[h]
\begin{center}
  \includegraphics[max width=\textwidth]{a3409b21-5bae-43c8-944a-d194fdc3524c-01}
\captionsetup{labelformat=empty}
\caption{Figure 20.1 The magnificent spectacle of the Aurora Borealis, or northern lights, glows in the northern sky above Bear Lake near Eielson Air Force Base, Alaska. Shaped by Earth's magnetic field, this light is produced by radiation spewed from solar storms. (credit: Senior Airman Joshua Strang, Flickr)}
\end{center}
\end{figure}

\section*{Chapter Outline}
20.1 Magnetic Fields, Field Lines, and Force\\
20.2 Motors, Generators, and Transformers\\
20.3 Electromagnetic Induction

\section*{Introduction}
\section*{Teacher Support}
Teacher Support Ask students what they know about magnets. Discuss the two poles of magnets that either attract each other or repel each other. Point out that magnetic force acts at a distance, which is similar to the electrostatic force studied earlier.

Review electric dipoles and the electric field that they create. Emphasize that dipoles have two charges, or poles, and that these poles will repulse, or attract, the same, or opposite, pole of another dipole. Also discuss how electric dipoles orient themselves to be parallel to electric field lines.

You may have encountered magnets for the first time as a small child playing with magnetic toys or refrigerator magnets. At the time, you likely noticed that\\
two magnets that repulse each other will attract each other if you flip one of them around. The force that acts across the air gaps between magnets is the same force that creates wonders such as the Aurora Borealis. In fact, magnetic effects pervade our lives in myriad ways, from electric motors to medical imaging and computer memory. In this chapter, we introduce magnets and learn how they work and how magnetic fields and electric currents interact.

\subsection*{20.2 Motors, Generators, and Transformers}
\section*{Section Learning Objectives}
By the end of this section, you will be able to do the following:

\begin{itemize}
  \item Explain how electric motors, generators, and transformers work
  \item Explain how commercial electric power is produced, transmitted, and distributed
\end{itemize}

\section*{Teacher Support}
Teacher Support The learning objectives in this section will help your students master the following standards:

\begin{itemize}
  \item (5) The student knows the nature of forces in the physical world. The student is expected to:
  \item ( G ) investigate and describe the relationship between electric and magnetic fields in applications such as generators, motors, and transformers.
\end{itemize}

\section*{Section Key Terms}
\section*{Electric Motors, Generators, and Transformers}
As we learned previously, a current-carrying wire in a magnetic field experiences a force-recall \(F=I \ell B \sin \theta\). Electric motors, which convert electrical energy into mechanical energy, are the most common application of magnetic force on current-carrying wires. Motors consist of loops of wire in a magnetic field. When current is passed through the loops, the magnetic field exerts a torque on the loops, which rotates a shaft. Electrical energy is converted to mechanical work in the process. Figure 20.23 shows a schematic drawing of an electric motor.

\begin{figure}[h]
\begin{center}
  \includegraphics[max width=\textwidth]{a3409b21-5bae-43c8-944a-d194fdc3524c-03}
\captionsetup{labelformat=empty}
\caption{Figure 20.23 Torque on a current loop. A vertical loop of wire in a horizontal magnetic field is attached to a vertical shaft. When current is passed through the wire loop, torque is exerted on it, making it turn the shaft.}
\end{center}
\end{figure}

Let us examine the force on each segment of the loop in Figure 20.23 to find the torques produced about the axis of the vertical shaft-this will lead to a useful equation for the torque on the loop. We take the magnetic field to be uniform over the rectangular loop, which has width \(w\) and height \(\ell\), as shown in the figure. First, consider the force on the top segment of the loop. To determine the direction of the force, we use the right-hand rule. The current goes from left to right into the page, and the magnetic field goes from left to right in the plane of the page. Curl your right fingers from the current vector to the magnetic field vector and your right thumb points down. Thus, the force on the top segment is downward, which produces no torque on the shaft. Repeating this analysis for the bottom segment-neglect the small gap where the lead wires go out-shows that the force on the bottom segment is upward, again producing no torque on the shaft.

Consider now the left vertical segment of the loop. Again using the right-hand rule, we find that the force exerted on this segment is perpendicular to the magnetic field, as shown in Figure 20.23. This force produces a torque on the shaft. Repeating this analysis on the right vertical segment of the loop shows that the force on this segment is in the direction opposite that of the force on the left segment, thereby producing an equal torque on the shaft. The total torque on the shaft is thus twice the toque on one of the vertical segments of the loop.

To find the magnitude of the torque as the wire loop spins, consider Figure 20.24, which shows a view of the wire loop from above. Recall that torque is defined as \(\tau=r F \sin \theta\), where \(F\) is the applied force, \(r\) is the distance from the pivot to where the force is applied, and is the angle between \(r\) and \(F\). Notice that, as the loop spins, the current in the vertical loop segments is always perpendicular to the magnetic field. Thus, the equation \(F=I \ell B \sin \theta\) gives the magnitude of the force on each vertical segment as \(F=I \ell B\). The distance \(r\) from the shaft to where this force is applied is \(w / 2\), so the torque created by this force is\\
\(\tau_{\text {segment }}=r F \sin \theta=w / 2 I \ell B \sin \theta=(w / 2) I \ell B \sin \theta\).\\
20.10

Because there are two vertical segments, the total torque is twice this, or\\
\(\tau=w I \ell B \sin \theta\).\\
20.11

If we have a multiple loop with \(N\) turns, we get \(N\) times the torque of a single loop. Using the fact that the area of the loop is \(A=w \ell\); the expression for the torque becomes\\
\(\tau=N I A B \sin \theta\).\\
20.12

This is the torque on a current-carrying loop in a uniform magnetic field. This equation can be shown to be valid for a loop of any shape.

\begin{figure}[h]
\begin{center}
  \includegraphics[max width=\textwidth]{a3409b21-5bae-43c8-944a-d194fdc3524c-05(1)}
\captionsetup{labelformat=empty}
\caption{Figure 20.24 View from above of the wire loop from Figure 20.23. The magnetic field generates a force \(F\) on each vertical segment of the wire loop, which generates a torque on the shaft. Notice that the currents \(I_{\text {in }}\) and \(I_{\text {out }}\) have the same magnitude because they both represent the current flowing in the wire loop, but \(I_{\text {in }}\) flows into the page and \(I_{\text {out }}\) flows out of the page.}
\end{center}
\end{figure}

From the equation \(\tau=N I A B \sin \theta\), we see that the torque is zero when \(\theta=0\). As the wire loop rotates, the torque increases to a maximum positive torque of \(w \ell B\) when \(\theta=90^{\circ}\). The torque then decreases back to zero as the wire loop rotates to \(\theta=180^{\circ}\). From \(\theta=180^{\circ}\) to \(\theta=360^{\circ}\), the torque is negative. Thus, the torque changes sign every half turn, so the wire loop will oscillate back and forth.

For the coil to continue rotating in the same direction, the current is reversed as the coil passes through \(\theta=0\) and \(\theta=180^{\circ}\) using automatic switches called brushes, as shown in Figure 20.25.

\begin{figure}[h]
\begin{center}
  \includegraphics[max width=\textwidth]{a3409b21-5bae-43c8-944a-d194fdc3524c-05}
\captionsetup{labelformat=empty}
\caption{Figure 20.25 (a) As the angular momentum of the coil carries it through \(\theta=0\), the brushes reverse the current and the torque remains clockwise. (b) The coil rotates continuously in the clockwise direction, with the current reversing each half revolution to maintain the clockwise torque.}
\end{center}
\end{figure}

Consider now what happens if we run the motor in reverse; that is, we attach a handle to the shaft and mechanically force the coil to rotate within the magnetic field, as shown in Figure 20.26. As per the equation \(F=q v B \sin \theta\)-where \(\theta\) is the angle between the vectors \(\vec{v}\) and \(\vec{B}\)-charges in the wires of the loop experience a magnetic force because they are moving in a magnetic field. Again using the right-hand rule, where we curl our fingers from vector \(\vec{v}\) to vector \(\vec{B}\), we find that charges in the top and bottom segments feel a force perpendicular to the wire, which does not cause a current. However, charges in the vertical wires experience forces parallel to the wire, causing a current to flow through the wire and through an external circuit if one is connected. A device such as this that converts mechanical energy into electrical energy is called a generator.

\begin{figure}[h]
\begin{center}
  \includegraphics[max width=\textwidth]{a3409b21-5bae-43c8-944a-d194fdc3524c-06}
\captionsetup{labelformat=empty}
\caption{Figure 20.26 When this coil is rotated through one-fourth of a revolution, the magnetic flux \(\Phi\) changes from its maximum to zero, inducing an emf, which drives a current through an external circuit.}
\end{center}
\end{figure}

Because current is induced only in the side wires, we can find the induced emf by only considering these wires. As explained in Induced Current in a Wire, motional emf in a straight wire moving at velocity \(v\) through a magnetic field \(B\) is \(E=B \ell v\), where the velocity is perpendicular to the magnetic field. In the generator, the velocity makes an angle \(\theta\) with \(B\) (see Figure 20.27), so the velocity component perpendicular to \(B\) is \(v \sin \theta\). Thus, in this case, the emf induced on each vertical wire segment is \(E=B \ell v \sin \theta\), and they are in the same direction. The total emf around the loop is then\\
\(E=2 B \ell v \sin \theta\).\\
20.13

Although this expression is valid, it does not give the emf as a function of time. To find how the emf evolves in time, we assume that the coil is rotated at a constant angular velocity \(\omega\). The angle \(\theta\) is related to the angular velocity by \(\theta=\omega t\), so that\\
\(E=2 B \ell v \sin \omega t\).\\
20.14

Recall that tangential velocity \(v\) is related to angular velocity \(\omega\) by \(v=r \omega\). Here, \(r=w / 2\), so that \(v=(w / 2) \omega\) and\\
\(E=2 B \ell\left(\frac{w}{2} \omega\right) \sin \omega t=B \ell w \omega \sin \omega t\).\\
20.15

Noting that the area of the loop is \(A=\ell w\) and allowing for \(N\) wire loops, we find that\\
\(E=N A B \omega \sin \omega t\)\\
20.16\\
is the emf induced in a generator coil of \(N\) turns and area \(A\) rotating at a constant angular velocity \(\omega\) in a uniform magnetic field \(B\). This can also be expressed as\\
\(E=E_{0} \sin \omega t\)\\
20.17\\
where\\
\(E_{0}=N A B \omega\)\\
20.18\\
is the maximum (peak) emf.\\
\includegraphics[max width=\textwidth, center]{a3409b21-5bae-43c8-944a-d194fdc3524c-07}

Figure 20.27 The instantaneous velocity of the vertical wire segments makes an angle \(\theta\) with the magnetic field. The velocity is shown in the figure by the green\\
arrow, and the angle \(\theta\) is indicated.\\
Figure 20.28 shows a generator connected to a light bulb and a graph of the emf vs. time. Note that the emf oscillates from a positive maximum of \(E_{0}\) to a negative maximum of \(-E_{0}\). In between, the emf goes through zero, which means that zero current flows through the light bulb at these times. Thus, the light bulb actually flickers on and off at a frequency of \(2 f\), because there are two zero crossings per period. Since alternating current such as this is used in homes around the world, why do we not notice the lights flickering on and off? In the United States, the frequency of alternating current is 60 Hz , so the lights flicker on and off at a frequency of 120 Hz . This is faster than the refresh rate of the human eye, so you don't notice the flicker of the lights. Also, other factors prevent various different types of light bulbs from switching on and off so fast, so the light output is smoothed out a bit.

\begin{figure}[h]
\begin{center}
  \includegraphics[max width=\textwidth]{a3409b21-5bae-43c8-944a-d194fdc3524c-08}
\captionsetup{labelformat=empty}
\caption{Figure 20.28 The emf of a generator is sent to a light bulb with the system of rings and brushes shown. The graph gives the emf of the generator as a function of time. \(E_{0}\) is the peak emf. The period is \(T=1 / f=2 \pi / \omega\), where \(f\) is the frequency at which the coil is rotated in the magnetic field.}
\end{center}
\end{figure}

\section*{Virtual Physics}
Generator Click to view content\\
Use this simulation to discover how an electrical generator works. Control the water supply that makes a water wheel turn a magnet. This induces an emf in a nearby wire coil, which is used to light a light bulb. You can also replace the light bulb with a voltmeter, which allows you to see the polarity of the voltage, which changes from positive to negative.

\section*{Grasp Check}
Set the number of wire loops to three, the bar-magnet strength to about 50 percent, and the loop area to 100 percent. Note the maximum voltage on the voltmeter. Assuming that one major division on the voltmeter is 5 V , what is\\
the maximum voltage when using only a single wire loop instead of three wire loops?\\
a. 5 V\\
b. 15 V\\
c. 125 V\\
d. 53 V

In real life, electric generators look a lot different than the figures in this section, but the principles are the same. The source of mechanical energy that turns the coil can be falling water-hydropower-steam produced by the burning of fossil fuels, or the kinetic energy of wind. Figure 20.29 shows a cutaway view of a steam turbine; steam moves over the blades connected to the shaft, which rotates the coil within the generator.

\begin{figure}[h]
\begin{center}
  \includegraphics[max width=\textwidth]{a3409b21-5bae-43c8-944a-d194fdc3524c-09}
\captionsetup{labelformat=empty}
\caption{Figure 20.29 Steam turbine generator. The steam produced by burning coal impacts the turbine blades, turning the shaft which is connected to the generator. (credit: Nabonaco, Wikimedia Commons)}
\end{center}
\end{figure}

Another very useful and common device that exploits magnetic induction is called a transformer. Transformers do what their name implies - they transform\\
voltages from one value to another; the term voltage is used rather than emf because transformers have internal resistance. For example, many cell phones, laptops, video games, power tools, and small appliances have a transformer built into their plug-in unit that changes 120 V or 240 V AC into whatever voltage the device uses. Figure 20.30 shows two different transformers. Notice the wire coils that are visible in each device. The purpose of these coils is explained below.

\begin{figure}[h]
\begin{center}
  \includegraphics[max width=\textwidth]{a3409b21-5bae-43c8-944a-d194fdc3524c-10}
\captionsetup{labelformat=empty}
\caption{Figure 20.30 On the left is a common laminated-core transformer, which is widely used in electric power transmission and electrical appliances. On the right is a toroidal transformer, which is smaller than the laminated-core transformer for the same power rating but is more expensive to make because of the equipment required to wind the wires in the doughnut shape.}
\end{center}
\end{figure}

Figure 20.31 shows a laminated-coil transformer, which is based on Faraday's law of induction and is very similar in construction to the apparatus Faraday used to demonstrate that magnetic fields can generate electric currents. The two wire coils are called the primary and secondary coils. In normal use, the input voltage is applied across the primary coil, and the secondary produces the transformed output voltage. Not only does the iron core trap the magnetic field created by the primary coil, but also its magnetization increases the field strength, which is analogous to how a dielectric increases the electric field strength in a capacitor. Since the input voltage is AC, a time-varying magnetic flux is sent through the secondary coil, inducing an AC output voltage.

\begin{figure}[h]
\begin{center}
  \includegraphics[max width=\textwidth]{a3409b21-5bae-43c8-944a-d194fdc3524c-10(1)}
\captionsetup{labelformat=empty}
\caption{Figure 20.31 A typical construction of a simple transformer has two coils wound}
\end{center}
\end{figure}

on a ferromagnetic core. The magnetic field created by the primary coil is mostly confined to and increased by the core, which transmits it to the secondary coil. Any change in current in the primary coil induces a current in the secondary coil.

\section*{Links To Physics}
Magnetic Rope Memory To send men to the moon, the Apollo program had to design an onboard computer system that would be robust, consume little power, and be small enough to fit onboard the spacecraft. In the 1960s, when the Apollo program was launched, entire buildings were regularly dedicated to housing computers whose computing power would be easily outstripped by today's most basic handheld calculator.

To address this problem, engineers at MIT and a major defense contractor turned to magnetic rope memory, which was an offshoot of a similar technology used prior to that time for creating random access memories. Unlike random access memory, magnetic rope memory was read-only memory that contained not only data but instructions as well. Thus, it was actually more than memory: It was a hard-wired computer program.

The components of magnetic rope memory were wires and iron rings-which were called cores. The iron cores served as transformers, such as that shown in the previous figure. However, instead of looping the wires multiple times around the core, individual wires passed only a single time through the cores, making these single-turn transformers. Up to 63 word wires could pass through a single core, along with a single bit wire. If a word wire passed through a given core, a voltage pulse on this wire would induce an emf in the bit wire, which would be interpreted as a one. If the word wire did not pass through the core, no emf would be induced on the bit wire, which would be interpreted as a zero.

Engineers would create programs that would be hard wired into these magnetic rope memories. The wiring process could take as long as a month to complete as workers painstakingly threaded wires through some cores and around others. If any mistakes were made either in the programming or the wiring, debugging would be extraordinarily difficult, if not impossible.

These modules did their job quite well. They are credited with correcting an astronaut mistake in the lunar landing procedure, thereby allowing Apollo 11 to land on the moon. It is doubtful that Michael Faraday ever imagined such an application for magnetic induction when he discovered it.

If the bit wire were looped twice around each core, how would the voltage induced in the bit wire be affected?\\
a. If number of loops around the wire is doubled, the emf is halved.\\
b. If number of loops around the wire is doubled, the emf is not affected.\\
c. If number of loops around the wire is doubled, the emf is also doubled.\\
d. If number of loops around the wire is doubled, the emf is four times the initial value.

For the transformer shown in Figure 20.31, the output voltage \(V_{\mathrm{S}}\) from the secondary coil depends almost entirely on the input voltage \(V_{\mathrm{P}}\) across the primary coil and the number of loops in the primary and secondary coils. Faraday's law of induction for the secondary coil gives its induced output voltage \(V_{\mathrm{S}}\) to be\\
\(V_{S}=-N_{S} \frac{\Delta \Phi}{\Delta t}\),\\
20.19\\
where \(N_{\mathrm{S}}\) is the number of loops in the secondary coil and \(\Delta \Phi / \Delta t\) is the rate of change of magnetic flux. The output voltage equals the induced emf \(\left(V_{\mathrm{S}}=E_{\mathrm{S}}\right)\), provided coil resistance is small-a reasonable assumption for transformers. The cross-sectional area of the coils is the same on each side, as is the magnetic field strength, and so \(\Delta \Phi / \Delta t\) is the same on each side. The input primary voltage \(V_{\mathrm{P}}\) is also related to changing flux by\\
\(V_{\mathrm{P}}=-N_{\mathrm{P}} \frac{\Delta \Phi}{\Delta t}\).\\
20.20

Taking the ratio of these last two equations yields the useful relationship\\
\(\frac{V_{\mathrm{S}}}{V_{\mathrm{P}}}=\frac{N_{\mathrm{S}}}{N_{\mathrm{P}}}(3.07)\).\\
20.21

This is known as the transformer equation. It simply states that the ratio of the secondary voltage to the primary voltage in a transformer equals the ratio of the number of loops in secondary coil to the number of loops in the primary coil.

\section*{Transmission of Electrical Power}
Transformers are widely used in the electric power industry to increase voltagescalled step-up transformers-before long-distance transmission via high-voltage wires. They are also used to decrease voltages - called step-down transformersto deliver power to homes and businesses. The overwhelming majority of electric power is generated by using magnetic induction, whereby a wire coil or copper disk is rotated in a magnetic field. The primary energy required to rotate the coils or disk can be provided by a variety of means. Hydroelectric power plants use the kinetic energy of water to drive electric generators. Coal or nuclear power plants create steam to drive steam turbines that turn the coils. Other sources of primary energy include wind, tides, or waves on water.

Once power is generated, it must be transmitted to the consumer, which often means transmitting power over hundreds of kilometers. To do this, the voltage of the power plant is increased by a step-up transformer, that is stepped up, and the current decreases proportionally because\\
\(P_{\text {transmitted }}=I_{\text {transmitted }} V_{\text {transmitted }}\).\\
20.22

The lower current \(I_{\text {transmitted }}\) in the transmission wires reduces the Joule losses, which is heating of the wire due to a current flow. This heating is caused by the small, but nonzero, resistance \(R_{\text {wire }}\) of the transmission wires. The power lost to the environment through this heat is\\
\(P_{\text {lost }}=I_{\text {transmitted }}^{2} R_{\text {wire }}\),\\
20.23\\
which is proportional to the current squared in the transmission wire. This is why the transmitted current \(I_{\text {transmitted }}\) must be as small as possible and, consequently, the voltage must be large to transmit the power \(P_{\text {transmitted }}\)

Voltages ranging from 120 to 700 kV are used for transmitting power over long distances. The voltage is stepped up at the exit of the power station by a step-up transformer, as shown in Figure 20.32.

\begin{figure}[h]
\begin{center}
  \includegraphics[max width=\textwidth]{a3409b21-5bae-43c8-944a-d194fdc3524c-13}
\captionsetup{labelformat=empty}
\caption{Figure 20.32 Transformers change voltages at several points in a power distribution system. Electric power is usually generated at greater than 10 kV , and transmitted long distances at voltages ranging from 120 kV to 700 kV to limit energy losses. Local power distribution to neighborhoods or industries goes through a substation and is sent short distances at voltages ranging from 5 to 13 kV . This is reduced to 120,240 , or 480 V for safety at the individual user site.}
\end{center}
\end{figure}

Once the power has arrived at a population or industrial center, the voltage is stepped down at a substation to between 5 and 30 kV . Finally, at individual homes or businesses, the power is stepped down again to 120,240 , or 480 V . Each step-up and step-down transformation is done with a transformer designed based on Faradays law of induction. We've come a long way since Queen Elizabeth asked Faraday what possible use could be made of electricity.

\section*{Check Your Understanding}
7.

What is an electric motor?\\
a. An electric motor transforms electrical energy into mechanical energy.\\
b. An electric motor transforms mechanical energy into electrical energy.\\
c. An electric motor transforms chemical energy into mechanical energy.\\
d. An electric motor transforms mechanical energy into chemical energy.\\
8.

What happens to the torque provided by an electric motor if you double the number of coils in the motor?\\
a. The torque would be doubled.\\
b. The torque would be halved.\\
c. The torque would be quadrupled.\\
d. The torque would be tripled.\\
9.

What is a step-up transformer?\\
a. A step-up transformer decreases the current to transmit power over short distances with minimum loss.\\
b. A step-up transformer increases the current to transmit power over short distances with minimum loss.\\
c. A step-up transformer increases voltage to transmit power over long distances with minimum loss.\\
d. A step-up transformer decreases voltage to transmit power over short distances with minimum loss.\\
10.

What should be the ratio of the number of output coils to the number of input coil in a step-up transformer to increase the voltage fivefold?\\
a. The ratio is five times.\\
b. The ratio is 10 times.\\
c. The ratio is 15 times.\\
d. The ratio is 20 times.

\subsection*{20.3 Electromagnetic Induction}
\section*{Section Learning Objectives}
By the end of this section, you will be able to do the following:

\begin{itemize}
  \item Explain how a changing magnetic field produces a current in a wire
  \item Calculate induced electromotive force and current
\end{itemize}

\section*{Teacher Support}
Teacher Support The learning objectives in this section will help your students master the following standards:

\begin{itemize}
  \item (5) The student knows the nature of forces in the physical world. The student is expected to:
  \item (G) investigate and describe the relationship between electric and magnetic fields in applications such as generators, motors, and transformers.
\end{itemize}

In addition, the OSX High School Physics Laboratory Manual addresses content in this section in the lab titled: Magnetism, as well as the following standards:

\begin{itemize}
  \item (5) Science concepts. The student knows the nature of forces in the physical world. The student is expected to:
  \item (G) investigate and describe the relationship between electric and magnetic fields in applications such as generators, motors, and transformers.
\end{itemize}

\section*{Section Key Terms}
\section*{Changing Magnetic Fields}
In the preceding section, we learned that a current creates a magnetic field. If nature is symmetrical, then perhaps a magnetic field can create a current. In 1831, some 12 years after the discovery that an electric current generates a magnetic field, English scientist Michael Faraday (1791-1862) and American scientist Joseph Henry (1797-1878) independently demonstrated that magnetic fields can produce currents. The basic process of generating currents with magnetic fields is called induction; this process is also called magnetic induction to distinguish it from charging by induction, which uses the electrostatic Coulomb force.

When Faraday discovered what is now called Faraday's law of induction, Queen Victoria asked him what possible use was electricity. "Madam," he replied,\\
"What good is a baby?" Today, currents induced by magnetic fields are essential to our technological society. The electric generator-found in everything from automobiles to bicycles to nuclear power plants-uses magnetism to generate electric current. Other devices that use magnetism to induce currents include pickup coils in electric guitars, transformers of every size, certain microphones, airport security gates, and damping mechanisms on sensitive chemical balances.

One experiment Faraday did to demonstrate magnetic induction was to move a bar magnet through a wire coil and measure the resulting electric current through the wire. A schematic of this experiment is shown in Figure 20.33. He found that current is induced only when the magnet moves with respect to the coil. When the magnet is motionless with respect to the coil, no current is induced in the coil, as in Figure 20.33. In addition, moving the magnet in the opposite direction (compare Figure 20.33 with Figure 20.33) or reversing the poles of the magnet (compare Figure 20.33 with Figure 20.33) results in a current in the opposite direction.

\begin{figure}[h]
\begin{center}
  \includegraphics[max width=\textwidth]{a3409b21-5bae-43c8-944a-d194fdc3524c-16}
\captionsetup{labelformat=empty}
\caption{Figure 20.33 Movement of a magnet relative to a coil produces electric currents as shown. The same currents are produced if the coil is moved relative to the magnet. The greater the speed, the greater the magnitude of the current, and the current is zero when there is no motion. The current produced by moving the magnet upward is in the opposite direction as the current produced by moving the magnet downward.}
\end{center}
\end{figure}

\section*{Virtual Physics}
Faraday s Law Click to view content\\
Try this simulation to see how moving a magnet creates a current in a circuit. A light bulb lights up to show when current is flowing, and a voltmeter shows the voltage drop across the light bulb. Try moving the magnet through a fourturn coil and through a two-turn coil. For the same magnet speed, which coil produces a higher voltage?

With the north pole to the left and moving the magnet from right to left, a positive voltage is produced as the magnet enters the coil. What sign voltage will be produced if the experiment is repeated with the south pole to the left?\\
a. The sign of voltage will change because the direction of current flow will change by moving south pole of the magnet to the left.\\
b. The sign of voltage will remain same because the direction of current flow will not change by moving south pole of the magnet to the left.\\
c. The sign of voltage will change because the magnitude of current flow will change by moving south pole of the magnet to the left.\\
d. The sign of voltage will remain same because the magnitude of current flow will not change by moving south pole of the magnet to the left.

\section*{Induced Electromotive Force}
If a current is induced in the coil, Faraday reasoned that there must be what he called an electromotive force pushing the charges through the coil. This interpretation turned out to be incorrect; instead, the external source doing the work of moving the magnet adds energy to the charges in the coil. The energy added per unit charge has units of volts, so the electromotive force is actually a potential. Unfortunately, the name electromotive force stuck and with it the potential for confusing it with a real force. For this reason, we avoid the term electromotive force and just use the abbreviation emf, which has the mathematical symbol \(\varepsilon\). The emf may be defined as the rate at which energy is drawn from a source per unit current flowing through a circuit. Thus, emf is the energy per unit charge added by a source, which contrasts with voltage, which is the energy per unit charge released as the charges flow through a circuit.

To understand why an emf is generated in a coil due to a moving magnet, consider Figure 20.34, which shows a bar magnet moving downward with respect to a wire loop. Initially, seven magnetic field lines are going through the loop (see left-hand image). Because the magnet is moving away from the coil, only five magnetic field lines are going through the loop after a short time \(\Delta t\) (see right-hand image). Thus, when a change occurs in the number of magnetic field lines going through the area defined by the wire loop, an emf is induced in the wire loop. Experiments such as this show that the induced emf is proportional to the rate of change of the magnetic field. Mathematically, we express this as \(\varepsilon \propto \frac{\Delta B}{\Delta t}\),\\
20.24\\
where \(\Delta B\) is the change in the magnitude in the magnetic field during time \(\Delta t\) and \(A\) is the area of the loop.

\begin{figure}[h]
\begin{center}
  \includegraphics[max width=\textwidth]{a3409b21-5bae-43c8-944a-d194fdc3524c-18(1)}
\captionsetup{labelformat=empty}
\caption{Figure 20.34 The bar magnet moves downward with respect to the wire loop, so that the number of magnetic field lines going through the loop decreases with time. This causes an emf to be induced in the loop, creating an electric current.}
\end{center}
\end{figure}

Note that magnetic field lines that lie in the plane of the wire loop do not actually pass through the loop, as shown by the left-most loop in Figure 20.35. In this figure, the arrow coming out of the loop is a vector whose magnitude is the area of the loop and whose direction is perpendicular to the plane of the loop. In Figure 20.35, as the loop is rotated from \(\theta=90^{\circ}\) to \(\theta=0^{\circ}\), the contribution of the magnetic field lines to the emf increases. Thus, what is important in generating an emf in the wire loop is the component of the magnetic field that is perpendicular to the plane of the loop, which is \(B \cos \theta\).

This is analogous to a sail in the wind. Think of the conducting loop as the sail and the magnetic field as the wind. To maximize the force of the wind on the sail, the sail is oriented so that its surface vector points in the same direction as the winds, as in the right-most loop in Figure 20.35. When the sail is aligned so that its surface vector is perpendicular to the wind, as in the left-most loop in Figure 20.35, then the wind exerts no force on the sail.

Thus, taking into account the angle of the magnetic field with respect to the area, the proportionality \(E \propto \Delta B / \Delta t\) becomes\\
\(E \propto \frac{\Delta B \cos \theta}{\Delta t}\).\\
20.25\\
\includegraphics[max width=\textwidth, center]{a3409b21-5bae-43c8-944a-d194fdc3524c-18}

Figure 20.35 The magnetic field lies in the plane of the left-most loop, so it cannot generate an emf in this case. When the loop is rotated so that the angle of the magnetic field with the vector perpendicular to the area of the loop increases to \(90^{\circ}\) (see right-most loop), the magnetic field contributes maximally to the emf in the loop. The dots show where the magnetic field lines intersect the plane defined by the loop.

Another way to reduce the number of magnetic field lines that go through the conducting loop in Figure 20.35 is not to move the magnet but to make the loop smaller. Experiments show that changing the area of a conducting loop in a stable magnetic field induces an emf in the loop. Thus, the emf produced in a conducting loop is proportional to the rate of change of the product of the perpendicular magnetic field and the loop area\\
\(\varepsilon \propto \frac{\Delta[(B \cos \theta) A]}{\Delta t}\),\\
20.26\\
where \(B \cos \theta\) is the perpendicular magnetic field and \(A\) is the area of the loop. The product \(B A \cos \theta\) is very important. It is proportional to the number of magnetic field lines that pass perpendicularly through a surface of area \(A\). Going back to our sail analogy, it would be proportional to the force of the wind on the sail. It is called the magnetic flux and is represented by \(\Phi\).\\
\(\Phi=B A \cos \theta\)\\
20.27

The unit of magnetic flux is the weber ( Wb ), which is magnetic field per unit area, or \(\mathrm{T} / \mathrm{m}^{2}\). The weber is also a volt second (Vs).

The induced emf is in fact proportional to the rate of change of the magnetic flux through a conducting loop.\\
\(\varepsilon \propto \frac{\Delta \Phi}{\Delta t}\)\\
20.28

Finally, for a coil made from \(N\) loops, the emf is \(N\) times stronger than for a single loop. Thus, the emf induced by a changing magnetic field in a coil of \(N\) loops is\\
\(\varepsilon \propto N \frac{\Delta B \cos \theta}{\Delta t} A\).\\
The last question to answer before we can change the proportionality into an equation is "In what direction does the current flow?" The Russian scientist Heinrich Lenz (1804-1865) explained that the current flows in the direction that creates a magnetic field that tries to keep the flux constant in the loop. For example, consider again Figure 20.34. The motion of the bar magnet causes the number of upward-pointing magnetic field lines that go through the loop to decrease. Therefore, an emf is generated in the loop that drives a current in the direction that creates more upward-pointing magnetic field lines. By using\\
the right-hand rule, we see that this current must flow in the direction shown in the figure. To express the fact that the induced emf acts to counter the change in the magnetic flux through a wire loop, a minus sign is introduced into the proportionality \(\varepsilon \propto \Delta \Phi / \Delta t\)., which gives Faraday's law of induction.\\
\(\varepsilon=-N \frac{\Delta \Phi}{\Delta t}\)\\
20.29

Lenz's law is very important. To better understand it, consider Figure 20.36, which shows a magnet moving with respect to a wire coil and the direction of the resulting current in the coil. In the top row, the north pole of the magnet approaches the coil, so the magnetic field lines from the magnet point toward the coil. Thus, the magnetic field \(\vec{B}_{\text {mag }}=B_{\text {mag }}(\hat{x})\) pointing to the right increases in the coil. According to Lenz's law, the emf produced in the coil will drive a current in the direction that creates a magnetic field \(\vec{B}_{\text {coil }}=B_{\text {coil }}(-\hat{x})\) inside the coil pointing to the left. This will counter the increase in magnetic flux pointing to the right. To see which way the current must flow, point your right thumb in the desired direction of the magnetic field \(\vec{B}_{\text {coil }}\), and the current will flow in the direction indicated by curling your right fingers. This is shown by the image of the right hand in the top row of Figure 20.36. Thus, the current must flow in the direction shown in Figure 4(a).

In Figure 4(b), the direction in which the magnet moves is reversed. In the coil, the right-pointing magnetic field \(\vec{B}_{\text {mag }}\) due to the moving magnet decreases. Lenz's law says that, to counter this decrease, the emf will drive a current that creates an additional right-pointing magnetic field \(\vec{B}_{\text {coil }}\) in the coil. Again, point your right thumb in the desired direction of the magnetic field, and the current will flow in the direction indicate by curling your right fingers (Figure 4(b)).

Finally, in Figure 4(c), the magnet is reversed so that the south pole is nearest the coil. Now the magnetic field \(\vec{B}_{\text {mag }}\) points toward the magnet instead of toward the coil. As the magnet approaches the coil, it causes the left-pointing magnetic field in the coil to increase. Lenz's law tells us that the emf induced in the coil will drive a current in the direction that creates a magnetic field pointing to the right. This will counter the increasing magnetic flux pointing to the left due to the magnet. Using the right-hand rule again, as indicated in the figure, shows that the current must flow in the direction shown in Figure 4(c).

\begin{figure}[h]
\begin{center}
  \includegraphics[max width=\textwidth]{a3409b21-5bae-43c8-944a-d194fdc3524c-21}
\captionsetup{labelformat=empty}
\caption{Figure 20.36 Lenz's law tells us that the magnetically induced emf will drive a current that resists the change in the magnetic flux through a circuit. This is shown in panels (a)-(c) for various magnet orientations and velocities. The right hands at right show how to apply the right-hand rule to find in which direction the induced current flows around the coil.}
\end{center}
\end{figure}

\section*{Virtual Physics}
Faraday s Electromagnetic Lab Click to view content\\
This simulation proposes several activities. For now, click on the tab Pickup Coil, which presents a bar magnet that you can move through a coil. As you do so, you can see the electrons move in the coil and a light bulb will light up or a voltmeter will indicate the voltage across a resistor. Note that the voltmeter allows you to see the sign of the voltage as you move the magnet about. You can also leave the bar magnet at rest and move the coil, although it is more difficult to observe the results.

PhET Explorations: Faraday's Electromagnetic Lab Play with a bar magnet and coils to learn about Faraday's law. Move a bar magnet near one or two coils to make a light bulb glow. View the magnetic field lines. A meter shows the direction and magnitude of the current. View the magnetic field lines or use a meter to show the direction and magnitude of the current. You can also play with electromagnets, generators and transformers!

Click to view content\\
Orient the bar magnet with the north pole facing to the right and place the pickup coil to the right of the bar magnet. Now move the bar mag-\\
net toward the coil and observe in which way the electrons move. This is the same situation as depicted below. Does the current in the simulation flow in the same direction as shown below? Explain why or why not.\\
\includegraphics[max width=\textwidth, center]{a3409b21-5bae-43c8-944a-d194fdc3524c-22}\\
a. Yes, the current in the simulation flows as shown because the direction of current is opposite to the direction of flow of electrons.\\
b. No, current in the simulation flows in the opposite direction because the direction of current is same to the direction of flow of electrons.

\section*{Watch Physics}
Induced Current in a Wire This video explains how a current can be induced in a straight wire by moving it through a magnetic field. The lecturer uses the cross product, which a type of vector multiplication. Don't worry if you are not familiar with this, it basically combines the right-hand rule for determining the force on the charges in the wire with the equation \(F=q v B \sin \theta\).

Click to view content

\section*{Grasp Check}
What emf is produced across a straight wire 0.50 m long moving at a velocity of \((1.5 \mathrm{~m} / \mathrm{s}) \hat{x}\) through a uniform magnetic field \((0.30 \mathrm{~T}) \hat{z}\) ? The wire lies in the \(\hat{y}\)-direction. Also, which end of the wire is at the higher potential-let the lower end of the wire be at \(y=0\) and the upper end at \(y=0.5 \mathrm{~m}\) )?\\
a. 0.15 V and the lower end of the wire will be at higher potential\\
b. 0.15 V and the upper end of the wire will be at higher potential\\
c. 0.075 V and the lower end of the wire will be at higher potential\\
d. 0.075 V and the upper end of the wire will be at higher potential

\section*{Worked Example}
EMF Induced in Conducing Coil by Moving Magnet Imagine a magnetic field goes through a coil in the direction indicated in Figure 20.37. The coil diameter is 2.0 cm . If the magnetic field goes from 0.020 to 0.010 T in 34 s , what is the direction and magnitude of the induced current? Assume the coil has a resistance of \(0.1 \Omega\).\\
\includegraphics[max width=\textwidth, center]{a3409b21-5bae-43c8-944a-d194fdc3524c-23}

Figure 20.37 A coil through which passes a magnetic field \(B\).

\section*{Strategy}
Use the equation\\
\(\varepsilon=-N \Delta \Phi / \Delta t\)\\
to find the induced emf in the coil, where \(\Delta t=34 \mathrm{~s}\). Counting the number of loops in the solenoid, we find it has 16 loops, so \(N=16\). Use the equation\\
\(\Phi=B A \cos \theta\)\\
to calculate the magnetic flux\\
\(\Phi=B A \cos \theta=B \pi\left(\frac{d}{2}\right)^{2}\),\\
20.30\\
where \(d\) is the diameter of the solenoid and we have used \(\cos 0^{\circ}=1\). Because the area of the solenoid does not vary, the change in the magnetic of the flux through the solenoid is\\
\(\Delta \Phi=\Delta B \pi\left(\frac{d}{2}\right)^{2}\).\\
20.31

Once we find the emf, we can use Ohm's law, \(\varepsilon=I R\), to find the current.\\
Finally, Lenz's law tells us that the current should produce a magnetic field that acts to oppose the decrease in the applied magnetic field. Thus, the current should produce a magnetic field to the right.

Solution\\
Combining equations \(\varepsilon=-N \Delta \Phi / \Delta t\) and \(\Phi=B A \cos \theta\) gives\\
\(\varepsilon=-N \frac{\Delta \Phi}{\Delta t}=-N \frac{\Delta B \pi d^{2}}{4 \Delta t}\).\\
20.32

Solving Ohm's law for the current and using this result gives

\[
\begin{aligned}
I & =\frac{\varepsilon}{R}=-N \frac{\Delta B \pi d^{2}}{4 R \Delta t} \\
& =-16 \frac{(-0.010 \mathrm{~T}) \pi(0.020 \mathrm{~m})^{2}}{4(0.10 \Omega)(34 \mathrm{~s})} \\
& =15 \mu \mathrm{~A}
\end{aligned}
\]

20.33

Lenz's law tells us that the current must produce a magnetic field to the right. Thus, we point our right thumb to the right and curl our right fingers around the solenoid. The current must flow in the direction in which our fingers are pointing, so it enters at the left end of the solenoid and exits at the right end.

Discussion\\
Let's see if the minus sign makes sense in Faraday's law of induction. Define the direction of the magnetic field to be the positive direction. This means the change in the magnetic field is negative, as we found above. The minus sign in Faraday's law of induction negates the negative change in the magnetic field, leaving us with a positive current. Therefore, the current must flow in the direction of the magnetic field, which is what we found.

Now try defining the positive direction to be the direction opposite that of the magnetic field, that is positive is to the left in Figure 20.37. In this case, you will find a negative current. But since the positive direction is to the left, a negative current must flow to the right, which again agrees with what we found by using Lenz's law.

\section*{Worked Example}
Magnetic Induction due to Changing Circuit Size The circuit shown in Figure 20.38 consists of a U-shaped wire with a resistor and with the ends connected by a sliding conducting rod. The magnetic field filling the area enclosed by the circuit is constant at 0.01 T . If the rod is pulled to the right at speed \(v=0.50 \mathrm{~m} / \mathrm{s}\), what current is induced in the circuit and in what direction does the current flow?

\begin{figure}[h]
\begin{center}
  \includegraphics[max width=\textwidth]{a3409b21-5bae-43c8-944a-d194fdc3524c-25}
\captionsetup{labelformat=empty}
\caption{Figure 20.38 A slider circuit. The magnetic field is constant and the rod is pulled to the right at speed \(v\). The changing area enclosed by the circuit induces an emf in the circuit.}
\end{center}
\end{figure}

\section*{Strategy}
We again use Faraday's law of induction, \(E=-N \frac{\Delta \Phi}{\Delta t}\), although this time the magnetic field is constant and the area enclosed by the circuit changes. The circuit contains a single loop, so \(N=1\). The rate of change of the area is \(\frac{\Delta A}{\Delta t}=v \ell\). Thus the rate of change of the magnetic flux is\\
\(\frac{\Delta \Phi}{\Delta t}=\frac{\Delta(B A \cos \theta)}{\Delta t}=B \frac{\Delta A}{\Delta t}=B v \ell\),\\
20.34\\
where we have used the fact that the angle \(\theta\) between the area vector and the magnetic field is \(0^{\circ}\). Once we know the emf, we can find the current by using Ohm's law. To find the direction of the current, we apply Lenz's law.

Solution\\
Faraday's law of induction gives\\
\(E=-N \frac{\Delta \Phi}{\Delta t}=-B v \ell\).\\
20.35

Solving Ohm's law for the current and using the previous result for emf gives\\
\(I=\frac{E}{R}=\frac{-B v \ell}{R}=\frac{-(0.010 \mathrm{~T})(0.50 \mathrm{~m} / \mathrm{s})(0.10 \mathrm{~m})}{20 \Omega}=25 \quad \mathrm{~A}\).\\
20.36

As the rod slides to the right, the magnetic flux passing through the circuit increases. Lenz's law tells us that the current induced will create a magnetic field that will counter this increase. Thus, the magnetic field created by the induced current must be into the page. Curling your right-hand fingers around the loop in the clockwise direction makes your right thumb point into the page,\\
which is the desired direction of the magnetic field. Thus, the current must flow in the clockwise direction around the circuit.

\section*{Discussion}
Is energy conserved in this circuit? An external agent must pull on the rod with sufficient force to just balance the force on a current-carrying wire in a magnetic field-recall that \(F=I \ell B \sin \theta\). The rate at which this force does work on the rod should be balanced by the rate at which the circuit dissipates power. Using \(F=I \ell B \sin \theta\), the force required to pull the wire at a constant speed \(v\) is\\
\(F_{\text {pull }}=I \ell B \sin \theta=I \ell B\),\\
20.37\\
where we used the fact that the angle \(\theta\) between the current and the magnetic field is \(90^{\circ}\). Inserting our expression above for the current into this equation gives\\
\(F_{\text {pull }}=I \ell B=-\frac{B v \ell}{R}(\ell B)=-\frac{B^{2} v \ell^{2}}{R}\).\\
20.38

The power contributed by the agent pulling the rod is \(F_{\text {pull }} v\), or\\
\(P_{\text {pull }}=F_{\text {pull }} v=-\frac{B^{2} v^{2} \ell^{2}}{R}\).\\
20.39

The power dissipated by the circuit is\\
\(P_{\text {dissipated }}=I^{2} R=\left(\frac{-B v \ell}{R}\right)^{2} R=\frac{B^{2} v^{2} \ell^{2}}{R}\).\\
20.40

We thus see that \(P_{\text {pull }}+P_{\text {dissipated }}=0\), which means that power is conserved in the system consisting of the circuit and the agent that pulls the rod. Thus, energy is conserved in this system.

\section*{Practice Problems}
11.

The magnetic flux through a single wire loop changes from 3.5 Wb to 1.5 Wb in 2.0 s . What emf is induced in the loop?\\
a. -2.0 V\\
b. -1.0 V\\
c. +1.0 V\\
d. +2.0 V\\
12.

What is the emf for a 10 -turn coil through which the flux changes at \(10 \mathrm{~Wb} / \mathrm{s}\) ?\\
a. -100 V\\
b. -10 V\\
c. +10 V\\
d. +100 V

\section*{Check Your Understanding}
13.

Given a bar magnet, how can you induce an electric current in a wire loop?\\
a. An electric current is induced if a bar magnet is placed near the wire loop.\\
b. An electric current is induced if a wire loop is wound around the bar magnet.\\
c. An electric current is induced if a bar magnet is moved through the wire loop.\\
d. An electric current is induced if a bar magnet is placed in contact with the wire loop.\\
14.

What factors can cause an induced current in a wire loop through which a magnetic field passes?\\
a. Induced current can be created by changing the size of the wire loop only.\\
b. Induced current can be created by changing the orientation of the wire loop only.\\
c. Induced current can be created by changing the strength of the magnetic field only.\\
d. Induced current can be created by changing the strength of the magnetic field, changing the size of the wire loop, or changing the orientation of the wire loop.

\section*{Key Terms}
Curie temperature well-defined temperature for ferromagnetic materials above which they cannot be magnetized\\
domain region within a magnetic material in which the magnetic poles of individual atoms are aligned\\
electric motor device that transforms electrical energy into mechanical energy\\
electromagnet device that uses electric current to make a magnetic field\\
electromagnetism study of electric and magnetic phenomena\\
emf rate at which energy is drawn from a source per unit current flowing through a circuit\\
ferromagnetic material such as iron, cobalt, nickel, or gadolinium that exhibits strong magnetic effects\\
generator device that transforms mechanical energy into electrical energy\\
induction rate at which energy is drawn from a source per unit current flowing through a circuit\\
magnetic dipole term that describes magnets because they always have two poles: north and south\\
magnetic field directional lines around a magnetic material that indicates the direction and magnitude of the magnetic force\\
magnetic flux component of the magnetic field perpendicular to the surface area through which it passes and multiplied by the area\\
magnetic pole part of a magnet that exerts the strongest force on other magnets or magnetic material\\
magnetized material that is induced to be magnetic or that is made into a permanent magnet\\
north pole part of a magnet that orients itself toward the geographic North Pole of Earth\\
permanent magnet material that retains its magnetic behavior for a long time, even when exposed to demagnetizing influences\\
right-hand rule rule involving curling the right-hand fingers from one vector to another; the direction in which the right thumb points is the direction of the resulting vector\\
solenoid uniform cylindrical coil of wire through which electric current is passed to produce a magnetic field\\
south pole part of a magnet that orients itself toward the geographic South Pole of Earth\\
transformer device that transforms voltages from one value to another

\section*{Key Equations}
20.1 Magnetic Fields, Field Lines, and Force

\subsection*{20.3 Electromagnetic Induction}
\section*{Section Summary}
\subsection*{20.1 Magnetic Fields, Field Lines, and Force}
\begin{itemize}
  \item All magnets have two poles: a north pole and a south pole. If the magnet is free to move, its north pole orients itself toward the geographic North Pole of Earth, and the south pole orients itself toward the geographic South Pole of Earth.
  \item A repulsive force occurs between the north poles of two magnets and likewise for two south poles. However, an attractive force occurs between the north pole of one magnet and the south pole of another magnet.
  \item A charged particle moving through a magnetic field experiences a force whose direction is determined by the right-hand rule.
  \item An electric current generates a magnetic field.
  \item Electromagnets are magnets made by passing a current through a system of wires.
\end{itemize}

\subsection*{20.2 Motors, Generators, and Transformers}
\begin{itemize}
  \item Electric motors contain wire loops in a magnetic field. Current is passed through the wire loops, which forces them to rotate in the magnetic field. The current is reversed every half rotation so that the torque on the loop is always in the same direction.
  \item Electric generators contain wire loops in a magnetic field. An external agent provides mechanical energy to force the loops to rotate in the magnetic field, which produces an AC voltage that drives an AC current through the loops.
  \item Transformers contain a ring made of magnetic material and, on opposite sides of the ring, two windings of wire wrap around the ring. A changing current in one wire winding creates a changing magnetic field, which is trapped in the ring and thus goes through the second winding and induces an emf in the second winding. The voltage in the second winding is proportional to the ratio of the number of loops in each winding.
  \item Transformers are used to step up and step down the voltage for power transmission.
  \item Over long distances, electric power is transmitted at high voltage to minimize the current and thereby minimize the Joule losses due to resistive heating.
\end{itemize}

\subsection*{20.3 Electromagnetic Induction}
\begin{itemize}
  \item Faraday s law of inductionstates that a changing magnetic flux that occurs within an area enclosed by a conducting loop induces an electric current in the loop.
  \item Lenz lawstates that an induced current flows in the direction such that it opposes the change that induced it.
\end{itemize}

\end{document}