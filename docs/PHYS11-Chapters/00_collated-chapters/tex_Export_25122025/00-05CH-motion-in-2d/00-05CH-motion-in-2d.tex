\documentclass[10pt]{article}
\usepackage[utf8]{inputenc}
\usepackage[T1]{fontenc}
\usepackage{graphicx}
\usepackage[export]{adjustbox}
\graphicspath{ {./images/} }
\usepackage{caption}
\usepackage{amsmath}
\usepackage{amsfonts}
\usepackage{amssymb}
\usepackage[version=4]{mhchem}
\usepackage{stmaryrd}

\DeclareUnicodeCharacter{2216}{\ifmmode\backslash\else{$\backslash$}\fi}
\DeclareUnicodeCharacter{27C2}{\ifmmode\perp\else{$\perp$}\fi}

\begin{document}
\captionsetup{singlelinecheck=false}
\begin{figure}[h]
\begin{center}
  \includegraphics[max width=\textwidth]{3b86ae0c-65b9-4aad-b74b-2c0d7621e1ae-01}
\captionsetup{labelformat=empty}
\caption{Figure 5.1 Billiard balls on a pool table are in motion after being hit with a cue stick. (Popperipopp, Wikimedia Commons)}
\end{center}
\end{figure}

\section*{Chapter Outline}
5.1 Vector Addition and Subtraction: Graphical Methods\\
5.2 Vector Addition and Subtraction: Analytical Methods\\
5.3 Projectile Motion\\
5.4 Inclined Planes\\
5.5 Simple Harmonic Motion

\section*{Introduction}
\section*{Teacher Support}
Teacher Support Physics learning objectives come from 112.39 (c) Knowledge and Skills

\section*{Teacher Support}
Teacher Support Point out to the students that most motion is in two or three dimensions and can be described in a similar fashion to one-dimensional motion. This chapter is about motion in two dimensions. Motion in two dimensions can be analyzed using vectors. We will first learn the practical skills of adding and subtracting vectors graphically (in drawings) and analytically (with\\
math). Once we're able to work with two-dimensional vectors, we can then apply these skills to problems of projectile motion, inclined planes, and harmonic motion.

In Chapter 2, we learned to distinguish between vectors and scalars; the difference being that a vector has magnitude and direction, whereas a scalar has only magnitude. We learned how to deal with vectors in physics by working straightforward one-dimensional vector problems, which may be treated mathematically in the same as scalars. In this chapter, we'll use vectors to expand our understanding of forces and motion into two dimensions. Most real-world physics problems (such as with the game of pool pictured here) are, after all, either two- or three-dimensional problems and physics is most useful when applied to real physical scenarios. We start by learning the practical skills of graphically adding and subtracting vectors (by using drawings) and analytically (with math). Once we're able to work with two-dimensional vectors, we apply these skills to problems of projectile motion, inclined planes, and harmonic motion.

\section*{Teacher Support}
Teacher Support Before students begin this chapter, review the concepts of displacement, velocity, acceleration, vectors, representing vectors, free-body diagrams.

\subsection*{5.1 Vector Addition and Subtraction: Graphical Methods}
\section*{Section Learning Objectives}
By the end of this section, you will be able to do the following:

\begin{itemize}
  \item Describe the graphical method of vector addition and subtraction
  \item Use the graphical method of vector addition and subtraction to solve physics problems
\end{itemize}

\section*{Teacher Support}
Teacher Support The learning objectives in this section will help your students master the following standards:

\begin{itemize}
  \item (4) Science concepts. The student knows and applies the laws governing motion in two dimensions for a variety of situations. The student is expected to:
  \item (E) develop and interpret free-body force diagrams.
\end{itemize}

\section*{Section Key Terms}
\section*{The Graphical Method of Vector Addition and Subtraction}
Recall that a vector is a quantity that has magnitude and direction. For example, displacement, velocity, acceleration, and force are all vectors. In onedimensional or straight-line motion, the direction of a vector can be given simply by a plus or minus sign. Motion that is forward, to the right, or upward is usually considered to be positive ( + ); and motion that is backward, to the left, or downward is usually considered to be negative ( - ).

In two dimensions, a vector describes motion in two perpendicular directions, such as vertical and horizontal. For vertical and horizontal motion, each vector is made up of vertical and horizontal components. In a one-dimensional problem, one of the components simply has a value of zero. For two-dimensional vectors, we work with vectors by using a frame of reference such as a coordinate system. Just as with one-dimensional vectors, we graphically represent vectors with an arrow having a length proportional to the vector's magnitude and pointing in the direction that the vector points.

\section*{Teacher Support}
Teacher Support [BL][OL]Review vectors and free body diagrams. Recall how velocity, displacement and acceleration vectors are represented.

Figure 5.2 shows a graphical representation of a vector; the total displacement for a person walking in a city. The person first walks nine blocks east and then five blocks north. Her total displacement does not match her path to her final destination. The displacement simply connects her starting point with her ending point using a straight line, which is the shortest distance. We use the notation that a boldface symbol, such as \(\mathbf{D}\), stands for a vector. Its magnitude is represented by the symbol in italics, \(D\), and its direction is given by an angle represented by the symbol \(\theta\). Note that her displacement would be the same if she had begun by first walking five blocks north and then walking nine blocks east.

\section*{Tips For Success}
In this text, we represent a vector with a boldface variable. For example, we represent a force with the vector \(\mathbf{F}\), which has both magnitude and direction. The magnitude of the vector is represented by the variable in italics, \(F\), and the direction of the variable is given by the angle \(\theta\).

\begin{figure}[h]
\begin{center}
  \includegraphics[max width=\textwidth]{3b86ae0c-65b9-4aad-b74b-2c0d7621e1ae-04}
\captionsetup{labelformat=empty}
\caption{Figure 5.2 A person walks nine blocks east and five blocks north. The displacement is 10.3 blocks at an angle \(29.1^{\circ}\) north of east.}
\end{center}
\end{figure}

The head-to-tail method is a graphical way to add vectors. The tail of the vector is the starting point of the vector, and the head (or tip) of a vector is the pointed end of the arrow. The following steps describe how to use the head-to-tail method for graphical vector addition.

\begin{enumerate}
  \item Let the \(x\)-axis represent the east-west direction. Using a ruler and protractor, draw an arrow to represent the first vector (nine blocks to the east), as shown in Figure 5.3(a).\\
\includegraphics[max width=\textwidth, center]{3b86ae0c-65b9-4aad-b74b-2c0d7621e1ae-05}\\
(a)
\end{enumerate}

Figure 5.3 The diagram shows a vector with a magnitude of nine units and a direction of \(0^{\circ}\).\\
2. Let the \(y\)-axis represent the north-south direction. Draw an arrow to represent the second vector (five blocks to the north). Place the tail of the second vector at the head of the first vector, as shown in Figure 5.4(b).\\
\includegraphics[max width=\textwidth, center]{3b86ae0c-65b9-4aad-b74b-2c0d7621e1ae-05(1)}

\begin{itemize}
  \item (b)
\end{itemize}

Figure 5.4 A vertical vector is added.\\
3. If there are more than two vectors, continue to add the vectors head-to-tail as described in step 2. In this example, we have only two vectors, so we\\
have finished placing arrows tip to tail.\\
4. Draw an arrow from the tail of the first vector to the head of the last vector, as shown in Figure 5.5(c). This is the resultant, or the sum, of the vectors.

\begin{figure}[h]
\begin{center}
  \includegraphics[max width=\textwidth]{3b86ae0c-65b9-4aad-b74b-2c0d7621e1ae-06}
\captionsetup{labelformat=empty}
\caption{Figure 5.5 The diagram shows the resultant vector, a ruler, and protractor.}
\end{center}
\end{figure}

\begin{enumerate}
  \setcounter{enumi}{4}
  \item To find the magnitude of the resultant, measure its length with a ruler. When we deal with vectors analytically in the next section, the magnitude will be calculated by using the Pythagorean theorem.
  \item To find the direction of the resultant, use a protractor to measure the angle it makes with the reference direction (in this case, the \(x\)-axis). When we deal with vectors analytically in the next section, the direction will be calculated by using trigonometry to find the angle.
\end{enumerate}

\section*{Teacher Support}
Teacher Support [AL] Ask two students to demonstrate pushing a table from two different directions. Ask students what they feel the direction of resultant motion will be. How would they represent this graphically? Recall that a vector's magnitude is represented by the length of the arrow. Demonstrate the head-to-tail method of adding vectors, using the example given in the chapter. Ask students to practice this method of addition using a scale and a protractor.\\[0pt]
[BL][OL][AL] Ask students if anything changes by moving the vector from one place to another on a graph. How about the order of addition? Would that make a difference? Introduce negative of a vector and vector subtraction.

\section*{Watch Physics}
Visualizing Vector Addition Examples This video shows four graphical representations of vector addition and matches them to the correct vector addition formula.

Click to view content\\
There are two vectors \(\backslash \operatorname{text}\{\mathrm{a}\}\) and \(\backslash \operatorname{text}\{\mathrm{b}\}\). The head of vector \(\backslash \operatorname{text}\{\mathrm{a}\}\) touches the tail of vector \(\backslash \operatorname{text}\{\mathrm{b}\}\). The addition of vectors \(\backslash \operatorname{text}\{\mathrm{a}\}\) and \(\backslash \operatorname{text}\{b\}\) gives a resultant vector \(\backslash \operatorname{text}\{c\}\). Can the addition of these two vectors can be represented by the following two equations? \textbackslash overrightarrow\{ \(\backslash \operatorname{text}\{\mathrm{a}\}\} +\backslash\) overrightarrow \(\{\backslash \operatorname{text}\{\mathrm{b}\}\}=\backslash\) overrightarrow \(\{\backslash \operatorname{text}\{\mathrm{c}\}\} ; \quad \backslash\) overrightarrow \(\{\backslash\) text \(\{\mathrm{b}\}\}+\backslash\) overrightarrow \(\{\backslash\) text \(\{\mathrm{a}\}\}=\backslash\) overrightarrow \(\{\backslash\) text \(\{\mathrm{c}\}\}\)\\
a. Yes, if we add the same two vectors in a different order it will still give the same resultant vector.\\
b. No, the resultant vector will change if we add the same vectors in a different order.

Vector subtraction is done in the same way as vector addition with one small change. We add the first vector to the negative of the vector that needs to be subtracted. A negative vector has the same magnitude as the original vector, but points in the opposite direction (as shown in Figure 5.6). Subtracting the vector \(\mathbf{B}\) from the vector \(\mathbf{A}\), which is written as \(\mathbf{A}-\mathbf{B}\), is the same as \(\mathbf{A}+ (-\mathbf{B})\). Since it does not matter in what order vectors are added, \(\mathbf{A}-\mathbf{B}\) is also equal to \((-\mathbf{B})+\mathbf{A}\). This is true for scalars as well as vectors. For example, \(5- 2=5+(-2)=(-2)+5\).

\begin{figure}[h]
\begin{center}
  \includegraphics[max width=\textwidth]{3b86ae0c-65b9-4aad-b74b-2c0d7621e1ae-07}
\captionsetup{labelformat=empty}
\caption{Figure 5.6 The diagram shows a vector, B , and the negative of this vector, -B .}
\end{center}
\end{figure}

Global angles are calculated in the counterclockwise direction. The clockwise direction is considered negative. For example, an angle of \(30^{\circ}\) south of west is the same as the global angle \(210^{\circ}\), which can also be expressed as \(-150^{\circ}\) from the positive \(x\)-axis.

\section*{Using the Graphical Method of Vector Addition and Subtraction to Solve Physics Problems}
Now that we have the skills to work with vectors in two dimensions, we can apply vector addition to graphically determine the resultant vector, which represents\\
the total force. Consider an example of force involving two ice skaters pushing a third as seen in Figure 5.7.

\begin{figure}[h]
\begin{center}
  \includegraphics[max width=\textwidth]{3b86ae0c-65b9-4aad-b74b-2c0d7621e1ae-08}
\captionsetup{labelformat=empty}
\caption{Figure 5.7 Part (a) shows an overhead view of two ice skaters pushing on a third. Forces are vectors and add like vectors, so the total force on the third skater is in the direction shown. In part (b), we see a free-body diagram representing the forces acting on the third skater.}
\end{center}
\end{figure}

In problems where variables such as force are already known, the forces can be represented by making the length of the vectors proportional to the magnitudes of the forces. For this, you need to create a scale. For example, each centimeter of vector length could represent 50 N worth of force. Once you have the initial vectors drawn to scale, you can then use the head-to-tail method to draw the resultant vector. The length of the resultant can then be measured and converted back to the original units using the scale you created.

You can tell by looking at the vectors in the free-body diagram in Figure 5.7 that the two skaters are pushing on the third skater with equal-magnitude forces, since the length of their force vectors are the same. Note, however, that the forces are not equal because they act in different directions. If, for example, each force had a magnitude of 400 N , then we would find the magnitude of the total external force acting on the third skater by finding the magnitude of the resultant vector. Since the forces act at a right angle to one another, we can use the Pythagorean theorem. For a triangle with sides \(\mathrm{a}, \mathrm{b}\), and c , the Pythagorean theorem tells us that\\
\(a^{2}+b^{2}=c^{2}\)\\
\(c=\sqrt{a^{2}+b^{2}}\).\\
Applying this theorem to the triangle made by \(\mathbf{F}_{1}, \mathbf{F}_{2}\), and \(\mathbf{F}_{\text {tot }}\) in Figure 5.7, we get\\
\(\mathbf{F}_{\text {tot }}^{2}=\sqrt{\mathbf{F}_{1}^{2}+\mathbf{F}_{1}^{2}}\),\\
or\\
\(\mathbf{F}_{\text {tot }}=\sqrt{(400 \mathrm{~N})^{2}+(400 \mathrm{~N})^{2}}=566 \mathrm{~N}\).\\
Note that, if the vectors were not at a right angle to each other ( \(90^{\circ}\) to one another), we would not be able to use the Pythagorean theorem to find the magnitude of the resultant vector. Another scenario where adding two-dimensional vectors is necessary is for velocity, where the direction may not be purely eastwest or north-south, but some combination of these two directions. In the next section, we cover how to solve this type of problem analytically. For now let's consider the problem graphically.

\section*{Worked Example}
Adding Vectors Graphically by Using the Head-to-Tail Method: A Woman Takes a Walk Use the graphical technique for adding vectors to find the total displacement of a person who walks the following three paths (displacements) on a flat field. First, she walks 25 m in a direction \(49^{\circ}\) north of east. Then, she walks 23 m heading \(15^{\circ}\) north of east. Finally, she turns and walks 32 m in a direction \(68^{\circ}\) south of east.

\section*{Strategy}
Graphically represent each displacement vector with an arrow, labeling the first \(\mathbf{A}\), the second \(\mathbf{B}\), and the third \(\mathbf{C}\). Make the lengths proportional to the distance of the given displacement and orient the arrows as specified relative to an east-west line. Use the head-to-tail method outlined above to determine the magnitude and direction of the resultant displacement, which we'll call \(\mathbf{R}\).

Solution\\
(1) Draw the three displacement vectors, creating a convenient scale (such as 1 cm of vector length on paper equals 1 m in the problem), as shown in Figure 5.8.

\begin{figure}[h]
\begin{center}
  \includegraphics[max width=\textwidth]{3b86ae0c-65b9-4aad-b74b-2c0d7621e1ae-09}
\captionsetup{labelformat=empty}
\caption{Figure 5.8 The three displacement vectors are drawn first.}
\end{center}
\end{figure}

(2) Place the vectors head to tail, making sure not to change their magnitude or direction, as shown in Figure 5.9.\\
\includegraphics[max width=\textwidth, center]{3b86ae0c-65b9-4aad-b74b-2c0d7621e1ae-10(1)}

Figure 5.9 Next, the vectors are placed head to tail.\\
(3) Draw the resultant vector \(\mathbf{R}\) from the tail of the first vector to the head of the last vector, as shown in Figure 5.10.

\begin{figure}[h]
\begin{center}
  \includegraphics[max width=\textwidth]{3b86ae0c-65b9-4aad-b74b-2c0d7621e1ae-10}
\captionsetup{labelformat=empty}
\caption{Figure 5.10 The resultant vector is drawn .}
\end{center}
\end{figure}

(4) Use a ruler to measure the magnitude of \(\mathbf{R}\), remembering to convert back to the units of meters using the scale. Use a protractor to measure the direction of R. While the direction of the vector can be specified in many ways, the easiest way is to measure the angle between the vector and the nearest horizontal or vertical axis. Since \(\mathbf{R}\) is south of the eastward pointing axis (the \(x\)-axis), we flip the protractor upside down and measure the angle between the eastward axis and the vector, as illustrated in Figure 5.11.

\begin{figure}[h]
\begin{center}
  \includegraphics[max width=\textwidth]{3b86ae0c-65b9-4aad-b74b-2c0d7621e1ae-11}
\captionsetup{labelformat=empty}
\caption{Figure 5.11 A ruler is used to measure the magnitude of \(\mathbf{R}\), and a protractor is used to measure the direction of \(\mathbf{R}\).}
\end{center}
\end{figure}

In this case, the total displacement \(\mathbf{R}\) has a magnitude of 50 m and points \(7^{\circ}\) south of east. Using its magnitude and direction, this vector can be expressed as\\
\(\mathrm{R}=50 \mathrm{~m}\)

\section*{5.1}
and\\
\(\theta=7^{\circ}\) south of east.

\section*{5.2}
\section*{Discussion}
The head-to-tail graphical method of vector addition works for any number of vectors. It is also important to note that it does not matter in what order the vectors are added. Changing the order does not change the resultant. For example, we could add the vectors as shown in Figure 5.12, and we would still get the same solution.

\begin{figure}[h]
\begin{center}
  \includegraphics[max width=\textwidth]{3b86ae0c-65b9-4aad-b74b-2c0d7621e1ae-12(1)}
\captionsetup{labelformat=empty}
\caption{Figure 5.12 Vectors can be added in any order to get the same result.}
\end{center}
\end{figure}

\section*{Teacher Support}
Teacher Support [BL][OL][AL] Ask three students to enact the situation shown in Figure 5.8. Recall how these forces can be represented in a freebody diagram. Giving values to these vectors, show how these can be added graphically.

\section*{Worked Example}
Subtracting Vectors Graphically: A Woman Sailing a Boat A woman sailing a boat at night is following directions to a dock. The instructions read to first sail 27.5 m in a direction \(66.0^{\circ}\) north of east from her current location, and then travel 30.0 m in a direction \(112^{\circ}\) north of east (or \(22.0^{\circ}\) west of north). If the woman makes a mistake and travels in the opposite direction for the second leg of the trip, where will she end up? The two legs of the woman's trip are illustrated in Figure 5.13.

\begin{figure}[h]
\begin{center}
  \includegraphics[max width=\textwidth]{3b86ae0c-65b9-4aad-b74b-2c0d7621e1ae-12}
\captionsetup{labelformat=empty}
\caption{Figure 5.13 In the diagram, the first leg of the trip is represented by vector A}
\end{center}
\end{figure}

and the second leg is represented by vector \(B\).

\section*{Strategy}
We can represent the first leg of the trip with a vector \(\mathbf{A}\), and the second leg of the trip that she was supposed to take with a vector \(\mathbf{B}\). Since the woman mistakenly travels in the opposite direction for the second leg of the journey, the vector for second leg of the trip she actually takes is \(-\mathbf{B}\). Therefore, she will end up at a location \(\mathbf{A}+(-\mathbf{B})\), or \(\mathbf{A}-\mathbf{B}\). Note that \(-\mathbf{B}\) has the same magnitude as \(\mathbf{B}(30.0 \mathrm{~m})\), but is in the opposite direction, \(68^{\circ}\left(180^{\circ}-112^{\circ}\right)\) south of east, as illustrated in Figure 5.14.\\
\includegraphics[max width=\textwidth, center]{3b86ae0c-65b9-4aad-b74b-2c0d7621e1ae-13(1)}

Figure 5.14 Vector - B represents traveling in the opposite direction of vector B . We use graphical vector addition to find where the woman arrives \(\mathbf{A}+(-\mathbf{B})\). Solution\\
(1) To determine the location at which the woman arrives by accident, draw vectors \(\mathbf{A}\) and \(-\mathbf{B}\).\\
(2) Place the vectors head to tail.\\
(3) Draw the resultant vector \(\mathbf{R}\).\\
(4) Use a ruler and protractor to measure the magnitude and direction of \(\mathbf{R}\).\\
\includegraphics[max width=\textwidth, center]{3b86ae0c-65b9-4aad-b74b-2c0d7621e1ae-13}

Figure 5.15 The vectors are placed head to tail.\\
In this case\\
\(R=23.0 \mathrm{~m}\)\\
5.3\\
and\\
\(\theta=7.5^{\circ}\) south of east.\\
5.4

Discussion\\
Because subtraction of a vector is the same as addition of the same vector with the opposite direction, the graphical method for subtracting vectors works the same as for adding vectors.

\section*{Worked Example}
Adding Velocities: A Boat on a River A boat attempts to travel straight across a river at a speed of \(3.8 \mathrm{~m} / \mathrm{s}\). The river current flows at a speed \(v_{\text {river }}\) of \(6.1 \mathrm{~m} / \mathrm{s}\) to the right. What is the total velocity and direction of the boat? You can represent each meter per second of velocity as one centimeter of vector length in your drawing.

\section*{Strategy}
We start by choosing a coordinate system with its x -axis parallel to the velocity of the river. Because the boat is directed straight toward the other shore, its velocity is perpendicular to the velocity of the river. We draw the two vectors, \(\mathbf{v}_{\text {boat }}\) and \(\mathbf{v}_{\text {river }}\), as shown in Figure 5.16.

Using the head-to-tail method, we draw the resulting total velocity vector from the tail of \(\mathbf{v}_{\text {boat }}\) to the head of \(\mathbf{v}_{\text {river }}\).

\begin{figure}[h]
\begin{center}
  \includegraphics[max width=\textwidth]{3b86ae0c-65b9-4aad-b74b-2c0d7621e1ae-15}
\captionsetup{labelformat=empty}
\caption{Figure 5.16 A boat attempts to travel across a river. What is the total velocity and direction of the boat?}
\end{center}
\end{figure}

Solution\\
By using a ruler, we find that the length of the resultant vector is 7.2 cm , which means that the magnitude of the total velocity is\\
\(\mathrm{v}_{\text {tot }}=7.2 \mathrm{~m} / \mathrm{s}\).\\
5.5

By using a protractor to measure the angle, we find \(\theta=32.0^{\circ}\).\\
Discussion\\
If the velocity of the boat and river were equal, then the direction of the total velocity would have been \(45^{\circ}\). However, since the velocity of the river is greater than that of the boat, the direction is less than \(45^{\circ}\) with respect to the shore, or \(x\) axis.

\section*{Teacher Support}
\section*{Teacher Support}
\section*{Teacher Demonstration}
Plot the way from the classroom to the cafeteria (or any two places in the school on the same level). Ask students to come up with approximate distances. Ask them to do a vector analysis of the path. What is the total distance travelled? What is the displacement?

\section*{Practice Problems}
1.

Vector \(\backslash\) overrightarrow \(\{\backslash \operatorname{text}\{\mathrm{A}\}\}\), having magnitude \(2.5 \backslash, \backslash \operatorname{text}\{\mathrm{~m}\}\), pointing \(37^{\wedge} \backslash\) circ \(\backslash\) ! south of east and vector \textbackslash overrightarrow \(\{\backslash \operatorname{text}\{\mathrm{B}\}\}\) having magnitude \(3.5 \backslash, \backslash \operatorname{text}\{\mathrm{~m}\}\), pointing \(20^{\wedge} \backslash\) circ \(\backslash!\) north of east are added. What is the magnitude of the resultant vector?\\
a. \(1.0 \backslash, \backslash \operatorname{text}\{\mathrm{~m}\}\)\\
b. \(5.3 \backslash, \backslash \operatorname{text}\{\mathrm{~m}\}\)\\
c. \(5.9 \backslash, \backslash \operatorname{text}\{\mathrm{~m}\}\)\\
d. \(6.0 \backslash, \backslash \operatorname{text}\{\mathrm{~m}\}\)\\
2.

A person walks \(32^{\wedge} \backslash\) circ \(\backslash\) ! north of west for \(94 \backslash, \backslash \operatorname{text}\{\mathrm{~m}\}\) and \(35^{\wedge} \backslash\) circ \(\backslash\) ! east of south for \(122 \backslash, \backslash \operatorname{text}\{\mathrm{~m}\}\). What is the magnitude of his displacement?\\
a. \(28 \backslash, \backslash \operatorname{text}\{\mathrm{~m}\}\)\\
b. \(51 \backslash, \backslash \operatorname{text}\{\mathrm{~m}\}\)\\
c. \(180 \backslash, \backslash \operatorname{text}\{\mathrm{~m}\}\)\\
d. \(216 \backslash, \backslash \operatorname{text}\{\mathrm{~m}\}\)

\section*{Virtual Physics}
Vector Addition In this simulation, you will experiment with adding vectors graphically. Click and drag the red vectors from the Grab One basket onto the graph in the middle of the screen. These red vectors can be rotated, stretched, or repositioned by clicking and dragging with your mouse. Check the Show Sum box to display the resultant vector (in green), which is the sum of all of the red vectors placed on the graph. To remove a red vector, drag it to the trash or click the Clear All button if you wish to start over. Notice that, if you click on any of the vectors, the \(|\mathbf{R}|\) is its magnitude, \(\theta\) is its direction with respect to the positive \(x\)-axis, \(\mathbf{R}_{\mathbf{x}}\) is its horizontal component, and \(R_{y}\) is its vertical component. You can check the resultant by lining up the vectors so that the head of the first vector touches the tail of the second. Continue until all of the vectors are aligned together head-to-tail. You will see that the resultant magnitude and angle is the same as the arrow drawn from the tail of the first vector to the head of the last vector. Rearrange the vectors in any order head-to-tail and compare. The resultant will always be the same.

Click to view content

\section*{Grasp Check}
True or False - The more long, red vectors you put on the graph, rotated in any direction, the greater the magnitude of the resultant green vector.\\
a. True\\
b. False

\section*{Check Your Understanding}
3.

While there is no single correct choice for the sign of axes, which of the following are conventionally considered positive?\\
a. backward and to the left\\
b. backward and to the right\\
c. forward and to the right\\
d. forward and to the left\\
4.

True or False-A person walks 2 blocks east and 5 blocks north. Another person walks 5 blocks north and then two blocks east. The displacement of the first person will be more than the displacement of the second person.\\
a. True\\
b. False

\section*{Teacher Support}
Teacher Support Use the Check Your Understanding questions to assess whether students achieve the learning objectives for this section. If students are struggling with a specific objective, the Check Your Understanding will help identify which objective is causing the problem and direct students to the relevant content.

\subsection*{5.2 Vector Addition and Subtraction: Anal tical Methods}
\section*{Section Learning Objectives}
By the end of this section, you will be able to do the following:

\begin{itemize}
  \item Define components of vectors
  \item Describe the analytical method of vector addition and subtraction
  \item Use the analytical method of vector addition and subtraction to solve problems
\end{itemize}

\section*{Teacher Support}
Teacher Support The learning objectives in this section will help your students master the following standards:

\begin{itemize}
  \item (3) Scientific processes. The student uses critical thinking, scientific reasoning, and problem solving to make informed decisions within and outside the classroom. The student is expected to:
  \item (F) express and interpret relationships symbolically in accordance with accepted theories to make predictions and solve problems mathematically, including problems requiring proportional reasoning and graphical vector addition
  \item (4) Science concepts. The student knows and applies the laws governing motion in two dimensions for a variety of situations. The student is expected to:
  \item (E) develop and interpret free-body force diagrams;
  \item (F) identify and describe motion relative to different frames of reference.
\end{itemize}

In addition, the High School Physics Laboratory Manual addresses content in this section in the lab titled: Motion in Two Dimensions, as well as the following standards:

\begin{itemize}
  \item (3) Scientific processes. The student uses critical thinking, scientific reasoning, and problem solving to make informed decisions within and outside the classroom. The student is expected to:
  \item (F) express and interpret relationships symbolically in accordance with accepted theories to make predictions and solve problems mathematically, including problems requiring proportional reasoning and graphical vector addition.
\end{itemize}

\section*{Section Key Terms}
\section*{Components of Vectors}
For the analytical method of vector addition and subtraction, we use some simple geometry and trigonometry, instead of using a ruler and protractor as we did for graphical methods. However, the graphical method will still come in handy to visualize the problem by drawing vectors using the head-to-tail method. The analytical method is more accurate than the graphical method, which is limited by the precision of the drawing. For a refresher on the definitions of the sine, cosine, and tangent of an angle, see Figure 5.17.\\
\includegraphics[max width=\textwidth, center]{3b86ae0c-65b9-4aad-b74b-2c0d7621e1ae-19}

Figure 5.17 For a right triangle, the sine, cosine, and tangent of are defined in terms of the adjacent side, the opposite side, or the hypotenuse. In this figure, \(x\) is the adjacent side, \(y\) is the opposite side, and \(h\) is the hypotenuse.

\section*{Teacher Support}
Teacher Support [BL][OL] Review trigonometric concepts of sine, cosine, tangent and the Pythagorean theorem.

Since, by definition, \(\cos \theta=x / h\), we can find the length \(x\) if we know \(h\) and \(\theta\) by using \(x=h \cos \theta\). Similarly, we can find the length of \(y\) by using \(y=h \sin \theta\). These trigonometric relationships are useful for adding vectors.

When a vector acts in more than one dimension, it is useful to break it down into its x and y components. For a two-dimensional vector, a component is a piece of a vector that points in either the x - or y -direction. Every \(2-\mathrm{d}\) vector can be expressed as a sum of its x and y components.

For example, given a vector like \(\mathbf{A}\) in Figure 5.18, we may want to find what two perpendicular vectors, \(\mathbf{A}_{x}\) and \(\mathbf{A}_{y}\), add to produce it. In this example, \(\mathbf{A}_{x}\) and \(\mathbf{A}_{y}\) form a right triangle, meaning that the angle between them is 90 degrees. This is a common situation in physics and happens to be the least complicated situation trigonometrically.

\begin{figure}[h]
\begin{center}
  \includegraphics[max width=\textwidth]{3b86ae0c-65b9-4aad-b74b-2c0d7621e1ae-20}
\captionsetup{labelformat=empty}
\caption{Figure 5.18 The vector \(\mathbf{A}\), with its tail at the origin of an \(x\) - \(y\)-coordinate system, is shown together with its \(x\) - and \(y\)-components, \(\mathbf{A}_{x}\) and \(\mathbf{A}_{y}\). These vectors form a right triangle.}
\end{center}
\end{figure}

\(\mathbf{A}_{x}\) and \(\mathbf{A}_{y}\) are defined to be the components of \(\mathbf{A}\) along the \(x\) - and \(y\)-axes. The three vectors, \(\mathbf{A}, \mathbf{A}_{x}\), and \(\mathbf{A}_{y}\), form a right triangle.\\
\(\mathbf{A}_{\mathbf{x}}+\mathbf{A}_{\mathbf{y}}=\mathbf{A}\)\\
If the vector \(\mathbf{A}\) is known, then its magnitude \(A\) (its length) and its angle \(\theta\) (its direction) are known. To find \(A_{x}\) and \(A_{y}\), its \(x\) - and \(y\)-components, we use the following relationships for a right triangle:\\
\(A_{x}=A \cos \theta\)\\
and\\
\(A_{y}=A \sin \theta\),\\
where \(A_{x}\) is the magnitude of \(\mathbf{A}\) in the \(x\)-direction, \(A_{y}\) is the magnitude of \(\mathbf{A}\) in the \(y\)-direction, and \(\theta\) is the angle of the resultant with respect to the \(x\)-axis, as shown in Figure 5.19.

\begin{figure}[h]
\begin{center}
  \includegraphics[max width=\textwidth]{3b86ae0c-65b9-4aad-b74b-2c0d7621e1ae-21}
\captionsetup{labelformat=empty}
\caption{Figure 5.19 The magnitudes of the vector components \(\mathbf{A}_{x}\) and \(\mathbf{A}_{y}\) can be related to the resultant vector \(\mathbf{A}\) and the angle \(\theta\) with trigonometric identities. Here we see that \(A_{x}=A \cos \theta\) and \(A_{y}=A \sin \theta\).}
\end{center}
\end{figure}

\section*{Teacher Support}
Teacher Support [BL][OL][AL] Derive the formula for getting the magnitude and direction of a vector.

\section*{Misconception Alert}
Students might be confused between the relationship \(\mathbf{A}_{\mathbf{x}}+\mathbf{A}_{\mathbf{y}}=\mathbf{A}\), which shows the addition of vectors and \(A=\sqrt{A_{x}^{2}+A_{y}^{2}}\) which shows the addition of magnitudes of vectors.

Suppose, for example, that \(\mathbf{A}\) is the vector representing the total displacement of the person walking in a city, as illustrated in Figure 5.20.

\begin{figure}[h]
\begin{center}
  \includegraphics[max width=\textwidth]{3b86ae0c-65b9-4aad-b74b-2c0d7621e1ae-22}
\captionsetup{labelformat=empty}
\caption{Figure 5.20 We can use the relationships \(A_{x}=A \cos \theta\) and \(A_{y}=A \sin \theta\) to determine the magnitude of the horizontal and vertical component vectors in this example.}
\end{center}
\end{figure}

Then \(\mathrm{A}=10.3\) blocks and \(\theta=29.1^{\circ}\), so that

\[
\begin{aligned}
A_{x} & =A \cos \theta \\
& =(10.3 \text { blocks })\left(\cos 29.1^{\circ}\right) \\
& =(10.3 \text { blocks })(0.874) \\
& =9.0 \text { blocks } .
\end{aligned}
\]

5.6

This magnitude indicates that the walker has traveled 9 blocks to the east-in other words, a 9 -block eastward displacement. Similarly,

\[
\begin{aligned}
A_{y} & =A \sin \theta \\
& =(10.3 \text { blocks })\left(\sin 29.1^{\circ}\right) \\
& =(10.3 \text { blocks })(0.846) \\
& =5.0 \text { blocks },
\end{aligned}
\]

\section*{5.7}
indicating that the walker has traveled 5 blocks to the north-a 5 -block northward displacement.

\section*{Analytical Method of Vector Addition and Subtraction}
Calculating a resultant vector (or vector addition) is the reverse of breaking the resultant down into its components. If the perpendicular components \(\mathbf{A}_{x}\) and \(\mathbf{A}_{y}\) of a vector \(\mathbf{A}\) are known, then we can find \(\mathbf{A}\) analytically. How do we do this? Since, by definition,\\
\(\tan \theta=y / x\) (or in this case \(\tan \theta=A_{y} / A_{x}\) ),\\
we solve for \(\theta\) to find the direction of the resultant.\\
\(\theta=\tan ^{-1}\left(A_{y} / A_{x}\right)\)\\
Since this is a right triangle, the Pythagorean theorem \(\left(\mathrm{x}^{2}+\mathrm{y}^{2}=\mathrm{h}^{2}\right)\) for finding the hypotenuse applies. In this case, it becomes\\
\(A^{2}=A_{x}^{2}+A_{y}^{2}\).\\
Solving for A gives\\
\(A=\sqrt{A_{x}^{2}+A_{y}^{2}}\).\\
In summary, to find the magnitude \(A\) and direction \(\theta\) of a vector from its perpendicular components \(\mathbf{A}_{x}\) and \(\mathbf{A}_{y}\), as illustrated in Figure 5.21, we use the following relationships:\\
\(A=\sqrt{A_{x}{ }^{2}+A_{y}{ }^{2}}\)

\[
\theta=\tan ^{-1}\left(A_{y} / A_{x}\right)
\]

\begin{center}
\includegraphics[max width=\textwidth]{3b86ae0c-65b9-4aad-b74b-2c0d7621e1ae-23}
\end{center}

Figure 5.21 The magnitude and direction of the resultant vector \(\mathbf{A}\) can be determined once the horizontal components \(\mathbf{A}_{x}\) and \(\mathbf{A}_{y}\) have been determined.

\section*{Teacher Support}
Teacher Support [BL][OL][AL] Demonstrate a problem involving displacement by physically walking along the specified direction. Show how this can be represented on a graph. Explain that even when solving problems analytically; representing it on a graph would make it easier to visualize the problem.

Sometimes, the vectors added are not perfectly perpendicular to one another. An example of this is the case below, where the vectors \(\mathbf{A}\) and \(\mathbf{B}\) are added to\\
produce the resultant \(\mathbf{R}\), as illustrated in Figure 5.22.

\begin{figure}[h]
\begin{center}
  \includegraphics[max width=\textwidth]{3b86ae0c-65b9-4aad-b74b-2c0d7621e1ae-24}
\captionsetup{labelformat=empty}
\caption{Figure 5.22 Vectors \(\mathbf{A}\) and \(\mathbf{B}\) are two legs of a walk, and \(\mathbf{R}\) is the resultant or total displacement. You can use analytical methods to determine the magnitude and direction of \(\mathbf{R}\).}
\end{center}
\end{figure}

If \(\mathbf{A}\) and \(\mathbf{B}\) represent two legs of a walk (two displacements), then \(\mathbf{R}\) is the total displacement. The person taking the walk ends up at the tip of \(\mathbf{R}\). There are many ways to arrive at the same point. The person could have walked straight ahead first in the \(x\)-direction and then in the \(y\)-direction. Those paths are the \(x\) - and \(y\)-components of the resultant, \(\mathbf{R}_{x}\) and \(\mathbf{R}_{y}\). If we know \(\mathbf{R}_{x}\) and \(\mathbf{R}_{y}\), we can find \(R\) and \(\theta\) using the equations \(R=\sqrt{R_{\mathrm{x}}^{2}+R_{\mathrm{y}}^{2}}\) and \(\theta=\tan ^{-1}\left(R_{y} / R_{x}\right)\).

\begin{enumerate}
  \item Draw in the \(x\) and \(y\) components of each vector (including the resultant) with a dashed line. Use the equations \(A_{x}=A \cos \theta\) and \(A_{y}=A \sin \theta\) to find the components. In Figure 5.23, these components are \(A_{x}, A_{y}, B_{x}\), and \(B_{y}\). Vector \(\mathbf{A}\) makes an angle of \(\theta_{A}\) with the \(x\)-axis, and vector \(\mathbf{B}\) makes and angle of \(\theta_{B}\) with its own \(x\)-axis (which is slightly above the \(x\)-axis used by vector \(\mathbf{A}\) ).
\end{enumerate}

\begin{figure}[h]
\begin{center}
  \includegraphics[max width=\textwidth]{3b86ae0c-65b9-4aad-b74b-2c0d7621e1ae-25}
\captionsetup{labelformat=empty}
\caption{Figure 5.23 To add vectors \(\mathbf{A}\) and \(\mathbf{B}\), first determine the horizontal and vertical components of each vector. These are the dotted vectors \(\mathbf{A}_{x}, \mathbf{A}_{y} \mathbf{B}_{y}\) shown in the image.}
\end{center}
\end{figure}

\begin{enumerate}
  \setcounter{enumi}{1}
  \item Find the \(x\) component of the resultant by adding the \(x\) component of the vectors
\end{enumerate}

\begin{itemize}
  \item \(R_{x}=A_{x}+B_{x}\)\\
and find the \(y\) component of the resultant (as illustrated in Figure 5.24) by adding the \(y\) component of the vectors.
\end{itemize}

\[
R_{y}=A_{y}+B_{y} .
\]

\begin{figure}[h]
\begin{center}
  \includegraphics[max width=\textwidth]{3b86ae0c-65b9-4aad-b74b-2c0d7621e1ae-26}
\captionsetup{labelformat=empty}
\caption{Figure 5.24 The vectors \(\mathbf{A}_{x}\) and \(\mathbf{B}_{x}\) add to give the magnitude of the resultant vector in the horizontal direction, \(R_{\mathrm{x}}\). Similarly, the vectors \(\mathbf{A}_{y}\) and \(\mathbf{B}_{y}\) add to give the magnitude of the resultant vector in the vertical direction, \(R_{\mathrm{y}}\).}
\end{center}
\end{figure}

Now that we know the components of \(\mathbf{R}\), we can find its magnitude and direction.\\
3. To get the magnitude of the resultant R , use the Pythagorean theorem.

\begin{itemize}
  \item \(R=\sqrt{R_{x}^{2}+R_{y}^{2}}\)
\end{itemize}

\begin{enumerate}
  \setcounter{enumi}{3}
  \item To get the direction of the resultant
\end{enumerate}

\begin{itemize}
  \item \(\theta=\tan ^{-1}\left(R_{y} / R_{x}\right)\).
\end{itemize}

\section*{Watch Physics}
Classifying Vectors and Quantities Example This video contrasts and compares three vectors in terms of their magnitudes, positions, and directions.

Click to view content\\
Three vectors, |overrightarrow \(\{\backslash \operatorname{text}\{\mathrm{u}\}\}\), |overrightarrow \(\{\backslash \operatorname{text}\{\mathrm{v}\}\}\), and \textbackslash overrightarrow\{ \(\backslash \operatorname{text}\{\mathrm{w}\}\}\), have the same magnitude of \(5 \backslash, \backslash \operatorname{text}\{\) units \(\}\). Vector \textbackslash overrightarrow\{ \(\backslash \operatorname{text}\{\mathrm{v}\}\) \} points to the northeast. Vector \textbackslash overrightarrow \(\{\backslash \operatorname{text}\{\mathrm{w}\}\}\) points to the southwest exactly opposite to vector \textbackslash overrightarrow \(\{\backslash \operatorname{text}\{\mathrm{u}\}\}\). Vector \textbackslash overrightarrow \(\{\backslash \operatorname{text}\{\mathrm{u}\}\) \} points in the northwest. If the vectors \textbackslash overrightarrow \(\{\backslash \operatorname{text}\{\mathrm{u}\}\}\), \(\backslash\) overrightarrow \(\{\backslash \operatorname{text}\{\mathrm{v}\}\}\), and\\
\textbackslash overrightarrow \(\{\backslash \operatorname{text}\{\mathrm{w}\}\}\) were added together, what would be the magnitude of the resultant vector? Why?\\
a. \(0 \backslash, \backslash \operatorname{text}\{\) units \(\}\). All of them will cancel each other out.\\
b. \(5 \backslash, \backslash\) text\{units\}. Two of them will cancel each other out.\\
c. \(10 \backslash, \backslash\) text\{units\}. Two of them will add together to give the resultant.\\
d. 15 units. All of them will add together to give the resultant.

\section*{Tips For Success}
In the video, the vectors were represented with an arrow above them rather than in bold. This is a common notation in math classes.

\section*{Using the Analytical Method of Vector Addition and Subtraction to Solve Problems}
Figure 5.25 uses the analytical method to add vectors.

\section*{Worked Example}
An Accelerating Subway Train\\
Add the vector \(\mathbf{A}\) to the vector \(\mathbf{B}\) shown in Figure 5.25, using the steps above. The \(x\)-axis is along the east-west direction, and the \(y\)-axis is along the northsouth directions. A person first walks 53.0 m in a direction \(20.0^{\circ}\) north of east, represented by vector \(\mathbf{A}\). The person then walks 34.0 m in a direction \(63.0^{\circ}\) north of east, represented by vector \(\mathbf{B}\).

\begin{figure}[h]
\begin{center}
  \includegraphics[max width=\textwidth]{3b86ae0c-65b9-4aad-b74b-2c0d7621e1ae-27}
\captionsetup{labelformat=empty}
\caption{Figure 5.25 You can use analytical models to add vectors.}
\end{center}
\end{figure}

\section*{Strategy}
The components of \(\mathbf{A}\) and \(\mathbf{B}\) along the \(x\) - and \(y\)-axes represent walking due east and due north to get to the same ending point. We will solve for these components and then add them in the x -direction and y -direction to find the resultant.

\section*{Solution}
First, we find the components of \(\mathbf{A}\) and \(\mathbf{B}\) along the \(x\) - and \(y\)-axes. From the problem, we know that \(A=53.0 \mathrm{~m}, \theta_{\mathrm{A}}=20.0^{\circ}, B=34.0 \mathrm{~m}\), and \(\theta_{\mathrm{B}}=63.0^{\circ}\). We find the \(x\)-components by using \(A_{x}=A \cos \theta\), which gives

\[
\begin{aligned}
A_{x} & =A \cos \theta_{A}=(53.0 \mathrm{~m})\left(\cos 20.0^{\circ}\right) \\
& =(53.0 \mathrm{~m})(0.940)=49.8 \mathrm{~m}
\end{aligned}
\]

and

\[
\begin{aligned}
B_{x} & =B \cos \theta_{B}=(34.0 \mathrm{~m})\left(\cos 63.0^{\circ}\right) \\
& =(34.0 \mathrm{~m})(0.454)=15.4 \mathrm{~m} .
\end{aligned}
\]

Similarly, the \(y\)-components are found using \(A_{y}=A \sin \theta_{A}\)

\[
\begin{aligned}
A_{y} & =A \sin \theta_{A}=(53.0 \mathrm{~m})\left(\sin 20.0^{\circ}\right) \\
& =(53.0 \mathrm{~m})(0.342)=18.1 \mathrm{~m}
\end{aligned}
\]

and

\[
\begin{aligned}
B_{y} & =B \sin \theta_{B}=(34.0 \mathrm{~m})\left(\sin 63.0^{\circ}\right) \\
& =(34.0 \mathrm{~m})(0.891)=30.3 \mathrm{~m} .
\end{aligned}
\]

The \(x\) - and \(y\)-components of the resultant are\\
\(R_{x}=A_{x}+B_{x}=49.8 \mathrm{~m}+15.4 \mathrm{~m}=65.2 \mathrm{~m}\)\\
and\\
\(R_{y}=A_{y}+B_{y}=18.1 \mathrm{~m}+30.3 \mathrm{~m}=48.4 \mathrm{~m}\).\\
Now we can find the magnitude of the resultant by using the Pythagorean theorem\\
\(R=\sqrt{R_{x}^{2}+R_{y}^{2}}=\sqrt{(65.2)^{2}+(48.4)^{2}} \mathrm{~m}\)\\
5.8\\
so that\\
\(R=\sqrt{6601 \mathrm{~m}}=81.2 \mathrm{~m}\).\\
Finally, we find the direction of the resultant\\
\(\theta=\tan ^{-1}\left(R_{y} / R_{x}\right)=+\tan ^{-1}(48.4 / 65.2)\).

This is\\
\(\theta=\tan ^{-1}(0.742)=36.6^{\circ}\).\\
Discussion\\
This example shows vector addition using the analytical method. Vector subtraction using the analytical method is very similar. It is just the addition of a negative vector. That is, \(\mathrm{A}-\mathrm{B} \equiv \mathrm{A}+(-\mathrm{B})\). The components of -B are the negatives of the components of B . Therefore, the \(x\) - and \(y\)-components of the resultant \(\mathrm{A}-\mathrm{B}=\mathrm{R}\) are\\
\(R_{x}=A_{x}+-B_{x}\)\\
and\\
\(R_{y}=A_{y}+-B_{y}\)\\
and the rest of the method outlined above is identical to that for addition.

\section*{Practice Problems}
5.

What is the magnitude of a vector whose \(x\)-component is 4 cm and whose \(y\) component is 3 cm ?\\
a. 1 cm\\
b. 5 cm\\
c. 7 cm\\
d. 25 cm\\
6.

What is the magnitude of a vector that makes an angle of \(30^{\circ}\) to the horizontal and whose \(x\)-component is 3 units?\\
a. 2.61 units\\
b. 3.00 units\\
c. 3.46 units\\
d. 6.00 units

\section*{Links To Physics}
\section*{Atmospheric Science}
\begin{figure}[h]
\begin{center}
  \includegraphics[max width=\textwidth]{3b86ae0c-65b9-4aad-b74b-2c0d7621e1ae-30}
\captionsetup{labelformat=empty}
\caption{Figure 5.26 This picture shows Bert Foord during a television Weather Forecast from the Meteorological Office in 1963. (BBC TV)}
\end{center}
\end{figure}

Atmospheric science is a physical science, meaning that it is a science based heavily on physics. Atmospheric science includes meteorology (the study of weather) and climatology (the study of climate). Climate is basically the average weather over a longer time scale. Weather changes quickly over time, whereas the climate changes more gradually.

The movement of air, water and heat is vitally important to climatology and meteorology. Since motion is such a major factor in weather and climate, this field uses vectors for much of its math.

Vectors are used to represent currents in the ocean, wind velocity and forces acting on a parcel of air. You have probably seen a weather map using vectors to show the strength (magnitude) and direction of the wind.

Vectors used in atmospheric science are often three-dimensional. We won't cover three-dimensional motion in this text, but to go from two-dimensions to three-dimensions, you simply add a third vector component. Three-dimensional motion is represented as a combination of \(x\)-, \(y\) - and \(z\) components, where \(z\) is the altitude.

Vector calculus combines vector math with calculus, and is often used to find the rates of change in temperature, pressure or wind speed over time or distance. This is useful information, since atmospheric motion is driven by changes in pressure or temperature. The greater the variation in pressure over a given distance, the stronger the wind to try to correct that imbalance. Cold air tends to be more dense and therefore has higher pressure than warm air. Higher pressure air rushes into a region of lower pressure and gets deflected by the spinning of the Earth, and friction slows the wind at Earth's surface.

Finding how wind changes over distance and multiplying vectors lets meteorologists, like the one shown in Figure 5.26, figure out how much rotation (spin) there is in the atmosphere at any given time and location. This is an important tool for tornado prediction. Conditions with greater rotation are more likely to produce tornadoes.

Why are vectors used so frequently in atmospheric science?\\
a. Vectors have magnitude as well as direction and can be quickly solved through scalar algebraic operations.\\
b. Vectors have magnitude but no direction, so it becomes easy to express physical quantities involved in the atmospheric science.\\
c. Vectors can be solved very accurately through geometry, which helps to make better predictions in atmospheric science.\\
d. Vectors have magnitude as well as direction and are used in equations that describe the three dimensional motion of the atmosphere.

\section*{Check Your Understanding}
7.

Between the analytical and graphical methods of vector additions, which is more accurate? Why?\\
a. The analytical method is less accurate than the graphical method, because the former involves geometry and trigonometry.\\
b. The analytical method is more accurate than the graphical method, because the latter involves some extensive calculations.\\
c. The analytical method is less accurate than the graphical method, because the former includes drawing all figures to the right scale.\\
d. The analytical method is more accurate than the graphical method, because the latter is limited by the precision of the drawing.\\
8.

What is a component of a two dimensional vector?\\
a. A component is a piece of a vector that points in either the \(x\) or \(y\) direction.\\
b. A component is a piece of a vector that has half of the magnitude of the original vector.\\
c. A component is a piece of a vector that points in the direction opposite to the original vector.\\
d. A component is a piece of a vector that points in the same direction as original vector but with double of its magnitude.\\
9.

How can we determine the global angle ∖ theta (measured counter-clockwise from positive x ) if we know A\_x and A\_y?\\
a. \(\backslash\) theta \(=\backslash \cos ^{\wedge}\{-1\} \backslash \operatorname{frac}\{\) A\_y \(\}\{\) A\_x \(\}\)\\
b. \(\backslash\) theta \(=\backslash \cot ^{\wedge}\{-1\} \backslash \operatorname{frac}\{\) A\_y \(\}\{\) A\_x \(\}\)\\
c. \(\backslash\) theta \(=\backslash \sin ^{\wedge}\{-1\} \backslash\) frac \(\{\) A\_y \(\}\{\) A\_x \(\}\)\\
d. \(\backslash\) theta \(=\backslash \tan ^{\wedge}\{-1\} \backslash \operatorname{frac}\{\) A\_y \(\}\{\) A\_x \(\}\)\\
10.

How can we determine the magnitude of a vector if we know the magnitudes of its components?\\
a. \(|\overrightarrow{\mathrm{A}}|=A_{x}+A_{y}\)\\
b. \(|\overrightarrow{\mathrm{A}}|=A_{x}{ }^{2}+A_{y}{ }^{2}\)\\
c. \(|\overrightarrow{\mathrm{A}}|=\sqrt{A_{x}{ }^{2}+A_{y}{ }^{2}}\)\\
d. \(|\overrightarrow{\mathrm{A}}|=\left(A_{x}{ }^{2}+A_{y}{ }^{2}\right)^{2}\)

\section*{Teacher Support}
Teacher Support Use the Check Your Understanding questions to assess whether students achieve the learning objectives for this section. If students are struggling with a specific objective, the Check Your Understanding will help identify which objective is causing the problem and direct students to the relevant content.

\subsection*{5.3 Projectile Motion}
\section*{Section Learning Objectives}
By the end of this section, you will be able to do the following:

\begin{itemize}
  \item Describe the properties of projectile motion
  \item Apply kinematic equations and vectors to solve problems involving projectile motion
\end{itemize}

\section*{Teacher Support}
\section*{Teacher Support}
The learning objectives in this section will help your students master the following standards:

\begin{itemize}
  \item (4) Science concepts. The student knows and applies the laws governing motion in two dimensions for a variety of situations. The student is expected to:
  \item (C) analyze and describe accelerated motion in two dimensions using equations.
\end{itemize}

In addition, the High School Physics Laboratory Manual addresses content in this section in the lab titled: Motion in Two Dimensions, as well as the following standards:

\begin{itemize}
  \item (4) Science concepts. The student knows and applies the laws governing motion in a variety of situations. The student is expected to:
  \item (C) analyze and describe accelerated motion in two dimensions using equations, including projectile and circular examples.
\end{itemize}

\section*{Section Key Terms}
\begin{center}
\begin{tabular}{|l|l|l|}
\hline
air resistance & maximum height (of a projectile) & projectile \\
\hline
projectile motion & range & trajectory \\
\hline
\end{tabular}
\end{center}

\section*{Properties of Projectile Motion}
Projectile motion is the motion of an object thrown (projected) into the air. After the initial force that launches the object, it only experiences the force of gravity. The object is called a projectile, and its path is called its trajectory. As an object travels through the air, it encounters a frictional force that slows its motion called air resistance. Air resistance does significantly alter trajectory motion, but due to the difficulty in calculation, it is ignored in introductory physics.

\section*{Teacher Support}
\section*{Teacher Support}
[BL][OL] Review addition of vectors graphically and analytically.\\[0pt]
[BL][OL][AL] Explain the term projectile motion. Ask students to guess what the motion of a projectile might depend on? Is the initial velocity important? Is the angle important? How will these things affect its height and the distance it covers? Introduce the concept of air resistance. Review kinematic equations.

The most important concept in projectile motion is that horizontal and vertical motions are independent, meaning that they don't influence one another. Figure 5.27 compares a cannonball in free fall (in blue) to a cannonball launched horizontally in projectile motion (in red). You can see that the cannonball in free fall falls at the same rate as the cannonball in projectile motion. Keep in mind that if the cannon launched the ball with any vertical component to the velocity, the vertical displacements would not line up perfectly.

Since vertical and horizontal motions are independent, we can analyze them separately, along perpendicular axes. To do this, we separate projectile motion into the two components of its motion, one along the horizontal axis and the other along the vertical.

\begin{figure}[h]
\begin{center}
  \includegraphics[max width=\textwidth]{3b86ae0c-65b9-4aad-b74b-2c0d7621e1ae-34(1)}
\captionsetup{labelformat=empty}
\caption{Figure 5.27 The diagram shows the projectile motion of a cannonball shot at a horizontal angle versus one dropped with no horizontal velocity. Note that both cannonballs have the same vertical position over time.}
\end{center}
\end{figure}

We'll call the horizontal axis the \(x\)-axis and the vertical axis the \(y\)-axis. For notation, \(\mathbf{d}\) is the total displacement, and \(\mathbf{x}\) and \(\mathbf{y}\) are its components along the horizontal and vertical axes. The magnitudes of these vectors are \(x\) and \(y\), as illustrated in Figure 5.28.

\begin{figure}[h]
\begin{center}
  \includegraphics[max width=\textwidth]{3b86ae0c-65b9-4aad-b74b-2c0d7621e1ae-34}
\captionsetup{labelformat=empty}
\caption{Figure 5.28 A boy kicks a ball at angle \(\theta\), and it is displaced a distance of \(\mathbf{s}\) along its trajectory.}
\end{center}
\end{figure}

As usual, we use velocity, acceleration, and displacement to describe motion. We must also find the components of these variables along the \(x\) - and \(y\)-axes. The components of acceleration are then very simple \(\mathbf{a}_{\mathbf{y}}=-\mathbf{g}=-9.80 \mathrm{~m} / \mathrm{s}^{2}\). Note that this definition defines the upwards direction as positive. Because gravity is vertical, \(\mathbf{a}_{\mathbf{x}}=0\). Both accelerations are constant, so we can use the kinematic equations. For review, the kinematic equations from a previous chapter are summarized in Table 5.1.

\begin{table}[h]
\begin{center}
\begin{tabular}{|l|}
\hline
\(\mathbf{x}=\mathbf{x}_{0}+\mathbf{v}_{\mathbf{a v g}} t(\) when \\
\(\mathbf{a}=\) constant \()\) \\
\hline
\(\mathbf{v}_{\mathbf{a v g}}=\frac{\mathbf{v}_{0}+\mathbf{v}}{2}(\) when \(\mathbf{a}=0)\) \\
\hline
\(\mathbf{v}=\mathbf{v}_{0}+\mathbf{a} t\) \\
\hline
\(\mathbf{x}=\mathbf{x}_{0}+\mathbf{v}_{0} t+\frac{1}{2} \mathbf{a} t^{2}\) \\
\hline
\(\mathbf{v}^{2}=\mathbf{v}_{0}^{2}+2 \mathbf{a}\left(\mathbf{x}-\mathbf{x}_{0}\right)\) \\
\hline
\end{tabular}
\captionsetup{labelformat=empty}
\caption{Table 5.1 Summary of Kinematic Equations (constant a)}
\end{center}
\end{table}

Where \(\mathbf{x}\) is position, \(\mathbf{x}_{0}\) is initial position, \(\mathbf{v}\) is velocity, \(\mathbf{v}_{\text {avg }}\) is average velocity, \(t\) is time and a is acceleration.

\section*{Solve Problems Involving Projectile Motion}
The following steps are used to analyze projectile motion:

\begin{enumerate}
  \item Separate the motion into horizontal and vertical components along the x - and y -axes.
\end{enumerate}

These axes are perpendicular, so \(A_{x}=A \cos \theta\) and \(A_{y}=A \sin \theta\) are used. The magnitudes of the displacement s along x - and y -axes are called \(x\) and \(y\). The magnitudes of the components of the velocity v are \(v_{x}=v \cos \theta\) and \(v_{y}=v \sin \theta\), where \(\boldsymbol{v}\) is the magnitude of the velocity and \(\theta\) is its direction. Initial values are denoted with a subscript 0 .\\
2. Treat the motion as two independent one-dimensional motions, one horizontal and the other vertical. The kinematic equations for horizontal and vertical motion take the following forms\\
Horizontal Motion( \(\mathbf{a}_{x}=0\) )\\
\(x=x_{0}+v_{x} t\)\\
\(v_{x}=v_{0 x}=\mathbf{v}_{\mathrm{x}}=\) velocity is a constant.\\
Vertical motion (assuming positive is up \(\mathbf{a}_{y}=-\mathbf{g}=-9.80 \mathrm{~m} / \mathrm{s}^{2}\) )\\
\(y=y_{0}+\frac{1}{2}\left(v_{0 y}+v_{y}\right) t\)\\
\(v_{y}=v_{0 y}-\mathbf{g} t\)\\
\(y=y_{0}+v_{0 y} t-\frac{1}{2} \mathbf{g} t^{2}\)\\
\(v_{y}^{2}=v_{0 y}^{2}-2 g\left(y-y_{0}\right)\)\\
3. Solve for the unknowns in the two separate motions (one horizontal and one vertical). Note that the only common variable between the motions is time \(t\). The problem solving procedures here are the same as for one-dimensional kinematics.\\
4. Recombine the two motions to find the total displacement s and velocity v . We can use the analytical method of vector addition, which uses \(A=\sqrt{A_{x}{ }^{2}+A_{y}{ }^{2}}\) and \(\theta= \tan ^{-1}\left(A_{y} / A_{x}\right)\) to find the magnitude and direction of the total displacement and velocity. Displacement\\
\(\mathbf{d}=\sqrt{x^{2}+y^{2}}\)\\
\(\theta=\tan ^{-1}(y / x)\)\\
Velocity\\
\(\mathbf{v}=\sqrt{\mathbf{v}_{x}^{2}+\mathbf{v}_{y}^{2}}\)\\
\(\theta_{v}=\tan ^{-1}\left(\mathbf{v}_{y} / \mathbf{v}_{x}\right)\)\\
\(\theta\) is the direction of the displacement d , and \(\theta_{\mathrm{v}}\) is the direction of the velocity v . (See\\
Figure 5.29

\begin{figure}[h]
\begin{center}
  \includegraphics[max width=\textwidth]{3b86ae0c-65b9-4aad-b74b-2c0d7621e1ae-37}
\captionsetup{labelformat=empty}
\caption{Figure 5.29 (a) We analyze two-dimensional projectile motion by breaking it into two independent one-dimensional motions along the vertical and horizontal axes. (b) The horizontal motion is simple, because \(\mathbf{a}_{x}=0\) and \(v_{x}\) is thus constant. (c) The velocity in the vertical direction begins to decrease as the object rises; at its highest point, the vertical velocity is zero. As the object falls towards the Earth again, the vertical velocity increases again in magnitude but points in the opposite direction to the initial vertical velocity. (d) The \(x\) - and \(y\)-motions are recombined to give the total velocity at any given point on the trajectory.}
\end{center}
\end{figure}

\section*{Teacher Support}
\section*{Teacher Support}
\section*{Teacher Demonstration}
Demonstrate the path of a projectile by doing a simple demonstration. Toss a dark beanbag in front of a white board so that students can get a good look at the projectile path. Vary the toss angles, so different paths can be displayed. This demonstration could be extended by using digital photography. Draw a reference grid on the whiteboard, then toss the bag at different angles while taking a video. Replay this in slow motion to observe and compare the altitudes and trajectories.

\section*{Tips For Success}
For problems of projectile motion, it is important to set up a coordinate system. The first step is to choose an initial position for \(\mathbf{x}\) and \(\mathbf{y}\). Usually, it is simplest to set the initial position of the object so that \(\mathbf{x}_{0}=0\) and \(\mathbf{y}_{0}=0\).

\section*{Watch Physics}
\section*{Projectile at an Angle}
This video presents an example of finding the displacement (or range) of a projectile launched at an angle. It also reviews basic trigonometry for finding the sine, cosine and tangent of an angle.

\section*{Click to view content}
Assume the ground is uniformly level. If the horizontal component of a projectile's velocity is doubled, but the vertical component is unchanged, what is the effect on the time of flight?\\
a. The time to reach the ground would remain the same since the vertical component is unchanged.\\
b. The time to reach the ground would remain the same since the vertical component of the velocity also gets doubled.\\
c. The time to reach the ground would be halved since the horizontal component of the velocity is doubled.\\
d. The time to reach the ground would be doubled since the horizontal component of the velocity is doubled.

\section*{Worked Example}
\section*{A Fireworks Projectile Explodes High and Away}
During a fireworks display like the one illustrated in Figure 5.30, a shell is shot into the air with an initial speed of \(70.0 \mathrm{~m} / \mathrm{s}\) at an angle of \(75^{\circ}\) above the horizontal. The fuse is timed to ignite the shell just as it reaches its highest point above the ground. (a) Calculate the height at which the shell explodes. (b) How much time passed between the launch of the shell and the explosion? (c) What is the horizontal displacement of the shell when it explodes?

\begin{figure}[h]
\begin{center}
  \includegraphics[max width=\textwidth]{3b86ae0c-65b9-4aad-b74b-2c0d7621e1ae-39}
\captionsetup{labelformat=empty}
\caption{Figure 5.30 The diagram shows the trajectory of a fireworks shell.}
\end{center}
\end{figure}

\section*{Strategy}
The motion can be broken into horizontal and vertical motions in which \(\mathbf{a}_{x}=0\) and \(\mathbf{a}_{y}=\mathbf{g}\). We can then define \(\mathbf{x}_{0}\) and \(\mathbf{y}_{0}\) to be zero and solve for the maximum height.

\section*{Solution for (a)}
By height we mean the altitude or vertical position \(\mathbf{y}\) above the starting point. The highest point in any trajectory, the maximum height, is reached when \(\mathbf{v}_{y}=0\); this is the moment when the vertical velocity switches from positive (upwards) to negative (downwards). Since we know the initial velocity, initial position, and the value of \(\mathbf{v}_{\mathrm{y}}\) when the firework reaches its maximum height, we use the following equation to find \(\mathbf{y}\)\\
\(\mathbf{v}_{y}^{2}=\mathbf{v}_{0 y}^{2}-2 \mathbf{g}\left(\mathbf{y}-y_{0}\right)\).\\
Because \(\mathbf{y}_{0}\) and \(\mathbf{v}_{y}\) are both zero, the equation simplifies to\\
\(0=\mathbf{v}_{0 y}^{2}-2 \mathbf{g y}\).\\
Solving for \(\mathbf{y}\) gives\\
\(\mathbf{y}=\frac{\mathbf{v}_{0 y}^{2}}{2 \mathbf{g}}\).\\
Now we must find \(\mathbf{v}_{0 y}\), the component of the initial velocity in the \(y\)-direction. It is given by \(\mathbf{v}_{0 y}=\mathbf{v}_{0} \sin \theta\), where \(\mathbf{v}_{0 y}\) is the initial velocity of \(70.0 \mathrm{~m} / \mathrm{s}\), and \(\theta=75^{\circ}\) is the initial angle. Thus,\\
\(\mathbf{v}_{0 y}=\mathbf{v}_{0} \sin \theta_{0}=(70.0 \mathrm{~m} / \mathrm{s})\left(\sin 75^{\circ}\right)=67.6 \mathrm{~m} / \mathrm{s}\)\\
and \(\mathbf{y}\) is\\
\(\mathbf{y}=\frac{(67.6 \mathrm{~m} / \mathrm{s})^{2}}{2\left(9.80 \mathrm{~m} / \mathrm{s}^{2}\right),}\)\\
so that\\
\(\mathbf{y}=233 \mathrm{~m}\).\\
Discussion for (a)\\
Since up is positive, the initial velocity and maximum height are positive, but the acceleration due to gravity is negative. The maximum height depends only on the vertical component of the initial velocity. The numbers in this example are reasonable for large fireworks displays, the shells of which do reach such heights before exploding.

Solution for (b)\\
There is more than one way to solve for the time to the highest point. In this case, the easiest method is to use \(\mathbf{y}=\mathbf{y}_{0}+\frac{1}{2}\left(\mathbf{v}_{0 y}+\mathbf{v}_{y}\right) t\). Because \(y_{0}\) is zero, this equation reduces to \(\mathbf{y}=\frac{1}{2}\left(\mathbf{v}_{0 y}+\mathbf{v}_{y}\right) t\).

Note that the final vertical velocity, \(\mathbf{v}_{y}\), at the highest point is zero. Therefore,

\[
\begin{aligned}
t & =\frac{2 \mathbf{y}}{\left(\mathbf{v}_{0 \mathbf{y}}+\mathbf{v}_{y}\right)}=\frac{2(233 \mathrm{~m})}{(67.6 \mathrm{~m} / \mathrm{s})} \\
& =6.90 \mathrm{~s}
\end{aligned}
\]

Discussion for (b)\\
This time is also reasonable for large fireworks. When you are able to see the launch of fireworks, you will notice several seconds pass before the shell explodes. Another way of finding the time is by using \(\mathbf{y}=\mathbf{y}_{0}+\mathbf{v}_{0 \mathbf{y}} t-\frac{1}{2} \mathbf{g} t^{2}\), and solving the quadratic equation for \(t\).

Solution for (c)\\
Because air resistance is negligible, \(\mathbf{a}_{\mathrm{x}}=0\) and the horizontal velocity is constant. The horizontal displacement is horizontal velocity multiplied by time as given by \(\mathbf{x}=\mathbf{x}_{0}+\mathbf{v}_{x} t\), where \(\mathbf{x}_{0}\) is equal to zero\\
\(\mathbf{x}=\mathbf{v}_{x} t\), where \(\mathbf{v}_{x}\) is the \(x\)-component of the velocity, which is given by \(\mathbf{v}_{x}=\mathbf{v}_{0} \cos \theta_{0}\). Now, \(\mathbf{v}_{x}=\mathbf{v}_{0} \cos \theta_{0}=(70.0 \mathrm{~m} / \mathrm{s})\left(\cos 75^{\circ}\right)=18.1 \mathrm{~m} / \mathrm{s}\).

The time \(t\) for both motions is the same, and so \(\mathbf{x}\) is\\
\(\mathbf{x}=(18.1 \mathrm{~m} / \mathrm{s})(6.90 \mathrm{~s})=125 \mathrm{~m}\).\\
Discussion for (c)\\
The horizontal motion is a constant velocity in the absence of air resistance. The horizontal displacement found here could be useful in keeping the fireworks fragments from falling on spectators. Once the shell explodes, air resistance has a major effect, and many fragments will land directly below, while some of the fragments may now have a velocity in the -x direction due to the forces of the explosion.

\section*{Teacher Support}
\section*{Teacher Support}
[BL][OL][AL] Talk about the sample problem. Discuss the variables or unknowns in each part of the problem Ask students which kinematic equations may be best suited to solve the different parts of the problem.

The expression we found for \(\mathbf{y}\) while solving part (a) of the previous problem works for any projectile motion problem where air resistance is negligible. Call the maximum height \(\mathbf{y}=h\); then,\\
\(h=\frac{\mathbf{v}_{0 y}^{2}}{2 \mathbf{g}}\).\\
This equation defines the maximum height of a projectile. The maximum height depends only on the vertical component of the initial velocity.

\section*{Worked Example}
\section*{Calculating Projectile Motion: Hot Rock Projectile}
Suppose a large rock is ejected from a volcano, as illustrated in Figure 5.31, with a speed of \(25.0 \mathrm{~m} / \mathrm{s}\) and at an angle \(35{ }^{\circ}\) above the horizontal. The rock strikes the side of the volcano at an altitude 20.0 m lower than its starting point. (a) Calculate the time it takes the rock to follow this path.

\begin{figure}[h]
\begin{center}
  \includegraphics[max width=\textwidth]{3b86ae0c-65b9-4aad-b74b-2c0d7621e1ae-42}
\captionsetup{labelformat=empty}
\caption{Figure 5.31 The diagram shows the projectile motion of a large rock from a volcano.}
\end{center}
\end{figure}

\section*{Strategy}
Breaking this two-dimensional motion into two independent one-dimensional motions will allow us to solve for the time. The time a projectile is in the air depends only on its vertical motion.

\section*{Solution}
While the rock is in the air, it rises and then falls to a final position 20.0 m lower than its starting altitude. We can find the time for this by using\\
\(\mathbf{y}=\mathbf{y}_{0}+\mathbf{v}_{0 \mathbf{y}} t-\frac{1}{2} \mathbf{g} t^{2}\).\\
If we take the initial position \(\mathbf{y}_{0}\) to be zero, then the final position is \(\mathbf{y}=-20.0 \mathrm{~m}\). Now the initial vertical velocity is the vertical component of the initial velocity, found from\\
\(\mathbf{v}_{0 \quad y}=\mathbf{v}_{0} \sin \theta_{0}=(25.0 \mathrm{~m} / \mathrm{s})\left(\sin 35^{\circ}\right)=14.3 \mathrm{~m} / \mathrm{s}\).\\
5.9

Substituting known values yields\\
\(-20.0 \mathrm{~m}=(14.3 \mathrm{~m} / \mathrm{s}) t-\left(4.90 \mathrm{~m} / \mathrm{s}^{2}\right) t^{2}\).\\
Rearranging terms gives a quadratic equation in \(t\)\\
\(\left(4.90 \mathrm{~m} / \mathrm{s}^{2}\right) t^{2}-(14.3 \mathrm{~m} / \mathrm{s}) t-(20.0 \mathrm{~m})=0\).\\
This expression is a quadratic equation of the form \(a t^{2}+b t+c=0\), where the constants are \(a =4.90, b=-14.3\), and \(c=-20.0\). Its solutions are given by the quadratic formula\\
\(t=\frac{-b \pm \sqrt{b^{2}-4 a c}}{2 a}\).

This equation yields two solutions \(t=3.96\) and \(t=-1.03\). You may verify these solutions as an exercise. The time is \(\mathrm{t}=3.96 \mathrm{~s}\) or -1.03 s . The negative value of time implies an event before the start of motion, so we discard it. Therefore,\\
\(t=3.96 \mathrm{~s}\).

\section*{Discussion}
The time for projectile motion is completely determined by the vertical motion. So any projectile that has an initial vertical velocity of \(14.3 \mathrm{~m} / \mathrm{s}\) and lands 20.0 m below its starting altitude will spend 3.96 s in the air.

\section*{Practice Problems}
11.

If an object is thrown horizontally, travels with an average x -component of its velocity equal to \(5 \backslash, \mid \operatorname{text}\{\mathrm{m} / \mathrm{s}\}\), and does not hit the ground, what will be the x -component of the displacement after \(20, \backslash \operatorname{text}\{\mathrm{~s}\}\) ?\\
a. \(\quad\{-100\} \backslash, \mid \operatorname{text}\{m\}\)\\
b. \(\{-4\} \backslash, \mid \operatorname{text}\{\mathrm{m}\}\)\\
c. \(\quad 4 \backslash, \mid \operatorname{text}\{\mathrm{m}\}\)\\
d. \(100 \backslash, \mid \operatorname{text}\{\mathrm{m}\}\)\\
12.

If a ball is thrown straight up with an initial velocity of \(20 \backslash, \backslash \operatorname{text}\{\mathrm{~m} / \mathrm{s}\}\) upward, what is the maximum height it will reach?\\
a. \(\quad\{-20.4\} \backslash, \backslash \operatorname{text}\{\mathrm{m}\}\)\\
b. \(\quad\{-1.02\} \backslash, \mid \operatorname{text}\{\mathrm{m}\}\)\\
c. \(\quad 1.02 \backslash, \backslash \operatorname{text}\{\mathrm{~m}\}\)\\
d. \(20.4 \backslash, \backslash \operatorname{text}\{\mathrm{~m}\}\)

The fact that vertical and horizontal motions are independent of each other lets us predict the range of a projectile. The range is the horizontal distance \(\mathbf{R}\) traveled by a projectile on level ground, as illustrated in Figure 5.32. Throughout history, people have been interested in finding the range of projectiles for practical purposes, such as aiming cannons.

\begin{figure}[h]
\begin{center}
  \includegraphics[max width=\textwidth]{3b86ae0c-65b9-4aad-b74b-2c0d7621e1ae-44}
\captionsetup{labelformat=empty}
\caption{Figure 5.32 Trajectories of projectiles on level ground. (a) The greater the initial speed \(v_{0}\), the greater the range for a given initial angle. (b) The effect of initial angle \(\theta_{0}\) on the range of a projectile with a given initial speed. Note that any combination of trajectories that add to 90 degrees will have the same range in the absence of air resistance, although the maximum heights of those paths are different.}
\end{center}
\end{figure}

How does the initial velocity of a projectile affect its range? Obviously, the greater the initial speed \(v_{0}\), the greater the range, as shown in the figure above. The initial angle \(\theta_{0}\) also has a dramatic effect on the range. When air resistance is negligible, the range \(R\) of a projectile on level ground is

\[
R=\frac{v_{0}^{2} \sin 2 \theta_{0}}{\mathbf{g}},
\]

where \(\boldsymbol{v}_{0}\) is the initial speed and \(\theta_{0}\) is the initial angle relative to the horizontal. It is important to note that the range doesn't apply to problems where the initial and final y position are different, or to cases where the object is launched perfectly horizontally.

\section*{Virtual Physics}
\section*{Projectile Motion}
In this simulation you will learn about projectile motion by blasting objects out of a cannon. You can choose between objects such as a tank shell, a golf ball or even a Buick. Experiment with changing the angle, initial speed, and mass, and adding in air resistance. Make a game out of this simulation by trying to hit the target.

\section*{Click to view content}
Consider the simulation. If a projectile is launched on level ground, what launch angle maximizes the range of the projectile?\\
a. \(\quad 0^{\wedge} \backslash\) circ\\
b. \(\quad 30^{\wedge} \backslash \mathrm{circ}\)\\
c. \(45^{\wedge} \backslash \mathrm{circ}\)\\
d. \(\quad 60^{\wedge} \backslash \mathrm{circ}\)

\section*{Check Your Understanding}
13.

What is projectile motion?\\
a. Projectile motion is the motion of an object projected into the air and moving under the influence of gravity.\\
b. Projectile motion is the motion of an object projected into the air and moving independently of gravity.\\
c. Projectile motion is the motion of an object projected vertically upward into the air and moving under the influence of gravity.\\
d. Projectile motion is the motion of an object projected horizontally into the air and moving independently of gravity.\\
14.

What is the force experienced by a projectile after the initial force that launched it into the air in the absence of air resistance?\\
a. The nuclear force\\
b. The gravitational force\\
c. The electromagnetic force\\
d. The contact force

\section*{Teacher Support}
\section*{Teacher Support}
Use the Check Your Understanding questions to assess whether students achieve the learning objectives for this section. If students are struggling with a specific objective, the Check Your

Understanding will help identify which objective is causing the problem and direct students to the relevant content.

\subsection*{5.4 Inclined Planes}
\section*{Section Learning Objectives}
By the end of this section, you will be able to do the following:

\begin{itemize}
  \item Distinguish between static friction and kinetic friction
  \item Solve problems involving inclined planes
\end{itemize}

\section*{Teacher Support}
Teacher Support The learning objectives in this section will help your students master the following standards:

\begin{itemize}
  \item (4) Science concepts. The student knows and applies the laws governing motion in two dimensions for a variety of situations. The student is expected to:
  \item (D) calculate the effect of forces on objects, including the law of inertia, the relationship between force and acceleration, and the nature of force pairs between objects.
\end{itemize}

\section*{Section Key Terms}
\section*{Static Friction and Kinetic Friction}
Recall from the previous chapter that friction is a force that opposes motion, and is around us all the time. Friction allows us to move, which you have discovered if you have ever tried to walk on ice.

There are different types of friction-kinetic and static. Kinetic friction acts on an object in motion, while static friction acts on an object or system at rest. The maximum static friction is usually greater than the kinetic friction between the objects.

\section*{Teacher Support}
Teacher Support [BL][OL] Review the concept of friction.\\[0pt]
[AL] Start a discussion about the two kinds of friction: static and kinetic. Ask students which one they think would be greater for two given surfaces. Explain the concept of coefficient of friction and what the number would imply in practical terms. Look at the table of static and kinetic friction and ask students to guess which other systems would have higher or lower coefficients.

Imagine, for example, trying to slide a heavy crate across a concrete floor. You may push harder and harder on the crate and not move it at all. This means\\
that the static friction responds to what you do-it increases to be equal to and in the opposite direction of your push. But if you finally push hard enough, the crate seems to slip suddenly and starts to move. Once in motion, it is easier to keep it in motion than it was to get it started because the kinetic friction force is less than the static friction force. If you were to add mass to the crate, (for example, by placing a box on top of it) you would need to push even harder to get it started and also to keep it moving. If, on the other hand, you oiled the concrete you would find it easier to get the crate started and keep it going.

Figure 5.33 shows how friction occurs at the interface between two objects. Magnifying these surfaces shows that they are rough on the microscopic level. So when you push to get an object moving (in this case, a crate), you must raise the object until it can skip along with just the tips of the surface hitting, break off the points, or do both. The harder the surfaces are pushed together (such as if another box is placed on the crate), the more force is needed to move them.

\begin{figure}[h]
\begin{center}
  \includegraphics[max width=\textwidth]{3b86ae0c-65b9-4aad-b74b-2c0d7621e1ae-48}
\captionsetup{labelformat=empty}
\caption{Figure 5.33 Frictional forces, such as \(\mathbf{f}\), always oppose motion or attempted motion between objects in contact. Friction arises in part because of the roughness of the surfaces in contact, as seen in the expanded view.}
\end{center}
\end{figure}

The magnitude of the frictional force has two forms: one for static friction, the other for kinetic friction. When there is no motion between the objects, the magnitude of static friction \(\mathbf{f}_{\mathbf{s}}\) is\\
\(\mathbf{f}_{\mathrm{s}} \leq \mu_{\mathrm{s}} \mathbf{N}_{\mathrm{s}}\),\\
where \(\mu_{\mathrm{s}}\) is the coefficient of static friction and \(\mathbf{N}\) is the magnitude of the normal force. Recall that the normal force opposes the force of gravity and acts perpendicular to the surface in this example, but not always.

Since the symbol \(\leq\) means less than or equal to, this equation says that static friction can have a maximum value of \(\mu_{\mathrm{s}} \mathbf{N}\). That is,\\
\(\mathbf{f}_{\mathrm{s}}(\max )=\mu_{\mathrm{s}} \mathbf{N}\).\\
Static friction is a responsive force that increases to be equal and opposite to whatever force is exerted, up to its maximum limit. Once the applied force exceeds \(\mathbf{f}_{\mathrm{s}}(\max )\), the object will move. Once an object is moving, the magnitude of kinetic friction \(\mathbf{f}_{\mathrm{k}}\) is given by\\
\(\mathbf{f}_{\mathrm{k}}=\mu_{\mathrm{k}} \mathbf{N}\).\\
where \(\mu_{\mathrm{k}}\) is the coefficient of kinetic friction.\\
Friction varies from surface to surface because different substances are rougher than others. Table 5.2 compares values of static and kinetic friction for different surfaces. The coefficient of the friction depends on the two surfaces that are in contact.

Table 5.2 Coefficients of Static and Kinetic Friction\\
Since the direction of friction is always opposite to the direction of motion, friction runs parallel to the surface between objects and perpendicular to the normal force. For example, if the crate you try to push (with a force parallel to the floor) has a mass of 100 kg , then the normal force would be equal to its weight\\
\(\mathbf{W}=m \mathbf{g}=(100 \mathrm{~kg})\left(9.80 \mathrm{~m} / \mathrm{s}^{2}\right)=980 \mathrm{~N}\), perpendicular to the floor. If the coefficient of static friction is 0.45 , you would have to exert a force parallel to the floor greater than\\
\(\mathbf{f}_{\mathrm{s}}(\max )=\mu_{\mathrm{s}} \mathbf{N}=(0.45)(980 \mathrm{~N})=440 \mathrm{~N}\)\\
to move the crate. Once there is motion, friction is less and the coefficient of kinetic friction might be 0.30 , so that a force of only 290 N\\
\(\mathbf{f}_{\mathrm{k}}=\mu_{\mathrm{k}} \mathbf{N}=(0.30)(980 \mathrm{~N})=290 \mathrm{~N}\)\\
would keep it moving at a constant speed. If the floor were lubricated, both coefficients would be much smaller than they would be without lubrication. The coefficient of friction is unitless and is a number usually between 0 and 1.0.

\section*{Working with Inclined Planes}
We discussed previously that when an object rests on a horizontal surface, there is a normal force supporting it equal in magnitude to its weight. Up until now,\\
we dealt only with normal force in one dimension, with gravity and normal force acting perpendicular to the surface in opposing directions (gravity downward, and normal force upward). Now that you have the skills to work with forces in two dimensions, we can explore what happens to weight and the normal force on a tilted surface such as an inclined plane. For inclined plane problems, it is easier breaking down the forces into their components if we rotate the coordinate system, as illustrated in Figure 5.34. The first step when setting up the problem is to break down the force of weight into components.

\begin{figure}[h]
\begin{center}
  \includegraphics[max width=\textwidth]{3b86ae0c-65b9-4aad-b74b-2c0d7621e1ae-50}
\captionsetup{labelformat=empty}
\caption{Figure 5.34 The diagram shows perpendicular and horizontal components of weight on an inclined plane.}
\end{center}
\end{figure}

\section*{Teacher Support}
Teacher Support [BL] Review the concepts of mass, weight, gravitation and normal force.\\[0pt]
[OL] Review vectors and components of vectors.\\
When an object rests on an incline that makes an angle \(\theta\) with the horizontal, the force of gravity acting on the object is divided into two components: A force acting perpendicular to the plane, \(\mathbf{w}_{\perp}\), and a force acting parallel to the plane, \(\mathbf{w}_{\|}\). The perpendicular force of weight, \(\mathbf{w}_{\perp}\), is typically equal in magnitude and opposite in direction to the normal force, \(\mathbf{N}\). The force acting parallel to the plane, \(\mathbf{w}_{\|}\), causes the object to accelerate down the incline. The force of friction, \(\mathbf{f}\), opposes the motion of the object, so it acts upward along the plane.

It is important to be careful when resolving the weight of the object into components. If the angle of the incline is at an angle \(\theta\) to the horizontal, then the magnitudes of the weight components are\\
\(\mathbf{w}_{\|}=\mathbf{w} \sin (\theta)=m \mathbf{g} \sin (\theta)\) and\\
\(\mathbf{w}_{\perp}=\mathbf{w} \cos (\theta)=m \mathbf{g} \cos (\theta)\).\\
Instead of memorizing these equations, it is helpful to be able to determine them from reason. To do this, draw the right triangle formed by the three\\
weight vectors. Notice that the angle of the incline is the same as the angle formed between \(\mathbf{w}\) and \(\mathbf{w}_{\perp}\). Knowing this property, you can use trigonometry to determine the magnitude of the weight components\\
\(\cos (\theta)=\frac{\mathbf{w}_{\perp}}{\mathbf{w}}\)\\
\(\mathbf{w}_{\perp}=\mathbf{w} \cos (\theta)=m \mathbf{g} \cos (\theta)\)\\
\(\sin (\theta)=\frac{\mathbf{w}_{\|}}{\mathbf{w}}\)\\
\(\mathbf{w}_{\|}=\mathbf{w} \sin (\theta)=m \mathbf{g} \sin (\theta)\).

\section*{Teacher Support}
Teacher Support [BL][OL][AL] Experiment with sliding different objects on inclined planes to understand static and kinetic friction. Which objects need a larger angle to slide down? What does this say about the coefficients of friction of those systems? Is a greater force required to start the motion of an object than to keep it in motion? What does this say about static and kinetic friction? When does an object slide down at constant velocity? What does this say about friction and normal force?

\section*{Watch Physics}
Inclined Plane Force Components This video shows how the weight of an object on an inclined plane is broken down into components perpendicular and parallel to the surface of the plane. It explains the geometry for finding the angle in more detail.

Click to view content\\
This video shows how the weight of an object on an inclined plane is broken down into components perpendicular and parallel to the surface of the plane. It explains the geometry for finding the angle in more detail.

When the surface is flat, you could say that one of the components of the gravitational force is zero; Which one? As the angle of the incline gets larger, what happens to the magnitudes of the perpendicular and parallel components of gravitational force?\\
a. When the angle is zero, the parallel component is zero and the perpendicular component is at a maximum. As the angle increases, the parallel component decreases and the perpendicular component increases. This is because the cosine of the angle shrinks while the sine of the angle increases.\\
b. When the angle is zero, the parallel component is zero and the perpendicular component is at a maximum. As the angle increases, the parallel component decreases and the perpendicular component increases. This\\
is because the cosine of the angle increases while the sine of the angle shrinks.\\
c. When the angle is zero, the parallel component is zero and the perpendicular component is at a maximum. As the angle increases, the parallel component increases and the perpendicular component decreases. This is because the cosine of the angle shrinks while the sine of the angle increases.\\
d. When the angle is zero, the parallel component is zero and the perpendicular component is at a maximum. As the angle increases, the parallel component increases and the perpendicular component decreases. This is because the cosine of the angle increases while the sine of the angle shrinks.

\section*{Tips For Success}
Normal force is represented by the variable N. This should not be confused with the symbol for the newton, which is also represented by the letter N. It is important to tell apart these symbols, especially since the units for normal force ( \(\mathbf{N}\) ) happen to be newtons ( N ). For example, the normal force, \(\mathbf{N}\), that the floor exerts on a chair might be \(\mathbf{N}=100 \mathbf{N}\). One important difference is that normal force is a vector, while the newton is simply a unit. Be careful not to confuse these letters in your calculations!

To review, the process for solving inclined plane problems is as follows:

\begin{enumerate}
  \item Draw a sketch of the problem.
  \item Identify known and unknown quantities, and identify the system of interest.
  \item Draw a free-body diagram (which is a sketch showing all of the forces acting on an object) with the coordinate system rotated at the same angle as the inclined plane. Resolve the vectors into horizontal and vertical components and draw them on the free-body diagram.
  \item Write Newton's second law in the horizontal and vertical directions and add the forces acting on the object. If the object does not accelerate in a particular direction (for example, the \(x\)-direction) then \(\mathbf{F}\) net \(x=0\). If the object does accelerate in that direction, Fnet \(x=m \mathbf{a}\).
  \item Check your answer. Is the answer reasonable? Are the units correct?
\end{enumerate}

\section*{Worked Example}
Finding the Coefficient of Kinetic Friction on an Inclined Plane A skier, illustrated in Figure 5.35(a), with a mass of 62 kg is sliding down a snowy slope at an angle of 25 degrees. Find the coefficient of kinetic friction for the skier if friction is known to be 45.0 N .

\begin{figure}[h]
\begin{center}
  \includegraphics[max width=\textwidth]{3b86ae0c-65b9-4aad-b74b-2c0d7621e1ae-53}
\captionsetup{labelformat=empty}
\caption{Figure 5.35 Use the diagram to help find the coefficient of kinetic friction for the skier.}
\end{center}
\end{figure}

\section*{Strategy}
The magnitude of kinetic friction was given as 45.0 N . Kinetic friction is related to the normal force \(\mathbf{N}\) as \(\mathbf{f}_{\mathrm{k}}=\mu_{\mathrm{k}} \mathbf{N}\). Therefore, we can find the coefficient of kinetic friction by first finding the normal force of the skier on a slope. The normal force is always perpendicular to the surface, and since there is no motion perpendicular to the surface, the normal force should equal the component of the skier's weight perpendicular to the slope.

That is,\\
\(\mathbf{N}=\mathbf{w}_{\perp}=\mathbf{w} \cos \left(25^{\circ}\right)=m \mathbf{g} \cos \left(25^{\circ}\right)\).\\
Substituting this into our expression for kinetic friction, we get\\
\(\mathbf{f}_{\mathrm{k}}=\mu_{\mathrm{k}} m \mathbf{g} \cos 25^{\circ}\),\\
which can now be solved for the coefficient of kinetic friction \({ }_{\mathrm{k}}\).\\
Solution\\
Solving for \(\mu_{\mathrm{k}}\) gives\\
\(\mu_{\mathrm{k}}=\frac{\mathbf{f}_{\mathrm{k}}}{\mathbf{w} \cos 25^{\circ}}=\frac{\mathbf{f}_{\mathrm{k}}}{m \mathbf{g} \cos 25^{\circ}}\).\\
Substituting known values on the right-hand side of the equation,\\
\(\mu_{\mathrm{k}}=\frac{45.0 \mathrm{~N}}{(62 \mathrm{~kg})\left(9.80 \mathrm{~m} / \mathrm{s}^{2}\right)(0.906)}=0.082\).\\
Discussion\\
This result is a little smaller than the coefficient listed in Table 5.2 for waxed wood on snow, but it is still reasonable since values of the coefficients of friction can vary greatly. In situations like this, where an object of mass \(m\) slides\\
down a slope that makes an angle with the horizontal, friction is given by \(\mathbf{f}_{\mathrm{k}}=\mu_{\mathrm{k}} m \mathbf{g} \cos \theta\).

\section*{Worked Example}
Weight on an Incline, a Two-Dimensional Problem The skier's mass, including equipment, is 60.0 kg . (See Figure 5.36(b).) (a) What is her acceleration if friction is negligible? (b) What is her acceleration if the frictional force is 45.0 N ?

\begin{figure}[h]
\begin{center}
  \includegraphics[max width=\textwidth]{3b86ae0c-65b9-4aad-b74b-2c0d7621e1ae-54}
\captionsetup{labelformat=empty}
\caption{Figure 5.36 Now use the diagram to help find the skier's acceleration if friction is negligible and if the frictional force is 45.0 N .}
\end{center}
\end{figure}

\section*{Strategy}
The most convenient coordinate system for motion on an incline is one that has one coordinate parallel to the slope and one perpendicular to the slope. Remember that motions along perpendicular axes are independent. We use the symbol ⟂ to mean perpendicular, and \(\|\) to mean parallel.

The only external forces acting on the system are the skier's weight, friction, and the normal force exerted by the ski slope, labeled \(\mathbf{w}, \mathbf{f}\), and \(\mathbf{N}\) in the free-body diagram. \(\mathbf{N}\) is always perpendicular to the slope and \(\mathbf{f}\) is parallel to it. But \(\mathbf{w}\) is not in the direction of either axis, so we must break it down into components along the chosen axes. We define \(\mathbf{w}_{\|}\)to be the component of weight parallel to the slope and \(\mathbf{w}_{\perp}\) the component of weight perpendicular to the slope. Once this is done, we can consider the two separate problems of forces parallel to the slope and forces perpendicular to the slope.

\section*{Solution}
The magnitude of the component of the weight parallel to the slope is \(\mathbf{w}_{\|}=\)\\
\(\boldsymbol{w} \sin \left(25^{\circ}\right)=m \boldsymbol{g} \sin \left(25^{\circ}\right)\), and the magnitude of the component of the weight perpendicular to the slope is \(\mathbf{w}_{\perp}=\mathbf{w} \cos \left(25^{\circ}\right)=m \mathbf{g} \cos \left(25^{\circ}\right)\).\\
(a) Neglecting friction: Since the acceleration is parallel to the slope, we only need to consider forces parallel to the slope. Forces perpendicular to the slope add to zero, since there is no acceleration in that direction. The forces parallel to the slope are the amount of the skier's weight parallel to the slope \(\mathbf{w}_{\|}\)and friction \(\mathbf{f}\). Assuming no friction, by Newton's second law the acceleration parallel to the slope is\\
\(\mathbf{a}_{\|}=\frac{\mathbf{F}_{\text {net } \|}}{m}\),\\
Where the net force parallel to the slope \(\mathbf{F}_{\text {net } \|}=\mathbf{w}_{\|}=m \mathbf{g} \sin \left(25^{\circ}\right)\), so that

\[
\begin{aligned}
\mathbf{a}_{\|} & =\frac{\mathbf{F}_{\text {net } \|}}{m}=\frac{m \operatorname{g} \sin \left(25^{\circ}\right)}{m}=\operatorname{g} \sin \left(25^{\circ}\right) \\
& =\left(9.80 \mathrm{~m} / \mathrm{s}^{2}\right)(0.423)=4.14 \mathrm{~m} / \mathrm{s}^{2}
\end{aligned}
\]

is the acceleration.\\
(b) Including friction: Here we now have a given value for friction, and we know its direction is parallel to the slope and it opposes motion between surfaces in contact. So the net external force is now\\
\(\mathbf{F}_{\text {net } \|}=\mathbf{w}_{\|}-\mathbf{f}\),\\
and substituting this into Newton's second law, \(a_{\|}=\frac{\mathbf{F}_{\text {net } \|}}{m}\) gives\\
\(\mathbf{a}_{\|}=\frac{\mathbf{F}_{n e t} \|}{m}=\frac{\mathbf{w}_{\|}-\mathbf{f}}{m}=\frac{m \mathbf{g s i n}\left(25^{\circ}\right)-\mathbf{f}}{m}\).\\
We substitute known values to get\\
\(\mathbf{a}_{\|}=\frac{(60.0 \mathrm{~kg})\left(9.80 \mathrm{~m} / \mathrm{s}^{2}\right)(0.423)-45.0 \mathrm{~N}}{60.0 \mathrm{~kg}}\),\\
or\\
\(\mathbf{a}_{\|}=3.39 \mathrm{~m} / \mathrm{s}^{2}\),\\
which is the acceleration parallel to the incline when there is 45 N opposing friction.

Discussion\\
Since friction always opposes motion between surfaces, the acceleration is smaller when there is friction than when there is not.

\section*{Practice Problems}
15.

When an object sits on an inclined plane that makes an angle with the horizontal, what is the expression for the component of the objects weight force that is parallel to the incline?\\
a. \(w_{\|}=w \cos \theta\)\\
b. \(w_{\|}=w \sin \theta\)\\
c. \(w_{\|}=w \sin \theta-\cos \theta\)\\
d. \(w_{\|}=w \cos \theta-\sin \theta\)\\
16.

An object with a mass of \(5 \backslash, \backslash \operatorname{text}\{\mathrm{~kg}\}\) rests on a plane inclined \(30^{\wedge} \backslash\) circ \(\backslash\) ! from horizontal. What is the component of the weight force that is parallel to the incline?\\
a. \(4.33 \backslash, \backslash \operatorname{text}\{\mathrm{~N}\}\)\\
b. \(5.0 \backslash, \backslash \operatorname{text}\{\mathrm{~N}\}\)\\
c. \(24.5 \backslash, \backslash \operatorname{text}\{\mathrm{~N}\}\)\\
d. \(42.43 \backslash, \backslash \operatorname{text}\{\mathrm{~N}\}\)

\section*{Snap Lab}
Friction at an Angle: Sliding a Coin An object will slide down an inclined plane at a constant velocity if the net force on the object is zero. We can use this fact to measure the coefficient of kinetic friction between two objects. As shown in the first Worked Example, the kinetic friction on a slope \(\mathbf{f}_{\mathrm{k}}=\mu_{\mathrm{k}} m \mathbf{g} \cos \theta\), and the component of the weight down the slope is equal to \(m \mathbf{g} \sin \theta\). These forces act in opposite directions, so when they have equal magnitude, the acceleration is zero. Writing these out

\[
\begin{aligned}
\mathbf{f}_{\mathrm{k}} & =\mathbf{F} \mathbf{g}_{x} \\
\mu_{\mathrm{k}} m \mathbf{g} \cos \theta & =m \mathbf{g} \sin \theta
\end{aligned}
\]

Solving for \(\mu_{\mathrm{k}}\), since \(\tan \theta=\sin \theta / \cos \theta\) we find that\\
\(\mu_{\mathrm{k}}=\frac{m \mathbf{g} \sin \theta}{m \mathbf{g} \cos \theta}=\tan \theta\).\\
5.10

\begin{itemize}
  \item 1 coin
  \item 1 book
  \item 1 protractor
\end{itemize}

\begin{enumerate}
  \item Put a coin flat on a book and tilt it until the coin slides at a constant velocity down the book. You might need to tap the book lightly to get the coin to move.
  \item Measure the angle of tilt relative to the horizontal and find \(\mu_{\mathrm{k}}\).
\end{enumerate}

\section*{Grasp Check}
True or False - If only the angles of two vectors are known, we can find the angle of their resultant addition vector.\\
a. True\\
b. False

\section*{Check Your Understanding}
17.

What is friction?\\
a. Friction is an internal force that opposes the relative motion of an object.\\
b. Friction is an internal force that accelerates an object's relative motion.\\
c. Friction is an external force that opposes the relative motion of an object.\\
d. Friction is an external force that increases the velocity of the relative motion of an object.\\
18.

What are the two varieties of friction? What does each one act upon?\\
a. Kinetic and static friction both act on an object in motion.\\
b. Kinetic friction acts on an object in motion, while static friction acts on an object at rest.\\
c. Kinetic friction acts on an object at rest, while static friction acts on an object in motion.\\
d. Kinetic and static friction both act on an object at rest.\\
19.

Given static and kinetic friction between two surfaces, which has a greater value? Why?\\
a. The kinetic friction has a greater value because the friction between the two surfaces is more when the two surfaces are in relative motion.\\
b. The static friction has a greater value because the friction between the two surfaces is more when the two surfaces are in relative motion.\\
c. The kinetic friction has a greater value because the friction between the two surfaces is less when the two surfaces are in relative motion.\\
d. The static friction has a greater value because the friction between the two surfaces is less when the two surfaces are in relative motion.

\section*{Teacher Support}
Teacher Support Use the Check Your Understanding questions to assess whether students achieve the learning objectives for this section. If students are struggling with a specific objective, the Check Your Understanding will help identify which objective is causing the problem and direct students to the relevant content.

\subsection*{5.5 Simple Harmonic Motion}
\section*{Section Learning Objectives}
By the end of this section, you will be able to do the following:

\begin{itemize}
  \item Describe Hooke's law and Simple Harmonic Motion
  \item Describe periodic motion, oscillations, amplitude, frequency, and period
  \item Solve problems in simple harmonic motion involving springs and pendulums
\end{itemize}

\section*{Teacher Support}
Teacher Support The learning objectives in this section will help your students master the following standards:

\begin{itemize}
  \item (7) Science concepts. The student knows the characteristics and behavior of waves. The student is expected to:
  \item (A) examine and describe oscillatory motion and wave propagation in various types of media.
\end{itemize}

In addition, the High School Physics Laboratory Manual addresses content in this section in the lab titled: Motion in Two Dimensions, as well as the following standards:

\begin{itemize}
  \item (7) Science concepts. The student knows the characteristics and behavior of waves. The student is expected to:
  \item (A) examine and describe oscillatory motion and wave propagation in various types of media.
\end{itemize}

\section*{Section Key Terms}
\section*{Hooke s Law and Simple Harmonic Motion}
Imagine a car parked against a wall. If a bulldozer pushes the car into the wall, the car will not move but it will noticeably change shape. A change in shape due to the application of a force is a deformation. Even very small forces are known to cause some deformation. For small deformations, two important things can happen. First, unlike the car and bulldozer example, the object returns to its original shape when the force is removed. Second, the size of the deformation is proportional to the force. This second property is known as Hooke's law. In equation form, Hooke's law is\\
\(\mathbf{F}=-\mathbf{k x}\),\\
where \(\mathbf{x}\) is the amount of deformation (the change in length, for example) produced by the restoring force \(\mathbf{F}\), and \(\mathbf{k}\) is a constant that depends on the shape and composition of the object. The restoring force is the force that brings the object back to its equilibrium position; the minus sign is there because the restoring force acts in the direction opposite to the displacement. Note that the restoring force is proportional to the deformation \(\mathbf{x}\). The deformation can also be thought of as a displacement from equilibrium. It is a change in position due to a force. In the absence of force, the object would rest at its equilibrium position. The force constant \(\mathbf{k}\) is related to the stiffness of a system. The larger the force constant, the stiffer the system. A stiffer system is more difficult to deform and requires a greater restoring force. The units of \(\mathbf{k}\) are newtons per meter ( \(\mathrm{N} / \mathrm{m}\) ). One of the most common uses of Hooke's law is solving problems involving springs and pendulums, which we will cover at the end of this section.

\section*{Teacher Support}
Teacher Support [BL] Review the concept of force.\\[0pt]
[BL][OL][AL] Introduce Hooke's law and force constant of a spring.

\section*{Oscillations and Periodic Motion}
What do an ocean buoy, a child in a swing, a guitar, and the beating of hearts all have in common? They all oscillate. That is, they move back and forth between two points, like the ruler illustrated in Figure 5.37. All oscillations involve force. For example, you push a child in a swing to get the motion started.

\begin{figure}[h]
\begin{center}
  \includegraphics[max width=\textwidth]{3b86ae0c-65b9-4aad-b74b-2c0d7621e1ae-59}
\captionsetup{labelformat=empty}
\caption{Figure 5.37 A ruler is displaced from its equilibrium position.}
\end{center}
\end{figure}

\section*{Teacher Support}
Teacher Support [BL][OL][AL] Find springs or rubber bands with different\\
amounts of stiffness. Ask students to attach weights to these to construct oscillators. Introduce the terms frequency and time period. Ask students to observe how the stiffness of the spring affects them. How does mass of the system affect them? How does the initial force applied affect them?

Newton's first law implies that an object oscillating back and forth is experiencing forces. Without force, the object would move in a straight line at a constant speed rather than oscillate. Consider, for example, plucking a plastic ruler to the left as shown in Figure 5.38. The deformation of the ruler creates a force in the opposite direction, known as a restoring force. Once released, the restoring force causes the ruler to move back toward its stable equilibrium position, where the net force on it is zero. However, by the time the ruler gets there, it gains momentum and continues to move to the right, producing the opposite deformation. It is then forced to the left, back through equilibrium, and the process is repeated until it gradually loses all of its energy. The simplest oscillations occur when the restoring force is directly proportional to displacement. Recall that Hooke's law describes this situation with the equation \(\mathbf{F}=-\mathbf{k x}\). Therefore, Hooke's law describes and applies to the simplest case of oscillation, known as simple harmonic motion.

\begin{figure}[h]
\begin{center}
  \includegraphics[max width=\textwidth]{3b86ae0c-65b9-4aad-b74b-2c0d7621e1ae-60}
\captionsetup{labelformat=empty}
\caption{Figure 5.38 (a) The plastic ruler has been released, and the restoring force is returning the ruler to its equilibrium position. (b) The net force is zero at the equilibrium position, but the ruler has momentum and continues to move to the right. (c) The restoring force is in the opposite direction. It stops the ruler and moves it back toward equilibrium again. (d) Now the ruler has momentum to the left. (e) In the absence of damping (caused by frictional forces), the ruler reaches its original position. From there, the motion will repeat itself.}
\end{center}
\end{figure}

When you pluck a guitar string, the resulting sound has a steady tone and lasts a long time. Each vibration of the string takes the same time as the previous one. Periodic motion is a motion that repeats itself at regular time intervals, such as with an object bobbing up and down on a spring or a pendulum swinging back and forth. The time to complete one oscillation (a complete cycle of motion) remains constant and is called the period \(T\). Its units are usually seconds.

Frequency \(f\) is the number of oscillations per unit time. The SI unit for frequency is the hertz (Hz), defined as the number of oscillations per second. The relationship between frequency and period is\\
\(f=1 / T\).

As you can see from the equation, frequency and period are different ways of expressing the same concept. For example, if you get a paycheck twice a month, you could say that the frequency of payment is two per month, or that the period between checks is half a month.

If there is no friction to slow it down, then an object in simple motion will oscillate forever with equal displacement on either side of the equilibrium position. The equilibrium position is where the object would naturally rest in the absence of force. The maximum displacement from equilibrium is called the amplitude \(\mathbf{X}\). The units for amplitude and displacement are the same, but depend on the type of oscillation. For the object on the spring, shown in Figure 5.39, the units of amplitude and displacement are meters.

\begin{figure}[h]
\begin{center}
  \includegraphics[max width=\textwidth]{3b86ae0c-65b9-4aad-b74b-2c0d7621e1ae-61}
\captionsetup{labelformat=empty}
\caption{Figure 5.39 An object attached to a spring sliding on a frictionless surface is a simple harmonic oscillator. When displaced from equilibrium, the object performs simple harmonic motion that has an amplitude \(\mathbf{X}\) and a period \(T\). The object's maximum speed occurs as it passes through equilibrium. The stiffer the spring is, the smaller the period \(T\). The greater the mass of the object is, the greater the period \(T\).}
\end{center}
\end{figure}

The mass \(m\) and the force constant \(\mathbf{k}\) are the only factors that affect the period and frequency of simple harmonic motion. The period of a simple harmonic oscillator is given by\\
\(T=2 \pi \sqrt{\frac{m}{\mathbf{k}}}\)\\
and, because \(f=1 / T\), the frequency of a simple harmonic oscillator is\\
\(f=\frac{1}{2 \pi} \sqrt{\frac{\mathbf{k}}{m}}\).

\section*{Watch Physics}
Introduction to Harmonic Motion This video shows how to graph the displacement of a spring in the x-direction over time, based on the period. Watch the first 10 minutes of the video (you can stop when the narrator begins to cover calculus).

Click to view content\\
Click to view content\\
If the amplitude of the displacement of a spring were larger, how would this affect the graph of displacement over time? What would happen to the graph if the period was longer?\\
a. Larger amplitude would result in taller peaks and troughs and a longer period would result in greater separation in time between peaks.\\
b. Larger amplitude would result in smaller peaks and troughs and a longer period would result in greater distance between peaks.\\
c. Larger amplitude would result in taller peaks and troughs and a longer period would result in shorter distance between peaks.\\
d. Larger amplitude would result in smaller peaks and troughs and a longer period would result in shorter distance between peaks.

\section*{Solving Spring and Pendulum Problems with Simple Harmonic Motion}
Before solving problems with springs and pendulums, it is important to first get an understanding of how a pendulum works. Figure 5.40 provides a useful illustration of a simple pendulum.\\
\includegraphics[max width=\textwidth, center]{3b86ae0c-65b9-4aad-b74b-2c0d7621e1ae-62}

Figure 5.40 A simple pendulum has a small-diameter bob and a string that has a very small mass but is strong enough not to stretch. The linear displacement from equilibrium is s , the length of the arc. Also shown are the forces on the bob, which result in a net force of \(-m g\) sin toward the equilibrium position - that is, a restoring force.

\section*{Teacher Support}
Teacher Support [BL] Review simple harmonic motion.\\
Everyday examples of pendulums include old-fashioned clocks, a child's swing, or the sinker on a fishing line. For small displacements of less than 15 degrees, a pendulum experiences simple harmonic oscillation, meaning that its restoring force is directly proportional to its displacement. A pendulum in simple harmonic motion is called a simple pendulum. A pendulum has an object with a small mass, also known as the pendulum bob, which hangs from a light wire or string. The equilibrium position for a pendulum is where the angle \(\theta\) is zero (that is, when the pendulum is hanging straight down). It makes sense that without any force applied, this is where the pendulum bob would rest.

\section*{Teacher Support}
Teacher Support [BL][OL][AL]Construct simple pendulums of different lengths. Ask students to measure their time periods or frequencies. Are they constant for a given pendulum? How does the mass impact the frequency? How does the initial displacement affect it? What happens if a small push is given to the pendulum to get it started? Does that change the frequency? In what way does the length affect the frequency?

The displacement of the pendulum bob is the arc length \(s\). The weight \(m \mathbf{g}\) has components \(m \mathbf{g} \cos \theta\) along the string and \(m \mathbf{g} \sin \theta\) tangent to the arc. Tension in the string exactly cancels the component \(m \mathbf{g} \cos \theta\) parallel to the string. This leaves a net restoring force back toward the equilibrium position that runs tangent to the arc and equals \(-m \mathbf{g} \sin \theta\).\\
For a simple pendulum, The period is \(T=2 \pi \sqrt{\frac{L}{\mathbf{g}}}\).\\
The only things that affect the period of a simple pendulum are its length and the acceleration due to gravity. The period is completely independent of other factors, such as mass or amplitude. However, note that \(T\) does depend on \(\mathbf{g}\). This means that if we know the length of a pendulum, we can actually use it to measure gravity! This will come in useful in Measuring Acceleration due to Gravity: The Period of a Pendulum.

\section*{Tips For Success}
Tension is represented by the variable \(\mathbf{T}\), and period is represented by the variable \(T\). It is important not to confuse the two, since tension is a force and period\\
is a length of time.

\section*{Worked Example}
Measuring Acceleration due to Gravity: The Period of a Pendulum What is the acceleration due to gravity in a region where a simple pendulum having a length 75.000 cm has a period of 1.7357 s ?

\section*{Strategy}
We are asked to find \(\mathbf{g}\) given the period \(T\) and the length \(L\) of a pendulum. We can solve \(T=2 \pi \sqrt{\frac{L}{\mathbf{g}}}\) for \(\mathbf{g}\), assuming that the angle of deflection is less than 15 degrees. Recall that when the angle of deflection is less than 15 degrees, the pendulum is considered to be in simple harmonic motion, allowing us to use this equation.

Solution

\begin{enumerate}
  \item Square \(T=2 \pi \sqrt{\frac{L}{\mathbf{g}}}\) and solve for \(g\).
\end{enumerate}

\begin{itemize}
  \item \(\mathbf{g}=4 \pi^{2} \frac{L}{T^{2}}\)
\end{itemize}

\begin{enumerate}
  \setcounter{enumi}{1}
  \item Substitute known values into the new equation.
\end{enumerate}

\begin{itemize}
  \item \(\mathbf{g}=4 \pi^{2} \frac{0.75000 \mathrm{~m}}{(1.7357 \mathrm{~s})^{2}}\)
\end{itemize}

\begin{enumerate}
  \setcounter{enumi}{2}
  \item Calculate to find \(\mathbf{g}\).
\end{enumerate}

\begin{itemize}
  \item \(\mathbf{g}=9.8281 \mathrm{~m} / \mathrm{s}^{2}\)
\end{itemize}

Discussion\\
This method for determining \(\mathbf{g}\) can be very accurate. This is why length and period are given to five digits in this example.

\section*{Worked Example}
Hooke s Law: How Stiff Are Car Springs? What is the force constant for the suspension system of a car, like that shown in Figure 5.41, that settles 1.20 cm when an \(80.0-\mathrm{kg}\) person gets in?\\
\includegraphics[max width=\textwidth, center]{3b86ae0c-65b9-4aad-b74b-2c0d7621e1ae-65}

Figure 5.41 A car in a parking lot. (exfordy, Flickr)

\section*{Strategy}
Consider the car to be in its equilibrium position \(\mathbf{x}=0\) before the person gets in. The car then settles down 1.20 cm , which means it is displaced to a position \(\mathbf{x}=-1.20 \times 10^{2} \mathrm{~m}\).

At that point, the springs supply a restoring force \(\mathbf{F}\) equal to the person's weight \(\mathbf{w}=m \mathbf{g}=(80.0 \mathrm{~kg})\left(9.80 \mathrm{~m} / \mathrm{s}^{2}\right)=784 \mathrm{~N}\). We take this force to be \(\mathbf{F}\) in Hooke's law.

Knowing \(\mathbf{F}\) and \(\mathbf{x}\), we can then solve for the force constant \(\mathbf{k}\).\\
Solution\\
Solve Hooke's law, \(\mathbf{F}=-\mathbf{k x}\), for \(\mathbf{k}\).\\
\(\mathbf{k}=\frac{\mathbf{F}}{\mathbf{x}}\)\\
Substitute known values and solve for \(\mathbf{k}\).

\[
\begin{aligned}
\mathbf{k} & =\frac{-784 \mathrm{~N}}{-1.20 \times 10^{-2} \mathrm{~m}} \\
& =6.53 \times 10^{4} \mathrm{~N} / \mathrm{m}
\end{aligned}
\]

Discussion\\
Note that \(\mathbf{F}\) and \(\mathbf{x}\) have opposite signs because they are in opposite directionsthe restoring force is up, and the displacement is down. Also, note that the car would oscillate up and down when the person got in, if it were not for the shock absorbers. Bouncing cars are a sure sign of bad shock absorbers.

\section*{Practice Problems}
20.

A force of \(70 \backslash, \backslash \operatorname{text}\{\mathrm{~N}\}\) applied to a spring causes it to be displaced by \(0.3 \backslash, \backslash \operatorname{text}\{\mathrm{~m}\}\). What is the force constant of the spring?\\
a. \(\{-233\} \backslash, \backslash \operatorname{text}\{\mathrm{N} / \mathrm{m}\}\)\\
b. \(\{-21\} \backslash, \backslash \operatorname{text}\{\mathrm{N} / \mathrm{m}\}\)\\
c. \(21 \backslash, \backslash \operatorname{text}\{\mathrm{~N} / \mathrm{m}\}\)\\
d. \(233 \backslash, \backslash \operatorname{text}\{\mathrm{~N} / \mathrm{m}\}\)\\
21.

What is the force constant for the suspension system of a car that settles \(3.3 \backslash, \backslash \operatorname{text}\{\mathrm{~cm}\}\) when a \(65 \backslash, \backslash \operatorname{text}\{\mathrm{~kg}\}\) person gets in?\\
a. \(1.93 \backslash\) times \(10^{\wedge} 4 \backslash, \backslash \operatorname{text}\{\mathrm{~N} / \mathrm{m}\}\)\\
b. \(1.97 \backslash\) times \(10^{\wedge} 3 \backslash, \backslash \operatorname{text}\{\mathrm{~N} / \mathrm{m}\}\)\\
c. \(1.93 \backslash\) times \(10^{\wedge} 2 \backslash, \backslash \operatorname{text}\{\mathrm{~N} / \mathrm{m}\}\)\\
d. \(1.97 \backslash\) times \(10^{\wedge} 1 \backslash, \backslash \operatorname{text}\{\mathrm{~N} / \mathrm{m}\}\)

\section*{Snap Lab}
Finding Gravity Using a Simple Pendulum Use a simple pendulum to find the acceleration due to gravity \(\mathbf{g}\) in your home or classroom.

\begin{itemize}
  \item 1 string
  \item 1 stopwatch
  \item 1 small dense object
\end{itemize}

\begin{enumerate}
  \item Cut a piece of a string or dental floss so that it is about 1 m long.
  \item Attach a small object of high density to the end of the string (for example, a metal nut or a car key).
  \item Starting at an angle of less than 10 degrees, allow the pendulum to swing and measure the pendulum's period for 10 oscillations using a stopwatch.
  \item Calculate g.
\end{enumerate}

\section*{Grasp Check}
How accurate is this measurement for \(g\) ? How might it be improved?\\
a. Accuracy for value of \(g\) will increase with an increase in the mass of a dense object.\\
b. Accuracy for the value of \(g\) will increase with increase in the length of the pendulum.\\
c. The value of \(g\) will be more accurate if the angle of deflection is more than \(15^{\circ}\).\\
d. The value of \(g\) will be more accurate if it maintains simple harmonic motion.

\section*{Check Your Understanding}
22.

What is deformation?\\
a. Deformation is the magnitude of the restoring force.\\
b. Deformation is the change in shape due to the application of force.\\
c. Deformation is the maximum force that can be applied on a spring.\\
d. Deformation is regaining the original shape upon the removal of an external force.\\
23.

According to Hooke's law, what is deformation proportional to?\\
a. Force\\
b. Velocity\\
c. Displacement\\
d. Force constant\\
24.

What are oscillations?\\
a. Motion resulting in small displacements\\
b. Motion which repeats itself periodically\\
c. Periodic, repetitive motion between two points\\
d. motion that is the opposite to the direction of the restoring force\\
25.

True or False-Oscillations can occur without force.\\
a. True\\
b. False

\section*{Teacher Support}
Teacher Support Use the Check Your Understanding questions to assess whether students achieve the learning objectives for this section. If students are struggling with a specific objective, the Check Your Understanding will help identify which objective is causing the problem and direct students to the relevant content.

\section*{Ke Terms}
air resistance a frictional force that slows the motion of objects as they travel through the air; when solving basic physics problems, air resistance is assumed to be zero\\
amplitude the maximum displacement from the equilibrium position of an object oscillating around the equilibrium position\\
analytical method the method of determining the magnitude and direction of a resultant vector using the Pythagorean theorem and trigonometric identities\\
component (of a \(\mathbf{2}\)-dimensional vector) a piece of a vector that points in either the vertical or the horizontal direction; every \(2-\mathrm{d}\) vector can be expressed as a sum of two vertical and horizontal vector components\\
deformation displacement from equilibrium, or change in shape due to the application of force\\
equilibrium position where an object would naturally rest in the absence of force\\
frequency number of events per unit of time\\
graphical method drawing vectors on a graph to add them using the head-to-tail method\\
head (of a vector) the end point of a vector; the location of the vector's arrow; also referred to as the tip\\
head-to-tail method a method of adding vectors in which the tail of each vector is placed at the head of the previous vector

Hooke s law proportional relationship between the force \(\mathbf{F}\) on a material and the deformation \(\Delta L\) it causes, \(\mathbf{F}=\mathbf{k} \Delta L\)\\
kinetic friction a force that opposes the motion of two systems that are in contact and moving relative to one another\\
maximum height (of a projectile) the highest altitude, or maximum displacement in the vertical position reached in the path of a projectile\\
oscillate moving back and forth regularly between two points\\
period time it takes to complete one oscillation\\
periodic motion motion that repeats itself at regular time intervals\\
projectile an object that travels through the air and experiences only acceleration due to gravity\\
projectile motion the motion of an object that is subject only to the acceleration of gravity\\
range the maximum horizontal distance that a projectile travels\\
restoring force force acting in opposition to the force caused by a deformation\\
resultant the sum of the a collection of vectors\\
resultant vector the vector sum of two or more vectors\\
simple harmonic motion the oscillatory motion in a system where the net force can be described by Hooke's law\\
simple pendulum an object with a small mass suspended from a light wire or string\\
static friction a force that opposes the motion of two systems that are in contact and are not moving relative to one another\\
tail the starting point of a vector; the point opposite to the head or tip of the arrow\\
trajectory the path of a projectile through the air\\
vector addition adding together two or more vectors

\section*{Ke Equations}
\subsection*{5.2 Vector Addition and Subtraction: Analytical Methods}
\subsection*{5.3 Projectile Motion}
\subsection*{5.4 Inclined Planes}
\subsection*{5.5 Simple Harmonic Motion}
\section*{Section Summar}
\subsection*{5.1 Vector Addition and Subtraction: Graphical Methods}
\begin{itemize}
  \item The graphical method of adding vectors \(\mathbf{A}\) and \(\mathbf{B}\) involves drawing vectors on a graph and adding them by using the head-to-tail method. The resultant vector \(\mathbf{R}\) is defined such that \(\mathbf{A}+\mathbf{B}=\mathbf{R}\). The magnitude and direction of R are then determined with a ruler and protractor.
  \item The graphical method of subtracting vectors \(\mathbf{A}\) and \(\mathbf{B}\) involves adding the opposite of vector \(\mathbf{B}\), which is defined as \(-\mathbf{B}\). In this case, \(\mathbf{A}-\mathbf{B}=\mathbf{A}+(-\mathbf{B})=\mathbf{R}\). Next, use the head-to-tail method as for vector addition to obtain the resultant vector R .
  \item Addition of vectors is independent of the order in which they are added; \(\mathbf{A}+\mathbf{B}=\mathbf{B}+\mathbf{A}\).
  \item The head-to-tail method of adding vectors involves drawing the first vector on a graph and then placing the tail of each subsequent vector at the head of the previous vector. The resultant vector is then drawn from the tail of the first vector to the head of the final vector.
  \item Variables in physics problems, such as force or velocity, can be represented with vectors by making the length of the vector proportional to the magnitude of the force or velocity.
  \item Problems involving displacement, force, or velocity may be solved graphically by measuring the resultant vector's magnitude with a ruler and measuring the direction with a protractor.
\end{itemize}

\subsection*{5.2 Vector Addition and Subtraction: Analytical Methods}
\begin{itemize}
  \item The analytical method of vector addition and subtraction uses the Pythagorean theorem and trigonometric identities to determine the magnitude and direction of a resultant vector.
  \item The steps to add vectors A and B using the analytical method are as follows:
\end{itemize}

\begin{enumerate}
  \item Determine the coordinate system for the vectors. Then, determine the horizontal and vertical components of each vector using the equations\\
\(A_{x}=A \cos \theta\)\\
\_ \(B_{x}=B \cos \theta\)\\
and\\
\(A_{y}=A \sin \theta\)\\
\(B_{y}=B \sin \theta\).
  \item Add the horizontal and vertical components of each vector to determine the components \(R_{x}\) and \(R_{y}\) of the resultant vector, R .
\end{enumerate}

\begin{itemize}
  \item \(R_{x}=A_{x}+B_{x}\)\\
and\\
\(R_{y}=A_{y}+B_{y}\).
\end{itemize}

\begin{enumerate}
  \setcounter{enumi}{2}
  \item Use the Pythagorean theorem to determine the magnitude, \(R\), of the resultant vector \(R\).\\
\(-R=\sqrt{R_{x}^{2}+R_{y}^{2}}\)
  \item Use a trigonometric identity to determine the direction, , of R . \(-\quad=\tan ^{-1}\left(R_{y} / R_{x}\right)\)
\end{enumerate}

\subsection*{5.3 Projectile Motion}
\begin{itemize}
  \item Projectile motion is the motion of an object through the air that is subject only to the acceleration of gravity.
  \item Projectile motion in the horizontal and vertical directions are independent of one another.
  \item The maximum height of an projectile is the highest altitude, or maximum displacement in the vertical position reached in the path of a projectile.
  \item The range is the maximum horizontal distance traveled by a projectile.
  \item To solve projectile problems: choose a coordinate system; analyze the motion in the vertical and horizontal direction separately; then, recombine the horizontal and vertical components using vector addition equations.
\end{itemize}

\subsection*{5.4 Inclined Planes}
\begin{itemize}
  \item Friction is a contact force between systems that opposes the motion or attempted motion between them. Simple friction is proportional to the normal force \(\mathbf{N}\) pushing the systems together. A normal force is always perpendicular to the contact surface between systems. Friction depends on both of the materials involved.
  \item \(\mu_{\mathrm{s}}\) is the coefficient of static friction, which depends on both of the materials.
  \item \(\mu_{\mathrm{k}}\) is the coefficient of kinetic friction, which also depends on both materials.
  \item When objects rest on an inclined plane that makes an angle with the horizontal surface, the weight of the object can be broken into components that act perpendicular ( \(\mathbf{w}_{\perp}\) ) and parallel ( \(\mathbf{w}_{\|}\)) to the surface of the plane.
\end{itemize}

\subsection*{5.5 Simple Harmonic Motion}
\begin{itemize}
  \item An oscillation is a back and forth motion of an object between two points of deformation.
  \item An oscillation may create a wave, which is a disturbance that propagates from where it was created.
  \item The simplest type of oscillations are related to systems that can be described by Hooke's law.
  \item Periodic motion is a repetitious oscillation.
  \item The time for one oscillation is the period T.
  \item The number of oscillations per unit time is the frequency
  \item A mass \(m\) suspended by a wire of length \(L\) is a simple pendulum and undergoes simple harmonic motion for amplitudes less than about 15 degrees.
\end{itemize}

\end{document}